\part{Lie groups and representations}

\chapter{Lie groups}

    Groups are the mathematical language of physics transformations or symmetries. This can be justified by noticing that the set of transformations must include the identity one, i.e.~nothing happens, the inverse one, i.e.~if you want to return back to the initial system, and the composition of two of them, i.e.~two consecutive transformations (from the first to the second system and then from the second to the third one) are equivalent as if you go from the first to the third system with the composition of the two of them. This is indeed the definition of a group and it will be the topic of this chapter. In particular, we are interested in infinitesimal continuous transformations, which are related to Lie groups, and in what happens at the identity, which is related to Lie algebras.

\section{Groups}

    \begin{definition}[Group]
        A group is a set of elements $G = \{g_i\}$ associated with a composition map 
        \begin{equation*}
        \begin{aligned}
            & G \times G \rightarrow G \\ & (g_1, g_2) \mapsto g_1 g_2 ~,
        \end{aligned}
        \end{equation*}
        such that it satisfies the following properties $\forall g,g_1,g_2,g_3 \in G$
        \begin{enumerate}
            \item closure, i.e.
            \begin{equation*}
                g_1 g_2 \in G ~,
            \end{equation*}
            \item associativity, i.e.
            \begin{equation*}
                (g_1 g_2)g_3 = g_1(g_2 g_3) = g_1 g_2 g_3 ~,
            \end{equation*}
            \item identity element, i.e.
            \begin{equation}\label{idgroup}
                \exists! g_0 \in G \colon g_0 g = g g_0 = g ~,
            \end{equation}
            \item inverse element, i.e.
            \begin{equation}\label{invgroup}
                \exists! g^{-1} \in G \colon g^{-1} g = g g^{-1} = g_0 ~.
            \end{equation}
        \end{enumerate}
    \end{definition}

    \begin{definition}[Abelian group]
        A group is said to be abelian if it satisfies the additional property
        \begin{enumerate}
        \setcounter{enumi}{4}
            \item commutativity, i.e.
            \begin{equation*}
                g_1 g_2 = g_2 g_1 \quad \forall g_1, g_2 \in G ~.
            \end{equation*}
        \end{enumerate}
    \end{definition}

    \begin{definition}[Subgroup]
        A subgroup is a subset $H \subset G$ of a group which is also a group itself with closed restricted composition map.
    \end{definition}
    \noindent The trivial subgroup $\{g_0\}$ is always a subgroup.

\subsection{Examples of groups}

    \begin{example}[Set groups]
        Examples of set groups are
        \begin{enumerate}
            \item $\mathbb Z$, $\mathbb Q$, $\mathbb R$, $\mathbb C$ with composition map $+$ and identity element $0$,
            \item $\mathbb Q\setminus\{0\}$, $\mathbb R\setminus\{0\}$ with composition map $\times$ and identity element $1$,
            \item $\mathbb Z_n = \mathbb Z / n \mathbb Z =  \set{z \in \mathbb N, z \in [0, n-1]}{a+n = a}$ with composition map $+$ and identity element $0$.
        \end{enumerate}
    \end{example}

    \begin{example}[Subgroups]
        Examples of sugroups are
        \begin{enumerate}
            \item $\mathbb Q \subset \mathbb R \subset \mathbb C$,
            \item $\mathbb Z_n \subset \mathbb Z$.
        \end{enumerate}
    \end{example}

    \begin{example}[Matrix groups]
        Matrices are a non-abelian group with matrix multiplication as composition map, even though in order to contain the inverse matrix they must have non-zero determinant:
        \begin{enumerate}
            \item general linear group, i.e. 
            \begin{equation}\label{gln}
                GL(n) = \set{M \in Mat_{n \times n}(\mathbb R)}{\det M \neq 0} ~.
            \end{equation}
        \end{enumerate}
        In particular, given a fixed invertible matrix $B \in Mat_{n \times n}$, a subgroup of $GL(n)$ is the set of matrices $M$ which preserve this matrix, i.e. $M^T B M = B$:
        \begin{enumerate}
        \setcounter{enumi}{1}
            \item orthogonal group, i.e. 
                \begin{equation*}
                    O(n) = \set{R \in Mat_{n \times n}(\mathbb R)}{R^T \mathbb I R = \mathbb I} ~,
                \end{equation*}
                where $B$ is the Euclidean metric $\mathbb I$,
            \item Lorentz group, i.e. 
                \begin{equation*}
                    O(1, 3) = \set{\Lambda \in Mat_{4 \times 4}(\mathbb R)}{\Lambda^T \eta \Lambda = \eta} ~,
                \end{equation*} 
                where $B$ is the Minkovskian metric $\eta$,
            \item symplectic group, i.e. 
                \begin{equation*}
                    Sp(n) = \set{M \in Mat_{2n \times 2n}(\mathbb R)}{M^T J M = J} ~,
                \end{equation*}
                where $B$ is the symplectic matrix $J = \begin{bmatrix} 0 & 1 \\ -1 & 0 \\ \end{bmatrix}$.
        \end{enumerate}
        Over the complex field:
        \begin{enumerate}
        \setcounter{enumi}{3}
            \item unitary group, i.e. 
                \begin{equation*}
                    U(n) = \set{U \in Mat_{n \times n}(\mathbb C)}{U^{\dagger} U = \mathds 1} ~.
                \end{equation*}
        \end{enumerate}
        It is useful to impose the additional condition of unit determinant  $\det M = 1$, which gives rise to subgroups called special groups:
        \begin{enumerate}
        \setcounter{enumi}{4}
            \item special linear group, i.e. 
                \begin{equation*}
                    SL(n) = \set{M \in GL(n)}{\det M = 1} ~,
                \end{equation*}
            \item special orthogonal group, i.e. 
                \begin{equation}\label{so(3)}
                    SO(n) = \set{R \in O(n)}{\det R = 1} ~,
                \end{equation}
            \item special Lorentz group, i.e. 
                \begin{equation}\label{so(1,3)}
                    SO(1, 3) = \set{\Lambda \in O(1, 3)}{\det \Lambda = 1} ~,
                \end{equation}
            \item special unitary group, i.e. 
                \begin{equation*}
                    SU(n) = \set{U \in U(n)}{\det U = 1} ~.
                \end{equation*}
        \end{enumerate}
    \end{example}

\section{Lie groups and Lie algebras}

    \begin{definition}[Lie group]
        A Lie group is a group endowed with a manifold structure such that the composition and the inverse maps are smooth, i.e.
        \begin{equation*}
        \begin{aligned}
            \mu \colon & G \times G \rightarrow G 
            \\ & ~~ (x, y) \mapsto x^{-1} y ~.
        \end{aligned}
        \end{equation*}
    \end{definition}

    In Lie groups, we can introduce the notions of closeness and power series. Infact, an infinitesimal transformation $M$ can be Taylor expand into $M = \mathbb I + X$ such that its action on a vector is $Mv = v + \delta v$, where $\delta v = X v$. Therefore, the set of infinitesimal trasformations $X$ is the tangent space at the identity element $g_0 \in G$. It is a linear space, such that its elements could be different from those of the Lie group. Furthermore, it has also a composition map, which in general is not invertible. 

    \begin{definition}[Lie algebra]
        A Lie algebra is a linear space equipped with an anti-symmetric product, called Lie brackets
        \begin{equation*}
            [~,~] \colon \mathfrak g \times \mathfrak g \rightarrow \mathfrak g ~,
        \end{equation*}
        such that it satisfies the following properties $\forall X, Y, Z \in \mathfrak g$, $\forall \alpha, \beta \in \mathbb R$
        \begin{enumerate}
            \item linearity, i.e.
            \begin{equation*}
                [\alpha X + \beta Y, Z] = \alpha [X, Z] + \beta [Y, Z] ~,
            \end{equation*}
            \item anti-symmetry, i.e.
            \begin{equation}\label{anti}
                [X, Y] = - [Y, X] ~,
            \end{equation}
            \item Jacobi identity, i.e.
            \begin{equation}\label{jacobi}
                [X, [Y, Z]] + [Y, [Z, X]] + [Z, [X, Y]] = 0 ~.
            \end{equation}
        \end{enumerate}
    \end{definition}

    \begin{definition}[Structure constants]
        Given a basis $\{T_i\}$ of $\mathfrak g$, the Lie algebra can be completely determines by the structure constants $f_{ijk} \in \mathbb R$ in the following way
        \begin{equation}\label{stconsts}
            [T_i, T_j] = f_{ijk} T_k ~.
        \end{equation}
    \end{definition}

    Since they must obey the anti-symmetry condition, they satisfy the property 
    \begin{equation*}
        f_{ijk} = - f_{jik} ~.
    \end{equation*}

    \begin{proof}
        We put~\eqref{stconts} into~\eqref{anti}
        \begin{equation*}
        \begin{aligned}
            f_{ijk} T_k = [T_i, T_j] = - [T_j, T_i] = - f_{jik} T_k ~.
        \end{aligned}
        \end{equation*}
        Hence 
        \begin{equation*}
            f_{ijk} = - f_{jik} ~.
        \end{equation*}
    \end{proof}

    Moreover, since they must obey the Jacobi identity, they satisfy the property
    \begin{equation*}
        f_{ilm} f_{jkl} + f_{jlm} f_{kil} + f_{klm} f_{ijl} = 0 ~.
    \end{equation*}

    \begin{proof}
        We put~\eqref{stconts} into~\eqref{jacobi}
        \begin{equation*}
        \begin{aligned}
            0 & = [T_i, [T_j, T_k]] + [T_j, [T_k, T_i]] + [T_k, [T_i, T_j]] \\ & = [T_i, f_{jkl} T_l] + [T_j, f_{kil} T_l] + [T_k, f_{ijl} T_l] \\ & = f_{jkl} [T_i, T_l] + f_{kil} [T_j, T_l] + f_{ijl} [T_k, T_l] \\ & = f_{jkl} f_{ilm} T_m + f_{kil} f_{jlm} T_m + f_{ijl} f_{klm} T_m \\ & = (f_{jkl} f_{ilm} + f_{kil} f_{jlm}+ f_{ijl} f_{klm} ) T_m \\ & = (f_{ilm} f_{jkl} + f_{jlm} f_{kil} + f_{klm} f_{ijl}) T_m ~.
        \end{aligned}
        \end{equation*}
        Since it is true for any $T_m$
        \begin{equation*}
            f_{ilm} f_{jkl} + f_{jlm} f_{kil} + f_{klm} f_{ijl} = 0 ~.
        \end{equation*}
    \end{proof}

\subsection{Exponential map and Baker-Campbell-Hausdorff formula}

    \begin{definition}[Exponential map]
        Given the existence of a unique path $\gamma \colon \mathbb R \rightarrow G$ such that $\gamma(0) = g_0$ and $\gamma(1) = g$, which is a one-parameter subgroup $\set{\gamma(s)}{s \in \mathbb R}$ such that the tangent vector at $g_0$ is $X$, the exponential map is the map that gives a group element $G \ni g = \exp(X)$ which is finitely away from $X \in \textgoth g$.
    \end{definition}
    \noindent Notice that we can exponentiate only elements which are connected to the identity element. 

    \begin{theorem}[Baker-Campbell-Hausdorff formula]
        Let $X, Y$ be infinitesimal matrices. Then the matrix $Z$, which is solution of the equation 
    \begin{equation} \label{eq}
        \exp(X)\exp(Y) = \exp(Z) ~,
    \end{equation}
        is 
    \begin{equation}\label{bch}
        Z = X + Y + \frac{1}{2} \comm{X}{Y} + O(X^3) + O(Y^3) ~.
    \end{equation}
    \end{theorem}

    \begin{proof}
        Given a matrix $X$, its exponential $\exp(X)$ can be calculated with the Taylor expansion
        \begin{equation} \label{exp}
            \exp(X) = \sum_{n=0}^{\infty} \frac{X^n}{n!} ~.
        \end{equation}
        Another useful Taylor expansion is the logarithmic one 
        \begin{equation} \label{log}
            \log(X) = \sum_{n=1}^{\infty} \frac{{(-1)}^{n+1}}{n} {(X-1)}^n ~.
        \end{equation}
        We isolate Z in the equation~\eqref{eq} and Taylor expand~\eqref{log} with $X = \exp(X)\exp(Y)$
        \begin{equation*}
        \begin{aligned}
            Z & = \log(\exp{X}\exp{Y}) \\ & = \sum_{n=1}^{\infty} \frac{{(-1)}^{n+1}}{n} {(\exp(X)\exp(Y) -1)}^n \\ & = \sum_{n=1}^{\infty} \frac{{(-1)}^{n+1}}{n} {(\sum_{p=0}^{\infty} \frac{X^p}{p!} \sum_{q=0}^{\infty} \frac{X^q}{q!} - 1)}^n ~.
        \end{aligned}
        \end{equation*}
        We neglect terms of third or higher order in order to compute the second order expansion
        \begin{equation*}
        \begin{aligned}
            Z & = \sum_{n=1}^{2} \frac{{(-1)}^{n+1}}{n} {(\sum_{p=0}^{\infty} \frac{X^p}{p!} \sum_{q=0}^{\infty} \frac{X^q}{q!} - 1)}^n \\ & = \sum_{p=0}^{\infty} \frac{X^p}{p!} \sum_{q=0}^{\infty} \frac{X^q}{q!} - 1 -\frac{1}{2} \Big(\sum_{p=0}^{\infty} \frac{X^p}{p!} \sum_{q=0}^{\infty} \frac{X^q}{q!} - 1\Big)^2 \\ & = (1 + X + \frac{1}{2} X^2 + O(X^3)) (1 + Y + \frac{1}{2} Y^2 + O(Y^3)) - 1 \\ & \quad ~ -\frac{1}{2} {\Big ((1 + X + O(X^2)) (1 + Y + O(Y^2)) - 1 \Big )}^2 \\ & = X + \frac{1}{2} X^2 + Y + \frac{1}{2} Y^2 + XY - \frac{1}{2} (X^2 + Y^2 + XY + YX) + O(X^3) + O(Y^3) \\ & = X + Y + \frac{1}{2} (XY - YX) + + O(X^3) + O(Y^3) \\ & = X + Y + \frac{1}{2} \comm{X}{Y} + O(X^3) + O(Y^3) ~.
        \end{aligned}
        \end{equation*}
    \end{proof}

    The procedure to go from the Lie group to the Lie algebra is the following one: from the identity element of the Lie group, we move infinitesimally around it and find the generators of the Lie algebra. Conversely, to go from the Lie algebra to an element of the Lie group, we start from the generators of the Lie algebra, and by using the exponential map and the BCH formula, we find an element of the Lie group.

\subsection{Symmetries in quantum mechanics}

    To clarify how symmetries are generated by infinitesimal transformations, recall what happens in quantum mechanics. For example, infinitesimal spatial translations generat the momentum operator 
    \begin{equation*}
        \bra{x} \hat P \ket{\psi} = - i \pdv{}{x} \psi(x) ~,
    \end{equation*}
    which leads to a finite spatial translation
    \begin{equation*}
        \bra{x} \exp(- \frac{i}{\hbar} y \hat P) \ket{\psi} = \psi(x + y) ~,
    \end{equation*}
    or infinitesimal time translations generate the energy operator via the Schroedinger equation
    \begin{equation*}
        \hat H \ket{\psi(t)} = i \hbar \dv{}{t} \ket{\psi(t)} ~,
    \end{equation*}
    which leads to a finite time evolution
    \begin{equation*}
        \ket{\psi(t + \tau)} = \exp(- \frac{i}{\hbar} \hat H \tau) \ket{\psi(t)} ~.
    \end{equation*}
    
\chapter{Representations}

    In physics, it is useful to study how groups act on objects, in particular matrix groups act on vectors belonging to linear spaces of different dimensions. In this chapter, we will study how a group, a Lie group or a  Lie algebra acts on objects living in a linear space, which is called a representation.

\section{Representations}

    \begin{definition}[Automorphism]
        The automorphism $Aut(V)$ of a linear space $V$ is the set of invertible linear maps into the the linear space itself.
    \end{definition}
    \begin{definition}[Representation]
        A linear representation $(\rho, V$) of a group $G$ is a group homomorphism into the set of automorphisms on $V$
        \begin{equation*}
            \rho \colon G \rightarrow Aut(V) 
        \end{equation*}
        such that it satisfies the following property $\forall g_1, g_2 \in G$
        \begin{enumerate}
            \item composition map, i.e.
            \begin{equation}\label{comprep}
                \rho(g_1 g_2) = \rho(g_1) \rho(g_2) ~.
            \end{equation}
        \end{enumerate}
    \end{definition}
    \noindent It is possible to derive other properties $\forall g \in G$
    \begin{enumerate}
    \setcounter{enumi}{1}
        \item identity element, i.e.
        \begin{equation}\label{repid}
            \rho(g_0) = \mathbb I_V ~,
        \end{equation}
        \item inverse element, i.e.
        \begin{equation*}
            \rho(g^{-1}) = \rho^{-1}(g) ~.
        \end{equation*}
    \end{enumerate}
    \begin{proof}
        For the identity element, we choose $g_1 = g_0$ and $g_2 = g$ in~\eqref{comprep}
        \begin{equation*}
            \rho(g_0 g) = \rho(g_0) \rho(g) ~,
        \end{equation*}
        and by~\eqref{idgroup} and the property of the identity $\mathbb I_V$
        \begin{equation*}
            \rho(g_0 g) = \rho(g) = \mathbb I_V \rho(g) ~.
        \end{equation*}
        Hence 
        \begin{equation*}
            \rho(g_0) = \mathbb I_V ~.
        \end{equation*}

        For the inverse element, we choose $g_1 = g^{-1}$ and $g_2 = g$ in~\eqref{comprep}
        \begin{equation*}
            \rho(g^{-1} g) = \rho(g^{-1}) \rho(g) ~,
        \end{equation*}
        and by~\eqref{invgroup} and~\eqref{repid}
        \begin{equation*}
            \rho(g^{-1} g) = \rho(g_0) = \mathbb I_V ~.
        \end{equation*}
        Hence 
        \begin{equation*}
            \rho(g^{-1}) \rho(g) = \mathbb I_V ~,
        \end{equation*}
        or, equivalently,
        \begin{equation*}
            \rho(g^{-1}) = \rho^{-1}(g) ~,
        \end{equation*}
    \end{proof}

    If we restrict to finite-dimensional linear spaces, i.e. $\dim V = n$, given a basis of $V$, we have $Aut(V) \simeq GL(n)$, where $GL(n)$ is~\eqref{gln}. Furthermore, $\dim (\rho,V) = \dim V$. A representation $(\rho,V)$ acts as a linear transformation on a vector $v \in V$ as $\rho(g) v$.

    \begin{definition}[Reducible, irreducible representation]
        A representation $(\rho, ~V)$ is reducible if 
        \begin{equation*}
            \nexists \emptyset \neq U \subset V  ~,
        \end{equation*}
        such that $\forall u \in U$
        \begin{equation*}
            \rho(g) u \in U ~.
        \end{equation*}
        Otherwise, it is irreducible.
    \end{definition}

    \noindent A reducible representation can be always put in block triangle form, by choosing a suitable basis 
    \begin{equation*}
        \rho(g) = 
        \begin{bmatrix}
            \rho_1(g) & B(g) \\
            0 & \rho_2(g) \\
        \end{bmatrix} ~,
    \end{equation*}
    where the invariant subspace is $U = \{\begin{bmatrix} u & 0 \\ \end{bmatrix} \in V\}$. In this way, a smaller dimensional representation can be construct. If $B(g)=0$, the representation is completely reducible and decomposes into the direct sum of $\rho = \rho_1 \oplus \rho_2$. 

    \begin{definition}[Equivalent representations]
        Two representations $\rho_1$ and $\rho_2$ of the same dimension are equivalent if $\exists S$ invertible such that $\forall g \in G$
        \begin{equation*}
            \rho_2(g) = S^{-1} \rho_1(g) S ~,
        \end{equation*}
        which means that there exists a basis change that relates the two representations.
    \end{definition}

    \begin{definition}[Faithful representation]
        A representation $\rho$ is faithful if 
        \begin{equation*}
            g_1 \neq g_2 ~\Rightarrow~ \rho(g_1) \neq \rho(g_1) ~.
        \end{equation*}
    \end{definition}

    For non-faithful representations, there exists $H \subset G$ such that $\rho(h) = \mathbb I \quad \forall h \in H$.

    \begin{definition}
        Over the complex field, i.e. $\rho \colon G \rightarrow GL(n, ~\mathbb C)$, a representation is unitary if 
        \begin{equation}\label{unitrep}
            \rho(g^{-1}) = \rho^{-1}(g) = {\rho(g)}^{\dagger} ~,
        \end{equation}
        where $\rho^\dagger = {(\rho^*)}^T$.
    \end{definition}
    \noindent The physical meaning of unitary representations is that they preserve probabilities in quantum mechanics. 
    \begin{proof}
        Infact through~\eqref{unitrep}
        \begin{equation*}
            ||\rho(g) \ket{\psi}||^2 = \bra{\psi} \rho^\dagger (g) \rho(g) \ket{\\psi} = \bra{\psi} \rho^{-1} (g) \rho(g) \ket{\psi} = \braket{\psi}{\psi} = ||\ket{\psi}||^2 ~.
        \end{equation*}
    \end{proof}

    For any group, there exists a trivial $1$-dimensional representation, $\rho(g) = \mathbb I$. For any matrix group, there exists a non-trivial $1$-dimensional representation, $\rho(g) = \det(g)$. The defining representation is $\rho(g) = g$. For a $n \times n$ matrix group, the defining representation has the dimension $n$.

\section{Representations of Lie groups and Lie algebras}

    \begin{definition}[Endomorphism]
        The endomorphism $End(V)$ of a linear space $V$ is the set of (not in general invertible) linear maps into the the linear space itself.
    \end{definition}

    \begin{definition}[Representation of a Lie algebra]
        A representation of a Lie algebra is a Lie algebra homomorphism into the set of endomorphisms on $V$
        \begin{equation*}
            \rho_{\mathfrak g} \colon \mathfrak g \rightarrow End(V) ~,
        \end{equation*}
        such that it satifies the following compatibility condition with the Lie brackets $\forall X, Y \in \mathfrak g$
        \begin{equation}\label{condlie}
            \rho_{\mathfrak g} [X, Y] = \rho_{\mathfrak g} X \rho_{\mathfrak g} Y - \rho_{\mathfrak g} Y \rho_{\mathfrak g} X ~,
        \end{equation}
        or, given a set of generators $\{T_i\}$,
        \begin{equation*}
            \rho_{\mathfrak g} [T_i, T_j] = f_{ijk} \rho_{\mathfrak g} T_k ~.
        \end{equation*}
    \end{definition}

\subsection{From Lie group representations to Lie algebra representations}

    Any representation of a Lie group $(\rho, V)$ induces a representation of its Lie algebra. 
    \begin{proof}
        Infact, an element of the group $g = \exp(tX)$ gives a path of transformations on $V$ and we can define a representation on $V$ with 
        \begin{equation}\label{repalg}
            \rho_{\mathfrak g} (X) (v) = \dv{}{t} \rho (\exp(tX)) \Big \vert_{t=0} ~.
        \end{equation}
        Hence, $\rho_{\mathfrak g} X $ is the same size of $\rho(g)$. 
        
        To show that it respects the Lie brackets~\eqref{condlie}, we use~\eqref{bch},~\eqref{comprep},~\eqref{anti} and the linearity property of the representation
        \begin{equation*}
        \begin{aligned}
            \rho_{\mathfrak g} X \rho_{\mathfrak g} Y - \rho_{\mathfrak g} Y \rho_{\mathfrak g} X & = \dv{}{t} (\rho (\exp(tX)) \rho (\exp(tY)) - \rho (\exp(tY)) \rho (\exp(tX)) ) \Big \vert_{t=0} \\ & = \dv{}{t} \Big (\rho (\exp(t(X + Y + \frac{1}{2}[X, Y]))) - \rho (\exp(t(Y + X + \frac{1}{2}[Y, X]))) \Big ) \Big \vert_{t=0} \\ & = \dv{}{t} \Big (\rho (\exp(t(X + Y + \frac{1}{2}[X, Y] - Y - X - \frac{1}{2}[Y, X]))) \Big ) \Big \vert_{t=0} \\ & = \dv{}{t} \Big (\rho (\exp(t(\frac{1}{2}[Y, X] - \frac{1}{2}[Y,X]))) \Big ) \Big \vert_{t=0} \\ & = \dv{}{t} \Big (\rho (\exp(t(\frac{1}{2}[Y, X] + \frac{1}{2}[X,Y]))) \Big ) \Big \vert_{t=0}  \\ & = \dv{}{t} \Big (\rho (\exp(t[Y, X])) \Big ) \Big \vert_{t=0} \\ & = \rho_{\mathfrak g} [X, Y] ~.
        \end{aligned}
        \end{equation*}      
        Hence $(\rho_{\mathfrak g}, V)$ is representation of the Lie algebra. 
    \end{proof}

    For a unitary representation, the Lie algebra representations are anti-Hermitian matrices, 
    \begin{equation*}
        \rho_{\mathfrak g}^\dagger (X) = - \rho_{\mathfrak g} (X) ~.
    \end{equation*} 
    \begin{proof}
        Because of unitarity~\eqref{unitrep}, 
        \begin{equation*}
            \rho^\dagger(\exp(tX)) \rho(\exp(tX)) = 1 ~,
        \end{equation*} 
        we have, by using~\eqref{repalg}
        \begin{equation*}
        \begin{aligned}
            0 & = \dv{}{t} \rho(\exp(tX))^\dagger \rho(\exp(tX))\Big \vert_{t=0} \\ & = \rho^\dagger_{\mathfrak g} (X) \rho^\dagger (\exp(tX)) \rho(\exp(tX))\Big \vert_{t=0} + \rho_{\mathfrak g} (X) \rho(\exp(tX))^\dagger \rho(\exp(tX))\Big \vert_{t=0} \\ & = (\rho^\dagger_{\mathfrak g} (X) + \rho_{\mathfrak g} (X)) \underbrace{\rho^\dagger(\exp(tX)) \rho(\exp(tX))\Big \vert_{t=0}}_{1} \\ & = \rho_{\mathfrak g}^\dagger (X) + \rho_{\mathfrak g} (X) ~.
        \end{aligned}
        \end{equation*}
        Hence 
        \begin{equation*}
            \rho_{\mathfrak g}^\dagger (X) = - \rho_{\mathfrak g} (X) ~.
        \end{equation*}
    \end{proof}

    In physics, it is more convenient working with Hermitian matrices, then, by introducing the related represention 
    \begin{equation*}
        \tilde \rho_{\mathfrak g} (X) = i \rho_{\mathfrak g} ~,
    \end{equation*}
    the commutator becomes 
    \begin{equation*}
        [\tilde \rho_{\mathfrak g} (T_i), \tilde \rho_{\mathfrak g} (T_j)] = - [ \rho_{\mathfrak g} (T_i), \rho_{\mathfrak g} (T_j)] = - f_{ijk} \rho_{\mathfrak g} (T_k) = i f_{ijk} \rho_{\mathfrak g} (T_k) ~.
    \end{equation*}

\subsection{From Lie algebra representations to Lie group representations}

    Not all the representations of a Lie algebra extend to a representation of the Lie groups, because the Lie algebra gives information only locally, around the identity, while the Lie group could have different global topology. 

    However, if the group is simply connected, i.e.~all closed paths are contractible (deformable to a point), the Lie algebra representation is also a Lie group representation. Furthermore, if the Lie group $G$ is not simply-connected, there is always another Lie group $\tilde G$, the universal cover, which has the same representation as the Lie algebra. Universal cover means that there is a surjective projection homomorphism $\phi \colon \tilde G \rightarrow G$. An important property between the two groups is that they have the same Lie algebra $\tilde{\mathfrak g} = \mathfrak g$. 

    \begin{proof}
        Geometrically, there is always an open neighbourhood such that the projection map is a diffeomorphism. This means that the tangent space at every point is isomorphic to the tangent space at the identity.
    \end{proof}

\subsection{Representations in quantum mechanics}

    In quantum mechanics, a state is not uniquely defined. Infact it is characterized by a Hilbert space vector up to a phase factor, which is called a ray in the Hilbert space. Two states are indeed the same physical state if 
    \begin{equation*}
        \ket{\psi} = \exp(i \lambda) \ket{\phi}
    \end{equation*}
    where $\lambda$ is the phase factor.

    Hence, we can relax the group structure and define another kind of representation, different from the linear one, called a projective representation. 
    \begin{definition}[Projective representation]
        A projective representation satisfies the property
        \begin{equation*}
            \rho(g) \rho(h) = \exp(i \phi(g, h)) \rho(gh)
        \end{equation*}
        where $\phi(g, h) \in \mathbb R$. 
    \end{definition}

    All finite-dimensional unitary projective representations come from a unitary linear representation of the universal covering group. 
