\part{Lie groups and representations}

\chapter{Groups}

    The justification of a group can be found in the structure of a physics transformation: given two transformation, the composition of them should be defined together with the unit one, i.e.~nothing happens, and the inverse one, i.e.~if you want to return back to the initial system.

    \begin{definition}[Group]
        A group is a set of elements $G = \{g_i\}$ associated with a composition map 
        \begin{equation*}
            \cdot \colon G \times G \rightarrow G
        \end{equation*}
        satisfying the following properties
        \begin{enumerate}
            \item closure, i.e.
            \begin{equation*}
                g_1 g_2 \in G \quad \forall g_1, g_2 \in G
            \end{equation*}
            \item associativity, i.e.
            \begin{equation*}
                (g_1 g_2)g_3 = g_1(g_2 g_3) = g_1 g_2 g_3 \quad \forall g_1, g_2, g_3 \in G
            \end{equation*}
            \item unit element, i.e.
            \begin{equation*}
                \exists! g_0 \in G \colon g_0 g = g g_0 = g \quad \forall g \in G
            \end{equation*}
            \item inverse element, i.e.
            \begin{equation*}
                \exists! g^{-1} \in G \colon g^{-1} g = g g^{-1} = g_0 \quad \forall g \in G
            \end{equation*}
        \end{enumerate}
    \end{definition}

    \begin{definition}[Abelian group]
        A group is said to abelian if 
        \begin{enumerate}
        \setcounter{enumi}{4}
            \item commutativity, i.e.
            \begin{equation*}
                g_1 g_2 = g_2 g_1 \quad \forall g_1, g_2 \in G
            \end{equation*}
        \end{enumerate}
    \end{definition}

    \begin{definition}[Subgroup]
        A subgroup is a subset $H \subset G$ of a group which is also a group itself with closed restricted composition map.
    \end{definition}

    \begin{example}[Groups]
        Examples of groups are
        \begin{enumerate}
            \item $\mathbb Z$, $\mathbb Q$, $\mathbb R$, $\mathbb C$ with composition map $+$ and unit element $0$,
            \item $\mathbb Q\setminus\{0\}$, $\mathbb Q\setminus\{0\}$ with composition map $\times$ and unit element $1$,
            \item $\mathbb Z_n = \set{z \in [0, ~n-1]}{a+n = a}$ with composition map $+$ and unit element $0$.
        \end{enumerate}
    \end{example}

    \begin{example}[Matrix groups]
        Matrices are a non-abelian group with matrix multiplication as composition map:
        \begin{enumerate}
            \item $GL(n) = \set{M \in Mat_{n \times n}(\mathbb R)}{\det M \neq 0}$
        \end{enumerate}
        Given a fixed invertible $n\times n$ matrix $B$, a subgroup of $GL(n)$ is the set of matrices $M$ which preserve this matrix, i.e. $M^{t}BM = B$:
        \begin{enumerate}
        \setcounter{enumi}{1}
            \item $O(n) = \set{R \in Mat_{n \times n}(\mathbb R)}{R^{t} \mathds 1 R = \mathds 1}$ with Euclidean metric $B = \mathds 1$,
            \item $O(1, ~n-1) = \set{\Lambda \in Mat_{n \times n}(\mathbb R)}{\Lambda^{t} \eta \Lambda = \eta}$ with Minkovskian metric $B = \eta$.
        \end{enumerate}
        Over the complex field
        \begin{enumerate}
        \setcounter{enumi}{3}
        \item $U(n) = \set{U \in Mat_{n \times n}(\mathbb C)}{U^{\dagger} U = \mathds 1}$
        \end{enumerate}
        Imposing $\det M = 1$, we find the special groups
        \begin{enumerate}
        \setcounter{enumi}{4}
            \item $SL(n) = \set{M \in GL(n)}{\det M = 1}$,
            \item $SO(n) = \set{R \in O(n)}{\det R = 1}$,
            \item $SO(1,~n-1) = \set{\Lambda \in O(1, ~n-1)}{\det \Lambda = 1}$,
            \item $SU(n) = \set{M \in U(n)}{\det U = 1}$.
        \end{enumerate}
    \end{example}

\chapter{Lie groups}

    \begin{definition}[Lie group]
        A Lie group is a group endowed with a manifold structure such that composition and inverse are smooth maps, i.e.
        \begin{equation*}
        \begin{aligned}
            \mu \colon & G \times G \rightarrow G 
            \\ & (x, y) \mapsto x^{-1} y
        \end{aligned}
        \end{equation*}
    \end{definition}

    In Lie groups, we can introduce the notions of closeness and power series. The tangent space at $g_0 \in G$, i.e.~elements of the group infinitesimally away from the unit element, gives rise of the Lie algebra.

    \begin{definition}[Lie algebra]
        A Lie algebra is a linear space equipped with an anti-symmetric product, called Lie brackets
        \begin{equation*}
            [~,~] \colon \textgoth g \times \textgoth g \rightarrow \textgoth g
        \end{equation*}
        satisfying the following properties
        \begin{enumerate}
            \item linearity, i.e.
            \begin{equation*}
                [\alpha X + \beta Y, ~Z] = \alpha [X, ~Z] + \beta [Y, ~Z] \quad \forall X, ~Y, ~Z \in \textgoth g \quad \forall \alpha, ~\beta \in \mathbb R
            \end{equation*}
            \item anti-symmetry, i.e.
            \begin{equation*}
                [X, ~Y] = - [Y, ~X] \quad \forall X, ~Y \in \textgoth g
            \end{equation*}
            \item Jacobi identity, i.e.
            \begin{equation*}
                [X, ~[Y, ~Z]] + [Y, ~[Z, ~X]] + [Z, ~[X, ~Y]] = 0 \quad \forall X, ~Y, ~Z \in \textgoth g
            \end{equation*}
        \end{enumerate}
        A Lie algebra can be encoded with the structure constants $f_{ijk}$
        \begin{equation*}
            [T_i, ~T_j] = f_{ijk} T_k
        \end{equation*}
        where $\{T_i\}$ is a basis of $\textgoth g$.
    \end{definition}

    The exponential map helps to construct a group element that is finitely away as $g = \exp(X)$ from $X \in \textgoth g$, tied to the existence of a unique path $\gamma \colon \mathbb R \rightarrow G$ such that $\gamma(0) = g_0$ and $\gamma(1) = g$ which is a one-parameter subgroup $\set{\gamma(s)}{s \in \mathbb R}$ whose tangent vector at $g_0$ is $X$.  

    A useful formula is the Baker-Campbell-Hausdorff one, which connects the group composition with the Lie brackets
    \begin{equation}\label{BCH}
        \exp(X)\exp(Y) = \exp(X + Y + \frac{1}{2} [X, ~Y] + \ldots)
    \end{equation}

    Summarizing, from structure constants we define a Lie algebra, from the exponential map and the BCH formula we define Lie group elements in terms of generators of the Lie algebra $\{T_i\}$.

\chapter{Representations}

    In physics, it is useful to study how groups act on objects, in particular matrix groups act on vectors belonging to linear spaces.

    \begin{definition}[Representation]
        A linear representation of a group $G$ is a group homomorphism 
        \begin{equation*}
            \rho \colon G \rightarrow Aut(V) 
        \end{equation*}
        satisfying the following property
        \begin{enumerate}
            \item composition map, i.e.
            \begin{equation*}
                \rho(g_1 g_2) = \rho(g_1) \rho(g_2) \quad \forall g_1, g_2 \in G
            \end{equation*}
        \end{enumerate}
    \end{definition}

    It is possible to derive other properties
    \begin{enumerate}
    \setcounter{enumi}{1}
        \item unit element, i.e.
        \begin{equation*}
            \rho(g_0) = \mathds 1_V
        \end{equation*}
        \item inverse element, i.e.
        \begin{equation*}
            \rho(g^{-1}) = \rho^{-1}(g) \quad \forall g \in G
        \end{equation*}
    \end{enumerate}

    For finite-dimensional linear spaces, i.e. $\dim V = n$, and $Aut(V) \simeq GL(n)$ after picking a basis. Furthermore, $\dim (\rho, ~V) = \dim V$. A representation $(\rho, ~V)$ acts as a linear transformation on a vector $v \in V$ as $\rho(g) v$.

    \begin{definition}[Reducible, irreducible representation]
        A representation $(\rho, ~V)$ is reducible if 
        \begin{equation*}
            \nexists U \subset V ~\colon~ \rho(g) u \in U \quad \forall u \in U
        \end{equation*}
        otherwise, it is irreducible.
    \end{definition}

    A reducible representation can be always put in block triangle form, choosing a suitable basis 
    \begin{equation*}
        \rho(g) = 
        \begin{bmatrix}
            \rho_1(g) & B(g) \\
            0 & \rho_2(g) \\
        \end{bmatrix}
    \end{equation*}
    with the invariant subspace is $U = \{(u, ~0) \in V\}$. If $B(g)=0$, the representation is completely reducible and decomposes into the direct sum of $\rho = \rho_1 \oplus \rho_2$. 

    \begin{definition}[Equivalent representations]
        Two representations $\rho_1$ and $\rho_2$ of the same dimension are equivalent if 
        \begin{equation*}
            \exists S invertible ~\colon~ \rho_2(g) = S^{-1} \rho_1(g) S \quad \forall g \in G
        \end{equation*}
        which means that there exists a basis change that relate the representations.
    \end{definition}

    \begin{definition}[Faithful representation]
        A representation $\rho$ is faithful if 
        \begin{equation*}
            g_1 \neq g_2 ~\Rightarrow~ \rho(g_1) \neq \rho(g_1)
        \end{equation*}
    \end{definition}

    For non-faithful representations, there exists $H \subset G$ such that $\rho(h) = \mathds 1 \quad \forall h \in H$.

    \begin{definition}
        Over the complex field, i.e. $\rho \colon G \rightarrow GL(n, ~\mathbb C)$, a representation is unitary if 
        \begin{equation*}
            \rho(g^{-1}) = \rho^{-1}(g) = \rho(g)^{\dagger}
        \end{equation*}
    \end{definition}

    For any group, there exists a trivial 1-dimensional representation,  $\rho(g) = 1$. For any matrix group, there exists a non-trivial 1-dimensional representation, $\rho(g) = \det(g)$.

    \section{Representations of Lie Groups and Lie Algebras}

    The defining representation is $\rho(g) = g$. For a $n \times n$ matrix group, the defining representation has the dimension $n$.

    A representation of a Lie algebra is a set of endomorphisms on a vector space $V$ 
    \begin{equation*}
        \rho_{\mathfrak g} \colon \mathfrak g \rightarrow End(V)
    \end{equation*}
    where $End$ means the set of linear maps $V \rightarrow V$. Given a basis, it becomes the set of matrices. Furthermore, the compatibility with the Lie algebra require the further condition 
    \begin{equation*}
        \rho_{\mathfrak g} [X, Y] = \rho_{\mathfrak g} X \rho_{\mathfrak g} Y - \rho_{\mathfrak g} Y \rho_{\mathfrak g} X
    \end{equation*}
    or, given a set of generators $\{T_i\}$,
    \begin{equation*}
        \rho_{\mathfrak g} [T_i, T_j] = f_{ijk} \rho_{\mathfrak g} T_k
    \end{equation*}

\section{From Lie group rep to Lie algebra rep}

    Any representation of a Lie group $(\rho, V)$ induces a representation of its Lie algebra. Infact, an element of the group $g = \exp(tX)$ gives a path of transformations on $V$ and we can define a representation on $V$ with 
    \begin{equation*}
        \rho_{\mathfrak g} (X) (v) = \dv{}{t} \rho (\exp(tX)) \Big \vert_{t=0}
    \end{equation*}
    Hence, $\rho_{\mathfrak g} X $ is the same size of $\rho(g)$ and respects the Lie brackets. $(\rho_{\mathfrak g}, V)$ is representation of the Lie algebra. 

    For a unitary representation, the Lie algebra representations are anti-Hermitian matrices, i.e. $\rho_{\mathfrak g}^\dagger (X) = - \rho_{\mathfrak g} (X)$. 

    \begin{proof}
        Because of unitarity, 
        \begin{equation*}
            \rho(\exp(tX))^\dagger \rho(\exp(tX)) = 1
        \end{equation*}
        and deriving it 
        \begin{equation*}
            0 = \dv{}{t} \rho(\exp(tX))^\dagger \rho(\exp(tX))\Big \vert_{t=0} = \dv{}{t} (\rho_{\mathfrak g}^\dagger (X) + \rho_{\mathfrak g} (X)) \rho(\exp(tX))^\dagger \rho(\exp(tX))\Big \vert_{t=0} = \rho_{\mathfrak g}^\dagger (X) + \rho_{\mathfrak g} (X)
        \end{equation*}
        Hence 
        \begin{equation*}
            \rho_{\mathfrak g}^\dagger (X) = i \rho_{\mathfrak g} (X)
        \end{equation*}
    \end{proof}

    In Physics, it is more convenient working with Hermitian matrices, then introducing the related represention 
    \begin{equation*}
        \tilde \rho_{\mathfrak g} (X) = i \rho_{\mathfrak g} 
    \end{equation*}
    and the commutator becomes 
    \begin{equation*}
        [\tilde \rho_{\mathfrak g} (T_i), \tilde \rho_{\mathfrak g} (T_j)] = - [ \rho_{\mathfrak g} (T_i), \rho_{\mathfrak g} (T_j)] = - f_{ijk} \rho_{\mathfrak g} (T_k) = i f_{ijk} \rho_{\mathfrak g} (T_k)
    \end{equation*}

\section{From Lie algebra rep to Lie group rep}

    Not all the representations of a Lie algebra extend to a rep of the Lie groups, because the Lie algebra gives information only locally, around the identity, while the Lie group could have different global topology. 

    However, if the group is simply connected, i.e. all closed paths are contractible (deformable to a point), the Lie algebra rep is also a Lie group rep. If the Lie group $G$ is not simply-connected, there is always another Lie group $\tilde G$, the universal cover, which has the same rep as the Lie algebra. Cover because there is a surjective projection homomorphism $\phi \colon \tilde G \rightarrow G$. Furthermore, they have the same Lie algebra $\tilde{\mathfrak g} = \mathfrak g$. 

    \begin{proof}
        Geometrically, there is always an open neighbourhood such that the projection map is a diffeomorphism. This means that the tangent space at evry point is isomorphic to the tangent space at the identity.
    \end{proof}

\section{Reps in QM}

    In quantum mechanics, a state is not uniquely defined. Infact it is characterized by a Hilbert space vector up to a phase factor. Two states are the same physical state if 
    \begin{equation*}
        \ket{\psi} = \exp(i \lambda) \ket{\phi}
    \end{equation*}

    Hence, we can relax the definition of the group structure, allowing also a projective representation
    \begin{equation*}
        \rho(g) \rho(h) = \exp(i \phi(g, h)) \rho(gh)
    \end{equation*}
    where $\phi(g, h) \in \mathbb R$. 

    All unitary projective representations come from a linear representation of the covering group. 
