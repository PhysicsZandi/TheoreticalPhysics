\part{SO(3) and SO(1,~3)}

\chapter{SO(3)}

\section{SO(3) as a Lie group}

    In this chapter, we will study the three-dimensional rotations group $O(3)$. Computing the determinant, we can decomposed the Lie group into two parts according to the sign of it:

    \begin{equation*}
        \Rightarrow O(3) = \underbrace{\{\det R = +1\}}_{SO(3)} ~ \cup ~ \{\det R = -1\} = SO(3) ~ \cup ~ {\det R = -1}
    \end{equation*}

    Since there is no continuos path that connect the two parts and only $SO(3)$ contains the identity, we are going to study $SO(3)$ and recover the other one with a reflexion along an axis.

    Any $SO(3)$ rotation can be parametrized by a unit vector, perpendicular to the rotation plane, and a rotation angle $\theta$:\

    \begin{equation*}
        R(\theta, ~ \mathbf n)_{ij} = \cos \theta ~ \delta_{ij} + (1 - \cos \theta) ~ n_i n_j - \sin \theta ~ \epsilon_{ijk} n_k
    \end{equation*}

    Hence, an infinitesimal rotation $\delta \theta$ near the identity $R(\theta = 0, ~ \mathbf n) = \delta_{ij}$ is 

    \begin{equation*}
        R(\delta \theta, ~ \mathbf n)_{ij} = \underbrace{\cos \delta \theta }_{1} ~ \delta_{ij} + (1 - \underbrace{\cos \delta \theta }_{1}) ~ n_i n_j - \underbrace{\sin \delta \theta}_{\delta \theta} ~ \epsilon_{ijk} n_k = \delta_{ij} - \delta \theta ~ \epsilon_{ijk} n_k
    \end{equation*}

    and its action on an arbitrary vector $\mathbf v$ is 

    \begin{equation*}
        R(\delta \theta, ~ \mathbf n)_{ij} v_i = \delta_{ij} v_i - \delta \theta ~ \epsilon_{ijk} v_i n_k = \delta_{ij} v_i + \delta \theta ~ \epsilon_{jik} v_i n_k 
    \end{equation*}

    or 

    \begin{equation*}
        R(\delta \theta, ~ \mathbf n) \mathbf v = \mathbf v + \delta \theta ~ \epsilon{jik} v_i n_k 
    \end{equation*}


\section{SO(3) representations}

    In this chapter, 

\chapter{SO(1,~3)}

    The Lorentz group is defined by the matrices $\Lambda$ such that preserve the Minkovski metric 
    \begin{equation}\label{lorentz}
        \Lambda^\alpha_{\phantom \alpha \mu} \Lambda^\beta{\phantom \beta \nu} \eta_{\alpha \beta} = \eta_{\mu \nu}
    \end{equation}

    First, the Lorentz group can be decomposed into two parts according to their determinant
    \begin{equation*}
        \det (\Lambda^T \eta \Lambda) = \det \Lambda^T \det \eta \det \Lambda = \det^2 \Lambda = \det \eta = 1
    \end{equation*}
    Hence 
    \begin{equation*}
        \det \Lambda = \pm 1
    \end{equation*}
    and the Lorentz group can be written as
    \begin{equation*}
        O(1,3) = \underbrace{\{\det \Lambda = +1\}}_{SO(1,3)} \cup \{\det \Lambda = - 1\} 
    \end{equation*}
    where $SO(1,3)$ is called the proper Lorentz group.

    Second, the proper Lorentz group can be decomposed into two parts according to their (0, 0) component
    \begin{equation*}
        \eta_{00} = \Lambda^\alpha_{\phantom \alpha 0} \Lambda^\beta_{\phantom \beta 0} \eta_{\alpha \beta}
    \end{equation*}
    \begin{equation*}
        -1 = -(\Lambda^0_{\phantom 0 0})^2 + (\Lambda^i_{\phantom i i})^2
    \end{equation*}
    \begin{equation*}
        (\Lambda^0_{\phantom 0 0})^2 = 1 + (\Lambda^i_{\phantom i i})^2 \geq 1
    \end{equation*}
    
    Hence 
    \begin{equation*}
        \Lambda^0_{\phantom 0 0} \in ]\infty, -1] \cup [1, \infty[
    \end{equation*}
    and the proper Lorentz group can be written as
    \begin{equation*}
        SO(1,3) = \underbrace{\{\Lambda^0_{\phantom 0 0} \in ]\infty, -1]\}}_{SO(1,3)^+} \cup \{\Lambda^0_{\phantom 0 0} \in [1, \infty[\} 
    \end{equation*}
    where $SO(1,3)^+$ is called the proper orthochronous Lorentz group.

    From now on, only the proper orthochronous Lorentz group will be studied because is the only group containing the identity.

\section{Lie algebra: generators of $SO(1,3)^+$}

    Consider an infinitesimal Lorentz transformation around the identity 
    \begin{equation}\label{inf}
        \Lambda^{\mu}_{\phantom \mu \nu} = \delta^{\mu}_{\phantom \mu \nu} + \omega^{\mu}_{\phantom \mu \nu}
    \end{equation}
    where $\omega^{\mu}_{\phantom \mu \nu} \ll 1$ is an infinitesimal matrix.
    
    Using~\eqref{lorentz},
    \begin{equation*}
        (\delta^{\alpha}_{\phantom \alpha \mu} + \omega^{\alpha}_{\phantom \alpha \mu} )( \delta^{\beta}_{\phantom \beta \nu} + \omega^{\beta}_{\phantom \beta \nu} )\eta_{\alpha \beta} = \eta_{\mu \nu}
    \end{equation*}
    \begin{equation*}
        \delta^{\alpha}_{\phantom \alpha \mu} \delta^{\beta}_{\phantom \beta \nu} \eta_{\alpha \beta} + \delta^{\alpha}_{\phantom \alpha \mu} \omega^{\beta}_{\phantom \beta \nu} \eta_{\alpha \beta} + \omega^{\alpha}_{\phantom \alpha \mu} \delta^{\beta}_{\phantom \beta \nu} \eta_{\alpha \beta} + \omega^{\alpha}_{\phantom \alpha \mu} \omega^{\beta}_{\phantom \beta \nu} \eta_{\alpha \beta} = \eta_{\mu \nu}
    \end{equation*}
    \begin{equation*}
        \cancel{\eta_{\mu \nu}} + \omega_{\mu \nu} + \omega_{\nu\mu} + O(\omega^2) = \cancel{\eta_{\mu \nu}}
    \end{equation*}
    Hence, the matrices $\omega_{\mu\nu}$ are anti-symmetric
    \begin{equation*}
        \omega_{\mu \nu} = \omega_{\nu \mu}
    \end{equation*}

    Using the exponential map, a generic $SO(1,3)^+$ transformation can be written as
    \begin{equation*}
        \Lambda^{\mu}_{\phantom \mu \nu} = \exp(- \frac{i}{2} \omega^{\alpha \beta} M_{\alpha \beta})^\mu_{\phantom \mu \nu}
    \end{equation*}
    where $M_{\alpha \beta}$ are the generators of the Lie algebra $\mathfrak{so} (1,3)$. Since they must be antisymmetric, otherwise they would vanish, there are six independent generators of $\mathfrak{so} (1,3)$.
    

\section{Lie algebra: commutators of $SO(1,3)^+$}

    To find the explicit expression of the commutator of two generators, first it will be computed the following expression using~\eqref{inf}
    \begin{equation*}
    \begin{aligned}
        (\tilde \Lambda^{-1})^{\mu}_{\phantom \mu \alpha} (\Lambda^{-1})^{\alpha}_{\phantom \alpha \beta} \tilde \Lambda^{\beta}_{\phantom \beta \gamma} \Lambda^{\gamma}_{\phantom \gamma \nu} & = (\delta^{\mu}_{\phantom \mu \alpha} - \tilde \omega^{\mu}_{\phantom \mu \alpha})(\delta^{\alpha}_{\phantom \alpha \beta} - \omega^{\alpha}_{\phantom \alpha \beta})(\delta^{\beta}_{\phantom \beta \gamma} + \tilde \omega^{\beta}_{\phantom \beta \gamma})(\delta^{\gamma}_{\phantom \gamma \nu} + \omega^{\gamma}_{\phantom \gamma \nu}) + O(\tilde \omega^2) + O(\omega^2) \\ & = \delta^{\mu}_{\phantom \mu \alpha} \delta^{\alpha}_{\phantom \alpha \beta} \delta^{\beta}_{\phantom \beta \gamma} \delta^{\gamma}_{\phantom \gamma \nu} - \tilde \omega^{\mu}_{\phantom \mu \alpha} \delta^{\alpha}_{\phantom \alpha \beta} \delta^{\beta}_{\phantom \beta \gamma} \delta^{\gamma}_{\phantom \gamma \nu} - \delta^{\mu}_{\phantom \mu \alpha} \omega^{\alpha}_{\phantom \alpha \beta}\delta^{\beta}_{\phantom \beta \gamma}\delta^{\gamma}_{\phantom \gamma \nu} + \delta^{\mu}_{\phantom \mu \alpha} \delta^{\alpha}_{\phantom \alpha \beta} \tilde \omega^{\beta}_{\phantom \beta \gamma}\delta^{\gamma}_{\phantom \gamma \nu} \\ & \quad + \delta^{\mu}_{\phantom \mu \alpha} \delta^{\alpha}_{\phantom \alpha \beta} \delta^{\beta}_{\phantom \beta \gamma} \tilde \omega^{\beta}_{\phantom \beta \gamma} + \tilde \omega^{\mu}_{\phantom \mu \alpha} \omega^{\alpha}_{\phantom \alpha \beta} \delta^{\beta}_{\phantom \beta \gamma} \delta^{\gamma}_{\phantom \gamma \nu} - \tilde \omega^{\mu}_{\phantom \mu \alpha} \delta^{\alpha}_{\phantom \alpha \beta} \delta^{\beta}_{\phantom \beta \gamma} \omega^{\gamma}_{\phantom \gamma \nu} - \delta^{\mu}_{\phantom \mu \alpha} \omega^{\alpha}_{\phantom \alpha \beta} \tilde \omega^{\beta}_{\phantom \beta \gamma} \delta^{\gamma}_{\phantom \gamma \nu} \\ & \quad + \delta^{\mu}_{\phantom \mu \alpha} \tilde \omega^{\alpha}_{\phantom \alpha \beta} \delta^{\beta}_{\phantom \beta \gamma} \omega^{\gamma}_{\phantom \gamma \nu} + O(\tilde \omega^2) + O(\omega^2) \\ & = \delta^{\mu}_{\phantom \mu \nu} - \cancel{\tilde \omega^{\mu}_{\phantom \mu \nu}} - \cancel{\omega^{\mu}_{\phantom \mu \nu}} + \cancel{\tilde \omega^{\mu}_{\phantom \mu \nu}} + \cancel{\omega^{\mu}_{\phantom \mu \nu}} + \tilde \omega^{\mu}_{\phantom \mu \alpha} \omega^{\alpha}_{\phantom \alpha \nu} - \tilde \omega^{\mu}_{\phantom \mu \alpha} \omega^{\alpha}_{\phantom \alpha \nu} - \omega^{\mu}_{\phantom \mu \gamma} \tilde \omega^{\alpha}_{\phantom \alpha \nu} + \tilde \omega^{\mu}_{\phantom \mu \alpha} \omega^{\alpha}_{\phantom \alpha \nu} \\ & \quad + O(\omega^2) + O(\tilde \omega^2) \\ & = \delta^{\mu}_{\phantom \mu \nu} + \cancel{\tilde \omega^{\mu}_{\phantom \mu \alpha} \omega^{\alpha}_{\phantom \alpha \nu}} - \cancel{\tilde \omega^{\mu}_{\phantom \mu \alpha} \omega^{\alpha}_{\phantom \alpha \nu}} - \omega^{\mu}_{\phantom \mu \gamma} \tilde \omega^{\alpha}_{\phantom \alpha \nu} + \tilde \omega^{\mu}_{\phantom \mu \alpha} \omega^{\alpha}_{\phantom \alpha \nu} + O(\omega^2) + O(\tilde \omega^2) \\ & = \delta^{\mu}_{\phantom \mu \nu} - \omega^{\mu}_{\phantom \mu \gamma} \tilde \omega^{\alpha}_{\phantom \alpha \nu} + \tilde \omega^{\mu}_{\phantom \mu \alpha} \omega^{\alpha}_{\phantom \alpha \nu} + O(\omega^2) + O(\tilde \omega^2)
    \end{aligned}
    \end{equation*}
    and second using the BCH formula~\eqref{BCH}
    \begin{equation*}
    \begin{aligned}
        (\tilde \Lambda^{-1})^{\mu}_{\phantom \mu \alpha} (\Lambda^{-1})^{\alpha}_{\phantom \alpha \beta} \tilde \Lambda^{\beta}_{\phantom \beta \gamma} \Lambda^{\gamma}_{\phantom \gamma \nu} & = 
    \end{aligned}
    \end{equation*}
