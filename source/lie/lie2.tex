\part{$SO(3)$ and $SO^+(1,3)$}

\chapter{$SO(3)$}

    In this chapter, we will apply all the notions we learnt in the previous ones to~\eqref{so(3)}, the unit-determinant $3$-dimensional rotation group $SO(3)$.

\section{$SO(3)$ as a Lie group}
    
    The determinant of a $O(3)$ rotation has only two possible values, i.e. $\det R = \pm 1$. 
    \begin{proof}
        By using~\eqref{so(3)} and the product property of the determinant
        \begin{equation*}
            1 = \det \mathbb I = \det R^T R = \det R^T \det R = {\det}^2 R ~.
        \end{equation*}
        Hence 
        \begin{equation*}
            \det R = \pm 1 ~.
        \end{equation*}
    \end{proof}

    We can decomposed $O(3)$ into two parts according to the sign of the determinant
    \begin{equation*}
        \Rightarrow O(3) = \underbrace{\{\det R = +1\}}_{SO(3)} \cup \{\det R = -1\} = SO(3) \cup \{\det R = -1\} ~.
    \end{equation*}
    Since there is no continuous path that connect the two parts and only $SO(3)$ contains the identity, we are going to study $SO(3)$ and recover the other one with a reflexion along an axis.

    Any $SO(3)$ rotation can be parametrized by a unit vector, perpendicular to the rotation plane, and a rotation angle $\theta$ by the formula
    \begin{equation*}
        R{(\theta, \mathbf n)}_{ij} = \cos \theta \delta_{ij} + (1 - \cos \theta) n_i n_j - \sin \theta \epsilon_{ijk} n_k
    \end{equation*}

    \begin{proof}
        Consider a unit vector $n$ and decompose another vector $v$into it 
        \begin{equation*}
            \mathbf v = \mathbf v_\parallel + \mathbf v_\perp ~,
        \end{equation*}
        where 
        \begin{equation*}
            v_\parallel = (\mathbf v \cdot \mathbf n) \mathbf n ~,
        \end{equation*} 
        and 
        \begin{equation*}
            \mathbf v_\perp = \mathbf v - \mathbf v_\parallel = \mathbf v - (\mathbf v \cdot \mathbf n) \mathbf n ~.
        \end{equation*}

        We exploit the triple vector product formula, with the unit norm of the unit vector
        \begin{equation*}
            \mathbf n \times (\mathbf n \times \mathbf v) = (\mathbf n \cdot \mathbf v) \mathbf n - \underbrace{(\mathbf n \cdot \mathbf n)}_{1}\mathbf v = (\mathbf n \cdot \mathbf v) \mathbf n - \mathbf v
        \end{equation*}
        and obtain 
        \begin{equation*}
            \mathbf v - (\mathbf n \cdot \mathbf v) \mathbf n = - \mathbf n \times (\mathbf n \times \mathbf v)
        \end{equation*}

        We put it in the previous equation and find 
        \begin{equation*}
            \mathbf v_\perp = - \mathbf n \times (\mathbf n \times \mathbf v)
        \end{equation*}

        Furthermore, we have the identity 
        \begin{equation*}
            \mathbf n \times \mathbf v = \mathbf n \times (\mathbf v_\parallel + \mathbf v_\perp) = \underbrace{\mathbf n \times \mathbf v_\parallel}_{0} + \mathbf n \times \mathbf v_\perp = \mathbf n \times \mathbf v_\perp
        \end{equation*}

        Now, we rotate around the unit vector $\mathbf n$ and go into polar coordinates 
        \begin{equation*}
            \mathbf {v'}_\perp = \cos \theta \mathbf v_\perp + \sin \theta \mathbf n \times \mathbf v_\perp = \cos \theta \mathbf v_\perp + \sin \theta \mathbf n \times \mathbf v
        \end{equation*}
        and 
        \begin{equation*}
            \mathbf {v'}_\parallel = \mathbf v_\parallel
        \end{equation*}

        Hence
        \begin{equation*}
        \begin{aligned}
            \mathbf {v'} & = R(\theta, \mathbf n) \mathbf v \\ & = \mathbf {v'}_\parallel + \mathbf {v'}_\perp \\ & = \mathbf v_\parallel + \cos \theta \mathbf v_\perp + \sin \theta \mathbf n \times \mathbf v \\ & = \mathbf v_\parallel + \cos \theta (\mathbf v - \mathbf v_\parallel) + \sin \theta \mathbf n \times \mathbf v \\ & = \cos \theta \mathbf v + (1 - \cos \theta)\mathbf v_\parallel + \sin \theta \mathbf n \times \mathbf v \\ &  = \cos \theta \mathbf v + (1 - \cos \theta) \mathbf v_\parallel + \sin \theta \mathbf n \times \mathbf v \\ & = \cos \theta \mathbf v + (1 - \cos \theta) (\mathbf v \cdot \mathbf n) \mathbf n  + \sin \theta \mathbf n \times \mathbf v 
        \end{aligned}
        \end{equation*}
        
        Since it is true for any vector $\mathbf v$, in index notation it becomes
        \begin{equation*}
            R{(\theta, \mathbf n)}_{ij} = \cos \theta \delta_{ij} + (1 - \cos \theta) n_i n_j - \sin \theta \epsilon_{ijk} n_k
        \end{equation*}
    \end{proof}

    An infinitesimal rotation $\delta \theta$ near the identity $R(\theta = 0, \mathbf n) = \delta_{ij}$ is 
    \begin{equation*}
        R{(\delta \theta, \mathbf n)}_{ij} = \underbrace{\cos \delta \theta }_{1} \delta_{ij} + (1 - \underbrace{\cos \delta \theta }_{1}) n_i n_j - \underbrace{\sin \delta \theta}_{\delta \theta} \epsilon_{ijk} n_k = \delta_{ij} - \delta \theta ~ \epsilon_{ijk} n_k
    \end{equation*}
    and its action on an arbitrary vector $\mathbf v$ is 
    \begin{equation*}
        R{(\delta \theta, \mathbf n)}_{ij} v_i = \delta_{ij} v_i - \delta \theta \epsilon_{ijk} v_i n_k = \delta_{ij} v_i + \delta \theta \epsilon_{jik} v_i n_k 
    \end{equation*}
    or 
    \begin{equation*}
        R{(\delta \theta, \mathbf n)} \mathbf v = \mathbf v + \delta \theta \epsilon_{jik} v_i n_k 
    \end{equation*}

    The generators $L_i$ of the Lie algebra, which are a basis of $\mathfrak{so}(3)$, are
    \begin{equation*}
        {(L_i)}_{jk} = - \epsilon_{ijk}
    \end{equation*}

    \begin{proof}
        If we substitute the unit vector $\mathbf n$ with the Euclidean unit vectors $\mathbf i$, $\mathbf j$ and $\mathbf k$, we find
        \begin{equation*}
            R(\delta \theta, \begin{bmatrix} 1 \\ 0 \\ 0 \\ \end{bmatrix} ) = \mathbf v \delta \theta \begin{bmatrix} 0 \\ -v_3 \\ v_2 \end{bmatrix} = \mathbf v + \delta \theta \underbrace{\begin{bmatrix} 0 & 0 & 0 \\ 0 & 0 & -1 \\ 0 & 1 & 0 \end{bmatrix}}_{L_1} \mathbf v
        \end{equation*}
        \begin{equation*}
            R(\delta \theta, \begin{bmatrix} 0 \\ 1 \\ 0 \\ \end{bmatrix} ) = \mathbf v \delta \theta \begin{bmatrix} v_3 \\ 0 \\ -v_1 \end{bmatrix} = \mathbf v + \delta \theta \underbrace{\begin{bmatrix} 0 & 0 & 1 \\ 0 & 0 & 0 \\ -1 & 0 & 0 \end{bmatrix}}_{L_2} \mathbf v
        \end{equation*}
        \begin{equation*}
            R(\delta \theta, \begin{bmatrix} 0 \\ 0 \\ 1 \\ \end{bmatrix} ) = \mathbf v \delta \theta \begin{bmatrix} -v_2 \\ v_1 \\ 0 \\ \end{bmatrix} = \mathbf v + \delta \theta \underbrace{\begin{bmatrix} 0 & -1 & 0 \\ 1 & 0 & 0 \\ 0 & 0 & 0 \end{bmatrix}}_{L_1} \mathbf v
        \end{equation*}
    \end{proof}

    Any rotation belonging ot $SO(3)$ can be written with the exponential map 
    \begin{equation*}
        R(\theta, \mathbf n) = \exp(\theta n_i L_i) = \exp(\theta \mathbf n \cdot \mathbf L)
    \end{equation*}

    The Lie algebra is $\mathfrak{so}(3) = \{\textnormal{anti-symmetric traceless} 3 \times 3 \textnormal{matrices}\}$, a $3$-dimensional real vector space.
    \begin{proof}
        Orthogonality require 
        \begin{equation*}
            \mathbb I = R(\theta, \mathbf n)^T R(\theta, \mathbf n) = \exp(\theta n_i (L_i^T + L_i) + \ldots)
        \end{equation*}
        then the generators are antisymmetric 
        \begin{equation*}
            L^T_i = - L^i
        \end{equation*}
        and the determinant condition $\det R = 1$ means 
        \begin{equation*}
            tr L_i = 0 
        \end{equation*}
    \end{proof}

    The commutation relations are 
    \begin{equation*}
        [L_i, L_j] = \epsilon_{ijk} L_k
    \end{equation*}

    In physics, it is useful define generators as $J_i = i L_i$. Thus 
    \begin{equation*}
        R(\theta, \mathbf n) = \exp(- i \theta \mathbf n \cdot \mathbf J)
    \end{equation*}
    and the commutation relations become
    \begin{equation*}
        [J_i, J_j] = i \epsilon_{ijk} J_k
    \end{equation*}
    
\section{$\mathfrak{so}(3)$ as a Lie algebra}

    Now we study the finite dimensional representation of $SO(3)$. In the previous section, we used the Cartesian basis: Therefore, by a basis change of the Lie algebra, we introduce the ladders operators  
    \begin{equation*}
        J_\pm = J_1 \pm i J_2
    \end{equation*}
    such that they satisfy the commutation relations 
    \begin{equation*}
        [J_3, J_\pm] = \pm J_\pm \qquad [J_+, J_-] = 2 J_3
    \end{equation*}
    Since $J_i$ are Hermitian, so $J_\pm$ are. 

    \begin{proof}
        First, the commutation relations between $J_3$ and $J_\pm$ are 
        \begin{equation*}
            [J_3, J_\pm] = [J_3, J_1 \pm i J_2] = [J_3, J_1] \pm i [J_3, J_2] = i J_2 \pm i (- i J_1) = J_1 \pm i J_2 = \pm J_\pm ~.
        \end{equation*}

        Secondly, the commutation relations between $J_+$ and $J_-$ are 
        \begin{equation*}
            [J_+, J_-] = [J_1 + i J_2, J_1 - i J_2] = \underbrace{[J_1, J_1]}_0 + i [J_2, J_1] - i [J_1, J_2] + \underbrace{[J_2, J_2]}_0 = i (-i J_3) - i (i J_3) = 2 J_3
        \end{equation*}
    \end{proof}
    
    We define the Hermitian square modulus operator 
    \begin{equation*}
        J^2 = J_1^2 + J_2^2 + J_3^2
    \end{equation*}
    It is a Casimir operator, i.e.~it commutes with all the Lie algebra generators.

    \begin{proof}
        First, the commutation relations between $J^2$ and $J_\pm$ are 
        \begin{equation*}
        \begin{aligned}
            [J^2, J_\pm] & = [J^2, J_1 \pm i J_2] \\ & = [J^2_1 + J^2_2 + J^2_3, J_1 \pm i J_2] \\ & = \underbrace{[J^2_1, J_1]}_0 + [J^2_2, J_1] + [J^2_3, J_1] \pm i[J^2_1, J_2] \pm i \underbrace{[J^2_2, J_2]}_0 \pm i [J^2_3, J_2] \\ & = J_2 [J_2, J_1] + [J_2, J_1] J_2 + J_3 [J_3, J_1] + [J_3, J_1] J_3 \pm i J_1 [J_1, J_2] \pm i [J_1, J_2] J_1 \pm i J_3 [J_3, J_2] \pm i [J_3, J_2] J_3 \\ & = - i \underbrace{J_2 J_3}_0 + - i \underbrace{J_3 J_2}_0 + i \underbrace{J_3 J_2}_0 + i \underbrace{J_2 J_3}_0 \mp \underbrace{J_1 J_3}_0 \mp \underbrace{J_3 J_1}_0 \pm \underbrace{J_3 J_1}_0 \pm \underbrace{J_1 J_3}_0 = 0
        \end{aligned}
        \end{equation*}

        Second, the commutation relations between $J^2$ and $J_i$ are 
        \begin{equation*}
        \begin{aligned}
            [J^2, J_i] & = [J^2_1 + J^2_2 + J^2_3, J_i] \\ & = [J_1^2, J_i] + [J_2^2, J_i] + [J_3^2, J_i] \\ & = J_1 [J_1, J_i] + [J_1, J_i] J_1 + J_2 [J_2, J_i] + [J_2, J_i] J_2 + J_3 [J_3, J_i] + [J_3, J_i] J_3
        \end{aligned}
        \end{equation*}
        If $i = 1$ 
        \begin{equation*}
        \begin{aligned}
            [J^2, J_1] & = J_1 \underbrace{[J_1, J_1]}_0 + \underbrace{[J_1, J_1]}_0 J_1 + J_2 [J_2, J_1] + [J_2, J_1] J_2 + J_3 [J_3, J_1] + [J_3, J_1] J_3 \\ & = - i \underbrace{J_2 J_3}_0 - i \underbrace{J_3 J_2}_0 + i \underbrace{J_3 J_2}_0 + i \underbrace{J_2 J_3}_0 = 0
        \end{aligned}
        \end{equation*}
        and similarly for $i = 2, 3$.
    \end{proof}

\subsection{Eigenvalues}

    Since $J_3$ and $J^2$ are commuting Hermitian matrices on a Hilbert space, the spectral theorem garanties us that there is a basis $\{\ket{\xi, m}\}$ such that it is orthonormal $\braket{\xi,m}{\xi', m'} = \delta_{mm'} \delta_{\xi\xi'}$ and it is a simultaneously eigenvector of the two operators, i.e. $J^2 \ket{\xi,m} = \xi \ket{\xi, m}$ and $J_3 \ket{\xi,m} = m \ket{\xi, m}$. There could be degeneracies, i.e.~same eigenvalue could have more than one eigenket, but the ladder operators prevent that because we can restrict to $1$-dimensional subspaces which correspond to same labels.

    By quantum mechanics similarity, we have the following eigenvalue relations 
    \begin{equation}\label{j3}
        J^2 \ket{j ,m} = j (j+1) \ket{j, m} \qquad J_3 \ket{j, m} = m \ket{j, m}
    \end{equation}
    where $j \in \frac{\mathbb N_0}{2}$ and $|m| \leq j$. Furthermore, the ladder operators act on these kets by 
    \begin{equation*}
        J_\pm \ket{j, m} = \sqrt{(j \mp m) (j \pm m + 1)} \ket{j, m \pm 1}
    \end{equation*}
    Notice that different values of $j$ correspons to different representations, which have dimension $2j + 1$. The irreducible representations are labelled by the eigenvalues of Casimir operator.

    For $j = 1$, we have a $3$-dimensional rep. For $j > 1$, we have a tensor product of irreps $V_1^{\otimes j} = V_1 \otimes \ldots \otimes V_1$. However half-integers are not a rep of $SO(3)$.

    \begin{proof}
        The entries of the representation matrices $M = \rho_j (X)$ for any $X \in \mathfrak{so}(3)$ are 
        \begin{equation*}
            M_{ab} = \bra{j, a} M \ket{j, b}
        \end{equation*} 
        and its action on kets is
        \begin{equation*}
            M \ket{j, m} = \sum_k \ket{j, k} M_{km}
        \end{equation*}
        By using the exponential map, we can find a representation of an element which could be in the universal cover. Infact, if we could take for istance the second unit vector $n_2$, we have 
        \begin{equation*}
            \rho_j {(R(\theta))}_{ab} = \bra{j, a} \exp(- i \theta J_2^{(j)}) \ket{j, b} = \bra{j, a} \exp(-\frac{1}{2} \theta (J_+^{(j)} - J_-^{(j)})) \ket{j, b}
        \end{equation*}
        where $J_k^{(j)} = \rho_j (L_k)$. 
    
        Explicitly, for $\theta = \pi$ we have 
        \begin{equation*}
            \rho(R(\pi))_{ab} = (-1)^{j-b} \delta_{a, -b}
        \end{equation*}
        Hence, a full rotation $2\pi$ is 
        \begin{equation*}
        \begin{aligned}
            \rho_j{(R(\pi) R(\pi))}_{ab} & = {(\rho_j(R(\pi)) \rho_j(R(\pi)))}_{ab} \\ & = \sum_c \rho{(R(\pi))}_{ac} \rho_j{(R(\pi))}_{cb} \\ & = \sum_c (-1)^{j-c} \delta_{a, -c} {(-1)}^{j-b} \delta_{c, -b} \\ & = \sum_c (-1)^{2j -c-b} \delta_{a,-c} \delta_{c, -b} \\ & = {(-1)}^{2j} \delta_{ab}
        \end{aligned}
        \end{equation*}
        Therefore $R(\pi) R(\pi) \neq \mathbb I$ and a full rotation is not the unit matrix for half-integer representations. 
    \end{proof}
    
    Observe that, since the difference is just a phase factor, it is indeed a projective rep.

\section{$SO(3)$ and $SU(2)$}

    The spinor representation of $SU(2)$ with $j=\frac{1}{2}$ is a $2$-dimensional representation with basis
    \begin{equation*}
        \ket{e_1} = \ket{\frac{1}{2}, \frac{1}{2}} \qquad \ket{e_2} = \ket{\frac{1}{2}, - \frac{1}{2}}
    \end{equation*}
    Hence, the matrix representation of the $J_3$ operator is
    \begin{equation*}
        J_3^{(\frac{1}{2})} = \begin{bmatrix}
            \frac{1}{2} & 0 \\ 0 & - \frac{1}{2} \\
        \end{bmatrix}
    \end{equation*}
    of the ladder operators are 
    \begin{equation*}
        J_-^{(\frac{1}{2})} = \begin{bmatrix}
            0 & 0 \\ 1 & 0 \\
        \end{bmatrix} \quad J_-^{(\frac{1}{2})} = \begin{bmatrix}
            0 & 1 \\ 0 & 0 \\
        \end{bmatrix}
    \end{equation*}
    and of the $J_i$ operators are
    \begin{equation*}
        J_1^{(\frac{1}{2})} = \begin{bmatrix}
            0 & \frac{1}{2} \\ \frac{1}{2} & 0 \\
        \end{bmatrix} \quad J_2^{(\frac{1}{2})} = \begin{bmatrix}
            0 & - \frac{i}{2} \\ \frac{i}{2} & 0 \\
        \end{bmatrix}
    \end{equation*}

    \begin{proof}
        First, the $J_3$ operator with~\eqref{j3}
        \begin{equation*}
            \bra{e_1} J_3^{(\frac{1}{2})} \ket{e_1} = - \bra{e_2} J_3^{(\frac{1}{2})} \ket{e_2} = - \frac{1}{2} \quad \bra{e_1} J_3^{(\frac{1}{2})} \ket{e_2} = \bra{e_2} J_3^{(\frac{1}{2})} \ket{e_1} = 0
        \end{equation*}
    
        Second, the ladder operators 
        \begin{equation*}
            J_-^{(\frac{1}{2})} \ket{e_1} =  \ket{e_2} \quad J_+^{(\frac{1}{2})} \ket{e_1} = J_-^{(\frac{1}{2})} \ket{e_2} = = 0 \quad J_+^{(\frac{1}{2})} \ket{e_2} =  \ket{e_1}
        \end{equation*}
        
        Third, the $J_i$ operators, with the inverse relations of~\eqref{jpm}
        \begin{equation*}
            J_1^{(\frac{1}{2})} = \frac{1}{2} (J_+^{(\frac{1}{2})} + J_-^{(\frac{1}{2})}) \quad J_2^{(\frac{1}{2})} = \frac{1}{2i} (J_+^{(\frac{1}{2})} - J_-^{(\frac{1}{2})}) ~.
        \end{equation*}
    \end{proof}    

    Notice that 
    \begin{equation*}
        J_i^{(\frac{1}{2})} = \frac{1}{2} \sigma_i
    \end{equation*}
    where $\sigma_i$ are the Pauli matrices 
    \begin{equation*}
        \sigma_1 = \begin{bmatrix} 0 & 1 \\ 1 & 0 \\ \end{bmatrix} \quad \sigma_2 = \begin{bmatrix} 0 & -i \\ i & 0 \\ \end{bmatrix} \quad \sigma_3 = \begin{bmatrix} 1 & 0 \\ 0 & -1 \\ \end{bmatrix}
    \end{equation*}
    such that thet satisfy the commutation relations
    \begin{equation*}
        [\sigma_i, \sigma_j] = 2 i \epsilon_{ijk} \sigma_k
    \end{equation*}
    This three matrices are linearly independent and they generate the $SU(2)$ group, i.e.~the set of Hermitian traceless matrices. 
    
    \begin{proof}
        This can be seen by writing down a complex $2 \times 2$ matrix 
        \begin{equation*}
            \begin{pmatrix}
                a & b \\ c & d \\
            \end{pmatrix}
        \end{equation*}
        and by requiring the condition to be a $SU(2)$ matrix, i.e. $b = c^*$ because of hermiticy and $a = -d$ because of tracelessness. Hence
        \begin{equation*}
            \begin{pmatrix}
                a & b \\ b^* & -a \\
            \end{pmatrix}
        \end{equation*}
        which leaves three degrees of freedom, because $a$ is real. Now we decompose into the real and the complex part of $b$
        \begin{equation*}
            \begin{pmatrix}
                a & b_r - i b_i \\ b_r + i b_i & - a\\
            \end{pmatrix} = b_r \begin{pmatrix}
                0 & 1 \\ 1 & 0 \\
            \end{pmatrix} + b_i \begin{pmatrix}
                0 & -i \\ i & 0 \\
            \end{pmatrix} + a \begin{pmatrix}
                1 & 0 \\ 0 & -1 \\
            \end{pmatrix} = b_r \sigma_1 + b_i \sigma_2 + a \sigma_3
        \end{equation*}
        which proves that they are indeed a basis for $\mathfrak{su}(2)$
    \end{proof}. 

    Thus a matrix $U \in SU(2)$ can be written in terms of the exponential map 
    \begin{equation*}
        U(\theta, \mathbf n) = \exp(- i \theta \mathbf n \cdot \mathbf J^{(\frac{1}{2})}) = \exp(- \frac{i}{2} \theta \mathbf n \cdot \mathbf \sigma) ~,
    \end{equation*}
    or explicitly 
    \begin{equation*}
        U(\theta, \mathbf n) = \begin{bmatrix}
            \cos \frac{\theta}{2} - i n_3 \sin \frac{\theta}{2} & -i(n_1 - i n_2) \sin \frac{\theta}{2} \\ - i (n_1 + i n_2) \sin \frac{\theta}{2} & \cos \frac{\theta}{2} + i n_3 \sin \frac{\theta}{2} \\
        \end{bmatrix} ~.
    \end{equation*}

    Alternatively, we can show that $SU(2) \simeq S^3$, by keeping complex parameters $a$ and $b$
    \begin{equation}\label{su2mat}
        U(a, b) = \begin{pmatrix}
            a & b \\ -b^* & a^* \\
        \end{pmatrix}
    \end{equation}
    such that $\det U = |a|^2 + |b|^2 = 1$.

\subsection{$\mathfrak{so}(3)$ = $\mathfrak{su}(2)$}

    In order to see that the Lie algebras are the same, we introduce a map between the two linear spaces
    \begin{equation*}
        X = \sum_i x_i \sigma_i \mapsto \mathbf x = \begin{pmatrix}
            x_1 \\ x_2 \\ x_3 \\
        \end{pmatrix} \in \mathbb R^3
    \end{equation*}
    with the Euclidean scalar product 
    \begin{equation*}
        \mathbf x \cdot \mathbf y = \frac{1}{2} tr(XY)
    \end{equation*}

    Thus, given an element $X \in \mathfrak{su}(2)$ we have for any $u \in SU(2)$ both the tracelessness and the hermiticy properties
    \begin{equation*}
        tr(U X U^{-1}) = tr(U^{-1} U X) = 0 \quad (UXU^{-1})^\dagger = (U^{-1})^\dagger X^\dagger U^\dagger = U X U^{-1}
    \end{equation*}
    and an action $\mathfrak{su}(2) \simeq \mathbb R^3 \rightarrow \mathbb R^3$ such that
    \begin{equation*}
        X \mapsto U X U^{-1}
    \end{equation*} 
    Furthermore, it is orthogonal 
    \begin{equation*}
        \mathbf x \cdot \mathbf y = \frac{1}{2} tr (XY) \mapsto \frac{1}{2} tr (UXU^{-1} U Y U^{-1}) = \frac{1}{2} (XY) = \mathbf x \cdot \mathbf y
    \end{equation*}
    and, since $SU(2)$ is connected, the identities are mapped and this defines a $3$-dimensional rep
    \begin{equation*}
        \rho_{ad} \colon SU(2) \rightarrow SO(3) \subset Aut(V) \quad V \ni X \mapsto \rho_{ad} (U) (X) = U X U^{-1}
    \end{equation*}
    called the adjoint representation. In this case, it coincides with the spin $j = 1$ representation of $SO(3)$.

    Explicitly, by using~\eqref{su2mat}
    \begin{equation*}
        U(a,b) \sigma_1 U(a,b)^{-1} = Re(a^2 - b^2) \sigma_1 - Im(a^2 - b^2) \sigma_2 + 2 Re (a b^*) \sigma_3
    \end{equation*}
    \begin{equation*}
        U(a,b) \sigma_2 U(a,b)^{-1} = Im(a^2 + b^2) \sigma_1 + Re(a^2 + b^2) \sigma_2 + 2 Im (a b^*) \sigma_3
    \end{equation*}
    \begin{equation*}
        U(a,b) \sigma_3 U(a,b)^{-1} = - Re(ab) \sigma_1 + 2 Im(ab) \sigma_2 + (|a|^2 - |b|^2) \sigma_3
    \end{equation*}
    or, in matrix notation,
    \begin{equation*}
        \rho_{ad} (U(a,b)) = \begin{bmatrix}
            Re(a^2 - b^2) & Im(a^2 + b^2) &  - Re(ab) \\ 
            - Im(a^2 - b^2) & Re(a^2 + b^2) & 2 Im(ab) \\
            2 Re (a b^*) & 2 Im (a b^*) & |a|^2 - |b|^2 \\
        \end{bmatrix}
    \end{equation*}
    
    
    We can recover rotation matrices around each Cartesian axis, by computing 
    \begin{equation*}
        R(\delta \theta, \begin{bmatrix} 1 \\ 0 \\ 0 \\ \end{bmatrix} ) = \begin{bmatrix}
            1 & 0 & 0 \\ 0 & \cos \theta & - \sin \theta \\ 0 & \sin \theta & \cos \theta \\
        \end{bmatrix} = \rho_{ad} (U(\cos \frac{\theta}{2} , -i \sin \frac{\theta}{2}))
    \end{equation*}
    \begin{equation*}
        R(\delta \theta, \begin{bmatrix} 0 \\ 1 \\ 0 \\ \end{bmatrix} ) = \begin{bmatrix}
            \cos \theta & 0 & \sin \theta \\ 0 & 1 & 0 \\  -\sin \theta & 0 & \cos \theta \\
        \end{bmatrix} = \rho_{ad} (U(\cos \frac{\theta}{2} , - \sin \frac{\theta}{2}))
    \end{equation*}
    \begin{equation*}
        R(\delta \theta, \begin{bmatrix} 0 \\ 0 \\ 1 \\ \end{bmatrix} ) = \begin{bmatrix}
            \cos \theta & - \sin \theta & 0 \\ \sin \theta & \cos \theta & 0 \\ 0 & 0 & 1 \\
        \end{bmatrix} = \rho_{ad} (U(\exp(-\frac{i}{2} \theta), 0))
    \end{equation*}

    Therefore, $\rho_{ad}$ defines a group homomorphism. The kernel is $\rho_{ad} (\mathbb I_2) = \rho_{ad} (- \mathbb I_2) = \mathbb I_3$. Hence, the universal cover $SU(2)$ is a two-fold cover of $SO(3)$, i.e. $SO(3) \simeq SU(2) / \ker (\rho_{ad}) = SU(2) / \mathbb Z_2$.

    \begin{proof}
        If $U(a,b)$ is mapped into the identity, then $Re(a b^*) = 0$, $|a|^2 - |b|^2 = 1$ and $Re(a^2 - b^2) = 1$. Hence $b=0$ and $a = \pm 1$. 
    \end{proof}

    For a general $SO(n)$ rotation group, since it is not connected, we have a two-fold universal cover $Spin(n)$, which is for $n \leq 6$
    \begin{equation*}
        Spin(3) \simeq SU(2) \quad Spin(4) \simeq SU(2) \times SU(2) \\
        Spin(5) \simeq Sp(2) \quad Spin(6) \simeq SU(6)
    \end{equation*}
    However, for $n > 6$, there is not a matrix group. Each group has a spinor representation. For $n = 2m$, it has dimension $2^{m-1}$ and for $n = 2m + 1$, it has dimension $2^m$.s

\chapter{$SO^+(1,3)$}

    In this chapter, we will apply all the notions we learnt in the previous ones to~\eqref{so(1,3)}, the proper Lorentz group $SO(1,3)$. In particular, we are interested in the proper orthochronous Lorentz group $SO^+(1,3)$.

\section{$SO^+(1,3)$ as a Lie group}

    The Lorentz group $O(1,3)$ is defined by the matrices $\Lambda$ such that they preserve the Minkovski metric 
    \begin{equation}\label{lorentz}
        \Lambda^\alpha_{\phantom \alpha \mu} \Lambda^\beta_{\phantom \beta \nu} \eta_{\alpha \beta} = \eta_{\mu \nu}
    \end{equation}

    First, the Lorentz group can be decomposed into two parts according to their determinant
    \begin{equation*}
        \det (\Lambda^T \eta \Lambda) = \det \Lambda^T \det \eta \det \Lambda = \det^2 \Lambda = \det \eta = 1
    \end{equation*}
    Hence 
    \begin{equation*}
        \det \Lambda = \pm 1
    \end{equation*}
    and the Lorentz group can be written as
    \begin{equation*}
        O(1,3) = \underbrace{\{\det \Lambda = +1\}}_{SO(1,3)} \cup \{\det \Lambda = - 1\} = SO(1,3) \cup \{\det \Lambda = - 1\}
    \end{equation*}
    where $SO(1,3)$ is called the proper Lorentz group.

    Second, the proper Lorentz group can be decomposed into two parts according to their $(0, 0)$ component
    \begin{equation*}
        \eta_{00} = \Lambda^\alpha_{\phantom \alpha 0} \Lambda^\beta_{\phantom \beta 0} \eta_{\alpha \beta}
    \end{equation*}
    \begin{equation*}
        -1 = -(\Lambda^0_{\phantom 0 0})^2 + (\Lambda^i_{\phantom i i})^2
    \end{equation*}
    \begin{equation*}
        (\Lambda^0_{\phantom 0 0})^2 = 1 + (\Lambda^i_{\phantom i i})^2 \geq 1
    \end{equation*}
    Hence 
    \begin{equation*}
        \Lambda^0_{\phantom 0 0} \in ]\infty, -1] \cup [1, \infty[
    \end{equation*}
    and the proper Lorentz group can be written as
    \begin{equation*}
        SO(1,3) = \underbrace{\{\Lambda^0_{\phantom 0 0} \in ]\infty, -1]\}}_{SO^+(1,3)} \cup \{\Lambda^0_{\phantom 0 0} \in [1, \infty[\} = SO^+(1,3) \cup \{\Lambda^0_{\phantom 0 0} \in [1, \infty[\} 
    \end{equation*}
    where $SO(1,3)^+$ is called the proper orthochronous Lorentz group.

    From now on, only the proper orthochronous Lorentz group will be studied because is the only group containing the identity. The rest of the group can be obtained by composing a proper orthochronous Lorentz transformation with time or spatial reversals.

\section{$\mathfrak{so}(1,3)$ as a Lie algebra}

    To find the Lie algebra, we consider an infinitesimal Lorentz transformation around the identity 
    \begin{equation}\label{inf}
        \Lambda^{\mu}_{\phantom \mu \nu} = \delta^{\mu}_{\phantom \mu \nu} + \omega^{\mu}_{\phantom \mu \nu}
    \end{equation}
    where $\omega^{\mu}_{\phantom \mu \nu} \ll 1$ is an infinitesimal matrix. 

    In order to preserve the metric, $\omega^{\mu}_{\phantom \mu \nu}$ is antisymmetric.
    \begin{proof}
        By using~\eqref{lorentz},
        \begin{equation*}
            (\delta^{\alpha}_{\phantom \alpha \mu} + \omega^{\alpha}_{\phantom \alpha \mu} )( \delta^{\beta}_{\phantom \beta \nu} + \omega^{\beta}_{\phantom \beta \nu} )\eta_{\alpha \beta} = \eta_{\mu \nu}
        \end{equation*}
        \begin{equation*}
            \delta^{\alpha}_{\phantom \alpha \mu} \delta^{\beta}_{\phantom \beta \nu} \eta_{\alpha \beta} + \delta^{\alpha}_{\phantom \alpha \mu} \omega^{\beta}_{\phantom \beta \nu} \eta_{\alpha \beta} + \omega^{\alpha}_{\phantom \alpha \mu} \delta^{\beta}_{\phantom \beta \nu} \eta_{\alpha \beta} + \omega^{\alpha}_{\phantom \alpha \mu} \omega^{\beta}_{\phantom \beta \nu} \eta_{\alpha \beta} = \eta_{\mu \nu}
        \end{equation*}
        \begin{equation*}
            \cancel{\eta_{\mu \nu}} + \omega_{\mu \nu} + \omega_{\nu\mu} + O(\omega^2) = \cancel{\eta_{\mu \nu}}
        \end{equation*}
        Hence, the matrices $\omega_{\mu\nu}$ are anti-symmetric
        \begin{equation*}
            \omega_{\mu \nu} = \omega_{\nu \mu}
        \end{equation*}
    \end{proof}

    By means of the exponential map, a generic $SO(1,3)^+$ transformation can be written as
    \begin{equation*}
        \Lambda^{\mu}_{\phantom \mu \nu} = \exp(- \frac{i}{2} \omega^{\alpha \beta} M_{\alpha \beta})^\mu_{\phantom \mu \nu}
    \end{equation*}
    where $M_{\alpha \beta}$ are the generators of the Lie algebra $\mathfrak{so} (1,3)$. Since they must be antisymmetric, otherwise they would vanish because $\omega$ are so as well, there are six independent generators of $\mathfrak{so} (1,3)$.


    The commutation relations between the generators are
    \begin{equation*}
        [M_{\alpha\beta}, M_{\sigma\rho}] = - i (\eta_{\alpha\sigma} M_{\beta \rho} - \eta_{\alpha\rho} M_{\beta\sigma} - \eta_{\beta\sigma} M_{\alpha\rho} + \eta_{\beta \rho} M_{\alpha \sigma})
    \end{equation*}

    \begin{proof}
        To find the explicit expression of the commutator of two generators, first it will be computed the following expression using~\eqref{inf}
        \begin{equation*}
        \begin{aligned}
            (\tilde \Lambda^{-1})^{\mu}_{\phantom \mu \alpha} (\Lambda^{-1})^{\alpha}_{\phantom \alpha \beta} \tilde \Lambda^{\beta}_{\phantom \beta \gamma} \Lambda^{\gamma}_{\phantom \gamma \nu} & = (\delta^{\mu}_{\phantom \mu \alpha} - \tilde \omega^{\mu}_{\phantom \mu \alpha})(\delta^{\alpha}_{\phantom \alpha \beta} - \omega^{\alpha}_{\phantom \alpha \beta})(\delta^{\beta}_{\phantom \beta \gamma} + \tilde \omega^{\beta}_{\phantom \beta \gamma})(\delta^{\gamma}_{\phantom \gamma \nu} + \omega^{\gamma}_{\phantom \gamma \nu}) + O(\tilde \omega^2) + O(\omega^2) \\ & = \delta^{\mu}_{\phantom \mu \alpha} \delta^{\alpha}_{\phantom \alpha \beta} \delta^{\beta}_{\phantom \beta \gamma} \delta^{\gamma}_{\phantom \gamma \nu} - \tilde \omega^{\mu}_{\phantom \mu \alpha} \delta^{\alpha}_{\phantom \alpha \beta} \delta^{\beta}_{\phantom \beta \gamma} \delta^{\gamma}_{\phantom \gamma \nu} - \delta^{\mu}_{\phantom \mu \alpha} \omega^{\alpha}_{\phantom \alpha \beta}\delta^{\beta}_{\phantom \beta \gamma}\delta^{\gamma}_{\phantom \gamma \nu} + \delta^{\mu}_{\phantom \mu \alpha} \delta^{\alpha}_{\phantom \alpha \beta} \tilde \omega^{\beta}_{\phantom \beta \gamma}\delta^{\gamma}_{\phantom \gamma \nu} \\ & \quad + \delta^{\mu}_{\phantom \mu \alpha} \delta^{\alpha}_{\phantom \alpha \beta} \delta^{\beta}_{\phantom \beta \gamma} \tilde \omega^{\beta}_{\phantom \beta \gamma} + \tilde \omega^{\mu}_{\phantom \mu \alpha} \omega^{\alpha}_{\phantom \alpha \beta} \delta^{\beta}_{\phantom \beta \gamma} \delta^{\gamma}_{\phantom \gamma \nu} - \tilde \omega^{\mu}_{\phantom \mu \alpha} \delta^{\alpha}_{\phantom \alpha \beta} \delta^{\beta}_{\phantom \beta \gamma} \omega^{\gamma}_{\phantom \gamma \nu} - \delta^{\mu}_{\phantom \mu \alpha} \omega^{\alpha}_{\phantom \alpha \beta} \tilde \omega^{\beta}_{\phantom \beta \gamma} \delta^{\gamma}_{\phantom \gamma \nu} \\ & \quad + \delta^{\mu}_{\phantom \mu \alpha} \tilde \omega^{\alpha}_{\phantom \alpha \beta} \delta^{\beta}_{\phantom \beta \gamma} \omega^{\gamma}_{\phantom \gamma \nu} + O(\tilde \omega^2) + O(\omega^2) \\ & = \delta^{\mu}_{\phantom \mu \nu} - \cancel{\tilde \omega^{\mu}_{\phantom \mu \nu}} - \cancel{\omega^{\mu}_{\phantom \mu \nu}} + \cancel{\tilde \omega^{\mu}_{\phantom \mu \nu}} + \cancel{\omega^{\mu}_{\phantom \mu \nu}} + \tilde \omega^{\mu}_{\phantom \mu \alpha} \omega^{\alpha}_{\phantom \alpha \nu} - \tilde \omega^{\mu}_{\phantom \mu \alpha} \omega^{\alpha}_{\phantom \alpha \nu} - \omega^{\mu}_{\phantom \mu \gamma} \tilde \omega^{\alpha}_{\phantom \alpha \nu} + \tilde \omega^{\mu}_{\phantom \mu \alpha} \omega^{\alpha}_{\phantom \alpha \nu} \\ & \quad + O(\omega^2) + O(\tilde \omega^2) \\ & = \delta^{\mu}_{\phantom \mu \nu} + \cancel{\tilde \omega^{\mu}_{\phantom \mu \alpha} \omega^{\alpha}_{\phantom \alpha \nu}} - \cancel{\tilde \omega^{\mu}_{\phantom \mu \alpha} \omega^{\alpha}_{\phantom \alpha \nu}} - \omega^{\mu}_{\phantom \mu \gamma} \tilde \omega^{\alpha}_{\phantom \alpha \nu} + \tilde \omega^{\mu}_{\phantom \mu \alpha} \omega^{\alpha}_{\phantom \alpha \nu} + O(\omega^2) + O(\tilde \omega^2) \\ & = \delta^{\mu}_{\phantom \mu \nu} - \omega^{\mu}_{\phantom \mu \gamma} \tilde \omega^{\alpha}_{\phantom \alpha \nu} + \tilde \omega^{\mu}_{\phantom \mu \alpha} \omega^{\alpha}_{\phantom \alpha \nu} + O(\omega^2) + O(\tilde \omega^2)
        \end{aligned}
        \end{equation*}
        and second using the BCH formula~\eqref{BCH}
        \begin{equation*}
        \begin{aligned}
            (\tilde \Lambda^{-1})^{\mu}_{\phantom \mu \alpha} (\Lambda^{-1})^{\alpha}_{\phantom \alpha \beta} \tilde \Lambda^{\beta}_{\phantom \beta \gamma} \Lambda^{\gamma}_{\phantom \gamma \nu} & = \exp(\frac{i}{2} \tilde \omega^{\alpha \beta} M_{\alpha \beta}) \exp(\frac{i}{2} \omega^{\sigma \rho} M_{\sigma \rho}) \exp(- \frac{i}{2} \tilde \omega^{\alpha \beta} M_{\alpha \beta}) \exp(- \frac{i}{2} \omega^{\sigma \rho} M_{\sigma \rho}) \\ & = \exp(\frac{i}{2} \tilde \omega^{\alpha \beta} M_{\alpha \beta} + \frac{i}{2} \omega^{\sigma \rho} M_{\sigma \rho} + [\frac{i}{2} \tilde \omega^{\alpha \beta} M_{\alpha \beta}, \frac{i}{2} \omega^{\sigma \rho} M_{\sigma \rho}]) \\ & \quad \exp(- \frac{i}{2} \tilde \omega^{\alpha \beta} M_{\alpha \beta} - \frac{i}{2} \omega^{\sigma \rho} M_{\sigma \rho} + [\frac{i}{2} \tilde \omega^{\alpha \beta} M_{\alpha \beta}, \frac{i}{2} \omega^{\sigma \rho} M_{\sigma \rho}]) + O(\omega^2) + O(\tilde \omega^2) \\ & =  \exp(\cancel{\frac{i}{2} \tilde \omega^{\alpha \beta} M_{\alpha \beta}} + \cancel{\frac{i}{2} \omega^{\sigma \rho} M_{\sigma \rho}} - \cancel{\frac{i}{2} \tilde \omega^{\alpha \beta} M_{\alpha \beta}} - \cancel{\frac{i}{2} \omega^{\sigma \rho} M_{\sigma \rho}} \\ & \quad + [\frac{i}{2} \tilde \omega^{\alpha \beta} M_{\alpha \beta}, \frac{i}{2} \omega^{\sigma \rho} M_{\sigma \rho}]) + O(\omega^2) + O(\tilde \omega^2) \\ & = \exp([\frac{i}{2} \tilde \omega^{\alpha \beta} M_{\alpha \beta}, \frac{i}{2} \omega^{\sigma \rho} M_{\sigma \rho}]) + O(\omega^2) + O(\tilde \omega^2)
        \end{aligned}
        \end{equation*}
        and we put them together
        \begin{equation*}
            \exp([\frac{i}{2} \tilde \omega^{\alpha \beta} M_{\alpha \beta}, \frac{i}{2} \omega^{\sigma \rho} M_{\sigma \rho}]) = \delta^{\mu}_{\phantom \mu \nu} - \omega^{\mu}_{\phantom \mu \gamma} \tilde \omega^{\alpha}_{\phantom \alpha \nu} + \tilde \omega^{\mu}_{\phantom \mu \alpha} \omega^{\alpha}_{\phantom \alpha \nu} \rightarrow \exp(-\frac{i}{2} (-\omega^{\mu}_{\phantom \mu \gamma} \tilde \omega^{\alpha}_{\phantom \alpha \nu} + \tilde \omega^{\mu}_{\phantom \mu \alpha} \omega^{\alpha}_{\phantom \alpha \nu}) M_\mu^{\phantom \mu\nu})
        \end{equation*}
        \begin{equation*}
            - \frac{1}{4} \tilde \omega^{\alpha \beta} \omega^{\sigma \rho}[M_{\alpha \beta}, M_{\sigma \rho}] = -\frac{i}{2} (-\omega^{\mu}_{\phantom \mu \alpha} \tilde \omega^{\alpha}_{\phantom \alpha \nu} + \tilde \omega^{\mu}_{\phantom \mu \alpha} \omega^{\alpha}_{\phantom \alpha \nu}) M_\mu^{\phantom \mu\nu} = - \frac{i}{2} \eta_{\alpha \gamma} (-\omega^{\mu\alpha} \tilde \omega^{\gamma\nu} + \tilde \omega^{\mu\alpha} \omega^{\gamma \nu}) M_{\mu\nu}
        \end{equation*}
        Hence 
        \begin{equation}\label{pr1}
            - \frac{1}{4} \tilde \omega^{\alpha \beta} \omega^{\sigma \rho}[M_{\alpha \beta}, M_{\sigma \rho}] = - \frac{i}{2} \eta_{\alpha \gamma} (\tilde \omega^{\mu\alpha} \omega^{\gamma \nu} -\omega^{\mu\alpha} \tilde \omega^{\gamma\nu}) M_{\mu\nu}
        \end{equation}
    
        Now we consider an ansatz 
        \begin{equation*}
            [M_{\alpha\beta}, M_{\sigma\rho}] = T^{(1)}_{\alpha\beta} M_{\sigma\rho} + T^{(2)}_{\alpha\sigma} M_{\beta\rho} + T^{(3)}_{\alpha\rho} M_{\beta\sigma} + T^{(4)}_{\beta\sigma} M_{\alpha\rho} + T^{(5)}_{\beta\rho} M_{\alpha\sigma } + T^{(6)}_{\sigma\rho} M_{\alpha\beta}
        \end{equation*}
        and inserting into the previous, there are no matching term with $T^{(1)} = T^{(2)} = 0$. 
    
        Furthermore, by using $M_{\mu\nu} = - M_{\nu\mu}$
        \begin{equation*}
            0 = [M_{\alpha\beta}, M_{\sigma \rho}] + [M_{\beta\alpha}. M_{\sigma\rho}]
        \end{equation*}
        we find that 
        \begin{equation*}
            0 = T^{(2)}_{\alpha\sigma} M_{\beta\rho} + T^{(3)}_{\alpha\rho} M_{\beta\sigma} + T^{(4)}_{\beta\sigma} M_{\alpha\rho} + T^{(5)}_{\beta\rho} M_{\alpha\sigma } + T^{(2)}_{\beta\sigma} M_{\alpha\rho} + T^{(3)}_{\beta\rho} M_{\alpha\sigma} + T^{(4)}_{\alpha\sigma} M_{\beta\rho} + T^{(5)}_{\alpha\rho} M_{\beta\sigma } 
        \end{equation*}
        therefore
        \begin{equation*}
            T^{(2)}_{\alpha\sigma} + T^{(4)}_{\alpha\sigma} = 0
        \end{equation*}
        and 
        \begin{equation*}
            T^{(3)}_{\alpha\rho} + T^{(5)}_{\alpha\rho} = 0
        \end{equation*}

        Similarly
        \begin{equation*}
            0 = [M_{\alpha\beta}, M_{\sigma \rho}] + [M_{\alpha\beta}. M_{\rho\sigma}]
        \end{equation*}
        we find that 
        \begin{equation*}
            0 = T^{(2)}_{\alpha\sigma} M_{\beta\rho} + T^{(3)}_{\alpha\rho} M_{\beta\sigma} + T^{(4)}_{\beta\sigma} M_{\alpha\rho} + T^{(5)}_{\beta\rho} M_{\alpha\sigma } + T^{(2)}_{\alpha\rho} M_{\beta\sigma} + T^{(3)}_{\alpha\sigma} M_{\beta\rho} + T^{(4)}_{\beta\rho} M_{\alpha\sigma} + T^{(5)}_{\beta\sigma} M_{\alpha\rho }
        \end{equation*}
        therefore
        \begin{equation*}
            T^{(2)}_{\alpha\sigma} + T^{(3)}_{\alpha\sigma} = 0
        \end{equation*}
        and 
        \begin{equation*}
            T^{(4)}_{\beta\sigma} + T^{(5)}_{\beta\sigma} = 0
        \end{equation*}

        Finally, we have
        \begin{equation*}
            T^(2) = - T^{(3)} = - T^{(4)} = T^{(5)} = T
        \end{equation*}
        and we insert it into~\eqref{pr1}
        \begin{equation*}
        \begin{aligned}
            - \frac{1}{4} \tilde \omega^{\alpha \beta} \omega^{\sigma \rho}[M_{\alpha \beta}, M_{\sigma \rho}] & = - \frac{1}{4} \tilde \omega^{\alpha \beta} \omega^{\sigma \rho} (T_{\alpha\sigma} M_{\beta\rho} - T_{\alpha\rho} M_{\beta\sigma} - T_{\beta\sigma} M_{\alpha\rho} + T_{\beta\rho} M_{\alpha\sigma }) \\ & = - \frac{1}{4} (T_{\alpha\sigma} \tilde \omega^{\alpha \beta} \omega^{\sigma \rho} M_{\beta\rho} - T_{\alpha\rho} \tilde \omega^{\alpha \beta} \omega^{\sigma \rho} M_{\beta\sigma} - T_{\beta\sigma} \tilde \omega^{\alpha \beta} \omega^{\sigma \rho} M_{\alpha\rho} + T_{\beta\rho} \tilde \omega^{\alpha \beta} \omega^{\sigma \rho} M_{\alpha\sigma }) \\ &  = - \frac{1}{4} (T_{\alpha\sigma} \tilde \omega^{\alpha \beta} \omega^{\sigma \rho} M_{\beta\rho} - T_{\alpha\sigma} \tilde \omega^{\alpha \beta} \omega^{\rho \sigma} M_{\beta\rho} - T_{\beta\sigma} \tilde \omega^{\alpha \beta} \omega^{\sigma \rho} M_{\alpha\rho} + T_{\beta\sigma} \tilde \omega^{\alpha \beta} \omega^{\rho \sigma} M_{\alpha\rho })
        \end{aligned}
        \end{equation*}
        where in the last step we only relabelled $\sigma \leftrightarrow \rho$. 

        Now, we use the antisymmetry of $\omega$ and $M$
        \begin{equation*}
        \begin{aligned}
            & = - \frac{1}{4} (T_{\alpha\sigma} \tilde \omega^{\alpha \beta} \omega^{\sigma \rho} M_{\beta\rho} + T_{\alpha\sigma} \tilde \omega^{\alpha \beta} \omega^{\sigma \rho} M_{\beta\rho} - T_{\beta\sigma} \tilde \omega^{\alpha \beta} \omega^{\sigma \rho} M_{\alpha\rho} - T_{\beta\sigma} \tilde \omega^{\alpha \beta} \omega^{\sigma \rho} M_{\alpha\rho }) \\ & = - \frac{1}{2} (T_{\alpha\sigma} \tilde \omega^{\alpha \beta} \omega^{\sigma \rho} M_{\beta\rho} - T_{\beta\sigma} \tilde \omega^{\alpha \beta} \omega^{\sigma \rho} M_{\alpha\rho}) \\ & = - \frac{1}{2} (T_{\gamma\lambda} \tilde \omega^{\gamma \nu} \omega^{\lambda \mu} M_{\nu\mu} - T_{\gamma\lambda} \tilde \omega^{\mu \gamma} \omega^{\lambda \nu} M_{\mu \nu}) \\ & = - \frac{1}{2} (T_{\alpha\sigma} \tilde \omega^{\alpha \beta} \omega^{\sigma \rho} M_{\beta\rho} - T_{\beta\sigma} \tilde \omega^{\alpha \beta} \omega^{\sigma \rho} M_{\alpha\rho}) \\ & = - \frac{1}{2} (T_{\gamma\lambda} \tilde \omega^{\gamma\nu} \omega^{\mu\lambda} M_{\mu\nu} - T_{\gamma\lambda} \tilde \omega^{\mu \gamma} \omega^{\lambda \nu} M_{\mu \nu}) \\ & = - \frac{1}{2} T_{\gamma\lambda} (\tilde \omega^{\gamma\nu} \omega^{\mu\lambda} - \tilde \omega^{\mu \gamma} \omega^{\lambda \nu}) M_{\mu \nu}
        \end{aligned}
        \end{equation*}
        where in the second step we only relabelled $\alpha \leftrightarrow \gamma$, $\beta \leftrightarrow \nu$, $\sigma \leftrightarrow \lambda$, $\rho \leftrightarrow \mu$ in the first term and $\alpha \leftrightarrow \mu$, $\beta \leftrightarrow \gamma$, $\sigma \leftrightarrow \lambda$, $\rho \leftrightarrow \nu$ in the second term. 

        We compare it to~\eqref{pr1}
        \begin{equation*}
            - \frac{1}{2} T_{\gamma\lambda} (\tilde \omega^{\gamma\nu} \omega^{\mu\lambda} - \tilde \omega^{\mu \gamma} \omega^{\lambda \nu}) M_{\mu \nu} = - \frac{i}{2} \eta_{\alpha \gamma} (\tilde \omega^{\mu\alpha} \omega^{\gamma \nu} -\omega^{\mu\alpha} \tilde \omega^{\gamma\nu}) M_{\mu\nu}
        \end{equation*}
        and we find that 
        \begin{equation*}
            T_{\gamma \lambda} = - i \eta_{\gamma\lambda}
        \end{equation*}
        Hence
        \begin{equation*}
            [M_{\alpha\beta}, M_{\sigma\rho}] = -i (\eta_{\alpha\sigma} M_{\beta\rho} - \eta_{\alpha\rho} M_{\beta\sigma} - \eta_{\beta\sigma} M_{\alpha\rho} + \eta_{\beta\rho} M_{\alpha\sigma})
        \end{equation*}
    \end{proof}

    It is also possbile to derive the $4 \times 4$ matrix representation, which is 
    \begin{equation*}
        (M_{\alpha\beta})^\mu_{\phantom \mu \nu} = i (\delta^\mu_{\phantom \mu \alpha} \eta_{\beta \nu} - \delta^\mu_{\phantom \mu \beta} \eta_{\alpha \nu})
    \end{equation*}

    Now, we restrict to spatial indices and the commutation relations become
    \begin{equation*}
        [M_{ij}, M_{kl}] = - i (\delta_{ik} M_{jl} - \delta_{il} M_{jk} - \delta_{jk} M_{il} + \delta_{ji} M_{ij})
    \end{equation*}
    which gives rise to three independent generators 
    \begin{equation*}
        J_i = - \frac{1}{2} \epsilon_{ijk} M_{jk}
    \end{equation*}
    o, equivalently,
    \begin{equation*}
        M_{ij} = - \epsilon_{ijk} J_k
    \end{equation*}
    with commutation relations 
    \begin{equation}\label{commjj}
        [J_m, J_n] = i \epsilon_{mnk} J_k
    \end{equation}
    Hence, the proper orthochronous Lorentz group contains the rotation group as subgroup with spatial indices.

    There are other generators $K_i = M_{0i}$ associated to temporal indices, which are the generators of boosts in the $i$-th direction. The commutation relations become 
    \begin{equation*}
        [K_i, K_j] = -i(\eta_{00} M_{ij} + \delta_{ij} M_{00}) = - iM_{ij} = - i \epsilon_{ijk} J_k
    \end{equation*}
    and 
    \begin{equation*}
        [J_i, K_j] = - \frac{1}{2} \epsilon_{imn} [M_{mn}, M_{0j}] = - \frac{i}{2} (- \delta_{mj} M_{n0} + \delta_{nj} M_{m0}) = i \epsilon_{ijn} K_n
    \end{equation*}

    If the velocity is $\mathbf v = c \tanh(r) \mathbf u$ where $r$ is the rapidity, then a generic boost transformation can be written as 
    \begin{equation*}
        \Lambda = \begin{pmatrix}
            \cosh(r) & \sinh(r) \mathbf u^T \\
            \sinh(r) \mathbf u & \mathbb I_3 + (\cosh(r)-1) \mathbf u \mathbf u^T \\
        \end{pmatrix}
    \end{equation*}
    and a general proper orthochronous Lorentz transformation is 
    \begin{equation*}
        \Lambda (\theta, \mathbf n, r, \mathbf u) = \exp(-i \theta \mathbf n \cdot \mathbf J - i \mathbf v \cdot \mathbf K) = \exp(- i \theta \mathbf n \cdot \mathbf J - i c \tanh(r) \mathbf u \cdot \mathbf K)
    \end{equation*}

\section{Representations of the Lorentz algebra}

    $SO^+(1,3)$ is not compact, since the values of the boosts do not have an upper-bound limit. For non-compact non-abelian Lie groups, any non-trivial unitary representation must be infinite-dimensional. However, let us focus in finite-dimensional representations, even though they are not unitary. We can justified this choice by noticing that for field representations, which become quantum operator, there is no scalar product between them. 

    The defining representation of the Lie algebra $\mathfrak{so}^+(1,3)$ is 
    \begin{equation*}
        (M_{\alpha\beta})^\mu_{\phantom \mu \nu} = i (\delta^\mu_{\phantom \mu \alpha} \eta_{\beta \nu} - \delta^\mu_{\phantom \mu \beta} \eta_{\alpha \nu})
    \end{equation*}

    The generators of rotations are
    \begin{equation*}
        J_i = \frac{1}{2} \epsilon_{ijk} M^{jk}
    \end{equation*}
    which are hermitian, i.e. $J_i^\dagger = J_i$, while the generators of boosts are
    \begin{equation*}
        K_i = M_{0i}
    \end{equation*}
    which are anti-hermitian, i.e. $K_i^\dagger = - K_i$, because they do not have a finite range. Explicitly, they are 
    \begin{equation}\label{explj}
        J_1 = \begin{bmatrix}
            0 & 0 & 0 & 0 \\ 
            0 & 0 & 0 & 0 \\ 
            0 & 0 & 0 & -i \\ 
            0 & 0 & i & 0 \\ 
        \end{bmatrix} \quad J_2 = \begin{bmatrix}
            0 & 0 & 0 & 0 \\ 
            0 & 0 & 0 & i \\ 
            0 & 0 & 0 & 0 \\ 
            0 & -i & 0 & 0 \\ 
        \end{bmatrix} \quad J_3 = \begin{bmatrix}
            0 & 0 & 0 & 0 \\ 
            0 & 0 & -i & 0 \\ 
            0 & i & 0 & 0 \\ 
            0 & 0 & 0 & 0 \\ 
        \end{bmatrix} 
    \end{equation}
    and 
    \begin{equation}\label{explk}
        K_1 = \begin{bmatrix}
            0 & i & 0 & 0 \\ 
            i & 0 & 0 & 0 \\ 
            0 & 0 & 0 & 0 \\ 
            0 & 0 & 0 & 0 \\ 
        \end{bmatrix} \quad K_2 = \begin{bmatrix}
            0 & 0 & i & 0 \\ 
            0 & 0 & 0 & 0 \\ 
            i & 0 & 0 & 0 \\ 
            0 & 0 & 0 & 0 \\ 
        \end{bmatrix} \quad K_3 = \begin{bmatrix}
            0 & 0 & 0 & i \\ 
            0 & 0 & 0 & 0 \\ 
            0 & 0 & 0 & 0 \\ 
            i & 0 & 0 & 0 \\ 
        \end{bmatrix} 
    \end{equation}

    \begin{proof}
        For $J_1$ and similarly for the others
        \begin{equation*}
            (J_1)^\mu_{\phantom \mu \nu} = - (M_{23})^\mu_{\phantom \mu \nu} = - i (\underbrace{\delta^\mu_{\phantom \mu 2}}_{\mu = 2} \underbrace{\eta_{3 \nu}}_{\nu = 3} + \underbrace{\delta^\mu_{\phantom \mu 3}}_{\mu = 3} \underbrace{\eta_{2 \nu}}_{\nu = 2}) = - i \delta^2_{\phantom 2 3} + i \delta^3_{\phantom 3 2}
        \end{equation*}

        For $K_1$ and similarly for the others 
        \begin{equation*}
            (K_1)^\mu_{\phantom \mu \nu} = - (M_{01})^\mu_{\phantom \mu \nu} = - i (\underbrace{\delta^\mu_{\phantom \mu 0}}_{\mu = 0} \underbrace{\eta_{1 \nu}}_{\nu = 1} + \underbrace{\delta^\mu_{\phantom \mu 1}}_{\mu = 1} \underbrace{\eta_{0 \nu}}_{\nu = 0}) = i \delta^0_{\phantom 0 1 } + i \delta^1_{\phantom 1 0}
        \end{equation*}
    \end{proof}

    The procedure to construct Lorentz algebra irreducible representations is called complexification and it consists of a complex linear combination of the generators $\mathfrak g_{\mathbb C} = \mathfrak g \oplus i \mathfrak g$. In particular, for the Lorentz algebra, we define new generators
    \begin{equation*}
        A_i = \frac{1}{2} (J_i + i K_i) \quad B_i = \frac{1}{2} (J_i - i K_i)
    \end{equation*}
    such that the new commutation relations are 
    \begin{equation*}
        [A_i, A_j] = i \epsilon_{ijk} A_K \quad [B_i, B_j] = i \epsilon_{ijk} B_K \quad [A_i, B_j] = 0
    \end{equation*}

    \begin{proof}
        Maybe in the future.
    \end{proof}


    Hence $\mathfrak{so}(1,3) \simeq \mathfrak{su}(2) \oplus \mathfrak{su}(2)$. In this way, we can use representations of $\mathfrak{su}(2)$ labelled by a pair $(j_1, j_2)$ to construct a representation of the whole Lorentz algebra. Formally, an irreducible representation of a direct sum $\mathfrak g \oplus \mathfrak h$ can be built by the tensor product of $(\rho_{\mathfrak g}, V_{\mathfrak g})$ and $(\rho_{\mathfrak h}, V_{\mathfrak h})$ by
    \begin{equation*}
        \rho \colon \mathfrak g \oplus \mathfrak h \rightarrow End(V) = End(V_{\mathfrak g} \otimes V_{\mathfrak h})
    \end{equation*}
    such that 
    \begin{equation*}
        \rho(X+Y) (v \otimes w) = \rho_{\mathfrak g}(v) \otimes w + v \otimes \rho_{\mathfrak h}(w)
    \end{equation*}
    In particular, for the Lorentz algebra 
    \begin{equation*}
        \rho_{j_1, j_2} \Big ( \sum_m \lambda_m A_m + \sum_n k_n B_n \Big) = \sum_m \lambda_m \rho_{j_1} (A_m) \otimes \mathbb I_{V_{j_2}} + \sum_n k_n \mathbb I_{V_{j_1}} \otimes \rho_{j_2} (B_n)
    \end{equation*}
    where $A_m$ and $B_n$ are independent generators of $\mathfrak{su}(2)$ and the dimension is $\dim (V_{j_1} \otimes V_{j_2}) = \dim V_{j_1} \dim V_{j_2} = (2j_1 + 1)(2j_2 + 1)$.

    If we restrict to the rotation algebra $J_i = A_i + B_i$, a similar closed algebra $\mathfrak{so}(3) \simeq \mathfrak{su}(2)$ can be studied inside $\mathfrak{so}(1,3)$. By means of the Clebsch-Gordan decomposition, this reduces into a sum of irreducible representation of spin $j$ such that $|j| \leq j_1 + j_2$, with total dimension $(2j_1 + 1) (2j_2 + 1)$.

    Notice that if they are both integers or half-integers, the sum representation is integer, like a bosonic one. On the other hand, if one is integer and the other one half-integer, the sum representation is half-integer, like a fermionic one. 

    $SO^+(1,3)$ is not simply connected. Therefore, we need its universal cover is $SL(2, \mathbb C)$. In particular, we have also here a two-fold cover $SO^+(1,3) \simeq SL(2, \mathbb C)/\mathbb Z_2$. Hence the spinor representations of $SO^+(1,3)$ is in $SL(2, \mathbb C)$.

\subsection{Spinor representations}

    Given the fact that we have two labels $j_1$ and $j_2$ for the representations, there are two different $2$-dimensional spinor representations: the left-handed Weyl spinor $\rho_L=\rho_{(\frac{1}{2}, 0)}$ with $(\frac{1}{2}, 0)$ and the right-handed Weyl spinor $\rho_R= \rho_{(0, \frac{1}{2})}$ with $(0, \frac{1}{2})$. They are called spinors because the only irreducible representation of rotations restriction are the one corresponding to $j = \frac{1}{2}$.

    Since the rep $j = \frac{1}{2}$ is generated by the Pauli matrices and $j=0$ is the trivial one, the generators of left spinors are 
    \begin{equation*}
        \rho_L(A_i) = \frac{1}{2} \sigma_i \otimes \mathbb I_1 = \frac{1}{2} \sigma_i \quad \rho_R(B_i) = \mathbb I_2 \otimes 0 = 0
    \end{equation*}
    and the generators of right spinors are
    \begin{equation*}
        \rho_R(A_i) = 0 \otimes \mathbb I_1 = 0 \quad \rho_R(B_i) = \mathbb I_1 \otimes \frac{1}{2} \sigma_i = \frac{1}{2} \sigma_i
    \end{equation*}
    Hence, by linearity, the original generators for left spinors are 
    \begin{equation*}
        \rho_L (J_i) = \rho_L(A_i) + \rho_L(B_i) = \frac{1}{2} \sigma_i \quad \rho_L(K_i) = - i (\rho_L(A_i) - \rho_L(B_i)) = - \frac{i}{2} \sigma_i
    \end{equation*}
    and the original generators for right spinors are 
    \begin{equation*}
        \rho_R (J_i) = \rho_R(A_i) + \rho_R(B_i) = \frac{1}{2} \sigma_i \quad \rho_R(K_i) = - i (\rho_R(A_i) - \rho_R(B_i)) = \frac{i}{2} \sigma_i
    \end{equation*}

    The associated representation is generated by real linear combinations: for left spinors is
    \begin{equation*}
        V_{(\frac{1}{2}, 0)} \ni \phi \mapsto \exp(-i \theta \vec n \cdot \rho_L(\vec J) - i \vec \cdot \rho_L(\vec K)) \phi
    \end{equation*}
    and for right spinors is
    \begin{equation*}
        V_{(0, \frac{1}{2})} \ni \phi \mapsto \exp(-i \theta \vec n \cdot \rho_R(\vec J) - i \vec \cdot \rho_R(\vec K)) \phi = \exp(-\frac{1}{2} (- i \theta \vec n + \vec v) \cdot \vec \sigma) \phi
    \end{equation*}

    They are the complex conjugates of each other. 
    
    \begin{proof}
        By using the identity
        \begin{equation*}
            (i \sigma_2) \sigma_i (- i \sigma_1) = (i \sigma_2) \overline \sigma_i (i \sigma_2)^{-1} = - \sigma_i
        \end{equation*}
        we define a new vector $\chi = i \sigma_2 \phi$ which maps as
        \begin{equation*}
        \begin{aligned}
            \chi \mapsto i \sigma_2 \phi & = i \sigma_2 \overline{\exp(-i \theta \vec n \cdot \rho_L(\vec J) - i \vec \cdot \rho_L(\vec K)) \phi} \\ & = i \sigma_2 \overline{\exp(-\frac{1}{2} (- i \theta \vec n + \vec v) \cdot \vec \sigma) \phi} \\ & = i \sigma_2 \exp(-\frac{1}{2} (- i \theta \vec n + \vec v) \cdot \overline{\vec \sigma}) \overline \phi \\ & = i \sigma_2 (i \sigma_2)^{-1} \exp(-\frac{1}{2} (- i \theta \vec n + \vec v) \cdot \vec \sigma) (i \sigma_2) \overline \phi \\ & = \exp(- i \theta \vec n \cdot \rho_R(\vec J) - i \vec v \cdot \rho_R (\vec K)) \chi
        \end{aligned}
        \end{equation*}
        Hence, up to a basis change, the complex conjugate of $\phi$ transforms like $\chi$.
    \end{proof} 

\subsection{Dirac representation}

    The direct sum of the two Weyl spinors is a reducible representation, called the bispinor representation. It is related to the Weyl one via 
    \begin{equation*}
        V_D = V_{\frac{1}{2}, 0} \oplus V_{0, \frac{1}{2}}
    \end{equation*}
    but with different basis 
    \begin{equation*}
        \psi_1 = \frac{\phi_1 + \chi_1}{\sqrt 2} \quad \psi_2 = \frac{\phi_2 + \chi_2}{\sqrt 2} \quad \psi_3 = \frac{\phi_1 - \chi_1}{\sqrt 2} \quad \psi_4 = \frac{\phi_2 - \chi_2}{\sqrt 2}
    \end{equation*}
    where $(\phi_1, \phi_2)$ is a basis of $V_{frac{1}{2}, 0}$ and $(\chi_1, \chi_2)$ of $V_{0, \frac{1}{2}}$. 


    Another way to introduce bispinors is through the Dirac matrices, i.e. $4 \times 4$ matrices such that they satisfy the anticommutator relations 
    \begin{equation*}
        \{\gamma_\mu, \gamma_\nu\} = \gamma_\mu \gamma_\nu - \gamma_\nu \gamma_\mu = 2 \eta_{\mu\nu}
    \end{equation*}
    They allow to construct a $4$-dimensional representation of the Lorentz algebra 
    \begin{equation*}
        M_{\alpha\beta} = \frac{i}{4} [\gamma_\alpha, \gamma_\beta] = \frac{i}{4} (\gamma_\alpha \gamma_\beta - \gamma_\beta \gamma_\alpha)
    \end{equation*}
    such that the commutator relations are
    \begin{equation*}
        [M_{\alpha\beta}, M_{\sigma\rho}] = - i (\eta_{\alpha\sigma} M_{\beta \rho} - \eta_{\alpha\rho} M_{\beta\sigma} - \eta_{\beta\sigma} M_{\alpha\rho} + \eta_{\beta\rho} M_{\alpha\sigma})
    \end{equation*}
    which leads to the Dirac equation. Further insights can be found in the RQM or in the QFT course.

\subsection{Vector representation}

    Now, we study the vector representation, which is a representation where the spin label is $(\frac{1}{2}, \frac{1}{2})$. Since its restrictions to rotations are $j = \frac{1}{2} + \frac{1}{2} = 1$ and $j = \frac{1}{2} - \frac{1}{2} = 0$, they both have integer spin and it corresponds to a bosonic field. 
    
    We pick a basis of $V_{\frac{1}{2}}$, which is 
    \begin{equation*}
        \{\ket{\frac{1}{2}, \frac{1}{2}}, \ket{\frac{1}{2}, -\frac{1}{2}}\} = \{\ket{+}, \ket{-}\}
    \end{equation*} 
    and it induces a natural basis for $V_{(\frac{1}{2}, \frac{1}{2})}$ 
    \begin{equation*}
        \ket{+} \oplus \ket{+} = \ket{++} \quad \ket{+} \oplus \ket{-} = \ket{+-} \quad \ket{-} \oplus \ket{+} = \ket{-+} \quad \ket{-} \oplus \ket{-} = \ket{--}
    \end{equation*}
    However, it can be more useful to introduce another basis for the same linear space
    \begin{equation*}
    \begin{aligned}
        & \ket{e_1} = \frac{\ket{+-} - \ket{-+}}{\sqrt 2} \quad \ket{e_2} = \frac{\ket{++} - \ket{--}}{\sqrt 2} \\ & \ket{e_3} = \frac{- \ket{++} - i\ket{--}}{\sqrt 2} \quad \ket{e_4} = \frac{- \ket{+-} - \ket{-+}}{\sqrt 2}
    \end{aligned}
    \end{equation*}

    It is equivalent to the defining rep of $\mathfrak{so}(1,3)$. 
 
    \begin{proof}
        The complex generators of the representation are
        \begin{equation*}
            \rho(A_i) = \rho_{\frac{1}{2}} (A_i) \otimes \mathbb J_2  = \mathcal J_i^{(\frac{1}{2})} \otimes \mathbb J_2
        \end{equation*}
        and
        \begin{equation*}
            \rho(B_i) = \mathbb J_2 \otimes \rho_{\frac{1}{2}} (B_i) = \mathbb J_2 \otimes \mathcal J_i^{(\frac{1}{2})} 
        \end{equation*}
        where their actions on the natural basis are
        \begin{equation*}
            \rho(A_i) \ket{+-} = \mathcal J_i^{(\frac{1}{2})} \ket{+} \otimes \ket{-} \quad \rho(B_i) \ket{+-} = \ket{+} \otimes \mathcal J_i^{(\frac{1}{2})}  \ket{-} 
        \end{equation*}
        Moreover, the original generators become 
        \begin{equation*}
            \rho(J_i) = \rho(A_i) + \rho(B_i) = \mathcal J_i^{(\frac{1}{2})} \otimes \mathbb J_2 + \mathbb J_2 \otimes \mathcal J_i^{(\frac{1}{2})} 
        \end{equation*}
        and 
        \begin{equation*}
            \rho(K_i) = -i (\rho(A_i) - \rho(B_i)) = - i \mathcal J_i^{(\frac{1}{2})} \otimes \mathbb J_2 + i \mathbb J_2 \otimes \mathcal J_i^{(\frac{1}{2})} 
        \end{equation*}
        where $\mathcal J^{(\frac{1}{2})}$ is the single $\mathfrak{su}(2)$ representation.
    
        For example, we take the generator $J_3$, whose action on the natural basis is
        \begin{equation*}
            \rho(J_3) \ket{++} = \ket{++} \quad \rho(J_3) \ket{--} = - \ket{--} \quad \rho(J_3) \ket{-+} = \rho(J_3) \ket{+-} = 0
        \end{equation*}
        or on the other basis is
        \begin{equation*}
            \rho(J_3) \ket{e_1} = \rho(J_3) \ket{e_4} = 0 \quad \rho(J_3) \ket{e_2} = i \ket{e_3} \quad \rho(J_3) \ket{e_2} = -i \ket{e_2}
        \end{equation*}
        Hence the corresponding matrix becomes 
        \begin{equation*}
            \rho(J_3) = \begin{pmatrix}
                0 & 0 & 0 & 0 \\
                0 & 0 & -i & 0 \\
                0 & i & 0 & 0 \\
                0 & 0 & 0 & 0 \\
            \end{pmatrix}
        \end{equation*}
        which is indeed the same matrix as~\eqref{explj}.
    
        For example, we take the generator $K_3$, whose action on the natural basis is
        \begin{equation*}
            \rho(K_3) \ket{++} = \rho(K_3) \ket{--} = 0 \quad \rho(K_3) \ket{+-} = - i \ket{+-} \quad \rho(K_3) \ket{-+} = i \ket{-+}
        \end{equation*}
        or on the other basis is
        \begin{equation*}
            \rho(K_3) \ket{e_1} = i \ket{e_4} \quad \rho(K_3) \ket{e_2} = \quad \rho(K_3) \ket{e_3} = 0 \quad \rho(K_3) \ket{e_4} = i \ket{e_1}
        \end{equation*}
        Hence the corresponding matrix becomes 
        \begin{equation*}
            \rho(K_3) = \begin{pmatrix}
                0 & 0 & 0 & i \\
                0 & 0 & 0 & 0 \\
                0 & 0 & 0 & 0 \\
                i & 0 & 0 & 0 \\
            \end{pmatrix}
        \end{equation*}
        which is indeed the same matrix as~\eqref{explk}.
    \end{proof}

\subsection{Other representations}

    We conclude this chapter with a list of some relevant representations in physics. See Table~\ref{table:1}.

    \begin{table}[h!]
        \centering
        \begin{tabular}{c | c | c}
            Representation & Field & Physics \\
            \hline
            $(0,0)$ & scalar & Higgs \\ 
            $(\frac{1}{2}, 0)$, $(0, \frac{1}{2})$ or $(\frac{1}{2}, 0) \oplus (0, \frac{1}{2})$ & fermionic  & matter \\ 
             $(\frac{1}{2}, \frac{1}{2})$ & bosonic  & forces \\ 
            $(1,0)$ (or $(0,1)) $ & (anti) self-dual $2$-form  & only string theory\\
            $(1, 0) \oplus (0, 1)$ & parity invariant 2-form & EM tensor \\
            $(1,\frac{1}{2}) \oplus (\frac{1}{2}, 1)$ & Rarita-Schwinger & gravitino \\
            $(1, 1)$ & traceless symmetric tensor & graviton \\
        \end{tabular}
        \caption{Other representations of the Lorentz group}\label{table:1}
    \end{table}

