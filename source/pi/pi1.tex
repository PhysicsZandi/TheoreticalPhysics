\part{Path integrals in configuration and phase space}

\chapter{Introduction}

    In quantum mechanics, there are two equivalent method to quantise: operatorial formalism and path integrals. The latter has been introduced by Feynman and it is useful for two reasons: quantisation of (non-abelian) gauge theories in the standard model and relation between quantum field theory and statistical mechanics.

\section{Two slit experiment}

    Consider an electron, created by a source, that passes through a barrier with two slits and reach a detector. The standard way to study such system is to highlight the wave behaviour of the electron and calculate the interference pattern with the Huygens principle. However, Feynman proposes an alternative way. The electron is a particle which accomplish all the possible path with an associated amplitude. Therefore, for each path $c_i$, there is an amplitude $A (c_i)$ and the total amplitude $A_{tot} = \sum_i A(c_i)$ is related to the probability to find the particle in a point of the detector screen $p = |A_{tot}|^2$. To each amplitude, we associated a unit norm and a phase equal to $\frac{S}{\hbar}$, where $S$ is the action of the particle, 
    \begin{equation*}
        A(c_i) = \exp( \frac{i}{\hbar}) S(c_i)
    \end{equation*}
    and the total amplitude is 
    \begin{equation*}
        A_{tot} = \sum_i \exp( \frac{i}{\hbar}) S(c_i) ~.
    \end{equation*}

\subsection{Heuristical free particle} 

    Consider a free particle with associated action 
    \begin{equation*}
        S[q] = \int_0^T dt ~ \frac{m}{2} \dot q^2 ~.
    \end{equation*}
    We suppose that the difference between the two path is $d \ll D$ and we evaluate the action fot the path $c_1$ 
    \begin{equation*}
        S(c_1) = \frac{m}{2} \frac{D^2}{T^2} T = \frac{m}{2} \frac{D^2}{T}
    \end{equation*}
    and for the path $c_2$ 
    \begin{equation*}
        S(c_2) = \frac{m}{2} \frac{(D+d)^2}{T^2} T = \frac{m}{2T} (D^2 + 2 D d + O(d^2)) = \frac{m}{2} \frac{D^2}{T} + \frac{m D d}{T} O(d^2) = \frac{m}{2} \frac{D^2}{T} + pd + O(d^2) = S(c_1) + pd + O(d^2) ~,
    \end{equation*}
    where we roughly estimate $p \sim \frac{m D}{T}$. Therefore, the total amplitude is 
    \begin{equation*}
        A_{tot} = A(c_1) + A(c_2) = \exp (\frac{i}{\hbar} S(c_1)) + \exp (\frac{i}{\hbar} S(c_2)) = \exp (\frac{i}{\hbar} S(c_1)) \Big (1 + \exp (\frac{i}{\hbar} pd)) + O(d^2) ~. 
    \end{equation*}
    We notice that the maximum probability is given when 
    \begin{equation*}
        \exp(\frac{i}{\hbar} pd) = 1 \quad \Rightarrow \quad \frac{pd}{\hbar} = 2 \pi n ~,
    \end{equation*}
    where $n \in \mathbb Z$. We recover quantum mechanics, since we recognise the Compton wavelength of the electron
    \begin{equation*}
        \lambda = \frac{\hbar}{p}
    \end{equation*}
    and we find the relation 
    \begin{equation*}
        \frac{d}{\lambda} = n ~.
    \end{equation*}

    We can generalise by increasing the number of slits and intermediary screens to have all the possible paths between the initial point (source) and the final point (detector). The action becomes 
    \begin{equation*}
        S[q] = \int{t_i}^{t_f} dt ~L(q, \dot q) 
    \end{equation*}
    and the transition amplitude in the continuum limit 
    \begin{equation*}
        A = \sum_i \exp(\frac{i}{\hbar} S(c_i)) = \int \mathcal D q ~ \exp (\frac{i}{\hbar} S[q]) ~,
    \end{equation*}
    where $S[q]$ is a functional, $A$ is an integral functional and $\mathcal D q$ is the measure in the path space.

    We can recover the classical limit by noticing that, for macroscopic systems, the rate $\frac{S}{\hbar}$ is very big and small variations $\frac{\delta S}{\hbar}$ are bigger than $i \pi$. Therefore, the amplitudes of nearby paths cancel by destructive interference, unless it is the real classical path, since $\delta S = 0$. 

\section{Schroedinger equation}

    Consider a $1$-dimensional particle. The lagrangian action in the configuration space 
    \begin{equation*}
        S[x(t)] = \int dt \Big ( \frac{m}{2} \dot x^2 - V(x)) ~. 
    \end{equation*}
    The momentum is defined as 
    \begin{equation*}
        p = \pdv{L}{\dot x} = m \dot x ~.
    \end{equation*}
    By a Legendre transformation, we can introduce the hamiltonian 
    \begin{equation*}
        H = p \dot x - L = p \frac{p}{m} + \frac{p^2}{2m} + V(x) = \frac{p^2}{2m} ~.
    \end{equation*}
    The hamiltonian action in the phase space 
    \begin{equation*}
        S[x(t)] = \int dt \Big ( p \dot x - \frac{p^2}{2m} - V(x)) ~. 
    \end{equation*}

    Given $2$ function in phase space $f$ and $g$, the Poisson brackets are 
    \begin{equation*}
        \{x,x\} = \{p,p\} = 0 ~, \quad \{x,p\} = 1 ~.
    \end{equation*}

    The canonical quantisation consists in promoting position and momentum to operators $\hat x$ and $\hat p$ such that they satisfy the canonical commutation relations 
    \begin{equation*}
        [\hat x, \hat x] = [\hat p, \hat p] = 0 ~, \quad [\hat x, \hat p] = i \hbar ~.
    \end{equation*}
    Therefore, also the hamiltonian is promoted to an operator acting on the Hilbert space $\mathcal H$ 
    \begin{equation*}
        \hat H = \frac{1}{2m} \hat p^2 + V(\hat x) ~.
    \end{equation*}

    The Schroedinger equation reads as 
    \begin{equation*}
        i \hbar \pdv{}{t} \ket{\psi} = \hat H \ket{\psi} ~,
    \end{equation*}
    where $\ket{\psi} \in \mathcal H$ is a state in the Hilbert space.

    In the position representation, the eigenstates of the position are 
    \begin{equation*}
        \hat x \ket{x} = x \ket{x} ~,
    \end{equation*}
    such that they satisfy 
    \begin{equation*}
        \braket{x}{x'} = \delta (x - x') ~, \quad \mathbb I = \int dx ~ \ket{x} \bra{x} ~.
    \end{equation*}
    The momentum operator in this representation is 
    \begin{equation*}
        \hat p = - i \hbar \pdv{}{x} ~.
    \end{equation*}
    Hence, the Schroedinger equation is 
    \begin{equation*}
        i \hbar \pdv{}{t} \psi(t,x) = \Big (\frac{1}{2m} \hat p^2 + V(\hat x) \Big ) \psi(t,x) = \Big (-\frac{\hbar^2}{2m} \pdvdu{x} + V(x) \Big) \psi(t,x) ~.
    \end{equation*}

    However, for any time-independent hamiltonian, we can solve the Schroedinger equation introducing an evolution operator 
    \begin{equation*}
        \ket{\psi(t)} = \exp(- \frac{i}{\hbar} \hat H t) \ket{\psi_i} ~.
    \end{equation*}
    \begin{proof}
        In fact, 
        \begin{equation*}
            i \hbar \pdv{}{t} \ket{\psi(t)} = i \hbar \Big ( - \frac{i}{\hbar} \hat H \Big) \ket{\psi(t)} = \hat H \ket{\psi(t)} ~. 
        \end{equation*}
    \end{proof}

    The transition amplitude to find the system from the intial state $\psi_i$ to the final state $\psi_f$ is 
    \begin{equation*}
        A(x_i, x_f, T) = \braket{\psi_f}{\psi(t)} = \bra{\psi_f} \exp(- \frac{i}{\hbar} \hat H T) \ket{\psi_i} ~.
    \end{equation*}
    \begin{proof}
        In fact, 
        \begin{equation*}
        \begin{aligned}
            \braket{\psi_f}{\psi(t)} & = \bra{\psi_f} \exp(- \frac{i}{\hbar} \hat H T) \ket{\psi_i} \\ & = \bra{\psi_f} \mathbb I \exp(- \frac{i}{\hbar} \hat H T) \mathbb I \ket{\psi_i} \\ & = \int dx_f ~ \underbrace{\braket{\psi_f}{x_i}}_{\psi_f(x_f)} \bra{x_i} \exp(- \frac{i}{\hbar} \hat H T) \int dx_i ~ \ket{x_i} \underbrace{\braket{x_i}{\psi_i}}_{\psi_i(x_i)} \\ & = \int dx_i \int dx_f ~ \psi_f^* (x_f) \psi_i (x_i) \bra{x_f} \exp(- \frac{i}{\hbar} \hat H T) \ket{x_i} ~,
        \end{aligned}
        \end{equation*}
        which shows that it is a matrix element between position eigenstates of the evolution operator.
    \end{proof}

\chapter{Path integrals in phase space}

    The path integral in phase space is 
    \begin{equation*}
    \begin{aligned}
        A & = \lim_{N \rightarrow \infty} \int \Big (\prod_{k=1}^{N-1} d x_k \Big ) \Big (\prod_{k=1}^{N} \frac{d p_k}{2\pi \hbar} \Big ) \exp(\frac{i \epsilon}{\hbar} \sum_{k=1}^{N} \Big (p_k \frac{x_k - x_{k-1}}{\epsilon} - H(x_{k-1}, p_k))) \\ & = \int \mathcal D x ~ \mathcal D p ~ \exp(\frac{i}{\hbar} S[x,p]) ~,
    \end{aligned}
    \end{equation*}
    where $\epsilon =\frac{T}{N} $.

    \begin{proof}
        In fact 
        \begin{equation*}
        \begin{aligned}
            A (x_i, x_f, T) & = \bra{x_f} \exp(- \frac{i}{\hbar} \hat H T) \ket{x_i} \\ & = \bra{x_f} \exp(- \frac{i}{\hbar} \hat H \underbrace{\frac{T}{N}}_\epsilon N) \ket{x_i} \\ & = \bra{x_f} \exp(- \frac{i}{\hbar} \hat H \epsilon)^N \ket{x_i} \\ & = \bra{x_f} \mathbb I \exp(- \frac{i}{\hbar} \hat H \epsilon) \mathbb I \ldots \mathbb I \exp(- \frac{i}{\hbar} \hat H \epsilon) \mathbb I \ket{x_i} \\ & = \int \Big ( \prod_{k=1}^{N-1} dx_k \Big) \Big ( \prod_{k=1}^{N} \bra{x_k} \exp(- \frac{i}{\hbar} \hat H \epsilon) \ket{x_{k-1}} \Big) \\ & = \int \Big ( \prod_{k=1}^{N-1} dx_k \Big) \Big ( \prod_{k=1}^{N} \bra{x_k} \mathbb I \exp(- \frac{i}{\hbar} \hat H \epsilon) \ket{x_{k-1}} \Big) \\ & = \int \Big ( \prod_{k=1}^{N-1} dx_k \Big) \Big ( \prod_{k=1}^{N} \frac{dp_k}{2 \pi \hbar} \Big) \prod_{k=1}^{N}\underbrace{\braket{x_k}{p_k}}_{\exp(\frac{i}{\hbar} p_k x_k)} \bra{p_k} \exp(- \frac{i}{\hbar} \hat H \epsilon) \ket{x_{k-1}} \\ & = \int \Big ( \prod_{k=1}^{N-1} dx_k \Big) \Big ( \prod_{k=1}^{N} \frac{dp_k}{2 \pi \hbar} \Big) \prod_{k=1}^{N} \exp(\frac{i}{\hbar} p_k x_k) \bra{p_k} \exp(- \frac{i}{\hbar} \hat H \epsilon) \ket{x_{k-1}} ~.
        \end{aligned}
        \end{equation*}

        So far, we compute the exaxt development. Now, we go infinitesimally with $N \rightarrow \infty$
        \begin{equation*}
        \begin{aligned}
            A & = \lim_{N \rightarrow \infty} \int \Big ( \prod_{k=1}^{N-1} dx_k \Big) \Big ( \prod_{k=1}^{N} \frac{dp_k}{2 \pi \hbar} \Big) \prod_{k=1}^{N} \exp(\frac{i}{\hbar} p_k x_k) \bra{p_k} \exp(- \frac{i}{\hbar} \hat H \epsilon) \ket{x_{k-1}} \\ & = \lim_{N \rightarrow \infty} \int \Big ( \prod_{k=1}^{N-1} dx_k \Big) \Big ( \prod_{k=1}^{N} \frac{dp_k}{2 \pi \hbar} \Big) \prod_{k=1}^{N} \exp(\frac{i}{\hbar} (p_k (x_k - x_{k-1}) - H(x_{k-1}, p_k) epsilon )) \\ & = \lim_{N \rightarrow \infty} \int \Big ( \prod_{k=1}^{N-1} dx_k \Big) \Big ( \prod_{k=1}^{N} \frac{dp_k}{2 \pi \hbar} \Big) \prod_{k=1}^{N} \exp(\frac{i}{\hbar} \underbrace{\epsilon}_{dt} (\underbrace{p_k}_p \underbrace{\frac{x_k-x_{k-1}}{\epsilon}}_{\dot x} - \underbrace{H(x_{k-1}, p_k)}_{H} )) \\ & = \int \mathcal D x ~ \mathcal D p ~ \exp(\frac{i}{\hbar} S[x,p]) ~,
        \end{aligned}
        \end{equation*}
        where we have used
        \begin{equation*}
            \bra{p_k} \exp(- \frac{i}{\hbar} \hat H \epsilon) \ket{x_{k-1}} = \braket{p_k}{x_{k-1}} \exp(- \frac{i}{\hbar} H (x_{k-1}, p_k) \epsilon) ~.
        \end{equation*}
    \end{proof}

\chapter{Path integrals in configuration space}

    The path integral in configuration space is 
    \begin{equation*}
        A = \lim_{N \rightarrow \infty} \int \Big (\prod_{k=1}^{N-1} d x_k \Big ) \Big (\frac{m}{2 \pi i \hbar \epsilon} \Big )^{\frac{N}{2}} \exp(\frac{i \epsilon}{\hbar} \sum_{k=1}^{N} \Big (\frac{m}{2} \frac{(x_k - x_{k-1})^2}{\epsilon^2} - V(x_{k-1})\Big )) = \int \mathcal D x ~ \exp(\frac{i}{\hbar} S[x]) ~.
    \end{equation*}
    \begin{proof}
        With the use of the gaussian integral, 
        \begin{equation*}
            \int_{- \infty}^\infty dp ~ \exp(- \frac{\alpha}{2} p^2 + \beta p) = \sqrt{\frac{2\pi}{\alpha}} \exp(\frac{\beta^2}{2\alpha}) ~,
        \end{equation*}
        where in our case they are 
        \begin{equation*}
            \alpha = \frac{i\epsilon}{\hbar m} ~, \quad \beta = \frac{i}{\hbar} (x_k - x_{k-1}) ~,
        \end{equation*}
        we have 
        \begin{equation*}
        \begin{aligned}
            A & = \lim_{N \rightarrow \infty} \int \Big (\prod_{k=1}^{N-1} d x_k \Big ) \Big (\frac{1}{2\pi\hbar} \Big (\frac{2\pi \hbar m}{i \epsilon} \Big)^{\frac{N}{2}}  \exp(\frac{i \epsilon}{\hbar} \sum_{k=1}^{N} \epsilon \Big (\frac{m (x_k - x_{k-1})^2}{\epsilon^2} - V(x_{k-1})) \Big ) \Big ) \\ & = \lim_{N \rightarrow \infty} \int \prod_{k=1}^{N-1} d x_k \Big (\frac{m}{2 \pi \hbar i \epsilon} \Big)^{\frac{N}{2}}  \exp(\frac{i \epsilon}{\hbar} \sum_{k=1}^{N} \epsilon \Big (\frac{m (x_k - x_{k-1})^2}{\epsilon^2} - V(x_{k-1})) \Big ) \\ & = \int \mathcal D x ~ \exp(\frac{i}{\hbar} S[x]) ~.
        \end{aligned}
        \end{equation*}
    \end{proof}

\chapter{Free particle}

    Consider a particle with $N = 1$ and $T = \epsilon$, the path integral is 
    \begin{equation*}
        A = \sqrt{\frac{m}{2\pi\hbar i T}} \exp(\frac{i}{\hbar} \frac{m(x_f - x_i)^2}{2T}) ~.
    \end{equation*}
    \begin{proof}
        In fact, heuristically for $x_{cl} (t) = x_i + \frac{x_f - x_i}{T} t$
        \begin{equation*}
            S[x_{cl}] = \int_0^T \frac{m}{2} \frac{(x_f - x_i)^2}{T^2} = T \frac{m}{2} \frac{(x_f - x_i)^2}{T^2} =  \frac{m(x_f - x_i)^2}{2T} ~.
        \end{equation*}

        Moreover, considering $x(t) = x_{cl}(t) + q(t)$
        \begin{equation*}
        \begin{aligned}
            A & = \int \mathcal D x \exp (\frac{i}{\hbar} S[x(t)]) \\ & = \int \mathcal D x \exp (\frac{i}{\hbar} S[x_{cl}(t) + q(t)]) \\ & = \int \mathcal D x \exp (\frac{i}{\hbar} S[x_{cl}(t)] + S [q(t)]) \\ & = \int \mathcal D x \exp (\frac{i}{\hbar} S[x_{cl}(t)]) \underbrace{\int \mathcal D x \exp (\frac{i}{\hbar} S [q(t)])}_N \\ & = N \exp(\frac{i}{\hbar} S[x_{cl}(t)]) ~,
        \end{aligned}
        \end{equation*}
        where we have used for the mixed term
        \begin{equation*}
            S[x] \propto \int dt \dv{}{t} (x_{cl} + q) \dv{}{t} (x_{cl} + q) = \int dt 2 \dot x \dot q = \int dt \underbrace{\ddot x}_0 q = 0
        \end{equation*}
    \end{proof}
    It is a semiclassical approximation, since it solves the Schroedinger equation 
    \begin{equation*}
        i \hbar \pdv{}{t} A(x_i, x_f, T) = - \frac{\hbar^2}{2m} A(x_i, x_f, T) ~.
    \end{equation*}

\chapter{Wick rotation} 

    Quantum mechanics can be related to statistical mechanics by an analitic continuation $T \rightarrow i \beta$, called a Wick rotation. 

    The Schroedinger equation becomes a diffusion equation 
    \begin{equation*}
        \pdv{}{\beta} A = \frac{1}{2m} \pdvdu{}{x_f} A ~.
    \end{equation*}
    \begin{proof}
        In fact
        \begin{equation*}
            - \frac{1}{2m} \pdvdu{}{x_f} A = - \frac{1}{2m} \frac{1}{i} \pdv{}{T} = - \frac{1}{2m} \frac{1}{i} \pdvdu{}{x_f} A = \frac{1}{2m} \pdvdu{}{x_f} A ~.
        \end{equation*}
    \end{proof}

    The fundamental solution of the heat equation is 
    \begin{equation*}
        \begin{cases}
            A(x_i, x_f, \beta) = \sqrt{\frac{m}{2\pi\beta}} \exp \Big (- \frac{m (x_f - x_i)^2}{2\beta} \Big) \\
            A(x_i, x_f, \beta = 0) = \delta (x_i - x_f) \\
        \end{cases} ~.
    \end{equation*}

    Through a Wick rotation, we can obtain an euclidean metric 
    \begin{equation*}
        ds^2 = - dt^2 + dx^2 \rightarrow ds^2 dt^2 + dx^2 ~.
    \end{equation*}
    Therefore, we have an euclidean action 
    \begin{equation*}
    \begin{aligned}
        i S[x] = i \int_0^T dt \Big ( \frac{m}{2} \dot x^2 - V(x) \Big) = \int_{0}^{\beta} da \Big (- \frac{m}{2} \Big (\dv{x}{a} \Big )^2 - V(x) \Big ) = - \int_{0}^{\beta} da \Big (\frac{m}{2} \Big (\dv{x}{a} \Big )^2 + V(x) \Big ) = - S_E [x] ~,
    \end{aligned}
    \end{equation*}
    with an euclidean path integral 
    \begin{equation*}
        \int \mathcal D x \exp(- S_E [x]) ~.
    \end{equation*}

    To summarise 
    \begin{equation*}
    \begin{aligned}
        & A(x_i, x_f, T) = \bra{x_f} \exp(- i t \hat H) \ket{x_i} = \int \mathcal D x \exp(i t S[x]) \\ & \rightarrow A(x_i, x_f, T) = \bra{x_f} \exp(- \beta \hat H) \ket{x_i} = \int \mathcal D x \exp(- S_E [x]) 
    \end{aligned}
    \end{equation*}
    and 
    \begin{equation*}
        i \pdv{}{t} A = \hat H A \rightarrow \pdv{}{\beta} A = - \hat H A ~.
    \end{equation*}

    The connection with statistical mechanics can be made by computing the canonical partition function, in quantum mechanics 
    \begin{equation*}
        Z = \tr \exp(- i \hat H t) = \sum_n \exp(-i E_n t) = \int dx \bra{x} \exp(- i \hat H t) \ket{x} = \int_{PBC} \mathcal D x \exp (i S [x]) ~,
    \end{equation*}
    where we have used a discrete energy spectrum (otherwise there would have been an integral), we have used position eigenstates and the path integral is on a circle with periodic boundary condition $x(T) = x(0)$; and in statistical mehcanics 
    \begin{equation*}
        Z_E = \tr \exp(- \beta \hat H) = \sum_n \exp (-\beta E_n) = \int dx \bra{x} \exp(- \beta \hat H) \ket{x} = \int_{PBC} \mathcal D x \exp (- S_E [x]) ~,
    \end{equation*}
    where the path integral is on a circle with periodic boundary condition $x(\beta) = x(0)$.

    Few comments are to be made. First, we need some (dimensional) regularisation, for instance we need an imaginary time to preserve gauge symmetry. Second, since so far we only have compute the $1$-dimensional case, we need to generalise for the finite or infinite degrees of freedom. For instance, in the $3$-dimensional particle
    \begin{equation*}
        \int_{\mathbf R^3} \mathcal D x ~ \exp (i S[x]) = \lim_{N \rightarrow \infty} \int \Big (\prod_{k=1}^{N-1} d^3 x_k \Big ) \Big (\frac{m}{2\pi i \hbar \epsilon} \Big)^{\frac{3N}{2}} \exp \Big (\frac{i}{\hbar} \sum_{k=1}^{N} \epsilon \Big (\frac{m}{2} \frac{(\mathbf x_k - \mathbf x_{k-1})^2}{\epsilon^2} - V (\mathbf x_{k-1}) \Big )  \Big )  ~.
    \end{equation*}
    Third, there is an ambiguity in choosing the argument of the potential: the prepoint discretisation chooses $V(x_{k-1})$, the midpoint discretisation chooses $V(\frac{x_k - x_{k-1}}{2})$ and the postpoint discretisation chooses $V(x_{k})$. For gauge theoru, the midpoint prescription is chosen. However all the quantisation methods (among ordering) are equivalent, since there are the same ambiguities in all methods. 

\chapter{Correlation functions}

    We define the normalised $n$-point correlation function as 
    \begin{equation*}
    \begin{aligned}
        \av{x(t_1), \ldots x(t_n)} & = \frac{\int \mathcal D x ~ x(t_1) \ldots x(t_n) \exp(\frac{i}{\hbar} S[x])}{\int \mathcal D x ~ \exp(\frac{i}{\hbar} S[x])} \\ & = \frac{1}{Z} \int \mathcal D x ~ x(t_1) \ldots x(t_n) \exp(\frac{i}{\hbar} S[x])~,
    \end{aligned}
    \end{equation*}
    where $x(t)$ is a dynamical variable. 

    The generating functional of correlation functions is 
    \begin{equation*}
        Z[J] = \int \mathcal D x ~ \exp \Big(\frac{i}{\hbar} S[x] + \frac{i}{\hbar} \int_0^T dt ~ J(t) x(t) \Big) ~,
    \end{equation*}
    where the functional $J(t)$ is called the source. 

    We can write the normalised $n$-point correlation function in terms of the generating functional in the following way 
    \begin{equation*}
        \av{x(t_1), \ldots x(t_n)} = \frac{1}{Z[0]} \Big ( \frac{\hbar}{i} \Big)^n \frac{\delta^n Z[J]}{\delta J (t_1) \ldots \delta J(t_n)} \Big \vert_{J=0} ~.
    \end{equation*}

    The $0$-point function is 
    \begin{equation*}
        \av{1} = \frac{Z}{Z} = 1~.
    \end{equation*}

    If you know $Z[J]$, you know all the correlation functions. 


    In the Schroedinger picture, the correlation function is 
    \begin{equation*}
        \av{x(t_1) \ldots x(t_n)} = \frac{1}{Z} \bra{x_f} \exp(- \frac{i}{\hbar} \hat H (t_f - T_n) \hat x \ldots \hat x \exp(- \frac{i}{\hbar} \hat H (T_1 - t_i))) \ket{x_i} ~,
    \end{equation*}
    where we have ordered to start from the earliest $T_1$ to the latest $T_n$. In the Heisenberg picture, the correlation function is 
    \begin{equation*}
        \av{x(t_1) \ldots x(t_n)} = \frac{1}{Z} \bra{x_f, t_f} T \hat h_H (t_1) \ldots \hat h_H (t_n) \ket{x_i, t_i} ~,
    \end{equation*}
    where $T$ is the time-ordered prescription. For example, a $2$-point correlation function is 
    \begin{equation*}
        \bra{0} T \hat \varphi (x) \hat \varphi (y) \ket{0} ~.
    \end{equation*}

\chapter{Gaussian integrals}

    We make a list of useful gaussian integrals 
    \begin{equation*}
        \int_{-\infty}^\infty \frac{d\phi}{2\pi} \exp(- \frac{1}{2} K \phi^2) = \frac{1}{\sqrt{K}} ~,
    \end{equation*}
    with $K \geq 0$.
    \begin{equation*}
        \int_{-\infty}^\infty \frac{d\phi}{2\pi} \exp(- \frac{1}{2} K \phi^2 + J \phi) = \frac{1}{\sqrt{K}} \exp(\frac{1}{2K} J^2) \quad K \geq 0 ~.
    \end{equation*}

    \begin{equation*}
        \int_{\mathbb R^n} \frac{d^n \phi}{(2\pi)^{\frac{n}{2}}} \exp(- \frac{1}{2} \phi^i K_{ij} \phi^j ) = \frac{1}{\sqrt{\det k_{ij}}} ~,
    \end{equation*}
    where $K_{ij}$ is symmetric, real, positive defined and $\det K_{ij}$ is the product of the eigenvalues.
    \begin{equation*}
        \int_{\mathbb R^n} \frac{d^n \phi}{(2\pi)^{\frac{n}{2}}} \exp(- \frac{1}{2} \phi^i K_{ij} \phi^j + J_i \phi^i) = \frac{1}{\sqrt{\det k_{ij}}} \exp( \frac{1}{2} J_i G^{ij} J_j ) ~,
    \end{equation*}
    where $G^{ij}$ is the inverse of $K_{ij}$ such that $K_{ij}G^{jl} = \delta_i^{\phantom i l}$.

    By analitic continuation, we have 
    \begin{equation*}
        \int_{\mathbb R^n} \frac{d^n \phi}{(-i2\pi)^{\frac{n}{2}}} \exp(- \frac{i}{2} \phi^i K_{ij} \phi^j + i J_i \phi^i) = \frac{1}{\sqrt{\det k_{ij}}} \exp( \frac{i}{2} J_i G^{ij} J_j ) ~,
    \end{equation*}

    but in order to converge, we must have $K_{ij} - i \epsilon \delta_{ij}$, so that a damping exponential compares.

\chapter{Hypercondensed notation}

    In the hypercondensed notation, we can write 
    \begin{equation*}
        x(t) \rightarrow \phi^i ~, \quad x \rightarrow \phi ~, t \rightarrow i ~,
    \end{equation*}
    \begin{equation*}
        A_\mu(x^\nu) \rightarrow \phi^i ~, \quad A \rightarrow \phi ~, (\phantom{|}_\mu, x^\nu) \rightarrow i ~,
    \end{equation*}
    Notice that now $i$ is a continuous index. Furthermore 
    \begin{equation*}
        \phi^i \phi_i = \int dt x(t) x(t) = \int dt \int dt' x(t) \delta (t - t') x(t') ~.
    \end{equation*}
    or
    \begin{equation*}
        \phi^i \phi_i = \int d^4 x A_\mu (x ) A^\mu (y) = \int d^4 x \int d^4 y A_\mu (x) \eta^{\mu\nu} \delta^4 (x-y) A_\nu (y) ~.
    \end{equation*}

    We can introduce the generating functional of connected correlation functional 
    \begin{equation*}
        W[J] = \frac{\hbar}{i} \ln Z{J} ~,
    \end{equation*}
    which implies that 
    \begin{equation*}
        Z{J} = \exp (\frac{i}{\hbar} W[J]) ~.
    \end{equation*}

    We can write the normalised $n$-point correlation function in terms of the generating functional of connected correlation functions in the following way 
    \begin{equation*}
        \av{\phi^{i_1} \ldots \phi^{i_n}}_c = \Big ( \frac{\hbar}{i} \Big)^{n-1} \frac{\delta^n W[J]}{\delta J_{i_1} \ldots \delta J_{i_n}} \Big \vert_{J=0} ~.
    \end{equation*}

    The effective action-generating functional of $1$-particle irreducible is 
    \begin{equation*}
        \Gamma (\varphi) = \min_J (W[J] - J^i \varphi_i) ~,
    \end{equation*}
    where the procedure to calculate $\varphi$ is 
    \begin{equation*}
        \varphi^i = \dvf{W[J]}{J_i} ~.
    \end{equation*}
    \begin{proof}
        The condition of minimum implies 
        \begin{equation*}
            \dvf{W[J]}{J_i} -  \varphi^i = 0 ~.
        \end{equation*}
        Therefore 
        \begin{equation*}
        \varphi^i = \dvf{W[J]}{J_i} ~,
        \end{equation*}
        and we can invert the relation from $\varphi^i = \varphi^i (J)$ to $T_i = T_i(\varphi)$.
    \end{proof}

    Notice that it is a Legendre transform. 


