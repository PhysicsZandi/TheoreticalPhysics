\part{Free theory}

\chapter{Free theory}   

    Consider a quadratic action 
    \begin{equation*}
        S[\phi] = \frac{1}{2} \phi^i K_{ij} \phi^j ~,
    \end{equation*}
    where $K_{ij}$ is an invertible matrix and the equations of motion are 
    \begin{equation*}
        K_{ij} \phi^j = 0 ~.
    \end{equation*}
    The generating functional is 
    \begin{equation*}
        Z[J] = \int \mathcal D x ~ \exp(i S[\phi] + i J_i \phi^i) = \det^{-1/2} K_{ij} \exp(\frac{1}{2} J_i G^{ij} J_j) ~,
    \end{equation*}
    through which we can obtain all normalised correlation functions. 
    The generating functional of connected correlation functions is 
    \begin{equation*}
        W[J] = \frac{1}{2} J_i G^{ij} J_i - \Lambda ~,
    \end{equation*} 
    where $\Lambda = \frac{i}{2} \ln \det K_{ij}$.
    \begin{proof}
        In fact 
        \begin{equation*}
            W[J] = - i \ln Z[J] = - \underbrace{\frac{i}{2} \ln \det K_{ij}}_\Lambda + \frac{1}{2}  J_i G^{ij} J_i = \frac{1}{2} J_i G^{ij} J_i - \Lambda  ~.
        \end{equation*}
    \end{proof}
    The effective action is 
    \begin{equation*}
        \Gamma[\varphi] = - \frac{1}{2} \varphi^i K_{ij} \varphi^j - \Lambda ~.
    \end{equation*}
    \begin{proof}
        In fact 
        \begin{equation*}
            \varphi^i = \dvf{W}{J_i} = G^{ij} J_j
        \end{equation*}
        and inverting it 
        \begin{equation*}
            J_i = K_{ij} \varphi^i ~.
        \end{equation*}
        Therefore
        \begin{equation*}
            \Gamma[\varphi] = \min_J (W[J] - J_i \varphi^i) = \frac{1}{2} J_i G^{ij} J_j - \Lambda - \underbrace{\varphi^i K_{ij} \varphi^j}_{J_i G^{ij} J_j} = - \frac{1}{2} \varphi^i K_{ij} \varphi^j - \Lambda ~.
        \end{equation*}
    \end{proof}

    The correlation function is
    \begin{equation*}
        \av{\phi^{i_1} \ldots \phi^{i_n}} = \frac{1}{Z} (\frac{1}{i})^n \frac{\delta^n Z[J]}{\delta J_{i_1} \ldots \delta J_{i_n}} \Big \vert_{J=0} ~.
    \end{equation*}
    Some examples are 
    \begin{equation*}
        \av{1} = \frac{Z}{Z} = 1 ~,
    \end{equation*}
    \begin{equation*}
        \av{\phi^i} = \frac{1}{Z} \frac{1}{i} \dvf{Z}{J_i} \Big \vert_{J=0} = \frac{1}{iZ} G^{ij} J_j \exp(\frac{1}{2} J_i G^{ij} J_j) \Big \vert_{J=0} = 0 ~,
    \end{equation*}
    \begin{equation*}
        \av{\phi^i \phi^j} = \frac{1}{Z} (\frac{1}{i})^2 \frac{\delta^2 Z[J]}{\delta J_{i} \delta J_j} \Big \vert_{J=0} = - i G^{ij} ~.
    \end{equation*}
    Notice that $1$ point correlation function vanishes since the action is invariant under $\phi' = - \phi$ and it can be generalise for all odd point functions. On the other hand, even point functions can be always decomposed into $2$ point ones. For example 
    \begin{equation*}
        \av{\phi^1 \phi^2 \phi^3 \phi^3} = (\frac{1}{i})^4 \frac{\delta^4 \exp(\frac{1}{2} J_i G^{ij} J_j)}{\delta J_1 \delta J_2 \delta J_3 \delta J_4} = - G^{12} G^{34} - G^{13} G^{24} - G^{14} G^{23} =  \av{\phi^1 \phi^2} \av{\phi^3 \phi^4} + \av{\phi^1 \phi^3} \av{\phi^2 \phi^4} + \av{\phi^1 \phi^4} \av{\phi^2 \phi^3} ~. 
    \end{equation*}
    This result is called the Wick's theorem, ehich states that in a quadratic theory, odd point functions vanish and even point functions can be expressed by $2$ point functions and connect in all possible ways all the points. There are $(n - 1)!!$ Wick contractions.

    Reintroducing $\hbar$, we obtain 
    \begin{equation*}
        S[\phi] = - \frac{1}{2} \phi^i K_{ij} \phi^j ~,
    \end{equation*}
    \begin{equation*}
        Z[J] = \det^{-1/2} K_{ij} \exp(\frac{1}{2 \hbar} J_i G^{ij} J_j) ~,
    \end{equation*}
    \begin{equation*}
        W[J] = \frac{1}{2} J_i G^{ij} J_j - \hbar \Lambda ~,
    \end{equation*}
    \begin{equation*}
        \Lambda[\varphi] = - \frac{1}{2} \phi^i K_{ij} \phi^j - \hbar \Lambda = S[\varphi] + \hbar \textnormal{corrections} ~,
    \end{equation*}
    \begin{equation*}
        \av{\phi^i \phi^j} = - i \hbar G^{ij} ~.
    \end{equation*}

\chapter{Harmonic oscillator}

    Consider an harmonic oscillator with mass $m=1$ and frequency $\omega$. Its action is 
    \begin{equation*}
    \begin{aligned}
        S[x] & = \int_{-\infty}^\infty dt ~ \Big ( \frac{1}{2} \dot x^2 - \frac{1}{2} \omega^2 x^2 \Big) \\ & = - \frac{1}{2} \int_{-\infty}^\infty dt ~ \Big (\dvd{}{t} + \omega^2 \Big) x^2 \\ & = - \frac{1}{2} \int_{-\infty}^\infty dt \int_{-\infty}^\infty dt' x(t) \Big (\dvd{}{t} + \omega^2 \Big) \delta (t - t') x(t') \\ & = - \frac{1}{2} \phi^i K_{ij} \phi^j ~,
    \end{aligned}
    \end{equation*}
    where we have intergated by parts and we define 
    \begin{equation*}
        K_{ij} = \Big (\dvd{}{t} + \omega^2 \Big) \delta (t - t') ~, \quad \phi^i = x(t) ~.
    \end{equation*}

    The Green function representation in Fourier space is 
    \begin{equation*}
        G(t, t') = \int \frac{dp}{2\pi} \frac{\exp(- i p (t - t'))}{-p^2 + \omega^2} ~.
    \end{equation*}
    \begin{proof}
        From the definition 
        \begin{equation*}
            K(t, t') G(t', t'') = \delta (t - t'') ~ ~,
        \end{equation*}
        we have 
        \begin{equation*}
            G(t, t') = \int \frac{dp}{2\pi} \frac{\exp(- i p (t - t'))}{-p^2 + \omega^2} ~.
        \end{equation*}
    \end{proof}

    Using the Feynman-Stueckelberg prescription, we obtain 
    \begin{equation*}
        G(t, t') = \frac{i}{2\pi} \exp(- i \omega |t - t'|) ~,
    \end{equation*}
    which means adding an $\epsilon \rightarrow 0^+$ factor in order to have a damping factor in the path integrals.
    Therefore, the path integral is  
    \begin{equation*}
        Z[J] = \mathcal N \exp(\frac{i}{2\hbar} \int dt \int dt' J(t) G(t, t'0 J(t'))) ~.
    \end{equation*}
    \begin{proof}
        In fact 
        \begin{equation*}
        \begin{aligned}
            Z[J] = & \int \mathcal D x ~ \exp(\frac{i}{\hbar} S[x] + \frac{i}{\hbar} \int dt ~ J(t) x(t)) \\ & = \int \mathcal D x ~ \exp(\int dt \int dt' ~ (\frac{1}{2} x(t) K(t, t') x(t') - J(t) \delta(t-t') x(t'))) \\ & = \int \mathcal D x ~ \exp(\int dt \int dt' ~ (\frac{1}{2} x(t) K(t, t') x(t') - J(t) \delta(t-t') x(t')) + \frac{1}{2} J(t) G(t,t') J(t') - \frac{1}{2} J(t) G(t,t') J(t')) \\ & = \exp \Big(\frac{i}{2\hbar} \int dt \int dt' ~ J(t) G(t, t') J(t') \Big) \int \mathcal D x ~ \exp(- \frac{i}{\hbar} \int dt \int dt' ~ \tilde x(t) K(t, t') \tilde x(t')) \\ & = \exp \Big(\frac{i}{2\hbar} \int dt \int dt' ~ J(t) G(t, t') J(t') \Big) \underbrace{\int \mathcal D \tilde x ~ \exp(- \frac{i}{\hbar} \int dt \int dt' ~ \tilde x(t) K(t, t') \tilde x(t'))}_{\mathcal N} \\ & = \mathcal N \exp(\frac{i}{2\hbar} \int dt \int dt' J(t) G(t, t'0 J(t'))) ~,
        \end{aligned}
        \end{equation*}
        where we have used the translation invariance of the measure and 
        \begin{equation*}
            \tilde x(t) = x(t) ~ \int dt'' ~ G(t, t'') J(t'') ~.
        \end{equation*}
    \end{proof}

    The $2$ point function is 
    \begin{equation*}
        \av{x(t) x(t')} = \frac{\int \mathcal D x ~ x(t) x(t) \exp(\frac{i}{\hbar} S[x])}{\int \mathcal D x ~ \exp(\frac{i}{\hbar} S[x])} = \frac{1}{Z[0]} (\frac{\hbar}{i})^2 \frac{\delta^2 Z[J]}{\delta J(t) \delta J(t')} = - i \hbar G(t, t') = \frac{\hbar}{2 \omega} \exp(- i \omega |t - t') ~.
    \end{equation*}

\section{Harmonic oscillator in Euclidean time}

    The path integral in Euclidean time, using a discrete diagonalised energy eigenbasis, 
    \begin{equation*}
        Z_E = \tr \exp(- \beta \hat H) = \sum_n \exp(- \beta E_n) \xrightarrow{\beta \rightarrow \infty} \exp(- \beta E_0) + \textnormal{subleading terms} = \bra{0} \exp(-\beta \hat H) \ket{0} ~,
    \end{equation*}
    which is the energy of the vacuum state. Therefore 
    \begin{equation*}
        Z_E [x] = \int_{PBC} \mathcal D x ~ \exp(- S_E[x] + \int d\tau ~ J(\tau) x(\tau)) \xrightarrow{\beta \rightarrow \infty} ~,
    \end{equation*}
    where 
    \begin{equation*}
        S_E[x] = \int_{-\beta/2}^{\beta/2} d\tau ~ ( \frac{1}{2} m \dot x^2 + \frac{\omega^2}{2} x^2)~.
    \end{equation*}

    The Green function is 
    \begin{equation*}
        G_E (\tau, \tau') = \int \frac{dp}{2\pi} \frac{\exp(- i p (t - t'))}{p_E^2 + \omega^2} = \frac{1}{2\omega} \exp(- \omega |\tau - \tau'|) ~.
    \end{equation*}
    The $2$ point function is 
    \begin{equation*}
        \av{x(\tau) x(\tau')} = G_E(\tau, \tau') = \frac{1}{2\omega} \exp(- \omega |\tau - \tau'|) ~.
    \end{equation*}

    Notice that in order to preserve the Fourier transform, we need to inverse Wick rotate the momentum, i.e. $t \rightarrow - i \tau$ and $p_M \rightarrow i P_E$. Hence, after a Wick rotation 
    \begin{equation*}
    \begin{aligned}
        \av{x(\tau) x(\tau')} & = G_E(\tau, \tau') = \int \frac{dp_E}{2\pi} \frac{\exp(- i p_E (t - t'))}{p_E^2 + \omega^2} \\ & \xrightarrow{\textnormal{Wick}} - i \int \frac{dp_M}{2\pi} \frac{\exp(- i p_M (t - t'))}{p_E^2 + \omega^2} = - i G(t,t') = \av{x(t) x(t')}
    \end{aligned}
    \end{equation*}
    and 
    \begin{equation*}
        \frac{1}{2\omega} \exp(- \omega |\tau - \tau'|) \xrightarrow{\textnormal{Wick}} \frac{1}{2\omega} \exp(- i \omega |t - t'|) ~.
    \end{equation*}

\chapter{Klein-Gordon theory}

    Consider a Klein-Gordon field. Its action is 
    \begin{equation*}
    \begin{aligned}
        S[x] & = \int d^4 x ~ \Big ( - \frac{1}{2} \partial_\mu \phi \partial^\mu \phi - \frac{1}{2} m^2 \phi^2 \Big) \\ & = \int d^4 x ~ ( \frac{1}{2} \phi \Box \phi - \frac{1}{2} \phi m^2 \phi) \\ & = - \frac{1}{2} \int d^4 x \int d^4 y \phi(x) (- \Box_x + m^2) \delta^4 (x - y) \phi(y) \\ & = - \frac{1}{2} \phi^i K_{ij} \phi^j ~,
    \end{aligned}
    \end{equation*}
    where we have intergated by parts and we define 
    \begin{equation*}
        K_{ij} = \Big (- \Box_x + m^2 \Big) \delta^4 (x - y) ~, \quad \phi^i = \phi(x) ~.
    \end{equation*}

    The Green function representation in Fourier space is 
    \begin{equation*}
        G(x, y) = \int \frac{d^4p}{(2\pi)^4} \frac{\exp(- i p_\mu (x^\mu - y^\mu))}{p^2 + \omega^2} ~.
    \end{equation*} 

    We can recover the harmonic oscillator from the Klein-Gordon theory by rescricting ourselves to the situation in which we have only the time $0$-dimension. In fact, the Klein-Gordon action becomes 
    \begin{equation*}
        \int dt (\frac{1}{2} \dot \phi^2 - \frac{1}{2} m^2 \phi^2)
    \end{equation*}
    where we can identify $\omega = m$. 

\chapter{Perturbation expansion}

    The model that we have in mind is the anharmonic oscillator, which action is
    \begin{equation*}
        S[x] = \int dt (\frac{1}{2} \dot x^2 - \frac{1}{2} \omega^2 x^2 - \frac{g}{3!} x^3 - \frac{\lambda}{4!} x^4 + \ldots) = S_0 + S_{int} ~
    \end{equation*}
    where $g$ and $\lambda$ are small perturbative quantity, called coupling constants.

    The path integral is 
    \begin{equation*}
    \begin{aligned}
        Z[J] & = \int \mathcal D x ~ \exp(\frac{i}{\hbar} S[x] + \frac{i}{\hbar} \int dt ~ J(t) x(t)) \\ & = \int \mathcal D x ~ \exp(\frac{i}{\hbar} S_{int}[x]) \underbrace{\exp(\frac{i}{\hbar} S_0[x] + \frac{i}{\hbar} \int dt ~ J(t) x(t))}_{Z_0[J]} \\ & = \int \mathcal D x ~ \Big ( 1 + (\frac{i}{\hbar} S_{int}[x]) + \frac{1}{2} (\frac{i}{\hbar} S_{int}[x])^2 + \ldots \Big ) Z_0[J] \\ & = \av{\exp(\frac{i}{\hbar} S_{int}[x])}_{0, U} \\ & = \av{1}_{0, U} + \av{\frac{i}{\hbar} S_{int}[x]}_{0, U} + \frac{1}{2} \av{(\frac{i}{\hbar} S_{int}[x])^2}_{0, U} ~.
    \end{aligned}
    \end{equation*}
    where $0$ means free theory and $U$ not normalised. We can intepret this expression as a differential operator
    \begin{equation*}
        \int \mathcal D x ~ \exp(\frac{i}{\hbar} S_{int}[\frac{\hbar}{i} \dvf{}{J}]) Z_0[J] = \int \mathcal D x ~ \Big ( 1 + (\frac{i}{\hbar} S_{int}[\frac{\hbar}{i} \dvf{}{J}]) + \frac{1}{2} (\frac{i}{\hbar} S_{int}[\frac{\hbar}{i} \dvf{}{J}])^2 + \ldots \Big ) Z_0[J] ~.
    \end{equation*}

\section{Vacuum diagrams}

    In this section, we will work in Euclidean time. The path integral is
    \begin{equation*}
        Z_E[J] = \int \mathcal D x ~ \exp(-S_E[x] + \int d\tau ~ x(\tau J(\tau))) ~,
    \end{equation*}
    where the Euclidean action is 
    \begin{equation*}
        S_E = \lim_{\beta \rightarrow \infty} \int_{-\beta/2}^{\beta/2} d\tau (\frac{1}{2} \dot x^2 + \frac{1}{2} \omega^2 x^2 + \frac{g}{3!} x^3 + \frac{\lambda}{4!} x^4) ~.
    \end{equation*}

    Consider the correction to the vacuum energy 
    \begin{equation*}
    \begin{aligned}
        Z_E[0] & = \int \mathcal D x ~ \exp(- S_E[x]) \\ & = \av{1}_U \\ & = \lim_{\beta \rightarrow \infty} \bra{0} \exp(- \beta \hat H) \ket{0} = \lim_{\beta \rightarrow \infty} \exp(- \beta E_0) \\ & = \av{\exp(- S_{E, int} [x])} _{U, 0} \\ & = \lim_{\beta \rightarrow \infty} \exp(- \beta (E_0^{(0)} + \Delta E)) ~,
    \end{aligned}
    \end{equation*}
    where 
    \begin{equation*}
        E_0 = E_0^{(0)} + \Delta E ~.
    \end{equation*}

    Now we compute the energy corrections $\Delta E$. 

\subsection{First case}
    Consider the case in which $g = 0$ and $\lambda \neq 0$. The path integral is 
    \begin{equation*}
    \begin{aligned}
        Z[0] & = \av{1 - S_{E, int} [x] + \ldots} \\ & = \av{1}_{U, 0} - \int_{- \beta / 2}^{\beta/2} d\tau ~ \frac{\lambda}{4!} \av{x^4(\tau)}_{U, 0} \frac{\av{1}}{\av{1}} + \ldots \\ & = \av{1} ( 1 - \int_{- \beta / 2}^{\beta/2} d\tau ~ \frac{\lambda}{4!} \av{x(\tau) x(\tau) x(\tau) x(\tau)}_{U, 0} ) ~.
    \end{aligned}
    \end{equation*} 
    Now, we have to compute the $4$ point function (which has $(n-1)!! = 3$ terms)
    \begin{equation*}
        \av{x(\tau) x(\tau) x(\tau) x(\tau)} = 3 \av{x(\tau) x(\tau)}^2 ~,
    \end{equation*}
    where we have used the Wick theorem. For the harmonic oscillator is 
    \begin{equation*}
        \av{x(\tau) x(\tau) x(\tau) x(\tau)} = \frac{3}{4 \omega^2} ~.
    \end{equation*}
    Therefore
    \begin{equation*}
    \begin{aligned}
        Z[0] & = \av{1} ( 1 - \int_{- \beta / 2}^{\beta/2} d\tau ~ \frac{\lambda}{4!} \frac{3}{4 \omega^2} + \ldots) \\ & = \av{1} ( 1 - \frac{\lambda \beta}{32 \omega^2} + \ldots) \\ & = \av{1} \exp(- \frac{\beta \lambda}{32 \omega^2} ) ~.
    \end{aligned}
    \end{equation*}
    Finally, the energy correction is 
    \begin{equation*}
        \Delta E = \frac{\lambda}{32 \omega^2} ~.
    \end{equation*}

\subsection{Second case}
    Consider the case in which $g \neq 0$ and $\lambda = 0$. The path integral is 
    \begin{equation*}
    \begin{aligned}
        Z[0] & = \av{1 - S_{E, int} [x] + \frac{1}{2} S^2_{E, int} [x]\ldots} \\ & = \av{1} ( 1 - \int_{- \beta / 2}^{\beta/2} d\tau ~ \frac{g}{3!} \cancel{\av{x^3(\tau)}} + \frac{1}{2} \int_{- \beta / 2}^{\beta/2} d\tau \int_{- \beta / 2}^{\beta/2} d\tau' ~ (\frac{g}{3!})^2 \av{x^3(\tau) x^3(\tau')}) ~.
    \end{aligned}
    \end{equation*} 
    Now, we have to compute the $6$ point function (which has $(n-1)!! = 15$ terms)
    \begin{equation*}
        \av{x^3(\tau) x^3(\tau')} = 6 G_E^3(\tau - \tau') + 9  G_E(\tau - \tau) G_E(\tau - \tau') G_E(\tau' - \tau') ~,
    \end{equation*}
    where we have used the Wick theorem. For the harmonic oscillator is 
    \begin{equation*}
        \av{x^3(\tau) x^3(\tau')} = \frac{6}{8 \omega^3} \exp(- 3 \omega |\tau - \tau'|) + \frac{9}{8 \omega^3} \exp(- \omega |\tau - \tau'|) ~.
    \end{equation*}
    Therefore
    \begin{equation*}
    \begin{aligned}
        Z[0] & = \av{1} ( 1 + \frac{1}{2} (\frac{g}{3!})^2 \int_{- \beta / 2}^{\beta/2} d\tau \int_{- \beta / 2}^{\beta/2} d\tau' ~ (\frac{6}{8 \omega^3} \exp(- 3 \omega |\tau - \tau'|) + \frac{9}{8 \omega^3} \exp(- \omega |\tau - \tau'|)) + \ldots) \\ & = \av{1} ( 1 + \frac{1}{2} (\frac{g}{3!})^2 \frac{1}{8 \omega^3} \int_{- \beta / 2}^{\beta/2} d\tau \int_{- \beta / 2}^{\beta/2} d\tau' ~ (6 \exp(- 3 \omega |\tau - \tau'|) + 9 \exp(- \omega |\tau - \tau'|)) + \ldots) ~.
    \end{aligned}
    \end{equation*}
    Now, we have to evaluate the integral
    \begin{equation*}
        \int_{- \beta / 2}^{\beta/2} d\tau \int_{- \beta / 2}^{\beta/2} d\tau' ~ (6 \exp(- 3 \omega |\tau - \tau'|) + 9 \exp(- \omega |\tau - \tau'|)) ~.
    \end{equation*}
    We make a change of variable $|\sigma| = |\tau - \tau'|$ and compute the limit $\beta \rightarrow \infty$
    \begin{equation*}
    \begin{aligned}
        & \underbrace{\int_{- \beta / 2}^{\beta/2} d\tau}_\beta \int_{-\infty}^{\infty} d\sigma ~ (6 \exp(- 3 \omega |\sigma|) + 9 \exp(- \omega |\sigma|)) \\ & = \beta (\int_{0}^{\infty} d\sigma ~ (6 \exp(- 3 \omega \sigma) + 9 \exp(- \omega \sigma))  + \int_{-\infty}^{0} d\sigma ~ (6 \exp(3 \omega \sigma) + 9 \exp(\omega \sigma)))  ~.
    \end{aligned}
    \end{equation*}
    We make a change of variable $\sigma = - \sigma$ in the second integrand 
    \begin{equation*}
    \begin{aligned}
        & \beta (\int_{0}^{\infty} d\sigma ~ (6 \exp(- 3 \omega \sigma) + 9 \exp(- \omega \sigma)) - \int_{-\infty}^{0} d\sigma ~ ( -6 \exp(3 \omega \sigma) + 9 \exp(- \omega \sigma))) \\ & = 2 \beta \int_{0}^{\infty} d\sigma ~ (6 \exp(- 3 \omega \sigma) + 9 \exp(- \omega \sigma)) \\ & = 2 \beta (\frac{6}{3 \omega} + \frac{9}{\omega}) ~.
    \end{aligned}
    \end{equation*}
    Therefore,
    \begin{equation*}
    \begin{aligned}
        Z[0] & = \av{1} ( 1 + \frac{1}{2} (\frac{g}{3!})^2 \frac{1}{8 \omega^3}2 \beta (\frac{6}{3 \omega} + \frac{9}{\omega}) + \ldots) \\ & = \av{1} ( 1 +  (\frac{g}{3!})^2 \frac{1}{8 \omega^3} \beta \frac{11}{\omega}  + \ldots) \\ & = \av{1} \exp(\frac{11 \beta g^4}{288 \omega^4} )~.
    \end{aligned}
    \end{equation*}

    Finally, the energy correction is 
    \begin{equation*}
        \Delta E = - \frac{11 g^4}{288 \omega^4} ~.
    \end{equation*}

    \section{Feynman diagrams}

