\part{Einstein's field equations}

\chapter{Principles}

    \begin{princ}[of Galileian relativity]
        The laws of Newtonian mechanics are the same for all inertial observers and time is absolute.
    \end{princ}

    \begin{princ}[of special relativity]
        The laws of physics are the same for all inertial observers and the speed of light in vacuum is invariant.
    \end{princ}

    \begin{princ}[of general relativity]
        The laws of Newtonian mechanics are the same for all observers (in all reference frames).
    \end{princ}

\section{Equivalence principles}

    \begin{princ}[weak equivalence]
        For all physics objects, the gravitational charge mass $m_g$ equals the inertial mass $m_i$.
    \end{princ}

    \begin{princ}[Einstein's equivalence]
        Motion in a uniform gravitational field cannot be distinguished from free fall.
    \end{princ}

    \begin{princ}[strong equivalence]
        There always exists a local reference frame in which all gravitational effects vanish.
    \end{princ}

\section{General covariance}

    \begin{princ}[weak equivalence]
        The laws of physics in a general frame are obtained from the laws of special relativity by replacing tensor quantities of the Lorentz group with tensor quantities of the spacetime manifold.
    \end{princ}

\chapter{Equations}

    In this chapter, we would like to answer to two questions: what are the gravitational test particles? What are the gravitational sources?

\section{Gravitational test particles}

    By the Einstein's equivalence principle, we can associate inertial observers to freely falling objects. Consider a test-particle with $4$-velocity $u^\mu = \dv{x^\mu}{\tau}$ subjected only to gravity. It must move along a straight line $x^i = x^i_0 + u^i_0 t$ which satisfies Newton's law 
    \begin{equation*}
        \dvd{x^i}{t} = 0 ~.
    \end{equation*}
    which can be put in a covariant form since $\dvd{t}{t} = 0$ 
    \begin{equation*}
        \dvd{x^\alpha}{\tau} = \gamma^2 \dvd{x^\alpha}{t} = 0 ~.
    \end{equation*}
    In a local inertial frame $\Gamma^\alpha_{\mu\nu} = 0$, therefore 
    \begin{equation*}
        0 = \dvd{x^\mu}{\alpha} = \dvd{x^\mu}{\alpha} + \Gamma^\alpha_{\mu\nu} \dv{x^\mu}{\tau} \dv{x^\nu}{\tau} = u^\mu \nabla_\mu u^\alpha ~,
    \end{equation*}
    which is the geodesic equation. Notice that this happens only at a point in which the test particle and the freely falling observer trajectories cross. However, it is a frame-independent result and it can be generalised by saying that test particles follows geodesic, called world-lines as well. Furthermore, if we consider the inertial observer a test particle at rest, even it itself follows a geodesic. In a different frame, $\Gamma^\alpha_{\mu\nu} \neq 0$ and we can consider the metric $g$ as the gravitational interactions potential
    \begin{equation*}
        \Gamma^\alpha_{\mu\nu} \sim g_{\mu\nu,\beta} ~.
    \end{equation*}

    The same reasoning can be applied to massless light. Infact the modulus of the parallely transported $4$-velocity $g(u, u)$ is conserved along a geodesic, since 
    \begin{equation*}
        0 = \nabla_u g(u,u) = 2 u^\nu u^\mu \nabla_\nu u_\mu = 2 u^\mu (u^\nu \nabla_\nu u_\mu) ~.
    \end{equation*}
    Therefore, by the principle of general covariance, in any reference frame, the modulus
    \begin{equation*}
        g(u,u) = \begin{cases}
            - 1 & \textnormal{for massive particles} \\
            0 & \textnormal{for light} \\
        \end{cases}
    \end{equation*}
    is conserved along a geodesic. However, for the light, there is no affine parsameter identified as proper time. This means that the metric encodes information about the causal structure of spacetime, because it governs the propagation of signals, like light.

\section{Gravitational sources}

    The metric $g$ is a $4 \times 4$ symmetric matrix that contains $10$ indepenent components. Hence, we need $10$ equation to completely indentify it, at most second order partial differential equations. The most reasonable guess for a tensor which contains the second derivative of the metric is the Einstein tensor 
    \begin{equation*}
        G_{\mu\nu} = R_{\mu\nu} - \frac{1}{2} R g_{\mu\nu} ~.
    \end{equation*}
    Notice that, by the second Bianchi identity, i.e.
    \begin{equation*}
        G^\mu_{\phantom \mu \nu, \mu} = 0 ~,
    \end{equation*}
    which means that is covariantly conserved and there are only $6$ independent components. 

    The source tensor must have the same properties: it is a symmetric, covariantly conserved $(0,2)$ tensor, containing information about matter. By the weak equivalence principle, the strength of gravity interaction is measured by the proper mass, and in special relativity, mass and energy are equivalent. Hence, the reasonable guess it that the source of gravity is encoded in the energy-momentum tensor. 

    For a perfect fluid with $4$-velocity $u$, the energy-momentum tensor is 
    \begin{equation*}
        T = \rho u \otimes u + p (g^{-1} + u \otimes u) = (\rho + p) u \otimes u + p g^{-1} ~,
    \end{equation*}
    or, in components,
    \begin{equation*}
        T^{\mu\nu} = \rho u^\mu u^\nu + p (g^{\mu\nu} + u^\mu u^\nu) = (\rho + p) u^\mu u^\nu + p g^{\mu\nu} ~,
    \end{equation*}
    where $\rho$ is the proper dentity and $p$ is the proper pressure, both measured by an observer comoving with the fluid, therefore it is a true scalar. 

    Notice that the pressure part is orthogonal to $u$
    \begin{equation*}
        (g^\mu\nu + u^\mu u^\nu) u_\nu = (g^{\mu\nu} + g^{\mu\nu} \underbrace{u^\mu u_\nu }_{-1}) u_\nu = g^{\mu\nu} - g^{\mu\nu} = 0~.
    \end{equation*}
    Infact, in the comoving frame, $u = (1, 0, 0, 0)$ and 
    \begin{equation*}
        T^{\mu\nu} = \begin{bmatrix}
            \rho & 0 \\ 0 & p g^{ij} \\
        \end{bmatrix} ~,
    \end{equation*}
    or, equivalently, 
    \begin{equation*}
        T^\mu_{\phantom \mu \nu} = \begin{bmatrix}
            \rho & 0 & 0 & 0 \\
            0 & p & 0 & 0 \\
            0 & 0 & p & 0 \\
            0 & 0 & 0 & p \\
        \end{bmatrix} ~.
    \end{equation*}
    
    TO COMPLETE. 

    In a general reference frame, the continuity equation is written as 
    \begin{equation*}
        nabla_\mu T^{\mu\nu} = 0 ~,
    \end{equation*}
    which enforces the natural candidate as the source of gravity. 

\section{Einstein's field equation} 

    The Einstein's field equations are therefore
    \begin{equation*}
        G_{\mu\nu} = R_{\mu\nu} - \frac{1}{2} R g_{\mu\nu} = k T_{\mu\nu} ~,
    \end{equation*}
    where $k = 8 \pi G_N$, due to Newtonian regime.
    \begin{proof}
        In fact, $[G^{\mu\nu}] = L^{-2}$ whereas $[G^{\mu\nu}] = M L^{-3}$. This means that we require a coupling constant $[G_N] = \frac{L}{M}$. 
    \end{proof}

\section{Geodesic equation} 

    Straight lines are replaced by geodesic lines. However, the geodesic equation is a second order differential equation and we can see the connection term as a force. Hence, gravity is geometry. The Einstein's field equation are non-lineare, therefore the effects of two gravitational sources is not their sum. Furthermore, for dust $(p = 0)$, geodesic motion follows from the covariant conservation of the energy-momentum tensor. 
    \begin{proof}
        The energy-momentum tensor becomes
        \begin{equation*}
            T^{\mu\nu} = \rho u^\mu u^\nu ~.
        \end{equation*}
        Hence, 
        \begin{equation*}
            \rho u^\mu \nabla_\mu u^\nu = - u^\nu \nabla_\mu (\rho u^\mu ) \propto u^\nu 
        \end{equation*}
        and 
        \begin{equation*}
            \rho u^\mu (u_\nu \nabla_\mu u^\nu) \propto \nabla_\mu (u_\nu u^\nu) = 0 = \nabla_\mu (\rho u^\mu) ~,
        \end{equation*}
        where the affine parameter is the proper time.
    \end{proof}

    Unlike electromagnetism, we do not need a Lorentz-like force to completeley determine gravity. 

\section{Einstein-Hilbert action} 

\chapter{Linearised}



\section{Einstein's field equations in linearised regime}

    In the weak field limit, local curvature is small. This means that there exists a reference frame in which the components of the metric 
    \begin{equation*}
        g_{\mu\nu} = \eta_{\mu\nu} + \epsilon h_{\mu\nu} ~.
    \end{equation*}

    The linearised Einstein's field equations in first order in $\epsilon$ are 
    \begin{equation*}
        - \Box h_{\mu\nu} + \eta_{\mu\nu} \Box h + \partial_\mu \partial^\lambda + \partial_\nu \partial^\lambda h_{\lambda \mu} - \eta_{\mu\nu} \partial^\lambda \partial^\rho h_{\lambda \rho} - \partial_\mu \partial_\nu h = 16 \pi G_N T_{\mu\nu} ~.
    \end{equation*}
    \begin{proof}
        The Christoffel symbols in first order in $\epsilon$ is 
        \begin{equation*}
        \begin{aligned}
            \Gamma^\alpha_{\mu\nu} & = \frac{1}{2} g^{\alpha\beta} (g_{\mu\beta, \nu} + g_{\nu\beta,\mu} - g_{\mu\nu,\beta}) \\ & = \frac{1}{2} g^{\alpha\beta} (\cancel{\eta_{\mu\beta, \nu}} + \epsilon h_{\mu\beta, \nu} + \cancel{\eta_{\nu\beta,\mu}} + \epsilon h_{\nu\beta,\mu} - \cancel{\eta_{\mu\nu,\beta} }- \epsilon h_{\mu\nu,\beta} ) \\ & = \frac{\epsilon}{2} g^{\alpha\beta} (h_{\mu\beta, \nu} + h_{\nu\beta,\mu} - h_{\mu\nu,\beta}) \\ & \simeq \frac{\epsilon}{2} (\eta^{\alpha\beta} - \epsilon h^{\alpha\beta}) (h_{\mu\beta, \nu} + h_{\nu\beta,\mu} - h_{\mu\nu,\beta}) \\ & \simeq \frac{1}{2} (\epsilon \eta^{\alpha\beta} - \cancel{\epsilon^2 h^{\alpha\beta}}) (h_{\mu\beta, \nu} + h_{\nu\beta,\mu} - h_{\mu\nu,\beta}) \\ & \simeq \frac{\epsilon}{2} \eta^{\alpha\beta} (h_{\mu\beta, \nu} + h_{\nu\beta,\mu} - h_{\mu\nu,\beta}) ~,
        \end{aligned}
        \end{equation*}
        where we have Taylor expanded the inverse of the metric 
        \begin{equation*}
            g^{\alpha\beta} = (g_{\alpha\beta})^{-1} = (\eta_{\alpha\beta} + \epsilon h_{\alpha\beta})^{-1} \simeq \eta_{\alpha\beta} - \epsilon h_{\alpha\beta}  ~.
        \end{equation*}

        The Riemann tensor in first order in $\epsilon$ is 
        \begin{equation*}
        \begin{aligned}
            R^\mu_{\phantom \mu \nu\alpha\beta} & = \Gamma^\mu_{\nu\beta,\alpha} - \Gamma^\mu_{\nu\alpha,\beta} + \Gamma^\lambda_{\nu\beta} \Gamma^\mu_{\lambda\alpha} - \Gamma^\lambda_{\nu\alpha} \Gamma^\mu_{\lambda\beta} \\ & \simeq \Gamma^\mu_{\nu\beta,\alpha} - \Gamma^\mu_{\nu\alpha,\beta} \\ & = \partial_\alpha (\frac{\epsilon}{2} \eta^{\mu\lambda} (h_{\nu\lambda, \beta} + h_{\beta\lambda,\nu} - h_{\nu\beta,\lambda}) ) - \partial_\beta (\frac{\epsilon}{2} \eta^{\mu\lambda} (h_{\nu\lambda, \alpha} + h_{\alpha\lambda,\nu} - h_{\nu\alpha,\lambda})) \\ & = \frac{\epsilon}{2} \cancel{\eta^{\mu\lambda}_{\phantom{\mu\lambda} , \alpha}} (h_{\nu\lambda, \beta} + h_{\beta\lambda,\nu} - h_{\nu\beta,\lambda}) + \frac{\epsilon}{2} \eta^{\mu\lambda} (\cancel{h_{\nu\lambda, \beta\alpha}} + h_{\beta\lambda,\nu\alpha} - h_{\nu\beta,\lambda\alpha}) \\ & \qquad - \frac{\epsilon}{2} \cancel{\eta^{\mu\lambda}_{\phantom{\mu\lambda} , \beta}} (h_{\nu\lambda, \alpha} + h_{\alpha\lambda,\nu} - h_{\nu\alpha,\lambda}) - \frac{\epsilon}{2} \eta^{\mu\lambda} (\cancel{h_{\nu\lambda, \alpha\beta}} + h_{\alpha\lambda,\nu\beta} - h_{\nu\alpha,\lambda\beta}) \\ & = \frac{\epsilon}{2} \eta^{\mu\lambda} (h_{\beta\lambda,\nu\alpha} - h_{\nu\beta,\lambda\alpha}) - \frac{\epsilon}{2} \eta^{\mu\lambda} (h_{\alpha\lambda,\nu\beta} - h_{\nu\alpha,\lambda\beta}) \\ & = \frac{\epsilon}{2} \eta^{\mu\lambda} (h_{\beta\lambda,\nu\alpha} - h_{\nu\beta,\lambda\alpha} - h_{\alpha\lambda,\nu\beta} + h_{\nu\alpha,\lambda\beta}) ~.
        \end{aligned}
        \end{equation*}

        The Ricci tensor in first order in $\epsilon$ is 
        \begin{equation*}
        \begin{aligned}
            R_{\nu\beta} & = R^\mu_{\phantom \mu \nu\mu\beta} \\ & = \frac{\epsilon}{2} \eta^{\mu\lambda} ( h_{\beta\lambda,\nu\mu} - h_{\nu\beta,\lambda\mu} - h_{\mu\lambda,\nu\beta} + h_{\nu\mu,\lambda\beta}) \\ & = \frac{\epsilon}{2} \eta^{\mu\lambda} (\partial_\nu \partial_\mu h_{\beta\lambda} - \partial _\lambda \partial_\mu h_{\nu\beta} - \partial_\nu \partial_\beta h_{\mu\lambda} + \partial_\lambda \partial_\beta h_{\nu\mu}) \\ & = \frac{\epsilon}{2} (\partial_\nu \partial^\mu h_{\beta\mu} - \partial^\mu \partial_\mu h_{\nu\beta} - \partial_\nu \partial_\beta h^\mu_{\phantom \mu \mu} + \partial^\mu \partial_\beta h_{\nu\mu}) \\ & = \frac{\epsilon}{2} (  \partial^\mu \partial_\beta h_{\nu\mu} + \partial^\mu \partial_\nu h_{\beta\mu} - \partial_\nu \partial_\beta h - \Box h_{\nu\beta} ) ~,
        \end{aligned}
        \end{equation*}
        where we have set $\lambda = \mu$.

        The Ricci scalar in first order in $\epsilon$ is 
        \begin{equation*}
        \begin{aligned}
            R & = R^\nu_{\phantom \nu \nu} \\ & = \eta^{\nu \beta} R_{\nu \beta} \\ & = \frac{\epsilon}{2} \eta^{\nu \beta} ( \partial^\mu \partial_\beta h_{\nu\mu} + \partial^\mu \partial_\nu h_{\beta\mu} - \partial_\nu \partial_\beta h - \Box h_{\nu\beta}) \\ & = \frac{\epsilon}{2} ( \partial^\mu \partial^\nu h_{\nu\mu} + \partial^\mu \partial^\beta h_{\beta\mu} - \partial_\nu \partial^\nu h - \Box h^\nu_{\phantom \nu \nu}) \\ & = \frac{\epsilon}{2} (2 \partial^\mu \partial^\nu h_{\nu\mu} - 2 \Box h) \\ & = \epsilon (\partial^\mu \partial^\nu h_{\nu\mu}- \Box h) ~,
        \end{aligned}
        \end{equation*}
        where we have set $\beta = \nu$.

        Finally, the Einstein tensor in first order in $\epsilon$ is 
        \begin{equation*}
        \begin{aligned}
            G_{\mu\nu} & = R_{\mu\nu} - \frac{1}{2} R g_{\mu\nu} \\ & = \frac{\epsilon}{2} (\partial^\beta \partial_\nu h_{\mu\beta} + \partial^\beta \partial_\mu h_{\nu\beta} - \partial_\mu \partial_\nu h - \Box h_{\mu\nu} ) - \frac{1}{2} \epsilon (\partial^\alpha \partial^\beta h_{\alpha\beta } - \Box h ) g_{\mu\nu} \\ & = \frac{\epsilon}{2} ( \eta_{\mu\nu} \Box h - \Box h_{\mu\nu} - \partial_\mu \partial_\nu h - \eta_{\mu\nu} \partial^\alpha \partial^\beta h_{\alpha\beta} + \partial^\beta \partial_\nu h_{\mu\beta} + \partial^\beta \partial_\mu h_{\nu\beta} ) ~.
        \end{aligned}
        \end{equation*}

        The linearised Einstein's field equations are 
        \begin{equation*}
            G_{\mu\nu} = 8 \pi G_N T_{\mu\nu} ~,
        \end{equation*}
        \begin{equation*}
            \frac{\epsilon}{2} ( \eta_{\mu\nu} \Box h - \Box h_{\mu\nu} - \partial_\mu \partial_\nu h - \eta_{\mu\nu} \partial^\alpha \partial^\beta h_{\alpha\beta}  + \partial^\beta \partial_\nu h_{\mu\beta} + \partial^\beta \partial_\mu h_{\nu\beta} ) = 8 \pi G_N \epsilon \epsilon T_{\mu\nu} ~,
        \end{equation*}
        \begin{equation*}
            \eta_{\mu\nu} \Box h - \Box h_{\mu\nu} - \partial_\mu \partial_\nu h - \eta_{\mu\nu} \partial^\alpha \partial^\beta h_{\alpha\beta}  + \partial^\beta \partial_\nu h_{\mu\beta} + \partial^\beta \partial_\mu h_{\nu\beta} = 16 \pi G_N \epsilon T_{\mu\nu} ~,
        \end{equation*}
        where we have expanded ${T'}_{\mu\nu} = T^{(0)}_{\mu\nu} + \epsilon T_{\mu\nu}$ and noticed that $T_{\mu\nu}^{(0)}= 0$, because the Minkovski metric solves Einstein's equations. 
    \end{proof}

    The second Bianchi identity allows us to define a condition, called gauge fixing, on the $h$ beacuse there are only $6$ independent components and not $10$. Using the De Donder gauge 
    \begin{equation*}
        2 \partial^\mu h_{\mu\nu} = \partial_\nu h ~,
    \end{equation*}
    the linearised Einstein's field equations becomes
    \begin{equation*}
        - \Box h_{\mu\nu} = 16 \pi G_N (T_{\mu\nu} - \frac{1}{2} \eta_{\mu\nu} T) ~.
    \end{equation*}
    \begin{proof}
        Infact, the right side 
        \begin{equation*}
        \begin{aligned}
            G_{\mu\nu} & = \eta_{\mu\nu} \Box h - \Box h_{\mu\nu} - \partial_\mu \partial_\nu h - \eta_{\mu\nu} \partial^\alpha \partial^\beta h_{\alpha\beta}  + \partial^\beta \partial_\nu h_{\mu\beta} + \partial^\beta \partial_\mu h_{\nu\beta} \\ & = \eta_{\mu\nu} \partial^\alpha \partial_\alpha h - \partial^\alpha \partial_\alpha h_{\mu\nu} - \partial_\mu \partial_\nu h - \eta_{\mu\nu} \partial^\alpha \partial^\beta h_{\alpha\beta}  + \partial^\beta \partial_\nu h_{\mu\beta} + \partial^\beta \partial_\mu h_{\nu\beta} \\ & = \eta_{\mu\nu} \partial^\alpha \partial_\alpha h - \partial^\alpha \partial_\alpha h_{\mu\nu} - \partial_\mu \partial_\nu h - \eta_{\mu\nu} \partial^\alpha \underbrace{\partial^\beta h_{\beta\alpha}}_{\frac{1}{2} \partial_\alpha h}  + \partial_\nu \underbrace{\partial^\beta h_{\beta\mu}}_{\frac{1}{2} \partial_\mu h} +  \partial_\mu \underbrace{\partial^\beta h_{\beta\nu}}_{\frac{1}{2} \partial_\nu h} \\ & = \eta_{\mu\nu} \partial^\alpha \partial_\alpha h - \partial^\alpha \partial_\alpha h_{\mu\nu} - \partial_\mu \partial_\nu h - \frac{1}{2} \eta_{\mu\nu} \partial^\alpha \partial_\alpha h  + \frac{1}{2} \partial_\mu h + \frac{1}{2} \partial_\mu \partial_\nu h \\ & = \eta_{\mu\nu} \partial^\alpha \partial_\alpha h - \partial^\alpha \partial_\alpha h_{\mu\nu} - \cancel{\partial_\mu \partial_\nu h} - \frac{1}{2} \eta_{\mu\nu} \partial^\alpha \partial_\alpha h  + \cancel{\frac{1}{2} \partial_\nu \partial_\mu h} + \cancel{\frac{1}{2} \partial_\mu \partial_\nu h } \\ & = - \Box h_{\mu\nu} + \frac{1}{2} \eta_{\mu\nu} \Box h ~.
        \end{aligned}
        \end{equation*}

        Hence 
        \begin{equation*}
            \Box h_{\mu\nu} + \frac{1}{2} \eta_{\mu\nu} \Box h = 16 \pi G_N T_{\mu\nu} ~.
        \end{equation*} 

        Furthermore, the trace of the Einstein tensor is 
        \begin{equation*}
        \begin{aligned}
            \eta^{\mu\nu} G_{\mu\nu} & = \eta^{\mu\nu} (\eta_{\mu\nu} \Box h - \Box h_{\mu\nu} - \partial_\mu \partial_\nu h - \eta_{\mu\nu} \partial^\alpha \partial^\beta h_{\alpha\beta} + \partial^\beta \partial_\nu h_{\mu\beta} + \partial^\beta \partial_\mu h_{\nu\beta}) \\ & = \underbrace{\eta^{\mu\nu}\eta_{\mu\nu}}_4 \Box h - \Box \underbrace{\eta^{\mu\nu} h_{\mu\nu}}_h - \eta^{\mu\nu}\partial_\mu \partial_\nu h - \underbrace{\eta^{\mu\nu} \eta_{\mu\nu}}_4 \partial^\alpha \partial^\beta h_{\alpha\beta} + \eta^{\mu\nu} \partial^\beta \partial_\nu h_{\mu\beta} + \eta^{\mu\nu} \partial^\beta \partial_\mu h_{\nu\beta} \\ & = 4 \Box h - \Box h - \partial^\nu \partial_\nu h - 4 \partial^\alpha \partial^\beta h_{\alpha\beta} + \partial^\beta \partial^\mu h_{\mu\beta} + \partial^\beta \partial^\nu h_{\nu\beta} \\ & = 2 \Box h - 2 \partial^\nu \underbrace{\partial^\mu h_{\mu\nu}}_{\frac{1}{2} \partial_\nu h} \\ & = \Box h ~,
        \end{aligned}
        \end{equation*}
        which is equal to 
        \begin{equation*}
            16 \pi G_N \underbrace{\eta^{\mu\nu} T_{\mu\nu}}_T = 16 \pi G_N T ~.
        \end{equation*}

        Putting together
        \begin{equation*}
            - \Box h_{\mu\nu} + \frac{1}{2} \eta_{\mu\nu} \Box h = 16 \pi G_N T_{\mu\nu} ~,
        \end{equation*}
        \begin{equation*}
            - \Box h_{\mu\nu} + \frac{1}{2} \eta_{\mu\nu} (16 \pi G_N T) = 16 \pi G_N T_{\mu\nu} ~,
        \end{equation*}
        \begin{equation*}
            - \Box h_{\mu\nu} = 16 \pi G_N \Big ( T_{\mu\nu} - \frac{1}{2} \eta_{\mu\nu} T) ~.
        \end{equation*}
    \end{proof}

    Moreoverm, in the Newtonian regime where the matter source is static 
    \begin{equation*}
        T^{\mu\nu} \simeq T_{00} = - T ~,
    \end{equation*}
    we recover the Poisson equation
    \begin{equation*}
        \nabla^2 V_N = 4 \pi G_N \rho
    \end{equation*}
    where $h_{00} = - 2 V_N$ and $G_N$ is indeed the Newton constant.
