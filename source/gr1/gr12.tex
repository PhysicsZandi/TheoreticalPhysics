\chapter{Linearised}

\section{Diffeomorphisms}

    Consider a point $P$ in two overlapping charts $x^\mu$ and $y^\mu$, a change of coordinates is a diffeomorphism if it reduces smoothly to the identity 
    \begin{equation}\label{diffeo}
        y^\mu (P) = x^\mu (P) + \epsilon \xi^\mu (P) ~.
    \end{equation}
    where $\epsilon$ is a parameter such that when $\epsilon = 0$ the diffeomorphism reduces to $y^\mu (P) = x^\mu (P)$. 

    If we see $\xi^\mu$ as the components of a smooth vector field in the basis $\pdv{}{x^\mu}$, i.e. 
    \begin{equation*}
        \xi = \xi^\mu \pdv{}{x^\mu} ~,
    \end{equation*}
    the bases $\pdv{}{x^\mu}$ and $\pdv{}{y^\mu}$ are related by 
    \begin{equation*}
        \pdv{}{x^\mu} = \underbrace{\pdv{y^\nu}{x^\mu}}_{\delta^\nu_{\phantom \nu \mu} + \epsilon \pdv{\xi^\nu}{x^\mu}} \pdv{}{y^\nu} = \Big (\delta^\nu_{\phantom \nu \mu} + \epsilon \pdv{\xi^\nu}{x^\mu}) \pdv{}{y^\mu}
    \end{equation*}


\section{Einstein's field equations in linearised regime}

    In the weak field limit, local curvature is small. This means that there exists a reference frame in which the components of the metric 
    \begin{equation*}
        g_{\mu\nu} = \eta_{\mu\nu} + \epsilon h_{\mu\nu} ~.
    \end{equation*}

    The linearised Einstein's field equations in first order in $\epsilon$ are 
    \begin{equation*}
        - \Box h_{\mu\nu} + \eta_{\mu\nu} \Box h + \partial_\mu \partial^\lambda + \partial_\nu \partial^\lambda h_{\lambda \mu} - \eta_{\mu\nu} \partial^\lambda \partial^\rho h_{\lambda \rho} - \partial_\mu \partial_\nu h = 16 \pi G_N T_{\mu\nu} ~.
    \end{equation*}
    \begin{proof}
        The Christoffel symbols in first order in $\epsilon$ is 
        \begin{equation*}
        \begin{aligned}
            \Gamma^\alpha_{\mu\nu} & = \frac{1}{2} g^{\alpha\beta} (g_{\mu\beta, \nu} + g_{\nu\beta,\mu} - g_{\mu\nu,\beta}) \\ & = \frac{1}{2} g^{\alpha\beta} (\cancel{\eta_{\mu\beta, \nu}} + \epsilon h_{\mu\beta, \nu} + \cancel{\eta_{\nu\beta,\mu}} + \epsilon h_{\nu\beta,\mu} - \cancel{\eta_{\mu\nu,\beta} }- \epsilon h_{\mu\nu,\beta} ) \\ & = \frac{\epsilon}{2} g^{\alpha\beta} (h_{\mu\beta, \nu} + h_{\nu\beta,\mu} - h_{\mu\nu,\beta}) \\ & \simeq \frac{\epsilon}{2} (\eta^{\alpha\beta} - \epsilon h^{\alpha\beta}) (h_{\mu\beta, \nu} + h_{\nu\beta,\mu} - h_{\mu\nu,\beta}) \\ & \simeq \frac{1}{2} (\epsilon \eta^{\alpha\beta} - \cancel{\epsilon^2 h^{\alpha\beta}}) (h_{\mu\beta, \nu} + h_{\nu\beta,\mu} - h_{\mu\nu,\beta}) \\ & \simeq \frac{\epsilon}{2} \eta^{\alpha\beta} (h_{\mu\beta, \nu} + h_{\nu\beta,\mu} - h_{\mu\nu,\beta}) ~,
        \end{aligned}
        \end{equation*}
        where we have Taylor expanded the inverse of the metric 
        \begin{equation*}
            g^{\alpha\beta} = (g_{\alpha\beta})^{-1} = (\eta_{\alpha\beta} + \epsilon h_{\alpha\beta})^{-1} \simeq \eta_{\alpha\beta} - \epsilon h_{\alpha\beta}  ~.
        \end{equation*}

        The Riemann tensor in first order in $\epsilon$ is 
        \begin{equation*}
        \begin{aligned}
            R^\mu_{\phantom \mu \nu\alpha\beta} & = \Gamma^\mu_{\nu\beta,\alpha} - \Gamma^\mu_{\nu\alpha,\beta} + \Gamma^\lambda_{\nu\beta} \Gamma^\mu_{\lambda\alpha} - \Gamma^\lambda_{\nu\alpha} \Gamma^\mu_{\lambda\beta} \\ & \simeq \Gamma^\mu_{\nu\beta,\alpha} - \Gamma^\mu_{\nu\alpha,\beta} \\ & = \partial_\alpha (\frac{\epsilon}{2} \eta^{\mu\lambda} (h_{\nu\lambda, \beta} + h_{\beta\lambda,\nu} - h_{\nu\beta,\lambda}) ) - \partial_\beta (\frac{\epsilon}{2} \eta^{\mu\lambda} (h_{\nu\lambda, \alpha} + h_{\alpha\lambda,\nu} - h_{\nu\alpha,\lambda})) \\ & = \frac{\epsilon}{2} \cancel{\eta^{\mu\lambda}_{\phantom{\mu\lambda} , \alpha}} (h_{\nu\lambda, \beta} + h_{\beta\lambda,\nu} - h_{\nu\beta,\lambda}) + \frac{\epsilon}{2} \eta^{\mu\lambda} (\cancel{h_{\nu\lambda, \beta\alpha}} + h_{\beta\lambda,\nu\alpha} - h_{\nu\beta,\lambda\alpha}) \\ & \qquad - \frac{\epsilon}{2} \cancel{\eta^{\mu\lambda}_{\phantom{\mu\lambda} , \beta}} (h_{\nu\lambda, \alpha} + h_{\alpha\lambda,\nu} - h_{\nu\alpha,\lambda}) - \frac{\epsilon}{2} \eta^{\mu\lambda} (\cancel{h_{\nu\lambda, \alpha\beta}} + h_{\alpha\lambda,\nu\beta} - h_{\nu\alpha,\lambda\beta}) \\ & = \frac{\epsilon}{2} \eta^{\mu\lambda} (h_{\beta\lambda,\nu\alpha} - h_{\nu\beta,\lambda\alpha}) - \frac{\epsilon}{2} \eta^{\mu\lambda} (h_{\alpha\lambda,\nu\beta} - h_{\nu\alpha,\lambda\beta}) \\ & = \frac{\epsilon}{2} \eta^{\mu\lambda} (h_{\beta\lambda,\nu\alpha} - h_{\nu\beta,\lambda\alpha} - h_{\alpha\lambda,\nu\beta} + h_{\nu\alpha,\lambda\beta}) ~.
        \end{aligned}
        \end{equation*}

        The Ricci tensor in first order in $\epsilon$ is 
        \begin{equation*}
        \begin{aligned}
            R_{\nu\beta} & = R^\mu_{\phantom \mu \nu\mu\beta} \\ & = \frac{\epsilon}{2} \eta^{\mu\lambda} ( h_{\beta\lambda,\nu\mu} - h_{\nu\beta,\lambda\mu} - h_{\mu\lambda,\nu\beta} + h_{\nu\mu,\lambda\beta}) \\ & = \frac{\epsilon}{2} \eta^{\mu\lambda} (\partial_\nu \partial_\mu h_{\beta\lambda} - \partial _\lambda \partial_\mu h_{\nu\beta} - \partial_\nu \partial_\beta h_{\mu\lambda} + \partial_\lambda \partial_\beta h_{\nu\mu}) \\ & = \frac{\epsilon}{2} (\partial_\nu \partial^\mu h_{\beta\mu} - \partial^\mu \partial_\mu h_{\nu\beta} - \partial_\nu \partial_\beta h^\mu_{\phantom \mu \mu} + \partial^\mu \partial_\beta h_{\nu\mu}) \\ & = \frac{\epsilon}{2} (  \partial^\mu \partial_\beta h_{\nu\mu} + \partial^\mu \partial_\nu h_{\beta\mu} - \partial_\nu \partial_\beta h - \Box h_{\nu\beta} ) ~,
        \end{aligned}
        \end{equation*}
        where we have set $\lambda = \mu$.

        The Ricci scalar in first order in $\epsilon$ is 
        \begin{equation*}
        \begin{aligned}
            R & = R^\nu_{\phantom \nu \nu} \\ & = \eta^{\nu \beta} R_{\nu \beta} \\ & = \frac{\epsilon}{2} \eta^{\nu \beta} ( \partial^\mu \partial_\beta h_{\nu\mu} + \partial^\mu \partial_\nu h_{\beta\mu} - \partial_\nu \partial_\beta h - \Box h_{\nu\beta}) \\ & = \frac{\epsilon}{2} ( \partial^\mu \partial^\nu h_{\nu\mu} + \partial^\mu \partial^\beta h_{\beta\mu} - \partial_\nu \partial^\nu h - \Box h^\nu_{\phantom \nu \nu}) \\ & = \frac{\epsilon}{2} (2 \partial^\mu \partial^\nu h_{\nu\mu} - 2 \Box h) \\ & = \epsilon (\partial^\mu \partial^\nu h_{\nu\mu}- \Box h) ~,
        \end{aligned}
        \end{equation*}
        where we have set $\beta = \nu$.

        Finally, the Einstein tensor in first order in $\epsilon$ is 
        \begin{equation*}
        \begin{aligned}
            G_{\mu\nu} & = R_{\mu\nu} - \frac{1}{2} R g_{\mu\nu} \\ & = \frac{\epsilon}{2} (\partial^\beta \partial_\nu h_{\mu\beta} + \partial^\beta \partial_\mu h_{\nu\beta} - \partial_\mu \partial_\nu h - \Box h_{\mu\nu} ) - \frac{1}{2} \epsilon (\partial^\alpha \partial^\beta h_{\alpha\beta } - \Box h ) g_{\mu\nu} \\ & = \frac{\epsilon}{2} ( \eta_{\mu\nu} \Box h - \Box h_{\mu\nu} - \partial_\mu \partial_\nu h - \eta_{\mu\nu} \partial^\alpha \partial^\beta h_{\alpha\beta} + \partial^\beta \partial_\nu h_{\mu\beta} + \partial^\beta \partial_\mu h_{\nu\beta} ) ~.
        \end{aligned}
        \end{equation*}

        The linearised Einstein's field equations are 
        \begin{equation*}
            G_{\mu\nu} = 8 \pi G_N T_{\mu\nu} ~,
        \end{equation*}
        \begin{equation*}
            \frac{\epsilon}{2} ( \eta_{\mu\nu} \Box h - \Box h_{\mu\nu} - \partial_\mu \partial_\nu h - \eta_{\mu\nu} \partial^\alpha \partial^\beta h_{\alpha\beta}  + \partial^\beta \partial_\nu h_{\mu\beta} + \partial^\beta \partial_\mu h_{\nu\beta} ) = 8 \pi G_N \epsilon \epsilon T_{\mu\nu} ~,
        \end{equation*}
        \begin{equation*}
            \eta_{\mu\nu} \Box h - \Box h_{\mu\nu} - \partial_\mu \partial_\nu h - \eta_{\mu\nu} \partial^\alpha \partial^\beta h_{\alpha\beta}  + \partial^\beta \partial_\nu h_{\mu\beta} + \partial^\beta \partial_\mu h_{\nu\beta} = 16 \pi G_N \epsilon T_{\mu\nu} ~,
        \end{equation*}
        where we have expanded ${T'}_{\mu\nu} = T^{(0)}_{\mu\nu} + \epsilon T_{\mu\nu}$ and noticed that $T_{\mu\nu}^{(0)}= 0$, because the Minkovski metric solves Einstein's equations. 
    \end{proof}

    The second Bianchi identity allows us to define a condition, called gauge fixing, on the $h$ beacuse there are only $6$ independent components and not $10$. Using the De Donder gauge 
    \begin{equation*}
        2 \partial^\mu h_{\mu\nu} = \partial_\nu h ~,
    \end{equation*}
    the linearised Einstein's field equations becomes
    \begin{equation*}
        - \Box h_{\mu\nu} = 16 \pi G_N (T_{\mu\nu} - \frac{1}{2} \eta_{\mu\nu} T) ~.
    \end{equation*}
    \begin{proof}
        Infact, the right side 
        \begin{equation*}
        \begin{aligned}
            G_{\mu\nu} & = \eta_{\mu\nu} \Box h - \Box h_{\mu\nu} - \partial_\mu \partial_\nu h - \eta_{\mu\nu} \partial^\alpha \partial^\beta h_{\alpha\beta}  + \partial^\beta \partial_\nu h_{\mu\beta} + \partial^\beta \partial_\mu h_{\nu\beta} \\ & = \eta_{\mu\nu} \partial^\alpha \partial_\alpha h - \partial^\alpha \partial_\alpha h_{\mu\nu} - \partial_\mu \partial_\nu h - \eta_{\mu\nu} \partial^\alpha \partial^\beta h_{\alpha\beta}  + \partial^\beta \partial_\nu h_{\mu\beta} + \partial^\beta \partial_\mu h_{\nu\beta} \\ & = \eta_{\mu\nu} \partial^\alpha \partial_\alpha h - \partial^\alpha \partial_\alpha h_{\mu\nu} - \partial_\mu \partial_\nu h - \eta_{\mu\nu} \partial^\alpha \underbrace{\partial^\beta h_{\beta\alpha}}_{\frac{1}{2} \partial_\alpha h}  + \partial_\nu \underbrace{\partial^\beta h_{\beta\mu}}_{\frac{1}{2} \partial_\mu h} +  \partial_\mu \underbrace{\partial^\beta h_{\beta\nu}}_{\frac{1}{2} \partial_\nu h} \\ & = \eta_{\mu\nu} \partial^\alpha \partial_\alpha h - \partial^\alpha \partial_\alpha h_{\mu\nu} - \partial_\mu \partial_\nu h - \frac{1}{2} \eta_{\mu\nu} \partial^\alpha \partial_\alpha h  + \frac{1}{2} \partial_\mu h + \frac{1}{2} \partial_\mu \partial_\nu h \\ & = \eta_{\mu\nu} \partial^\alpha \partial_\alpha h - \partial^\alpha \partial_\alpha h_{\mu\nu} - \cancel{\partial_\mu \partial_\nu h} - \frac{1}{2} \eta_{\mu\nu} \partial^\alpha \partial_\alpha h  + \cancel{\frac{1}{2} \partial_\nu \partial_\mu h} + \cancel{\frac{1}{2} \partial_\mu \partial_\nu h } \\ & = - \Box h_{\mu\nu} + \frac{1}{2} \eta_{\mu\nu} \Box h ~.
        \end{aligned}
        \end{equation*}

        Hence 
        \begin{equation*}
            \Box h_{\mu\nu} + \frac{1}{2} \eta_{\mu\nu} \Box h = 16 \pi G_N T_{\mu\nu} ~.
        \end{equation*} 

        Furthermore, the trace of the Einstein tensor is 
        \begin{equation*}
        \begin{aligned}
            \eta^{\mu\nu} G_{\mu\nu} & = \eta^{\mu\nu} (\eta_{\mu\nu} \Box h - \Box h_{\mu\nu} - \partial_\mu \partial_\nu h - \eta_{\mu\nu} \partial^\alpha \partial^\beta h_{\alpha\beta} + \partial^\beta \partial_\nu h_{\mu\beta} + \partial^\beta \partial_\mu h_{\nu\beta}) \\ & = \underbrace{\eta^{\mu\nu}\eta_{\mu\nu}}_4 \Box h - \Box \underbrace{\eta^{\mu\nu} h_{\mu\nu}}_h - \eta^{\mu\nu}\partial_\mu \partial_\nu h - \underbrace{\eta^{\mu\nu} \eta_{\mu\nu}}_4 \partial^\alpha \partial^\beta h_{\alpha\beta} + \eta^{\mu\nu} \partial^\beta \partial_\nu h_{\mu\beta} + \eta^{\mu\nu} \partial^\beta \partial_\mu h_{\nu\beta} \\ & = 4 \Box h - \Box h - \partial^\nu \partial_\nu h - 4 \partial^\alpha \partial^\beta h_{\alpha\beta} + \partial^\beta \partial^\mu h_{\mu\beta} + \partial^\beta \partial^\nu h_{\nu\beta} \\ & = 2 \Box h - 2 \partial^\nu \underbrace{\partial^\mu h_{\mu\nu}}_{\frac{1}{2} \partial_\nu h} \\ & = \Box h ~,
        \end{aligned}
        \end{equation*}
        which is equal to 
        \begin{equation*}
            16 \pi G_N \underbrace{\eta^{\mu\nu} T_{\mu\nu}}_T = 16 \pi G_N T ~.
        \end{equation*}

        Putting together
        \begin{equation*}
            - \Box h_{\mu\nu} + \frac{1}{2} \eta_{\mu\nu} \Box h = 16 \pi G_N T_{\mu\nu} ~,
        \end{equation*}
        \begin{equation*}
            - \Box h_{\mu\nu} + \frac{1}{2} \eta_{\mu\nu} (16 \pi G_N T) = 16 \pi G_N T_{\mu\nu} ~,
        \end{equation*}
        \begin{equation*}
            - \Box h_{\mu\nu} = 16 \pi G_N \Big ( T_{\mu\nu} - \frac{1}{2} \eta_{\mu\nu} T) ~.
        \end{equation*}
    \end{proof}

    Moreover, in the Newtonian regime where the matter source is static 
    \begin{equation*}
        T^{\mu\nu} \simeq T_{00} = - T ~,
    \end{equation*}
    we recover the Poisson equation
    \begin{equation*}
        \nabla^2 V_N = 4 \pi G_N \rho
    \end{equation*}
    where $h_{00} = - 2 V_N$ and $G_N$ is indeed the Newton constant.

\chapter{Gravitational waves}

\section{Gauge invariance}

    Notice that the linearised Einstein's field equations are invariant under a diffeomorphism~\eqref{diffeo}, since a variation of the metric 
    \begin{equation*}
        {h'}_{\mu\nu} = h_{\mu\nu} - (\xi_{\mu,\nu} + \xi_{\nu,\mu}) ~,
    \end{equation*}
    leaves unchanged the equations of motion~\eqref{lineq}.
    \begin{proof}
        Maybe in the future.
    \end{proof}

    This frees four degrees of freedom to fullfil. We choose the De Donder gauge 
    \begin{equation*}
        2 \partial^\mu {h'}_{\mu\nu} - \partial_\nu h' = 0 ~,
    \end{equation*}
    which can be used to determine explicitly the equation which $\xi$ must satisfy in order to have the De Donder gauge 
    \begin{equation*}
        2 \Box \xi_\nu = 2 \partial^\mu h_{\mu\nu} ~.
    \end{equation*}
    \begin{proof}
        Maybe in the future.
    \end{proof}

    In particular, we can define a new metric, called the transverse tensor
    \begin{equation*}
        h^T_{\mu\nu} = {h'}_{\mu\nu} - \frac{1}{2} \eta_{\mu\nu} h' ~,
    \end{equation*}
    which in the De Donder gauge satsfies the trasversality condition 
    \begin{equation*}
        \partial^\mu h^T_{\mu\nu} = 0 ~.
    \end{equation*}
    \begin{proof}
        Maybe in the future.
    \end{proof}

    Moreover, this tensor is called the trace-inverse 
    \begin{equation*}
        h^T = - h' = - h ~.
    \end{equation*}
    \begin{proof}
        Maybe in the future.
    \end{proof}

    Finally, we obtain the inhomogeneous linearised Einstein's field equation in the De Donder gauge
    \begin{equation*}
        - \Box h^T_{\mu\nu} = 16 \pi G_N T_{\mu\nu} ~.
    \end{equation*}
    \begin{proof}
        Maybe in the future.
    \end{proof}

\section{Vacuum gauge}

    In the vacuum, where $T^{\mu\nu} = 0$, we obtain the homogeneous linearised Einstein's field equation in the De Donder gauge
    \begin{equation*}
        - \Box h^T_{\mu\nu} = 0 ~.
    \end{equation*}
    Furthermore, notice that the De Donder gauge does not uniquely determine the choice of coordinates. In fact, choosing 
    \begin{equation*}
        \Box {\xi'}^\nu = 0 ~,
    \end{equation*}
    we can see that by a change of metric 
    \begin{equation*}
        {h''} = {h'}_{\mu\nu} - ({\xi'}_{\mu,\nu} + {\xi}_{\nu,\mu}) ~,
    \end{equation*}
    we see that we have a residual gauge, since it does not transform the equations of motion.
    \begin{proof}
        Maybe in the future.
    \end{proof}

    Choosing a purely coordinate perturbation, in which $h_{\mu\nu}$,
    \begin{equation*}
        {h'}_{\mu\nu} = \xi_{\mu,\nu} + \xi_{\nu,\mu} ~,
    \end{equation*}
    with the physical meaning that there is no perturbation of the metric but only a change of coordinates, we obtain the same condition of the De Donder gauge 
    \begin{equation*}
        \Box \xi_\nu = 0~.
    \end{equation*} 