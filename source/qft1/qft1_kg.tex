\part{Klein-Gordon theory}

\chapter{Canonical or second quantisation}

    In Schoedinger picture, where states evolve in time while operators do not, recall that standard quantisation from classical mechanics to quantum mechanics works in this way: 
    \begin{enumerate}
        \item hamiltonian formalism $H \mapsto$ hamiltonian operator $\hat H$~,
        \item generalised coordinates and conjugate momenta $(q_i, p^i = \pdv{L}{\dot q_i}) \mapsto$ operators on a Hilbert space $\hat q_i$ and $\hat p^i$~,
        \item Poissons brackets $\{q_i, p^j\} = \delta_i^{\phantom i j}$ and $\{p^i, p^j\} = \{q_i, q_j\} = 0 \mapsto$ commutators $[q_i, p^j] = i \delta_i^{\phantom i j}$ and $[p^i, p^j] = [q_i, q_j] = 0$~.
    \end{enumerate}

    Similarly, the second quantisation from classical field theory to quantum field theory works in this way:
    \begin{enumerate}
        \item fields and conjugate fields $(\varphi_i(t, \mathbf x), \pi^i (t, \mathbf x) = \pdv{\mathcal L}{\dot \varphi_i}) \mapsto$ operators on a Fock space $\hat \varphi_i(t, \mathbf x)$ and $\hat \pi^i (t, \mathbf x)$~,
        \item canonical commutation relations $[\hat \varphi_i(t, \mathbf x), \hat \pi^j (t, \mathbf y)] = i \delta_i^{\phantom i j} \delta^3(\mathbf x - \mathbf y)$ and $[\hat \varphi_i(t, \mathbf x), \hat \varphi_j(t, \mathbf y)] = [\hat \pi^i (t, \mathbf x), \hat \pi^j (t, \mathbf y)] = 0$~.
    \end{enumerate}

    States which live in the Fock state $\ket{\psi}$ evolve in time via the Schoedinger equation 
    \begin{equation*}
        i \pdv{}{t} \ket{\psi} = \hat H \ket{\psi} 
    \end{equation*}
    where $\ket{\psi}$ is a wave functional such that its modulus square gives the density probability to find the field in a certain configuration and $\hat H (\varphi_i(t, \mathbf x), \pi^i (t, \mathbf x))$ is an operator, since $\varphi_i(t, \mathbf x)$ and $\pi^i (t, \mathbf x)$ are.

    In order to solve the theory, we need to find the eigenstates of $\hat H$, but it is too difficult expect in the case of a free theory, which the lagrangian is quadratic and the equations od motion are linear and solvable.

\section{Harmonic oscillator}

    Recall some feature of the harmonic oscillator.

\section{Dirac delta}

    Recall that the integral representation of the Dirac delta is 
    \begin{equation}\label{deltaint}
        \delta^3 (\mathbf x - \mathbf y) = \int \frac{d^3 p}{(2\pi)^3} \exp(i \mathbf p \cdot (\mathbf x - \mathbf y)) = \int \frac{d^3 p}{(2\pi)^3} \exp(- i \mathbf p \cdot (\mathbf x - \mathbf y)) ~.
    \end{equation}

\chapter{Single real Klein-Gordon field}

\section{Hamiltonian}

    The simplest relativistic field theory is the Klein-Gordon theory of a single real scalar field chargeless and spinless. Its lagrangian is 
    \begin{equation*}
        \mathcal L = \frac{1}{2} \partial_\mu \varphi \partial^\mu \varphi - \frac{1}{2} m^2 \varphi^2 
    \end{equation*}
    and its equations of motion are 
    \begin{equation}\label{kgeq}
        (\Box + m^2) \varphi(x) = 0 ~.
    \end{equation}
    \begin{proof}
        Infact, using~\eqref{eleq} 
        \begin{equation*}
        \begin{aligned}
            0 & = \pdv{\mathcal L}{\varphi} - \partial_\mu \pdv{\mathcal L}{\partial_\mu \varphi} \\ & = \pdv{}{\varphi} \Big ( \cancel{\frac{1}{2} \partial_\mu \varphi \partial^\mu \varphi} - \underbrace{\frac{1}{2} m^2 \varphi^2}_{m^2 \varphi} \Big) + \partial_\mu \pdv{}{\partial_\mu \varphi} \Big ( \underbrace{\frac{1}{2} \partial_\mu \varphi \partial^\mu \varphi}_{\partial_\mu \partial^\mu \varphi} - \cancel{\frac{1}{2} m^2 \varphi^2} \Big) \\ & = \underbrace{\partial_\mu \partial^\mu}_\Box \varphi + m^2 \varphi \\ & = (\Box + m^2) \varphi ~.
        \end{aligned}
        \end{equation*}
    \end{proof}

    It is a system of infinitely many degrees of freedom and to decouple them we need to perform a Fourier transform 
    \begin{equation}\label{fourkg}
        \varphi (t, \mathbf x) = \int \frac{d^3 p}{(2\pi)^3} \exp(i \mathbf p \cdot \mathbf x) \tilde \varphi(t, \mathbf p) ~,
    \end{equation}
    which in momentum space becomes 
    \begin{equation*}
        \Big ( \pdvdu{}{t} + |\mathbf p|^2 + m^2 \Big) \tilde \varphi(t, \mathbf x) = 0
    \end{equation*}
    and its solution is an harmonic oscillator for each $\mathbf p$ of frequency 
    \begin{equation}\label{kgenergy}
        \omega_{\mathbf p} = \sqrt{|\mathbf p|^2 + m^2}~.
    \end{equation}
    Hence, the most general solution of the Klein-Gordon equation~\eqref{kgeq} is a superposition of simple harmonic oscillators, each vibrating with different frequency and amplitude. To quantise the theory and $\varphi$, we need to quantise this set of infinitely decoupled harmonic oscillators.
    \begin{proof}
        We decompose~\eqref{kgeq} into time and space components
        \begin{equation*}
            0 = (\Box + m^2) \varphi = (\underbrace{\partial_0}_{\partial^0} \partial^0 + \underbrace{\partial_i}_{-\partial^i} \partial^i + m^2) \varphi = ((\partial^0)^2 - (\partial^i)^2 + m^2) \varphi = (\pdvdu{}{t} - \nabla^2 + m^2) \varphi ~,
        \end{equation*}
        and we substitute~\eqref{fourkg}
        \begin{equation*}
        \begin{aligned}
            0 & = (\pdvdu{}{t} - \nabla^2 + m^2) \int \frac{d^3 p}{(2\pi)^3} \exp(i \mathbf p \cdot \mathbf x) \tilde \varphi(t, \mathbf p) \\ & = \int \frac{d^3 p}{(2\pi)^3} (\pdvdu{}{t} - \underbrace{\nabla^2}_{- i^2 |\mathbf p|^2} + m^2) (\exp(i \mathbf p \cdot \mathbf x) \tilde \varphi(t, \mathbf p)) \\ & = \int \frac{d^3 p}{(2\pi)^3} (\pdvdu{}{t} - i^2 |\mathbf p|^2 + m^2) \exp(i \mathbf p \cdot \mathbf x) \tilde \varphi(t, \mathbf p) \\ & = \int \frac{d^3 p}{(2\pi)^3} (\pdvdu{}{t} + |\mathbf p|^2 + m^2) \exp(i \mathbf p \cdot \mathbf x) \tilde \varphi(t, \mathbf p) ~,
        \end{aligned}
        \end{equation*}
        where the integrand vanishes with the exponential. Finally, we define the energy~\eqref{kgenergy} and we obtain 
        \begin{equation*}
            (\pdvdu{}{t} + \omega_{\mathbf p})^2 \tilde \varphi(t, \mathbf p) = 0 ~,
        \end{equation*} 
        which is indeed the equation of an harmonic oscillator in the form $\ddot x + \omega^2 x = 0$.
    \end{proof}

    By analogy with the simple quantum harmonic oscillator, we define the field operator 
    \begin{equation}\label{kgfop}
        \hat \varphi (\mathbf x) = \int \frac{d^3 p}{{(2\pi)}^3} \frac{1}{\sqrt{2 \omega_{\mathbf p}}} \Big (\hat a_{\mathbf p} \exp(i \mathbf p \cdot \mathbf x) + \hat a_{\mathbf p}^\dagger \exp(- i \mathbf p \cdot \mathbf x) \Big)
    \end{equation}
    and the conjugate operator
    \begin{equation}\label{kgpop}
        \hat \pi (\mathbf x) = \int \frac{d^3 p}{{(2\pi)}^3} \Big (- i\sqrt{\frac{\omega_{\mathbf p}}{2}} \Big ) \Big (\hat a_{\mathbf p} \exp(i \mathbf p \cdot \mathbf x) - \hat a_{\mathbf p}^\dagger \exp(- i \mathbf p \cdot \mathbf x) \Big) ~,
    \end{equation}
    such that they satisfies the commutation relations for annihilation and creation operators
    \begin{equation}\label{anncrea}
        [\hat a_{\mathbf p}, \hat a_{\mathbf q}] = [\hat a_{\mathbf p}^\dagger, \hat a_{\mathbf q}^\dagger] = 0 ~, \quad [\hat a_{\mathbf p}, \hat a_{\mathbf q}^\dagger] = (2\pi)^3 \delta^3 (\mathbf p - \mathbf q) ~.
    \end{equation}
    Therefore, the canonical commutation relations become 
    \begin{equation*}
        [\hat \varphi(\mathbf x), \hat \varphi (\mathbf y)] = [\hat \pi(\mathbf x), \hat \pi (\mathbf y)]  = 0
    \end{equation*}
    and 
    \begin{equation*}
        [\hat \varphi(\mathbf x), \hat \pi (\mathbf y)] = i \delta^3 (\mathbf x - \mathbf y) ~.
    \end{equation*}
    \begin{proof}
        For the field-field commutator, using~\eqref{anncrea},~\eqref{kgfop} and~\eqref{deltaint}
        \begin{equation*}
        \begin{aligned}
            [\hat \varphi(\mathbf x), \hat \varphi (\mathbf y)] & = [\int \frac{d^3 p}{{(2\pi)}^3} \frac{1}{\sqrt{2 \omega_{\mathbf p}}} \Big (\hat a_{\mathbf p} \exp(i \mathbf p \cdot \mathbf x) + \hat a_{\mathbf p}^\dagger \exp(- i \mathbf p \cdot \mathbf x) \Big), \\ & \qquad \int \frac{d^3 q}{{(2\pi)}^3} \frac{1}{\sqrt{2 \omega_{\mathbf q}}} \Big (\hat a_{\mathbf q} \exp(i \mathbf q \cdot \mathbf y) + \hat a_{\mathbf q}^\dagger \exp(- i \mathbf q \cdot \mathbf y) \Big)] \\ &  = \int \frac{d^3 p ~ d^3 q}{{(2\pi)}^6} \frac{1}{2 \sqrt{\omega_{\mathbf p}} \omega_{\mathbf q}} [\hat a_{\mathbf p} \exp(i \mathbf p \cdot \mathbf x) + \hat a_{\mathbf p}^\dagger \exp(- i \mathbf p \cdot \mathbf x), \\ & \qquad \hat a_{\mathbf q} \exp(i \mathbf q \cdot \mathbf y) + \hat a_{\mathbf q}^\dagger \exp(- i \mathbf q \cdot \mathbf y)] \\ & = \int \frac{d^3 p ~ d^3 q}{{(2\pi)}^6} \frac{1}{2 \sqrt{\omega_{\mathbf p}} \omega_{\mathbf q}} \Big ( \underbrace{[\hat a_{\mathbf p}, \hat a_{\mathbf q}]}_0 \exp(i (\mathbf p \cdot \mathbf x + \mathbf q \cdot \mathbf y)) + \underbrace{[\hat a_{\mathbf p}, \hat a_{\mathbf q}^\dagger]}_{(2\pi)^3 \delta^3 (\mathbf p - \mathbf q)} \exp(i (\mathbf p \cdot \mathbf x - \mathbf q \cdot \mathbf y)) \\ & \qquad + \underbrace{[\hat a_{\mathbf p}^\dagger, \hat a_{\mathbf q}]}_{- (2\pi)^3 \delta^3 (\mathbf q - \mathbf p)} \exp(i (- \mathbf p \cdot \mathbf x + \mathbf q \cdot \mathbf y)) + \underbrace{[\hat a_{\mathbf p}^\dagger, \hat a_{\mathbf q}^\dagger]}_0 \exp(i (- \mathbf p \cdot \mathbf x - \mathbf q \cdot \mathbf y))\Big) \\ & = \int \frac{d^3 p ~ d^3 q}{{(2\pi)}^3} \frac{1}{2 \sqrt{\omega_{\mathbf p}} \omega_{\mathbf q}} \Big ( \underbrace{\delta^3 (\mathbf p - \mathbf q) \exp(i (\mathbf p \cdot \mathbf x - \mathbf q \cdot \mathbf y))}_{\mathbf p = \mathbf q} \\ & \qquad - \underbrace{\delta^3 (\mathbf q - \mathbf p) \exp(i (- \mathbf p \cdot \mathbf x + \mathbf q \cdot \mathbf y))}_{\mathbf p = \mathbf q} \Big) \\ & = \int \frac{d^3 p}{{(2\pi)}^3} \frac{1}{2 \omega_{\mathbf p}} \Big (\underbrace{\exp(i \mathbf p \cdot (\mathbf x - \mathbf y))}_{\delta^3 (\mathbf x - \mathbf y)} - \underbrace{\exp(i \mathbf p \cdot (- \mathbf x + \mathbf y))}_{\delta^3 (\mathbf x - \mathbf y)}\Big) = 0 ~.
        \end{aligned}
        \end{equation*}

        For the conjugate-conjugate commutator, using~\eqref{anncrea},~\eqref{kgpop} and~\eqref{deltaint}
        \begin{equation*}
        \begin{aligned}
            [\hat \pi(\mathbf x), \hat \pi (\mathbf y)] & = [\int \frac{d^3 p}{{(2\pi)}^3} \Big (- i\sqrt{\frac{\omega_{\mathbf p}}{2}} \Big )  \Big (\hat a_{\mathbf p} \exp(i \mathbf p \cdot \mathbf x) - \hat a_{\mathbf p}^\dagger \exp(- i \mathbf p \cdot \mathbf x) \Big), \\ & \qquad \int \frac{d^3 q}{{(2\pi)}^3} \Big (- i \sqrt{\frac{\omega_{\mathbf q}}{2}} \Big )  \Big (\hat a_{\mathbf q} \exp(i \mathbf q \cdot \mathbf y) - \hat a_{\mathbf q}^\dagger \exp(- i \mathbf q \cdot \mathbf y) \Big)] \\ &  = \int \frac{d^3 p ~ d^3 q}{{(2\pi)}^6} \Big (- \frac{1}{2} \sqrt{\omega_{\mathbf p}\omega_{\mathbf q}} \Big ) [\hat a_{\mathbf p} \exp(i \mathbf p \cdot \mathbf x) - \hat a_{\mathbf p}^\dagger \exp(- i \mathbf p \cdot \mathbf x), \\ & \qquad \hat a_{\mathbf q} \exp(i \mathbf q \cdot \mathbf y) - \hat a_{\mathbf q}^\dagger \exp(- i \mathbf q \cdot \mathbf y)] \\ & = \int \frac{d^3 p ~ d^3 q}{{(2\pi)}^6} \Big (- \frac{1}{2} \sqrt{\omega_{\mathbf p}\omega_{\mathbf q}} \Big ) \Big (\underbrace{[\hat a_{\mathbf p}, \hat a_{\mathbf q}]}_0 \exp(i (\mathbf p \cdot \mathbf x + \mathbf q \cdot \mathbf y)) - \underbrace{[\hat a_{\mathbf p}, \hat a_{\mathbf q}^\dagger]}_{(2\pi)^3 \delta^3 (\mathbf p - \mathbf q)} \exp(i (\mathbf p \cdot \mathbf x - \mathbf q \cdot \mathbf y)) \\ & \qquad - \underbrace{[\hat a_{\mathbf p}^\dagger, \hat a_{\mathbf q}]}_{- (2\pi)^3 \delta^3 (\mathbf q - \mathbf p)} \exp(i (- \mathbf p \cdot \mathbf x + \mathbf q \cdot \mathbf y)) + \underbrace{[\hat a_{\mathbf p}^\dagger, \hat a_{\mathbf q}^\dagger]}_0 \exp(i (- \mathbf p \cdot \mathbf x - \mathbf q \cdot \mathbf y))\Big) \\ & = \int \frac{d^3 p ~ d^3 q}{{(2\pi)}^3} \Big (- \frac{1}{2} \sqrt{\omega_{\mathbf p}\omega_{\mathbf q}} \Big ) \Big ( - \underbrace{\delta^3 (\mathbf p - \mathbf q) \exp(i (\mathbf p \cdot \mathbf x - \mathbf q \cdot \mathbf y))}_{\mathbf p = \mathbf q} \\ & \qquad + \underbrace{\delta^3 (\mathbf q - \mathbf p) \exp(i (- \mathbf p \cdot \mathbf x + \mathbf q \cdot \mathbf y))}_{\mathbf p = \mathbf q} \Big) \\ & = \int \frac{d^3 p}{{(2\pi)}^3} \Big (- \frac{\omega_{\mathbf p}}{2} \Big ) \Big (-\underbrace{\exp(i \mathbf p \cdot (\mathbf x - \mathbf y))}_{\delta^3 (\mathbf x - \mathbf y)} + \underbrace{\exp(i \mathbf p \cdot (- \mathbf x + \mathbf y))}_{\delta^3 (\mathbf x - \mathbf y)}\Big) = 0 ~.
        \end{aligned}
        \end{equation*}

        For the field-conjugate commutator, using~\eqref{anncrea},~\eqref{kgfop},~\eqref{kgpop} and~\eqref{deltaint}
        \begin{equation*}
        \begin{aligned}
            [\hat \varphi(\mathbf x), \hat \pi (\mathbf y)] & = [\int \frac{d^3 p}{{(2\pi)}^3} \frac{1}{\sqrt{2 \omega_{\mathbf p}}} \Big (\hat a_{\mathbf p} \exp(i \mathbf p \cdot \mathbf x) + \hat a_{\mathbf p}^\dagger \exp(- i \mathbf p \cdot \mathbf x) \Big), \\ & \qquad \int \frac{d^3 q}{{(2\pi)}^3} \Big (- i \sqrt{\frac{\omega_{\mathbf q}}{2}} \Big )  \Big (\hat a_{\mathbf q} \exp(i \mathbf q \cdot \mathbf y) - \hat a_{\mathbf q}^\dagger \exp(- i \mathbf q \cdot \mathbf y) \Big)] \\ &  = \int \frac{d^3 p ~ d^3 q}{{(2\pi)}^6} \Big (- \frac{i}{2}\sqrt{\frac{\omega_{\mathbf q}}{\omega_{\mathbf p}}} \Big ) [\hat a_{\mathbf p} \exp(i \mathbf p \cdot \mathbf x) + \hat a_{\mathbf p}^\dagger \exp(- i \mathbf p \cdot \mathbf x), \\ & \qquad \hat a_{\mathbf q} \exp(i \mathbf q \cdot \mathbf y) - \hat a_{\mathbf q}^\dagger \exp(- i \mathbf q \cdot \mathbf y)] \\ & = \int \frac{d^3 p ~ d^3 q}{{(2\pi)}^6} \Big (- \frac{i}{2}\sqrt{\frac{\omega_{\mathbf q}}{\omega_{\mathbf p}}} \Big ) \Big ( \underbrace{[\hat a_{\mathbf p}, \hat a_{\mathbf q}]}_0 \exp(i (\mathbf p \cdot \mathbf x + \mathbf q \cdot \mathbf y)) - \underbrace{[\hat a_{\mathbf p}, \hat a_{\mathbf q}^\dagger]}_{(2\pi)^3 \delta^3 (\mathbf p - \mathbf q)} \exp(i (\mathbf p \cdot \mathbf x - \mathbf q \cdot \mathbf y)) \\ & \qquad + \underbrace{[\hat a_{\mathbf p}^\dagger, \hat a_{\mathbf q}]}_{- (2\pi)^3 \delta^3 (\mathbf q - \mathbf p)} \exp(i (- \mathbf p \cdot \mathbf x + \mathbf q \cdot \mathbf y)) - \underbrace{[\hat a_{\mathbf p}^\dagger, \hat a_{\mathbf q}^\dagger]}_0 \exp(i (- \mathbf p \cdot \mathbf x - \mathbf q \cdot \mathbf y))\Big) \\ & = \int \frac{d^3 p ~ d^3 q}{{(2\pi)}^3} \Big (- \frac{i}{2}\sqrt{\frac{\omega_{\mathbf q}}{\omega_{\mathbf p}}} \Big ) \Big ( - \underbrace{\delta^3 (\mathbf p - \mathbf q) \exp(i (\mathbf p \cdot \mathbf x - \mathbf q \cdot \mathbf y))}_{\mathbf p = \mathbf q} \\ & \qquad - \underbrace{\delta^3 (\mathbf q - \mathbf p) \exp(i (- \mathbf p \cdot \mathbf x + \mathbf q \cdot \mathbf y))}_{\mathbf p = \mathbf q} \Big) \\ & = \int \frac{d^3 p}{{(2\pi)}^3} \Big (\frac{i}{2} \Big ) \Big (\underbrace{\exp(i \mathbf p \cdot (\mathbf x - \mathbf y))}_{\delta^3 (\mathbf x - \mathbf y)} + \underbrace{\exp(i \mathbf p \cdot (- \mathbf x + \mathbf y))}_{\delta^3 (\mathbf x - \mathbf y)}\Big) \\ & = \frac{i}{2} 2 \delta^3 (\mathbf x - \mathbf y) = i \delta^3 (\mathbf x - \mathbf y) ~.
        \end{aligned}
        \end{equation*}
    \end{proof}

    The hamiltonian is 
    \begin{equation*}
        H = \frac{1}{2} \int d^3 x ~ (\pi^2 + (\boldsymbol \nabla \varphi)^2 + m^2 \varphi^2) ~.
    \end{equation*}
    If we make a function study of the classical hamiltonian, we notice that it has quadratic terms and a minimum at $\varphi_0 (t, \mathbf x) = const$ which we could consider as the ground state with $\varphi_0 = 0$. Quantising the theory means that we consider quantum (small) fluctuations $\delta \varphi$ around this ground state such that 
    \begin{equation*}
        \varphi(t, \mathbf x) = \underbrace{\varphi(t, \mathbf x)_0}_0 + \delta \varphi(t, \mathbf x) ~.
    \end{equation*} 
    The hamiltonian operator in quantum field theory becomes
    \begin{equation}\label{hamkg}
        \hat H = \int \frac{d^3 p}{(2\pi)^3} \omega_{\mathbf p} \hat a_{\mathbf p}^\dagger \hat a_{\mathbf p} + \frac{1}{2} \int d^3 p ~ \omega_{\mathbf p} \delta^3 (0) ~.
    \end{equation}
    \begin{proof}
        Infact, the conjugate field is 
        \begin{equation}\label{conjfield}
        \begin{aligned}
            \pi = \pdv{\mathcal L}{\dot \varphi} = \pdv{\mathcal L}{\partial_0 \varphi} = \partial_0 \varphi = \dot \varphi 
        \end{aligned}
        \end{equation}
        and using~\eqref{energ} and~\eqref{kglan}
        \begin{equation*}
        \begin{aligned}
            H & = \int d^3 x ~ T^{00} \\ & = \int d^3 x ~(\pi \underbrace{\dot \varphi}_\pi - \mathcal L) \\ & = \int d^3 x ~(\pi^2 - \frac{1}{2} \partial_\mu \varphi \partial^\mu \varphi + \frac{1}{2} m^2 \varphi^2) \\ & = \int d^3 x ~(\pi^2 - \frac{1}{2} \partial_0 \varphi \partial^0 \varphi - \frac{1}{2} \partial_i \varphi \partial^i \varphi + \frac{1}{2} m^2 \varphi^2) \\ & = \int d^3 x ~(\pi^2 - \frac{1}{2} \underbrace{\partial_0 \varphi \partial^0 \varphi}_{\pi^2} - \frac{1}{2} \underbrace{\partial_i \varphi \partial^i \varphi}_{- \nabla^2 \varphi} + \frac{1}{2} m^2 \varphi^2) \\ & = \frac{1}{2} \int d^3 x ~ (\pi^2 + (\boldsymbol \nabla \varphi)^2 + m^2 \varphi^2) ~.
        \end{aligned}
        \end{equation*}

        Furthermore, using~\eqref{anncrea},~\eqref{kgfop},~\eqref{kgpop} and~\eqref{deltaint}
        \begin{equation*}
        \begin{aligned}
            \hat H & = \frac{1}{2} \int d^3 x ~ \Big (\hat \pi^2 + (\boldsymbol \nabla \hat \varphi)^2 + m^2 \hat \varphi^2) \\ & = \frac{1}{2} \int d^3 x ~ (\int \frac{d^3 p}{{(2\pi)}^3} \Big (- i\sqrt{\frac{\omega_{\mathbf p}}{2}} \Big ) \Big (\hat a_{\mathbf p} \exp(i \mathbf p \cdot \mathbf x) - \hat a_{\mathbf p}^\dagger \exp(- i \mathbf p \cdot \mathbf x) \Big) \Big ) \\ & \qquad \Big (\int \frac{d^3 q}{{(2\pi)}^3} \Big (- i\sqrt{\frac{\omega_{\mathbf q}}{2}} \Big ) \Big (\hat a_{\mathbf q} \exp(i \mathbf q \cdot \mathbf x) - \hat a_{\mathbf q}^\dagger \exp(- i \mathbf q \cdot \mathbf x) \Big) \Big ) \\ & \qquad + \nabla \Big ( \int \frac{d^3 p}{{(2\pi)}^3} \frac{1}{\sqrt{2 \omega_{\mathbf p}}} \Big (\hat a_{\mathbf p} \exp(i \mathbf p \cdot \mathbf x) + \hat a_{\mathbf p}^\dagger \exp(- i \mathbf p \cdot \mathbf x) \Big) \Big) \\ & \qquad \nabla \Big ( \int \frac{d^3 q}{{(2\pi)}^3} \frac{1}{\sqrt{2 \omega_{\mathbf q}}} \Big (\hat a_{\mathbf q} \exp(i \mathbf q \cdot \mathbf x) + \hat a_{\mathbf q}^\dagger \exp(- i \mathbf q \cdot \mathbf x) \Big) \Big) \\ & \qquad + m^2 \Big (\int \frac{d^3 p}{{(2\pi)}^3} \frac{1}{\sqrt{2 \omega_{\mathbf p}}} \Big (\hat a_{\mathbf p} \exp(i \mathbf p \cdot \mathbf x) + \hat a_{\mathbf p}^\dagger \exp(- i \mathbf p \cdot \mathbf x) \Big) \Big ) \\ & \qquad \Big ( \int \frac{d^3 q}{{(2\pi)}^3} \frac{1}{\sqrt{2 \omega_{\mathbf q}}} \Big (\hat a_{\mathbf q} \exp(i \mathbf q \cdot \mathbf x) + \hat a_{\mathbf q}^\dagger \exp(- i \mathbf q \cdot \mathbf x) \Big) \Big)
        \end{aligned}
        \end{equation*}
        \begin{equation*}
        \begin{aligned}
            \phantom{\hat H} & = \frac{1}{2} \int \frac{d^3 x ~ d^3 p ~d^3 q}{(2\pi)^6} ~ \Big (\Big (- \frac{1}{2} \sqrt{\omega_{\mathbf p} \omega_{\mathbf q}} \Big ) \Big (\hat a_{\mathbf p} \hat a_{\mathbf q} \exp(i (\mathbf p + \mathbf q) \cdot \mathbf x) - \hat a_{\mathbf p} \hat a_{\mathbf q}^\dagger \exp(i (\mathbf p - \mathbf q) \cdot \mathbf x) \\ & \qquad - \hat a_{\mathbf p}^\dagger \hat a_{\mathbf q} \exp(i (- \mathbf p + \mathbf q) \cdot \mathbf x) + \hat a_{\mathbf p}^\dagger \hat a_{\mathbf q}^\dagger \exp(i (- \mathbf p - \mathbf q) \cdot \mathbf x) \Big) \\ & \qquad + \frac{1}{2 \sqrt{\omega_{\mathbf p} \omega_{\mathbf q}}} \Big (i \mathbf p \hat a_{\mathbf p} \exp(i \mathbf p \cdot \mathbf x) - i \mathbf p \hat a_{\mathbf p}^\dagger \exp(- i \mathbf p \cdot \mathbf x) \Big) \cdot \\ & \qquad \Big ( i \mathbf q \hat a_{\mathbf q} \exp(i \mathbf q \cdot \mathbf x) - i \mathbf q \hat a_{\mathbf q}^\dagger \exp(- i \mathbf q \cdot \mathbf x) \Big) \\ & \qquad + m^2 \frac{1}{2 \sqrt{\omega_{\mathbf p} \omega_{\mathbf q}}} \Big (\hat a_{\mathbf p} \hat a_{\mathbf q} \exp(i (\mathbf p + \mathbf q) \cdot \mathbf x) + \hat a_{\mathbf p} \hat a_{\mathbf q}^\dagger \exp(i (\mathbf p - \mathbf q) \cdot \mathbf x) \\ & \qquad + \hat a_{\mathbf p}^\dagger \hat a_{\mathbf q} \exp(i (- \mathbf p + \mathbf q) \cdot \mathbf x) + \hat a_{\mathbf p}^\dagger \hat a_{\mathbf q}^\dagger \exp(i (- \mathbf p - \mathbf q) \cdot \mathbf x) \Big) \Big) 
        \end{aligned}
        \end{equation*}
        \begin{equation*}
        \begin{aligned}
            \phantom{\hat H} & = \frac{1}{2} \int \frac{d^3 x ~ d^3 p ~d^3 q}{(2\pi)^6} \Big (\Big (- \frac{1}{2} \sqrt{\omega_{\mathbf p} \omega_{\mathbf q}} \Big ) \Big (\hat a_{\mathbf p} \hat a_{\mathbf q} \underbrace{\exp(i (\mathbf p + \mathbf q) \cdot \mathbf x)}_{\delta^3 (\mathbf p + \mathbf q)} - \hat a_{\mathbf p} \hat a_{\mathbf q}^\dagger \underbrace{\exp(i (\mathbf p - \mathbf q) \cdot \mathbf x)}_{\delta^3 (\mathbf p - \mathbf q)} \\ & \qquad - \hat a_{\mathbf p}^\dagger \hat a_{\mathbf q} \underbrace{\exp(i (- \mathbf p + \mathbf q) \cdot \mathbf x)}_{\delta^3 (\mathbf p - \mathbf q)} + \hat a_{\mathbf p}^\dagger \hat a_{\mathbf q}^\dagger \underbrace{\exp(i (- \mathbf p - \mathbf q) \cdot \mathbf x)}_{\delta^3 (\mathbf p + \mathbf q)} \Big) \\ & \qquad + \frac{1}{2 \sqrt{\omega_{\mathbf p} \omega_{\mathbf q}}} \Big (- \mathbf p \cdot \mathbf q \hat a_{\mathbf p} \hat a_{\mathbf q} \underbrace{\exp(i (\mathbf p + \mathbf q) \cdot \mathbf x)}_{\delta^3 (\mathbf p + \mathbf q)} + \mathbf p \cdot \mathbf q \hat a_{\mathbf p} \hat a_{\mathbf q}^\dagger \underbrace{\exp(i (\mathbf p - \mathbf q) \cdot \mathbf x)}_{\delta^3 (\mathbf p - \mathbf q)} \\ & \qquad + \mathbf p \cdot \mathbf q \hat a_{\mathbf p}^\dagger \hat a_{\mathbf q} \underbrace{\exp(i (- \mathbf p + \mathbf q) \cdot \mathbf x)}_{\delta^3 (\mathbf p - \mathbf q)} - \mathbf p \cdot \mathbf q \hat a_{\mathbf p}^\dagger \hat a_{\mathbf q}^\dagger \underbrace{\exp(i (- \mathbf p - \mathbf q) \cdot \mathbf x)}_{\delta^3 (\mathbf p + \mathbf q)} \Big) \\ & \qquad + \frac{m^2}{2 \sqrt{\omega_{\mathbf p} \omega_{\mathbf q}}} \Big (\hat a_{\mathbf p} \hat a_{\mathbf q} \underbrace{\exp(i (\mathbf p + \mathbf q) \cdot \mathbf x)}_{\delta^3 (\mathbf p + \mathbf q)} + \hat a_{\mathbf p} \hat a_{\mathbf q}^\dagger \underbrace{\exp(i (\mathbf p - \mathbf q) \cdot \mathbf x)}_{\delta^3 (\mathbf p - \mathbf q)} \\ & \qquad + \hat a_{\mathbf p}^\dagger \hat a_{\mathbf q} \underbrace{\exp(i (- \mathbf p + \mathbf q) \cdot \mathbf x)}_{\delta^3 (\mathbf p - \mathbf q)} + \hat a_{\mathbf p}^\dagger \hat a_{\mathbf q}^\dagger \underbrace{\exp(i (- \mathbf p - \mathbf q) \cdot \mathbf x)}_{\delta^3 (\mathbf p + \mathbf q)} \Big) \Big) 
        \end{aligned}
        \end{equation*}
        \begin{equation*}
        \begin{aligned}
            \phantom{\hat H} & = \frac{1}{2} \int \frac{d^3 p ~d^3 q}{(2\pi)^3} \Big (\Big (- \frac{1}{2} \sqrt{\omega_{\mathbf p} \omega_{\mathbf q}} \Big ) \Big (\hat a_{\mathbf p} \hat a_{\mathbf q} \underbrace{\delta^3 (\mathbf p + \mathbf q)}_{\mathbf p = - \mathbf q} - \hat a_{\mathbf p} \hat a_{\mathbf q}^\dagger \underbrace{\delta^3 (\mathbf p - \mathbf q)}_{\mathbf p = \mathbf q} \\ & \qquad - \hat a_{\mathbf p}^\dagger \hat a_{\mathbf q} \underbrace{\delta^3 (\mathbf p - \mathbf q)}_{\mathbf p = \mathbf q} + \hat a_{\mathbf p}^\dagger \hat a_{\mathbf q}^\dagger \underbrace{\delta^3 (\mathbf p + \mathbf q)}_{\mathbf p = - \mathbf q} \Big) \\ & \qquad + \frac{1}{2 \sqrt{\omega_{\mathbf p} \omega_{\mathbf q}}} \Big (- \mathbf p \cdot \mathbf q \hat a_{\mathbf p} \hat a_{\mathbf q} \underbrace{\delta^3 (\mathbf p + \mathbf q)}_{\mathbf p = - \mathbf q} + \mathbf p \cdot \mathbf q \hat a_{\mathbf p} \hat a_{\mathbf q}^\dagger \underbrace{\delta^3 (\mathbf p - \mathbf q)}_{\mathbf p = \mathbf q} \\ & \qquad + \mathbf p \cdot \mathbf q \hat a_{\mathbf p}^\dagger \hat a_{\mathbf q} \underbrace{\delta^3 (\mathbf p - \mathbf q)}_{\mathbf p = \mathbf q} - \mathbf p \cdot \mathbf q \hat a_{\mathbf p}^\dagger \hat a_{\mathbf q}^\dagger \underbrace{\delta^3 (\mathbf p + \mathbf q)}_{\mathbf p = - \mathbf q} \Big) \\ & \qquad + \frac{m^2}{2 \sqrt{\omega_{\mathbf p} \omega_{\mathbf q}}} \Big (\hat a_{\mathbf p} \hat a_{\mathbf q} \underbrace{\delta^3 (\mathbf p + \mathbf q)}_{\mathbf p = - \mathbf q} + \hat a_{\mathbf p} \hat a_{\mathbf q}^\dagger \underbrace{\delta^3 (\mathbf p - \mathbf q)}_{\mathbf p = \mathbf q} \\ & \qquad + \hat a_{\mathbf p}^\dagger \hat a_{\mathbf q} \underbrace{\delta^3 (\mathbf p - \mathbf q)}_{\mathbf p = \mathbf q} + \hat a_{\mathbf p}^\dagger \hat a_{\mathbf q}^\dagger \underbrace{\delta^3 (\mathbf p + \mathbf q)}_{\mathbf p = - \mathbf q} \Big) \Big)  
        \end{aligned}
        \end{equation*}
        \begin{equation*}
        \begin{aligned}
            \phantom{\hat H} & = \frac{1}{2} \int \frac{d^3 p}{(2\pi)^3} \Big ( \Big (-\frac{\omega_{\mathbf p}}{2} \Big) \Big (\hat a_{\mathbf p} \hat a_{- \mathbf p} - \hat a_{\mathbf p} \hat a_{\mathbf p}^\dagger - \hat a_{\mathbf p}^\dagger \hat a_{\mathbf p} + \hat a_{\mathbf p}^\dagger \hat a_{- \mathbf p}^\dagger \Big) \\ & \qquad + \Big (\frac{|\mathbf p|^2}{2 \omega_{\mathbf p}} \Big) \Big (\hat a_{\mathbf p} \hat a_{- \mathbf p} + \hat a_{\mathbf p} \hat a_{\mathbf p}^\dagger + \hat a_{\mathbf p}^\dagger \hat a_{\mathbf p} + \hat a_{\mathbf p}^\dagger \hat a_{- \mathbf p}^\dagger \Big) \\ & \qquad + \Big ( \frac{m^2}{2 \omega_{\mathbf p}} \Big) \Big (\hat a_{\mathbf p} \hat a_{- \mathbf p}+ \hat a_{\mathbf p} \hat a_{\mathbf p}^\dagger  + \hat a_{\mathbf p}^\dagger \hat a_{\mathbf p} + \hat a_{\mathbf p}^\dagger \hat a_{- \mathbf p}^\dagger \Big) \Big)
        \end{aligned}
        \end{equation*}
        \begin{equation*}
        \begin{aligned}
            \phantom{\hat H} & = \frac{1}{4} \int \frac{d^3 p}{(2\pi)^3} \frac{1}{\omega_{\mathbf p}} \Big ( (\hat a_{\mathbf p} \hat a_{- \mathbf p} + \hat a_{\mathbf p}^\dagger \hat a_{- \mathbf p}^\dagger ) \underbrace{(- \omega_{\mathbf p}^2 + | \mathbf p|^2 + m^2 )}_0 \\ & \qquad + (\hat a_{\mathbf p} \hat a_{\mathbf p}^\dagger + \hat a_{\mathbf p}^\dagger \hat a_{\mathbf p} ) \underbrace{(\omega_{\mathbf p}^2 + | \mathbf p|^2 + m^2 )}_{2\omega_{\mathbf p}^2} \Big) \\ & = \frac{1}{4} \int \frac{d^3 p}{(2\pi)^3} \frac{2 \omega_{\mathbf p}^{\cancel{2}}}{\cancel{\omega_{\mathbf p}}} (\hat a_{\mathbf p} \hat a_{\mathbf p}^\dagger + \hat a_{\mathbf p}^\dagger \hat a_{\mathbf p}) \\ & = \frac{1}{2} \int \frac{d^3 p}{(2\pi)^3} \omega_{\mathbf p} (\underbrace{\hat a_{\mathbf p} \hat a_{\mathbf p}^\dagger}_{[\hat a_{\mathbf p}, \hat a_{\mathbf p}^\dagger] + \hat a_{\mathbf p}^\dagger \hat a_{\mathbf p}} + \hat a_{\mathbf p}^\dagger \hat a_{\mathbf p}) \\ & = \frac{1}{2} \int \frac{d^3 p}{(2\pi)^3} \omega_{\mathbf p} (\underbrace{[\hat a_{\mathbf p}, \hat a_{\mathbf p}^\dagger]}_{(2\pi)^3 \delta^3 (\mathbf p - \mathbf p)} + 2 \hat a_{\mathbf p}^\dagger \hat a_{\mathbf p}) \\ & = \frac{1}{2} \int d^3 p ~ \omega_{\mathbf p} \delta^3 (0) + \int \frac{d^3 p}{(2\pi)^3} ~ \omega_{\mathbf p}\hat a_{\mathbf p}^\dagger \hat a_{\mathbf p} ~,
        \end{aligned}
        \end{equation*}
        where we have used the fact that $\omega_{- \mathbf p} = \sqrt{| - \mathbf p|^2 + m^2} = \sqrt{|\mathbf p|^2 + m^2} = \omega_{\mathbf p}$. 

    \end{proof}

    The first term of~\eqref{hamkg} counts simply how what is the relativistic energy of each particle $\omega_{\mathbf p}$ and through the number operator $\hat N_{\mathbf p} = \hat a_{\mathbf p}^\dagger \hat a_{\mathbf p}$ and the integral, we sum all over the possible value of $\mathbf p$. However, most of them may be zero and we do not have to worry about divergences. 

\section{Vacuum energy}
    Things are different if we look at the second term of~\eqref{hamkg}, beacuse, in analogy with the energy of the single harmonic oscillator, we interpret it as the energy of the vacuum and it diverges for two reasons
    \begin{enumerate}
        \item infrared divergence, i.e. 
            \begin{equation*}
                \delta^3 (0) \rightarrow \infty~,
            \end{equation*}
        \item ultraviolet divergence, i.e. for $|\mathbf p| \rightarrow \infty$
            \begin{equation*}
            \int d^3 p ~ \omega_{\mathbf p} \rightarrow \infty ~,
        \end{equation*} 
            since for $|\mathbf p| \rightarrow \infty$
            \begin{equation*}
                \omega_{\mathbf p} = \sqrt{|\mathbf p|^2 + m^2} \simeq |\mathbf p| ~.
            \end{equation*}
    \end{enumerate}

    This can be better understood by applying the hamiltonian operator to the vacuum state $\ket{0}$, i.e.~the state such that it is annihilated by all the annihilation operators is for all $\mathbf p$
    \begin{equation*}
        \hat a_{\mathbf p} \ket{0} = 0 \quad \forall \mathbf p ~.
    \end{equation*}
    Therefore 
    \begin{equation*}
        \hat H \ket{0} = E_0 \ket{0} = \infty \ket{0}
    \end{equation*}
    and the vaccum energy is infinite.
    \begin{proof}
        Infact, using~\eqref{hamkg}
        \begin{equation*}
            \hat H \ket{0} = \int \frac{d^3 p}{(2\pi)^3} \omega_{\mathbf p} \hat a_{\mathbf p}^\dagger \underbrace{\hat a_{\mathbf p} \ket{0}}_0 + \Big (\underbrace{\frac{1}{2} \int d^3 p ~ \omega_{\mathbf p} \delta^3 (0)}_\infty \Big ) \ket{0} = \infty \ket{0} = E_0 \ket{0} ~.
        \end{equation*}
    \end{proof}

\subsection{IR divergence}

    The infrared divergence is due to the fact that space is infinitely large. This means that in every point of spacetime there is an harmonic oscilators. To prove this, consider a box of sides $L$ and periodic boundary conditions for the fields. The volume of the box is just the Dirac delta inside the integrand of the energy vacuum. Infact 
    \begin{equation*}
        (2\pi)^3 \delta^3 (0) = \lim_{L \rightarrow \infty} \int_{-\frac{L}{2}}^{\frac{L}{2}} \int_{-\frac{L}{2}}^{\frac{L}{2}} \int_{-\frac{L}{2}}^{\frac{L}{2}} d^3 x ~ \exp(- i \mathbf p \cdot \mathbf x) \Big \vert_{\mathbf p = 0} = \lim_{L \rightarrow \infty} \int_{-\frac{L}{2}}^{\frac{L}{2}} \int_{-\frac{L}{2}}^{\frac{L}{2}} \int_{-\frac{L}{2}}^{\frac{L}{2}} d^3 x = L^3 = V ~.
    \end{equation*}
    This divergence can be removed by studying energy densities instead of pure energies. 
    \begin{equation*}
        \mathcal E_0 = \frac{E_0}{V} = \int \frac{d^3}{(2\pi)^3} \frac{\omega_{\mathbf p}}{2} ~.
    \end{equation*}

\subsection{UV divergence}

    However, still the energy density is infinite because of the ultraviolet divergence, since for $|\mathbf p| \rightarrow \infty$
    \begin{equation*}
        \mathcal E_0 \rightarrow \infty ~.
    \end{equation*}
    
    The reason is the following: we made a strong assumption considering the theory valid for any large value of energy and now we have found where the theory breaks, since this divergence arises indeed from the fact that our theory is not valid for arbitrarily high energies. What we need to do id to introduce a cut-off, i.e. a maximum energy after which the theory is not anymore valid. Since gravity cannot be neglected and becomes strongly coupled at Planck mass $M_P \simeq 10^{19} GeV$, we therefore set the cut-off at this energy. 

    Computationally, we measure only energy differences between excited estates, which are particles, and the vacuum energy, which becomes irrelevant and it can be set to zero. This procedure is called \textit{normal ordering}. We define a new hamiltonian operator 
    \begin{equation*}
        \colon \hat H \colon = \hat H - E_0 = \hat H - \bra{0} \hat H \ket{0} ~,
    \end{equation*}
    such that 
    \begin{equation*}
        \colon \hat H \colon \ket{0} = \underbrace{\hat H \ket{0}}_{E_0 \ket{0}} - E_0 \ket{0} = 0~.
    \end{equation*}
    The difference between $\hat H$ and $\colon \hat H \colon$ is due to an ambiguity in going from classical to quantum theory. Infact, normal ordering means to set a rule to order annihilation and creation operators: all annihilation operators are pleced to the right and, consequently, creation operatore to the left (dagger always first). We emphasise that in the interaction theory, vaccume energy cannot be anymore set to zero.

    As we said, different ordering in the classical hamiltonians bring different hamiltonian operators. Infact, if we rewrite the hamiltonian of the classical harmonic oscillator
    \begin{equation*}
        H = \frac{p^2}{2m} + \frac{1}{2} \omega^2 q^2 = \frac{1}{2} (\omega q - i p) (\omega q + i p) ~,
    \end{equation*}
    we notice that the first one leads us to
    \begin{equation*}
        \hat H = \omega (\hat a^\dagger \hat a + \frac{\mathbb I}{2}) ~,
    \end{equation*}
    while the second one to 
    \begin{equation*}
        \hat H = \omega a^\dagger \hat a ~.
    \end{equation*}
    \begin{proof}
        For the first hamiltonian 
        \begin{equation*}
        \begin{aligned}
            \hat H & = \frac{1}{2} (-i \sqrt{\frac{\omega}{2}} (\hat a - \hat a^\dagger))^2 + \frac{1}{2} \omega^2 (\frac{1}{\sqrt{2 \omega}} (\hat a + \hat a^\dagger))^2 \\ & = - \frac{\omega}{4} (\cancel{\hat a^2} - \hat a \hat a^\dagger - \hat a^\dagger \hat a + \cancel{(\hat a^\dagger)^2}) + \frac{\omega}{4} (\cancel{\hat a^2} + \hat a \hat a^\dagger + \hat a^\dagger \hat a + \cancel{(\hat a^\dagger)^2}) \\ & = \frac{\omega}{4} (\hat a \hat a^\dagger + \hat a^\dagger \hat a + \hat a \hat a^\dagger + \hat a^\dagger \hat a) \\ & = \frac{\omega}{2} (\underbrace{\hat a \hat a^\dagger}_{[\hat a, \hat a^\dagger] + \hat a^\dagger \hat a} + \hat a^\dagger \hat a) \\ & = \frac{\omega}{2} (\underbrace{[\hat a, \hat a^\dagger]}_{\mathbb I} + 2 \hat a^\dagger \hat a) \\ & = \omega (\frac{\mathbb I}{2}+ \hat a^\dagger \hat a) ~,
        \end{aligned}
        \end{equation*}
        while for the second hamiltonian
        \begin{equation*}
        \begin{aligned}
            \hat H & = \frac{1}{2} \Big (\omega \frac{1}{\sqrt{2 \omega}} (\hat a + \hat a^\dagger) - i (-i \sqrt{\frac{\omega}{2}} (\hat a - \hat a^\dagger)) \Big ) \Big (\omega \frac{1}{\sqrt{2 \omega}} (\hat a + \hat a^\dagger) + i (-i \sqrt{\frac{\omega}{2}} (\hat a - \hat a^\dagger)) \Big) \\ & = \frac{\omega}{4} ( \cancel{\hat a} + \hat a^\dagger - \cancel{\hat a} + \hat a^\dagger ) (\hat a + \cancel{\hat a^\dagger} + \hat a - \cancel{\hat a^\dagger}) \\ & = \omega \hat a^\dagger \hat a ~.
        \end{aligned}
        \end{equation*}
    \end{proof}

    Finally, the normal ordered hamiltonian of the Klein-Gordon theory is 
    \begin{equation} \label{hamop}
        \colon \hat H \colon = \int \frac{d^3 p}{(2\pi)^3} \omega_{\mathbf p} \hat a_{\mathbf p}^\dagger \hat a_{\mathbf p} ~.
    \end{equation}
    \begin{proof}
        Infact, since
        \begin{equation*}
            \hat H = \frac{1}{2} \int \frac{d^3 p}{(2\pi)^3} \omega_{\mathbf p} (\hat a_{\mathbf p} \hat a_{\mathbf p}^\dagger + \hat a_{\mathbf p}^\dagger \hat a_{\mathbf p}) ~,
        \end{equation*}
        we have 
        \begin{equation*}
            \colon \hat H \colon = \frac{1}{2} \int \frac{d^3 p}{(2\pi)^3} \omega_{\mathbf p} (\hat a_{\mathbf p}^\dagger \hat a_{\mathbf p} + \hat a_{\mathbf p}^\dagger \hat a_{\mathbf p}) = \int \frac{d^3 p}{(2\pi)^3} \omega_{\mathbf p} \hat a_{\mathbf p}^\dagger \hat a_{\mathbf p} ~.
        \end{equation*}
    \end{proof}

    Furthermore, by analogy of the harmonic oscillator, the hamiltonian~\eqref{hamkg} and the annihilation and creation operators satisfies the commutation relations 
    \begin{equation*}
        [\hat H, \hat a_{\mathbf p}] = - \omega_{\mathbf p} \hat a_{\mathbf p} ~, \quad [\hat H, \hat a_{\mathbf p}^\dagger] = \omega_{\mathbf p} \hat a_{\mathbf p}^\dagger ~.
    \end{equation*}
    \begin{proof}
        For the first commutator
        \begin{equation*}
        \begin{aligned}
            [\hat H, \hat a_{\mathbf p}] & = \int \frac{d^3 q}{(2\pi)^3} \omega_{\mathbf q} [\hat a_{\mathbf q}^\dagger \hat a_{\mathbf q}, \hat a_{\mathbf p}] \\ & = \int \frac{d^3 q}{(2\pi)^3} \omega_{\mathbf q} (\hat a_{\mathbf q}^\dagger \underbrace{[\hat a_{\mathbf q}, \hat a_{\mathbf p}]}_0 + \underbrace{[\hat a_{\mathbf q}^\dagger, \hat a_{\mathbf p}]}_{- (2\pi)^3 \delta^3 (\mathbf p - \mathbf q)} \hat a_{\mathbf q}) \\ & = - \int \frac{d^3 q}{\cancel{(2\pi)^3}} \omega_{\mathbf q} \cancel{(2\pi)^3} \delta^3 (\mathbf p - \mathbf q) \hat a_{\mathbf q} \\ & = - \omega_{\mathbf p} \hat a_{\mathbf p} ~.
        \end{aligned}
        \end{equation*}

        For the second commutator
        \begin{equation*}
        \begin{aligned}
            [\hat H, \hat a_{\mathbf p}^\dagger] & = \int \frac{d^3 q}{(2\pi)^3} \omega_{\mathbf q} [\hat a_{\mathbf q}^\dagger \hat a_{\mathbf q}, \hat a_{\mathbf p}^\dagger] \\ & = \int \frac{d^3 q}{(2\pi)^3} \omega_{\mathbf q} (\hat a_{\mathbf q}^\dagger \underbrace{[\hat a_{\mathbf q}, \hat a_{\mathbf p}^\dagger]}_ {(2\pi)^3 \delta^3 (\mathbf p - \mathbf q)} + \underbrace{[\hat a_{\mathbf q}^\dagger, \hat a_{\mathbf p}^\dagger]}_{0} \hat a_{\mathbf q}) \\ & = \int \frac{d^3 q}{\cancel{(2\pi)^3}} \omega_{\mathbf q} \cancel{(2\pi)^3} \delta^3 (\mathbf p - \mathbf q) \hat a_{\mathbf q}^\dagger \\ & = \omega_{\mathbf p} \hat a_{\mathbf p}^\dagger ~.
        \end{aligned}
        \end{equation*}
    \end{proof}

    The momentum operator is defined as 
    \begin{equation}\label{momop}
        \hat{\mathbf P} = - \int d^3 x ~ \hat \pi \boldsymbol \nabla \hat \varphi = \int \frac{d^3 p}{(2\pi)^3} \mathbf p \hat a_{\mathbf p}^\dagger \hat a_{\mathbf p}~.
    \end{equation}
    \begin{proof}
        Infact, using~\eqref{momen}
        \begin{equation*}
        \begin{aligned}
            \hat{\mathbf P} & = \int d^3 x ~ T^{0i} \\ & = \int d^3 x ~ \hat \pi \boldsymbol \nabla \hat \varphi ~.
        \end{aligned}
        \end{equation*}

        Furthermore, using~\eqref{anncrea},~\eqref{kgfop},~\eqref{kgpop} and~\eqref{deltaint}
        \begin{equation*}
        \begin{aligned}
            \hat{\mathbf P} & = - \int d^3 x ~ \Big ( \int \frac{d^3 p}{{(2\pi)}^3} \Big (- i\sqrt{\frac{\omega_{\mathbf p}}{2}} \Big ) \Big (\hat a_{\mathbf p} \exp(i \mathbf p \cdot \mathbf x) - \hat a_{\mathbf p}^\dagger \exp(- i \mathbf p \cdot \mathbf x) \Big) \\ & \qquad \nabla \int \frac{d^3 q}{{(2\pi)}^3} \frac{1}{\sqrt{2 \omega_{\mathbf q}}} \Big (\hat a_{\mathbf q} \exp(i \mathbf q \cdot \mathbf x) + \hat a_{\mathbf q}^\dagger \exp(- i \mathbf q \cdot \mathbf x) \Big) \Big ) \\ & = - \int d^3 x ~ \Big ( \int \frac{d^3 p}{{(2\pi)}^3} \Big (- i\sqrt{\frac{\omega_{\mathbf p}}{2}} \Big ) \Big (\hat a_{\mathbf p} \exp(i \mathbf p \cdot \mathbf x) - \hat a_{\mathbf p}^\dagger \exp(- i \mathbf p \cdot \mathbf x) \Big) \\ & \qquad \int \frac{d^3 q}{{(2\pi)}^3} \frac{1}{\sqrt{2 \omega_{\mathbf q}}} \Big (i \mathbf q \hat a_{\mathbf q} \exp(i \mathbf q \cdot \mathbf x) - i \mathbf q \hat a_{\mathbf q}^\dagger \exp(- i \mathbf q \cdot \mathbf x) \Big) \Big ) \\ & = - \int \frac{d^3 x ~ d^3 p ~ d^3 q}{{(2\pi)}^6} \Big (- \frac{i}{2} \sqrt{\frac{\omega_{\mathbf p}}{\omega_{\mathbf q}}} \Big ) (i \mathbf q \hat a_{\mathbf p} \hat a_{\mathbf q} \exp(i (\mathbf p + \mathbf q) \cdot \mathbf x) - i \mathbf q  \hat a_{\mathbf p} \hat a_{\mathbf q}^\dagger \exp(i (\mathbf p - \mathbf q) \cdot \mathbf x) \\ & \qquad - i \mathbf q \hat a_{\mathbf p}^\dagger \hat a_{\mathbf q} \exp(i (- \mathbf p + \mathbf q) \cdot \mathbf x) + i \mathbf q \hat a_{\mathbf p}^\dagger \hat a_{\mathbf q}^\dagger \exp(i (- \mathbf p - \mathbf q) \cdot \mathbf x) )
        \end{aligned}
        \end{equation*}
        \begin{equation*}
        \begin{aligned}
            \phantom{\hat{\mathbf P}} & = - \int \frac{d^3 x ~ d^3 p ~ d^3 q}{{(2\pi)}^6} \Big (\frac{\mathbf q}{2} \sqrt{\frac{\omega_{\mathbf p}}{\omega_{\mathbf q}}} \Big ) (\hat a_{\mathbf p} \hat a_{\mathbf q} \underbrace{\exp(i (\mathbf p + \mathbf q) \cdot \mathbf x)}_{\delta^3 (\mathbf p + \mathbf q)} - \hat a_{\mathbf p} \hat a_{\mathbf q}^\dagger \underbrace{\exp(i (\mathbf p - \mathbf q) \cdot \mathbf x)}_{\delta^3 (\mathbf p - \mathbf q)} \\ & \qquad - \hat a_{\mathbf p}^\dagger \hat a_{\mathbf q} \underbrace{\exp(i (-\mathbf p + \mathbf q) \cdot \mathbf x)}_{\delta^3 (\mathbf p - \mathbf q)} + \hat a_{\mathbf p}^\dagger \hat a_{\mathbf q}^\dagger \underbrace{\exp(i (-\mathbf p - \mathbf q) \cdot \mathbf x)}_{\delta^3 (\mathbf p + \mathbf q)} ) \\ & = - \int \frac{d^3 p ~ d^3 q}{{(2\pi)}^3} \Big (\frac{\mathbf q}{2} \sqrt{\frac{\omega_{\mathbf p}}{\omega_{\mathbf q}}} \Big ) (\hat a_{\mathbf p} \hat a_{\mathbf q} \underbrace{\delta^3 (\mathbf p + \mathbf q) }_{\mathbf q = - \mathbf p} - \hat a_{\mathbf p} \hat a_{\mathbf q}^\dagger \underbrace{\delta^3 (\mathbf p - \mathbf q) }_{\mathbf q = \mathbf p} \\ & \qquad - \hat a_{\mathbf p}^\dagger \hat a_{\mathbf q} \underbrace{\delta^3 (\mathbf p - \mathbf q) }_{\mathbf q = \mathbf p} + \hat a_{\mathbf p}^\dagger \hat a_{\mathbf q}^\dagger \underbrace{\delta^3 (\mathbf p + \mathbf q) }_{\mathbf q = - \mathbf p}) \\ & = - \int \frac{d^3 p}{{(2\pi)}^3} \Big (\frac{\mathbf p}{2} (- \hat a_{\mathbf p}^\dagger \hat a_{\mathbf p} - \hat a_{\mathbf p} \hat a_{\mathbf p}^\dagger) - \frac{\mathbf p}{2} (\hat a_{\mathbf p} \hat a_{- \mathbf p} + \hat a_{\mathbf p}^\dagger \hat a_{- \mathbf p}^\dagger) \Big) \\ & = \int \frac{d^3 p}{{(2\pi)}^3} \Big (\frac{\mathbf p}{2} (\hat a_{\mathbf p}^\dagger \hat a_{\mathbf p} + \hat a_{\mathbf p} \hat a_{\mathbf p}^\dagger) + \underbrace{\frac{\mathbf p}{2} (\hat a_{\mathbf p} \hat a_{- \mathbf p} + \hat a_{\mathbf p}^\dagger \hat a_{- \mathbf p}^\dagger)}_{0} \Big)  ~,
        \end{aligned}
        \end{equation*}
        where in the last row, the second term vanishes because it is an odd function integrated all over $\mathbb R^3$. Finally, in normal ordering  
        \begin{equation*}
            \hat{\mathbf P} = \int \frac{d^3 p}{{(2\pi)}^3} \Big (\frac{\mathbf p}{2} (\hat a_{\mathbf p}^\dagger \hat a_{\mathbf p} + \hat a_{\mathbf p}^\dagger \hat a_{\mathbf p}^\dagger) \Big) = \int \frac{d^3 p}{{(2\pi)}^3} \mathbf p \hat a_{\mathbf p}^\dagger \hat a_{\mathbf p} ~.
        \end{equation*}
    \end{proof}

\section{$1$-particle states}

    Now, we build the energy eigenstates of a $1$-particle state. In analogy with the harmonic oscillator, we require the following properties:
    \begin{enumerate}
        \item the vacuum state is annihilated by all the annihilation operators for all $\mathbf p$ 
            \begin{equation*}
                \hat a_{\mathbf p} \ket{0} = 0 \quad \forall \mathbf p ~,
            \end{equation*}
        \item a generic state can be defined by the creation operators actiong on the vacuum
            \begin{equation*}
                \ket{\mathbf p} = \hat a_{\mathbf p}^\dagger \ket{0} ~.
            \end{equation*}
    \end{enumerate}

    The state $\ket{\mathbf p}$ is the momentum eigentstate of a single scalar (spinless) particle with mass $m$. Infact, it is the momentum eigenstate 
    \begin{equation*}
        \hat{\mathbf P} \ket{\mathbf p} = \mathbf p \ket{\mathbf p} ~,
    \end{equation*}
    \begin{proof}
        Infact, using~\eqref{momop}
        \begin{equation*}
        \begin{aligned}
            \hat{\mathbf P} \ket{\mathbf p} & = \hat{\mathbf P} \hat a_{\mathbf p}^\dagger \ket{0} \\ & = \int \frac{d^3 q}{(2\pi)^3} \mathbf q \hat a_{\mathbf q}^\dagger \underbrace{\hat a_{\mathbf q} \hat a_{\mathbf p}^\dagger}_{[\hat a_{\mathbf q}, \hat a_{\mathbf p}^\dagger] + \hat a_{\mathbf q}^\dagger \hat a_{\mathbf p}} \ket{0} \\ & = \int \frac{d^3 q}{(2\pi)^3} \mathbf q \hat a_{\mathbf q}^\dagger (\underbrace{[\hat a_{\mathbf q}, \hat a_{\mathbf p}^\dagger]}_{(2\pi)^3 \delta^3 (\mathbf p - \mathbf q)} + \hat a_{\mathbf q}^\dagger \underbrace{\hat a_{\mathbf p}) \ket{0}}_0 \\ & = \int \frac{d^3 q}{\cancel{(2\pi)^3}} \mathbf q \hat a_{\mathbf q}^\dagger  \cancel{(2\pi)^3} \underbrace{\delta^3 (\mathbf p - \mathbf q)}_{\mathbf q = \mathbf p} \ket{0} \\ & = \mathbf p \hat a_{\mathbf p}^\dagger \ket{0} \\ & = \mathbf p \ket{\mathbf p}  ~.
        \end{aligned}
        \end{equation*}
    \end{proof}
    Furthermore, this states is also the energy eigenstate, since it is a function of the momentum 
    \begin{equation*}
        \hat H \ket{\mathbf p} = E_{\mathbf p} \ket{\mathbf p} = \omega_{\mathbf p} \ket{\mathbf p} ~.
    \end{equation*}
    \begin{proof}
        Infact, using~\eqref{hamop}
        \begin{equation*}
        \begin{aligned}
            \hat H \ket{\mathbf p} & = \hat H \hat a_{\mathbf p}^\dagger \ket{0} \\ & = \int \frac{d^3 q}{(2\pi)^3} \omega_{\mathbf q} \hat a_{\mathbf q}^\dagger \underbrace{\hat a_{\mathbf q} \hat a_{\mathbf p}^\dagger}_{[\hat a_{\mathbf q}, \hat a_{\mathbf p}^\dagger] + \hat a_{\mathbf q}^\dagger \hat a_{\mathbf p}} \ket{0} \\ & = \int \frac{d^3 q}{(2\pi)^3} \omega_{\mathbf q} \hat a_{\mathbf q}^\dagger (\underbrace{[\hat a_{\mathbf q}, \hat a_{\mathbf p}^\dagger]}_{(2\pi)^3 \delta^3 (\mathbf p - \mathbf q)} + \hat a_{\mathbf q}^\dagger \underbrace{\hat a_{\mathbf p}) \ket{0}}_0 \\ & = \int \frac{d^3 q}{\cancel{(2\pi)^3}} \omega_{\mathbf q} \hat a_{\mathbf q}^\dagger  \cancel{(2\pi)^3} \underbrace{\delta^3 (\mathbf p - \mathbf q)}_{\mathbf q = \mathbf p} \ket{0} \\ & = \omega_{\mathbf p} \hat a_{\mathbf p}^\dagger \ket{0} \\ & = \omega_{\mathbf p} \ket{\mathbf p}  ~.
        \end{aligned}
        \end{equation*}
    \end{proof}

\section{$n$-particle states}

    We can generalise for a system composed by $n$ particles. The state becomes 
    \begin{equation*}
        \ket{\mathbf p_1, \ldots \mathbf p_n} = \hat a_{\mathbf p_1}^\dagger \ldots \hat a_{\mathbf p_n}^\dagger \ket{0} ~.
    \end{equation*}
    
    Notice that the state is symmetric under exchange of any two particles, since 
    \begin{equation*}
        [\hat a_{\mathbf p_i}^\dagger, \hat a_{\mathbf p_j}^\dagger ] = 0 ~.
    \end{equation*}
    \begin{proof}
        For instance, given two particles of momenta $\mathbf p$ and $\mathbf q$, we have 
        \begin{equation*}
            \ket{\mathbf p, \mathbf q} = \hat a_{\mathbf p}^\dagger \hat a_{\mathbf q}^\dagger \ket{0} = \hat a_{\mathbf q}^\dagger  \hat a_{\mathbf p}^\dagger \ket{0} = \ket{\mathbf q, \mathbf p} ~.
        \end{equation*}
    \end{proof}
    This means that the Klein-Gordon theory describes bosons. It is indeed spin-statistics relation and it is a consequence of quantum field theory and the commutation relations imposed to quantise (not quantum mechanics).

    A basis of the Fock space is built by all the possible combination of creation operators acting on the vacuum state 
    \begin{equation*}
        \{\ket{0}, \hat a_{\mathbf p_1}^\dagger \ket{0}, \hat a_{\mathbf p_1}^\dagger \hat a_{\mathbf p_2}^\dagger \ket{0}, \ldots \} 
    \end{equation*},
    where $\ket{0}$ is the vacuum state, $\hat a_{\mathbf p_1}^\dagger \ket{0}$ is the $1$-particle state, $\hat a_{\mathbf p_1}^\dagger \hat a_{\mathbf p_2}^\dagger \ket{0}$ is the $2$-particles state, etc. The total Fock space is 
    \begin{equation*}
        \mathcal F = \bigoplus_n \mathcal H_n
    \end{equation*}
    where $\mathcal H_n$ is the Hilbert space for $n$ particles.

    We can define the number operator wich counts the number of particle in a given state 
    \begin{equation}\label{nop}
        \hat N = \int \frac{d^3 p}{(2\pi)^3} \hat a_{\mathbf p}^\dagger \hat a_{\mathbf p} ~,
    \end{equation}
    such that 
    \begin{equation*}
        \hat N \ket{\mathbf p_1, \ldots \mathbf p_n} = \hat N \hat a_{\mathbf p_1}^\dagger \ldots \hat a_{\mathbf p_n}^\dagger \ket{0} =  n \ket{\mathbf p_1, \ldots \mathbf p_n} ~.
    \end{equation*}

    Notice that the particle number is conserved, since
    \begin{equation*}
        [\hat H, \hat N] = 0 ~.
    \end{equation*}
    This means that if the system has initially $n$ particles, this number will remain the same. This happens only in a free theory, because interactions move the system between different sectors of the Fock space.
    \begin{proof}
        Infact, using~\eqref{hamop} and~\eqref{nop}
        \begin{equation*}
        \begin{aligned}
            [\hat H, \hat N] & = [\int \frac{d^3 p}{(2\pi)^3} \omega_{\mathbf p} \hat a_{\mathbf p}^\dagger \hat a_{\mathbf p}, \int \frac{d^3 q}{(2\pi)^3} \hat a_{\mathbf q}^\dagger \hat a_{\mathbf q}] \\ & = \int \frac{d^3 p ~ d^3q}{(2\pi)^6} \omega_{\mathbf p} [\hat a_{\mathbf p}^\dagger \hat a_{\mathbf p}, \hat a_{\mathbf q}^\dagger \hat a_{\mathbf q}] \\ & = \int \frac{d^3 p ~ d^3q}{(2\pi)^6} \omega_{\mathbf p} ( \hat a_{\mathbf p}^\dagger [\hat a_{\mathbf p}, \hat a_{\mathbf q}^\dagger \hat a_{\mathbf q}] + [\hat a_{\mathbf p}^\dagger, \hat a_{\mathbf q}^\dagger \hat a_{\mathbf q}] \hat a_{\mathbf p} ) \\ & = \int \frac{d^3 p ~ d^3q}{(2\pi)^6} \omega_{\mathbf p} ( \hat a_{\mathbf p}^\dagger \hat a_{\mathbf q}^\dagger \underbrace{[\hat a_{\mathbf p}, \hat a_{\mathbf q}]}_0 + \hat a_{\mathbf p}^\dagger \underbrace{[\hat a_{\mathbf p}, \hat a_{\mathbf q}^\dagger]}_{(2\pi)^3 \delta^3 (\mathbf p - \mathbf q)}  \hat a_{\mathbf q} + \hat a_{\mathbf q}^\dagger \underbrace{[\hat a_{\mathbf p}^\dagger, \hat a_{\mathbf q}]}_{- (2\pi)^3 \delta^3 (\mathbf p - \mathbf q)} \hat a_{\mathbf p} + \underbrace{[\hat a_{\mathbf p}^\dagger, \hat a_{\mathbf q}^\dagger]}_0 \hat a_{\mathbf q} \hat a_{\mathbf p} ) \\ & = \int \frac{d^3 p ~ d^3q}{(2\pi)^3} \omega_{\mathbf p} (\hat a_{\mathbf p}^\dagger \underbrace{\delta^3 (\mathbf p - \mathbf q)}_{\mathbf p = \mathbf q} \hat a_{\mathbf q} - \hat a_{\mathbf q}^\dagger \underbrace{\delta^3 (\mathbf p - \mathbf q)}_{\mathbf p = \mathbf q} \hat a_{\mathbf p} ) \\ & = \int \frac{d^3 p}{(2\pi)^3} \omega_{\mathbf p} (\cancel{\hat a_{\mathbf p}^\dagger \hat a_{\mathbf p}} - \cancel{\hat a_{\mathbf p}^\dagger \hat a_{\mathbf p}} ) = 0 ~.
        \end{aligned}
        \end{equation*}
    \end{proof}

\chapter{Two real (or complex) Klein-Gordon field}

    Consider two real Klein-Gordon fields $\varphi_1$ and $\varphi_2$ with different masses $m_1 \neq m_2$. Their lagrangian is $\forall i = 1,2$
    \begin{equation*}
        \mathcal L = \sum_{i=1}^{2} \Big ( \frac{1}{2} \partial_\mu \varphi_i \partial^\mu \varphi_i - \frac{1}{2} m_i^2 \varphi^2_i \Big) ~,
    \end{equation*}
    such that the equations of motion are $2$ independent Klein-Gordon equations for each field
    \begin{equation*}
        (\Box + m_i^2) \varphi_i (x) = 0 ~.
    \end{equation*}
    Since the fields are not interacting, in normal ordering the total hamiltonian operator is 
    \begin{equation*}
        \hat H = \hat H_1 + \hat H_2 ~,
    \end{equation*}
    where 
    \begin{equation*}
        \hat H_i = \int \frac{d^3 p}{(2\pi)^3} \omega_{i, \mathbf p} \hat a_{i, \mathbf p}^\dagger \hat a_{i, \mathbf p} 
    \end{equation*}
    and 
    \begin{equation*}
        \omega_{i, \mathbf p} = \sqrt{|\mathbf p|^2 + m_i^2} ~,
    \end{equation*}
    the total momentum operator is 
    \begin{equation*}
        \hat{\mathbf P} = \hat{\mathbf P_1} + \hat{\mathbf P_2}  ~,
    \end{equation*}
    where 
    \begin{equation*}
        \hat{\mathbf P_i} = \int \frac{d^3 p}{(2\pi)^3} \mathbf p \hat a_{i, \mathbf p}^\dagger \hat a_{i, \mathbf p} 
    \end{equation*}
    and the total number operator is 
    \begin{equation*}
        \hat N = \hat N_1 + \hat N_2 ~,
    \end{equation*}
    where 
    \begin{equation*}
        \hat N_i = \int \frac{d^3 p}{(2\pi)^3} \hat a_{i, \mathbf p}^\dagger \hat a_{i, \mathbf p} ~.
    \end{equation*}

    For particle states, the vacuum state is 
    \begin{equation*}
        \hat a_{i, \mathbf p} \ket{0} = 0 
    \end{equation*} 
    and the action of $\hat a_{i, \mathbf p}^\dagger$ on it creates a relativistic particle with mass $m_i$ 
    \begin{equation*}
        \ket{\mathbf p_i} = \hat a_{i, \mathbf p}^\dagger \ket{0} ~.
    \end{equation*}
    Furthermore, they are eigenstates of the total operators  
    \begin{equation*}
        \hat H \ket{\mathbf p_i} = \omega_{i, \mathbf p} \ket{\mathbf p_i} ~, \quad \hat{\mathbf P} \ket{\mathbf p_i} = \mathbf p \ket{\mathbf p_i} ~, \quad \hat N \ket{\mathbf p_i} = 1 \ket{\mathbf p_i} ~.
    \end{equation*}
    Notice that they seem degenerate in $\mathbf p$ since they have the same momentum, but they can always be distinguished by a measurement of their energy since it is different
    \begin{equation*}
        \omega_{1, \mathbf p} = \sqrt{|\mathbf p|^2 + m_1^2} \neq \omega_{2, \mathbf p} = \sqrt{|\mathbf p|^2 + m_2^2} ~.
    \end{equation*}

\section{Electrical charge via $O(2)$ symmetry}

    The more interesting case is the equal-mass one $m_1 = m_2$, because it arises a new symmetry of the action. Rewriting the lagrangian in term of a vector and its transpose one
    \begin{equation*}
        \boldsymbol \varphi = \begin{bmatrix}
            \varphi_1 \\ \varphi_2
        \end{bmatrix} ~, \boldsymbol \varphi^T = \begin{bmatrix}
            \varphi_1 & \varphi_2
        \end{bmatrix} ~,
    \end{equation*}
    we obtain 
    \begin{equation*}
        \mathcal L = \frac{1}{2} (\partial_\mu \boldsymbol \varphi^T) (\partial^\mu \boldsymbol \varphi) - \frac{1}{2} m^2 \boldsymbol \varphi^T \boldsymbol \varphi ~.
    \end{equation*}
    Since this lagrangian is invariant by an $O(2)$ rotation in the field space, the Noether's theorem allows us to define a charge operator
    \begin{equation*}
        \hat Q = - i \int \frac{d^3 p}{(2\pi)^3} (\hat a_{1, \mathbf p} \hat a_{2, \mathbf p}^\dagger - \hat a_{2, \mathbf p} \hat a_{1, \mathbf p}^\dagger)
    \end{equation*}
    where we have not used normal ordering. 
    \begin{proof}
        The lagrangian is invariant under an $O(2)$ rotation in the $(\varphi_1, \varphi_2)$ plane. Infact, for a rotation 
        \begin{equation*}
            \boldsymbol \varphi' = R \boldsymbol \varphi ~,
        \end{equation*}
        we have 
        \begin{equation*}
        \begin{aligned}
            \mathcal L' & = \frac{1}{2} (\partial_\mu (\boldsymbol \varphi')^T) (\partial^\mu \boldsymbol \varphi') - \frac{1}{2} m^2 (\boldsymbol \varphi')^T \boldsymbol \varphi' \\ & \\ & = \frac{1}{2} (\partial_\mu (R \boldsymbol \varphi)^T) (\partial^\mu R \boldsymbol \varphi) - \frac{1}{2} m^2 (R \boldsymbol \varphi)^T R \boldsymbol \varphi \\ & = \frac{1}{2} (\partial_\mu \boldsymbol \varphi^T) \underbrace{R^T R}_{\mathbb I} (\partial^\mu \boldsymbol \varphi) - \frac{1}{2} m^2 \boldsymbol \varphi^T \underbrace{R^T R}_{\mathbb I} \boldsymbol \varphi \\ & = \frac{1}{2} (\partial_\mu \boldsymbol \varphi^T) (\partial^\mu \boldsymbol \varphi) - \frac{1}{2} m^2 \boldsymbol \varphi^T \boldsymbol \varphi = \mathcal L ~,
        \end{aligned}
        \end{equation*}
        since the lagrangian depends only on the lenght of $|\varphi|^2 = \boldsymbol \varphi^T \boldsymbol \varphi$. Now, we compute the conserved current by considering an infinitesimal transformation matrix 
        \begin{equation*}
            R = \begin{bmatrix}
                \cos \theta & \sin \theta \\ - \sin \theta & \cos \theta \\
            \end{bmatrix} \simeq \begin{bmatrix}
                1 & \theta \\ - \theta & 1 \\
            \end{bmatrix} ~,
        \end{equation*}
        and for the fields 
        \begin{equation*}
            \begin{bmatrix}
                {\varphi'}_1 \\ {\varphi'}_2\\
            \end{bmatrix} = \begin{bmatrix}
                1 & \theta \\ - \theta & 1 \\
            \end{bmatrix} \begin{bmatrix}
                \varphi_1 \\ \varphi_2 \\
            \end{bmatrix} 
        \end{equation*} 
        which implies an infinitesimal transformation of the fields
        \begin{equation*}
            \delta \varphi_1 = {\varphi'}_1 - \varphi_1 = \theta \varphi_2 ~, \quad \delta \varphi_2 = {\varphi'}_2 - \varphi_2 = - \theta \varphi_1 ~.
        \end{equation*}

        By the Noether's theorem, the conserved current~\eqref{conscurr} is 
        \begin{equation*}
            J^\mu = \underbrace{\pdv{\mathcal L}{\partial_\mu \varphi_i}}_{\partial^\mu \varphi_i} \delta \varphi_i = \partial^\mu \varphi_1 \delta \varphi_1 + \partial^\mu \varphi_2 \delta \varphi_2 = \theta ((\partial^\mu \varphi_1) \varphi_2 - (\partial^\mu \varphi_2) \varphi_1) ~,
        \end{equation*}
        where $K^\mu = 0$, and conserved charge is 
        \begin{equation*}
            Q = \int d^3 x ~ J^0 = \int d^3 x ~ ((\partial^0 \varphi_1) \varphi_2 - (\partial^0 \varphi_2) \varphi_1) = \int d^3 x ~ (\dot \varphi_1 \varphi_2 - \dot \varphi_2 \varphi_1)
        \end{equation*}
        where we have omitted a constant $\theta$. 

        Finally, we promote it to charge operator 
        \begin{equation*}
        \begin{aligned}
            \hat Q & = \int d^3 x ~ (\hat \pi_1 \hat \varphi_2 - \hat \pi_2 \hat \varphi_1) \\ & = \int d^3 x ~ \Big ( \int \frac{d^3 q}{{(2\pi)}^3} \Big ( - i \sqrt{\frac{\omega_{\mathbf q}}{2}} \Big ) \Big ( \hat a_{1, \mathbf q} \exp(i \mathbf q \cdot \mathbf x) - \hat a_{1, \mathbf q}^\dagger \exp(- i \mathbf q \cdot \mathbf x) \Big) \\ & \qquad \int \frac{d^3 p}{{(2\pi)}^3} \frac{1}{\sqrt{2 \omega_{\mathbf p}}} \Big ( \hat a_{2, \mathbf p} \exp(i \mathbf p \cdot \mathbf x) + \hat a_{2, \mathbf p}^\dagger \exp(- i \mathbf p \cdot \mathbf x) \Big) \\ & \qquad - \int \frac{d^3 q}{{(2\pi)}^3} \Big ( - i \sqrt{\frac{\omega_{\mathbf q}}{2}} \Big ) \Big ( \hat a_{2, \mathbf q} \exp(i \mathbf q \cdot \mathbf x) - \hat a_{2, \mathbf q}^\dagger \exp(- i \mathbf q \cdot \mathbf x) \Big) \\ & \qquad \int \frac{d^3 p}{{(2\pi)}^3} \frac{1}{\sqrt{2 \omega_{\mathbf p}}} \Big ( \hat a_{1, \mathbf p} \exp(i \mathbf p \cdot \mathbf x) + \hat a_{1, \mathbf p}^\dagger \exp(- i \mathbf p \cdot \mathbf x) \Big)  \Big) \\ & = - \frac{i}{2} \int \frac{d^3 x ~ d^3 p ~ d^3 q}{(2\pi)^6} \sqrt{\frac{\omega_{\mathbf q}}{\omega_{\mathbf p}}} \Big ( \hat a_{1, \mathbf q} \hat a_{2, \mathbf p} \underbrace{\exp(i (\mathbf p + \mathbf q) \cdot \mathbf x)}_{\delta^3 (\mathbf q + \mathbf p)} + \hat a_{1, \mathbf q} \hat a_{2, \mathbf p}^\dagger \underbrace{\exp(i (- \mathbf p + \mathbf q) \cdot \mathbf x)}_{\delta^3 (\mathbf q - \mathbf p)} \\ & \qquad - \hat a_{1, \mathbf q}^\dagger \hat a_{2, \mathbf p} \underbrace{\exp(i (\mathbf p - \mathbf q) \cdot \mathbf x)}_{\delta^3 (\mathbf q - \mathbf p)} - \hat a_{1, \mathbf q}^\dagger \hat a_{2, \mathbf p}^\dagger \underbrace{\exp(i (- \mathbf p - \mathbf q) \cdot \mathbf x)}_{\delta^3 (\mathbf q + \mathbf p)} \\ & \qquad - \hat a_{2, \mathbf q} \hat a_{1, \mathbf p} \underbrace{\exp(i (\mathbf p + \mathbf q) \cdot \mathbf x)}_{\delta^3 (\mathbf q + \mathbf p)} - \hat a_{2, \mathbf q} \hat a_{1, \mathbf p}^\dagger \underbrace{\exp(i (- \mathbf p + \mathbf q) \cdot \mathbf x)}_{\delta^3 (\mathbf q - \mathbf p)} \\ & \qquad + \hat a_{2, \mathbf q}^\dagger \hat a_{1, \mathbf p} \underbrace{\exp(i (\mathbf p - \mathbf q) \cdot \mathbf x)}_{\delta^3 (\mathbf q - \mathbf p)} + \hat a_{2, \mathbf q}^\dagger \hat a_{1, \mathbf p}^\dagger \underbrace{\exp(i (- \mathbf p - \mathbf q) \cdot \mathbf x)}_{\delta^3 (\mathbf q + \mathbf p)} \Big) 
        \end{aligned}
        \end{equation*}
        \begin{equation*}
        \begin{aligned}
            \phantom{\hat Q} & = - \frac{i}{2} \int \frac{d^3 p ~ d^3 q}{(2\pi)^3} \sqrt{\frac{\omega_{\mathbf q}}{\omega_{\mathbf p}}} \Big ( \hat a_{1, \mathbf q} \hat a_{2, \mathbf p} \underbrace{\delta^3 (\mathbf q + \mathbf p)}_{\mathbf q = - \mathbf p} + \hat a_{1, \mathbf q} \hat a_{2, \mathbf p}^\dagger \underbrace{\delta^3 (\mathbf q - \mathbf p)}_{\mathbf q = \mathbf p} \\ & \qquad - \hat a_{1, \mathbf q}^\dagger \hat a_{2, \mathbf p} \underbrace{\delta^3 (\mathbf q - \mathbf p)}_{\mathbf q = \mathbf p} - \hat a_{1, \mathbf q}^\dagger \hat a_{2, \mathbf p}^\dagger \underbrace{\delta^3 (\mathbf q + \mathbf p)}_{\mathbf q = - \mathbf p} \\ & \qquad - \hat a_{2, \mathbf q} \hat a_{1, \mathbf p} \underbrace{\delta^3 (\mathbf q + \mathbf p)}_{\mathbf q = - \mathbf p} - \hat a_{2, \mathbf q} \hat a_{1, \mathbf p}^\dagger \underbrace{\delta^3 (\mathbf q - \mathbf p)}_{\mathbf q = \mathbf p} \\ & \qquad + \hat a_{2, \mathbf q}^\dagger \hat a_{1, \mathbf p} \underbrace{\delta^3 (\mathbf q - \mathbf p)}_{\mathbf q = \mathbf p} + \hat a_{2, \mathbf q}^\dagger \hat a_{1, \mathbf p}^\dagger \underbrace{\delta^3 (\mathbf q + \mathbf p)}_{\mathbf q = - \mathbf p} \Big) \\ & = - \frac{i}{2} \int \frac{d^3 p}{(2\pi)^3} ( \hat a_{1, - \mathbf p} \hat a_{2, \mathbf p} + \hat a_{1, \mathbf p} \hat a_{2, \mathbf p}^\dagger - \hat a_{1, \mathbf p}^\dagger \hat a_{2, \mathbf p} - \hat a_{1, - \mathbf p}^\dagger \hat a_{2, \mathbf p}^\dagger \\ & \qquad - \hat a_{2, - \mathbf p} \hat a_{1, \mathbf p} - \hat a_{2, \mathbf p} \hat a_{1, \mathbf p}^\dagger  + \hat a_{2, \mathbf p}^\dagger \hat a_{1, \mathbf p} + \hat a_{2, - \mathbf p}^\dagger \hat a_{1, \mathbf p}^\dagger ) \\ & = - \frac{i}{2} \Big (\int \frac{d^3 p}{(2\pi)^3} (\hat a_{1, - \mathbf p} \hat a_{2, \mathbf p} - \hat a_{2, - \mathbf p} \hat a_{1, \mathbf p} ) + \int \frac{d^3 p}{(2\pi)^3} (\hat a_{2, - \mathbf p}^\dagger \hat a_{1, \mathbf p}^\dagger - \hat a_{1, - \mathbf p}^\dagger \hat a_{2, \mathbf p}^\dagger) \\ & \qquad + \int \frac{d^3 p}{(2\pi)^3}(\hat a_{1, \mathbf p} \hat a_{2, \mathbf p}^\dagger - \hat a_{1, \mathbf p}^\dagger \hat a_{2, \mathbf p} - \hat a_{2, \mathbf p} \hat a_{1, \mathbf p}^\dagger  + \hat a_{2, \mathbf p}^\dagger \hat a_{1, \mathbf p}) \Big) \\ & = - \frac{i}{2} \int \frac{d^3 p}{(2\pi)^3} (\hat a_{1, \mathbf p} \hat a_{2, \mathbf p}^\dagger - \underbrace{\hat a_{1, \mathbf p}^\dagger \hat a_{2, \mathbf p}}_{ \hat a_{2, \mathbf p} \hat a_{1, \mathbf p}^\dagger} - \hat a_{2, \mathbf p} \hat a_{1, \mathbf p}^\dagger + \underbrace{\hat a_{2, \mathbf p}^\dagger \hat a_{1, \mathbf p}}_{\hat a_{1, \mathbf p} \hat a_{2, \mathbf p}^\dagger} ) \\ & = - \frac{i}{2} \int \frac{d^3 p}{(2\pi)^3} (\hat a_{1, \mathbf p} \hat a_{2, \mathbf p}^\dagger - \hat a_{2, \mathbf p} \hat a_{1, \mathbf p}^\dagger - \hat a_{2, \mathbf p} \hat a_{1, \mathbf p}^\dagger + \hat a_{1, \mathbf p} \hat a_{2, \mathbf p}^\dagger ) \\ & = - i \int \frac{d^3 p}{(2\pi)^3} (\hat a_{1, \mathbf p} \hat a_{2, \mathbf p}^\dagger - \hat a_{2, \mathbf p} \hat a_{1, \mathbf p}^\dagger)
        \end{aligned}
        \end{equation*}
        where in the fourth row from end, the first two integrals vanish because they are odd functions since they commute. 
    \end{proof}
    It is hermitian 
    \begin{equation*}
        \hat Q^\dagger = \hat Q ~.
    \end{equation*}
    \begin{proof}
        Infact, 
        \begin{equation*}
        \begin{aligned}
            \hat Q^\dagger & = i \int \frac{d^3 p}{(2\pi)^3} ((\hat a_{1, \mathbf p} \hat a_{2, \mathbf p}^\dagger)^\dagger - (\hat a_{2, \mathbf p} \hat a_{1, \mathbf p}^\dagger)^\dagger) \\ & = i \int \frac{d^3 p}{(2\pi)^3} (\hat a_{2, \mathbf p} \hat a_{1, \mathbf p}^\dagger - \hat a_{1, \mathbf p} \hat a_{2, \mathbf p}^\dagger ) \\ & = - i \int \frac{d^3 p}{(2\pi)^3} (\hat a_{1, \mathbf p} \hat a_{2, \mathbf p}^\dagger - \hat a_{2, \mathbf p} \hat a_{1, \mathbf p}^\dagger) = \hat Q ~.
        \end{aligned}
        \end{equation*}
    \end{proof}
    It is conserved by the hamiltonian 
    \begin{equation*}
        [\hat Q, \hat H] = 0 ~.
    \end{equation*}
    \begin{proof}
        Infact, 
        \begin{equation*}
        \begin{aligned}
            [\hat Q, \hat H] & = [- i \int \frac{d^3 p}{(2\pi)^3} (\hat a_{1, \mathbf p} \hat a_{2, \mathbf p}^\dagger - \hat a_{2, \mathbf p} \hat a_{1, \mathbf p}^\dagger), \int \frac{d^3 q}{(2\pi)^3} (\omega_{1, \mathbf q} \hat a_{1, \mathbf q}^\dagger \hat a_{1, \mathbf q} + \omega_{2, \mathbf q} \hat a_{2, \mathbf q}^\dagger \hat a_{2, \mathbf q})] \\ & = - i \int \frac{d^3 p ~ d^3 q}{(2\pi)^6} [\hat a_{1, \mathbf p} \hat a_{2, \mathbf p}^\dagger - \hat a_{2, \mathbf p} \hat a_{1, \mathbf p}^\dagger, \omega_{1, \mathbf q} \hat a_{1, \mathbf q}^\dagger \hat a_{1, \mathbf q} + \omega_{2, \mathbf q} \hat a_{2, \mathbf q}^\dagger \hat a_{2, \mathbf q}] \\ & = - i \int \frac{d^3 p ~ d^3 q}{(2\pi)^6} (\omega_{1, \mathbf q} [\hat a_{1, \mathbf p} \hat a_{2, \mathbf p}^\dagger, \hat a_{1, \mathbf q}^\dagger \hat a_{1, \mathbf q}] - \omega_{1, \mathbf q} [\hat a_{2, \mathbf p} \hat a_{1, \mathbf p}^\dagger, \hat a_{1, \mathbf q}^\dagger \hat a_{1, \mathbf q}] \\ & \qquad + \omega_{2, \mathbf q} [\hat a_{1, \mathbf p} \hat a_{2, \mathbf p}^\dagger, \hat a_{2, \mathbf q}^\dagger \hat a_{2, \mathbf q}] - \omega_{2, \mathbf q} [\hat a_{2, \mathbf p} \hat a_{1, \mathbf p}^\dagger, \hat a_{2, \mathbf q}^\dagger \hat a_{2, \mathbf q}]) \\ & = - i \int \frac{d^3 p ~ d^3 q}{(2\pi)^6} (\omega_{1, \mathbf q} \hat a_{1, \mathbf p} [\hat a_{2, \mathbf p}^\dagger, \hat a_{1, \mathbf q}^\dagger \hat a_{1, \mathbf q}] + \omega_{1, \mathbf q} [\hat a_{1, \mathbf p}, \hat a_{1, \mathbf q}^\dagger \hat a_{1, \mathbf q}] \hat a_{2, \mathbf p}^\dagger \\ & \qquad - \omega_{1, \mathbf q} \hat a_{2, \mathbf p} [\hat a_{1, \mathbf p}^\dagger, \hat a_{1, \mathbf q}^\dagger \hat a_{1, \mathbf q}] - \omega_{1, \mathbf q} [\hat a_{2, \mathbf p}, \hat a_{1, \mathbf q}^\dagger \hat a_{1, \mathbf q}]  \hat a_{1, \mathbf p}^\dagger \\ & \qquad + \omega_{2, \mathbf q} \hat a_{1, \mathbf p} [\hat a_{2, \mathbf p}^\dagger, \hat a_{2, \mathbf q}^\dagger \hat a_{2, \mathbf q}] + \omega_{2, \mathbf q}  [\hat a_{1, \mathbf p}, \hat a_{2, \mathbf q}^\dagger \hat a_{2, \mathbf q}] \hat a_{2, \mathbf p}^\dagger \\ & \qquad - \omega_{2, \mathbf q} \hat a_{2, \mathbf p} [\hat a_{1, \mathbf p}^\dagger, \hat a_{2, \mathbf q}^\dagger \hat a_{2, \mathbf q}] - \omega_{2, \mathbf q} [\hat a_{2, \mathbf p}, \hat a_{2, \mathbf q}^\dagger \hat a_{2, \mathbf q}] \hat a_{1, \mathbf p}^\dagger) 
        \end{aligned}
        \end{equation*}
        \begin{equation*}
        \begin{aligned}
            & = - i \int \frac{d^3 p ~ d^3 q}{(2\pi)^6} ( \omega_{1, \mathbf q} \hat a_{1, \mathbf p} \hat a_{1, \mathbf q}^\dagger \underbrace{[\hat a_{2, \mathbf p}^\dagger, \hat a_{1, \mathbf q}]}_0 + \omega_{1, \mathbf q} \hat a_{1, \mathbf p} \underbrace{[\hat a_{2, \mathbf p}^\dagger, \hat a_{1, \mathbf q}^\dagger]}_0 \hat a_{1, \mathbf q} \\ & \qquad + \omega_{1, \mathbf q} \hat a_{1, \mathbf q}^\dagger \underbrace{[\hat a_{1, \mathbf p}, \hat a_{1, \mathbf q}]}_0 \hat a_{2, \mathbf p}^\dagger + \omega_{1, \mathbf q} \underbrace{[\hat a_{1, \mathbf p}, \hat a_{1, \mathbf q}^\dagger]}_{(2\pi)^3 \delta^3 (\mathbf p - \mathbf q)} \hat a_{1, \mathbf q} \hat a_{2, \mathbf p}^\dagger \\ & \qquad - \omega_{1, \mathbf q} \hat a_{2, \mathbf p} \hat a_{1, \mathbf q}^\dagger  \underbrace{[\hat a_{1, \mathbf p}^\dagger, \hat a_{1, \mathbf q}]}_{- (2 \pi)^3 \delta^3 (\mathbf p - \mathbf q)} - \omega_{1, \mathbf q} \hat a_{2, \mathbf p} \underbrace{[\hat a_{1, \mathbf p}^\dagger, \hat a_{1, \mathbf q}^\dagger ]}_0 \hat a_{1, \mathbf q} \\ & \qquad - \omega_{1, \mathbf q} \hat a_{1, \mathbf q}^\dagger \underbrace{[\hat a_{2, \mathbf p}, \hat a_{1, \mathbf q}]}_0  \hat a_{1, \mathbf p}^\dagger - \omega_{1, \mathbf q} \underbrace{[\hat a_{2, \mathbf p}, \hat a_{1, \mathbf q}^\dagger]}_0 \hat a_{1, \mathbf q} \hat a_{1, \mathbf p}^\dagger  \\ & \qquad + \omega_{2, \mathbf q} \hat a_{1, \mathbf p} \hat a_{2, \mathbf q}^\dagger \underbrace{[\hat a_{2, \mathbf p}^\dagger, \hat a_{2, \mathbf q}]}_{-(2\pi)^3 \delta^3 (\mathbf p - \mathbf q)} + \omega_{2, \mathbf q} \hat a_{1, \mathbf p} \underbrace{[\hat a_{2, \mathbf p}^\dagger, \hat a_{2, \mathbf q}^\dagger]}_0 \hat a_{2, \mathbf q} \\ & \qquad + \omega_{2, \mathbf q} \hat a_{2, \mathbf q}^\dagger \underbrace{[\hat a_{1, \mathbf p}, \hat a_{2, \mathbf q}]}_0 \hat a_{2, \mathbf p}^\dagger + \omega_{2, \mathbf q}  \underbrace{[\hat a_{1, \mathbf p}, \hat a_{2, \mathbf q}^\dagger]}_0 \hat a_{2, \mathbf q} \hat a_{2, \mathbf p}^\dagger \\ & \qquad - \omega_{2, \mathbf q} \hat a_{2, \mathbf p} \hat a_{2, \mathbf q}^\dagger \underbrace{[\hat a_{1, \mathbf p}^\dagger, \hat a_{2, \mathbf q}]}_0 - \omega_{2, \mathbf q} \hat a_{2, \mathbf p} \underbrace{[\hat a_{1, \mathbf p}^\dagger, \hat a_{2, \mathbf q}^\dagger ]}_0 \hat a_{2, \mathbf q} \\ & \qquad - \omega_{2, \mathbf q} \hat a_{2, \mathbf q}^\dagger \underbrace{[\hat a_{2, \mathbf p}, \hat a_{2, \mathbf q}] }_0 \hat a_{1, \mathbf p}^\dagger - \omega_{2, \mathbf q} \underbrace{[\hat a_{2, \mathbf p}, \hat a_{2, \mathbf q}^\dagger]}_{(2\pi)^3 \delta^3 (\mathbf p - \mathbf q)} \hat a_{2, \mathbf q} \hat a_{1, \mathbf p}^\dagger) \\ & = - i \int \frac{d^3 p ~ d^3 q}{(2\pi)^6} ( \omega_{1, \mathbf q} \underbrace{[\hat a_{1, \mathbf p}, \hat a_{1, \mathbf q}^\dagger]}_{(2\pi)^3 \delta^3 (\mathbf p - \mathbf q)} \hat a_{1, \mathbf q} \hat a_{2, \mathbf p}^\dagger - \omega_{1, \mathbf q} \hat a_{2, \mathbf p} \hat a_{1, \mathbf q}^\dagger  \underbrace{[\hat a_{1, \mathbf p}^\dagger, \hat a_{1, \mathbf q}]}_{- (2 \pi)^3 \delta^3 (\mathbf p - \mathbf q)} \\ & \qquad + \omega_{2, \mathbf q} \hat a_{1, \mathbf p} \hat a_{2, \mathbf q}^\dagger \underbrace{[\hat a_{2, \mathbf p}^\dagger, \hat a_{2, \mathbf q}]}_{-(2\pi)^3 \delta^3 (\mathbf p - \mathbf q)} - \omega_{2, \mathbf q} \underbrace{[\hat a_{2, \mathbf p}, \hat a_{2, \mathbf q}^\dagger]}_{(2\pi)^3 \delta^3 (\mathbf p - \mathbf q)} \hat a_{2, \mathbf q} \hat a_{1, \mathbf p}^\dagger) \\ & = - i \int \frac{d^3 p ~ d^3 q}{(2\pi)^3} ( \omega_{1, \mathbf q} \underbrace{\delta^3 (\mathbf p - \mathbf q)}_{\mathbf q = \mathbf p} \hat a_{1, \mathbf q} \hat a_{2, \mathbf p}^\dagger + \omega_{1, \mathbf q} \hat a_{2, \mathbf p} \hat a_{1, \mathbf q}^\dagger \underbrace{\delta^3 (\mathbf p - \mathbf q)}_{\mathbf q = \mathbf p} \\ & \qquad - \omega_{2, \mathbf q} \hat a_{1, \mathbf p} \hat a_{2, \mathbf q}^\dagger \underbrace{\delta^3 (\mathbf p - \mathbf q)}_{\mathbf q = \mathbf p} - \omega_{2, \mathbf q} \underbrace{\delta^3 (\mathbf p - \mathbf q)}_{\mathbf q = \mathbf p} \hat a_{2, \mathbf q} \hat a_{1, \mathbf p}^\dagger) \\ & = - i \int \frac{d^3 p}{(2\pi)^3} ( \omega_{1, \mathbf p} \hat a_{1, \mathbf p} \hat a_{2, \mathbf p}^\dagger + \omega_{1, \mathbf p} \hat a_{2, \mathbf p} \hat a_{1, \mathbf p}^\dagger - \omega_{2, \mathbf p} \hat a_{1, \mathbf p} \hat a_{2, \mathbf p}^\dagger - \omega_{2, \mathbf p} \hat a_{2, \mathbf p} \hat a_{1, \mathbf p}^\dagger)
        \end{aligned}
        \end{equation*}
    \end{proof}
    
    Notice that the definition of the charge operator is ambiguous, since we may define a new charge operator 
    \begin{equation*}
        \hat Q' = c_1 \hat Q + c_2 ~,
    \end{equation*}
    where $c_1, c_2 \in \mathbb R$ and it still satisfy the conservation law (in the Heisenberg picture)
    \begin{equation*}
        \dv{\hat Q'}{t} = 0 ~.
    \end{equation*}
    \begin{proof}
        Infact 
        \begin{equation*}
            \dv{\hat Q'}{t} = \dv{c_1 Q + c_2}{t} = c_1 \dv{Q}{t} = 0 ~.
        \end{equation*}
    \end{proof}
    The ambiguity of $c_1$ can be used to set the units, while the one of $c_2$ can be exploit to go in normal ordering. Infact, setting $c_1 = 1$ and using
    \begin{equation*}
        \hat Q \ket{0} = 0 ~,
    \end{equation*}
    we obtain 
    \begin{equation*}
        \hat Q' = c_2 ~.
    \end{equation*}
    \begin{proof}
        Infact 
        \begin{equation}
            \hat Q' \ket{0} = \hat Q \ket{0} + c_2 \ket{0} = c_2 \ket{0} ~,
        \end{equation}
        \begin{equation*}
            \bra{0} \hat Q' \ket{0} = c_2 \braket{0}{0} = c_2
        \end{equation*}
        and 
        \begin{equation*}
            \colon \hat Q' \colon = \hat Q' - \bra{0} \hat Q' \ket{0} = \hat Q + c_2 - c_2 = \hat Q ~.
        \end{equation*}
    \end{proof}

    We define new ladder operators
    \begin{equation*}
        \hat a_{\pm, \mathbf p} = \frac{\hat a_{1, \mathbf p} \pm i \hat a_{2, \mathbf p}}{\sqrt{2}} ~, \quad \hat a_{\pm, \mathbf p}^\dagger = \frac{\hat a_{1, \mathbf p}^\dagger \mp i \hat a_{2, \mathbf p}^\dagger}{\sqrt{2}} ~,
    \end{equation*}
    such that they satisfy 
    \begin{equation*}
        [\hat Q, \hat a_{\pm, \mathbf p}] = \mp \hat a_{\pm, \mathbf p} ~, \quad [\hat Q, \hat a_{\pm, \mathbf p}^\dagger] = \pm \hat a_{\pm, \mathbf p}^\dagger ~.
    \end{equation*}
    \begin{proof}
        For the annihilation commutator 
        \begin{equation*}
        \begin{aligned}
            [\hat Q, \hat a_{\pm, \mathbf p}] & = \frac{[\hat Q, \hat a_{1, \mathbf p}] \pm i [\hat Q, \hat a_{2, \mathbf p}]}{\sqrt{2}} \\ & = - \frac{i}{\sqrt{2}} \int \frac{d^3 q}{(2\pi)^3} ( [\hat a_{1, \mathbf q} \hat a_{2, \mathbf q}^\dagger - \hat a_{2, \mathbf q} \hat a_{1, \mathbf q}^\dagger, \hat a_{1, \mathbf p}]  \pm i  [\hat a_{1, \mathbf q} \hat a_{2, \mathbf q}^\dagger - \hat a_{2, \mathbf q} \hat a_{1, \mathbf q}^\dagger, \hat a_{2, \mathbf p}]) \\ & = - \frac{i}{\sqrt{2}} \int \frac{d^3 q}{(2\pi)^3} ( [\hat a_{1, \mathbf q} \hat a_{2, \mathbf q}^\dagger, \hat a_{1, \mathbf p}] - [\hat a_{2, \mathbf q} \hat a_{1, \mathbf q}^\dagger, \hat a_{1, \mathbf p}] \pm i [\hat a_{1, \mathbf q} \hat a_{2, \mathbf q}^\dagger, \hat a_{2, \mathbf p}] \mp i [\hat a_{2, \mathbf q} \hat a_{1, \mathbf q}^\dagger, \hat a_{2, \mathbf p}]) \\ & = - \frac{i}{\sqrt{2}} \int \frac{d^3 q}{(2\pi)^3} ( \hat a_{1, \mathbf q} \underbrace{[\hat a_{2, \mathbf q}^\dagger, \hat a_{1, \mathbf p}]}_0 + \underbrace{[\hat a_{1, \mathbf q}, \hat a_{1, \mathbf p}]}_0 \hat a_{2, \mathbf q}^\dagger - \hat a_{2, \mathbf q}  \underbrace{[\hat a_{1, \mathbf q}^\dagger, \hat a_{1, \mathbf p}]}_{- (2\pi) \delta^3 (\mathbf p - \mathbf q)} - \underbrace{[\hat a_{2, \mathbf q}, \hat a_{1, \mathbf p}]}_0 \hat a_{1, \mathbf q}^\dagger \\ & \qquad \pm i \hat a_{1, \mathbf q} \underbrace{[\hat a_{2, \mathbf q}^\dagger, \hat a_{2, \mathbf p}]}_{- (2\pi)^3 \delta^3 (\mathbf p - \mathbf q)} \pm i \underbrace{[\hat a_{1, \mathbf q} , \hat a_{2, \mathbf p}]}_0 \hat a_{2, \mathbf q}^\dagger \mp i \hat a_{2, \mathbf q} \underbrace{[ \hat a_{1, \mathbf q}^\dagger, \hat a_{2, \mathbf p}]}_0 \mp i \underbrace{[\hat a_{2, \mathbf q}, \hat a_{2, \mathbf p}]}_0 \hat a_{1, \mathbf q}^\dagger) \\ & = - \frac{i}{\sqrt{2}} \int d^3 q ~ ( \hat a_{2, \mathbf q} \underbrace{\delta^3 (\mathbf p - \mathbf q)}_{\mathbf p = \mathbf q} \mp i \hat a_{1, \mathbf q} \underbrace{\delta^3 (\mathbf p - \mathbf q)}_{\mathbf p = \mathbf q} ) \\ & = - \frac{i}{\sqrt{2}} \hat a_{2, \mathbf p} - \frac{i}{\sqrt{2}} (\mp i \hat a_{1, \mathbf p}) \\ & = \frac{\mp \hat a_{1, \mathbf p} - i \hat a_{2, \mathbf p}}{\sqrt{2}} \\ & = \mp \frac{\hat a_{1, \mathbf p} \pm i \hat a_{2, \mathbf p}}{\sqrt{2}} = \mp \hat a_{\pm, \mathbf p} ~.
        \end{aligned}
        \end{equation*}

        For the creation commutator 
        \begin{equation*}
        \begin{aligned}
            [\hat Q, \hat a_{\pm, \mathbf p}^\dagger] & = \frac{[\hat Q, \hat a_{1, \mathbf p}^\dagger] \mp i [\hat Q, \hat a_{2, \mathbf p}^\dagger]}{\sqrt{2}} \\ & = - \frac{i}{\sqrt{2}} \int \frac{d^3 q}{(2\pi)^3} ( [\hat a_{1, \mathbf q} \hat a_{2, \mathbf q}^\dagger - \hat a_{2, \mathbf q} \hat a_{1, \mathbf q}^\dagger, \hat a_{1, \mathbf p}^\dagger]  \mp i  [\hat a_{1, \mathbf q} \hat a_{2, \mathbf q}^\dagger - \hat a_{2, \mathbf q} \hat a_{1, \mathbf q}^\dagger, \hat a_{2, \mathbf p}^\dagger]) \\ & = - \frac{i}{\sqrt{2}} \int \frac{d^3 q}{(2\pi)^3} ( [\hat a_{1, \mathbf q} \hat a_{2, \mathbf q}^\dagger, \hat a_{1, \mathbf p}^\dagger] - [\hat a_{2, \mathbf q} \hat a_{1, \mathbf q}^\dagger, \hat a_{1, \mathbf p}^\dagger] \mp i [\hat a_{1, \mathbf q} \hat a_{2, \mathbf q}^\dagger, \hat a_{2, \mathbf p}^\dagger] \pm i [\hat a_{2, \mathbf q} \hat a_{1, \mathbf q}^\dagger, \hat a_{2, \mathbf p}^\dagger]) \\ & = - \frac{i}{\sqrt{2}} \int \frac{d^3 q}{(2\pi)^3} ( \hat a_{1, \mathbf q} \underbrace{[\hat a_{2, \mathbf q}^\dagger, \hat a_{1, \mathbf p}^\dagger]}_0 + \underbrace{[\hat a_{1, \mathbf q}, \hat a_{1, \mathbf p}^\dagger]}_{(2\pi)^3 \delta^3 (\mathbf q - \mathbf p)} \hat a_{2, \mathbf q}^\dagger - \hat a_{2, \mathbf q}  \underbrace{[\hat a_{1, \mathbf q}^\dagger, \hat a_{1, \mathbf p}^\dagger]}_0 - \underbrace{[\hat a_{2, \mathbf q}, \hat a_{1, \mathbf p}]}_0 \hat a_{1, \mathbf q}^\dagger \\ & \qquad \mp i \hat a_{1, \mathbf q} \underbrace{[\hat a_{2, \mathbf q}^\dagger, \hat a_{2, \mathbf p}^\dagger]}_0 \mp i \underbrace{[\hat a_{1, \mathbf q} , \hat a_{2, \mathbf p}]}_0 \hat a_{2, \mathbf q}^\dagger \pm i \hat a_{2, \mathbf q} \underbrace{[ \hat a_{1, \mathbf q}^\dagger, \hat a_{2, \mathbf p}]}_0 \pm i \underbrace{[\hat a_{2, \mathbf q}, \hat a_{2, \mathbf p}^\dagger]}_{(2\pi)^3 \delta^3 (\mathbf p - \mathbf q)} \hat a_{1, \mathbf q}^\dagger) \\ & = - \frac{i}{\sqrt{2}} \int d^3 q ~ ( \underbrace{\delta^3 (\mathbf q - \mathbf p)}_{\mathbf p = \mathbf q} \hat a_{2, \mathbf q}^\dagger \pm i \underbrace{\delta^3 (\mathbf p - \mathbf q)}_{\mathbf p = \mathbf q} \hat a_{1, \mathbf q}^\dagger ) \\ & = - \frac{i}{\sqrt{2}} \hat a_{2, \mathbf p}^\dagger \pm \frac{1}{\sqrt{2}} \hat a_{1, \mathbf p}^\dagger \\ & = \pm \frac{\hat a_{1, \mathbf p}^\dagger \mp i \hat a_{2, \mathbf p}^\dagger}{\sqrt{2}} = \pm \hat a_{\pm, \mathbf p}^\dagger ~.
        \end{aligned}
        \end{equation*}
    \end{proof}

    Now, we study the spectrum of $\hat Q$. We define the eigenstate of the charge operators as 
    \begin{equation*}
        \hat Q \ket{S} = q \ket{S} ~,
    \end{equation*}
    where $q$ is the charge. The action of the ladder operators~\eqref{ladd} is to add or subtract a unit of charge in the system
    \begin{equation*}
        \hat Q \hat a_{\pm, \mathbf p}^\dagger \ket{S} = (q \pm 1) \hat a_{\pm, \mathbf p}^\dagger ~.
    \end{equation*}
    \begin{proof}
        Infact, 
        \begin{equation*}
            \underbrace{\hat Q \hat a_{\pm, \mathbf p}^\dagger}_{[\hat Q \hat,  a_{\pm, \mathbf p}^\dagger] + a_{\pm, \mathbf p}^\dagger \hat Q } \ket{S} = \underbrace{[\hat Q \hat,  a_{\pm, \mathbf p}^\dagger]}_{\pm a_{\pm, \mathbf p}^\dagger} \ket{S} + a_{\pm, \mathbf p}^\dagger \hat Q \ket{S} = \pm a_{\pm, \mathbf p}^\dagger \ket{S} + a_{\pm, \mathbf p}^\dagger \underbrace{\hat Q \ket{S}}_{q }\ket{S} = (q \pm 1) \hat a_{\pm, \mathbf p}^\dagger ~.
        \end{equation*}
    \end{proof}

    Since $\hat Q$ commute with $\hat H$ and $\hat{\mathbf P}$ and the three operators are linear combinations of the ladder operators, $\ket{S}$ is a common eigenstate of them. Consider a system of $n$ particles such that they have charge $\pm q$. Therefore the common eigenstates $\ket{S_n^\pm}$ are defined by 
    \begin{equation*}
        \hat Q \ket{0} = 0 ~, \quad \ket{S_n^\pm} = \prod_{i=1}^n \hat a_{\pm, \mathbf p}^\dagger \ket{0} ~,
    \end{equation*}
    where $\quad \ket{S_n^+}$ corresponds to $n$ positively-charge particles and $\quad \ket{S_n^-}$ corresponds to $n$ negatively-charge particles, such that they satisfy the properties
    \begin{equation*}
        \hat H \ket{S_n^\pm} = \Big (\sum_{i = 1}^{n} \omega_{\mathbf p_i} \Big ) \ket{S_n^\pm} ~, \quad \hat{\mathbf P} \ket{S_n^\pm} = \Big (\sum_{i = 1}^{n} \mathbf p_i \Big ) \ket{S_n^\pm} ~, \quad \hat N \ket{S_n^\pm} = n \ket{S_n^\pm}
    \end{equation*}
    and 
    \begin{equation*}
        \hat Q \ket{S_n^\pm} = \pm n \ket{S_n^\pm} ~.
    \end{equation*}
    This means that a particle states is characterised by its energy, momentum and charge eigestate. Furthermore, notice that the last espression give us the physical interpretation of the charge operator: $q$ is indeed the electric charge such that positively-charged states are particles and negatively-charged states are antiparticles. To be more precise, we could allow all also all the linear combination between them, e.g. the first is $q$, the second is $-q$, etc.

    However, a single Klein-Gordon field can describe only chargeless particles, since for $\varphi_1 = \varphi_2$ we have 
    \begin{equation*}
        \hat Q = 0 ~.
    \end{equation*}
    Hence, you need at least two degrees of freedom to descrive particles and antiparticles with non-zero electric charge.
    \begin{proof}
        Infact for $\varphi_1 = \varphi_2 = \varphi$
        \begin{equation*}
            Q = \int d^3 x ~ (\dot \varphi_1 \varphi_2 - \dot \varphi_2 \varphi_1) = \int d^3 x ~ (\dot \varphi^2 - \dot \varphi^2) = 0 ~.
        \end{equation*}
    \end{proof}

\section{Complex Klein-Gordon field}

    The description of two real Klein-Gordon fields is equivalent to a complex Klein-Gordon field, since the degrees of freedom are still two. For a the latter, they are 
    \begin{equation*}
        \varphi = \frac{\varphi_1 + i \varphi_2}{\sqrt{2}} ~, \quad \varphi^* = \frac{\varphi_1 - i \varphi_2}{\sqrt{2}} ~
    \end{equation*}
    and the corresponding lagrangian is 
    \begin{equation*}
        \mathcal L = \partial_\mu \varphi^* \partial^\mu \varphi - m^2 \varphi^* \varphi ~.
    \end{equation*}
    \begin{proof}
        Infact 
        \begin{equation*}
        \begin{aligned}
            \mathcal L & = \partial_\mu \varphi^* \partial^\mu \varphi - m^2 \varphi^* \varphi \\ & = \partial_\mu \frac{\varphi_1 - i \varphi_2}{\sqrt{2}} \partial^\mu \frac{\varphi_1 + i \varphi_2}{\sqrt{2}} - m^2 \frac{\varphi_1 - i \varphi_2}{\sqrt{2}} \frac{\varphi_1 + i \varphi_2}{\sqrt{2}} \\ & = \frac{1}{2} \partial_\mu \varphi_1 \partial^\mu \varphi_2 - \frac{1}{2} m^2 \varphi_1 \varphi_2 ~.
        \end{aligned}
        \end{equation*}
    \end{proof}

\section{Electric charge via $U(1)$ symmetry}

    This lagrangian is invariant with an $U(1)$ rotation (which is equivalent to an $O(2)$ rotation) and the Noether's theorem allows us to define a charge operator
    \begin{equation*}
        \hat Q = \int \frac{d^3 p}{(2\pi)^3} (\hat a_{+, \mathbf p}^\dagger \hat a_{+, \mathbf p} - \hat a_{-, \mathbf p}^\dagger \hat a_{-, \mathbf p}) = \hat N_+ - \hat N_- ~,
    \end{equation*}
    where we have used normal ordering and the number operators are 
    \begin{equation*}
        \hat N_\pm = \int \frac{d^3 p}{(2\pi)^3} \hat a_{\pm, \mathbf p}^\dagger \hat a_{\pm, \mathbf p} ~.
    \end{equation*}
    This can be seen by the definition of the field operator 
    \begin{equation}\label{fieldk}
        \hat \varphi (\mathbf x) = \int \frac{d^3 p}{(2\pi)^3} \frac{1}{\sqrt{2 \omega_{\mathbf p}}} \Big ( \hat a_{+, \mathbf p} \exp(i \mathbf p \cdot \mathbf x) + \hat a_{-, \mathbf p}^\dagger \exp(- i \mathbf p \cdot \mathbf x) \Big)
    \end{equation}
    and the conjugate field operator 
    \begin{equation}\label{conj}
        \hat \varphi^* (\mathbf x) = \int \frac{d^3 p}{(2\pi)^3} \frac{1}{\sqrt{2 \omega_{\mathbf p}}} \Big ( \hat a_{-, \mathbf p} \exp(i \mathbf p \cdot \mathbf x) + \hat a_{+, \mathbf p}^\dagger \exp(- i \mathbf p \cdot \mathbf x) \Big) ~,
    \end{equation}
    where $\hat a_{+, \mathbf p}^\dagger$ create a particle with momentum $\mathbf p$ and energy $\omega_{\mathbf p}$, $\hat a_{-, \mathbf p}^\dagger$ create an antiparticle with momentum $\mathbf p$ and energy $\omega_{\mathbf p}$, $\hat a_{+, \mathbf p}$ destroys a particle with momentum $\mathbf p$ and energy $\omega_{\mathbf p}$ and $\hat a_{-, \mathbf p}$ destroys an antiparticle with momentum $\mathbf p$ and energy $\omega_{\mathbf p}$.
    \begin{proof}
        The lagrangian is invariant under a global $U(1)$ rotation. Infact, for a rotation 
        \begin{equation*}
            \varphi' = \exp(i \theta) \varphi ~, \quad {\varphi'}^* = \exp(- i \theta) \varphi^* ~,
        \end{equation*}
        we have 
        \begin{equation*}
        \begin{aligned}
            \mathcal L' & = (\partial_\mu (\varphi')^*) (\partial^\mu \varphi') - m^2 (\varphi')^* \varphi' \\ & = (\partial_\mu \varphi^*) \cancel{\exp(- i \theta)} \cancel{\exp(i \theta)} (\partial^\mu \varphi) - m^2 \varphi^* \cancel{\exp(- i \theta)} \cancel{\exp(i \theta)} \varphi \\ & = \partial_\mu \varphi^* \partial^\mu \varphi - m^2 \varphi^* \varphi = \mathcal L ~.
        \end{aligned}
        \end{equation*}
       
        Now, we compute the conserved current by considering an infinitesimal transformation for the fields 
        \begin{equation*}
            \varphi' = \exp(i \theta) \varphi \simeq \varphi + i\theta \varphi ~, \quad {\varphi'}^* = \exp(- i \theta) \varphi^* \simeq \varphi^* - i \theta \varphi^* ~,
        \end{equation*} 
        which implies an infinitesimal transformation of the fields
        \begin{equation*}
            \delta \varphi = {\varphi'} - \varphi = i \theta \varphi ~, \quad \delta \varphi^* = {\varphi'}^* - \varphi^* = - \theta \varphi^* ~.
        \end{equation*}

        By the Noether's theorem, the conserved current~\eqref{conscurr} is 
        \begin{equation*}
            J^\mu = \underbrace{\pdv{\mathcal L}{\partial_\mu \varphi_i}}_{\partial^\mu \varphi_i} \delta \varphi_i = \partial^\mu \varphi \delta \varphi + \partial^\mu \varphi^* \delta \varphi^* = i \theta ((\partial^\mu \varphi^*) \varphi - (\partial^\mu \varphi) \varphi^*) ~,
        \end{equation*}
        where $K^\mu = 0$, and conserved charge is 
        \begin{equation*}
            Q = \int d^3 x ~ J^0 = i \int d^3 x ~ ((\partial^0 \varphi^*) \varphi - (\partial^0 \varphi) \varphi^*) = i\int d^3 x ~ (\dot \varphi^* \varphi - \dot \varphi \varphi^*)
        \end{equation*}
        where we have omitted a constant $\theta$. 

        Hence, the charge is 
        \begin{equation*}   
            Q = i \int d^3 x ~(\varphi \dot \varphi^* - \varphi^* \dot \varphi) ~.
        \end{equation*}

        Now, we find the field operator, using~\eqref{kgfop} 
        \begin{equation*}
        \begin{aligned}
            \varphi & = \frac{\varphi_1 + i \varphi_2}{\sqrt{2}} \\ & = \frac{1}{\sqrt{2}} \Big (\int \frac{d^3 p}{{(2\pi)}^3} \frac{1}{\sqrt{2 \omega_{\mathbf p}}} \Big (\hat a_{1, \mathbf p} \exp(i \mathbf p \cdot \mathbf x) + \hat a_{1, \mathbf p}^\dagger \exp(- i \mathbf p \cdot \mathbf x) \Big) \\ & \qquad + i \int \frac{d^3 p}{{(2\pi)}^3} \frac{1}{\sqrt{2 \omega_{\mathbf p}}} \Big (\hat a_{2, \mathbf p} \exp(i \mathbf p \cdot \mathbf x) + \hat a_{2, \mathbf p}^\dagger \exp(- i \mathbf p \cdot \mathbf x) \Big) \Big) \\ & = \int \frac{d^3 p}{{(2\pi)}^3} \frac{1}{\sqrt{2 \omega_{\mathbf p}}} \Big ( \underbrace{\frac{\hat a_{1, \mathbf p} + i \hat a_{2, \mathbf p}}{\sqrt{2}}}_{\hat a_{+, \mathbf p}} \exp(i \mathbf p \cdot \mathbf x) + \underbrace{\frac{\hat a_{1, \mathbf p}^\dagger + i \hat a_{2, \mathbf p}^\dagger}{\sqrt{2}}}_{\hat a_{-, \mathbf p}^\dagger} \exp(- i \mathbf p \cdot \mathbf x) \Big) \\ & = \int \frac{d^3 p}{{(2\pi)}^3} \frac{1}{\sqrt{2 \omega_{\mathbf p}}} \Big ( \hat a_{+, \mathbf p} \exp(i \mathbf p \cdot \mathbf x) + \hat a_{-, \mathbf p}^\dagger \exp(- i \mathbf p \cdot \mathbf x) \Big) 
        \end{aligned}
        \end{equation*}
        and the complex conjugate field operator is
        \begin{equation*}
        \begin{aligned}
            \varphi^* & = \frac{\varphi_1 - i \varphi_2}{\sqrt{2}} \\ & = \frac{1}{\sqrt{2}} \Big (\int \frac{d^3 p}{{(2\pi)}^3} \frac{1}{\sqrt{2 \omega_{\mathbf p}}} \Big (\hat a_{1, \mathbf p} \exp(i \mathbf p \cdot \mathbf x) + \hat a_{1, \mathbf p}^\dagger \exp(- i \mathbf p \cdot \mathbf x) \Big) \\ & \qquad - i \int \frac{d^3 p}{{(2\pi)}^3} \frac{1}{\sqrt{2 \omega_{\mathbf p}}} \Big (\hat a_{2, \mathbf p} \exp(i \mathbf p \cdot \mathbf x) + \hat a_{2, \mathbf p}^\dagger \exp(- i \mathbf p \cdot \mathbf x) \Big) \Big) \\ & = \int \frac{d^3 p}{{(2\pi)}^3} \frac{1}{\sqrt{2 \omega_{\mathbf p}}} \Big ( \underbrace{\frac{\hat a_{1, \mathbf p} - i \hat a_{2, \mathbf p}}{\sqrt{2}}}_{\hat a_{-, \mathbf p}} \exp(i \mathbf p \cdot \mathbf x) + \underbrace{\frac{\hat a_{1, \mathbf p}^\dagger - i \hat a_{2, \mathbf p}^\dagger}{\sqrt{2}}}_{\hat a_{+, \mathbf p}^\dagger} \exp(- i \mathbf p \cdot \mathbf x) \Big) \\ & = \int \frac{d^3 p}{{(2\pi)}^3} \frac{1}{\sqrt{2 \omega_{\mathbf p}}} \Big ( \hat a_{-, \mathbf p} \exp(i \mathbf p \cdot \mathbf x) + \hat a_{+, \mathbf p}^\dagger \exp(- i \mathbf p \cdot \mathbf x) \Big)  ~.
        \end{aligned}
        \end{equation*}
        Furthermore, the conjugate field is 
        \begin{equation*}
            \pi = \pdv{\mathcal L}{\dot \varphi} = \dot \varphi^*
        \end{equation*}
        and the complex conjugate of the conjugate field is 
        \begin{equation*}
            \pi^* = \pdv{\mathcal L}{\dot \varphi^*} = \dot \varphi ~.
        \end{equation*}
        Hence 
        \begin{equation*}
        \begin{aligned}
            \pi^* & = \frac{\pi_1 + i \pi_2}{\sqrt{2}} \\ & = \frac{1}{\sqrt{2}} \Big (\int \frac{d^3 p}{{(2\pi)}^3} \Big ( - i \sqrt{\frac{\omega_{\mathbf p}}{2}} \Big ) \Big (\hat a_{1, \mathbf p} \exp(i \mathbf p \cdot \mathbf x) - \hat a_{1, \mathbf p}^\dagger \exp(- i \mathbf p \cdot \mathbf x) \Big) \\ & \qquad + i \int \frac{d^3 p}{{(2\pi)}^3} \Big ( - i \sqrt{\frac{\omega_{\mathbf p}}{2}} \Big ) \Big (\hat a_{2, \mathbf p} \exp(i \mathbf p \cdot \mathbf x) - \hat a_{2, \mathbf p}^\dagger \exp(- i \mathbf p \cdot \mathbf x) \Big) \Big) \\ & = \int \frac{d^3 p}{{(2\pi)}^3} \Big ( - i \sqrt{\frac{\omega_{\mathbf p}}{2}} \Big ) \Big ( \underbrace{\frac{\hat a_{1, \mathbf p} + i \hat a_{2, \mathbf p}}{\sqrt{2}}}_{\hat a_{+, \mathbf p}} \exp(i \mathbf p \cdot \mathbf x) - \underbrace{\frac{\hat a_{1, \mathbf p}^\dagger + i \hat a_{2, \mathbf p}^\dagger}{\sqrt{2}}}_{\hat a_{+, \mathbf p}^\dagger} \exp(- i \mathbf p \cdot \mathbf x) \Big) \\ & = \int \frac{d^3 p}{{(2\pi)}^3} \Big ( - i \sqrt{\frac{\omega_{\mathbf p}}{2}} \Big ) \Big ( \hat a_{+, \mathbf p} \exp(i \mathbf p \cdot \mathbf x) - \hat a_{-, \mathbf p}^\dagger \exp(- i \mathbf p \cdot \mathbf x) \Big)  
        \end{aligned}
        \end{equation*}
        and 
        \begin{equation*}
        \begin{aligned}
            \pi & = \frac{\pi_1 - i \pi_2}{\sqrt{2}} \\ & = \frac{1}{\sqrt{2}} \Big (\int \frac{d^3 p}{{(2\pi)}^3} \Big ( - i \sqrt{\frac{\omega_{\mathbf p}}{2}} \Big ) \Big (\hat a_{1, \mathbf p} \exp(i \mathbf p \cdot \mathbf x) - \hat a_{1, \mathbf p}^\dagger \exp(- i \mathbf p \cdot \mathbf x) \Big) \\ & \qquad - i \int \frac{d^3 p}{{(2\pi)}^3} \Big ( - i \sqrt{\frac{\omega_{\mathbf p}}{2}} \Big ) \Big (\hat a_{2, \mathbf p} \exp(i \mathbf p \cdot \mathbf x) - \hat a_{2, \mathbf p}^\dagger \exp(- i \mathbf p \cdot \mathbf x) \Big) \Big) \\ & = \int \frac{d^3 p}{{(2\pi)}^3} \Big ( - i \sqrt{\frac{\omega_{\mathbf p}}{2}} \Big ) \Big ( \underbrace{\frac{\hat a_{1, \mathbf p} - i \hat a_{2, \mathbf p}}{\sqrt{2}}}_{\hat a_{-, \mathbf p}} \exp(i \mathbf p \cdot \mathbf x) - \underbrace{\frac{\hat a_{1, \mathbf p}^\dagger - i \hat a_{2, \mathbf p}^\dagger}{\sqrt{2}}}_{\hat a_{+, \mathbf p}^\dagger} \exp(- i \mathbf p \cdot \mathbf x) \Big) \\ & = \int \frac{d^3 p}{{(2\pi)}^3} \Big ( - i \sqrt{\frac{\omega_{\mathbf p}}{2}} \Big ) \Big ( \hat a_{-, \mathbf p} \exp(i \mathbf p \cdot \mathbf x) - \hat a_{+, \mathbf p}^\dagger \exp(- i \mathbf p \cdot \mathbf x) \Big)  ~.
        \end{aligned}
        \end{equation*}
        Putting together
        \begin{equation*}
        \begin{aligned}
            \hat Q & = i \int d^3 x ~ (\hat \varphi \hat \pi - \hat \varphi^* \hat \pi^*) \\ & = i \int d^3 x ~ \Big ( \int \frac{d^3 p}{{(2\pi)}^3} \frac{1}{\sqrt{2 \omega_{\mathbf p}}} \Big ( \hat a_{+, \mathbf p} \exp(i \mathbf p \cdot \mathbf x) + \hat a_{-, \mathbf p}^\dagger \exp(- i \mathbf p \cdot \mathbf x) \Big) \\ & \qquad \int \frac{d^3 q}{{(2\pi)}^3} \Big ( - i \sqrt{\frac{\omega_{\mathbf q}}{2}} \Big ) \Big ( \hat a_{-, \mathbf q} \exp(i \mathbf q \cdot \mathbf x) - \hat a_{+, \mathbf q}^\dagger \exp(- i \mathbf q \cdot \mathbf x) \Big) \\ & \qquad - \int \frac{d^3 p}{{(2\pi)}^3} \frac{1}{\sqrt{2 \omega_{\mathbf p}}} \Big ( \hat a_{-, \mathbf p} \exp(i \mathbf p \cdot \mathbf x) + \hat a_{+, \mathbf p}^\dagger \exp(- i \mathbf p \cdot \mathbf x) \Big) \\ & \qquad \int \frac{d^3 q}{{(2\pi)}^3} \Big ( - i \sqrt{\frac{\omega_{\mathbf q}}{2}} \Big ) \Big ( \hat a_{+, \mathbf q} \exp(i \mathbf p \cdot \mathbf x) - \hat a_{-, \mathbf q}^\dagger \exp(- i \mathbf p \cdot \mathbf x) \Big) \Big) \\ & = \frac{1}{2} \int \frac{d^3 x ~ d^3 p ~ d^3 q}{(2\pi)^6} \sqrt{\frac{\omega_{\mathbf q}}{\omega_{\mathbf p}}} \Big ( \hat a_{+, \mathbf p} \hat a_{-, \mathbf q} \underbrace{\exp(i (\mathbf p + \mathbf q) \cdot \mathbf x)}_{\delta^3 (\mathbf q + \mathbf p)} - \hat a_{+, \mathbf p} \hat a_{+, \mathbf q}^\dagger \underbrace{\exp(i (\mathbf p - \mathbf q) \cdot \mathbf x)}_{\delta^3 (\mathbf q - \mathbf p)} \\ & \qquad + \hat a_{-, \mathbf p}^\dagger \hat a_{-, \mathbf q} \underbrace{\exp(i (- \mathbf p + \mathbf q) \cdot \mathbf x)}_{\delta^3 (\mathbf q - \mathbf p)} - \hat a_{-, \mathbf p}^\dagger \hat a_{+, \mathbf q}^\dagger \underbrace{\exp(i (- \mathbf p - \mathbf q) \cdot \mathbf x)}_{\delta^3 (\mathbf q + \mathbf p)} \\ & \qquad - \hat a_{-, \mathbf p} \hat a_{+, \mathbf q} \underbrace{\exp(i (\mathbf p + \mathbf q) \cdot \mathbf x)}_{\delta^3 (\mathbf q + \mathbf p)} + \hat a_{-, \mathbf p} \hat a_{-, \mathbf q}^\dagger \underbrace{\exp(i (\mathbf p - \mathbf q) \cdot \mathbf x)}_{\delta^3 (\mathbf q - \mathbf p)} \\ & \qquad - \hat a_{+, \mathbf p}^\dagger \hat a_{+, \mathbf q} \underbrace{\exp(i (- \mathbf p + \mathbf q) \cdot \mathbf x)}_{\delta^3 (\mathbf q - \mathbf p)} + \hat a_{+, \mathbf p}^\dagger \hat a_{-, \mathbf q}^\dagger \underbrace{\exp(i (- \mathbf p - \mathbf q) \cdot \mathbf x)}_{\delta^3 (\mathbf q + \mathbf p)} \Big) 
        \end{aligned}
        \end{equation*}
        \begin{equation*}
        \begin{aligned}
            \phantom{\hat Q} & = \frac{1}{2} \int \frac{d^3 p ~ d^3 q}{(2\pi)^3} \sqrt{\frac{\omega_{\mathbf q}}{\omega_{\mathbf p}}} \Big ( \hat a_{+, \mathbf p} \hat a_{-, \mathbf q} \underbrace{\delta^3 (\mathbf q + \mathbf p)}_{\mathbf q = - \mathbf p} - \hat a_{+, \mathbf p} \hat a_{+, \mathbf q}^\dagger \underbrace{\delta^3 (\mathbf q - \mathbf p)}_{\mathbf q = \mathbf p} \\ & \qquad + \hat a_{-, \mathbf p}^\dagger \hat a_{-, \mathbf q} \underbrace{\delta^3 (\mathbf q - \mathbf p)}_{\mathbf q = \mathbf p} - \hat a_{-, \mathbf p}^\dagger \hat a_{+, \mathbf q}^\dagger \underbrace{\delta^3 (\mathbf q + \mathbf p)}_{\mathbf q = - \mathbf p} \\ & \qquad - \hat a_{-, \mathbf p} \hat a_{+, \mathbf q} \underbrace{\delta^3 (\mathbf q + \mathbf p)}_{\mathbf q = - \mathbf p} + \hat a_{-, \mathbf p} \hat a_{-, \mathbf q}^\dagger \underbrace{\delta^3 (\mathbf q - \mathbf p)}_{\mathbf q = \mathbf p} \\ & \qquad - \hat a_{+, \mathbf p}^\dagger \hat a_{+, \mathbf q} \underbrace{\delta^3 (\mathbf q - \mathbf p)}_{\mathbf q = \mathbf p} + \hat a_{+, \mathbf p}^\dagger \hat a_{-, \mathbf q}^\dagger \underbrace{\delta^3 (\mathbf q + \mathbf p)}_{\mathbf q = - \mathbf p} \Big) \\ & = \frac{1}{2} \int \frac{d^3 p}{(2\pi)^3} ( \hat a_{+, \mathbf p} \hat a_{-, - \mathbf p} - \hat a_{+, \mathbf p} \hat a_{+, \mathbf p}^\dagger + \hat a_{-, \mathbf p}^\dagger \hat a_{-, \mathbf p} - \hat a_{-, \mathbf p}^\dagger \hat a_{+, - \mathbf p}^\dagger \\ & \qquad - \hat a_{-, \mathbf p} \hat a_{+, - \mathbf p} + \hat a_{-, \mathbf p} \hat a_{-, \mathbf p}^\dagger - \hat a_{+, \mathbf p}^\dagger \hat a_{+, \mathbf p} + \hat a_{+, \mathbf p}^\dagger \hat a_{-, - \mathbf p}^\dagger ) \\ & = \frac{1}{2} \int \frac{d^3 p}{(2\pi)^3} ( \hat a_{+, \mathbf p} \hat a_{-, - \mathbf p} - \hat a_{-, \mathbf p} \hat a_{+, - \mathbf p} ) + \frac{1}{2} \int \frac{d^3 p}{(2\pi)^3} (\hat a_{+, \mathbf p}^\dagger \hat a_{-, - \mathbf p}^\dagger - \hat a_{-, \mathbf p}^\dagger \hat a_{+, - \mathbf p}^\dagger) \\ & \qquad + \frac{1}{2} \int \frac{d^3 p}{(2\pi)^3} (\hat a_{-, \mathbf p}^\dagger \hat a_{-, \mathbf p} + \hat a_{-, \mathbf p} \hat a_{-, \mathbf p}^\dagger - \hat a_{+, \mathbf p} \hat a_{+, \mathbf p}^\dagger - \hat a_{+, \mathbf p}^\dagger \hat a_{+, \mathbf p} ) ~,
        \end{aligned}
        \end{equation*}
        where in the last row, the first two integrals vanish because they are odd functions since they commute. Finally, in normal ordering, we obtain 
        \begin{equation*}
            \hat Q = \int \frac{d^3 p}{(2\pi)^3} ( \hat a_{-, \mathbf p}^\dagger \hat a_{-, \mathbf p} - \hat a_{+, \mathbf p}^\dagger \hat a_{+, \mathbf p} ) ~.
        \end{equation*}
    \end{proof}

    We remark that this is possible because the theory is free. If there are interactions, the number operators are not anymore conserved but the charge operator still is: interactions create and destroy particles and antiparticles under the constrain of conserved total charge.

\chapter{Manifestly Lorentz covariance}

\section{Lorentz covariance}

    The vacuum state is normalised 
    \begin{equation*}
        \braket{0}{0} = 1 ~,
    \end{equation*}
    while $1$-particle states satisfiy the orthogonality relation 
    \begin{equation*}
        \braket{\mathbf p}{\mathbf q} = (2\pi)^3 \delta^3 (\mathbf p - \mathbf q) 
    \end{equation*}
    and the completeness relation 
    \begin{equation*}
        \mathbb I = \int \frac{d^3 p}{(2 \pi)^3} \ket{\mathbf p} \bra{\mathbf p} ~,
    \end{equation*}
    where $\mathbb I$ is the identity operator.
    \begin{proof}
        Maybe in the future.
    \end{proof}

    However, we want Lorentz covariance, since the identity operator is so but the right side of the completeness relation is not, given that the measure $\int d^3 p$ and the projector $\ket{\mathbf p} \bra{\mathbf p}$ are not separately so. We know that $\in d^4 p$ is Lorentz covariant, because 
    \begin{equation*}
        d^4 p' = \underbrace{|\det \Lambda|}_1 d^4 p = d^4 p ~.
    \end{equation*}
    Therefore, we change the orthogonality relation into 
    \begin{equation*}
        \braket{p}{q} = (2\pi)^3 2 \sqrt{E_{\mathbf p} E_{\mathbf q}} \delta^3 (\mathbf p - \mathbf q) 
    \end{equation*}
    and the completeness relation into 
    \begin{equation*}
        \mathbb I = \int \frac{d^4 p}{(2\pi)^3} \delta (p^2_0 - |\mathbf p|^2 - m^2) \theta(p_0) \ket{\mathbf p} \bra{\mathbf p} ~,
    \end{equation*}
    where $p_0 = E_{\mathbf p} = \sqrt{|\mathbf p|^2 + m^2}$ and the manifestly invariant states are 
    \begin{equation}
        \ket{p} = \sqrt{2E_{\mathbf p}} \ket{\mathbf p} ~.
    \end{equation}
    \begin{proof}
        Maybe in the future.
    \end{proof}

\section{Heisenberg picture}

    Classical theory is Lorentz-invariant since the lagrangian $\mathcal L$ is manifestly so. However, so far in the Schroedinger picture, we worked at a preferred time in which field operators are $\hat \varphi (\mathbf x)$ and $\hat \pi (\mathbf x)$. The time evolution is governed by the Schroedinger equation
    \begin{equation*}
        i \dv{}{t} \ket{\mathbf p(t)} = \hat H \ket{\mathbf p(t)} ~,
    \end{equation*}
    in which a state evolves as 
    \begin{equation*}
        \ket{\mathbf p(t)} = \exp(- i E_{\mathbf p} t) \ket{\mathbf p} ~.
    \end{equation*}

    In Heisenberg's picture, an operator is related to the Schroedinger's one as 
    \begin{equation*}
        \hat O_H = \exp(i \hat H t) \hat O_S \exp(- i \hat H t) ~,
    \end{equation*}
    where its time evolution is governed by the Heisenberg equation 
    \begin{equation*}
        \dv{}{\hat O_H} = i [\hat H, \hat O_H] ~.
    \end{equation*}
    \begin{proof}
        In fact 
        \begin{equation*}
        \begin{aligned}
            \dv{}{t} \hat O_H & = \dv{}{t} \Big ( \exp(i \hat H t) \hat O_S \exp(- i \hat H t) \Big) \\ & = \dv{}{t} \Big ( \exp(i \hat H t) \Big) \hat O_S \exp(- i \hat H t) + \exp(i \hat H t) \cancel{\dv{}{t} \Big (  \hat O_S  \Big)} \exp(- i \hat H t) + \exp(i \hat H t) \hat O_S \dv{}{t} \Big (  \exp(- i \hat H t) \Big ) \\ & = i \hat H \underbrace{\exp(i \hat H t) \hat O_S \exp(- i \hat H)}_{\hat O_H} - i \underbrace{\exp(i \hat H t) \hat O_S \exp(- i \hat H)}_{\hat O_H} \hat H \\ & = i \hat H \hat O_H - i \hat O_H \hat H \\ & = i [\hat H, \hat O_H] ~.
        \end{aligned}
        \end{equation*}
    \end{proof}

    Therefore, in Schoredinger picture we have $\hat \varphi(\mathbf x)$ while in Heisenberg picture we have $\hat \varphi(x)$, where they agree at $t=0$. The commutation relation at equal time $t$ becomes 
    \begin{equation*}
        [\hat \varphi(t, \mathbf x), \hat \varphi(t, \mathbf y)] = [\hat \pi(t, \mathbf x), \hat \pi(t, \mathbf y)] = 0 ~, \quad [\hat \varphi(t, \mathbf x), \hat \pi(t, \mathbf y)] = i \delta^3 (\mathbf x - \mathbf y) ~.
    \end{equation*}

    The time evolution of $\hat \varphi (x)$ is 
    \begin{equation*}
        \pdv{}{t} \hat \varphi (x) = \hat \pi (x) ~.
    \end{equation*}
    \begin{proof}
        In fact 
        \begin{equation*}
        \begin{aligned}
            \pdv{}{t} \hat \varphi (x) & = i [\hat H, \hat \varphi (x)] \\ & = i [\frac{1}{2} \int d^3 y (\hat \pi^2 (t, \mathbf y) + \nabla^2 \hat \varphi (t, \mathbf y) + m^2 \hat \varphi^2 (t, \mathbf y)) , \hat \varphi (t, \mathbf x)] \\ & = \frac{i}{2} \int d^3 y \Big (\underbrace{[\hat \pi^2 (t, \mathbf y), \hat \varphi (t, \mathbf x)]}_{\pi (t, \mathbf y) [\hat \pi (t, \mathbf y), \hat \varphi (t, \mathbf x)] + [\hat \pi (t, \mathbf y), \hat \varphi (t, \mathbf x)] \pi (t, \mathbf y)} + \underbrace{[\nabla^2 \hat \varphi (t, \mathbf y), \hat \varphi (t, \mathbf x)]}_0 + m^2 \underbrace{[\hat \varphi^2 (t, \mathbf y), \hat \varphi (t, \mathbf x)]}_0 \Big ) \\ & = \frac{i}{2} \int d^3 y \Big (\pi (t, \mathbf y) \underbrace{[\hat \pi (t, \mathbf y), \hat \varphi (t, \mathbf x)]}_{- i \delta^3 (\mathbf x - \mathbf y)} + \underbrace{[\hat \pi (t, \mathbf y), \hat \varphi (t, \mathbf x)] }_{- i \delta^3 (\mathbf x - \mathbf y)} \pi (t, \mathbf y) \Big ) \\ & = \frac{1}{2} \int d^3 y \Big (\pi (t, \mathbf y) \underbrace{\delta^3 (\mathbf x - \mathbf y)}_{\mathbf x = \mathbf y} + \underbrace{\delta^3 (\mathbf x - \mathbf y)}_{\mathbf x = \mathbf y} \pi (t, \mathbf y) \Big ) \\ & = \frac{1}{2} (\pi (t, \mathbf x) + \pi (t, \mathbf x)) \\ & = \pi (t, \mathbf x) ~,
        \end{aligned}
        \end{equation*}
        where we have used $[\boldsymbol \nabla_{\mathbf y} \hat \varphi (t, \mathbf y), \hat \varphi (t, \mathbf x)] = 0$ since the right-handed side depends on $\mathbf y$ and the left-handed side depends on $\mathbf x$.
    \end{proof}

    The time evolution of $\hat \pi (x)$ is 
    \begin{equation*}
        \pdv{}{t} \hat \pi (x) = (\nabla^2 - m^2) \hat \varphi (x) ~.
    \end{equation*}
    \begin{proof}
        In fact 
        \begin{equation*}
        \begin{aligned}
            \pdv{}{t} \hat \pi (x) & = i [\hat H, \hat \pi (x)] \\ & = i [\frac{1}{2} \int d^3 y (\hat \pi^2 (t, \mathbf y) + \nabla^2 \hat \varphi (t, \mathbf y) + m^2 \hat \varphi^2 (t, \mathbf y)) , \hat \pi (t, \mathbf x)] \\ & = \frac{i}{2} \int d^3 y \Big (\underbrace{[\hat \pi^2 (t, \mathbf y), \hat \pi (t, \mathbf x)]}_0 + \underbrace{[\nabla^2 \hat \varphi (t, \mathbf y), \hat \pi (t, \mathbf x)]}_{[ \boldsymbol \nabla_{\mathbf y} \hat \varphi (t, \mathbf y) \cdot \boldsymbol \nabla_{\mathbf y} \hat \varphi (t, \mathbf y), \hat \pi (t, \mathbf x)]} + m^2 \underbrace{[\hat \varphi^2 (t, \mathbf y), \hat \pi (t, \mathbf x)]}_{\hat \varphi (t, \mathbf y) [\hat \varphi (t, \mathbf y), \hat \pi (t, \mathbf x)] + [\hat \varphi (t, \mathbf y), \hat \pi (t, \mathbf x)] \hat \varphi (t, \mathbf y)} \Big ) \\ & = \frac{i}{2} \int d^3 y \Big (\underbrace{[\boldsymbol \nabla_{\mathbf y} \hat \varphi (t, \mathbf y) \cdot \boldsymbol \nabla_{\mathbf y} \hat \varphi (t, \mathbf y), \hat \pi (t, \mathbf x)]}_{\boldsymbol \nabla_{\mathbf y} \hat \varphi (t, \mathbf y) \cdot [\boldsymbol \nabla_{\mathbf y} \hat \varphi (t, \mathbf y), \hat \pi (t, \mathbf x)] + [\boldsymbol \nabla_{\mathbf y} \hat \varphi (t, \mathbf y), \hat \pi (t, \mathbf x)] \cdot \boldsymbol \nabla_{\mathbf y} \hat \varphi (t, \mathbf y)} \\ & \qquad + m^2 \hat \varphi (t, \mathbf y) \underbrace{[\hat \varphi (t, \mathbf y), \hat \pi (t, \mathbf x)]}_{i \delta^3 (\mathbf x - \mathbf y)} + m^2 \underbrace{[\hat \varphi (t, \mathbf y), \hat \pi (t, \mathbf x)]}_{i \delta^3 (\mathbf x - \mathbf y)} \hat 
            \varphi (t, \mathbf y) \Big ) 
        \end{aligned}    
        \end{equation*}
        \begin{equation*}
        \begin{aligned}
            \phantom{\pdv{}{t} \hat \pi (x)} & = \frac{i}{2} \int d^3 y \Big (\boldsymbol \nabla_{\mathbf y} \hat \varphi (t, \mathbf y) \cdot \underbrace{[\boldsymbol \nabla_{\mathbf y} \hat \varphi (t, \mathbf y), \hat \pi (t, \mathbf x)]}_{\boldsymbol \nabla_{\mathbf y} [\hat \varphi (t, \mathbf y), \hat \pi (t, \mathbf x)]} + \underbrace{[\boldsymbol \nabla_{\mathbf y} \hat \varphi (t, \mathbf y), \hat \pi (t, \mathbf x)]}_{\boldsymbol \nabla_{\mathbf y} [\hat \varphi (t, \mathbf y), \hat \pi (t, \mathbf x)]} \cdot \boldsymbol \nabla_{\mathbf y} \hat \varphi (t, \mathbf y) \\ & \qquad + i m^2 \hat \varphi (t, \mathbf y) i \delta^3 (\mathbf x - \mathbf y) + i m^2 \delta^3 (\mathbf x - \mathbf y) \hat \varphi (t, \mathbf y) \Big ) \\ & = \frac{i}{2} \int d^3 y \Big (\boldsymbol \nabla_{\mathbf y} \hat \varphi (t, \mathbf y) \cdot \boldsymbol \nabla_{\mathbf y} \underbrace{[\hat \varphi (t, \mathbf y), \hat \pi (t, \mathbf x)]}_{i \delta^3 (\mathbf x - \mathbf y)} + \boldsymbol \nabla_{\mathbf y} \underbrace{[\hat \varphi (t, \mathbf y), \hat \pi (t, \mathbf x)]}_{i \delta^3 (\mathbf x - \mathbf y)} \cdot \boldsymbol \nabla_{\mathbf y} \hat \varphi (t, \mathbf y) \\ & \qquad + i m^2 \hat \varphi (t, \mathbf y) i \delta^3 (\mathbf x - \mathbf y) + i m^2 \delta^3 (\mathbf x - \mathbf y) \hat \varphi (t, \mathbf y) \Big ) \\ & = \frac{i}{2} \int d^3 y \Big (i \underbrace{\boldsymbol \nabla_{\mathbf y} \hat \varphi (t, \mathbf y) \cdot \boldsymbol \nabla_{\mathbf y} \delta^3 (\mathbf x - \mathbf y)}_{- \boldsymbol \nabla_{\mathbf y} \cdot \boldsymbol \nabla_{\mathbf y} \hat \varphi (t, \mathbf y) \delta^3 (\mathbf x - \mathbf y) + \textnormal{boundary term}} + i \underbrace{\boldsymbol \nabla_{\mathbf y} \delta^3 (\mathbf x - \mathbf y) \cdot \boldsymbol \nabla_{\mathbf y} \hat \varphi (t, \mathbf y)}_{- \delta^3 (\mathbf x - \mathbf y) \boldsymbol \nabla_{\mathbf y} \cdot \boldsymbol \nabla_{\mathbf y} \hat \varphi (t, \mathbf y) + \textnormal{boundary term}} \\ & \qquad + i m^2 \hat \varphi (t, \mathbf y) i \delta^3 (\mathbf x - \mathbf y) + i m^2 \delta^3 (\mathbf x - \mathbf y) \hat \varphi (t, \mathbf y) \Big ) \\ & = \frac{i}{2} \int d^3 y \Big (- i \boldsymbol \nabla_{\mathbf y} \cdot \boldsymbol \nabla_{\mathbf y} \hat \varphi (t, \mathbf y) \underbrace{\delta^3 (\mathbf x - \mathbf y) }_{\mathbf x = \mathbf y} - i \underbrace{\delta^3 (\mathbf x - \mathbf y) }_{\mathbf x = \mathbf y} \boldsymbol \nabla_{\mathbf y} \cdot \boldsymbol \nabla_{\mathbf y} \hat \varphi (t, \mathbf y) \\ & \qquad + i m^2 \hat \varphi (t, \mathbf y) i \underbrace{\delta^3 (\mathbf x - \mathbf y) }_{\mathbf x = \mathbf y} + i m^2 \underbrace{\delta^3 (\mathbf x - \mathbf y) }_{\mathbf x = \mathbf y} \hat \varphi (t, \mathbf y) \Big ) \\ & = \boldsymbol \nabla_{\mathbf x} \cdot \boldsymbol \nabla_{\mathbf x} \hat \varphi (t, \mathbf x) - m^2 \hat \varphi (t, \mathbf x) = (\nabla^2 - m^2) \hat \varphi (t, \mathbf x) ~.
        \end{aligned}
        \end{equation*}
    \end{proof}

    Combining the two time evolutions, we obtain the Klein-Gordon equation~\eqref{kgeq}
    \begin{equation*}
        (\Box + m^2) \hat \varphi (x) ~.
    \end{equation*}
    \begin{proof}
        In fact, 
        \begin{equation*}
            \hat{\ddot \varphi} (x) =  \hat{\dot \pi} (x) = (\nabla^2 - m^2) \hat \psi(x) ~,
        \end{equation*}
        hence
        \begin{equation*}
            0 = (\pdvdu{}{t} - \nabla^2 + m^2) \hat \varphi (x) = (\Box + m^2) \hat \varphi (x) ~.
        \end{equation*}
    \end{proof}

    Since field operators evolve in time and they depend on the ladder operators, the latters evolve in time as well 
    \begin{equation*}
        (\hat a_{\mathbf p})_H = \exp(- i E_{\mathbf p} t) (\hat a_{\mathbf p})_S ~, \quad (\hat a^\dagger_{\mathbf p})_H = \exp(i E_{\mathbf p} t) (\hat a^\dagger_{\mathbf p})_S~.
    \end{equation*}
    \begin{proof}
        Using the formula $ \hat H^n \hat a_{\mathbf p} = \hat a_{\mathbf p} (\hat H - E_{\mathbf p})^n$ which can be proved 
        \begin{equation*}
            [\hat H, \hat a_{\mathbf p}] = \hat H \hat a_{\mathbf p} - \hat a_{\mathbf p} \hat H = - \omega_{\mathbf p} \hat a_{\mathbf p} = - E_{\mathbf p} \hat a_{\mathbf p} ~,
        \end{equation*}
        \begin{equation*}
            \hat H \hat a_{\mathbf p} = \hat a_{\mathbf p} (\hat H - E_{\mathbf p}) ~,
        \end{equation*}
        hence 
        \begin{equation}
            \hat H^2 \hat a_{\mathbf p} = \hat H \hat a_{\mathbf p} (\hat H - E_{\mathbf p}) = \hat a_{\mathbf p} (\hat H - E_{\mathbf p})^2 ~,
        \end{equation}
        by induction 
        \begin{equation*}
            \hat H^n \hat a_{\mathbf p} = \hat a_{\mathbf p} (\hat H - E_{\mathbf p})^n ~.
        \end{equation*}

        For the annihilation operator 
        \begin{equation*}
        \begin{aligned}
            (\hat a_{\mathbf p})_H & = \exp(i \hat H t) (\hat a_{\mathbf p})_S \exp(- i \hat H t) \\ & = \underbrace{\exp(i \hat H t)}_{\sum_n \frac{(i t)^n}{n!} \hat H^n} \hat a_{\mathbf p} \exp(- i \hat H t) \\ & = \sum_n \frac{(i t)^n}{n!} \underbrace{\hat H^n \hat a_{\mathbf p}}_{\hat a_{\mathbf p} (\hat H - E_{\mathbf p})^n} \exp(- i \hat H t) \\ & = \underbrace{\sum_n \frac{1}{n!}  (i t (\hat H - E_{\mathbf p}))^n }_{\exp(i t (\hat H - E_{\mathbf p}))}  \hat a_{\mathbf p} \exp(- i \hat H t) \\ & = \exp(i t (\cancel{\hat H} - E_{\mathbf p})) \hat a_{\mathbf p} \cancel{\exp(- i \hat H t)} \\ & = \exp(- i E_{\mathbf p} t) \hat a_{\mathbf p} ~.
        \end{aligned}
        \end{equation*}

        Using the formula $\hat H^n \hat a_{\mathbf p} = \hat a_{\mathbf p} (\hat H - E_{\mathbf p})^n$ which can be proved 
        \begin{equation*}
            [\hat H, \hat a^\dagger_{\mathbf p}] = \hat H \hat a^\dagger_{\mathbf p} - \hat a^\dagger_{\mathbf p} \hat H = \omega_{\mathbf p} \hat a^\dagger_{\mathbf p} = E_{\mathbf p} \hat a^\dagger_{\mathbf p} ~,
        \end{equation*}
        \begin{equation*}
            \hat H \hat a^\dagger_{\mathbf p} = \hat a^\dagger_{\mathbf p} (\hat H + E_{\mathbf p}) ~,
        \end{equation*}
        hence 
        \begin{equation}
            \hat H^2 \hat a^\dagger_{\mathbf p} = \hat H \hat a^\dagger_{\mathbf p} (\hat H + E_{\mathbf p}) = \hat a^\dagger_{\mathbf p} (\hat H + E_{\mathbf p})^2 ~,
        \end{equation}
        by induction 
        \begin{equation*}
            \hat H^n \hat a^\dagger_{\mathbf p} = \hat a^\dagger_{\mathbf p} (\hat H + E_{\mathbf p})^n ~.
        \end{equation*}

        For the annihilation operator 
        \begin{equation*}
        \begin{aligned}
            (\hat a^\dagger_{\mathbf p})_H & = \exp(i \hat H t) (\hat a^\dagger_{\mathbf p})_S \exp(- i \hat H t) \\ & = \underbrace{\exp(i \hat H t)}_{\sum_n \frac{(i t)^n}{n!} \hat H^n} \hat a^\dagger_{\mathbf p} \exp(- i \hat H t) \\ & = \sum_n \frac{(i t)^n}{n!} \underbrace{\hat H^n \hat a^\dagger_{\mathbf p}}_{\hat a^\dagger_{\mathbf p} (\hat H + E_{\mathbf p})^n} \exp(- i \hat H t) \\ & = \underbrace{\sum_n \frac{1}{n!}  (i t (\hat H + E_{\mathbf p}))^n }_{\exp(i t (\hat H + E_{\mathbf p}))}  \hat a^\dagger_{\mathbf p} \exp(- i \hat H t) \\ & = \exp(i t (\cancel{\hat H} + E_{\mathbf p})) \hat a^\dagger_{\mathbf p} \cancel{\exp(- i \hat H t)} \\ & = \exp(i E_{\mathbf p} t) \hat a^\dagger_{\mathbf p} ~.
        \end{aligned}
        \end{equation*}
    \end{proof}

    Finally, the field operators are 
    \begin{equation*}
        \hat \varphi (x) = \int \frac{d^3 p}{(2\pi)^3} \frac{1}{\sqrt{2 E_{\mathbf p}}} (\hat a^\dagger_{\mathbf p} \exp(i p x) + \hat a_{\mathbf p} \exp(-i p x))
    \end{equation*}
    and 
    \begin{equation*}
        \hat \pi (x) = \int \frac{d^3 p}{(2\pi)^3} \Big (- i \sqrt{\frac{ E_{\mathbf p}}{2}} \Big ) (\hat a_{\mathbf p} \exp(-i p x) - \hat a^\dagger_{\mathbf p} \exp(i p x) ) ~,
    \end{equation*}
    where $p x = p^\mu x_\mu = E_{\mathbf p} t - \mathbf p \cdot \mathbf x$. 
    \begin{proof}
        For the field operator 
        \begin{equation}
        \begin{aligned}
            \hat \varphi (x) & = \int \frac{d^3 p}{(2\pi)^3} \frac{1}{\sqrt{2 E_{\mathbf p}}} ((\hat a^\dagger_{\mathbf p})_H \exp(-i \mathbf p \cdot \mathbf x) + (\hat a_{\mathbf p})_H \exp(i \mathbf p \cdot \mathbf x)) \\ & = \int \frac{d^3 p}{(2\pi)^3} \frac{1}{\sqrt{2 E_{\mathbf p}}} (\hat a^\dagger_{\mathbf p} \exp(i (E_{\mathbf p} t - \mathbf p \cdot \mathbf x)) + \hat a_{\mathbf p} \exp(-i (E_{\mathbf p} t - \mathbf p \cdot \mathbf x))) \\ & = \int \frac{d^3 p}{(2\pi)^3} \frac{1}{\sqrt{2 E_{\mathbf p}}} (\hat a^\dagger_{\mathbf p} \exp(i px) + \hat a_{\mathbf p} \exp(-i px)) ~.
        \end{aligned}
        \end{equation}

        For the conjugate field operator 
        \begin{equation}
        \begin{aligned}
            \hat \pi (x) & = \int \frac{d^3 p}{(2\pi)^3} \Big (- i \sqrt{\frac{ E_{\mathbf p}}{2}} \Big ) (- (\hat a^\dagger_{\mathbf p})_H \exp(-i \mathbf p \cdot \mathbf x) + (\hat a_{\mathbf p})_H \exp(i \mathbf p \cdot \mathbf x)) \\ & = \int \frac{d^3 p}{(2\pi)^3} \Big (- i \sqrt{\frac{ E_{\mathbf p}}{2}} \Big ) (- \hat a^\dagger_{\mathbf p} \exp(i (E_{\mathbf p} t - \mathbf p \cdot \mathbf x)) + \hat a_{\mathbf p} \exp(-i (E_{\mathbf p} t - \mathbf p \cdot \mathbf x))) \\ & = \int \frac{d^3 p}{(2\pi)^3} \Big (- i \sqrt{\frac{ E_{\mathbf p}}{2}} \Big ) ( - \hat a^\dagger_{\mathbf p} \exp(i px) + \hat a_{\mathbf p} \exp(-i px)) ~.
        \end{aligned}
        \end{equation}
    \end{proof}

\section{Casuality} 

    So far, almost everything is manifestly Lorentz invariant except the commutation relations, because they privileged a time since they must be evaluated at equal time. We need to work out commutation relations at arbitrary times 
    \begin{equation*}
        [\hat O_1 (t, \mathbf x), \hat O_2 (t', \mathbf y)] = [\hat O_1 (x), \hat O_2 (y)] ~.
    \end{equation*}
    Its physical meaning is the effects of one onto the other one. This means that in order to preserve causality, it must be zero outside the light cone since the signal cannot travel faster than light. Obviously, if it is inside the light cone, it can be non-zero. This means that
    \begin{equation*}
        [\hat O_1 (x), \hat O_2 (y)] = 0 \quad \forall (x - y)^2 = (x^\mu - y^\mu) (x_\mu - y_\mu) < 0 ~. 
    \end{equation*}

    We define the quantity 
    \begin{equation*}
        \Delta (x - y) = [\hat \varphi (x), \hat \varphi (y)] ~.
    \end{equation*}

    It is Lorentz invariant
    \begin{equation*}
        \Delta (x - y) = \int \frac{d^3 p}{(2 \pi)^3} \frac{1}{2 E_{\mathbf p}} \Big ( \exp(- i p (x - y)) - \exp(i p (x - y)) \Big) ~.
    \end{equation*}
    \begin{proof}
        In fact, 
        \begin{equation*}
        \begin{aligned}
            \Delta (x - y) & = [\hat \varphi (x), \hat \varphi (y)] \\ & = [\int \frac{d^3 p}{{(2\pi)}^3} \frac{1}{\sqrt{2 \omega_{\mathbf p}}} \Big (\hat a_{\mathbf p} \exp(- i p x) + \hat a_{\mathbf p}^\dagger \exp(i p x) \Big) , \\ & \qquad  \int \frac{d^3 q}{{(2\pi)}^3} \frac{1}{\sqrt{2 \omega_{\mathbf p}}} \Big (\hat a_{\mathbf q} \exp(- i q y) + \hat a_{\mathbf q}^\dagger \exp( i q y) \Big)] \\ & = \int \frac{d^3 p ~ d^3 q}{{(2\pi)}^6} \frac{1}{2 \sqrt{\omega_{\mathbf p}}\omega_{\mathbf q}}  \Big ( \underbrace{[\hat a_{\mathbf p}, \hat a_{\mathbf q}]}_0 exp (- i p x - i q y) + \underbrace{[\hat a_{\mathbf p}^\dagger, \hat a_{\mathbf q}]}_{- (2\pi)^3 \delta (p - q)} exp (i p x - i q y) \\ & \qquad + \underbrace{[\hat a_{\mathbf p}, \hat a_{\mathbf q}^\dagger]}_{(2\pi)^3 \delta ( p - q)} exp (- i p x + i q y) + \underbrace{[\hat a_{\mathbf p}^\dagger, \hat a_{\mathbf p}^\dagger]}_0 exp (i p x + i q y) \Big) \\ & = \int \frac{d^3 p ~ d^3 q}{{(2\pi)}^3} \frac{1}{2 \sqrt{\omega_{\mathbf p}}\omega_{\mathbf q}}  \Big ( - \underbrace{\delta (p - q)}_{p = q} exp (i p x - i q y) + \underbrace{\delta ( p - q)}_{p = q} exp (- i p x + i q y) \Big) \\ & =  = \int \frac{d^3 p ~ d^3 q}{{(2\pi)}^3} \frac{1}{2 \omega_{\mathbf p}} \Big ( exp (- i p (x -  y)) - exp (i p (x - y) )  \Big) \\ & =
        \end{aligned}
        \end{equation*}
    \end{proof}

    It satisfies the causality constrains
    \begin{enumerate}
        \item inside the light cone, for timelike separations
            \begin{equation*}
                \Delta (x - y) \neq 0 \quad \forall (x - y)^2 > 0 ~,
            \end{equation*}
        \item outside the light cone, for spacelike separations
            \begin{equation*}
                \Delta (x - y) = 0 \quad \forall (x - y)^2 < 0 ~.
            \end{equation*}
    \end{enumerate}
    \begin{proof}
        For inside the light cone, we choose one particular case with $x^\mu = (t, 0, 0, 0)$ and $y^\mu = (0, 0, 0,0 0)$, where we are static in space. Hence 
        \begin{equation*}
        \begin{aligned}
        \Delta (x - y) & = \int \frac{d^3 p}{(2 \pi)^3} \frac{1}{2 E_{\mathbf p}} \Big ( \exp(- i \underbrace{p (x - y)}_{E_{\mathbf p} t}) - \exp(i \underbrace{p (x - y)}_{E_{\mathbf p} t}) \Big) \\ & = \int \frac{d^3 p}{(2 \pi)^3} \frac{1}{2 E_{\mathbf p}} \Big ( \exp(- i E_{\mathbf p} t) - \exp(i E_{\mathbf p} t) \Big) ~.
        \end{aligned}
        \end{equation*}
        In polar coordinates $(|p|, \theta, \varphi)$
        \begin{equation*}
        \begin{aligned}
        \Delta (x - y) & = \int_0^\infty \frac{d|p|}{(2 \pi)^3} \frac{|p|^2}{2 E_{\mathbf p}} \Big ( \exp(- i E_{\mathbf p} t) - \exp(i E_{\mathbf p} t) \Big) \underbrace{\int_0^{2\pi} d\varphi \int_0^\pi d\theta \sin \theta}_{4\pi} \\ & = 4 \pi \int_0^\infty \frac{d|p|}{(2 \pi)^3} \frac{|p|^2}{2 \sqrt{|p|^2 + m^2}} \Big ( \exp(- i \sqrt{|p|^2 + m^2} t) - \exp(i \sqrt{|p|^2 + m^2} t) \Big) ~.
        \end{aligned}
        \end{equation*}
        By a change of variable $|p| = E_{\mathbf p}$ with differential $|p|^2 d|p| = E_{\mathbf p} d E_{\mathbf p} \sqrt{E_{\mathbf p}^2 - m^2}$
        \begin{equation*}
        \begin{aligned}
            \Delta (x-y) & = \frac{1}{4\pi^2} \int_m^\infty d E_{\mathbf p} \sqrt{E_{\mathbf p} - m^2} (\exp(- i E_{\mathbf p} t) - \exp(i E_{\mathbf p} t)) \\ & = \frac{m}{8 \pi t} (Y_1(mt) + i J_1(mt) - Y_1(-mt) - i J_1(-mt)) ~,
        \end{aligned}
        \end{equation*}
        where have used the Bessel function of first order. notice that their behaviour at infinity is 
        \begin{equation*}
            J_1(x) \xrightarrow{x \rightarrow \infty} \sqrt{\frac{2\pi}{x}} \cos x ~, \quad Y_1(x) \xrightarrow{x \rightarrow \infty} \sqrt{\frac{2}{\pi x}} \sin x ~,
        \end{equation*}
        hence 
        \begin{equation*}
            Y_1(mt) + m J_2(mt) \xrightarrow{t \rightarrow \infty} \sqrt{\frac{2}{\pi mt}} (\sin (mt) + i \cos (mt)) = i \sqrt{\frac{2}{\pi mt}} \exp(- i mt)
        \end{equation*}
        and 
        \begin{equation*}
            \Delta (x-y) \xrightarrow{t \rightarrow \infty} \propto \exp(-imt) - \exp(i mt) \neq 0~.
        \end{equation*}

        For outside the light cone, because of the Lorentz invariance, we need to prove only a particular spacelike separations and it becomes true for all spacelike separations. We choose the one at the same t
        \begin{equation*}
        \begin{aligned}
            \Delta (x - y) & = \int \frac{d^3 p}{(2 \pi)^3} \frac{1}{2 E_{\mathbf p}} \Big ( \exp(- i \underbrace{p (x - y)}_{- \mathbf p \cdot (\mathbf x - \mathbf y)}) - \exp(i \underbrace{p (x - y)}_{- \mathbf p \cdot (\mathbf x - \mathbf y)}) \Big) \\ & = \int \frac{d^3 p}{(2 \pi)^3} \frac{1}{2 E_{\mathbf p}} \Big ( \exp( i\mathbf p \cdot (\mathbf x - \mathbf y)) - \exp( - i \mathbf p \cdot (\mathbf x - \mathbf y)) \Big) = 0 ~,
        \end{aligned}
        \end{equation*}
        where we echanged $\mathbf p$ in $-\mathbf p$ in the second term of the integrand.
    \end{proof}

    We have just proved that the Klein-Gordon theory preserves causality.
    
\section{Correlators}

    Another way to study is via the propagators. Consider a particle at a spacetime point $y$. What is the probability to find it st $x$? We define the propagator 
    \begin{equation*}
        D(x - y) = \bra{0} \hat \varphi (x) \hat \varphi (y) \ket{0} ~.
    \end{equation*}
    which is 
    \begin{equation}
        D(x-y) = \int \frac{d^3 p}{(2\pi)^3} \frac{1}{2 E_{\mathbf p}} \exp(- i p (x - y)) ~.
    \end{equation}
    \begin{proof}
        In fact 
        \begin{equation*}
        \begin{aligned}
            D(x- y) & = \bra{0} \hat \varphi (x) \hat \varphi (y) \ket{0} \\ & = \bra{0} \int \frac{d^3 p}{(2\pi)^3} \frac{1}{\sqrt{2 E_{\mathbf p}}} (\hat a^\dagger_{\mathbf p} \exp(i p x) + \hat a_{\mathbf p} \exp(-i p x)) \\ & \qquad \int \frac{d^3 q}{(2\pi)^3} \frac{1}{\sqrt{2 E_{\mathbf q}}} (\hat a^\dagger_{\mathbf q} \exp(i q y) + \hat a_{\mathbf q} \exp(-i q y)) \ket{0} \\ & = \int \frac{d^3 p ~ d^3 q}{(2\pi)^6} \frac{1}{2 \sqrt{E_{\mathbf p} E_{\mathbf p}}} \bra{0} \Big ( \hat a_{\mathbf p} \hat a_{\mathbf q} \exp (- i p x - i q y) + \hat a_{\mathbf p}^\dagger \hat a_{\mathbf q} \exp (i p x - i q y) \\ & \qquad + \hat a_{\mathbf p} \hat a_{\mathbf q}^\dagger \exp (- i p x + i q y) + \hat a_{\mathbf p}^\dagger \hat a_{\mathbf p}^\dagger \exp (i p x + i q y) \Big) \ket{0} \\ & = \int \frac{d^3 p ~ d^3 q}{(2\pi)^6} \frac{1}{2 \sqrt{E_{\mathbf p} E_{\mathbf p}}} \Big ( \bra{0} \hat a_{\mathbf p} \underbrace{\hat a_{\mathbf q} \ket{0}}_0 \exp (- i p x - i q y) + \bra{0} \hat a_{\mathbf p}^\dagger \underbrace{\hat a_{\mathbf q} \ket{0}}_0  \exp (i p x - i q y) \\ & \qquad + \bra{0} \underbrace{\hat a_{\mathbf p} \hat a_{\mathbf q}^\dagger}_{[\hat a_{\mathbf p}, \hat a_{\mathbf q}^\dagger] + \hat a_{\mathbf q}^\dagger \hat a_{\mathbf p}} \ket{0} \exp (- i p x + i q y) + \underbrace{\bra{0} \hat a_{\mathbf p}^\dagger}_0 \hat a_{\mathbf p}^\dagger \ket{0}  \exp (i p x + i q y) \Big) \\ & = \int \frac{d^3 p ~ d^3 q}{(2\pi)^6} \frac{1}{2 \sqrt{E_{\mathbf p} E_{\mathbf p}}} \Big (\bra{0} \underbrace{[\hat a_{\mathbf p}, \hat a_{\mathbf q}^\dagger]}_{(2\pi)^3 \delta (\mathbf p - \mathbf q)} \ket{0} \exp (- i p x + i q y) + \bra{0} \hat a_{\mathbf q}^\dagger \underbrace{\hat a_{\mathbf p} \ket{0}}_0 \exp (- i p x + i q y) \Big ) \\ & = \int \frac{d^3 p ~ d^3 q}{(2\pi)^3} \frac{1}{2 \sqrt{E_{\mathbf p} E_{\mathbf p}}} 
            \underbrace{\delta (\mathbf p - \mathbf q)}_{\mathbf p = \mathbf q} \underbrace{\braket{0}{0}}_1 \exp (- i p x + i q y) \\ & = \int \frac{d^3 p}{(2\pi)^3} \frac{1}{2 E_{\mathbf p}} \exp (- i p (x-y)) ~.
        \end{aligned}
        \end{equation*}
    \end{proof}

    Outside the light cone, it does not vanish.
    \begin{proof}
        For spacelike separation, like $(x-y) = (0, \mathbf r)$, we have 
        \begin{equation*}
            D(x-y) = \frac{1}{(2\pi)^2 r} \int_m^\infty dy ~ \frac{y \exp(- yr)}{\sqrt{y^2 - m^1}} ~,
        \end{equation*}
        or with asymptotic behaviour at infinity 
        \begin{equation*}
            D(x-y) \xrightarrow{r \rightarrow \infty} \propto \exp(-mr) ~,
        \end{equation*}
        which means that the propagators decays exponentially quickly outside the light cone but it does not vanish.
    \end{proof}

    However, causality is not violated because even though the propagator is non-zero, the commutator $\Delta (x-y)$ is so and it can be written as 
    \begin{equation*}
        \Delta (x-y) = D(x-y) - D(y-x) ~,
    \end{equation*}
    which means that there is a decostructive interference, with the Feynman interpretation as 
    \begin{equation*}
        \Delta (x-y) = \bra{0} \hat \varphi (x) \hat \varphi (y) \ket{0} - \bra{0} \hat \varphi (y) \hat \varphi (x) \ket{0} ~,
    \end{equation*}
    where the first term is the probability amplitude to go from $\mathbf y$ to $\mathbf x$ at a speed larger than that of light and the latter term is the probability amplitude to go from $\mathbf x$ to $\mathbf y$ at a speed larger than that of light. The physical intuition, for a complex Klein-Gordon field, is a particle and an antiparticle that travels backward in time
    \begin{equation*}
        \Delta (x-y) = \bra{0} \hat \varphi (x) \hat \varphi^* (y) \ket{0} - \bra{0} \hat \varphi^* (y) \hat \varphi (x) \ket{0} ~.
    \end{equation*} 
    
    
    
