\documentclass[a4paper, 12pt]{memoir}

\usepackage[a4paper, top = 4cm, bottom = 4cm, left = 3cm, right = 3cm]{geometry}

\usepackage[T1]{fontenc}
\usepackage[utf8]{inputenc}
\usepackage{pythontex} 
\usepackage{nopageno} 
\usepackage{pgf}

\usepackage{tocloft}
\newcommand{\listequationsname}{List of Equations}
\newlistof{listofequations}{equ}{\listequationsname}
\newcommand{\myequation}[1]{%
	\addcontentsline{equ}{equation}{\protect\numberline{\theequation}#1}\par
}
\makeatletter
\let\l@equation\l@figure
\makeatother

\usepackage{xcolor}
\xdefinecolor{mycolor}{RGB}{0,175,179} 
\usepackage{hyperref}
\hypersetup{colorlinks, linkcolor={mycolor}, citecolor={mycolor}, urlcolor={mycolor}}

\usepackage{lipsum}

\renewcommand{\aftertoctitle}{\afterchaptertitle\par\nobreak\hfill{\normalfont{Page}}\par\nobreak}

\usepackage{titlesec}
\titleformat{\part}[display]
  {\normalfont\HUGE\bfseries\color{mycolor}\centering}
  {Part \thepart}{20pt}{\HUGE\normalfont\color{black}}
\titleformat{\chapter}[display]
  {\normalfont\HUGE\bfseries\color{mycolor}\centering}
  {Chapter \thechapter}{20pt}{\HUGE\normalfont\color{black}}
\titleformat{\section}
  {\normalfont\Large\bfseries\color{mycolor}\centering}
  {\thesection}{1em}{}
\titleformat{\subsection}
  {\normalfont\large\bfseries\color{mycolor}\centering}
  {\thesubsection}{1em}{}

\renewcommand{\printtoctitle}[1]{\HUGE\normalfont\color{black}#1}

\usepackage[backend=bibtex, sorting=none]{biblatex}
\addbibresource{../bibliography.bib}

\usepackage{amsmath}
\usepackage{amsthm}
\usepackage{thmtools}
\usepackage{mathtools}

\newtheorem{principle}{Principle}[chapter]
\newtheorem{lemma}{Lemma}[chapter]
\theoremstyle{definition}
\newtheorem{example}{Example}[chapter]
\renewcommand\qedsymbol{q.e.d.}

\theoremstyle{remark}
\newtheorem{case}{Case}

\newcommand{\dv}[2]{\frac{d#1}{d#2}}
\newcommand{\dvin}[3]{\frac{d#1}{d#2}\Big\vert_{#3}}
\newcommand{\dvd}[2]{\frac{d^2#1}{d#2^2}}
\newcommand{\dvf}[2]{\frac{\delta #1}{\delta #2}}
\newcommand{\pdv}[2]{\frac{\partial#1}{\partial#2}}
\newcommand{\pdvd}[3]{\frac{\partial^2 #1}{\partial#2 \partial#3}}
\newcommand{\pdvdu}[2]{\frac{\partial^2 #1}{\partial#2^2}}
\newcommand{\integ}[3]{\int_{#1}^{#2}d#3~}
\newcommand{\poi}[2]{[#1,~#2]}
\newcommand{\poiexp}[2]{\pdv{#1}{q^i} \pdv{#2}{p_i} - \pdv{#2}{q^i} \pdv{#1}{p_i}}

\newcommand{\comm}[2]{[#1,~#2]}
\newcommand{\set}[2]{\{#1\colon#2\}}
\newcommand{\inner}[2]{\langle#1,~#2\rangle}
\newcommand{\av}[1]{\langle#1\rangle}
\newcommand{\avp}[2]{\langle#1\rangle_{#2}}
\newcommand{\ket}[1]{\vert#1\rangle}
\newcommand{\bra}[1]{\langle#1\vert}
\newcommand{\braket}[2]{\langle#1\vert#2\rangle}

\newtheoremstyle{colored}{}{}{\itshape}{}{\color{mycolor}\normalfont\bfseries\indent}{}{\newline}{}

\declaretheorem[
  style=colored,
  name=Definition,
  numberwithin=chapter,
]{definition}

\declaretheorem[
  style=colored,
  name=Theorem,
  numberwithin=chapter,
]{theorem}

\declaretheorem[
  style=colored,
  name=Corollary,
  numberwithin=chapter,
]{corollary}

\declaretheorem[
  style=colored,
  name=Law,
  numberwithin=chapter,
]{law}

\declaretheorem[
  style=colored,
  name=Principle,
  numberwithin=chapter,
]{princ}

\usepackage{amsfonts}
\usepackage{dsfont}
\usepackage{yfonts}
\usepackage{amssymb}

\let\oldproof\proof
\renewcommand{\proof}{\color{darkgray}\oldproof}

\let\oldexample\example
\renewcommand{\example}{\color{darkgray}\oldexample}

\usepackage{cancel}
\usepackage{indentfirst}

\usepackage{tikz}
\usepackage{amssymb}
\usepackage{pgfplots}
\usepgfplotslibrary{patchplots}
\usetikzlibrary{patterns, positioning, arrows}
\pgfplotsset{compat=1.15}

\DeclareMathOperator{\tr}{tr}
\DeclareMathOperator{\str}{str}
\DeclareMathOperator{\real}{Re}
\DeclareMathOperator{\imm}{Im}
\DeclareMathOperator{\sgn}{sgn}
\DeclareMathOperator{\spann}{span}
\DeclareMathOperator{\vol}{vol}

\usetikzlibrary{positioning, arrows.meta}




\title{quantum field theory I}
\date{\today}

\newcommand{\subt}{second quantisation of free theories}

\begin{document}

\frontmatter

\pagestyle{empty}
{\raggedleft\vspace*{\baselineskip}
{\LARGE Matteo Zandi}\\[0.35\textheight]
{\HUGE \textcolor{mycolor}{\textbf{On~\thetitle:}}}\\[\baselineskip]
{\LARGE \subt }\\[\baselineskip]
{\large \thedate}\par
\vspace*{2\baselineskip}
\vfill
{\large matteo.zandi2@studio.unibo.it}\par
\vspace*{\baselineskip}}
\clearpage
\pagestyle{headings}

\blankpage

\tableofcontents

\mainmatter

\begin{pycode}
import sympy as sy
def plot1(x, f, rangex, rangey, fig, leg, negx, negy):
    rangexx = rangex
    rangeyy = rangey
    if negx == True:
        rangexx = 0
    if negy == True:
        rangeyy = 0
    x = sy.Symbol('x')
    p = sy.plot((f, (x, -rangexx, rangex)), ylim=[-rangeyy, rangey], legend= leg, show=False, line_color='#00AFB3')
    p.save(f'fig/fig{fig}.pgf')
    print(r'\input{fig/fig'+ rf'{fig}' + r'.pgf}')

def plot4(x, f, g, h, l, rangex, rangey, fig, leg, negx, negy):
    rangexx = rangex
    rangeyy = rangey
    if negx == True:
        rangexx = 0
    if negy == True:
        rangeyy = 0
    x = sy.Symbol('x')
    p = sy.plot((f, (x, -rangexx, rangex)), (g, (x, -rangex, rangex)), (h, (x, -rangex, rangex)), (l, (x, -rangex, rangex)), ylim=[-rangeyy, rangey], legend= leg, show=False, line_color='#00AFB3')
    p[3].line_color='black'
    p.save(f'fig/fig{fig}.pgf')
    print(r'\input{fig/fig'+ rf'{fig}' + r'.pgf}')

def der(y, x):
    x = sy.Symbol(x) 
    derivative = sy.diff(y, x)
    return sy.latex(derivative) 
 
def indint(integrand, x): 
    x = sy.Symbol(x) 
    integral = sy.integrate(integrand,x) 
    return sy.latex(integral) 

def defint(integrand, x, min, max): 
    x = sy.Symbol(x) 
    integral = sy.integrate(integrand, (x, min, max)) 
    return sy.latex(integral) 

def infint(integrand, x): 
    x = sy.Symbol(x) 
    integral = sy.integrate(integrand, (x, float('-inf'), float('inf'))) 
    return sy.latex(integral) 

def ode(ode, y, x): 
    x = sy.Symbol(x) 
    y = sy.Function(y) 
    lhs, rhs = ode.split('=') 
    ode = sy.Eq(sy.S(lhs),sy.S(rhs)) 
    sol = sy.dsolve(ode,y(x)) 
    return sy.latex(sol) 

# \py{ode("Derivative(y(x),x,x) + y(x) = 0", "y", "x")} ~.
 
def odeic(ode, y, x, ic): 
    x  = sy.Symbol(x) 
    y  = sy.Function(y) 
    lhs,rhs = ode.split('=') 
    ode = sy.Eq(sy.S(lhs),sy.S(rhs)) 
    sol = sy.dsolve(ode,y(x), ics= sy.S(ic)) 
    return sy.latex(sol) 

#\py{odeic("Derivative(y(x),x,x) + y(x) = 0", "y", "x", "{y(0):1, y(x).diff(x).subs(x, 0): 0}")} ~.

def matrixmult(A, B):
    C = A*B
    return sy.latex(C)

def Taylor(x, f, point, order):
    x = sy.Symbol('x')
    ts = sy.series(f, x, point, order) 
    return sy.latex(ts)

def limit(x, f, point):
    x = sy.Symbol('x')
    lim = sy.limit(f, x, point) 
    return sy.latex(lim)

\end{pycode}

\chapter*{Abstract}

    In these notes, we will study the mathematical framework of quantum field of free theory, i.e.~quadratic without interactions. Quantum fields arise out from second quantisation of classical theories, like Klein-Gordon, Dirac or Maxwell. In the first part, we will introduce the passage from standard quantum mechanics to quantum field theory, with the failures of relativistic quantum mechanics and the example of the mechanical model of a string. In the second part, we will review classical field theories: Euler-Lagrange equations and Noether's theorem. In the third part, we will quantise free theories, focusing on field operators, anticommutation/commutation relations and Noether's charges of quantum fields, describing spin $0$, $1/2$ and $1$ particles.
    
\part{Introduction}

\chapter{Relativistic quantum mechanics}

    In this chapter, we will see why relativistic quantum mechanics is not the right framework and, conversely, why quantum field theory is consistent with relativity and quantum mechanics axioms.

\section{QM + SR}

    Nature is described by quantum mechanics at short distances and by special relativity at large velocities. However, quantum mechanics does not include relativistic effects, since there is no limiting speed and energy used is non-relativistic
    \begin{equation*}
        E_{cm} = \frac{m}{2} v^2 \neq E_{sr} = \sqrt{p^2 c^2 + m^2 c^4} ~.
    \end{equation*}
    On the other hand, special relativity does not include quantum effects, since observables are not operators acting on a Hilbert space and energy is not quantised. 
    
    Recall that quantum mechanics framework consists into promoting observables to operators via canonical commutation relations 
    \begin{equation*}
        [\hat x, \hat p] = i \hbar \Rightarrow \hat p = - i \hbar \dv{}{x} ~, \quad [\hat t, \hat H] = i \hbar \Rightarrow \hat H = - i \hbar \dv{}{t} ~.
    \end{equation*}
    This operators act on an Hilbert space, which is a complex infinite-dimensional linear space endowed with an inner product. As basis of this space, we could use the eigenstates of the energy and the eigenvalues are the exact values of the energy. A system of fixed number of particles is described by a state in this space.

    A first attempt to build a theory compatible with quantum mechanics and relativity could be to use the quantum mechanics framework with the relativistic hamiltonian for the generalised Schroedinger equation
    \begin{equation*}
        i \hbar \pdv{}{t} \psi(t, x) = \hat H \psi(t, x) = \sqrt{\hat p^2 c^2 + m^2 c^4} \psi(t, x) ~.
    \end{equation*}
    This solution fails for three main reasons
    \begin{enumerate}
        \item the number of particle is not fixed and conserved;
        \item it violates causality, i.e.~information cannot travel faster than light;
        \item there is an infinite tower of negative energy states.
    \end{enumerate}

\section{Number of particle is not fixed}

    Experimentally, the number of particles is not conserved. Into colliders, like LHC, there is creation of new particles and destruction of old ones. Into decays processes as well, a particle can decay into more different particles. This happens due to the existence of antiparticles. 

    Heuristically, we can describe a situation in which the number of particles changes. Consider a particle of mass $m$ in a box of length $L$. In order to be inside the box, we must have $\Delta x \sim L$. By the uncertainty principle, $\Delta x \Delta p \sim L \delta p \sim \hbar$, hence, $\Delta p \sim \hbar / L$. In the ultra-relativistic limit, the energy is $E \sim p c$. Therefore, $\Delta E \sim c \Delta p \sim \hbar c / L$ and, if $L \rightarrow 0$, then $E \rightarrow \infty$. But we could have enough energy to produce two particles of mass $m$ from the vacuum, i.e. $\delta E \sim 2 m c^2$, where they must have opposite charge in order to preserve charge conservation, called particles and antiparticles. They would survive only for $\Delta t \sim \hbar / \Delta E$ and they would annihilate into the vaccum. However, for $L \rightarrow 0$ and $E \rightarrow \infty$, there will be more and more particles created. Important effects will happen when $c \hbar / L \sim m c^2$, which means for length of the order of the Compton wavelength
    \begin{equation*}
        \lambda_C \sim \frac{\hbar}{mc^2} .
    \end{equation*}
    This means that at distances shorter than the Compton wavelength, there is a non-zero probability to detect pair creation of a particle and an antiparticle, but states in quantum mechanics have a fixed number of particles and they would not be able to describe this system.

\section{Causality}

    Quantum mechanics violates causality. In fact, if we compute the probability to freely propagate from a point $\mathbf x$ to $\mathbf y$ after a time $t$, we obtain 
    \begin{equation*}
        A_{\mathbf x \rightarrow \mathbf y} (t) = \Big ( \frac{m}{2 \pi i \hbar t} \Big)^{3/2} \exp(\frac{i m}{2 \hbar t} |\mathbf x - \mathbf y|^2) \neq 0 ~,
    \end{equation*}
    which means that for points outside the light cone $|\mathbf x - \mathbf y| \gg ct$, there is a non-zero probability to exchange information. Causality is indeed violated, but it is okay, since we did not assume locality in the axioms of quantum mechanics and there is no limiting speed.
    \begin{proof}
        Given the Hamiltonian 
        \begin{equation*}
            \hat H = \frac{\hat p^2}{2m} ~,
        \end{equation*}
        we find 
        \begin{equation*}
        \begin{aligned}
            A_{\mathbf x \rightarrow \mathbf y} (t) & = \bra{\mathbf y} \exp(- \frac{i}{\hbar} \hat H t) \ket{\mathbf x} = \int \frac{d^3 p}{(2 \pi \hbar)^3} \bra{\mathbf y} \exp(- \frac{i}{\hbar} t \underbrace{\hat H) \ket{\mathbf p}}_{E_{\mathbf p} \ket{\mathbf p}} \underbrace{\braket{\mathbf p}{\mathbf x} }_{\exp(- \frac{i}{\hbar} \mathbf p \cdot \mathbf x)} \\ & = \int \frac{d^3 p}{(2 \pi \hbar)^3} \underbrace{\braket{\mathbf y}{\mathbf p}}_{\exp( \frac{i}{\hbar} \mathbf p \cdot \mathbf y)} \exp(- \frac{i}{\hbar} \frac{p^2}{2m} t)  \exp(- \frac{i}{\hbar} \mathbf p \cdot \mathbf x) \\ & = \int \frac{d^3 p}{(2 \pi \hbar)^3} \exp(- \frac{i}{\hbar} \frac{p^2}{2m} t)  \exp(- \frac{i}{\hbar} \mathbf p \cdot (\mathbf x - \mathbf y)) \\ & = \int \frac{d^3 p}{(2 \pi \hbar)^3} \exp(- \frac{it}{2m\hbar} p^2 - \frac{i}{\hbar} \mathbf p \cdot (\mathbf x - \mathbf y)) ~,
        \end{aligned}
        \end{equation*}
        where we have used the completeness relation for $\mathbf p$, the change of basis and the eigenvalue relation. Now, we use the Gaussian integral 
        \begin{equation*}
            \int_{-\infty}^\infty dx ~ \exp(- a x^2 + bx) = \sqrt{\frac{\pi}{a}}  \exp(\frac{b^2}{4 a}) ~,
        \end{equation*}
        with 
        \begin{equation*}
            a = \frac{i t}{2 m \hbar} ~, \quad b = - \frac{i}{\hbar} |\mathbf x - \mathbf y| ~,
        \end{equation*}
        hence, we find for three Gaussian integrals
        \begin{equation*}
            \begin{aligned}
            A_{\mathbf x \rightarrow \mathbf y} (t) & = \frac{1}{(2 \pi \hbar)^3} \Big ( \frac{\pi 2 m \hbar}{i t} \Big)^{3/2} \exp( \frac{i^2}{\hbar^2} |\mathbf x - \mathbf y|^2 \frac{2 m \hbar}{4 i t}) \\ & = \Big ( \frac{m}{2 \pi i \hbar t} \Big)^{3/2} \exp(\frac{i m}{2 \hbar t} |\mathbf x - \mathbf y|^2) ~.
        \end{aligned}
        \end{equation*}
    \end{proof}

    Relativistic quantum mechanics violates causality as well. In fact, if we compute the probability to freely propagate from a point $\mathbf x$ to $\mathbf y$ after a time $t$, we obtain 
    \begin{equation*}
        A_{\mathbf x \rightarrow \mathbf y} (t) \neq 0 ~,
    \end{equation*}
    which means that for points outside the light cone $|\mathbf x - \mathbf y| \gg ct$, there is a non-zero probability to exchange information. Causality is indeed violated, but it is okay, since we did not assume locality in the axioms of quantum mechanics and there is no limiting speed.
    \begin{proof}
        Given the Hamiltonian 
        \begin{equation*}
            \hat H = \sqrt{\hat p^2 c^2 + m^2 c^4} ~,
        \end{equation*}
        we find, using $\hbar = 1$
        \begin{equation*}
        \begin{aligned}
            A_{\mathbf x \rightarrow \mathbf y} (t) & = \bra{\mathbf y} \exp(- i \hat H t) \ket{\mathbf x} = \int \frac{d^3 p}{(2 \pi )^3} \bra{\mathbf y} \exp(- i t \underbrace{\hat H) \ket{\mathbf p}}_{E_{\mathbf p} \ket{\mathbf p}} \underbrace{\braket{\mathbf p}{\mathbf x} }_{\exp(- i \mathbf p \cdot \mathbf x)} \\ & = \int \frac{d^3 p}{(2 \pi )^3} \underbrace{\braket{\mathbf y}{\mathbf p}}_{\exp( i \mathbf p \cdot \mathbf y)} \exp(- i \sqrt{p^2 c^2 + m^2 c^4} t)  \exp(- i \mathbf p \cdot \mathbf x) \\ & = \int \frac{d^3 p}{(2 \pi )^3} \exp(- i \sqrt{p^2 c^2 + m^2 c^4} t)  \exp(- i \mathbf p \cdot (\mathbf x - \mathbf y)) \\ & = \int \frac{d^3 p}{(2 \pi )^3} \exp(- i t E_{\mathbf p} - i \mathbf p \cdot \mathbf r) ~,
        \end{aligned}
        \end{equation*}
        where we have used the completeness relation for $\mathbf p$, the change of basis, the eigenvalue relation and we have called $r = |\mathbf x - \mathbf y|$. Now, we use the polar coordinates in momentum space $(p, \theta, \phi)$
        \begin{equation*}
        \begin{aligned}
            A_{\mathbf x \rightarrow \mathbf y} (t) & = \frac{2\pi}{(2 \pi^3} \int_0^\infty dp ~p^2 \int_0^\pi \underbrace{d\theta ~ \sin\theta}_{d (- \cos\theta)} \exp(- i t E_{\mathbf p} - i p r \cos \theta) \\ & = \frac{1}{4 \pi^2} \int_0^\infty dp ~p^2 \int_{-1}^1 d (\cos\theta) ~ \exp(- i t E_{\mathbf p} - i p r \cos \theta) \\ & = \frac{1}{4 \pi^2} \int_0^\infty dp ~p^2 \frac{\exp(- i t E_{\mathbf p} + i p r \cos \theta) }{i p r} \Big \vert_{\cos \theta = -1}^{\cos \theta = 1} \\ & = \frac{1}{4 \pi^2 i r} \int_0^\infty dp ~ p (\exp(i p r) - \exp(- i p r))\exp(- i t E_{\mathbf p}) \\ & = \frac{1}{4 \pi^2 i r} \int_{-\infty}^\infty dp ~ p \exp(i p r - i t E_{\mathbf p}) ~,
        \end{aligned}
        \end{equation*}
        where we have exchanged $p \rightarrow -p$ in order to have the integration domain from $-\infty$ to $\infty$. The result of this intergal can be expressed in terms of the Bessel functions. However, we will make an estimate complexificating $p$ and integrating in the comples plane. Our integral becomes 
        \begin{equation*}
            A_{\mathbf x \rightarrow \mathbf y} (t) = frac{1}{4 \pi^2 i r} \int_{-\infty}^\infty dp ~ p \exp(i p r - i t \sqrt{(p + im) (p - im)}) ~,
        \end{equation*}
        with two branch cuts in $[-i\infty, -im]$ and $[im, i\infty]$. Using the Cauchy theorem, we draw a countour integral and we decomposed it into $C = C_1 + C_2 + C_3 + C_4 + C_L + C_\epsilon$, therefore the integral on the real axis is given by the limit of $\epsilon \rightarrow 0$ and $L \rightarrow \infty$ of
        \begin{equation*}
            \int_{C_L} = - \int_{C_1} - \int_{C_2} - \int_{C_3} - \int_{C_4} - \int_{C_\epsilon} ~.
        \end{equation*}
        For $C_\epsilon$, we use the Darboux inequality to find 
        \begin{equation*}
            |\int_{C_\epsilon} dp ~ f(p)| \leq L_{C_\epsilon} \sup_{p \in C_\epsilon} |f(p)| \simeq \frac{2\pi m \epsilon}{4 \pi^2 r}  \exp(-mr) \exp(- it \sqrt{2im \epsilon i \theta_{max}}) \xrightarrow{\epsilon \rightarrow 0} 0 ~,
        \end{equation*}
        where we have used $p = \epsilon \exp(i \theta)$. 
        For $C_4$, we have 
        \begin{equation*}
        \begin{aligned}
            & \exp(ipr) \exp(- i \sqrt{p^2 + m^2}t )  \\ & = \exp(i \real (p) r - \real (\sqrt{p^2 - m^2}t)) \exp(-i \imm (p) r + \imm (\sqrt{p^2 - m^2}t)) \\ & = \exp(i \real (p) r - \real (\sqrt{p^2 - m^2}t)) \exp(-i \imm (p) r - |\imm (\sqrt{p^2 - m^2}t|)) \xrightarrow{p \rightarrow \infty} 0 ~.
        \end{aligned}
        \end{equation*}
        Now, we make an approximation by considering only points outside the light cone with $r \gg t$. Hence, we find 
        \begin{equation*}
            \exp(i p r) \exp(- i \sqrt{p^2 + m^2}t) = \exp(i \theta) \exp(- \imm (p) r) \exp(\imm (\sqrt{p^2 + m^2}) t) \xrightarrow{p \rightarrow \infty} 0~.
        \end{equation*}
        It remains only $C_2$ and $C_3$, which with a change of variable $p dp = - y dy$ we obtain
        \begin{equation*}
        \begin{aligned}
            A_{\mathbf x \rightarrow \mathbf y} (t) & = \frac{i}{4 \pi^2 r} \int_m^\infty dy ~ y \exp(- y r) (\exp(\sqrt{y^2 - m^2})t - \exp(\sqrt{y^2 - m^2})t)\\ &  = \frac{i}{2 \pi^2 r} \int_m^\infty dy ~ y \exp(- y r) \sinh (\sqrt(y^2 - m^2) t) ~,
        \end{aligned}
        \end{equation*}
        where there is a minus sign in the second exponential because we are on the left of the branchcut. However, notice that for $y \geq m$ we have $\exp(-yr) \geq \exp(-mr)$ and $\sinh (\sqrt(y^2 - m^2) t) \geq 0$, therefore
        \begin{equation*}
            y \exp(-yr) \sinh(\sqrt{y^2 - m^2} t) \geq 0 
        \end{equation*}
        and 
        \begin{equation*}
            A_{\mathbf x \rightarrow \mathbf y} (t) \neq 0 ~.
        \end{equation*}
    \end{proof}

\section{Unites and scales}

    In theroetical physics, it is useful to change units from the international system to natural units. In the description of natural phenomena, we find $3$ fundamental constants 
    \begin{enumerate}
        \item $c$, speed of light;
        \item $\hbar$, Planck's constant;
        \item $G_N$, Newton's constant.
    \end{enumerate}
    Their dimensional analysis is 
    \begin{equation*}
        [c] = [L][T]^{-1} ~, \quad [\hbar] = [M] [L]^2 [T]^{-1} ~, \quad [G_N] = [L]^3 [M]^{-1} [T]^{-2} ~.
    \end{equation*}
    \begin{proof}
        For $c$, which is a velocity,
        \begin{equation*}
            [c] = [L][T]^{-1} ~.
        \end{equation*}
        For $\hbar$, which is an action,
        \begin{equation*}
            [c] = [E][T] = [M] [L]^2 [T]^{-2} [T] = [M] [L]^2 [T]^{-1}  ~.
        \end{equation*}
        For $G_N$, which is an energy per length divided by mass square,
        \begin{equation*}
            [G_N] = [E] [L] [M]^{-2} = [M] [L]^2 [T]^{-2} [L] [M]{-2} = [L]^3 [M]^{-1} [T]^{-2} ~.
        \end{equation*}
    \end{proof}
    Now, we introduce the natural units, in which everything is measured in masses (or equivalently in length)
    \begin{equation*}
        \hbar = c = 1 ~.
    \end{equation*}
    Therefore 
    \begin{equation*}
        [L] = [T] = [E]^{-1} = [M]^{-1} ~.
    \end{equation*}
    The Newton's constant is 
    \begin{equation*}
        [G] = [M]^{-2} ~.
    \end{equation*}

    Combining the three constants, we can define important quantities
    \begin{enumerate}
        \item Planck's mass \begin{equation*}
            m_P \sim 10^{19} GeV ~,
        \end{equation*}
        \item Planck's length \begin{equation*}
            l_p = \frac{1}{M_p} \sim 10^{-35} m ~,
        \end{equation*}
    \end{enumerate}
    where we have $1 Gev = 10^{16} m^{-1}$.

    Relevant energy scales in Nature are 
    \begin{enumerate}
        \item $m_P \sim 10^{19} GeV$, 
        \item $E_{GUT} \sim 10^{16} GeV$, 
        \item $E_{LHC}\sim 10^{13} GeV$, 
        \item $m_{top} \sim 170 GeV$, 
        \item $m_{Higgs} \sim 125 GeV$, 
        \item $m_{W, Z} \sim 90 GeV$, 
        \item $m_{\mu} \sim 0.1 GeV$, 
        \item $m_{e} \sim 10^{-3} GeV$, 
        \item $m_{\nu} \sim 10^{-11} GeV$, 
        \item $H_0 \sim 10^{16} GeV$.
    \end{enumerate}

\section{Quantum field theory}

    By the considerations made in the previous sections, relativistic quantum mechanics is not good. We need to change paradigm and introduce the new framework of quantum field theory. Classical fields, like the electromagentic and the gravitational one, were introduced to avoid action at a distance and make laws of Nature local. Looking at photons, fields seem the more fundamental quantity, whereas looking at electrons, particles seem the more fundamental quantity. In quantum field theory, fundamental quantities are quantum fields and each particle emerges as a quanta/oscillation of its own associated field, which is spread everywhere in the spacetime. This means that if there is nothing, it is in the groud state, whereas if there is the particle, it is in an excited state. Examples are photons for the electromagentic field, electron for the electronic field and Higgs bosons for the Higgs field. This new point of view enesures locality, since interactions are local; causality, since nothing can travel faster than light; fields can describe systems with arbitrary number of particles, since there is creation and annihilation of particles and antiparticles and fields has infinite number of degrees of freedom. For example, they can describe decay processes in which outcoming fields are excitated and incomping ones are put in ground state. There is a change also in the coordinates in which we describe the system: in quantum mechanics, position and time were the finite number of degrees of freedom, whereas, in quantum field theory, field configuration are the infinite number of degrees of freedom, since they are functions of the spacetime coordinates. In fact, $t$ and $\mathbf x$ are only labels to indicate which field of the infinite continuous number that there are. Fields are therefore promoted to operators acting on a Fock space $\hat \psi (t, \mathbf x)$ via generalised commutations relations. 
    Moreover, there are other features that emerge from this description. All particles of the same kind are identical, even if they are produced far far away, since they are quanta of the same field. Protons produced in a supernova billion light years away are the same produced on Earth. This implies that to be more fundamental is the field underneath them. In quantum mechanics, the correct spin-statistics relations are imposed by hand, whereas in quantum field theory, they emerge from the theory. Given a wavefunction which described $n$ identical particles $\psi(x_1, \ldots x_n)$, we can have two alternatives 
    \begin{enumerate}
        \item $\psi$ is symmetric 
            \begin{equation*}
                \psi(\ldots, x_i, \ldots, x_j, \ldots) = \psi(\ldots, x_j, \ldots, x_i, \ldots) ~,
             \end{equation*}
            which is the Bose-Einstein statistics and it happens for integer-spin particles, called bosons;
        \item $\psi$ is antisymmetric 
            \begin{equation*}
                \psi(\ldots, x_i, \ldots, x_j, \ldots) = - \psi(\ldots, x_j, \ldots, x_i, \ldots) ~,
             \end{equation*}
            which is the Fermi-Dirac statistics and it happens for half integer-spin particles, called fermions;
    \end{enumerate}
    In quantum field theory, this is realised by imposing commutation relations fo bosons and anticommutation relations for fermions in order to maintain consistency in the theory.

\chapter{Mechanical model of a quantum string}

    Consider an elastic string, which is the continuum limit of a $1$-dimensional lattics of $N$ atoms. Our discrete model is therefore composed by $N$ atoms of a discrete string of mass $m$ and they are coupled via a spring of elastic constant $k$. This preserve locality, since eacj only interacts with its own neighbourhood. 

    An example can be a metastable system, for which the equilibrium position is $x = 0$, i.e. $F(x = 0) = 0$ and it can be Taylor expanded into 
    \begin{equation}
        F(x) = F(x = 0) + \underbrace{\dv{F}{x} \Big \vert_{x = 0}}_{-k} x + O(x^2) = - k x ~.
    \end{equation}

    Let us focus on the displacement along the $y$-axis with fixed lattice spacing $\Delta$ in $x$. The equilibrium points are $y_j = 0$ and the system can be viewed as a system of $N$ coupled harmonic oscillators. In the continuum limit $\Delta \rightarrow 0$ and $ N \rightarrow \infty$, the $y$-displacements $u_j (t)$ become a field $\phi(t, x)$.

\section{Classical string}

    The lagrangian of the system is 
    \begin{equation*}
        L = \sum_{j = 1}^{N} \Big ( \frac{m}{2} \dot y_j^2 (t) - \frac{k}{2} (y_j(t) - y_{j+1} (t) )^2 \Big) ~,
    \end{equation*}
    where the displacement with resepct to the equilibrium position is $y_j(t) - y_{j+1} (t)$. Furthermore, we impose periodic boundary condition, for which $y_{j + N} (t) = y_{j} (t)$. We have a closed string with translational invariance for $j \rightarrow j + N$.

    We assume the regime of small oscillations, i.e. 
    \begin{equation*}
        \frac{y_j - y_{j+1}}{\Delta} \ll 1 ~.
    \end{equation*}

    The lagrangian becomes 
    \begin{equation*}
    \begin{aligned}
        L & = \sum_{j = 1}^{N} \Big ( \frac{m}{2} \dot y_j^2 (t) - \frac{k}{2} (y_j(t) - y_{j+1} (t) )^2 \Big) \\ & = \sum_{j = 1}^{N} \Big ( \frac{m}{2} \dot y_j^2 (t) - \frac{k \Delta}{2} (\frac{y_j(t) - y_{j+1} (t) }{\Delta})^2 \Big) \\ & = \sum_{j = 1}^{N} \frac{m}{2}\Big ( \dot y_j^2 (t) - v^2 (\frac{y_j(t) - y_{j+1} (t) }{\Delta})^2 \Big) ~,
    \end{aligned}
    \end{equation*}
    where we have defined 
    \begin{equation*}
        \tilde k = k \Delta^2 = m v^2 ~,
    \end{equation*}
    since $[\tilde K] = [E]$.

    The equations of motion are 
    \begin{equation*}
        \ddot y_j (t) = - \frac{v}{\Delta^2} ( 2 y_j(t) - y_{j+1} (t) - y_{j-1}(t)) ~.
    \end{equation*}
    \begin{proof}
        Using the Euler-Lagrange equations and noticing that in the sum $y_j$ compare where the index is $j - 1$ and $j$, we have
        \begin{equation*}
        \begin{aligned}
            0 & = \pdv{L}{y_j} - \dv{}{t} \pdv{L}{\dot y_j} \\ & = - \frac{v^2}{\Delta^2} \pdv{}{y_j} ((y_{j-1} - y_j)^2 + (y_j - y_{j+1})^2 ) - 2 m \ddot y_j \\ & = - \frac{ v^2}{\Delta^2} ( - 2 (y_{j-1} - y_j) + 2 (y_j - y_{j+1})) - 2 m \ddot y_j \\ & = - \frac{v^2}{\Delta^2} (4 y_j - 2 y_{j-1} - 2 y_{j+1}) - 2 m \ddot y_j ~, 
        \end{aligned}
        \end{equation*}
        hence 
        \begin{equation*}
            \ddot y_j = - \frac{v}{\Delta^2} ( 2 y_j(t) - y_{j+1} (t) - y_{j-1}(t)) ~.
        \end{equation*}
    \end{proof}

    In order to solve this problem containing $N$ coupled harmonic oscillators, we used the discrete Fourier transform
    \begin{equation*}
        y_j (t) = \frac{1}{\sqrt{N}} \sum_{s = 1}^N \exp(i \frac{2\pi}{N} s j) \tilde y_s(t) ~.
    \end{equation*}
    We obtained a system of $N$ decoupled simple harmonic oscillators, one for each $s$,
    \begin{equation*}
        \ddot{\tilde y_s} (t) = - \Big ( \frac{2 v}{N} \sin \frac{\pi s}{N}\Big)^2  \tilde y_s (t) ~,
    \end{equation*}
    of $N$ different frequencies $\omega_s = 2 v / \Delta \sin (\pi s /N)$. The equations of motion are 
    \begin{equation*}
        \tilde y_s (t) + \omega^2_s y_s (t) = 0 ~.
    \end{equation*}
    \begin{proof}
        Maybe in the future.
    \end{proof}

    Therefore, the solution for each $s$ is 
    \begin{equation*}
        \tilde y_s (t) = A_s \exp(- i \omega_s t) 
    \end{equation*}
    and the total solution is a linear combinations of sound waves 
    \begin{equation*}
        y_j (t) = \frac{1}{\sqrt{N}} \sum_{s = 1}^N \exp(i (\frac{2\pi}{N} s j - \omega_s t)) ~,
    \end{equation*}
    with $N$ discrete wave numbers $k_s = 2 \pi s / N \Delta = 2 \pi / \lambda_s$ or $N$ discrete wavelength $\lambda_s = N \Delta / s$, which are $N\Delta, N\Delta/2, \ldots$. Each sound waves has different velocity 
    \begin{equation*}
        v_s = \frac{\omega_s}{k_s} = \frac{\frac{2 v }{\Delta} \sin \frac{\pi s}{N}}{\frac{2 \pi s}{N \Delta}} = \frac{v N}{\pi s} \sin \frac{\pi s}{N} = v \frac{\sin \frac{\pi s}{N}}{\frac{\pi s}{N}} ~.
    \end{equation*}
    \begin{proof}
        Maybe in the future.
    \end{proof}

    In the continuum limit, $j \Delta \rightarrow x$, $k \rightarrow p$, $v_s \rightarrow v$.

    We have $N-1$ degrees of freedom, because there is $\omega_{s = N} = 0$. However, we can exploit the symmetries of the sine functions to reduce $2$ real degrees of freedom into $1$ complex degree of freedom. $y$ must be real but $\tilde y$ is complex such that $y = \exp(i \ldots) \tilde y$ is real. In fact, $\tilde y_s$ and $\tilde y_{N - s}$ have the same frequency. Therefore 
    \begin{equation*}
        y_j (t) = \frac{1}{\sqrt{N}} \sum_{s = 1}^{N/2 -1} \Big ( \exp(\frac{2\pi}{N} j s) \tilde y_s + \exp(- \frac{2\pi}{N} j s) \tilde y_{N - s} \Big) ~,
    \end{equation*}
    \begin{equation*}
        y_j^* (t) = \frac{1}{\sqrt{N}} \sum_{s = 1}^{N/2 -1} \Big (- \exp(\frac{2\pi}{N} j s) \tilde y_s^* + \exp(\frac{2\pi}{N} j s) \tilde y_{N - s}^* \Big) ~.
    \end{equation*}
    \begin{proof}
        Maybe in the future.
    \end{proof}
    The conditions in order to have $y$ real, i.e. $y = y^*$, are 
    \begin{equation*}
        \tilde y_s = \tilde y_{N-s}^* ~, \quad \tilde y_s^* = \tilde y_{N-s} ~.
    \end{equation*}
    \begin{proof}
        Maybe in the future.
    \end{proof}

    We could have use directly the lagrangian, and diagonalise it to decouple the equations of motion, to obtain the same results 
    \begin{equation*}
        K = \frac{1}{2} m \sum_{j=1}^{N} \dot y_j^2 = \frac{1}{2} m \sum_{s = 1}^{N/2 - 1} 2 \dot{\tilde y_s} \dot{\tilde y_{N-s}} = m \sum_{s=1}^{N/2 - 1} |\dot{\tilde y}|^2 ~,
    \end{equation*}
    \begin{equation*}
        V = \frac{1}{2} \frac{m v^2}{\Delta^2} \sum_{j=1}^{N} (y_j - y_{j-1})^2 = m \sum_{s=1}^{N/2 - 1} \omega^2_s |\tilde y|^2 ~.
    \end{equation*}
    \begin{proof}
        Maybe in the future.
    \end{proof}

\section{Harmonic oscillator}

    Now, we quantise the system, which means to quantise $N$ decoupled harmonic oscillators. Recall the simple quantum harmonic oscillator of $m = 1$. The hamiltonian is 
    \begin{equation*}
        \hat H = \frac{\hat p^2}{2} + \frac{\omega^2}{2} \hat y^2 ~,
    \end{equation*}
    such that 
    \begin{equation*}
        [\hat y, \hat p] = i ~.
    \end{equation*}
    The equilibrium point is $y = 0$. The creation and annihilation operators are 
    \begin{equation*}
        \hat a = \sqrt{\frac{\omega}{2}} \hat y + \frac{i}{\sqrt{2\omega}} \hat p ~, \quad \hat a^\dagger = \sqrt{\frac{\omega}{2}} \hat y - \frac{i}{\sqrt{2\omega}} \hat p ~,
    \end{equation*}
    which inverted looks like 
    \begin{equation*}
        \hat y = \frac{\hat a + \hat a^\dagger}{\sqrt{2 \omega}} ~, \quad \hat p = i \sqrt{\frac{\omega}{2}} (\hat a - \hat a^\dagger) ~,
    \end{equation*}
    such that 
    \begin{equation*}
        [\hat a, \hat a^\dagger] = 1~.
    \end{equation*}
    Therefore, the hamiltonian becomes 
    \begin{equation*}
        \hat H = \omega (\hat a^\dagger \hat a + \frac{1}{2}) = \omega (\hat n + \frac{1}{2}) ~,
    \end{equation*}
    where the number operator is $\hat n$. The zero point energy (ground state) is $\omega/2$ and a generic $n$ point energy is $n \omega /2$. The commutation relations are
    \begin{equation*}
        [\hat H, \hat a^\dagger] = \omega \hat a^\dagger ~, \quad [\hat H, \hat a] = - \omega \hat a ~.
    \end{equation*}
    The eigenstates of the number operator are the same of the hamiltonian 
    \begin{equation*}
        \hat H \ket{n} = E_{n} \ket{n} ~,
    \end{equation*}
    therefore 
    \begin{equation*}
        \hat H \hat a^\dagger \ket{n} = [\hat H, \hat a^\dagger] \ket{n} + \hat a^\dagger \hat H \ket{n} = (\omega + E_n) \hat a^\dagger \ket{n} ~.
    \end{equation*}
    This means that 
    \begin{equation*}
        \hat a^\dagger \ket{n} = c_{n+1} \ket{n+1} ~, \quad \hat a \ket{n} = c_{n-1} \ket{n-1} ~.
    \end{equation*}
    For the ground state, we have $\hat a \ket{0} = 0$, whereas for a generic excited state $(\hat a^\dagger)^n \ket{0} = \sqrt{n!} \ket{n}$.

\section{Quantum string}

    To quantise, we decompose $\tilde y_s$ into its real and complex part 
    \begin{equation*}
        \tilde y_s = \frac{1}{\sqrt{2}} (\real \tilde y_s + i \imm \tilde y_s) ~,
    \end{equation*}
    and we promote them to operators 
    \begin{equation*}
        \hat{(\real \tilde y_s)} = \frac{1}{\sqrt{\omega_s}} (\hat a_s^{(R)} + \hat a_s^{\dagger(R)}) ~, \quad \hat{(\imm \tilde y_s)} = \frac{1}{\sqrt{\omega_s}} (\hat a_s^{(I)} + \hat a_s^{\dagger(I)}) ~,
    \end{equation*}
    \begin{equation*}
        \hat p_s^{(R)} = - i \sqrt{\frac{\omega_s}{2}} (\hat a_s^{(R)} - \hat a_s^{\dagger(R)}) ~, \quad \hat p_s^{(I)} = - i \sqrt{\frac{\omega_s}{2}} (\hat a_s^{(I)} - \hat a_s^{\dagger(I)}) ~,
    \end{equation*}
    where $\omega_s = 2 v / \Delta \sin (\pi s / N)$ for $s = 1, \ldots N - 1 /2$. 

\section{Fock space}

    There are several Hilbert spaces. Each quantum harmonic oscillator has its own Hilbert space spanned by the eigenstates of the hamiltonian and created by the action on the vacuum of the creation operators. These quantised excitation of vibrations of a solid (string) are called quasi-particles or phonons. The total space is called Fock space 
    \begin{equation*}
        \mathcal F = \sum_{n = 1}^\infty \mathcal H = \mathcal H_1 \oplus \ldots \oplus \mathcal H_n ~,
    \end{equation*}
    where 
    \begin{equation*}
        \mathcal H_n = \otimes_{n} \mathcal H_1 = \mathcal H_1 \otimes \ldots \otimes \mathcal H_1 ~.
    \end{equation*}
    Recall that in $\oplus$ we take the same of the axis and in $otimes$ we take all the combinations. In our case 
    \begin{equation*}
        \mathcal H_1 = \oplus_{s = 1}^{N/2 - 1} \mathcal H_{1, s}
    \end{equation*}
    where each phonons can have frequency $\omega_s$. We can build our Fock space with the ladder operators. $\mathcal H_1$ is spanned by the states $\hat a_i^{\dagger(R, I)} \ket{0} = \hat a_i^\dagger \ket{0}$. $\mathcal H_2$ is spanned by the states $\hat a_i^\dagger \hat a_j^\dagger \ket{0}$ and $\hat a_j^\dagger \hat a_i^\dagger \ket{0}$ which in general are not the same. $\mathcal H_N$ is spanned by the states $(\hat a_{i_1}^\dagger)^{n_1}  \ldots (\hat a_{i_l}^\dagger)^{n_l} \ket{0}$ and $\hat a_j^\dagger \hat a_i^\dagger \ket{0}$ with constrain $n_1 + \ldots n_l = n$.

    The vacuum is defined as the state with no phonons, in which all the harmonic oscillators are in the ground state
    \begin{equation*}
        \ket{0, \ldots 0} = \ket{0}_{\omega_1}^{(R)} \otimes \ket{0}_{\omega_1}^{(I)} \otimes \ldots \otimes \ket{0}_{\omega_{N/2-1}}^{(R)} \otimes \ket{0}_{\omega_{N/2-1}}^{(I)} ~.
    \end{equation*}
    Hence, for example
    \begin{equation*}
        \ket{1, 0, \ldots 0} = \hat a^{\dagger(R)}_1 \ket{0} ~, \quad \ket{1, 0, 1, 0 \ldots 0} = \hat a^{\dagger(R)}_1 \hat a^{\dagger(R)}_2 \ket{0} ~.
    \end{equation*}
    An arbitrary basis element can be written as 
    \begin{equation*}
        \ket{n_1^{(R)}, n_1^{(I)}, \ldots, n_{N/2-1}^{(R)}, n_{N/2-1}^{(I)}} = C (\hat a_1^{\dagger (R)})^{n_1^{(R)}} (\hat a_1^{\dagger (R)})^{n_1^{(R)}} \ldots (\hat a_{N/2-1}^{\dagger (R)})^{n_{N/2-1}{(R)}} (\hat a_{N/2-1}^{\dagger (R)})^{n_{N/2-1}^{(R)}} \ket{0} ~,
    \end{equation*}
    where $\mathcal C$ is a normalisation constant.
    An arbitrary state of the Fock state can be written as 
    \begin{equation*}
        \ket{\psi} = \sum_{n_1^{(R)} = 1}^\infty \sum_{n_1^{(I)} = 1}^\infty \ldots \sum_{n_{N/2-1}^{(R)} = 1}^\infty \sum_{n_{N/2-1}^{(I)} = 1}^\infty C_{n_1^{(R)}} C_{n_1^{(I)}} \ldots C_{n_{N/2-1}^{(R)}} C_{n_{N/2-1}^{(I)}} \ket{n_1^{(R)}, n_1^{(I)}, \ldots, n_{N/2-1}^{(R)}, n_{N/2-1}^{(I)}} ~,
    \end{equation*}
    where $|C_{n_1^{(R)}} C_{n_1^{(I)}} \ldots C_{n_{N/2-1}^{(R)}} C_{n_{N/2-1}^{(I)}}|^2$ gives the probability that a quantum elastic fiscrete string is in a state with $n_1^{(R)} + n_1^{(I)}$ phonons with frequency $\omega_1$, $n_2^{(R)} + n_2^{(I)}$ phonons with frequency $\omega_2$, etc. The normalisation condition holds 
    \begin{equation*}
        ||\ket{\psi}||^2 = \braket{\psi}{\psi} = \sum_{n_1^{(R)} = 1}^\infty \sum_{n_1^{(I)} = 1}^\infty \ldots \sum_{n_{N/2-1}^{(R)} = 1}^\infty \sum_{n_{N/2-1}^{(I)} = 1}^\infty |C_{n_1^{(R)}} C_{n_1^{(I)}} \ldots C_{n_{N/2-1}^{(R)}} C_{n_{N/2-1}^{(I)}}|^2 = 1 ~.
    \end{equation*}
    The hamiltonian becomes 
    \begin{equation*}
        \hat H = \sum_{s=1}^{N/2-1} \omega_s (\hat a^{\dagger(R)}_s \hat a^{(R)}_s + \hat a^{\dagger(I)}_s \hat a^{(I)}_s + 1) = \sum_{s=1}^{N/2-1} \omega_s (\hat N^{(R)}_s + \hat N^{(I)}_s + 1) = \sum_{s=1}^{N/2-1} \omega_s (\hat N_s + 1) ~,
    \end{equation*}
    where $\hat N^{(R)}_s + \hat N^{(I)}_s = \hat N_s$. Therefore, the vacuum energy is 
    \begin{equation*}
        E_0 = \sum_{s=1}^{N/2-1} \omega_s ~.
    \end{equation*}

\section{Continuum limit}

    Now, we go into the continuum limit, for which $\Delta \rightarrow 0$ and $N \rightarrow \infty$. The following quantities become continuum variables 
    \begin{equation*}
        k_s \rightarrow k ~, \quad \lambda_s = \frac{2\pi}{k_s} \rightarrow \lambda = \frac{2\pi}{k}~, \quad j \Delta \rightarrow x ~, \quad y_j(t) \rightarrow \phi(t, x) ~.
    \end{equation*}
    Furthermore, 
    \begin{equation*}
        v_s = \frac{\omega_s}{k_s} = v_s \sin(\frac{\pi s}{N}) \frac{N}{\pi s} \rightarrow v \frac{\pi s}{N} \frac{N}{\pi s} = v ~.
    \end{equation*}
    Upon quantisation, we find phonons (quanta of excitations of elestic string) characterised by $\omega = v k$. If $v = c$, we find 
    \begin{equation*}
        \omega = c k ~, \quad \hbar \omega = c \hbar k ~, \quad E_p = \hbar p ~,
    \end{equation*}
    which means that each phonon is a photon, since it is the energy relation for massless particles. There are infinitely many decoupled harmonic oscillators each for a value of $p$.

\section{Particle intepretation}

    $\phi(t, x)$ can be seen as the field associated to a massless relativistic particle upon quantisation part of the momentum $p = \hbar k$, which emerges as quanta of excitations over its own ground state. $\hat a^\dagger_k$ and $\hat a_k$ are ladder operators of a particle with momentum $p$. The mass can be obtained by means of the addition of springs with respect to $y_j = 0$ with elastic constant $k_\mu$. The lagrangian becomes 
    \begin{equation*}
        L = \sum_{j = 1}^{N} \Big ( \frac{m}{2} \dot y_j^2 (t) - \frac{k}{2} (y_j(t) - y_{j+1} (t) )^2 - \frac{1}{2} k_\mu y_j^2(t) \Big) = \frac{m}{2} \Big ( \dot y_j^2 (t) - v^2 (\frac{y_j(t) - y_{j+1} (t)}{\Delta} )^2 - \frac{1}{2} \frac{k_\mu}{m} y_j^2(t) \Big) ~,
    \end{equation*}
    which in the continuum limit, for $v = 1$, we have
    \begin{equation*}
        \sum_{j = 1}^{N} \rightarrow \int dx ~, \quad y_j \rightarrow \phi ~. \quad \partial_0 \phi = \dot y_j \rightarrow \dot \phi ~, \quad \frac{y_j(t) - y_{j+1} (t)}{\Delta} \rightarrow \partial_x \phi ~,
    \end{equation*}
    hence 
    \begin{equation*}
        L = \int dx (\frac{m}{2} \partial_0 \phi \partial_0 \phi - \frac{m}{2} \partial_x \phi \partial_x \phi  - \frac{1}{2} k_\mu \phi^2 ) = \int dx \mathcal L~.
    \end{equation*}
    We calle $\mathcal L$ lagrangian density, which in covariant notation becomes 
    \begin{equation*}
        \mathcal L = \frac{1}{2} m \partial_\mu \phi \partial^\mu \phi - \frac{1}{2} k_\mu \phi^2 =  \frac{1}{2} m \partial_0 \phi \partial^0 \phi - \frac{1}{2} m \partial_x \phi \partial^x \phi- \frac{1}{2} k_\mu \phi^2 ~.
    \end{equation*}
    The equations of motion are 
    \begin{equation*}
        \pdv{\mathcal L}{\phi} - \partial_\mu \pdv{\mathcal L}{\partial_\mu \phi} = m \partial^\mu \partial_\mu \phi + k_\mu = 0 ~,
    \end{equation*}
    \begin{equation*}
        (\Box + \frac{k_\mu }{m}) \phi = 0 ~,
    \end{equation*}
    which in non-covariant notation is 
    \begin{equation*}
        \ddot \phi(t, x) = \phi_{xx} (t, x) - \frac{k_\mu}{m} \phi(t, x) ~.
    \end{equation*}
    It is always a system of infinitely many coupled harmonic oscillators, which can be decoupled via Fourier transform 
    \begin{equation*}
        y_j(t) = \frac{1}{\sqrt{N}} \sum_{j=1}^N \exp(i \frac{2 \pi}{N} s j) \tilde y(t) = \frac{1}{\sqrt{N}} \sum_{j=1}^N \exp(i k_s j \Delta) \tilde y(t) \rightarrow \phi(t, x) = \frac{1}{2\pi} \int dk \exp(i k x) \tilde \phi(t, k) ~.
    \end{equation*}
    The equations of motion becomes 
    \begin{equation*}
        \ddot{\tilde \phi} (t, k) = - (k^2 + \frac{k_\mu}{m}) \tilde \phi(t, k) ~,
    \end{equation*}
    which are a system of decoupled harmonic oscillators with frequency $\omega^2 = k^2 - k_\mu / m$. 
    \begin{proof}
        In fact, given 
        \begin{equation*}
            \phi(t, x) = \frac{1}{2\pi} \int dk \exp(i k x) \tilde \phi(t, k)  ~,
        \end{equation*}
        we have 
        \begin{equation*}
            \dot \phi(t, x) = \frac{1}{2\pi} \int dk \exp(i k x) \dot{\tilde \phi(t, k)} 
        \end{equation*}
        and 
        \begin{equation*}
            \ddot \phi(t, x) = \frac{1}{2\pi} \int dk \exp(i k x) \ddot{\tilde \phi(t, k)} ~. 
        \end{equation*}
        Hence 
        \begin{equation*}
            \ddot \phi(t, x) = \phi_{xx} (t, x) - \frac{k_\mu}{m} \phi(t, x) 
        \end{equation*}
        becomes 
        \begin{equation*}
            \cancel{\frac{1}{2\pi} \int dk \exp(i k x)} \ddot{\tilde \phi(t, k)} = k^2 \cancel{ \frac{1}{2\pi} \int dk \exp(i k x)} \tilde \phi(t, k) - \frac{k_\mu}{m} \cancel{\frac{1}{2\pi} \int dk \exp(i k x)} \tilde \phi(t, k) ~.
        \end{equation*}
        Finally, 
        \begin{equation*}
            \ddot{\tilde \phi} (t, k) = - (k^2 + \frac{k_\mu}{m}) \tilde \phi(t, k) ~.
        \end{equation*}
    \end{proof}

    Furthermore 
    \begin{equation*}
        \omega^2 = k^2 - \frac{k_\mu}{m} ~, \quad \hbar^2 \omega^2 = c^2 \hbar^2 k^2 - \hbar^2 \frac{k_\mu}{m} ~, \quad E^2 = c^2 p^2 + \mu^2 c^4 ~,
    \end{equation*}
    where $k_\mu = c^2 \mu^2 m / \hbar^2$. Putting $c = \hbar = 1$, we obtain 
    \begin{equation*}
        (\Box + \mu^2) \phi(t, x) = 0 ~,
    \end{equation*}
    which is the Klein-Gordon equation.

    Now, we quantise with the operators 
    \begin{equation*}
        \hat \phi_k = \frac{1}{\sqrt{2 \omega_k}} (\hat a_k + \hat a^\dagger_k) ~, \quad \hat \pi_k = - i \sqrt{\frac{\omega_k}{2}} (\hat a_k - \hat a^\dagger_k)  ~.
    \end{equation*}
    A generic basis element of the Fock space is 
    \begin{equation*}
        (\hat a_{k_1}^\dagger)^{n_1} \ldots (\hat a_{k_l}^\dagger)^{n_l} \ket{0} ~,
    \end{equation*}
    which is a state of defined energy 
    \begin{equation*}
        E = \sum_{i=1}^{l} n_i \omega_i + \frac{1}{2} \int dk \omega_k ~.
    \end{equation*}
    Notice that $\omega_k \sim k$ and the second intergal, which is the vacuum energy, diverges. The first reason is that it is an integral over an infinite range of $\omega$ and the second because $\omega$ goes to infinity as $k$ increases. Vacuum energy is infinite. This means that there is a regime in which the theory is wrong, which is for energies at the scale of the Planck's mass, where gravity becomes no more negligible. Notice that this divergence is a consequenceof the continuum limit. Therefore, if we set a cutoff for $k \Delta \sim 1$, $E_0$ becomes finite. Notice also that at fixed energy, we can have different states with different number of particle which have the same energy 
    \begin{equation*}
        E = \sum_{i = 1}^{l} n_i \omega_i = \sum_{i' = 1}^{l'} n_i' \omega_i' ~.
    \end{equation*}
    Hence, we can describe transitions between different states with the same energy but with the number of particles that changes. This is impossible in quantum mechanics. For example 
    \begin{equation*}
        E = 2 \omega + 2 \omega \rightarrow E = \omega + \omega + \omega + \omega ~.
    \end{equation*}
    However, this is possible only with interactions that can change the configuration of the system. Interaction can be introduced with small perturbation of the free theory, which are subleading terms in the Taylor expansion of the lagrangian
    \begin{equation*}
        \mathcal L = \frac{1}{2} \partial_\mu \phi \partial^\mu \phi - V(\phi) ~,
    \end{equation*}
    where 
    \begin{equation*}
        V(\phi) = \frac{\mu}{2} \phi^2 + c_3 \phi^3 + c_4 \phi^4 + \ldots ~.
    \end{equation*}
    This is an expansion around the equilibrium position $\phi = 0$, $V(0) = 0$ and it is a maximum $V'' > 0$
    \begin{equation*}
        V(\phi) = V(\phi_{min}) + V_\phi (\phi_{min}) (\phi - \phi_{min}) + \frac{1}{2} \underbrace{V_{\phi \phi} (\phi_{min})}_{\mu^2} (\phi - \phi_{min})^2 + \underbrace{\frac{1}{3!} V_{\phi \phi \phi} (\phi_{min})}_{c_3} (\phi - \phi_{min})^3 + \underbrace{\frac{1}{4!} V_{\phi \phi \phi \phi} (\phi_{min})}_{c_4} (\phi - \phi_{min})^4 + \ldots ~.
    \end{equation*} 
    $c_3$ and $c_4$ are coupling constant that describe how strong is the interaction.

\part{Classical field theory}

\chapter{Action}

    In this chapter, we will study how to treat classical fields with the Lagrangian formalism: principle of stationary action, Euler-Lagrange equations, Noether's theorem and energy-momentum tensor.

\section{Euler-Lagrange equations}

    A field is a physical quantity $\phi(t, \mathbf x) = \phi(x)$ defined at every point in spacetime. In classical mechanics, particles are described by generalised coordinates $q_i(t)$ with a finite number of degrees of freedom and labelled by $i$. Instead, fields are described by $\phi_i (x)$ and labelled by discrete $i$ and by continuous $x$, which are infinite degrees of freedom, since for each point in spacetime, there are $i$ degrees of freedom. The dynamics of a field is governed by an action, which is a functional that associates a real number to each field configuration for a fixed time interval $t \in [t_1, t_2]$
    \begin{equation}\label{action}
        S[\phi_i(x), \partial_\mu \phi_i(x)] = \int_{t_1}^{t_2} dt ~ L = \int_{t_1}^{t_2} dt ~ \int d^3x ~ \mathcal L = \int d^4 x ~ \mathcal L (\phi_i, \partial_\mu \phi_i) ~,
    \end{equation}
    where $\mathcal L$ is the Lagrangian density, defined by 
    \begin{equation*}
        L = \int d^3x ~ \mathcal L ~.
    \end{equation*}
    In natural units, the dimensional analysis is 
    \begin{equation*}
        [S] = 0 ~, \quad  [d^4 x]=-4 ~, \quad  [\mathcal L] = 4 ~.
    \end{equation*}
    By the principle of stationary action, it is possible to find the equations that governs the spacetime evolution of the field. 
    \begin{princ}
        The system evolves from an initial configuration at time $t_1$ to a final configuration at time $t_2$ along a path in configuration space which extremises the action~\eqref{action}, i.e.
        \begin{equation}\label{statact}
            \delta S = 0 ~.
        \end{equation}
        with the additional conditions 
        \begin{enumerate}
            \item fields vanish at spatial infinity
                \begin{equation*}
                    \phi_i(t, \mathbf x) \rightarrow 0 \quad |\mathbf x| \rightarrow \infty ~,
                \end{equation*}
                hence
                \begin{equation}\label{space}
                    \delta \phi_i (t, \pm \infty) = 0 ~,
                \end{equation}
            \item fields vanish at time extremes
                \begin{equation}\label{time}
                    \delta \phi_i (t_1, \mathbf x) = \delta \phi_i (t_2, \mathbf x) = 0 ~.
                \end{equation}
        \end{enumerate}
    \end{princ}
    The equation of motion of the system are called the Euler-Lagrange equations and are
    \begin{equation}\label{eleq}
        \pdv{\mathcal L}{\phi_i} - \partial_\mu \pdv{\mathcal L}{\partial_\mu \phi_i} = 0 ~.
    \end{equation}
    \begin{proof}
        The variation of the action is 
        \begin{equation*}
            \delta S = \int d^4 x ~ \Big ( \pdv{\mathcal L}{\phi_i} \delta \phi_i + \pdv{\mathcal L}{\partial_\mu \phi_i} \delta \partial_\mu \phi_i \Big) = \int d^4 x ~ \Big ( \pdv{\mathcal L}{\phi_i} \delta \phi_i + \pdv{\mathcal L}{\partial_\mu \phi_i} \partial_\mu \delta \phi_i \Big) ~,
        \end{equation*}
        where
        \begin{equation*}
            \delta \phi_i = {\phi'}_i(x) - \phi_i(x) ~,
        \end{equation*} 
        and 
        \begin{equation*}
            \delta \partial_\mu \phi_i(x) = \partial_\mu {\phi'}_i - \partial_\mu \phi(x) = \partial_\mu ({\phi'}_i(x) - \phi_i(x)) = \partial_\mu \delta \phi(x) ~.
        \end{equation*}
        By integration by parts, we obtain
        \begin{equation*}
        \begin{aligned}
            \delta S & = \int d^4 x ~ \Big ( \pdv{\mathcal L}{\phi_i} \delta \phi_i + \pdv{\mathcal L}{\partial_\mu \phi_i} \partial_\mu \delta \phi_i \Big) \\ & = \int d^4 x ~ \Big( \pdv{\mathcal L}{\phi_i} \delta \phi_i - \partial_\mu \pdv{\mathcal L}{\partial_\mu \phi_i} \delta \phi_i \Big) + \int d^4x ~ \partial_\mu \Big( \pdv{\mathcal L }{\partial_\mu \phi_i} \delta \phi_i \Big) ~.
        \end{aligned}
        \end{equation*}
        Notice that the last term is a total derivative and it vanishes at the boundary by the condition~\eqref{time} and~\eqref{space}
        \begin{equation*}
            \int d^4x ~ \partial_\mu \Big( \pdv{\mathcal L }{\partial_\mu \phi_i} \delta \phi_i \Big) = \int_{\partial \mathcal M} d \sigma_\mu ~ \pdv{\mathcal L }{\partial_\mu \phi_i} \delta \phi_i = 0 ~.
        \end{equation*}
        Hence, we find 
        \begin{equation*}
            \delta S = \int d^4 x ~ \Big( \pdv{\mathcal L}{\phi_i} \delta \phi_i - \partial_\mu \pdv{\mathcal L}{\partial_\mu \phi_i} \delta \phi_i \Big) ~,
        \end{equation*}
        and, by the principle of stationary action~\eqref{statact}, 
        \begin{equation*}
            \int d^4 x ~ \delta \phi_i  \Big( \pdv{\mathcal L}{\phi_i} - \partial_\mu \pdv{\mathcal L}{\partial_\mu \phi_i} \Big) = 0 ~.
        \end{equation*}
        Finally, since $\delta_i \phi$ is arbitrary, we obtain
        \begin{equation*}
            \pdv{\mathcal L}{\phi_i} - \partial_\mu \pdv{\mathcal L}{\partial_\mu \phi_i} = 0 ~.
        \end{equation*}
    \end{proof}
    Since canonical quantisation is made through the hamiltonian fomalism, we introduce the conjugate field $\pi^i(x)$ associated to the field $\phi_i$ is 
    \begin{equation*}
        \pi^i(x) = \pdv{\mathcal L}{\dot \phi_i(x)}
    \end{equation*}
    and the Hamiltonian density is given by the Legendre transformation 
    \begin{equation*}
        \mathcal H = \pi^i \dot \phi_i - \mathcal L ~,
    \end{equation*}
    where the Hamiltonian is 
    \begin{equation*}
        H = \int d^3 x ~ \mathcal H ~.
    \end{equation*}

\section{Noether's theorem}

    Symmetries are fundamental in quantum field theory and they can be classified into Table~\ref{table:symm}.

    \begin{table}[h!]
        \centering
        \begin{tabular}{c | c |c | c }
            - & - & spacetime & internal\\
            \hline
            global & continuous & Poincaré & $U(1)$, $SU(3)_{flavour}$ \\ 
            global & discrete & parity $P$, time reversal $T$ & $\mathcal Z_2$ \\ 
            local & continuous & general coordinate transformations & $SU(3) \times SU(2) \times U(1)$ \\ 
            local & discrete & $P(x)$ & $\mathcal Z_2 (x)$ \\ 
        \end{tabular}
        \caption{Symmetries in physics.}
        \label{table:symm}
    \end{table}

    Through the Noether's theorem, we can associate conserved quantities to continuous symmetries.

    \begin{theorem}[Noether's]
        Every continuous symmetry $\delta \phi_i$ of the action~\eqref{action} gives rise to a conserved current 
        \begin{equation}\label{conscurr}
            J^\mu = \pdv{\mathcal L}{\partial_\mu \phi_i} \delta \phi - K^\mu ~,
        \end{equation}
        such that it satisfies a continuity equation 
        \begin{equation}\label{cont}
            \partial_\mu J^\mu = 0
        \end{equation}
    \end{theorem}
    \begin{proof}
        We consider an infinitesimal transformation for a continuous symmetry of the system
        \begin{equation*}
            {\phi'}_i = \phi_i + \delta \phi_i ~,
        \end{equation*}
        which induces a transformation of the Lagrangian 
        \begin{equation*}
            \mathcal L' = \mathcal L + \delta \mathcal L ~.
        \end{equation*}
        In order to be a symmetry of the system, we require that the action is not invariant, but we allow to be up to a boundary term $K^\mu(\phi_i)$, because the dynamics of the system, i.e.~the equations of tmotion, do not change with a boundary term. Hence 
        \begin{equation*}
            S' = S + \int \partial_\mu K^\mu(\phi_i) ~,
        \end{equation*}
        but 
        \begin{equation}\label{symm}
            \delta S = \int \partial_\mu K^\mu(\phi_i) ~.
        \end{equation}
        Explicitly, we obtain 
        \begin{equation*}
        \begin{aligned}
            \delta S & = \delta \int d^4 x ~ \mathcal L = \int d^4 x ~ \Big ( \pdv{\mathcal L}{\phi_i} \delta \phi_i + \pdv{\mathcal L}{\partial_\mu \phi_i} \delta \partial_\mu \phi_i\Big) =  \int d^4 x ~ \Big ( \pdv{\mathcal L}{\phi_i} \delta \phi_i + \pdv{\mathcal L}{\partial_\mu \phi_i} \partial_\mu \delta \phi_i \Big) \\ & =  \int d^4 x ~ \Big ( \pdv{\mathcal L}{\phi_i} \delta \phi_i - \partial_\mu \pdv{\mathcal L}{\partial_\mu \phi_i} \delta \phi_i \Big) + \int d^4 x ~ \partial_\mu \Big ( \pdv{\mathcal L}{\phi} \delta \phi_i \Big)\\ & =  \int d^4 x ~ \delta \phi_i \Big ( \underbrace{\pdv{\mathcal L}{\phi_i} - \partial_\mu \pdv{\mathcal L}{\partial_\mu \phi_i}}_0 \Big) + \int d^4 x ~ \partial_\mu \Big ( \pdv{\mathcal L}{\phi} \delta \phi_i \Big)  = \int d^4 x ~ \partial_\mu \Big ( \pdv{\mathcal L}{\phi} \delta \phi_i \Big) ~,
        \end{aligned}
        \end{equation*}
        where we used the fact that partial derivatives and symmetries commute, the equation of motions~\eqref{eleq} and we integrated by parts. Hence, by requiring that it is a symmetry
        \begin{equation*}
            \delta S = \int d^4 x ~ \partial_\mu \Big ( \pdv{\mathcal L}{\phi} \delta \phi_i \Big) = \int d^4 x \partial_\mu K^\mu ~,
        \end{equation*}
        or equivalently 
        \begin{equation*}
            \int d^4 x ~ \partial_\mu \Big ( \pdv{\mathcal L}{\phi} \delta \phi_i - K^\mu \Big) = 0 ~.
        \end{equation*}
        Since it is for arbitrary integration, the integrand vanishes and 
        \begin{equation*}
            \partial_\mu J^\mu = 0 ~,
        \end{equation*}
        with 
        \begin{equation*}
            J^\mu = \pdv{\mathcal L}{\partial_\mu \phi_i} \delta \phi_i - K^\mu ~.
        \end{equation*}
    \end{proof}
    Notice that every conserved current can be related to a conserved quantity $Q$ by 
    \begin{equation*}
        Q = \int_{\mathbb R^3} d^3 x ~J^0 ~.
    \end{equation*}
    This means that $Q$ is conserved locally, i.e.~any charge carrier leaving a finite volume $V$ is associated to a flow of current $\mathbf J$ out of the volume.
    \begin{proof}
        Infact, by using~\eqref{cont}
        \begin{equation*}
        \begin{aligned}
            \dv{Q}{t} & = \dv{}{t} \int_{\mathbb R^3} d^3 x ~ J^0 = \int_{\mathbb R^3} d^3 x ~ \pdv{J^0}{t} = - \int_{\mathbb R^3} d^3 x ~ \nabla \cdot \mathbf J = 0 = -  \int_{\partial \mathbb R^3} d \mathbf S \cdot \mathbf J = 0
        \end{aligned}
        \end{equation*}
        where we used the Stoke's theorem and the fact that $\mathbf J \rightarrow 0$ for $|\mathbf x| \rightarrow 0$.
    \end{proof}

\section{Energy-momentum tensor}

    Spacetime translations give rise to $4$ conserved currents, which corresponds to the conservation of energy (1) and momentum (3). In fact, consider an infinitesimal spacetime translation 
    \begin{equation*}
        {x'}^\mu = x^\mu - \epsilon^\mu ~,
    \end{equation*}
    such that fields change by 
    \begin{equation*}
        {\phi'}_i = \phi_i (x + \epsilon) = \phi(x) + \epsilon^\mu \partial_\mu \phi_i (x) ~.
    \end{equation*}
    We considered an active transformation, where there is not a change of frame but fields themselves are translated into new fields such that 
    \begin{equation*}
        {\phi'}_i (x') = \phi(x) = \phi(x' + \epsilon) ~.
    \end{equation*}
    A passive transformation would have acted as 
    \begin{equation*}
        {\phi'}_i = \phi_i (x - \epsilon)  ~.
    \end{equation*}
    Since the Lagrangian is a function of the coordinates via fields, we have the following transformation
    \begin{equation*}
        \delta \mathcal L = \mathcal L' - \mathcal L = \epsilon^\mu \partial_\mu \mathcal L = \epsilon^\mu \partial_\nu(\delta^\nu_{\phantom \nu \mu} \mathcal L)
    \end{equation*}
    Hence, the boundary term is 
    \begin{equation*}
        K^\mu = \delta^\mu_{\phantom \mu \nu} \mathcal L ~.
    \end{equation*}
    Now, we apply the Noether's theorem~\eqref{conscurr} and find $4$ different conserved currents labelled by $\nu$
    \begin{equation*}
        {(J^\mu)}_\nu = \pdv{\mathcal L}{\partial_\mu \phi_i} \partial_\nu \phi_i - \delta^\mu_{\phantom \mu \nu} \mathcal L
    \end{equation*}
    and we define the energy-momentum tensor, or stress-energy tensor, 
    \begin{equation*}\label{emten}
        T^\mu_{\phantom \mu \nu} = {(J^\mu)}_\nu ~,
    \end{equation*}
    such that 
    \begin{equation*}
        \partial_\mu T^\mu_{\phantom \mu \nu} = 0 ~.
    \end{equation*}
    In natural units, the dimensional analysis is 
    \begin{equation*}
        T^\mu_{\phantom \mu \nu} = [\mathcal L] = 4 ~.
    \end{equation*}
    The $4$ conserved charges are 
    \begin{equation*}
        Q_\nu = \int_{\mathbb R^3} d^3 x ~ (J^0)_\nu = \int_{\mathbb R^3} d^3 x ~ T^0_{\phantom 0 \nu}  ~,
    \end{equation*}
    which correspond to the $4$-momentum
    \begin{equation*}
        P^\mu = \int_{\mathbb R^3} d^3 x ~ T^{0\mu} ~.
    \end{equation*}
    In particular, the $0$-th component is the energy 
    \begin{equation}\label{energ}
    \begin{aligned}
        P^0 & = \int d^3 x ~ T^{00} = \int d^3 x ~ \Big ( \pdv{\mathcal L}{\partial_0 \phi_i} \partial^0 \phi_i - \delta^{0 0} \mathcal L \Big) \\ & = \int d^3 x ~ \Big ( \underbrace{\pdv{\mathcal L}{\dot \phi_i}}_{\pi^i} \dot \phi_i - \mathcal L \Big ) = \int d^3 x ~ (\underbrace{\pi^i \dot \phi_i - \mathcal L}_{\mathcal H}) = \int d^3 x ~ \mathcal H = H
    \end{aligned}
    \end{equation}
    such that 
    \begin{equation*}
        \dv{H}{t} = 0 ~,
    \end{equation*} 
    and the $j$-th components are the momentum 
    \begin{equation}\label{momen}
    \begin{aligned}
        P^j & = \int d^3 x ~ T^{0j} = \int d^3 x ~ \Big ( \pdv{\mathcal L}{\partial_0 \phi_i} \partial^j \phi_i - \underbrace{\delta^{0j}}_0 \mathcal L \Big)  = \int d^3 x ~ \Big ( \underbrace{\pdv{\mathcal L}{\dot \phi_i}}_{\pi^i} \partial^j \phi_i \Big) = \int d^3 x ~ \pi^i \partial^j \phi_i ~,
    \end{aligned}
    \end{equation}
    which in vector notation becomes 
    \begin{equation*}
        \mathbf P = - \int d^3 x ~ \pi^i \boldsymbol \nabla \phi_i ~,
    \end{equation*}
    such that 
    \begin{equation*}
        \dv{P^i}{t} = 0 ~.
    \end{equation*}

\section{Electrodynamics}

    Maxwell's equations 
    \begin{equation}\label{eqmax}
        \nabla \cdot \mathbf B = 0 ~, \quad \nabla \times \mathbf E + \pdv{\mathbf B}{t} = 0 ~, \quad \nabla \cdot E = \rho ~, \quad \nabla \times \mathbf B - \pdv{\mathbf E}{t} = \mathbf J 
    \end{equation}
    can be written in covariant form 
    \begin{equation*}
        \partial_\mu F^{\mu\nu} = J^\nu ~, \quad \partial_\mu {F^*}^{\mu\nu} = 0 ~,
    \end{equation*}
    where $F^{\mu\nu}$ is the electromagnetic tensor and ${F^*}^{\mu\nu} = \frac{1}{2} \epsilon^{\mu\nu\sigma\rho} F_{\sigma\rho}$ is its dual. 
    Furthermore, they can be written in terms of the scalar $\phi$ and the vector potentials $\mathbf A$, defined by
    \begin{equation*}
        \mathbf E = - \nabla \phi - \pdv{\mathbf A}{t} ~, \quad \mathbf B = \nabla \times \mathbf A
    \end{equation*}
    Maxwell's equations do not change under this transformation. 
    \begin{proof}
       In fact, 
    \end{proof}
    In covariant form, we can write the electromagnetic tensor as 
    \begin{equation*}
        F^{\mu\nu} = \partial^\mu A^\nu - \partial^\nu A^\mu
    \end{equation*}
    Maxwell's equations can be seen as the equations of motion of the electromagnetic Lagrangian 
    \begin{equation*}
        \mathcal L = - \frac{1}{4} F^{\mu\nu} F_{\mu\nu} - J_\mu A^\mu
    \end{equation*}
    or, equivalenty, written in terms of the $4$-potential, 
    \begin{equation*}
        \mathcal L = - \frac{1}{2} \partial_\mu A_\nu \partial^\mu A^\nu + \frac{1}{2} (\partial_\mu A^\mu) ^2 - A_\mu J^\mu ~.
    \end{equation*}

    \begin{proof}
        First, we prove that they are equivalent.
        Maybe in the future. 

        Second, we prove that it leads to the Maxwell's equations.
        Maybe in the future.
    \end{proof}

    In natural units, the dimensional analysis is 
    \begin{equation*}
        [F^{\mu\nu}] = 2 \quad [A_\mu] = 1 \quad [J^\mu] = 3
    \end{equation*}

    The minus sign garanties that the kinetic energy has a positive one 
    \begin{equation*}
        - \frac{1}{2} \partial_0 A_i \underbrace{\partial^0}_{\partial_0} \underbrace{A^i}_{-A_i} = \frac{1}{2} \dot A_i^2
    \end{equation*}

    The fourth field $A_0$ is not a dynamical quantity, since there is no kinetic energy in terms of $\dot A_0^2$, because the first $- \frac{1}{2} \partial_0 A_0 \partial^0 A^0$ cancels out with $\frac{1}{2} (\partial_0 A_0)^2$. Therefore, there are only $3$ degrees of freedom. However, since electrodynamics is a gauge theory, it is possible to restrict to only $2$ degrees of freedom, which correspond to the $2$ transversal polarisations direction of an electromagnetic wave.

    The energy-momentum tensor is 
    \begin{equation}\label{emtem}
        T^{\mu\nu} = \partial^\nu A^\mu \partial_\rho A^\rho - \partial^\mu A^\rho \partial^\nu A_\rho + \frac{1}{4} \eta^{\mu\nu} F_{\rho \sigma} F^{\rho\sigma}
    \end{equation}

    \begin{proof}
        Maybe in the future.
    \end{proof}

    However, the first term in~\eqref{emtem} is not symmetric under change $\mu \leftrightarrow \nu$, but in order to take into account general relativity, this tensor must be symmetric, since $R_{\mu\nu}$ and $g_{\mu\nu}$ are so in
    \begin{equation*}
        R_{\mu\nu} - \frac{1}{2} R g_{\mu\nu} + \Lambda g_{\mu\nu} = \frac{8 \pi G}{c^4} T_{\mu\nu}
    \end{equation*}

    To do it, we defined a new energy-momentum tensor starting from the old one with the addition of an extra term: the partial derivative of a $3$ indices anti-symmetric in the first $2$ indices tensor $K^{\lambda\mu\nu} = - K^{\mu\lambda\nu}$
    \begin{equation*}
        \tilde T^{\mu\nu} = T^{\mu\nu} + \partial_\lambda K^{\lambda\mu\nu}
    \end{equation*} 
    This garanties that it is conserved as well 
    \begin{equation*}
        \partial_\mu \tilde T^{\mu\nu} = \partial_\mu T^{\mu\nu} + \underbrace{\partial_\mu \partial_\lambda}_{symm} \underbrace{K^{\lambda\mu\nu}}_{anti} = \partial_\mu T^{\mu\nu} = 0
    \end{equation*}

    In the electromagnetic case, we choose $K$ to be 
    \begin{equation*}
        K^{\lambda\mu\nu} = F^{\mu\lambda} A^\nu
    \end{equation*}
    and the symmetric energy-momentum tensor becomes 
    \begin{equation*}
        \tilde T^{\mu\nu} = F^{\mu\lambda} F_{\lambda}^{\phantom \lambda \nu} + \frac{1}{4} \eta^{\mu\nu} F^{\rho\sigma} F_{\rho\sigma}
    \end{equation*}
    which is called the Belifante-Rosenfeld tensor.

    \begin{proof}
        Maybe in the future.
    \end{proof}

    The energy density is 
    \begin{equation*}
        \mathcal E = \frac{1}{2} (|\mathbf E|^2 + |\mathbf B|^2)
    \end{equation*}

    \begin{proof}
        Maybe in the future.
    \end{proof}

    The momentum density is 
    \begin{equation*}
        \mathcal P^i = (\mathbf E \times \mathbf B)^i
    \end{equation*}

    \begin{proof}
        Maybe in the future.
    \end{proof}


\part{Klein-Gordon theory}

\chapter{Canonical or second quantisation}

    In Schoedinger picture, where states evolve in time while operators do not, recall that standard quantisation from classical mechanics to quantum mechanics works in this way: 
    \begin{enumerate}
        \item hamiltonian formalism $H \mapsto$ hamiltonian operator $\hat H$~,
        \item generalised coordinates and conjugate momenta $(q_i, p^i = \pdv{L}{\dot q_i}) \mapsto$ operators on a Hilbert space $\hat q_i$ and $\hat p^i$~,
        \item Poissons brackets $\{q_i, p^j\} = \delta_i^{\phantom i j}$ and $\{p^i, p^j\} = \{q_i, q_j\} = 0 \mapsto$ commutators $[q_i, p^j] = i \delta_i^{\phantom i j}$ and $[p^i, p^j] = [q_i, q_j] = 0$~.
    \end{enumerate}

    Similarly, the second quantisation from classical field theory to quantum field theory works in this way:
    \begin{enumerate}
        \item fields and conjugate fields $(\varphi_i(t, \mathbf x), \pi^i (t, \mathbf x) = \pdv{\mathcal L}{\dot \varphi_i}) \mapsto$ operators on a Fock space $\hat \varphi_i(t, \mathbf x)$ and $\hat \pi^i (t, \mathbf x)$~,
        \item canonical commutation relations $[\hat \varphi_i(t, \mathbf x), \hat \pi^j (t, \mathbf y)] = i \delta_i^{\phantom i j} \delta^3(\mathbf x - \mathbf y)$ and $[\hat \varphi_i(t, \mathbf x), \hat \varphi_j(t, \mathbf y)] = [\hat \pi^i (t, \mathbf x), \hat \pi^j (t, \mathbf y)] = 0$~.
    \end{enumerate}

    States which live in the Fock state $\ket{\psi}$ evolve in time via the Schoedinger equation 
    \begin{equation*}
        i \pdv{}{t} \ket{\psi} = \hat H \ket{\psi} 
    \end{equation*}
    where $\ket{\psi}$ is a wave functional such that its modulus square gives the density probability to find the field in a certain configuration and $\hat H (\varphi_i(t, \mathbf x), \pi^i (t, \mathbf x))$ is an operator, since $\varphi_i(t, \mathbf x)$ and $\pi^i (t, \mathbf x)$ are.

    In order to solve the theory, we need to find the eigenstates of $\hat H$, but it is too difficult expect in the case of a free theory, which the lagrangian is quadratic and the equations od motion are linear and solvable.

\section{Harmonic oscillator}

    Recall some feature of the harmonic oscillator.

\section{Dirac delta}

    Recall that the integral representation of the Dirac delta is 
    \begin{equation}\label{deltaint}
        \delta^3 (\mathbf x - \mathbf y) = \int \frac{d^3 p}{(2\pi)^3} \exp(i \mathbf p \cdot (\mathbf x - \mathbf y)) = \int \frac{d^3 p}{(2\pi)^3} \exp(- i \mathbf p \cdot (\mathbf x - \mathbf y)) ~.
    \end{equation}

\chapter{Single real Klein-Gordon field}

\section{Hamiltonian}

    The simplest relativistic field theory is the Klein-Gordon theory of a single real scalar field chargeless and spinless. Its lagrangian is 
    \begin{equation*}
        \mathcal L = \frac{1}{2} \partial_\mu \varphi \partial^\mu \varphi - \frac{1}{2} m^2 \varphi^2 
    \end{equation*}
    and its equations of motion are 
    \begin{equation}\label{kgeq}
        (\Box + m^2) \varphi(x) = 0 ~.
    \end{equation}
    \begin{proof}
        Infact, using~\eqref{eleq} 
        \begin{equation*}
        \begin{aligned}
            0 & = \pdv{\mathcal L}{\varphi} - \partial_\mu \pdv{\mathcal L}{\partial_\mu \varphi} \\ & = \pdv{}{\varphi} \Big ( \cancel{\frac{1}{2} \partial_\mu \varphi \partial^\mu \varphi} - \underbrace{\frac{1}{2} m^2 \varphi^2}_{m^2 \varphi} \Big) + \partial_\mu \pdv{}{\partial_\mu \varphi} \Big ( \underbrace{\frac{1}{2} \partial_\mu \varphi \partial^\mu \varphi}_{\partial_\mu \partial^\mu \varphi} - \cancel{\frac{1}{2} m^2 \varphi^2} \Big) \\ & = \underbrace{\partial_\mu \partial^\mu}_\Box \varphi + m^2 \varphi \\ & = (\Box + m^2) \varphi ~.
        \end{aligned}
        \end{equation*}
    \end{proof}

    It is a system of infinitely many degrees of freedom and to decouple them we need to perform a Fourier transform 
    \begin{equation}\label{fourkg}
        \varphi (t, \mathbf x) = \int \frac{d^3 p}{(2\pi)^3} \exp(i \mathbf p \cdot \mathbf x) \tilde \varphi(t, \mathbf p) ~,
    \end{equation}
    which in momentum space becomes 
    \begin{equation*}
        \Big ( \pdvdu{}{t} + |\mathbf p|^2 + m^2 \Big) \tilde \varphi(t, \mathbf x) = 0
    \end{equation*}
    and its solution is an harmonic oscillator for each $\mathbf p$ of frequency 
    \begin{equation}\label{kgenergy}
        \omega_{\mathbf p} = \sqrt{|\mathbf p|^2 + m^2}~.
    \end{equation}
    Hence, the most general solution of the Klein-Gordon equation~\eqref{kgeq} is a superposition of simple harmonic oscillators, each vibrating with different frequency and amplitude. To quantise the theory and $\varphi$, we need to quantise this set of infinitely decoupled harmonic oscillators.
    \begin{proof}
        We decompose~\eqref{kgeq} into time and space components
        \begin{equation*}
            0 = (\Box + m^2) \varphi = (\underbrace{\partial_0}_{\partial^0} \partial^0 + \underbrace{\partial_i}_{-\partial^i} \partial^i + m^2) \varphi = ((\partial^0)^2 - (\partial^i)^2 + m^2) \varphi = (\pdvdu{}{t} - \nabla^2 + m^2) \varphi ~,
        \end{equation*}
        and we substitute~\eqref{fourkg}
        \begin{equation*}
        \begin{aligned}
            0 & = (\pdvdu{}{t} - \nabla^2 + m^2) \int \frac{d^3 p}{(2\pi)^3} \exp(i \mathbf p \cdot \mathbf x) \tilde \varphi(t, \mathbf p) \\ & = \int \frac{d^3 p}{(2\pi)^3} (\pdvdu{}{t} - \underbrace{\nabla^2}_{- i^2 |\mathbf p|^2} + m^2) (\exp(i \mathbf p \cdot \mathbf x) \tilde \varphi(t, \mathbf p)) \\ & = \int \frac{d^3 p}{(2\pi)^3} (\pdvdu{}{t} - i^2 |\mathbf p|^2 + m^2) \exp(i \mathbf p \cdot \mathbf x) \tilde \varphi(t, \mathbf p) \\ & = \int \frac{d^3 p}{(2\pi)^3} (\pdvdu{}{t} + |\mathbf p|^2 + m^2) \exp(i \mathbf p \cdot \mathbf x) \tilde \varphi(t, \mathbf p) ~,
        \end{aligned}
        \end{equation*}
        where the integrand vanishes with the exponential. Finally, we define the energy~\eqref{kgenergy} and we obtain 
        \begin{equation*}
            (\pdvdu{}{t} + \omega_{\mathbf p})^2 \tilde \varphi(t, \mathbf p) = 0 ~,
        \end{equation*} 
        which is indeed the equation of an harmonic oscillator in the form $\ddot x + \omega^2 x = 0$.
    \end{proof}

    By analogy with the simple quantum harmonic oscillator, we define the field operator 
    \begin{equation}\label{kgfop}
        \hat \varphi (\mathbf x) = \int \frac{d^3 p}{{(2\pi)}^3} \frac{1}{\sqrt{2 \omega_{\mathbf p}}} \Big (\hat a_{\mathbf p} \exp(i \mathbf p \cdot \mathbf x) + \hat a_{\mathbf p}^\dagger \exp(- i \mathbf p \cdot \mathbf x) \Big)
    \end{equation}
    and the conjugate operator
    \begin{equation}\label{kgpop}
        \hat \pi (\mathbf x) = \int \frac{d^3 p}{{(2\pi)}^3} \Big (- i\sqrt{\frac{\omega_{\mathbf p}}{2}} \Big ) \Big (\hat a_{\mathbf p} \exp(i \mathbf p \cdot \mathbf x) - \hat a_{\mathbf p}^\dagger \exp(- i \mathbf p \cdot \mathbf x) \Big) ~,
    \end{equation}
    such that they satisfies the commutation relations for annihilation and creation operators
    \begin{equation}\label{anncrea}
        [\hat a_{\mathbf p}, \hat a_{\mathbf q}] = [\hat a_{\mathbf p}^\dagger, \hat a_{\mathbf q}^\dagger] = 0 ~, \quad [\hat a_{\mathbf p}, \hat a_{\mathbf q}^\dagger] = (2\pi)^3 \delta^3 (\mathbf p - \mathbf q) ~.
    \end{equation}
    Therefore, the canonical commutation relations become 
    \begin{equation*}
        [\hat \varphi(\mathbf x), \hat \varphi (\mathbf y)] = [\hat \pi(\mathbf x), \hat \pi (\mathbf y)]  = 0
    \end{equation*}
    and 
    \begin{equation*}
        [\hat \varphi(\mathbf x), \hat \pi (\mathbf y)] = i \delta^3 (\mathbf x - \mathbf y) ~.
    \end{equation*}
    \begin{proof}
        For the field-field commutator, using~\eqref{anncrea},~\eqref{kgfop} and~\eqref{deltaint}
        \begin{equation*}
        \begin{aligned}
            [\hat \varphi(\mathbf x), \hat \varphi (\mathbf y)] & = [\int \frac{d^3 p}{{(2\pi)}^3} \frac{1}{\sqrt{2 \omega_{\mathbf p}}} \Big (\hat a_{\mathbf p} \exp(i \mathbf p \cdot \mathbf x) + \hat a_{\mathbf p}^\dagger \exp(- i \mathbf p \cdot \mathbf x) \Big), \\ & \qquad \int \frac{d^3 q}{{(2\pi)}^3} \frac{1}{\sqrt{2 \omega_{\mathbf q}}} \Big (\hat a_{\mathbf q} \exp(i \mathbf q \cdot \mathbf y) + \hat a_{\mathbf q}^\dagger \exp(- i \mathbf q \cdot \mathbf y) \Big)] \\ &  = \int \frac{d^3 p ~ d^3 q}{{(2\pi)}^6} \frac{1}{2 \sqrt{\omega_{\mathbf p}} \omega_{\mathbf q}} [\hat a_{\mathbf p} \exp(i \mathbf p \cdot \mathbf x) + \hat a_{\mathbf p}^\dagger \exp(- i \mathbf p \cdot \mathbf x), \\ & \qquad \hat a_{\mathbf q} \exp(i \mathbf q \cdot \mathbf y) + \hat a_{\mathbf q}^\dagger \exp(- i \mathbf q \cdot \mathbf y)] \\ & = \int \frac{d^3 p ~ d^3 q}{{(2\pi)}^6} \frac{1}{2 \sqrt{\omega_{\mathbf p}} \omega_{\mathbf q}} \Big ( \underbrace{[\hat a_{\mathbf p}, \hat a_{\mathbf q}]}_0 \exp(i (\mathbf p \cdot \mathbf x + \mathbf q \cdot \mathbf y)) + \underbrace{[\hat a_{\mathbf p}, \hat a_{\mathbf q}^\dagger]}_{(2\pi)^3 \delta^3 (\mathbf p - \mathbf q)} \exp(i (\mathbf p \cdot \mathbf x - \mathbf q \cdot \mathbf y)) \\ & \qquad + \underbrace{[\hat a_{\mathbf p}^\dagger, \hat a_{\mathbf q}]}_{- (2\pi)^3 \delta^3 (\mathbf q - \mathbf p)} \exp(i (- \mathbf p \cdot \mathbf x + \mathbf q \cdot \mathbf y)) + \underbrace{[\hat a_{\mathbf p}^\dagger, \hat a_{\mathbf q}^\dagger]}_0 \exp(i (- \mathbf p \cdot \mathbf x - \mathbf q \cdot \mathbf y))\Big) \\ & = \int \frac{d^3 p ~ d^3 q}{{(2\pi)}^3} \frac{1}{2 \sqrt{\omega_{\mathbf p}} \omega_{\mathbf q}} \Big ( \underbrace{\delta^3 (\mathbf p - \mathbf q) \exp(i (\mathbf p \cdot \mathbf x - \mathbf q \cdot \mathbf y))}_{\mathbf p = \mathbf q} \\ & \qquad - \underbrace{\delta^3 (\mathbf q - \mathbf p) \exp(i (- \mathbf p \cdot \mathbf x + \mathbf q \cdot \mathbf y))}_{\mathbf p = \mathbf q} \Big) \\ & = \int \frac{d^3 p}{{(2\pi)}^3} \frac{1}{2 \omega_{\mathbf p}} \Big (\underbrace{\exp(i \mathbf p \cdot (\mathbf x - \mathbf y))}_{\delta^3 (\mathbf x - \mathbf y)} - \underbrace{\exp(i \mathbf p \cdot (- \mathbf x + \mathbf y))}_{\delta^3 (\mathbf x - \mathbf y)}\Big) = 0 ~.
        \end{aligned}
        \end{equation*}

        For the conjugate-conjugate commutator, using~\eqref{anncrea},~\eqref{kgpop} and~\eqref{deltaint}
        \begin{equation*}
        \begin{aligned}
            [\hat \pi(\mathbf x), \hat \pi (\mathbf y)] & = [\int \frac{d^3 p}{{(2\pi)}^3} \Big (- i\sqrt{\frac{\omega_{\mathbf p}}{2}} \Big )  \Big (\hat a_{\mathbf p} \exp(i \mathbf p \cdot \mathbf x) - \hat a_{\mathbf p}^\dagger \exp(- i \mathbf p \cdot \mathbf x) \Big), \\ & \qquad \int \frac{d^3 q}{{(2\pi)}^3} \Big (- i \sqrt{\frac{\omega_{\mathbf q}}{2}} \Big )  \Big (\hat a_{\mathbf q} \exp(i \mathbf q \cdot \mathbf y) - \hat a_{\mathbf q}^\dagger \exp(- i \mathbf q \cdot \mathbf y) \Big)] \\ &  = \int \frac{d^3 p ~ d^3 q}{{(2\pi)}^6} \Big (- \frac{1}{2} \sqrt{\omega_{\mathbf p}\omega_{\mathbf q}} \Big ) [\hat a_{\mathbf p} \exp(i \mathbf p \cdot \mathbf x) - \hat a_{\mathbf p}^\dagger \exp(- i \mathbf p \cdot \mathbf x), \\ & \qquad \hat a_{\mathbf q} \exp(i \mathbf q \cdot \mathbf y) - \hat a_{\mathbf q}^\dagger \exp(- i \mathbf q \cdot \mathbf y)] \\ & = \int \frac{d^3 p ~ d^3 q}{{(2\pi)}^6} \Big (- \frac{1}{2} \sqrt{\omega_{\mathbf p}\omega_{\mathbf q}} \Big ) \Big (\underbrace{[\hat a_{\mathbf p}, \hat a_{\mathbf q}]}_0 \exp(i (\mathbf p \cdot \mathbf x + \mathbf q \cdot \mathbf y)) - \underbrace{[\hat a_{\mathbf p}, \hat a_{\mathbf q}^\dagger]}_{(2\pi)^3 \delta^3 (\mathbf p - \mathbf q)} \exp(i (\mathbf p \cdot \mathbf x - \mathbf q \cdot \mathbf y)) \\ & \qquad - \underbrace{[\hat a_{\mathbf p}^\dagger, \hat a_{\mathbf q}]}_{- (2\pi)^3 \delta^3 (\mathbf q - \mathbf p)} \exp(i (- \mathbf p \cdot \mathbf x + \mathbf q \cdot \mathbf y)) + \underbrace{[\hat a_{\mathbf p}^\dagger, \hat a_{\mathbf q}^\dagger]}_0 \exp(i (- \mathbf p \cdot \mathbf x - \mathbf q \cdot \mathbf y))\Big) \\ & = \int \frac{d^3 p ~ d^3 q}{{(2\pi)}^3} \Big (- \frac{1}{2} \sqrt{\omega_{\mathbf p}\omega_{\mathbf q}} \Big ) \Big ( - \underbrace{\delta^3 (\mathbf p - \mathbf q) \exp(i (\mathbf p \cdot \mathbf x - \mathbf q \cdot \mathbf y))}_{\mathbf p = \mathbf q} \\ & \qquad + \underbrace{\delta^3 (\mathbf q - \mathbf p) \exp(i (- \mathbf p \cdot \mathbf x + \mathbf q \cdot \mathbf y))}_{\mathbf p = \mathbf q} \Big) \\ & = \int \frac{d^3 p}{{(2\pi)}^3} \Big (- \frac{\omega_{\mathbf p}}{2} \Big ) \Big (-\underbrace{\exp(i \mathbf p \cdot (\mathbf x - \mathbf y))}_{\delta^3 (\mathbf x - \mathbf y)} + \underbrace{\exp(i \mathbf p \cdot (- \mathbf x + \mathbf y))}_{\delta^3 (\mathbf x - \mathbf y)}\Big) = 0 ~.
        \end{aligned}
        \end{equation*}

        For the field-conjugate commutator, using~\eqref{anncrea},~\eqref{kgfop},~\eqref{kgpop} and~\eqref{deltaint}
        \begin{equation*}
        \begin{aligned}
            [\hat \varphi(\mathbf x), \hat \pi (\mathbf y)] & = [\int \frac{d^3 p}{{(2\pi)}^3} \frac{1}{\sqrt{2 \omega_{\mathbf p}}} \Big (\hat a_{\mathbf p} \exp(i \mathbf p \cdot \mathbf x) + \hat a_{\mathbf p}^\dagger \exp(- i \mathbf p \cdot \mathbf x) \Big), \\ & \qquad \int \frac{d^3 q}{{(2\pi)}^3} \Big (- i \sqrt{\frac{\omega_{\mathbf q}}{2}} \Big )  \Big (\hat a_{\mathbf q} \exp(i \mathbf q \cdot \mathbf y) - \hat a_{\mathbf q}^\dagger \exp(- i \mathbf q \cdot \mathbf y) \Big)] \\ &  = \int \frac{d^3 p ~ d^3 q}{{(2\pi)}^6} \Big (- \frac{i}{2}\sqrt{\frac{\omega_{\mathbf q}}{\omega_{\mathbf p}}} \Big ) [\hat a_{\mathbf p} \exp(i \mathbf p \cdot \mathbf x) + \hat a_{\mathbf p}^\dagger \exp(- i \mathbf p \cdot \mathbf x), \\ & \qquad \hat a_{\mathbf q} \exp(i \mathbf q \cdot \mathbf y) - \hat a_{\mathbf q}^\dagger \exp(- i \mathbf q \cdot \mathbf y)] \\ & = \int \frac{d^3 p ~ d^3 q}{{(2\pi)}^6} \Big (- \frac{i}{2}\sqrt{\frac{\omega_{\mathbf q}}{\omega_{\mathbf p}}} \Big ) \Big ( \underbrace{[\hat a_{\mathbf p}, \hat a_{\mathbf q}]}_0 \exp(i (\mathbf p \cdot \mathbf x + \mathbf q \cdot \mathbf y)) - \underbrace{[\hat a_{\mathbf p}, \hat a_{\mathbf q}^\dagger]}_{(2\pi)^3 \delta^3 (\mathbf p - \mathbf q)} \exp(i (\mathbf p \cdot \mathbf x - \mathbf q \cdot \mathbf y)) \\ & \qquad + \underbrace{[\hat a_{\mathbf p}^\dagger, \hat a_{\mathbf q}]}_{- (2\pi)^3 \delta^3 (\mathbf q - \mathbf p)} \exp(i (- \mathbf p \cdot \mathbf x + \mathbf q \cdot \mathbf y)) - \underbrace{[\hat a_{\mathbf p}^\dagger, \hat a_{\mathbf q}^\dagger]}_0 \exp(i (- \mathbf p \cdot \mathbf x - \mathbf q \cdot \mathbf y))\Big) \\ & = \int \frac{d^3 p ~ d^3 q}{{(2\pi)}^3} \Big (- \frac{i}{2}\sqrt{\frac{\omega_{\mathbf q}}{\omega_{\mathbf p}}} \Big ) \Big ( - \underbrace{\delta^3 (\mathbf p - \mathbf q) \exp(i (\mathbf p \cdot \mathbf x - \mathbf q \cdot \mathbf y))}_{\mathbf p = \mathbf q} \\ & \qquad - \underbrace{\delta^3 (\mathbf q - \mathbf p) \exp(i (- \mathbf p \cdot \mathbf x + \mathbf q \cdot \mathbf y))}_{\mathbf p = \mathbf q} \Big) \\ & = \int \frac{d^3 p}{{(2\pi)}^3} \Big (\frac{i}{2} \Big ) \Big (\underbrace{\exp(i \mathbf p \cdot (\mathbf x - \mathbf y))}_{\delta^3 (\mathbf x - \mathbf y)} + \underbrace{\exp(i \mathbf p \cdot (- \mathbf x + \mathbf y))}_{\delta^3 (\mathbf x - \mathbf y)}\Big) \\ & = \frac{i}{2} 2 \delta^3 (\mathbf x - \mathbf y) = i \delta^3 (\mathbf x - \mathbf y) ~.
        \end{aligned}
        \end{equation*}
    \end{proof}

    The hamiltonian is 
    \begin{equation*}
        H = \frac{1}{2} \int d^3 x ~ (\pi^2 + (\boldsymbol \nabla \varphi)^2 + m^2 \varphi^2) ~.
    \end{equation*}
    If we make a function study of the classical hamiltonian, we notice that it has quadratic terms and a minimum at $\varphi_0 (t, \mathbf x) = const$ which we could consider as the ground state with $\varphi_0 = 0$. Quantising the theory means that we consider quantum (small) fluctuations $\delta \varphi$ around this ground state such that 
    \begin{equation*}
        \varphi(t, \mathbf x) = \underbrace{\varphi(t, \mathbf x)_0}_0 + \delta \varphi(t, \mathbf x) ~.
    \end{equation*} 
    The hamiltonian operator in quantum field theory becomes
    \begin{equation}\label{hamkg}
        \hat H = \int \frac{d^3 p}{(2\pi)^3} \omega_{\mathbf p} \hat a_{\mathbf p}^\dagger \hat a_{\mathbf p} + \frac{1}{2} \int d^3 p ~ \omega_{\mathbf p} \delta^3 (0) ~.
    \end{equation}
    \begin{proof}
        Infact, the conjugate field is 
        \begin{equation}\label{conjfield}
        \begin{aligned}
            \pi = \pdv{\mathcal L}{\dot \varphi} = \pdv{\mathcal L}{\partial_0 \varphi} = \partial_0 \varphi = \dot \varphi 
        \end{aligned}
        \end{equation}
        and using~\eqref{energ} and~\eqref{kglan}
        \begin{equation*}
        \begin{aligned}
            H & = \int d^3 x ~ T^{00} \\ & = \int d^3 x ~(\pi \underbrace{\dot \varphi}_\pi - \mathcal L) \\ & = \int d^3 x ~(\pi^2 - \frac{1}{2} \partial_\mu \varphi \partial^\mu \varphi + \frac{1}{2} m^2 \varphi^2) \\ & = \int d^3 x ~(\pi^2 - \frac{1}{2} \partial_0 \varphi \partial^0 \varphi - \frac{1}{2} \partial_i \varphi \partial^i \varphi + \frac{1}{2} m^2 \varphi^2) \\ & = \int d^3 x ~(\pi^2 - \frac{1}{2} \underbrace{\partial_0 \varphi \partial^0 \varphi}_{\pi^2} - \frac{1}{2} \underbrace{\partial_i \varphi \partial^i \varphi}_{- \nabla^2 \varphi} + \frac{1}{2} m^2 \varphi^2) \\ & = \frac{1}{2} \int d^3 x ~ (\pi^2 + (\boldsymbol \nabla \varphi)^2 + m^2 \varphi^2) ~.
        \end{aligned}
        \end{equation*}

        Furthermore, using~\eqref{anncrea},~\eqref{kgfop},~\eqref{kgpop} and~\eqref{deltaint}
        \begin{equation*}
        \begin{aligned}
            \hat H & = \frac{1}{2} \int d^3 x ~ \Big (\hat \pi^2 + (\boldsymbol \nabla \hat \varphi)^2 + m^2 \hat \varphi^2) \\ & = \frac{1}{2} \int d^3 x ~ (\int \frac{d^3 p}{{(2\pi)}^3} \Big (- i\sqrt{\frac{\omega_{\mathbf p}}{2}} \Big ) \Big (\hat a_{\mathbf p} \exp(i \mathbf p \cdot \mathbf x) - \hat a_{\mathbf p}^\dagger \exp(- i \mathbf p \cdot \mathbf x) \Big) \Big ) \\ & \qquad \Big (\int \frac{d^3 q}{{(2\pi)}^3} \Big (- i\sqrt{\frac{\omega_{\mathbf q}}{2}} \Big ) \Big (\hat a_{\mathbf q} \exp(i \mathbf q \cdot \mathbf x) - \hat a_{\mathbf q}^\dagger \exp(- i \mathbf q \cdot \mathbf x) \Big) \Big ) \\ & \qquad + \nabla \Big ( \int \frac{d^3 p}{{(2\pi)}^3} \frac{1}{\sqrt{2 \omega_{\mathbf p}}} \Big (\hat a_{\mathbf p} \exp(i \mathbf p \cdot \mathbf x) + \hat a_{\mathbf p}^\dagger \exp(- i \mathbf p \cdot \mathbf x) \Big) \Big) \\ & \qquad \nabla \Big ( \int \frac{d^3 q}{{(2\pi)}^3} \frac{1}{\sqrt{2 \omega_{\mathbf q}}} \Big (\hat a_{\mathbf q} \exp(i \mathbf q \cdot \mathbf x) + \hat a_{\mathbf q}^\dagger \exp(- i \mathbf q \cdot \mathbf x) \Big) \Big) \\ & \qquad + m^2 \Big (\int \frac{d^3 p}{{(2\pi)}^3} \frac{1}{\sqrt{2 \omega_{\mathbf p}}} \Big (\hat a_{\mathbf p} \exp(i \mathbf p \cdot \mathbf x) + \hat a_{\mathbf p}^\dagger \exp(- i \mathbf p \cdot \mathbf x) \Big) \Big ) \\ & \qquad \Big ( \int \frac{d^3 q}{{(2\pi)}^3} \frac{1}{\sqrt{2 \omega_{\mathbf q}}} \Big (\hat a_{\mathbf q} \exp(i \mathbf q \cdot \mathbf x) + \hat a_{\mathbf q}^\dagger \exp(- i \mathbf q \cdot \mathbf x) \Big) \Big)
        \end{aligned}
        \end{equation*}
        \begin{equation*}
        \begin{aligned}
            \phantom{\hat H} & = \frac{1}{2} \int \frac{d^3 x ~ d^3 p ~d^3 q}{(2\pi)^6} ~ \Big (\Big (- \frac{1}{2} \sqrt{\omega_{\mathbf p} \omega_{\mathbf q}} \Big ) \Big (\hat a_{\mathbf p} \hat a_{\mathbf q} \exp(i (\mathbf p + \mathbf q) \cdot \mathbf x) - \hat a_{\mathbf p} \hat a_{\mathbf q}^\dagger \exp(i (\mathbf p - \mathbf q) \cdot \mathbf x) \\ & \qquad - \hat a_{\mathbf p}^\dagger \hat a_{\mathbf q} \exp(i (- \mathbf p + \mathbf q) \cdot \mathbf x) + \hat a_{\mathbf p}^\dagger \hat a_{\mathbf q}^\dagger \exp(i (- \mathbf p - \mathbf q) \cdot \mathbf x) \Big) \\ & \qquad + \frac{1}{2 \sqrt{\omega_{\mathbf p} \omega_{\mathbf q}}} \Big (i \mathbf p \hat a_{\mathbf p} \exp(i \mathbf p \cdot \mathbf x) - i \mathbf p \hat a_{\mathbf p}^\dagger \exp(- i \mathbf p \cdot \mathbf x) \Big) \cdot \\ & \qquad \Big ( i \mathbf q \hat a_{\mathbf q} \exp(i \mathbf q \cdot \mathbf x) - i \mathbf q \hat a_{\mathbf q}^\dagger \exp(- i \mathbf q \cdot \mathbf x) \Big) \\ & \qquad + m^2 \frac{1}{2 \sqrt{\omega_{\mathbf p} \omega_{\mathbf q}}} \Big (\hat a_{\mathbf p} \hat a_{\mathbf q} \exp(i (\mathbf p + \mathbf q) \cdot \mathbf x) + \hat a_{\mathbf p} \hat a_{\mathbf q}^\dagger \exp(i (\mathbf p - \mathbf q) \cdot \mathbf x) \\ & \qquad + \hat a_{\mathbf p}^\dagger \hat a_{\mathbf q} \exp(i (- \mathbf p + \mathbf q) \cdot \mathbf x) + \hat a_{\mathbf p}^\dagger \hat a_{\mathbf q}^\dagger \exp(i (- \mathbf p - \mathbf q) \cdot \mathbf x) \Big) \Big) 
        \end{aligned}
        \end{equation*}
        \begin{equation*}
        \begin{aligned}
            \phantom{\hat H} & = \frac{1}{2} \int \frac{d^3 x ~ d^3 p ~d^3 q}{(2\pi)^6} \Big (\Big (- \frac{1}{2} \sqrt{\omega_{\mathbf p} \omega_{\mathbf q}} \Big ) \Big (\hat a_{\mathbf p} \hat a_{\mathbf q} \underbrace{\exp(i (\mathbf p + \mathbf q) \cdot \mathbf x)}_{\delta^3 (\mathbf p + \mathbf q)} - \hat a_{\mathbf p} \hat a_{\mathbf q}^\dagger \underbrace{\exp(i (\mathbf p - \mathbf q) \cdot \mathbf x)}_{\delta^3 (\mathbf p - \mathbf q)} \\ & \qquad - \hat a_{\mathbf p}^\dagger \hat a_{\mathbf q} \underbrace{\exp(i (- \mathbf p + \mathbf q) \cdot \mathbf x)}_{\delta^3 (\mathbf p - \mathbf q)} + \hat a_{\mathbf p}^\dagger \hat a_{\mathbf q}^\dagger \underbrace{\exp(i (- \mathbf p - \mathbf q) \cdot \mathbf x)}_{\delta^3 (\mathbf p + \mathbf q)} \Big) \\ & \qquad + \frac{1}{2 \sqrt{\omega_{\mathbf p} \omega_{\mathbf q}}} \Big (- \mathbf p \cdot \mathbf q \hat a_{\mathbf p} \hat a_{\mathbf q} \underbrace{\exp(i (\mathbf p + \mathbf q) \cdot \mathbf x)}_{\delta^3 (\mathbf p + \mathbf q)} + \mathbf p \cdot \mathbf q \hat a_{\mathbf p} \hat a_{\mathbf q}^\dagger \underbrace{\exp(i (\mathbf p - \mathbf q) \cdot \mathbf x)}_{\delta^3 (\mathbf p - \mathbf q)} \\ & \qquad + \mathbf p \cdot \mathbf q \hat a_{\mathbf p}^\dagger \hat a_{\mathbf q} \underbrace{\exp(i (- \mathbf p + \mathbf q) \cdot \mathbf x)}_{\delta^3 (\mathbf p - \mathbf q)} - \mathbf p \cdot \mathbf q \hat a_{\mathbf p}^\dagger \hat a_{\mathbf q}^\dagger \underbrace{\exp(i (- \mathbf p - \mathbf q) \cdot \mathbf x)}_{\delta^3 (\mathbf p + \mathbf q)} \Big) \\ & \qquad + \frac{m^2}{2 \sqrt{\omega_{\mathbf p} \omega_{\mathbf q}}} \Big (\hat a_{\mathbf p} \hat a_{\mathbf q} \underbrace{\exp(i (\mathbf p + \mathbf q) \cdot \mathbf x)}_{\delta^3 (\mathbf p + \mathbf q)} + \hat a_{\mathbf p} \hat a_{\mathbf q}^\dagger \underbrace{\exp(i (\mathbf p - \mathbf q) \cdot \mathbf x)}_{\delta^3 (\mathbf p - \mathbf q)} \\ & \qquad + \hat a_{\mathbf p}^\dagger \hat a_{\mathbf q} \underbrace{\exp(i (- \mathbf p + \mathbf q) \cdot \mathbf x)}_{\delta^3 (\mathbf p - \mathbf q)} + \hat a_{\mathbf p}^\dagger \hat a_{\mathbf q}^\dagger \underbrace{\exp(i (- \mathbf p - \mathbf q) \cdot \mathbf x)}_{\delta^3 (\mathbf p + \mathbf q)} \Big) \Big) 
        \end{aligned}
        \end{equation*}
        \begin{equation*}
        \begin{aligned}
            \phantom{\hat H} & = \frac{1}{2} \int \frac{d^3 p ~d^3 q}{(2\pi)^3} \Big (\Big (- \frac{1}{2} \sqrt{\omega_{\mathbf p} \omega_{\mathbf q}} \Big ) \Big (\hat a_{\mathbf p} \hat a_{\mathbf q} \underbrace{\delta^3 (\mathbf p + \mathbf q)}_{\mathbf p = - \mathbf q} - \hat a_{\mathbf p} \hat a_{\mathbf q}^\dagger \underbrace{\delta^3 (\mathbf p - \mathbf q)}_{\mathbf p = \mathbf q} \\ & \qquad - \hat a_{\mathbf p}^\dagger \hat a_{\mathbf q} \underbrace{\delta^3 (\mathbf p - \mathbf q)}_{\mathbf p = \mathbf q} + \hat a_{\mathbf p}^\dagger \hat a_{\mathbf q}^\dagger \underbrace{\delta^3 (\mathbf p + \mathbf q)}_{\mathbf p = - \mathbf q} \Big) \\ & \qquad + \frac{1}{2 \sqrt{\omega_{\mathbf p} \omega_{\mathbf q}}} \Big (- \mathbf p \cdot \mathbf q \hat a_{\mathbf p} \hat a_{\mathbf q} \underbrace{\delta^3 (\mathbf p + \mathbf q)}_{\mathbf p = - \mathbf q} + \mathbf p \cdot \mathbf q \hat a_{\mathbf p} \hat a_{\mathbf q}^\dagger \underbrace{\delta^3 (\mathbf p - \mathbf q)}_{\mathbf p = \mathbf q} \\ & \qquad + \mathbf p \cdot \mathbf q \hat a_{\mathbf p}^\dagger \hat a_{\mathbf q} \underbrace{\delta^3 (\mathbf p - \mathbf q)}_{\mathbf p = \mathbf q} - \mathbf p \cdot \mathbf q \hat a_{\mathbf p}^\dagger \hat a_{\mathbf q}^\dagger \underbrace{\delta^3 (\mathbf p + \mathbf q)}_{\mathbf p = - \mathbf q} \Big) \\ & \qquad + \frac{m^2}{2 \sqrt{\omega_{\mathbf p} \omega_{\mathbf q}}} \Big (\hat a_{\mathbf p} \hat a_{\mathbf q} \underbrace{\delta^3 (\mathbf p + \mathbf q)}_{\mathbf p = - \mathbf q} + \hat a_{\mathbf p} \hat a_{\mathbf q}^\dagger \underbrace{\delta^3 (\mathbf p - \mathbf q)}_{\mathbf p = \mathbf q} \\ & \qquad + \hat a_{\mathbf p}^\dagger \hat a_{\mathbf q} \underbrace{\delta^3 (\mathbf p - \mathbf q)}_{\mathbf p = \mathbf q} + \hat a_{\mathbf p}^\dagger \hat a_{\mathbf q}^\dagger \underbrace{\delta^3 (\mathbf p + \mathbf q)}_{\mathbf p = - \mathbf q} \Big) \Big)  
        \end{aligned}
        \end{equation*}
        \begin{equation*}
        \begin{aligned}
            \phantom{\hat H} & = \frac{1}{2} \int \frac{d^3 p}{(2\pi)^3} \Big ( \Big (-\frac{\omega_{\mathbf p}}{2} \Big) \Big (\hat a_{\mathbf p} \hat a_{- \mathbf p} - \hat a_{\mathbf p} \hat a_{\mathbf p}^\dagger - \hat a_{\mathbf p}^\dagger \hat a_{\mathbf p} + \hat a_{\mathbf p}^\dagger \hat a_{- \mathbf p}^\dagger \Big) \\ & \qquad + \Big (\frac{|\mathbf p|^2}{2 \omega_{\mathbf p}} \Big) \Big (\hat a_{\mathbf p} \hat a_{- \mathbf p} + \hat a_{\mathbf p} \hat a_{\mathbf p}^\dagger + \hat a_{\mathbf p}^\dagger \hat a_{\mathbf p} + \hat a_{\mathbf p}^\dagger \hat a_{- \mathbf p}^\dagger \Big) \\ & \qquad + \Big ( \frac{m^2}{2 \omega_{\mathbf p}} \Big) \Big (\hat a_{\mathbf p} \hat a_{- \mathbf p}+ \hat a_{\mathbf p} \hat a_{\mathbf p}^\dagger  + \hat a_{\mathbf p}^\dagger \hat a_{\mathbf p} + \hat a_{\mathbf p}^\dagger \hat a_{- \mathbf p}^\dagger \Big) \Big)
        \end{aligned}
        \end{equation*}
        \begin{equation*}
        \begin{aligned}
            \phantom{\hat H} & = \frac{1}{4} \int \frac{d^3 p}{(2\pi)^3} \frac{1}{\omega_{\mathbf p}} \Big ( (\hat a_{\mathbf p} \hat a_{- \mathbf p} + \hat a_{\mathbf p}^\dagger \hat a_{- \mathbf p}^\dagger ) \underbrace{(- \omega_{\mathbf p}^2 + | \mathbf p|^2 + m^2 )}_0 \\ & \qquad + (\hat a_{\mathbf p} \hat a_{\mathbf p}^\dagger + \hat a_{\mathbf p}^\dagger \hat a_{\mathbf p} ) \underbrace{(\omega_{\mathbf p}^2 + | \mathbf p|^2 + m^2 )}_{2\omega_{\mathbf p}^2} \Big) \\ & = \frac{1}{4} \int \frac{d^3 p}{(2\pi)^3} \frac{2 \omega_{\mathbf p}^{\cancel{2}}}{\cancel{\omega_{\mathbf p}}} (\hat a_{\mathbf p} \hat a_{\mathbf p}^\dagger + \hat a_{\mathbf p}^\dagger \hat a_{\mathbf p}) \\ & = \frac{1}{2} \int \frac{d^3 p}{(2\pi)^3} \omega_{\mathbf p} (\underbrace{\hat a_{\mathbf p} \hat a_{\mathbf p}^\dagger}_{[\hat a_{\mathbf p}, \hat a_{\mathbf p}^\dagger] + \hat a_{\mathbf p}^\dagger \hat a_{\mathbf p}} + \hat a_{\mathbf p}^\dagger \hat a_{\mathbf p}) \\ & = \frac{1}{2} \int \frac{d^3 p}{(2\pi)^3} \omega_{\mathbf p} (\underbrace{[\hat a_{\mathbf p}, \hat a_{\mathbf p}^\dagger]}_{(2\pi)^3 \delta^3 (\mathbf p - \mathbf p)} + 2 \hat a_{\mathbf p}^\dagger \hat a_{\mathbf p}) \\ & = \frac{1}{2} \int d^3 p ~ \omega_{\mathbf p} \delta^3 (0) + \int \frac{d^3 p}{(2\pi)^3} ~ \omega_{\mathbf p}\hat a_{\mathbf p}^\dagger \hat a_{\mathbf p} ~,
        \end{aligned}
        \end{equation*}
        where we have used the fact that $\omega_{- \mathbf p} = \sqrt{| - \mathbf p|^2 + m^2} = \sqrt{|\mathbf p|^2 + m^2} = \omega_{\mathbf p}$. 

    \end{proof}

    The first term of~\eqref{hamkg} counts simply how what is the relativistic energy of each particle $\omega_{\mathbf p}$ and through the number operator $\hat N_{\mathbf p} = \hat a_{\mathbf p}^\dagger \hat a_{\mathbf p}$ and the integral, we sum all over the possible value of $\mathbf p$. However, most of them may be zero and we do not have to worry about divergences. 

\section{Vacuum energy}
    Things are different if we look at the second term of~\eqref{hamkg}, beacuse, in analogy with the energy of the single harmonic oscillator, we interpret it as the energy of the vacuum and it diverges for two reasons
    \begin{enumerate}
        \item infrared divergence, i.e. 
            \begin{equation*}
                \delta^3 (0) \rightarrow \infty~,
            \end{equation*}
        \item ultraviolet divergence, i.e. for $|\mathbf p| \rightarrow \infty$
            \begin{equation*}
            \int d^3 p ~ \omega_{\mathbf p} \rightarrow \infty ~,
        \end{equation*} 
            since for $|\mathbf p| \rightarrow \infty$
            \begin{equation*}
                \omega_{\mathbf p} = \sqrt{|\mathbf p|^2 + m^2} \simeq |\mathbf p| ~.
            \end{equation*}
    \end{enumerate}

    This can be better understood by applying the hamiltonian operator to the vacuum state $\ket{0}$, i.e.~the state such that it is annihilated by all the annihilation operators is for all $\mathbf p$
    \begin{equation*}
        \hat a_{\mathbf p} \ket{0} = 0 \quad \forall \mathbf p ~.
    \end{equation*}
    Therefore 
    \begin{equation*}
        \hat H \ket{0} = E_0 \ket{0} = \infty \ket{0}
    \end{equation*}
    and the vaccum energy is infinite.
    \begin{proof}
        Infact, using~\eqref{hamkg}
        \begin{equation*}
            \hat H \ket{0} = \int \frac{d^3 p}{(2\pi)^3} \omega_{\mathbf p} \hat a_{\mathbf p}^\dagger \underbrace{\hat a_{\mathbf p} \ket{0}}_0 + \Big (\underbrace{\frac{1}{2} \int d^3 p ~ \omega_{\mathbf p} \delta^3 (0)}_\infty \Big ) \ket{0} = \infty \ket{0} = E_0 \ket{0} ~.
        \end{equation*}
    \end{proof}

\subsection{IR divergence}

    The infrared divergence is due to the fact that space is infinitely large. This means that in every point of spacetime there is an harmonic oscilators. To prove this, consider a box of sides $L$ and periodic boundary conditions for the fields. The volume of the box is just the Dirac delta inside the integrand of the energy vacuum. Infact 
    \begin{equation*}
        (2\pi)^3 \delta^3 (0) = \lim_{L \rightarrow \infty} \int_{-\frac{L}{2}}^{\frac{L}{2}} \int_{-\frac{L}{2}}^{\frac{L}{2}} \int_{-\frac{L}{2}}^{\frac{L}{2}} d^3 x ~ \exp(- i \mathbf p \cdot \mathbf x) \Big \vert_{\mathbf p = 0} = \lim_{L \rightarrow \infty} \int_{-\frac{L}{2}}^{\frac{L}{2}} \int_{-\frac{L}{2}}^{\frac{L}{2}} \int_{-\frac{L}{2}}^{\frac{L}{2}} d^3 x = L^3 = V ~.
    \end{equation*}
    This divergence can be removed by studying energy densities instead of pure energies. 
    \begin{equation*}
        \mathcal E_0 = \frac{E_0}{V} = \int \frac{d^3}{(2\pi)^3} \frac{\omega_{\mathbf p}}{2} ~.
    \end{equation*}

\subsection{UV divergence}

    However, still the energy density is infinite because of the ultraviolet divergence, since for $|\mathbf p| \rightarrow \infty$
    \begin{equation*}
        \mathcal E_0 \rightarrow \infty ~.
    \end{equation*}
    
    The reason is the following: we made a strong assumption considering the theory valid for any large value of energy and now we have found where the theory breaks, since this divergence arises indeed from the fact that our theory is not valid for arbitrarily high energies. What we need to do id to introduce a cut-off, i.e. a maximum energy after which the theory is not anymore valid. Since gravity cannot be neglected and becomes strongly coupled at Planck mass $M_P \simeq 10^{19} GeV$, we therefore set the cut-off at this energy. 

    Computationally, we measure only energy differences between excited estates, which are particles, and the vacuum energy, which becomes irrelevant and it can be set to zero. This procedure is called \textit{normal ordering}. We define a new hamiltonian operator 
    \begin{equation*}
        \colon \hat H \colon = \hat H - E_0 = \hat H - \bra{0} \hat H \ket{0} ~,
    \end{equation*}
    such that 
    \begin{equation*}
        \colon \hat H \colon \ket{0} = \underbrace{\hat H \ket{0}}_{E_0 \ket{0}} - E_0 \ket{0} = 0~.
    \end{equation*}
    The difference between $\hat H$ and $\colon \hat H \colon$ is due to an ambiguity in going from classical to quantum theory. Infact, normal ordering means to set a rule to order annihilation and creation operators: all annihilation operators are pleced to the right and, consequently, creation operatore to the left (dagger always first). We emphasise that in the interaction theory, vaccume energy cannot be anymore set to zero.

    As we said, different ordering in the classical hamiltonians bring different hamiltonian operators. Infact, if we rewrite the hamiltonian of the classical harmonic oscillator
    \begin{equation*}
        H = \frac{p^2}{2m} + \frac{1}{2} \omega^2 q^2 = \frac{1}{2} (\omega q - i p) (\omega q + i p) ~,
    \end{equation*}
    we notice that the first one leads us to
    \begin{equation*}
        \hat H = \omega (\hat a^\dagger \hat a + \frac{\mathbb I}{2}) ~,
    \end{equation*}
    while the second one to 
    \begin{equation*}
        \hat H = \omega a^\dagger \hat a ~.
    \end{equation*}
    \begin{proof}
        For the first hamiltonian 
        \begin{equation*}
        \begin{aligned}
            \hat H & = \frac{1}{2} (-i \sqrt{\frac{\omega}{2}} (\hat a - \hat a^\dagger))^2 + \frac{1}{2} \omega^2 (\frac{1}{\sqrt{2 \omega}} (\hat a + \hat a^\dagger))^2 \\ & = - \frac{\omega}{4} (\cancel{\hat a^2} - \hat a \hat a^\dagger - \hat a^\dagger \hat a + \cancel{(\hat a^\dagger)^2}) + \frac{\omega}{4} (\cancel{\hat a^2} + \hat a \hat a^\dagger + \hat a^\dagger \hat a + \cancel{(\hat a^\dagger)^2}) \\ & = \frac{\omega}{4} (\hat a \hat a^\dagger + \hat a^\dagger \hat a + \hat a \hat a^\dagger + \hat a^\dagger \hat a) \\ & = \frac{\omega}{2} (\underbrace{\hat a \hat a^\dagger}_{[\hat a, \hat a^\dagger] + \hat a^\dagger \hat a} + \hat a^\dagger \hat a) \\ & = \frac{\omega}{2} (\underbrace{[\hat a, \hat a^\dagger]}_{\mathbb I} + 2 \hat a^\dagger \hat a) \\ & = \omega (\frac{\mathbb I}{2}+ \hat a^\dagger \hat a) ~,
        \end{aligned}
        \end{equation*}
        while for the second hamiltonian
        \begin{equation*}
        \begin{aligned}
            \hat H & = \frac{1}{2} \Big (\omega \frac{1}{\sqrt{2 \omega}} (\hat a + \hat a^\dagger) - i (-i \sqrt{\frac{\omega}{2}} (\hat a - \hat a^\dagger)) \Big ) \Big (\omega \frac{1}{\sqrt{2 \omega}} (\hat a + \hat a^\dagger) + i (-i \sqrt{\frac{\omega}{2}} (\hat a - \hat a^\dagger)) \Big) \\ & = \frac{\omega}{4} ( \cancel{\hat a} + \hat a^\dagger - \cancel{\hat a} + \hat a^\dagger ) (\hat a + \cancel{\hat a^\dagger} + \hat a - \cancel{\hat a^\dagger}) \\ & = \omega \hat a^\dagger \hat a ~.
        \end{aligned}
        \end{equation*}
    \end{proof}

    Finally, the normal ordered hamiltonian of the Klein-Gordon theory is 
    \begin{equation} \label{hamop}
        \colon \hat H \colon = \int \frac{d^3 p}{(2\pi)^3} \omega_{\mathbf p} \hat a_{\mathbf p}^\dagger \hat a_{\mathbf p} ~.
    \end{equation}
    \begin{proof}
        Infact, since
        \begin{equation*}
            \hat H = \frac{1}{2} \int \frac{d^3 p}{(2\pi)^3} \omega_{\mathbf p} (\hat a_{\mathbf p} \hat a_{\mathbf p}^\dagger + \hat a_{\mathbf p}^\dagger \hat a_{\mathbf p}) ~,
        \end{equation*}
        we have 
        \begin{equation*}
            \colon \hat H \colon = \frac{1}{2} \int \frac{d^3 p}{(2\pi)^3} \omega_{\mathbf p} (\hat a_{\mathbf p}^\dagger \hat a_{\mathbf p} + \hat a_{\mathbf p}^\dagger \hat a_{\mathbf p}) = \int \frac{d^3 p}{(2\pi)^3} \omega_{\mathbf p} \hat a_{\mathbf p}^\dagger \hat a_{\mathbf p} ~.
        \end{equation*}
    \end{proof}

    Furthermore, by analogy of the harmonic oscillator, the hamiltonian~\eqref{hamkg} and the annihilation and creation operators satisfies the commutation relations 
    \begin{equation*}
        [\hat H, \hat a_{\mathbf p}] = - \omega_{\mathbf p} \hat a_{\mathbf p} ~, \quad [\hat H, \hat a_{\mathbf p}^\dagger] = \omega_{\mathbf p} \hat a_{\mathbf p}^\dagger ~.
    \end{equation*}
    \begin{proof}
        For the first commutator
        \begin{equation*}
        \begin{aligned}
            [\hat H, \hat a_{\mathbf p}] & = \int \frac{d^3 q}{(2\pi)^3} \omega_{\mathbf q} [\hat a_{\mathbf q}^\dagger \hat a_{\mathbf q}, \hat a_{\mathbf p}] \\ & = \int \frac{d^3 q}{(2\pi)^3} \omega_{\mathbf q} (\hat a_{\mathbf q}^\dagger \underbrace{[\hat a_{\mathbf q}, \hat a_{\mathbf p}]}_0 + \underbrace{[\hat a_{\mathbf q}^\dagger, \hat a_{\mathbf p}]}_{- (2\pi)^3 \delta^3 (\mathbf p - \mathbf q)} \hat a_{\mathbf q}) \\ & = - \int \frac{d^3 q}{\cancel{(2\pi)^3}} \omega_{\mathbf q} \cancel{(2\pi)^3} \delta^3 (\mathbf p - \mathbf q) \hat a_{\mathbf q} \\ & = - \omega_{\mathbf p} \hat a_{\mathbf p} ~.
        \end{aligned}
        \end{equation*}

        For the second commutator
        \begin{equation*}
        \begin{aligned}
            [\hat H, \hat a_{\mathbf p}^\dagger] & = \int \frac{d^3 q}{(2\pi)^3} \omega_{\mathbf q} [\hat a_{\mathbf q}^\dagger \hat a_{\mathbf q}, \hat a_{\mathbf p}^\dagger] \\ & = \int \frac{d^3 q}{(2\pi)^3} \omega_{\mathbf q} (\hat a_{\mathbf q}^\dagger \underbrace{[\hat a_{\mathbf q}, \hat a_{\mathbf p}^\dagger]}_ {(2\pi)^3 \delta^3 (\mathbf p - \mathbf q)} + \underbrace{[\hat a_{\mathbf q}^\dagger, \hat a_{\mathbf p}^\dagger]}_{0} \hat a_{\mathbf q}) \\ & = \int \frac{d^3 q}{\cancel{(2\pi)^3}} \omega_{\mathbf q} \cancel{(2\pi)^3} \delta^3 (\mathbf p - \mathbf q) \hat a_{\mathbf q}^\dagger \\ & = \omega_{\mathbf p} \hat a_{\mathbf p}^\dagger ~.
        \end{aligned}
        \end{equation*}
    \end{proof}

    The momentum operator is defined as 
    \begin{equation}\label{momop}
        \hat{\mathbf P} = - \int d^3 x ~ \hat \pi \boldsymbol \nabla \hat \varphi = \int \frac{d^3 p}{(2\pi)^3} \mathbf p \hat a_{\mathbf p}^\dagger \hat a_{\mathbf p}~.
    \end{equation}
    \begin{proof}
        Infact, using~\eqref{momen}
        \begin{equation*}
        \begin{aligned}
            \hat{\mathbf P} & = \int d^3 x ~ T^{0i} \\ & = \int d^3 x ~ \hat \pi \boldsymbol \nabla \hat \varphi ~.
        \end{aligned}
        \end{equation*}

        Furthermore, using~\eqref{anncrea},~\eqref{kgfop},~\eqref{kgpop} and~\eqref{deltaint}
        \begin{equation*}
        \begin{aligned}
            \hat{\mathbf P} & = - \int d^3 x ~ \Big ( \int \frac{d^3 p}{{(2\pi)}^3} \Big (- i\sqrt{\frac{\omega_{\mathbf p}}{2}} \Big ) \Big (\hat a_{\mathbf p} \exp(i \mathbf p \cdot \mathbf x) - \hat a_{\mathbf p}^\dagger \exp(- i \mathbf p \cdot \mathbf x) \Big) \\ & \qquad \nabla \int \frac{d^3 q}{{(2\pi)}^3} \frac{1}{\sqrt{2 \omega_{\mathbf q}}} \Big (\hat a_{\mathbf q} \exp(i \mathbf q \cdot \mathbf x) + \hat a_{\mathbf q}^\dagger \exp(- i \mathbf q \cdot \mathbf x) \Big) \Big ) \\ & = - \int d^3 x ~ \Big ( \int \frac{d^3 p}{{(2\pi)}^3} \Big (- i\sqrt{\frac{\omega_{\mathbf p}}{2}} \Big ) \Big (\hat a_{\mathbf p} \exp(i \mathbf p \cdot \mathbf x) - \hat a_{\mathbf p}^\dagger \exp(- i \mathbf p \cdot \mathbf x) \Big) \\ & \qquad \int \frac{d^3 q}{{(2\pi)}^3} \frac{1}{\sqrt{2 \omega_{\mathbf q}}} \Big (i \mathbf q \hat a_{\mathbf q} \exp(i \mathbf q \cdot \mathbf x) - i \mathbf q \hat a_{\mathbf q}^\dagger \exp(- i \mathbf q \cdot \mathbf x) \Big) \Big ) \\ & = - \int \frac{d^3 x ~ d^3 p ~ d^3 q}{{(2\pi)}^6} \Big (- \frac{i}{2} \sqrt{\frac{\omega_{\mathbf p}}{\omega_{\mathbf q}}} \Big ) (i \mathbf q \hat a_{\mathbf p} \hat a_{\mathbf q} \exp(i (\mathbf p + \mathbf q) \cdot \mathbf x) - i \mathbf q  \hat a_{\mathbf p} \hat a_{\mathbf q}^\dagger \exp(i (\mathbf p - \mathbf q) \cdot \mathbf x) \\ & \qquad - i \mathbf q \hat a_{\mathbf p}^\dagger \hat a_{\mathbf q} \exp(i (- \mathbf p + \mathbf q) \cdot \mathbf x) + i \mathbf q \hat a_{\mathbf p}^\dagger \hat a_{\mathbf q}^\dagger \exp(i (- \mathbf p - \mathbf q) \cdot \mathbf x) )
        \end{aligned}
        \end{equation*}
        \begin{equation*}
        \begin{aligned}
            \phantom{\hat{\mathbf P}} & = - \int \frac{d^3 x ~ d^3 p ~ d^3 q}{{(2\pi)}^6} \Big (\frac{\mathbf q}{2} \sqrt{\frac{\omega_{\mathbf p}}{\omega_{\mathbf q}}} \Big ) (\hat a_{\mathbf p} \hat a_{\mathbf q} \underbrace{\exp(i (\mathbf p + \mathbf q) \cdot \mathbf x)}_{\delta^3 (\mathbf p + \mathbf q)} - \hat a_{\mathbf p} \hat a_{\mathbf q}^\dagger \underbrace{\exp(i (\mathbf p - \mathbf q) \cdot \mathbf x)}_{\delta^3 (\mathbf p - \mathbf q)} \\ & \qquad - \hat a_{\mathbf p}^\dagger \hat a_{\mathbf q} \underbrace{\exp(i (-\mathbf p + \mathbf q) \cdot \mathbf x)}_{\delta^3 (\mathbf p - \mathbf q)} + \hat a_{\mathbf p}^\dagger \hat a_{\mathbf q}^\dagger \underbrace{\exp(i (-\mathbf p - \mathbf q) \cdot \mathbf x)}_{\delta^3 (\mathbf p + \mathbf q)} ) \\ & = - \int \frac{d^3 p ~ d^3 q}{{(2\pi)}^3} \Big (\frac{\mathbf q}{2} \sqrt{\frac{\omega_{\mathbf p}}{\omega_{\mathbf q}}} \Big ) (\hat a_{\mathbf p} \hat a_{\mathbf q} \underbrace{\delta^3 (\mathbf p + \mathbf q) }_{\mathbf q = - \mathbf p} - \hat a_{\mathbf p} \hat a_{\mathbf q}^\dagger \underbrace{\delta^3 (\mathbf p - \mathbf q) }_{\mathbf q = \mathbf p} \\ & \qquad - \hat a_{\mathbf p}^\dagger \hat a_{\mathbf q} \underbrace{\delta^3 (\mathbf p - \mathbf q) }_{\mathbf q = \mathbf p} + \hat a_{\mathbf p}^\dagger \hat a_{\mathbf q}^\dagger \underbrace{\delta^3 (\mathbf p + \mathbf q) }_{\mathbf q = - \mathbf p}) \\ & = - \int \frac{d^3 p}{{(2\pi)}^3} \Big (\frac{\mathbf p}{2} (- \hat a_{\mathbf p}^\dagger \hat a_{\mathbf p} - \hat a_{\mathbf p} \hat a_{\mathbf p}^\dagger) - \frac{\mathbf p}{2} (\hat a_{\mathbf p} \hat a_{- \mathbf p} + \hat a_{\mathbf p}^\dagger \hat a_{- \mathbf p}^\dagger) \Big) \\ & = \int \frac{d^3 p}{{(2\pi)}^3} \Big (\frac{\mathbf p}{2} (\hat a_{\mathbf p}^\dagger \hat a_{\mathbf p} + \hat a_{\mathbf p} \hat a_{\mathbf p}^\dagger) + \underbrace{\frac{\mathbf p}{2} (\hat a_{\mathbf p} \hat a_{- \mathbf p} + \hat a_{\mathbf p}^\dagger \hat a_{- \mathbf p}^\dagger)}_{0} \Big)  ~,
        \end{aligned}
        \end{equation*}
        where in the last row, the second term vanishes because it is an odd function integrated all over $\mathbb R^3$. Finally, in normal ordering  
        \begin{equation*}
            \hat{\mathbf P} = \int \frac{d^3 p}{{(2\pi)}^3} \Big (\frac{\mathbf p}{2} (\hat a_{\mathbf p}^\dagger \hat a_{\mathbf p} + \hat a_{\mathbf p}^\dagger \hat a_{\mathbf p}^\dagger) \Big) = \int \frac{d^3 p}{{(2\pi)}^3} \mathbf p \hat a_{\mathbf p}^\dagger \hat a_{\mathbf p} ~.
        \end{equation*}
    \end{proof}

\section{$1$-particle states}

    Now, we build the energy eigenstates of a $1$-particle state. In analogy with the harmonic oscillator, we require the following properties:
    \begin{enumerate}
        \item the vacuum state is annihilated by all the annihilation operators for all $\mathbf p$ 
            \begin{equation*}
                \hat a_{\mathbf p} \ket{0} = 0 \quad \forall \mathbf p ~,
            \end{equation*}
        \item a generic state can be defined by the creation operators actiong on the vacuum
            \begin{equation*}
                \ket{\mathbf p} = \hat a_{\mathbf p}^\dagger \ket{0} ~.
            \end{equation*}
    \end{enumerate}

    The state $\ket{\mathbf p}$ is the momentum eigentstate of a single scalar (spinless) particle with mass $m$. Infact, it is the momentum eigenstate 
    \begin{equation*}
        \hat{\mathbf P} \ket{\mathbf p} = \mathbf p \ket{\mathbf p} ~,
    \end{equation*}
    \begin{proof}
        Infact, using~\eqref{momop}
        \begin{equation*}
        \begin{aligned}
            \hat{\mathbf P} \ket{\mathbf p} & = \hat{\mathbf P} \hat a_{\mathbf p}^\dagger \ket{0} \\ & = \int \frac{d^3 q}{(2\pi)^3} \mathbf q \hat a_{\mathbf q}^\dagger \underbrace{\hat a_{\mathbf q} \hat a_{\mathbf p}^\dagger}_{[\hat a_{\mathbf q}, \hat a_{\mathbf p}^\dagger] + \hat a_{\mathbf q}^\dagger \hat a_{\mathbf p}} \ket{0} \\ & = \int \frac{d^3 q}{(2\pi)^3} \mathbf q \hat a_{\mathbf q}^\dagger (\underbrace{[\hat a_{\mathbf q}, \hat a_{\mathbf p}^\dagger]}_{(2\pi)^3 \delta^3 (\mathbf p - \mathbf q)} + \hat a_{\mathbf q}^\dagger \underbrace{\hat a_{\mathbf p}) \ket{0}}_0 \\ & = \int \frac{d^3 q}{\cancel{(2\pi)^3}} \mathbf q \hat a_{\mathbf q}^\dagger  \cancel{(2\pi)^3} \underbrace{\delta^3 (\mathbf p - \mathbf q)}_{\mathbf q = \mathbf p} \ket{0} \\ & = \mathbf p \hat a_{\mathbf p}^\dagger \ket{0} \\ & = \mathbf p \ket{\mathbf p}  ~.
        \end{aligned}
        \end{equation*}
    \end{proof}
    Furthermore, this states is also the energy eigenstate, since it is a function of the momentum 
    \begin{equation*}
        \hat H \ket{\mathbf p} = E_{\mathbf p} \ket{\mathbf p} = \omega_{\mathbf p} \ket{\mathbf p} ~.
    \end{equation*}
    \begin{proof}
        Infact, using~\eqref{hamop}
        \begin{equation*}
        \begin{aligned}
            \hat H \ket{\mathbf p} & = \hat H \hat a_{\mathbf p}^\dagger \ket{0} \\ & = \int \frac{d^3 q}{(2\pi)^3} \omega_{\mathbf q} \hat a_{\mathbf q}^\dagger \underbrace{\hat a_{\mathbf q} \hat a_{\mathbf p}^\dagger}_{[\hat a_{\mathbf q}, \hat a_{\mathbf p}^\dagger] + \hat a_{\mathbf q}^\dagger \hat a_{\mathbf p}} \ket{0} \\ & = \int \frac{d^3 q}{(2\pi)^3} \omega_{\mathbf q} \hat a_{\mathbf q}^\dagger (\underbrace{[\hat a_{\mathbf q}, \hat a_{\mathbf p}^\dagger]}_{(2\pi)^3 \delta^3 (\mathbf p - \mathbf q)} + \hat a_{\mathbf q}^\dagger \underbrace{\hat a_{\mathbf p}) \ket{0}}_0 \\ & = \int \frac{d^3 q}{\cancel{(2\pi)^3}} \omega_{\mathbf q} \hat a_{\mathbf q}^\dagger  \cancel{(2\pi)^3} \underbrace{\delta^3 (\mathbf p - \mathbf q)}_{\mathbf q = \mathbf p} \ket{0} \\ & = \omega_{\mathbf p} \hat a_{\mathbf p}^\dagger \ket{0} \\ & = \omega_{\mathbf p} \ket{\mathbf p}  ~.
        \end{aligned}
        \end{equation*}
    \end{proof}

\section{$n$-particle states}

    We can generalise for a system composed by $n$ particles. The state becomes 
    \begin{equation*}
        \ket{\mathbf p_1, \ldots \mathbf p_n} = \hat a_{\mathbf p_1}^\dagger \ldots \hat a_{\mathbf p_n}^\dagger \ket{0} ~.
    \end{equation*}
    
    Notice that the state is symmetric under exchange of any two particles, since 
    \begin{equation*}
        [\hat a_{\mathbf p_i}^\dagger, \hat a_{\mathbf p_j}^\dagger ] = 0 ~.
    \end{equation*}
    \begin{proof}
        For instance, given two particles of momenta $\mathbf p$ and $\mathbf q$, we have 
        \begin{equation*}
            \ket{\mathbf p, \mathbf q} = \hat a_{\mathbf p}^\dagger \hat a_{\mathbf q}^\dagger \ket{0} = \hat a_{\mathbf q}^\dagger  \hat a_{\mathbf p}^\dagger \ket{0} = \ket{\mathbf q, \mathbf p} ~.
        \end{equation*}
    \end{proof}
    This means that the Klein-Gordon theory describes bosons. It is indeed spin-statistics relation and it is a consequence of quantum field theory and the commutation relations imposed to quantise (not quantum mechanics).

    A basis of the Fock space is built by all the possible combination of creation operators acting on the vacuum state 
    \begin{equation*}
        \{\ket{0}, \hat a_{\mathbf p_1}^\dagger \ket{0}, \hat a_{\mathbf p_1}^\dagger \hat a_{\mathbf p_2}^\dagger \ket{0}, \ldots \} 
    \end{equation*},
    where $\ket{0}$ is the vacuum state, $\hat a_{\mathbf p_1}^\dagger \ket{0}$ is the $1$-particle state, $\hat a_{\mathbf p_1}^\dagger \hat a_{\mathbf p_2}^\dagger \ket{0}$ is the $2$-particles state, etc. The total Fock space is 
    \begin{equation*}
        \mathcal F = \bigoplus_n \mathcal H_n
    \end{equation*}
    where $\mathcal H_n$ is the Hilbert space for $n$ particles.

    We can define the number operator wich counts the number of particle in a given state 
    \begin{equation}\label{nop}
        \hat N = \int \frac{d^3 p}{(2\pi)^3} \hat a_{\mathbf p}^\dagger \hat a_{\mathbf p} ~,
    \end{equation}
    such that 
    \begin{equation*}
        \hat N \ket{\mathbf p_1, \ldots \mathbf p_n} = \hat N \hat a_{\mathbf p_1}^\dagger \ldots \hat a_{\mathbf p_n}^\dagger \ket{0} =  n \ket{\mathbf p_1, \ldots \mathbf p_n} ~.
    \end{equation*}

    Notice that the particle number is conserved, since
    \begin{equation*}
        [\hat H, \hat N] = 0 ~.
    \end{equation*}
    This means that if the system has initially $n$ particles, this number will remain the same. This happens only in a free theory, because interactions move the system between different sectors of the Fock space.
    \begin{proof}
        Infact, using~\eqref{hamop} and~\eqref{nop}
        \begin{equation*}
        \begin{aligned}
            [\hat H, \hat N] & = [\int \frac{d^3 p}{(2\pi)^3} \omega_{\mathbf p} \hat a_{\mathbf p}^\dagger \hat a_{\mathbf p}, \int \frac{d^3 q}{(2\pi)^3} \hat a_{\mathbf q}^\dagger \hat a_{\mathbf q}] \\ & = \int \frac{d^3 p ~ d^3q}{(2\pi)^6} \omega_{\mathbf p} [\hat a_{\mathbf p}^\dagger \hat a_{\mathbf p}, \hat a_{\mathbf q}^\dagger \hat a_{\mathbf q}] \\ & = \int \frac{d^3 p ~ d^3q}{(2\pi)^6} \omega_{\mathbf p} ( \hat a_{\mathbf p}^\dagger [\hat a_{\mathbf p}, \hat a_{\mathbf q}^\dagger \hat a_{\mathbf q}] + [\hat a_{\mathbf p}^\dagger, \hat a_{\mathbf q}^\dagger \hat a_{\mathbf q}] \hat a_{\mathbf p} ) \\ & = \int \frac{d^3 p ~ d^3q}{(2\pi)^6} \omega_{\mathbf p} ( \hat a_{\mathbf p}^\dagger \hat a_{\mathbf q}^\dagger \underbrace{[\hat a_{\mathbf p}, \hat a_{\mathbf q}]}_0 + \hat a_{\mathbf p}^\dagger \underbrace{[\hat a_{\mathbf p}, \hat a_{\mathbf q}^\dagger]}_{(2\pi)^3 \delta^3 (\mathbf p - \mathbf q)}  \hat a_{\mathbf q} + \hat a_{\mathbf q}^\dagger \underbrace{[\hat a_{\mathbf p}^\dagger, \hat a_{\mathbf q}]}_{- (2\pi)^3 \delta^3 (\mathbf p - \mathbf q)} \hat a_{\mathbf p} + \underbrace{[\hat a_{\mathbf p}^\dagger, \hat a_{\mathbf q}^\dagger]}_0 \hat a_{\mathbf q} \hat a_{\mathbf p} ) \\ & = \int \frac{d^3 p ~ d^3q}{(2\pi)^3} \omega_{\mathbf p} (\hat a_{\mathbf p}^\dagger \underbrace{\delta^3 (\mathbf p - \mathbf q)}_{\mathbf p = \mathbf q} \hat a_{\mathbf q} - \hat a_{\mathbf q}^\dagger \underbrace{\delta^3 (\mathbf p - \mathbf q)}_{\mathbf p = \mathbf q} \hat a_{\mathbf p} ) \\ & = \int \frac{d^3 p}{(2\pi)^3} \omega_{\mathbf p} (\cancel{\hat a_{\mathbf p}^\dagger \hat a_{\mathbf p}} - \cancel{\hat a_{\mathbf p}^\dagger \hat a_{\mathbf p}} ) = 0 ~.
        \end{aligned}
        \end{equation*}
    \end{proof}

\chapter{Two real (or complex) Klein-Gordon field}

    Consider two real Klein-Gordon fields $\varphi_1$ and $\varphi_2$ with different masses $m_1 \neq m_2$. Their lagrangian is $\forall i = 1,2$
    \begin{equation*}
        \mathcal L = \sum_{i=1}^{2} \Big ( \frac{1}{2} \partial_\mu \varphi_i \partial^\mu \varphi_i - \frac{1}{2} m_i^2 \varphi^2_i \Big) ~,
    \end{equation*}
    such that the equations of motion are $2$ independent Klein-Gordon equations for each field
    \begin{equation*}
        (\Box + m_i^2) \varphi_i (x) = 0 ~.
    \end{equation*}
    Since the fields are not interacting, in normal ordering the total hamiltonian operator is 
    \begin{equation*}
        \hat H = \hat H_1 + \hat H_2 ~,
    \end{equation*}
    where 
    \begin{equation*}
        \hat H_i = \int \frac{d^3 p}{(2\pi)^3} \omega_{i, \mathbf p} \hat a_{i, \mathbf p}^\dagger \hat a_{i, \mathbf p} 
    \end{equation*}
    and 
    \begin{equation*}
        \omega_{i, \mathbf p} = \sqrt{|\mathbf p|^2 + m_i^2} ~,
    \end{equation*}
    the total momentum operator is 
    \begin{equation*}
        \hat{\mathbf P} = \hat{\mathbf P_1} + \hat{\mathbf P_2}  ~,
    \end{equation*}
    where 
    \begin{equation*}
        \hat{\mathbf P_i} = \int \frac{d^3 p}{(2\pi)^3} \mathbf p \hat a_{i, \mathbf p}^\dagger \hat a_{i, \mathbf p} 
    \end{equation*}
    and the total number operator is 
    \begin{equation*}
        \hat N = \hat N_1 + \hat N_2 ~,
    \end{equation*}
    where 
    \begin{equation*}
        \hat N_i = \int \frac{d^3 p}{(2\pi)^3} \hat a_{i, \mathbf p}^\dagger \hat a_{i, \mathbf p} ~.
    \end{equation*}

    For particle states, the vacuum state is 
    \begin{equation*}
        \hat a_{i, \mathbf p} \ket{0} = 0 
    \end{equation*} 
    and the action of $\hat a_{i, \mathbf p}^\dagger$ on it creates a relativistic particle with mass $m_i$ 
    \begin{equation*}
        \ket{\mathbf p_i} = \hat a_{i, \mathbf p}^\dagger \ket{0} ~.
    \end{equation*}
    Furthermore, they are eigenstates of the total operators  
    \begin{equation*}
        \hat H \ket{\mathbf p_i} = \omega_{i, \mathbf p} \ket{\mathbf p_i} ~, \quad \hat{\mathbf P} \ket{\mathbf p_i} = \mathbf p \ket{\mathbf p_i} ~, \quad \hat N \ket{\mathbf p_i} = 1 \ket{\mathbf p_i} ~.
    \end{equation*}
    Notice that they seem degenerate in $\mathbf p$ since they have the same momentum, but they can always be distinguished by a measurement of their energy since it is different
    \begin{equation*}
        \omega_{1, \mathbf p} = \sqrt{|\mathbf p|^2 + m_1^2} \neq \omega_{2, \mathbf p} = \sqrt{|\mathbf p|^2 + m_2^2} ~.
    \end{equation*}

\section{Electrical charge via $O(2)$ symmetry}

    The more interesting case is the equal-mass one $m_1 = m_2$, because it arises a new symmetry of the action. Rewriting the lagrangian in term of a vector and its transpose one
    \begin{equation*}
        \boldsymbol \varphi = \begin{bmatrix}
            \varphi_1 \\ \varphi_2
        \end{bmatrix} ~, \boldsymbol \varphi^T = \begin{bmatrix}
            \varphi_1 & \varphi_2
        \end{bmatrix} ~,
    \end{equation*}
    we obtain 
    \begin{equation*}
        \mathcal L = \frac{1}{2} (\partial_\mu \boldsymbol \varphi^T) (\partial^\mu \boldsymbol \varphi) - \frac{1}{2} m^2 \boldsymbol \varphi^T \boldsymbol \varphi ~.
    \end{equation*}
    Since this lagrangian is invariant by an $O(2)$ rotation in the field space, the Noether's theorem allows us to define a charge operator
    \begin{equation*}
        \hat Q = - i \int \frac{d^3 p}{(2\pi)^3} (\hat a_{1, \mathbf p} \hat a_{2, \mathbf p}^\dagger - \hat a_{2, \mathbf p} \hat a_{1, \mathbf p}^\dagger)
    \end{equation*}
    where we have not used normal ordering. 
    \begin{proof}
        The lagrangian is invariant under an $O(2)$ rotation in the $(\varphi_1, \varphi_2)$ plane. Infact, for a rotation 
        \begin{equation*}
            \boldsymbol \varphi' = R \boldsymbol \varphi ~,
        \end{equation*}
        we have 
        \begin{equation*}
        \begin{aligned}
            \mathcal L' & = \frac{1}{2} (\partial_\mu (\boldsymbol \varphi')^T) (\partial^\mu \boldsymbol \varphi') - \frac{1}{2} m^2 (\boldsymbol \varphi')^T \boldsymbol \varphi' \\ & \\ & = \frac{1}{2} (\partial_\mu (R \boldsymbol \varphi)^T) (\partial^\mu R \boldsymbol \varphi) - \frac{1}{2} m^2 (R \boldsymbol \varphi)^T R \boldsymbol \varphi \\ & = \frac{1}{2} (\partial_\mu \boldsymbol \varphi^T) \underbrace{R^T R}_{\mathbb I} (\partial^\mu \boldsymbol \varphi) - \frac{1}{2} m^2 \boldsymbol \varphi^T \underbrace{R^T R}_{\mathbb I} \boldsymbol \varphi \\ & = \frac{1}{2} (\partial_\mu \boldsymbol \varphi^T) (\partial^\mu \boldsymbol \varphi) - \frac{1}{2} m^2 \boldsymbol \varphi^T \boldsymbol \varphi = \mathcal L ~,
        \end{aligned}
        \end{equation*}
        since the lagrangian depends only on the lenght of $|\varphi|^2 = \boldsymbol \varphi^T \boldsymbol \varphi$. Now, we compute the conserved current by considering an infinitesimal transformation matrix 
        \begin{equation*}
            R = \begin{bmatrix}
                \cos \theta & \sin \theta \\ - \sin \theta & \cos \theta \\
            \end{bmatrix} \simeq \begin{bmatrix}
                1 & \theta \\ - \theta & 1 \\
            \end{bmatrix} ~,
        \end{equation*}
        and for the fields 
        \begin{equation*}
            \begin{bmatrix}
                {\varphi'}_1 \\ {\varphi'}_2\\
            \end{bmatrix} = \begin{bmatrix}
                1 & \theta \\ - \theta & 1 \\
            \end{bmatrix} \begin{bmatrix}
                \varphi_1 \\ \varphi_2 \\
            \end{bmatrix} 
        \end{equation*} 
        which implies an infinitesimal transformation of the fields
        \begin{equation*}
            \delta \varphi_1 = {\varphi'}_1 - \varphi_1 = \theta \varphi_2 ~, \quad \delta \varphi_2 = {\varphi'}_2 - \varphi_2 = - \theta \varphi_1 ~.
        \end{equation*}

        By the Noether's theorem, the conserved current~\eqref{conscurr} is 
        \begin{equation*}
            J^\mu = \underbrace{\pdv{\mathcal L}{\partial_\mu \varphi_i}}_{\partial^\mu \varphi_i} \delta \varphi_i = \partial^\mu \varphi_1 \delta \varphi_1 + \partial^\mu \varphi_2 \delta \varphi_2 = \theta ((\partial^\mu \varphi_1) \varphi_2 - (\partial^\mu \varphi_2) \varphi_1) ~,
        \end{equation*}
        where $K^\mu = 0$, and conserved charge is 
        \begin{equation*}
            Q = \int d^3 x ~ J^0 = \int d^3 x ~ ((\partial^0 \varphi_1) \varphi_2 - (\partial^0 \varphi_2) \varphi_1) = \int d^3 x ~ (\dot \varphi_1 \varphi_2 - \dot \varphi_2 \varphi_1)
        \end{equation*}
        where we have omitted a constant $\theta$. 

        Finally, we promote it to charge operator 
        \begin{equation*}
        \begin{aligned}
            \hat Q & = \int d^3 x ~ (\hat \pi_1 \hat \varphi_2 - \hat \pi_2 \hat \varphi_1) \\ & = \int d^3 x ~ \Big ( \int \frac{d^3 q}{{(2\pi)}^3} \Big ( - i \sqrt{\frac{\omega_{\mathbf q}}{2}} \Big ) \Big ( \hat a_{1, \mathbf q} \exp(i \mathbf q \cdot \mathbf x) - \hat a_{1, \mathbf q}^\dagger \exp(- i \mathbf q \cdot \mathbf x) \Big) \\ & \qquad \int \frac{d^3 p}{{(2\pi)}^3} \frac{1}{\sqrt{2 \omega_{\mathbf p}}} \Big ( \hat a_{2, \mathbf p} \exp(i \mathbf p \cdot \mathbf x) + \hat a_{2, \mathbf p}^\dagger \exp(- i \mathbf p \cdot \mathbf x) \Big) \\ & \qquad - \int \frac{d^3 q}{{(2\pi)}^3} \Big ( - i \sqrt{\frac{\omega_{\mathbf q}}{2}} \Big ) \Big ( \hat a_{2, \mathbf q} \exp(i \mathbf q \cdot \mathbf x) - \hat a_{2, \mathbf q}^\dagger \exp(- i \mathbf q \cdot \mathbf x) \Big) \\ & \qquad \int \frac{d^3 p}{{(2\pi)}^3} \frac{1}{\sqrt{2 \omega_{\mathbf p}}} \Big ( \hat a_{1, \mathbf p} \exp(i \mathbf p \cdot \mathbf x) + \hat a_{1, \mathbf p}^\dagger \exp(- i \mathbf p \cdot \mathbf x) \Big)  \Big) \\ & = - \frac{i}{2} \int \frac{d^3 x ~ d^3 p ~ d^3 q}{(2\pi)^6} \sqrt{\frac{\omega_{\mathbf q}}{\omega_{\mathbf p}}} \Big ( \hat a_{1, \mathbf q} \hat a_{2, \mathbf p} \underbrace{\exp(i (\mathbf p + \mathbf q) \cdot \mathbf x)}_{\delta^3 (\mathbf q + \mathbf p)} + \hat a_{1, \mathbf q} \hat a_{2, \mathbf p}^\dagger \underbrace{\exp(i (- \mathbf p + \mathbf q) \cdot \mathbf x)}_{\delta^3 (\mathbf q - \mathbf p)} \\ & \qquad - \hat a_{1, \mathbf q}^\dagger \hat a_{2, \mathbf p} \underbrace{\exp(i (\mathbf p - \mathbf q) \cdot \mathbf x)}_{\delta^3 (\mathbf q - \mathbf p)} - \hat a_{1, \mathbf q}^\dagger \hat a_{2, \mathbf p}^\dagger \underbrace{\exp(i (- \mathbf p - \mathbf q) \cdot \mathbf x)}_{\delta^3 (\mathbf q + \mathbf p)} \\ & \qquad - \hat a_{2, \mathbf q} \hat a_{1, \mathbf p} \underbrace{\exp(i (\mathbf p + \mathbf q) \cdot \mathbf x)}_{\delta^3 (\mathbf q + \mathbf p)} - \hat a_{2, \mathbf q} \hat a_{1, \mathbf p}^\dagger \underbrace{\exp(i (- \mathbf p + \mathbf q) \cdot \mathbf x)}_{\delta^3 (\mathbf q - \mathbf p)} \\ & \qquad + \hat a_{2, \mathbf q}^\dagger \hat a_{1, \mathbf p} \underbrace{\exp(i (\mathbf p - \mathbf q) \cdot \mathbf x)}_{\delta^3 (\mathbf q - \mathbf p)} + \hat a_{2, \mathbf q}^\dagger \hat a_{1, \mathbf p}^\dagger \underbrace{\exp(i (- \mathbf p - \mathbf q) \cdot \mathbf x)}_{\delta^3 (\mathbf q + \mathbf p)} \Big) 
        \end{aligned}
        \end{equation*}
        \begin{equation*}
        \begin{aligned}
            \phantom{\hat Q} & = - \frac{i}{2} \int \frac{d^3 p ~ d^3 q}{(2\pi)^3} \sqrt{\frac{\omega_{\mathbf q}}{\omega_{\mathbf p}}} \Big ( \hat a_{1, \mathbf q} \hat a_{2, \mathbf p} \underbrace{\delta^3 (\mathbf q + \mathbf p)}_{\mathbf q = - \mathbf p} + \hat a_{1, \mathbf q} \hat a_{2, \mathbf p}^\dagger \underbrace{\delta^3 (\mathbf q - \mathbf p)}_{\mathbf q = \mathbf p} \\ & \qquad - \hat a_{1, \mathbf q}^\dagger \hat a_{2, \mathbf p} \underbrace{\delta^3 (\mathbf q - \mathbf p)}_{\mathbf q = \mathbf p} - \hat a_{1, \mathbf q}^\dagger \hat a_{2, \mathbf p}^\dagger \underbrace{\delta^3 (\mathbf q + \mathbf p)}_{\mathbf q = - \mathbf p} \\ & \qquad - \hat a_{2, \mathbf q} \hat a_{1, \mathbf p} \underbrace{\delta^3 (\mathbf q + \mathbf p)}_{\mathbf q = - \mathbf p} - \hat a_{2, \mathbf q} \hat a_{1, \mathbf p}^\dagger \underbrace{\delta^3 (\mathbf q - \mathbf p)}_{\mathbf q = \mathbf p} \\ & \qquad + \hat a_{2, \mathbf q}^\dagger \hat a_{1, \mathbf p} \underbrace{\delta^3 (\mathbf q - \mathbf p)}_{\mathbf q = \mathbf p} + \hat a_{2, \mathbf q}^\dagger \hat a_{1, \mathbf p}^\dagger \underbrace{\delta^3 (\mathbf q + \mathbf p)}_{\mathbf q = - \mathbf p} \Big) \\ & = - \frac{i}{2} \int \frac{d^3 p}{(2\pi)^3} ( \hat a_{1, - \mathbf p} \hat a_{2, \mathbf p} + \hat a_{1, \mathbf p} \hat a_{2, \mathbf p}^\dagger - \hat a_{1, \mathbf p}^\dagger \hat a_{2, \mathbf p} - \hat a_{1, - \mathbf p}^\dagger \hat a_{2, \mathbf p}^\dagger \\ & \qquad - \hat a_{2, - \mathbf p} \hat a_{1, \mathbf p} - \hat a_{2, \mathbf p} \hat a_{1, \mathbf p}^\dagger  + \hat a_{2, \mathbf p}^\dagger \hat a_{1, \mathbf p} + \hat a_{2, - \mathbf p}^\dagger \hat a_{1, \mathbf p}^\dagger ) \\ & = - \frac{i}{2} \Big (\int \frac{d^3 p}{(2\pi)^3} (\hat a_{1, - \mathbf p} \hat a_{2, \mathbf p} - \hat a_{2, - \mathbf p} \hat a_{1, \mathbf p} ) + \int \frac{d^3 p}{(2\pi)^3} (\hat a_{2, - \mathbf p}^\dagger \hat a_{1, \mathbf p}^\dagger - \hat a_{1, - \mathbf p}^\dagger \hat a_{2, \mathbf p}^\dagger) \\ & \qquad + \int \frac{d^3 p}{(2\pi)^3}(\hat a_{1, \mathbf p} \hat a_{2, \mathbf p}^\dagger - \hat a_{1, \mathbf p}^\dagger \hat a_{2, \mathbf p} - \hat a_{2, \mathbf p} \hat a_{1, \mathbf p}^\dagger  + \hat a_{2, \mathbf p}^\dagger \hat a_{1, \mathbf p}) \Big) \\ & = - \frac{i}{2} \int \frac{d^3 p}{(2\pi)^3} (\hat a_{1, \mathbf p} \hat a_{2, \mathbf p}^\dagger - \underbrace{\hat a_{1, \mathbf p}^\dagger \hat a_{2, \mathbf p}}_{ \hat a_{2, \mathbf p} \hat a_{1, \mathbf p}^\dagger} - \hat a_{2, \mathbf p} \hat a_{1, \mathbf p}^\dagger + \underbrace{\hat a_{2, \mathbf p}^\dagger \hat a_{1, \mathbf p}}_{\hat a_{1, \mathbf p} \hat a_{2, \mathbf p}^\dagger} ) \\ & = - \frac{i}{2} \int \frac{d^3 p}{(2\pi)^3} (\hat a_{1, \mathbf p} \hat a_{2, \mathbf p}^\dagger - \hat a_{2, \mathbf p} \hat a_{1, \mathbf p}^\dagger - \hat a_{2, \mathbf p} \hat a_{1, \mathbf p}^\dagger + \hat a_{1, \mathbf p} \hat a_{2, \mathbf p}^\dagger ) \\ & = - i \int \frac{d^3 p}{(2\pi)^3} (\hat a_{1, \mathbf p} \hat a_{2, \mathbf p}^\dagger - \hat a_{2, \mathbf p} \hat a_{1, \mathbf p}^\dagger)
        \end{aligned}
        \end{equation*}
        where in the fourth row from end, the first two integrals vanish because they are odd functions since they commute. 
    \end{proof}
    It is hermitian 
    \begin{equation*}
        \hat Q^\dagger = \hat Q ~.
    \end{equation*}
    \begin{proof}
        Infact, 
        \begin{equation*}
        \begin{aligned}
            \hat Q^\dagger & = i \int \frac{d^3 p}{(2\pi)^3} ((\hat a_{1, \mathbf p} \hat a_{2, \mathbf p}^\dagger)^\dagger - (\hat a_{2, \mathbf p} \hat a_{1, \mathbf p}^\dagger)^\dagger) \\ & = i \int \frac{d^3 p}{(2\pi)^3} (\hat a_{2, \mathbf p} \hat a_{1, \mathbf p}^\dagger - \hat a_{1, \mathbf p} \hat a_{2, \mathbf p}^\dagger ) \\ & = - i \int \frac{d^3 p}{(2\pi)^3} (\hat a_{1, \mathbf p} \hat a_{2, \mathbf p}^\dagger - \hat a_{2, \mathbf p} \hat a_{1, \mathbf p}^\dagger) = \hat Q ~.
        \end{aligned}
        \end{equation*}
    \end{proof}
    It is conserved by the hamiltonian 
    \begin{equation*}
        [\hat Q, \hat H] = 0 ~.
    \end{equation*}
    \begin{proof}
        Infact, 
        \begin{equation*}
        \begin{aligned}
            [\hat Q, \hat H] & = [- i \int \frac{d^3 p}{(2\pi)^3} (\hat a_{1, \mathbf p} \hat a_{2, \mathbf p}^\dagger - \hat a_{2, \mathbf p} \hat a_{1, \mathbf p}^\dagger), \int \frac{d^3 q}{(2\pi)^3} (\omega_{1, \mathbf q} \hat a_{1, \mathbf q}^\dagger \hat a_{1, \mathbf q} + \omega_{2, \mathbf q} \hat a_{2, \mathbf q}^\dagger \hat a_{2, \mathbf q})] \\ & = - i \int \frac{d^3 p ~ d^3 q}{(2\pi)^6} [\hat a_{1, \mathbf p} \hat a_{2, \mathbf p}^\dagger - \hat a_{2, \mathbf p} \hat a_{1, \mathbf p}^\dagger, \omega_{1, \mathbf q} \hat a_{1, \mathbf q}^\dagger \hat a_{1, \mathbf q} + \omega_{2, \mathbf q} \hat a_{2, \mathbf q}^\dagger \hat a_{2, \mathbf q}] \\ & = - i \int \frac{d^3 p ~ d^3 q}{(2\pi)^6} (\omega_{1, \mathbf q} [\hat a_{1, \mathbf p} \hat a_{2, \mathbf p}^\dagger, \hat a_{1, \mathbf q}^\dagger \hat a_{1, \mathbf q}] - \omega_{1, \mathbf q} [\hat a_{2, \mathbf p} \hat a_{1, \mathbf p}^\dagger, \hat a_{1, \mathbf q}^\dagger \hat a_{1, \mathbf q}] \\ & \qquad + \omega_{2, \mathbf q} [\hat a_{1, \mathbf p} \hat a_{2, \mathbf p}^\dagger, \hat a_{2, \mathbf q}^\dagger \hat a_{2, \mathbf q}] - \omega_{2, \mathbf q} [\hat a_{2, \mathbf p} \hat a_{1, \mathbf p}^\dagger, \hat a_{2, \mathbf q}^\dagger \hat a_{2, \mathbf q}]) \\ & = - i \int \frac{d^3 p ~ d^3 q}{(2\pi)^6} (\omega_{1, \mathbf q} \hat a_{1, \mathbf p} [\hat a_{2, \mathbf p}^\dagger, \hat a_{1, \mathbf q}^\dagger \hat a_{1, \mathbf q}] + \omega_{1, \mathbf q} [\hat a_{1, \mathbf p}, \hat a_{1, \mathbf q}^\dagger \hat a_{1, \mathbf q}] \hat a_{2, \mathbf p}^\dagger \\ & \qquad - \omega_{1, \mathbf q} \hat a_{2, \mathbf p} [\hat a_{1, \mathbf p}^\dagger, \hat a_{1, \mathbf q}^\dagger \hat a_{1, \mathbf q}] - \omega_{1, \mathbf q} [\hat a_{2, \mathbf p}, \hat a_{1, \mathbf q}^\dagger \hat a_{1, \mathbf q}]  \hat a_{1, \mathbf p}^\dagger \\ & \qquad + \omega_{2, \mathbf q} \hat a_{1, \mathbf p} [\hat a_{2, \mathbf p}^\dagger, \hat a_{2, \mathbf q}^\dagger \hat a_{2, \mathbf q}] + \omega_{2, \mathbf q}  [\hat a_{1, \mathbf p}, \hat a_{2, \mathbf q}^\dagger \hat a_{2, \mathbf q}] \hat a_{2, \mathbf p}^\dagger \\ & \qquad - \omega_{2, \mathbf q} \hat a_{2, \mathbf p} [\hat a_{1, \mathbf p}^\dagger, \hat a_{2, \mathbf q}^\dagger \hat a_{2, \mathbf q}] - \omega_{2, \mathbf q} [\hat a_{2, \mathbf p}, \hat a_{2, \mathbf q}^\dagger \hat a_{2, \mathbf q}] \hat a_{1, \mathbf p}^\dagger) 
        \end{aligned}
        \end{equation*}
        \begin{equation*}
        \begin{aligned}
            & = - i \int \frac{d^3 p ~ d^3 q}{(2\pi)^6} ( \omega_{1, \mathbf q} \hat a_{1, \mathbf p} \hat a_{1, \mathbf q}^\dagger \underbrace{[\hat a_{2, \mathbf p}^\dagger, \hat a_{1, \mathbf q}]}_0 + \omega_{1, \mathbf q} \hat a_{1, \mathbf p} \underbrace{[\hat a_{2, \mathbf p}^\dagger, \hat a_{1, \mathbf q}^\dagger]}_0 \hat a_{1, \mathbf q} \\ & \qquad + \omega_{1, \mathbf q} \hat a_{1, \mathbf q}^\dagger \underbrace{[\hat a_{1, \mathbf p}, \hat a_{1, \mathbf q}]}_0 \hat a_{2, \mathbf p}^\dagger + \omega_{1, \mathbf q} \underbrace{[\hat a_{1, \mathbf p}, \hat a_{1, \mathbf q}^\dagger]}_{(2\pi)^3 \delta^3 (\mathbf p - \mathbf q)} \hat a_{1, \mathbf q} \hat a_{2, \mathbf p}^\dagger \\ & \qquad - \omega_{1, \mathbf q} \hat a_{2, \mathbf p} \hat a_{1, \mathbf q}^\dagger  \underbrace{[\hat a_{1, \mathbf p}^\dagger, \hat a_{1, \mathbf q}]}_{- (2 \pi)^3 \delta^3 (\mathbf p - \mathbf q)} - \omega_{1, \mathbf q} \hat a_{2, \mathbf p} \underbrace{[\hat a_{1, \mathbf p}^\dagger, \hat a_{1, \mathbf q}^\dagger ]}_0 \hat a_{1, \mathbf q} \\ & \qquad - \omega_{1, \mathbf q} \hat a_{1, \mathbf q}^\dagger \underbrace{[\hat a_{2, \mathbf p}, \hat a_{1, \mathbf q}]}_0  \hat a_{1, \mathbf p}^\dagger - \omega_{1, \mathbf q} \underbrace{[\hat a_{2, \mathbf p}, \hat a_{1, \mathbf q}^\dagger]}_0 \hat a_{1, \mathbf q} \hat a_{1, \mathbf p}^\dagger  \\ & \qquad + \omega_{2, \mathbf q} \hat a_{1, \mathbf p} \hat a_{2, \mathbf q}^\dagger \underbrace{[\hat a_{2, \mathbf p}^\dagger, \hat a_{2, \mathbf q}]}_{-(2\pi)^3 \delta^3 (\mathbf p - \mathbf q)} + \omega_{2, \mathbf q} \hat a_{1, \mathbf p} \underbrace{[\hat a_{2, \mathbf p}^\dagger, \hat a_{2, \mathbf q}^\dagger]}_0 \hat a_{2, \mathbf q} \\ & \qquad + \omega_{2, \mathbf q} \hat a_{2, \mathbf q}^\dagger \underbrace{[\hat a_{1, \mathbf p}, \hat a_{2, \mathbf q}]}_0 \hat a_{2, \mathbf p}^\dagger + \omega_{2, \mathbf q}  \underbrace{[\hat a_{1, \mathbf p}, \hat a_{2, \mathbf q}^\dagger]}_0 \hat a_{2, \mathbf q} \hat a_{2, \mathbf p}^\dagger \\ & \qquad - \omega_{2, \mathbf q} \hat a_{2, \mathbf p} \hat a_{2, \mathbf q}^\dagger \underbrace{[\hat a_{1, \mathbf p}^\dagger, \hat a_{2, \mathbf q}]}_0 - \omega_{2, \mathbf q} \hat a_{2, \mathbf p} \underbrace{[\hat a_{1, \mathbf p}^\dagger, \hat a_{2, \mathbf q}^\dagger ]}_0 \hat a_{2, \mathbf q} \\ & \qquad - \omega_{2, \mathbf q} \hat a_{2, \mathbf q}^\dagger \underbrace{[\hat a_{2, \mathbf p}, \hat a_{2, \mathbf q}] }_0 \hat a_{1, \mathbf p}^\dagger - \omega_{2, \mathbf q} \underbrace{[\hat a_{2, \mathbf p}, \hat a_{2, \mathbf q}^\dagger]}_{(2\pi)^3 \delta^3 (\mathbf p - \mathbf q)} \hat a_{2, \mathbf q} \hat a_{1, \mathbf p}^\dagger) \\ & = - i \int \frac{d^3 p ~ d^3 q}{(2\pi)^6} ( \omega_{1, \mathbf q} \underbrace{[\hat a_{1, \mathbf p}, \hat a_{1, \mathbf q}^\dagger]}_{(2\pi)^3 \delta^3 (\mathbf p - \mathbf q)} \hat a_{1, \mathbf q} \hat a_{2, \mathbf p}^\dagger - \omega_{1, \mathbf q} \hat a_{2, \mathbf p} \hat a_{1, \mathbf q}^\dagger  \underbrace{[\hat a_{1, \mathbf p}^\dagger, \hat a_{1, \mathbf q}]}_{- (2 \pi)^3 \delta^3 (\mathbf p - \mathbf q)} \\ & \qquad + \omega_{2, \mathbf q} \hat a_{1, \mathbf p} \hat a_{2, \mathbf q}^\dagger \underbrace{[\hat a_{2, \mathbf p}^\dagger, \hat a_{2, \mathbf q}]}_{-(2\pi)^3 \delta^3 (\mathbf p - \mathbf q)} - \omega_{2, \mathbf q} \underbrace{[\hat a_{2, \mathbf p}, \hat a_{2, \mathbf q}^\dagger]}_{(2\pi)^3 \delta^3 (\mathbf p - \mathbf q)} \hat a_{2, \mathbf q} \hat a_{1, \mathbf p}^\dagger) \\ & = - i \int \frac{d^3 p ~ d^3 q}{(2\pi)^3} ( \omega_{1, \mathbf q} \underbrace{\delta^3 (\mathbf p - \mathbf q)}_{\mathbf q = \mathbf p} \hat a_{1, \mathbf q} \hat a_{2, \mathbf p}^\dagger + \omega_{1, \mathbf q} \hat a_{2, \mathbf p} \hat a_{1, \mathbf q}^\dagger \underbrace{\delta^3 (\mathbf p - \mathbf q)}_{\mathbf q = \mathbf p} \\ & \qquad - \omega_{2, \mathbf q} \hat a_{1, \mathbf p} \hat a_{2, \mathbf q}^\dagger \underbrace{\delta^3 (\mathbf p - \mathbf q)}_{\mathbf q = \mathbf p} - \omega_{2, \mathbf q} \underbrace{\delta^3 (\mathbf p - \mathbf q)}_{\mathbf q = \mathbf p} \hat a_{2, \mathbf q} \hat a_{1, \mathbf p}^\dagger) \\ & = - i \int \frac{d^3 p}{(2\pi)^3} ( \omega_{1, \mathbf p} \hat a_{1, \mathbf p} \hat a_{2, \mathbf p}^\dagger + \omega_{1, \mathbf p} \hat a_{2, \mathbf p} \hat a_{1, \mathbf p}^\dagger - \omega_{2, \mathbf p} \hat a_{1, \mathbf p} \hat a_{2, \mathbf p}^\dagger - \omega_{2, \mathbf p} \hat a_{2, \mathbf p} \hat a_{1, \mathbf p}^\dagger)
        \end{aligned}
        \end{equation*}
    \end{proof}
    
    Notice that the definition of the charge operator is ambiguous, since we may define a new charge operator 
    \begin{equation*}
        \hat Q' = c_1 \hat Q + c_2 ~,
    \end{equation*}
    where $c_1, c_2 \in \mathbb R$ and it still satisfy the conservation law (in the Heisenberg picture)
    \begin{equation*}
        \dv{\hat Q'}{t} = 0 ~.
    \end{equation*}
    \begin{proof}
        Infact 
        \begin{equation*}
            \dv{\hat Q'}{t} = \dv{c_1 Q + c_2}{t} = c_1 \dv{Q}{t} = 0 ~.
        \end{equation*}
    \end{proof}
    The ambiguity of $c_1$ can be used to set the units, while the one of $c_2$ can be exploit to go in normal ordering. Infact, setting $c_1 = 1$ and using
    \begin{equation*}
        \hat Q \ket{0} = 0 ~,
    \end{equation*}
    we obtain 
    \begin{equation*}
        \hat Q' = c_2 ~.
    \end{equation*}
    \begin{proof}
        Infact 
        \begin{equation}
            \hat Q' \ket{0} = \hat Q \ket{0} + c_2 \ket{0} = c_2 \ket{0} ~,
        \end{equation}
        \begin{equation*}
            \bra{0} \hat Q' \ket{0} = c_2 \braket{0}{0} = c_2
        \end{equation*}
        and 
        \begin{equation*}
            \colon \hat Q' \colon = \hat Q' - \bra{0} \hat Q' \ket{0} = \hat Q + c_2 - c_2 = \hat Q ~.
        \end{equation*}
    \end{proof}

    We define new ladder operators
    \begin{equation*}
        \hat a_{\pm, \mathbf p} = \frac{\hat a_{1, \mathbf p} \pm i \hat a_{2, \mathbf p}}{\sqrt{2}} ~, \quad \hat a_{\pm, \mathbf p}^\dagger = \frac{\hat a_{1, \mathbf p}^\dagger \mp i \hat a_{2, \mathbf p}^\dagger}{\sqrt{2}} ~,
    \end{equation*}
    such that they satisfy 
    \begin{equation*}
        [\hat Q, \hat a_{\pm, \mathbf p}] = \mp \hat a_{\pm, \mathbf p} ~, \quad [\hat Q, \hat a_{\pm, \mathbf p}^\dagger] = \pm \hat a_{\pm, \mathbf p}^\dagger ~.
    \end{equation*}
    \begin{proof}
        For the annihilation commutator 
        \begin{equation*}
        \begin{aligned}
            [\hat Q, \hat a_{\pm, \mathbf p}] & = \frac{[\hat Q, \hat a_{1, \mathbf p}] \pm i [\hat Q, \hat a_{2, \mathbf p}]}{\sqrt{2}} \\ & = - \frac{i}{\sqrt{2}} \int \frac{d^3 q}{(2\pi)^3} ( [\hat a_{1, \mathbf q} \hat a_{2, \mathbf q}^\dagger - \hat a_{2, \mathbf q} \hat a_{1, \mathbf q}^\dagger, \hat a_{1, \mathbf p}]  \pm i  [\hat a_{1, \mathbf q} \hat a_{2, \mathbf q}^\dagger - \hat a_{2, \mathbf q} \hat a_{1, \mathbf q}^\dagger, \hat a_{2, \mathbf p}]) \\ & = - \frac{i}{\sqrt{2}} \int \frac{d^3 q}{(2\pi)^3} ( [\hat a_{1, \mathbf q} \hat a_{2, \mathbf q}^\dagger, \hat a_{1, \mathbf p}] - [\hat a_{2, \mathbf q} \hat a_{1, \mathbf q}^\dagger, \hat a_{1, \mathbf p}] \pm i [\hat a_{1, \mathbf q} \hat a_{2, \mathbf q}^\dagger, \hat a_{2, \mathbf p}] \mp i [\hat a_{2, \mathbf q} \hat a_{1, \mathbf q}^\dagger, \hat a_{2, \mathbf p}]) \\ & = - \frac{i}{\sqrt{2}} \int \frac{d^3 q}{(2\pi)^3} ( \hat a_{1, \mathbf q} \underbrace{[\hat a_{2, \mathbf q}^\dagger, \hat a_{1, \mathbf p}]}_0 + \underbrace{[\hat a_{1, \mathbf q}, \hat a_{1, \mathbf p}]}_0 \hat a_{2, \mathbf q}^\dagger - \hat a_{2, \mathbf q}  \underbrace{[\hat a_{1, \mathbf q}^\dagger, \hat a_{1, \mathbf p}]}_{- (2\pi) \delta^3 (\mathbf p - \mathbf q)} - \underbrace{[\hat a_{2, \mathbf q}, \hat a_{1, \mathbf p}]}_0 \hat a_{1, \mathbf q}^\dagger \\ & \qquad \pm i \hat a_{1, \mathbf q} \underbrace{[\hat a_{2, \mathbf q}^\dagger, \hat a_{2, \mathbf p}]}_{- (2\pi)^3 \delta^3 (\mathbf p - \mathbf q)} \pm i \underbrace{[\hat a_{1, \mathbf q} , \hat a_{2, \mathbf p}]}_0 \hat a_{2, \mathbf q}^\dagger \mp i \hat a_{2, \mathbf q} \underbrace{[ \hat a_{1, \mathbf q}^\dagger, \hat a_{2, \mathbf p}]}_0 \mp i \underbrace{[\hat a_{2, \mathbf q}, \hat a_{2, \mathbf p}]}_0 \hat a_{1, \mathbf q}^\dagger) \\ & = - \frac{i}{\sqrt{2}} \int d^3 q ~ ( \hat a_{2, \mathbf q} \underbrace{\delta^3 (\mathbf p - \mathbf q)}_{\mathbf p = \mathbf q} \mp i \hat a_{1, \mathbf q} \underbrace{\delta^3 (\mathbf p - \mathbf q)}_{\mathbf p = \mathbf q} ) \\ & = - \frac{i}{\sqrt{2}} \hat a_{2, \mathbf p} - \frac{i}{\sqrt{2}} (\mp i \hat a_{1, \mathbf p}) \\ & = \frac{\mp \hat a_{1, \mathbf p} - i \hat a_{2, \mathbf p}}{\sqrt{2}} \\ & = \mp \frac{\hat a_{1, \mathbf p} \pm i \hat a_{2, \mathbf p}}{\sqrt{2}} = \mp \hat a_{\pm, \mathbf p} ~.
        \end{aligned}
        \end{equation*}

        For the creation commutator 
        \begin{equation*}
        \begin{aligned}
            [\hat Q, \hat a_{\pm, \mathbf p}^\dagger] & = \frac{[\hat Q, \hat a_{1, \mathbf p}^\dagger] \mp i [\hat Q, \hat a_{2, \mathbf p}^\dagger]}{\sqrt{2}} \\ & = - \frac{i}{\sqrt{2}} \int \frac{d^3 q}{(2\pi)^3} ( [\hat a_{1, \mathbf q} \hat a_{2, \mathbf q}^\dagger - \hat a_{2, \mathbf q} \hat a_{1, \mathbf q}^\dagger, \hat a_{1, \mathbf p}^\dagger]  \mp i  [\hat a_{1, \mathbf q} \hat a_{2, \mathbf q}^\dagger - \hat a_{2, \mathbf q} \hat a_{1, \mathbf q}^\dagger, \hat a_{2, \mathbf p}^\dagger]) \\ & = - \frac{i}{\sqrt{2}} \int \frac{d^3 q}{(2\pi)^3} ( [\hat a_{1, \mathbf q} \hat a_{2, \mathbf q}^\dagger, \hat a_{1, \mathbf p}^\dagger] - [\hat a_{2, \mathbf q} \hat a_{1, \mathbf q}^\dagger, \hat a_{1, \mathbf p}^\dagger] \mp i [\hat a_{1, \mathbf q} \hat a_{2, \mathbf q}^\dagger, \hat a_{2, \mathbf p}^\dagger] \pm i [\hat a_{2, \mathbf q} \hat a_{1, \mathbf q}^\dagger, \hat a_{2, \mathbf p}^\dagger]) \\ & = - \frac{i}{\sqrt{2}} \int \frac{d^3 q}{(2\pi)^3} ( \hat a_{1, \mathbf q} \underbrace{[\hat a_{2, \mathbf q}^\dagger, \hat a_{1, \mathbf p}^\dagger]}_0 + \underbrace{[\hat a_{1, \mathbf q}, \hat a_{1, \mathbf p}^\dagger]}_{(2\pi)^3 \delta^3 (\mathbf q - \mathbf p)} \hat a_{2, \mathbf q}^\dagger - \hat a_{2, \mathbf q}  \underbrace{[\hat a_{1, \mathbf q}^\dagger, \hat a_{1, \mathbf p}^\dagger]}_0 - \underbrace{[\hat a_{2, \mathbf q}, \hat a_{1, \mathbf p}]}_0 \hat a_{1, \mathbf q}^\dagger \\ & \qquad \mp i \hat a_{1, \mathbf q} \underbrace{[\hat a_{2, \mathbf q}^\dagger, \hat a_{2, \mathbf p}^\dagger]}_0 \mp i \underbrace{[\hat a_{1, \mathbf q} , \hat a_{2, \mathbf p}]}_0 \hat a_{2, \mathbf q}^\dagger \pm i \hat a_{2, \mathbf q} \underbrace{[ \hat a_{1, \mathbf q}^\dagger, \hat a_{2, \mathbf p}]}_0 \pm i \underbrace{[\hat a_{2, \mathbf q}, \hat a_{2, \mathbf p}^\dagger]}_{(2\pi)^3 \delta^3 (\mathbf p - \mathbf q)} \hat a_{1, \mathbf q}^\dagger) \\ & = - \frac{i}{\sqrt{2}} \int d^3 q ~ ( \underbrace{\delta^3 (\mathbf q - \mathbf p)}_{\mathbf p = \mathbf q} \hat a_{2, \mathbf q}^\dagger \pm i \underbrace{\delta^3 (\mathbf p - \mathbf q)}_{\mathbf p = \mathbf q} \hat a_{1, \mathbf q}^\dagger ) \\ & = - \frac{i}{\sqrt{2}} \hat a_{2, \mathbf p}^\dagger \pm \frac{1}{\sqrt{2}} \hat a_{1, \mathbf p}^\dagger \\ & = \pm \frac{\hat a_{1, \mathbf p}^\dagger \mp i \hat a_{2, \mathbf p}^\dagger}{\sqrt{2}} = \pm \hat a_{\pm, \mathbf p}^\dagger ~.
        \end{aligned}
        \end{equation*}
    \end{proof}

    Now, we study the spectrum of $\hat Q$. We define the eigenstate of the charge operators as 
    \begin{equation*}
        \hat Q \ket{S} = q \ket{S} ~,
    \end{equation*}
    where $q$ is the charge. The action of the ladder operators~\eqref{ladd} is to add or subtract a unit of charge in the system
    \begin{equation*}
        \hat Q \hat a_{\pm, \mathbf p}^\dagger \ket{S} = (q \pm 1) \hat a_{\pm, \mathbf p}^\dagger ~.
    \end{equation*}
    \begin{proof}
        Infact, 
        \begin{equation*}
            \underbrace{\hat Q \hat a_{\pm, \mathbf p}^\dagger}_{[\hat Q \hat,  a_{\pm, \mathbf p}^\dagger] + a_{\pm, \mathbf p}^\dagger \hat Q } \ket{S} = \underbrace{[\hat Q \hat,  a_{\pm, \mathbf p}^\dagger]}_{\pm a_{\pm, \mathbf p}^\dagger} \ket{S} + a_{\pm, \mathbf p}^\dagger \hat Q \ket{S} = \pm a_{\pm, \mathbf p}^\dagger \ket{S} + a_{\pm, \mathbf p}^\dagger \underbrace{\hat Q \ket{S}}_{q }\ket{S} = (q \pm 1) \hat a_{\pm, \mathbf p}^\dagger ~.
        \end{equation*}
    \end{proof}

    Since $\hat Q$ commute with $\hat H$ and $\hat{\mathbf P}$ and the three operators are linear combinations of the ladder operators, $\ket{S}$ is a common eigenstate of them. Consider a system of $n$ particles such that they have charge $\pm q$. Therefore the common eigenstates $\ket{S_n^\pm}$ are defined by 
    \begin{equation*}
        \hat Q \ket{0} = 0 ~, \quad \ket{S_n^\pm} = \prod_{i=1}^n \hat a_{\pm, \mathbf p}^\dagger \ket{0} ~,
    \end{equation*}
    where $\quad \ket{S_n^+}$ corresponds to $n$ positively-charge particles and $\quad \ket{S_n^-}$ corresponds to $n$ negatively-charge particles, such that they satisfy the properties
    \begin{equation*}
        \hat H \ket{S_n^\pm} = \Big (\sum_{i = 1}^{n} \omega_{\mathbf p_i} \Big ) \ket{S_n^\pm} ~, \quad \hat{\mathbf P} \ket{S_n^\pm} = \Big (\sum_{i = 1}^{n} \mathbf p_i \Big ) \ket{S_n^\pm} ~, \quad \hat N \ket{S_n^\pm} = n \ket{S_n^\pm}
    \end{equation*}
    and 
    \begin{equation*}
        \hat Q \ket{S_n^\pm} = \pm n \ket{S_n^\pm} ~.
    \end{equation*}
    This means that a particle states is characterised by its energy, momentum and charge eigestate. Furthermore, notice that the last espression give us the physical interpretation of the charge operator: $q$ is indeed the electric charge such that positively-charged states are particles and negatively-charged states are antiparticles. To be more precise, we could allow all also all the linear combination between them, e.g. the first is $q$, the second is $-q$, etc.

    However, a single Klein-Gordon field can describe only chargeless particles, since for $\varphi_1 = \varphi_2$ we have 
    \begin{equation*}
        \hat Q = 0 ~.
    \end{equation*}
    Hence, you need at least two degrees of freedom to descrive particles and antiparticles with non-zero electric charge.
    \begin{proof}
        Infact for $\varphi_1 = \varphi_2 = \varphi$
        \begin{equation*}
            Q = \int d^3 x ~ (\dot \varphi_1 \varphi_2 - \dot \varphi_2 \varphi_1) = \int d^3 x ~ (\dot \varphi^2 - \dot \varphi^2) = 0 ~.
        \end{equation*}
    \end{proof}

\section{Complex Klein-Gordon field}

    The description of two real Klein-Gordon fields is equivalent to a complex Klein-Gordon field, since the degrees of freedom are still two. For a the latter, they are 
    \begin{equation*}
        \varphi = \frac{\varphi_1 + i \varphi_2}{\sqrt{2}} ~, \quad \varphi^* = \frac{\varphi_1 - i \varphi_2}{\sqrt{2}} ~
    \end{equation*}
    and the corresponding lagrangian is 
    \begin{equation*}
        \mathcal L = \partial_\mu \varphi^* \partial^\mu \varphi - m^2 \varphi^* \varphi ~.
    \end{equation*}
    \begin{proof}
        Infact 
        \begin{equation*}
        \begin{aligned}
            \mathcal L & = \partial_\mu \varphi^* \partial^\mu \varphi - m^2 \varphi^* \varphi \\ & = \partial_\mu \frac{\varphi_1 - i \varphi_2}{\sqrt{2}} \partial^\mu \frac{\varphi_1 + i \varphi_2}{\sqrt{2}} - m^2 \frac{\varphi_1 - i \varphi_2}{\sqrt{2}} \frac{\varphi_1 + i \varphi_2}{\sqrt{2}} \\ & = \frac{1}{2} \partial_\mu \varphi_1 \partial^\mu \varphi_2 - \frac{1}{2} m^2 \varphi_1 \varphi_2 ~.
        \end{aligned}
        \end{equation*}
    \end{proof}

\section{Electric charge via $U(1)$ symmetry}

    This lagrangian is invariant with an $U(1)$ rotation (which is equivalent to an $O(2)$ rotation) and the Noether's theorem allows us to define a charge operator
    \begin{equation*}
        \hat Q = \int \frac{d^3 p}{(2\pi)^3} (\hat a_{+, \mathbf p}^\dagger \hat a_{+, \mathbf p} - \hat a_{-, \mathbf p}^\dagger \hat a_{-, \mathbf p}) = \hat N_+ - \hat N_- ~,
    \end{equation*}
    where we have used normal ordering and the number operators are 
    \begin{equation*}
        \hat N_\pm = \int \frac{d^3 p}{(2\pi)^3} \hat a_{\pm, \mathbf p}^\dagger \hat a_{\pm, \mathbf p} ~.
    \end{equation*}
    This can be seen by the definition of the field operator 
    \begin{equation}\label{fieldk}
        \hat \varphi (\mathbf x) = \int \frac{d^3 p}{(2\pi)^3} \frac{1}{\sqrt{2 \omega_{\mathbf p}}} \Big ( \hat a_{+, \mathbf p} \exp(i \mathbf p \cdot \mathbf x) + \hat a_{-, \mathbf p}^\dagger \exp(- i \mathbf p \cdot \mathbf x) \Big)
    \end{equation}
    and the conjugate field operator 
    \begin{equation}\label{conj}
        \hat \varphi^* (\mathbf x) = \int \frac{d^3 p}{(2\pi)^3} \frac{1}{\sqrt{2 \omega_{\mathbf p}}} \Big ( \hat a_{-, \mathbf p} \exp(i \mathbf p \cdot \mathbf x) + \hat a_{+, \mathbf p}^\dagger \exp(- i \mathbf p \cdot \mathbf x) \Big) ~,
    \end{equation}
    where $\hat a_{+, \mathbf p}^\dagger$ create a particle with momentum $\mathbf p$ and energy $\omega_{\mathbf p}$, $\hat a_{-, \mathbf p}^\dagger$ create an antiparticle with momentum $\mathbf p$ and energy $\omega_{\mathbf p}$, $\hat a_{+, \mathbf p}$ destroys a particle with momentum $\mathbf p$ and energy $\omega_{\mathbf p}$ and $\hat a_{-, \mathbf p}$ destroys an antiparticle with momentum $\mathbf p$ and energy $\omega_{\mathbf p}$.
    \begin{proof}
        The lagrangian is invariant under a global $U(1)$ rotation. Infact, for a rotation 
        \begin{equation*}
            \varphi' = \exp(i \theta) \varphi ~, \quad {\varphi'}^* = \exp(- i \theta) \varphi^* ~,
        \end{equation*}
        we have 
        \begin{equation*}
        \begin{aligned}
            \mathcal L' & = (\partial_\mu (\varphi')^*) (\partial^\mu \varphi') - m^2 (\varphi')^* \varphi' \\ & = (\partial_\mu \varphi^*) \cancel{\exp(- i \theta)} \cancel{\exp(i \theta)} (\partial^\mu \varphi) - m^2 \varphi^* \cancel{\exp(- i \theta)} \cancel{\exp(i \theta)} \varphi \\ & = \partial_\mu \varphi^* \partial^\mu \varphi - m^2 \varphi^* \varphi = \mathcal L ~.
        \end{aligned}
        \end{equation*}
       
        Now, we compute the conserved current by considering an infinitesimal transformation for the fields 
        \begin{equation*}
            \varphi' = \exp(i \theta) \varphi \simeq \varphi + i\theta \varphi ~, \quad {\varphi'}^* = \exp(- i \theta) \varphi^* \simeq \varphi^* - i \theta \varphi^* ~,
        \end{equation*} 
        which implies an infinitesimal transformation of the fields
        \begin{equation*}
            \delta \varphi = {\varphi'} - \varphi = i \theta \varphi ~, \quad \delta \varphi^* = {\varphi'}^* - \varphi^* = - \theta \varphi^* ~.
        \end{equation*}

        By the Noether's theorem, the conserved current~\eqref{conscurr} is 
        \begin{equation*}
            J^\mu = \underbrace{\pdv{\mathcal L}{\partial_\mu \varphi_i}}_{\partial^\mu \varphi_i} \delta \varphi_i = \partial^\mu \varphi \delta \varphi + \partial^\mu \varphi^* \delta \varphi^* = i \theta ((\partial^\mu \varphi^*) \varphi - (\partial^\mu \varphi) \varphi^*) ~,
        \end{equation*}
        where $K^\mu = 0$, and conserved charge is 
        \begin{equation*}
            Q = \int d^3 x ~ J^0 = i \int d^3 x ~ ((\partial^0 \varphi^*) \varphi - (\partial^0 \varphi) \varphi^*) = i\int d^3 x ~ (\dot \varphi^* \varphi - \dot \varphi \varphi^*)
        \end{equation*}
        where we have omitted a constant $\theta$. 

        Hence, the charge is 
        \begin{equation*}   
            Q = i \int d^3 x ~(\varphi \dot \varphi^* - \varphi^* \dot \varphi) ~.
        \end{equation*}

        Now, we find the field operator, using~\eqref{kgfop} 
        \begin{equation*}
        \begin{aligned}
            \varphi & = \frac{\varphi_1 + i \varphi_2}{\sqrt{2}} \\ & = \frac{1}{\sqrt{2}} \Big (\int \frac{d^3 p}{{(2\pi)}^3} \frac{1}{\sqrt{2 \omega_{\mathbf p}}} \Big (\hat a_{1, \mathbf p} \exp(i \mathbf p \cdot \mathbf x) + \hat a_{1, \mathbf p}^\dagger \exp(- i \mathbf p \cdot \mathbf x) \Big) \\ & \qquad + i \int \frac{d^3 p}{{(2\pi)}^3} \frac{1}{\sqrt{2 \omega_{\mathbf p}}} \Big (\hat a_{2, \mathbf p} \exp(i \mathbf p \cdot \mathbf x) + \hat a_{2, \mathbf p}^\dagger \exp(- i \mathbf p \cdot \mathbf x) \Big) \Big) \\ & = \int \frac{d^3 p}{{(2\pi)}^3} \frac{1}{\sqrt{2 \omega_{\mathbf p}}} \Big ( \underbrace{\frac{\hat a_{1, \mathbf p} + i \hat a_{2, \mathbf p}}{\sqrt{2}}}_{\hat a_{+, \mathbf p}} \exp(i \mathbf p \cdot \mathbf x) + \underbrace{\frac{\hat a_{1, \mathbf p}^\dagger + i \hat a_{2, \mathbf p}^\dagger}{\sqrt{2}}}_{\hat a_{-, \mathbf p}^\dagger} \exp(- i \mathbf p \cdot \mathbf x) \Big) \\ & = \int \frac{d^3 p}{{(2\pi)}^3} \frac{1}{\sqrt{2 \omega_{\mathbf p}}} \Big ( \hat a_{+, \mathbf p} \exp(i \mathbf p \cdot \mathbf x) + \hat a_{-, \mathbf p}^\dagger \exp(- i \mathbf p \cdot \mathbf x) \Big) 
        \end{aligned}
        \end{equation*}
        and the complex conjugate field operator is
        \begin{equation*}
        \begin{aligned}
            \varphi^* & = \frac{\varphi_1 - i \varphi_2}{\sqrt{2}} \\ & = \frac{1}{\sqrt{2}} \Big (\int \frac{d^3 p}{{(2\pi)}^3} \frac{1}{\sqrt{2 \omega_{\mathbf p}}} \Big (\hat a_{1, \mathbf p} \exp(i \mathbf p \cdot \mathbf x) + \hat a_{1, \mathbf p}^\dagger \exp(- i \mathbf p \cdot \mathbf x) \Big) \\ & \qquad - i \int \frac{d^3 p}{{(2\pi)}^3} \frac{1}{\sqrt{2 \omega_{\mathbf p}}} \Big (\hat a_{2, \mathbf p} \exp(i \mathbf p \cdot \mathbf x) + \hat a_{2, \mathbf p}^\dagger \exp(- i \mathbf p \cdot \mathbf x) \Big) \Big) \\ & = \int \frac{d^3 p}{{(2\pi)}^3} \frac{1}{\sqrt{2 \omega_{\mathbf p}}} \Big ( \underbrace{\frac{\hat a_{1, \mathbf p} - i \hat a_{2, \mathbf p}}{\sqrt{2}}}_{\hat a_{-, \mathbf p}} \exp(i \mathbf p \cdot \mathbf x) + \underbrace{\frac{\hat a_{1, \mathbf p}^\dagger - i \hat a_{2, \mathbf p}^\dagger}{\sqrt{2}}}_{\hat a_{+, \mathbf p}^\dagger} \exp(- i \mathbf p \cdot \mathbf x) \Big) \\ & = \int \frac{d^3 p}{{(2\pi)}^3} \frac{1}{\sqrt{2 \omega_{\mathbf p}}} \Big ( \hat a_{-, \mathbf p} \exp(i \mathbf p \cdot \mathbf x) + \hat a_{+, \mathbf p}^\dagger \exp(- i \mathbf p \cdot \mathbf x) \Big)  ~.
        \end{aligned}
        \end{equation*}
        Furthermore, the conjugate field is 
        \begin{equation*}
            \pi = \pdv{\mathcal L}{\dot \varphi} = \dot \varphi^*
        \end{equation*}
        and the complex conjugate of the conjugate field is 
        \begin{equation*}
            \pi^* = \pdv{\mathcal L}{\dot \varphi^*} = \dot \varphi ~.
        \end{equation*}
        Hence 
        \begin{equation*}
        \begin{aligned}
            \pi^* & = \frac{\pi_1 + i \pi_2}{\sqrt{2}} \\ & = \frac{1}{\sqrt{2}} \Big (\int \frac{d^3 p}{{(2\pi)}^3} \Big ( - i \sqrt{\frac{\omega_{\mathbf p}}{2}} \Big ) \Big (\hat a_{1, \mathbf p} \exp(i \mathbf p \cdot \mathbf x) - \hat a_{1, \mathbf p}^\dagger \exp(- i \mathbf p \cdot \mathbf x) \Big) \\ & \qquad + i \int \frac{d^3 p}{{(2\pi)}^3} \Big ( - i \sqrt{\frac{\omega_{\mathbf p}}{2}} \Big ) \Big (\hat a_{2, \mathbf p} \exp(i \mathbf p \cdot \mathbf x) - \hat a_{2, \mathbf p}^\dagger \exp(- i \mathbf p \cdot \mathbf x) \Big) \Big) \\ & = \int \frac{d^3 p}{{(2\pi)}^3} \Big ( - i \sqrt{\frac{\omega_{\mathbf p}}{2}} \Big ) \Big ( \underbrace{\frac{\hat a_{1, \mathbf p} + i \hat a_{2, \mathbf p}}{\sqrt{2}}}_{\hat a_{+, \mathbf p}} \exp(i \mathbf p \cdot \mathbf x) - \underbrace{\frac{\hat a_{1, \mathbf p}^\dagger + i \hat a_{2, \mathbf p}^\dagger}{\sqrt{2}}}_{\hat a_{+, \mathbf p}^\dagger} \exp(- i \mathbf p \cdot \mathbf x) \Big) \\ & = \int \frac{d^3 p}{{(2\pi)}^3} \Big ( - i \sqrt{\frac{\omega_{\mathbf p}}{2}} \Big ) \Big ( \hat a_{+, \mathbf p} \exp(i \mathbf p \cdot \mathbf x) - \hat a_{-, \mathbf p}^\dagger \exp(- i \mathbf p \cdot \mathbf x) \Big)  
        \end{aligned}
        \end{equation*}
        and 
        \begin{equation*}
        \begin{aligned}
            \pi & = \frac{\pi_1 - i \pi_2}{\sqrt{2}} \\ & = \frac{1}{\sqrt{2}} \Big (\int \frac{d^3 p}{{(2\pi)}^3} \Big ( - i \sqrt{\frac{\omega_{\mathbf p}}{2}} \Big ) \Big (\hat a_{1, \mathbf p} \exp(i \mathbf p \cdot \mathbf x) - \hat a_{1, \mathbf p}^\dagger \exp(- i \mathbf p \cdot \mathbf x) \Big) \\ & \qquad - i \int \frac{d^3 p}{{(2\pi)}^3} \Big ( - i \sqrt{\frac{\omega_{\mathbf p}}{2}} \Big ) \Big (\hat a_{2, \mathbf p} \exp(i \mathbf p \cdot \mathbf x) - \hat a_{2, \mathbf p}^\dagger \exp(- i \mathbf p \cdot \mathbf x) \Big) \Big) \\ & = \int \frac{d^3 p}{{(2\pi)}^3} \Big ( - i \sqrt{\frac{\omega_{\mathbf p}}{2}} \Big ) \Big ( \underbrace{\frac{\hat a_{1, \mathbf p} - i \hat a_{2, \mathbf p}}{\sqrt{2}}}_{\hat a_{-, \mathbf p}} \exp(i \mathbf p \cdot \mathbf x) - \underbrace{\frac{\hat a_{1, \mathbf p}^\dagger - i \hat a_{2, \mathbf p}^\dagger}{\sqrt{2}}}_{\hat a_{+, \mathbf p}^\dagger} \exp(- i \mathbf p \cdot \mathbf x) \Big) \\ & = \int \frac{d^3 p}{{(2\pi)}^3} \Big ( - i \sqrt{\frac{\omega_{\mathbf p}}{2}} \Big ) \Big ( \hat a_{-, \mathbf p} \exp(i \mathbf p \cdot \mathbf x) - \hat a_{+, \mathbf p}^\dagger \exp(- i \mathbf p \cdot \mathbf x) \Big)  ~.
        \end{aligned}
        \end{equation*}
        Putting together
        \begin{equation*}
        \begin{aligned}
            \hat Q & = i \int d^3 x ~ (\hat \varphi \hat \pi - \hat \varphi^* \hat \pi^*) \\ & = i \int d^3 x ~ \Big ( \int \frac{d^3 p}{{(2\pi)}^3} \frac{1}{\sqrt{2 \omega_{\mathbf p}}} \Big ( \hat a_{+, \mathbf p} \exp(i \mathbf p \cdot \mathbf x) + \hat a_{-, \mathbf p}^\dagger \exp(- i \mathbf p \cdot \mathbf x) \Big) \\ & \qquad \int \frac{d^3 q}{{(2\pi)}^3} \Big ( - i \sqrt{\frac{\omega_{\mathbf q}}{2}} \Big ) \Big ( \hat a_{-, \mathbf q} \exp(i \mathbf q \cdot \mathbf x) - \hat a_{+, \mathbf q}^\dagger \exp(- i \mathbf q \cdot \mathbf x) \Big) \\ & \qquad - \int \frac{d^3 p}{{(2\pi)}^3} \frac{1}{\sqrt{2 \omega_{\mathbf p}}} \Big ( \hat a_{-, \mathbf p} \exp(i \mathbf p \cdot \mathbf x) + \hat a_{+, \mathbf p}^\dagger \exp(- i \mathbf p \cdot \mathbf x) \Big) \\ & \qquad \int \frac{d^3 q}{{(2\pi)}^3} \Big ( - i \sqrt{\frac{\omega_{\mathbf q}}{2}} \Big ) \Big ( \hat a_{+, \mathbf q} \exp(i \mathbf p \cdot \mathbf x) - \hat a_{-, \mathbf q}^\dagger \exp(- i \mathbf p \cdot \mathbf x) \Big) \Big) \\ & = \frac{1}{2} \int \frac{d^3 x ~ d^3 p ~ d^3 q}{(2\pi)^6} \sqrt{\frac{\omega_{\mathbf q}}{\omega_{\mathbf p}}} \Big ( \hat a_{+, \mathbf p} \hat a_{-, \mathbf q} \underbrace{\exp(i (\mathbf p + \mathbf q) \cdot \mathbf x)}_{\delta^3 (\mathbf q + \mathbf p)} - \hat a_{+, \mathbf p} \hat a_{+, \mathbf q}^\dagger \underbrace{\exp(i (\mathbf p - \mathbf q) \cdot \mathbf x)}_{\delta^3 (\mathbf q - \mathbf p)} \\ & \qquad + \hat a_{-, \mathbf p}^\dagger \hat a_{-, \mathbf q} \underbrace{\exp(i (- \mathbf p + \mathbf q) \cdot \mathbf x)}_{\delta^3 (\mathbf q - \mathbf p)} - \hat a_{-, \mathbf p}^\dagger \hat a_{+, \mathbf q}^\dagger \underbrace{\exp(i (- \mathbf p - \mathbf q) \cdot \mathbf x)}_{\delta^3 (\mathbf q + \mathbf p)} \\ & \qquad - \hat a_{-, \mathbf p} \hat a_{+, \mathbf q} \underbrace{\exp(i (\mathbf p + \mathbf q) \cdot \mathbf x)}_{\delta^3 (\mathbf q + \mathbf p)} + \hat a_{-, \mathbf p} \hat a_{-, \mathbf q}^\dagger \underbrace{\exp(i (\mathbf p - \mathbf q) \cdot \mathbf x)}_{\delta^3 (\mathbf q - \mathbf p)} \\ & \qquad - \hat a_{+, \mathbf p}^\dagger \hat a_{+, \mathbf q} \underbrace{\exp(i (- \mathbf p + \mathbf q) \cdot \mathbf x)}_{\delta^3 (\mathbf q - \mathbf p)} + \hat a_{+, \mathbf p}^\dagger \hat a_{-, \mathbf q}^\dagger \underbrace{\exp(i (- \mathbf p - \mathbf q) \cdot \mathbf x)}_{\delta^3 (\mathbf q + \mathbf p)} \Big) 
        \end{aligned}
        \end{equation*}
        \begin{equation*}
        \begin{aligned}
            \phantom{\hat Q} & = \frac{1}{2} \int \frac{d^3 p ~ d^3 q}{(2\pi)^3} \sqrt{\frac{\omega_{\mathbf q}}{\omega_{\mathbf p}}} \Big ( \hat a_{+, \mathbf p} \hat a_{-, \mathbf q} \underbrace{\delta^3 (\mathbf q + \mathbf p)}_{\mathbf q = - \mathbf p} - \hat a_{+, \mathbf p} \hat a_{+, \mathbf q}^\dagger \underbrace{\delta^3 (\mathbf q - \mathbf p)}_{\mathbf q = \mathbf p} \\ & \qquad + \hat a_{-, \mathbf p}^\dagger \hat a_{-, \mathbf q} \underbrace{\delta^3 (\mathbf q - \mathbf p)}_{\mathbf q = \mathbf p} - \hat a_{-, \mathbf p}^\dagger \hat a_{+, \mathbf q}^\dagger \underbrace{\delta^3 (\mathbf q + \mathbf p)}_{\mathbf q = - \mathbf p} \\ & \qquad - \hat a_{-, \mathbf p} \hat a_{+, \mathbf q} \underbrace{\delta^3 (\mathbf q + \mathbf p)}_{\mathbf q = - \mathbf p} + \hat a_{-, \mathbf p} \hat a_{-, \mathbf q}^\dagger \underbrace{\delta^3 (\mathbf q - \mathbf p)}_{\mathbf q = \mathbf p} \\ & \qquad - \hat a_{+, \mathbf p}^\dagger \hat a_{+, \mathbf q} \underbrace{\delta^3 (\mathbf q - \mathbf p)}_{\mathbf q = \mathbf p} + \hat a_{+, \mathbf p}^\dagger \hat a_{-, \mathbf q}^\dagger \underbrace{\delta^3 (\mathbf q + \mathbf p)}_{\mathbf q = - \mathbf p} \Big) \\ & = \frac{1}{2} \int \frac{d^3 p}{(2\pi)^3} ( \hat a_{+, \mathbf p} \hat a_{-, - \mathbf p} - \hat a_{+, \mathbf p} \hat a_{+, \mathbf p}^\dagger + \hat a_{-, \mathbf p}^\dagger \hat a_{-, \mathbf p} - \hat a_{-, \mathbf p}^\dagger \hat a_{+, - \mathbf p}^\dagger \\ & \qquad - \hat a_{-, \mathbf p} \hat a_{+, - \mathbf p} + \hat a_{-, \mathbf p} \hat a_{-, \mathbf p}^\dagger - \hat a_{+, \mathbf p}^\dagger \hat a_{+, \mathbf p} + \hat a_{+, \mathbf p}^\dagger \hat a_{-, - \mathbf p}^\dagger ) \\ & = \frac{1}{2} \int \frac{d^3 p}{(2\pi)^3} ( \hat a_{+, \mathbf p} \hat a_{-, - \mathbf p} - \hat a_{-, \mathbf p} \hat a_{+, - \mathbf p} ) + \frac{1}{2} \int \frac{d^3 p}{(2\pi)^3} (\hat a_{+, \mathbf p}^\dagger \hat a_{-, - \mathbf p}^\dagger - \hat a_{-, \mathbf p}^\dagger \hat a_{+, - \mathbf p}^\dagger) \\ & \qquad + \frac{1}{2} \int \frac{d^3 p}{(2\pi)^3} (\hat a_{-, \mathbf p}^\dagger \hat a_{-, \mathbf p} + \hat a_{-, \mathbf p} \hat a_{-, \mathbf p}^\dagger - \hat a_{+, \mathbf p} \hat a_{+, \mathbf p}^\dagger - \hat a_{+, \mathbf p}^\dagger \hat a_{+, \mathbf p} ) ~,
        \end{aligned}
        \end{equation*}
        where in the last row, the first two integrals vanish because they are odd functions since they commute. Finally, in normal ordering, we obtain 
        \begin{equation*}
            \hat Q = \int \frac{d^3 p}{(2\pi)^3} ( \hat a_{-, \mathbf p}^\dagger \hat a_{-, \mathbf p} - \hat a_{+, \mathbf p}^\dagger \hat a_{+, \mathbf p} ) ~.
        \end{equation*}
    \end{proof}

    We remark that this is possible because the theory is free. If there are interactions, the number operators are not anymore conserved but the charge operator still is: interactions create and destroy particles and antiparticles under the constrain of conserved total charge.

\chapter{Manifestly Lorentz covariance}

\section{Lorentz covariance}

    The vacuum state is normalised 
    \begin{equation*}
        \braket{0}{0} = 1 ~,
    \end{equation*}
    while $1$-particle states satisfiy the orthogonality relation 
    \begin{equation*}
        \braket{\mathbf p}{\mathbf q} = (2\pi)^3 \delta^3 (\mathbf p - \mathbf q) 
    \end{equation*}
    and the completeness relation 
    \begin{equation*}
        \mathbb I = \int \frac{d^3 p}{(2 \pi)^3} \ket{\mathbf p} \bra{\mathbf p} ~,
    \end{equation*}
    where $\mathbb I$ is the identity operator.
    \begin{proof}
        Maybe in the future.
    \end{proof}

    However, we want Lorentz covariance, since the identity operator is so but the right side of the completeness relation is not, given that the measure $\int d^3 p$ and the projector $\ket{\mathbf p} \bra{\mathbf p}$ are not separately so. We know that $\in d^4 p$ is Lorentz covariant, because 
    \begin{equation*}
        d^4 p' = \underbrace{|\det \Lambda|}_1 d^4 p = d^4 p ~.
    \end{equation*}
    Therefore, we change the orthogonality relation into 
    \begin{equation*}
        \braket{p}{q} = (2\pi)^3 2 \sqrt{E_{\mathbf p} E_{\mathbf q}} \delta^3 (\mathbf p - \mathbf q) 
    \end{equation*}
    and the completeness relation into 
    \begin{equation*}
        \mathbb I = \int \frac{d^4 p}{(2\pi)^3} \delta (p^2_0 - |\mathbf p|^2 - m^2) \theta(p_0) \ket{\mathbf p} \bra{\mathbf p} ~,
    \end{equation*}
    where $p_0 = E_{\mathbf p} = \sqrt{|\mathbf p|^2 + m^2}$ and the manifestly invariant states are 
    \begin{equation}
        \ket{p} = \sqrt{2E_{\mathbf p}} \ket{\mathbf p} ~.
    \end{equation}
    \begin{proof}
        Maybe in the future.
    \end{proof}

\section{Heisenberg picture}

    Classical theory is Lorentz-invariant since the lagrangian $\mathcal L$ is manifestly so. However, so far in the Schroedinger picture, we worked at a preferred time in which field operators are $\hat \varphi (\mathbf x)$ and $\hat \pi (\mathbf x)$. The time evolution is governed by the Schroedinger equation
    \begin{equation*}
        i \dv{}{t} \ket{\mathbf p(t)} = \hat H \ket{\mathbf p(t)} ~,
    \end{equation*}
    in which a state evolves as 
    \begin{equation*}
        \ket{\mathbf p(t)} = \exp(- i E_{\mathbf p} t) \ket{\mathbf p} ~.
    \end{equation*}

    In Heisenberg's picture, an operator is related to the Schroedinger's one as 
    \begin{equation*}
        \hat O_H = \exp(i \hat H t) \hat O_S \exp(- i \hat H t) ~,
    \end{equation*}
    where its time evolution is governed by the Heisenberg equation 
    \begin{equation*}
        \dv{}{\hat O_H} = i [\hat H, \hat O_H] ~.
    \end{equation*}
    \begin{proof}
        In fact 
        \begin{equation*}
        \begin{aligned}
            \dv{}{t} \hat O_H & = \dv{}{t} \Big ( \exp(i \hat H t) \hat O_S \exp(- i \hat H t) \Big) \\ & = \dv{}{t} \Big ( \exp(i \hat H t) \Big) \hat O_S \exp(- i \hat H t) + \exp(i \hat H t) \cancel{\dv{}{t} \Big (  \hat O_S  \Big)} \exp(- i \hat H t) + \exp(i \hat H t) \hat O_S \dv{}{t} \Big (  \exp(- i \hat H t) \Big ) \\ & = i \hat H \underbrace{\exp(i \hat H t) \hat O_S \exp(- i \hat H)}_{\hat O_H} - i \underbrace{\exp(i \hat H t) \hat O_S \exp(- i \hat H)}_{\hat O_H} \hat H \\ & = i \hat H \hat O_H - i \hat O_H \hat H \\ & = i [\hat H, \hat O_H] ~.
        \end{aligned}
        \end{equation*}
    \end{proof}

    Therefore, in Schoredinger picture we have $\hat \varphi(\mathbf x)$ while in Heisenberg picture we have $\hat \varphi(x)$, where they agree at $t=0$. The commutation relation at equal time $t$ becomes 
    \begin{equation*}
        [\hat \varphi(t, \mathbf x), \hat \varphi(t, \mathbf y)] = [\hat \pi(t, \mathbf x), \hat \pi(t, \mathbf y)] = 0 ~, \quad [\hat \varphi(t, \mathbf x), \hat \pi(t, \mathbf y)] = i \delta^3 (\mathbf x - \mathbf y) ~.
    \end{equation*}

    The time evolution of $\hat \varphi (x)$ is 
    \begin{equation*}
        \pdv{}{t} \hat \varphi (x) = \hat \pi (x) ~.
    \end{equation*}
    \begin{proof}
        In fact 
        \begin{equation*}
        \begin{aligned}
            \pdv{}{t} \hat \varphi (x) & = i [\hat H, \hat \varphi (x)] \\ & = i [\frac{1}{2} \int d^3 y (\hat \pi^2 (t, \mathbf y) + \nabla^2 \hat \varphi (t, \mathbf y) + m^2 \hat \varphi^2 (t, \mathbf y)) , \hat \varphi (t, \mathbf x)] \\ & = \frac{i}{2} \int d^3 y \Big (\underbrace{[\hat \pi^2 (t, \mathbf y), \hat \varphi (t, \mathbf x)]}_{\pi (t, \mathbf y) [\hat \pi (t, \mathbf y), \hat \varphi (t, \mathbf x)] + [\hat \pi (t, \mathbf y), \hat \varphi (t, \mathbf x)] \pi (t, \mathbf y)} + \underbrace{[\nabla^2 \hat \varphi (t, \mathbf y), \hat \varphi (t, \mathbf x)]}_0 + m^2 \underbrace{[\hat \varphi^2 (t, \mathbf y), \hat \varphi (t, \mathbf x)]}_0 \Big ) \\ & = \frac{i}{2} \int d^3 y \Big (\pi (t, \mathbf y) \underbrace{[\hat \pi (t, \mathbf y), \hat \varphi (t, \mathbf x)]}_{- i \delta^3 (\mathbf x - \mathbf y)} + \underbrace{[\hat \pi (t, \mathbf y), \hat \varphi (t, \mathbf x)] }_{- i \delta^3 (\mathbf x - \mathbf y)} \pi (t, \mathbf y) \Big ) \\ & = \frac{1}{2} \int d^3 y \Big (\pi (t, \mathbf y) \underbrace{\delta^3 (\mathbf x - \mathbf y)}_{\mathbf x = \mathbf y} + \underbrace{\delta^3 (\mathbf x - \mathbf y)}_{\mathbf x = \mathbf y} \pi (t, \mathbf y) \Big ) \\ & = \frac{1}{2} (\pi (t, \mathbf x) + \pi (t, \mathbf x)) \\ & = \pi (t, \mathbf x) ~,
        \end{aligned}
        \end{equation*}
        where we have used $[\boldsymbol \nabla_{\mathbf y} \hat \varphi (t, \mathbf y), \hat \varphi (t, \mathbf x)] = 0$ since the right-handed side depends on $\mathbf y$ and the left-handed side depends on $\mathbf x$.
    \end{proof}

    The time evolution of $\hat \pi (x)$ is 
    \begin{equation*}
        \pdv{}{t} \hat \pi (x) = (\nabla^2 - m^2) \hat \varphi (x) ~.
    \end{equation*}
    \begin{proof}
        In fact 
        \begin{equation*}
        \begin{aligned}
            \pdv{}{t} \hat \pi (x) & = i [\hat H, \hat \pi (x)] \\ & = i [\frac{1}{2} \int d^3 y (\hat \pi^2 (t, \mathbf y) + \nabla^2 \hat \varphi (t, \mathbf y) + m^2 \hat \varphi^2 (t, \mathbf y)) , \hat \pi (t, \mathbf x)] \\ & = \frac{i}{2} \int d^3 y \Big (\underbrace{[\hat \pi^2 (t, \mathbf y), \hat \pi (t, \mathbf x)]}_0 + \underbrace{[\nabla^2 \hat \varphi (t, \mathbf y), \hat \pi (t, \mathbf x)]}_{[ \boldsymbol \nabla_{\mathbf y} \hat \varphi (t, \mathbf y) \cdot \boldsymbol \nabla_{\mathbf y} \hat \varphi (t, \mathbf y), \hat \pi (t, \mathbf x)]} + m^2 \underbrace{[\hat \varphi^2 (t, \mathbf y), \hat \pi (t, \mathbf x)]}_{\hat \varphi (t, \mathbf y) [\hat \varphi (t, \mathbf y), \hat \pi (t, \mathbf x)] + [\hat \varphi (t, \mathbf y), \hat \pi (t, \mathbf x)] \hat \varphi (t, \mathbf y)} \Big ) \\ & = \frac{i}{2} \int d^3 y \Big (\underbrace{[\boldsymbol \nabla_{\mathbf y} \hat \varphi (t, \mathbf y) \cdot \boldsymbol \nabla_{\mathbf y} \hat \varphi (t, \mathbf y), \hat \pi (t, \mathbf x)]}_{\boldsymbol \nabla_{\mathbf y} \hat \varphi (t, \mathbf y) \cdot [\boldsymbol \nabla_{\mathbf y} \hat \varphi (t, \mathbf y), \hat \pi (t, \mathbf x)] + [\boldsymbol \nabla_{\mathbf y} \hat \varphi (t, \mathbf y), \hat \pi (t, \mathbf x)] \cdot \boldsymbol \nabla_{\mathbf y} \hat \varphi (t, \mathbf y)} \\ & \qquad + m^2 \hat \varphi (t, \mathbf y) \underbrace{[\hat \varphi (t, \mathbf y), \hat \pi (t, \mathbf x)]}_{i \delta^3 (\mathbf x - \mathbf y)} + m^2 \underbrace{[\hat \varphi (t, \mathbf y), \hat \pi (t, \mathbf x)]}_{i \delta^3 (\mathbf x - \mathbf y)} \hat 
            \varphi (t, \mathbf y) \Big ) 
        \end{aligned}    
        \end{equation*}
        \begin{equation*}
        \begin{aligned}
            \phantom{\pdv{}{t} \hat \pi (x)} & = \frac{i}{2} \int d^3 y \Big (\boldsymbol \nabla_{\mathbf y} \hat \varphi (t, \mathbf y) \cdot \underbrace{[\boldsymbol \nabla_{\mathbf y} \hat \varphi (t, \mathbf y), \hat \pi (t, \mathbf x)]}_{\boldsymbol \nabla_{\mathbf y} [\hat \varphi (t, \mathbf y), \hat \pi (t, \mathbf x)]} + \underbrace{[\boldsymbol \nabla_{\mathbf y} \hat \varphi (t, \mathbf y), \hat \pi (t, \mathbf x)]}_{\boldsymbol \nabla_{\mathbf y} [\hat \varphi (t, \mathbf y), \hat \pi (t, \mathbf x)]} \cdot \boldsymbol \nabla_{\mathbf y} \hat \varphi (t, \mathbf y) \\ & \qquad + i m^2 \hat \varphi (t, \mathbf y) i \delta^3 (\mathbf x - \mathbf y) + i m^2 \delta^3 (\mathbf x - \mathbf y) \hat \varphi (t, \mathbf y) \Big ) \\ & = \frac{i}{2} \int d^3 y \Big (\boldsymbol \nabla_{\mathbf y} \hat \varphi (t, \mathbf y) \cdot \boldsymbol \nabla_{\mathbf y} \underbrace{[\hat \varphi (t, \mathbf y), \hat \pi (t, \mathbf x)]}_{i \delta^3 (\mathbf x - \mathbf y)} + \boldsymbol \nabla_{\mathbf y} \underbrace{[\hat \varphi (t, \mathbf y), \hat \pi (t, \mathbf x)]}_{i \delta^3 (\mathbf x - \mathbf y)} \cdot \boldsymbol \nabla_{\mathbf y} \hat \varphi (t, \mathbf y) \\ & \qquad + i m^2 \hat \varphi (t, \mathbf y) i \delta^3 (\mathbf x - \mathbf y) + i m^2 \delta^3 (\mathbf x - \mathbf y) \hat \varphi (t, \mathbf y) \Big ) \\ & = \frac{i}{2} \int d^3 y \Big (i \underbrace{\boldsymbol \nabla_{\mathbf y} \hat \varphi (t, \mathbf y) \cdot \boldsymbol \nabla_{\mathbf y} \delta^3 (\mathbf x - \mathbf y)}_{- \boldsymbol \nabla_{\mathbf y} \cdot \boldsymbol \nabla_{\mathbf y} \hat \varphi (t, \mathbf y) \delta^3 (\mathbf x - \mathbf y) + \textnormal{boundary term}} + i \underbrace{\boldsymbol \nabla_{\mathbf y} \delta^3 (\mathbf x - \mathbf y) \cdot \boldsymbol \nabla_{\mathbf y} \hat \varphi (t, \mathbf y)}_{- \delta^3 (\mathbf x - \mathbf y) \boldsymbol \nabla_{\mathbf y} \cdot \boldsymbol \nabla_{\mathbf y} \hat \varphi (t, \mathbf y) + \textnormal{boundary term}} \\ & \qquad + i m^2 \hat \varphi (t, \mathbf y) i \delta^3 (\mathbf x - \mathbf y) + i m^2 \delta^3 (\mathbf x - \mathbf y) \hat \varphi (t, \mathbf y) \Big ) \\ & = \frac{i}{2} \int d^3 y \Big (- i \boldsymbol \nabla_{\mathbf y} \cdot \boldsymbol \nabla_{\mathbf y} \hat \varphi (t, \mathbf y) \underbrace{\delta^3 (\mathbf x - \mathbf y) }_{\mathbf x = \mathbf y} - i \underbrace{\delta^3 (\mathbf x - \mathbf y) }_{\mathbf x = \mathbf y} \boldsymbol \nabla_{\mathbf y} \cdot \boldsymbol \nabla_{\mathbf y} \hat \varphi (t, \mathbf y) \\ & \qquad + i m^2 \hat \varphi (t, \mathbf y) i \underbrace{\delta^3 (\mathbf x - \mathbf y) }_{\mathbf x = \mathbf y} + i m^2 \underbrace{\delta^3 (\mathbf x - \mathbf y) }_{\mathbf x = \mathbf y} \hat \varphi (t, \mathbf y) \Big ) \\ & = \boldsymbol \nabla_{\mathbf x} \cdot \boldsymbol \nabla_{\mathbf x} \hat \varphi (t, \mathbf x) - m^2 \hat \varphi (t, \mathbf x) = (\nabla^2 - m^2) \hat \varphi (t, \mathbf x) ~.
        \end{aligned}
        \end{equation*}
    \end{proof}

    Combining the two time evolutions, we obtain the Klein-Gordon equation~\eqref{kgeq}
    \begin{equation*}
        (\Box + m^2) \hat \varphi (x) ~.
    \end{equation*}
    \begin{proof}
        In fact, 
        \begin{equation*}
            \hat{\ddot \varphi} (x) =  \hat{\dot \pi} (x) = (\nabla^2 - m^2) \hat \psi(x) ~,
        \end{equation*}
        hence
        \begin{equation*}
            0 = (\pdvdu{}{t} - \nabla^2 + m^2) \hat \varphi (x) = (\Box + m^2) \hat \varphi (x) ~.
        \end{equation*}
    \end{proof}

    Since field operators evolve in time and they depend on the ladder operators, the latters evolve in time as well 
    \begin{equation*}
        (\hat a_{\mathbf p})_H = \exp(- i E_{\mathbf p} t) (\hat a_{\mathbf p})_S ~, \quad (\hat a^\dagger_{\mathbf p})_H = \exp(i E_{\mathbf p} t) (\hat a^\dagger_{\mathbf p})_S~.
    \end{equation*}
    \begin{proof}
        Using the formula $ \hat H^n \hat a_{\mathbf p} = \hat a_{\mathbf p} (\hat H - E_{\mathbf p})^n$ which can be proved 
        \begin{equation*}
            [\hat H, \hat a_{\mathbf p}] = \hat H \hat a_{\mathbf p} - \hat a_{\mathbf p} \hat H = - \omega_{\mathbf p} \hat a_{\mathbf p} = - E_{\mathbf p} \hat a_{\mathbf p} ~,
        \end{equation*}
        \begin{equation*}
            \hat H \hat a_{\mathbf p} = \hat a_{\mathbf p} (\hat H - E_{\mathbf p}) ~,
        \end{equation*}
        hence 
        \begin{equation}
            \hat H^2 \hat a_{\mathbf p} = \hat H \hat a_{\mathbf p} (\hat H - E_{\mathbf p}) = \hat a_{\mathbf p} (\hat H - E_{\mathbf p})^2 ~,
        \end{equation}
        by induction 
        \begin{equation*}
            \hat H^n \hat a_{\mathbf p} = \hat a_{\mathbf p} (\hat H - E_{\mathbf p})^n ~.
        \end{equation*}

        For the annihilation operator 
        \begin{equation*}
        \begin{aligned}
            (\hat a_{\mathbf p})_H & = \exp(i \hat H t) (\hat a_{\mathbf p})_S \exp(- i \hat H t) \\ & = \underbrace{\exp(i \hat H t)}_{\sum_n \frac{(i t)^n}{n!} \hat H^n} \hat a_{\mathbf p} \exp(- i \hat H t) \\ & = \sum_n \frac{(i t)^n}{n!} \underbrace{\hat H^n \hat a_{\mathbf p}}_{\hat a_{\mathbf p} (\hat H - E_{\mathbf p})^n} \exp(- i \hat H t) \\ & = \underbrace{\sum_n \frac{1}{n!}  (i t (\hat H - E_{\mathbf p}))^n }_{\exp(i t (\hat H - E_{\mathbf p}))}  \hat a_{\mathbf p} \exp(- i \hat H t) \\ & = \exp(i t (\cancel{\hat H} - E_{\mathbf p})) \hat a_{\mathbf p} \cancel{\exp(- i \hat H t)} \\ & = \exp(- i E_{\mathbf p} t) \hat a_{\mathbf p} ~.
        \end{aligned}
        \end{equation*}

        Using the formula $\hat H^n \hat a_{\mathbf p} = \hat a_{\mathbf p} (\hat H - E_{\mathbf p})^n$ which can be proved 
        \begin{equation*}
            [\hat H, \hat a^\dagger_{\mathbf p}] = \hat H \hat a^\dagger_{\mathbf p} - \hat a^\dagger_{\mathbf p} \hat H = \omega_{\mathbf p} \hat a^\dagger_{\mathbf p} = E_{\mathbf p} \hat a^\dagger_{\mathbf p} ~,
        \end{equation*}
        \begin{equation*}
            \hat H \hat a^\dagger_{\mathbf p} = \hat a^\dagger_{\mathbf p} (\hat H + E_{\mathbf p}) ~,
        \end{equation*}
        hence 
        \begin{equation}
            \hat H^2 \hat a^\dagger_{\mathbf p} = \hat H \hat a^\dagger_{\mathbf p} (\hat H + E_{\mathbf p}) = \hat a^\dagger_{\mathbf p} (\hat H + E_{\mathbf p})^2 ~,
        \end{equation}
        by induction 
        \begin{equation*}
            \hat H^n \hat a^\dagger_{\mathbf p} = \hat a^\dagger_{\mathbf p} (\hat H + E_{\mathbf p})^n ~.
        \end{equation*}

        For the annihilation operator 
        \begin{equation*}
        \begin{aligned}
            (\hat a^\dagger_{\mathbf p})_H & = \exp(i \hat H t) (\hat a^\dagger_{\mathbf p})_S \exp(- i \hat H t) \\ & = \underbrace{\exp(i \hat H t)}_{\sum_n \frac{(i t)^n}{n!} \hat H^n} \hat a^\dagger_{\mathbf p} \exp(- i \hat H t) \\ & = \sum_n \frac{(i t)^n}{n!} \underbrace{\hat H^n \hat a^\dagger_{\mathbf p}}_{\hat a^\dagger_{\mathbf p} (\hat H + E_{\mathbf p})^n} \exp(- i \hat H t) \\ & = \underbrace{\sum_n \frac{1}{n!}  (i t (\hat H + E_{\mathbf p}))^n }_{\exp(i t (\hat H + E_{\mathbf p}))}  \hat a^\dagger_{\mathbf p} \exp(- i \hat H t) \\ & = \exp(i t (\cancel{\hat H} + E_{\mathbf p})) \hat a^\dagger_{\mathbf p} \cancel{\exp(- i \hat H t)} \\ & = \exp(i E_{\mathbf p} t) \hat a^\dagger_{\mathbf p} ~.
        \end{aligned}
        \end{equation*}
    \end{proof}

    Finally, the field operators are 
    \begin{equation*}
        \hat \varphi (x) = \int \frac{d^3 p}{(2\pi)^3} \frac{1}{\sqrt{2 E_{\mathbf p}}} (\hat a^\dagger_{\mathbf p} \exp(i p x) + \hat a_{\mathbf p} \exp(-i p x))
    \end{equation*}
    and 
    \begin{equation*}
        \hat \pi (x) = \int \frac{d^3 p}{(2\pi)^3} \Big (- i \sqrt{\frac{ E_{\mathbf p}}{2}} \Big ) (\hat a_{\mathbf p} \exp(-i p x) - \hat a^\dagger_{\mathbf p} \exp(i p x) ) ~,
    \end{equation*}
    where $p x = p^\mu x_\mu = E_{\mathbf p} t - \mathbf p \cdot \mathbf x$. 
    \begin{proof}
        For the field operator 
        \begin{equation}
        \begin{aligned}
            \hat \varphi (x) & = \int \frac{d^3 p}{(2\pi)^3} \frac{1}{\sqrt{2 E_{\mathbf p}}} ((\hat a^\dagger_{\mathbf p})_H \exp(-i \mathbf p \cdot \mathbf x) + (\hat a_{\mathbf p})_H \exp(i \mathbf p \cdot \mathbf x)) \\ & = \int \frac{d^3 p}{(2\pi)^3} \frac{1}{\sqrt{2 E_{\mathbf p}}} (\hat a^\dagger_{\mathbf p} \exp(i (E_{\mathbf p} t - \mathbf p \cdot \mathbf x)) + \hat a_{\mathbf p} \exp(-i (E_{\mathbf p} t - \mathbf p \cdot \mathbf x))) \\ & = \int \frac{d^3 p}{(2\pi)^3} \frac{1}{\sqrt{2 E_{\mathbf p}}} (\hat a^\dagger_{\mathbf p} \exp(i px) + \hat a_{\mathbf p} \exp(-i px)) ~.
        \end{aligned}
        \end{equation}

        For the conjugate field operator 
        \begin{equation}
        \begin{aligned}
            \hat \pi (x) & = \int \frac{d^3 p}{(2\pi)^3} \Big (- i \sqrt{\frac{ E_{\mathbf p}}{2}} \Big ) (- (\hat a^\dagger_{\mathbf p})_H \exp(-i \mathbf p \cdot \mathbf x) + (\hat a_{\mathbf p})_H \exp(i \mathbf p \cdot \mathbf x)) \\ & = \int \frac{d^3 p}{(2\pi)^3} \Big (- i \sqrt{\frac{ E_{\mathbf p}}{2}} \Big ) (- \hat a^\dagger_{\mathbf p} \exp(i (E_{\mathbf p} t - \mathbf p \cdot \mathbf x)) + \hat a_{\mathbf p} \exp(-i (E_{\mathbf p} t - \mathbf p \cdot \mathbf x))) \\ & = \int \frac{d^3 p}{(2\pi)^3} \Big (- i \sqrt{\frac{ E_{\mathbf p}}{2}} \Big ) ( - \hat a^\dagger_{\mathbf p} \exp(i px) + \hat a_{\mathbf p} \exp(-i px)) ~.
        \end{aligned}
        \end{equation}
    \end{proof}

\section{Casuality} 

    So far, almost everything is manifestly Lorentz invariant except the commutation relations, because they privileged a time since they must be evaluated at equal time. We need to work out commutation relations at arbitrary times 
    \begin{equation*}
        [\hat O_1 (t, \mathbf x), \hat O_2 (t', \mathbf y)] = [\hat O_1 (x), \hat O_2 (y)] ~.
    \end{equation*}
    Its physical meaning is the effects of one onto the other one. This means that in order to preserve causality, it must be zero outside the light cone since the signal cannot travel faster than light. Obviously, if it is inside the light cone, it can be non-zero. This means that
    \begin{equation*}
        [\hat O_1 (x), \hat O_2 (y)] = 0 \quad \forall (x - y)^2 = (x^\mu - y^\mu) (x_\mu - y_\mu) < 0 ~. 
    \end{equation*}

    We define the quantity 
    \begin{equation*}
        \Delta (x - y) = [\hat \varphi (x), \hat \varphi (y)] ~.
    \end{equation*}

    It is Lorentz invariant
    \begin{equation*}
        \Delta (x - y) = \int \frac{d^3 p}{(2 \pi)^3} \frac{1}{2 E_{\mathbf p}} \Big ( \exp(- i p (x - y)) - \exp(i p (x - y)) \Big) ~.
    \end{equation*}
    \begin{proof}
        In fact, 
        \begin{equation*}
        \begin{aligned}
            \Delta (x - y) & = [\hat \varphi (x), \hat \varphi (y)] \\ & = [\int \frac{d^3 p}{{(2\pi)}^3} \frac{1}{\sqrt{2 \omega_{\mathbf p}}} \Big (\hat a_{\mathbf p} \exp(- i p x) + \hat a_{\mathbf p}^\dagger \exp(i p x) \Big) , \\ & \qquad  \int \frac{d^3 q}{{(2\pi)}^3} \frac{1}{\sqrt{2 \omega_{\mathbf p}}} \Big (\hat a_{\mathbf q} \exp(- i q y) + \hat a_{\mathbf q}^\dagger \exp( i q y) \Big)] \\ & = \int \frac{d^3 p ~ d^3 q}{{(2\pi)}^6} \frac{1}{2 \sqrt{\omega_{\mathbf p}}\omega_{\mathbf q}}  \Big ( \underbrace{[\hat a_{\mathbf p}, \hat a_{\mathbf q}]}_0 exp (- i p x - i q y) + \underbrace{[\hat a_{\mathbf p}^\dagger, \hat a_{\mathbf q}]}_{- (2\pi)^3 \delta (p - q)} exp (i p x - i q y) \\ & \qquad + \underbrace{[\hat a_{\mathbf p}, \hat a_{\mathbf q}^\dagger]}_{(2\pi)^3 \delta ( p - q)} exp (- i p x + i q y) + \underbrace{[\hat a_{\mathbf p}^\dagger, \hat a_{\mathbf p}^\dagger]}_0 exp (i p x + i q y) \Big) \\ & = \int \frac{d^3 p ~ d^3 q}{{(2\pi)}^3} \frac{1}{2 \sqrt{\omega_{\mathbf p}}\omega_{\mathbf q}}  \Big ( - \underbrace{\delta (p - q)}_{p = q} exp (i p x - i q y) + \underbrace{\delta ( p - q)}_{p = q} exp (- i p x + i q y) \Big) \\ & =  = \int \frac{d^3 p ~ d^3 q}{{(2\pi)}^3} \frac{1}{2 \omega_{\mathbf p}} \Big ( exp (- i p (x -  y)) - exp (i p (x - y) )  \Big) \\ & =
        \end{aligned}
        \end{equation*}
    \end{proof}

    It satisfies the causality constrains
    \begin{enumerate}
        \item inside the light cone, for timelike separations
            \begin{equation*}
                \Delta (x - y) \neq 0 \quad \forall (x - y)^2 > 0 ~,
            \end{equation*}
        \item outside the light cone, for spacelike separations
            \begin{equation*}
                \Delta (x - y) = 0 \quad \forall (x - y)^2 < 0 ~.
            \end{equation*}
    \end{enumerate}
    \begin{proof}
        For inside the light cone, we choose one particular case with $x^\mu = (t, 0, 0, 0)$ and $y^\mu = (0, 0, 0,0 0)$, where we are static in space. Hence 
        \begin{equation*}
        \begin{aligned}
        \Delta (x - y) & = \int \frac{d^3 p}{(2 \pi)^3} \frac{1}{2 E_{\mathbf p}} \Big ( \exp(- i \underbrace{p (x - y)}_{E_{\mathbf p} t}) - \exp(i \underbrace{p (x - y)}_{E_{\mathbf p} t}) \Big) \\ & = \int \frac{d^3 p}{(2 \pi)^3} \frac{1}{2 E_{\mathbf p}} \Big ( \exp(- i E_{\mathbf p} t) - \exp(i E_{\mathbf p} t) \Big) ~.
        \end{aligned}
        \end{equation*}
        In polar coordinates $(|p|, \theta, \varphi)$
        \begin{equation*}
        \begin{aligned}
        \Delta (x - y) & = \int_0^\infty \frac{d|p|}{(2 \pi)^3} \frac{|p|^2}{2 E_{\mathbf p}} \Big ( \exp(- i E_{\mathbf p} t) - \exp(i E_{\mathbf p} t) \Big) \underbrace{\int_0^{2\pi} d\varphi \int_0^\pi d\theta \sin \theta}_{4\pi} \\ & = 4 \pi \int_0^\infty \frac{d|p|}{(2 \pi)^3} \frac{|p|^2}{2 \sqrt{|p|^2 + m^2}} \Big ( \exp(- i \sqrt{|p|^2 + m^2} t) - \exp(i \sqrt{|p|^2 + m^2} t) \Big) ~.
        \end{aligned}
        \end{equation*}
        By a change of variable $|p| = E_{\mathbf p}$ with differential $|p|^2 d|p| = E_{\mathbf p} d E_{\mathbf p} \sqrt{E_{\mathbf p}^2 - m^2}$
        \begin{equation*}
        \begin{aligned}
            \Delta (x-y) & = \frac{1}{4\pi^2} \int_m^\infty d E_{\mathbf p} \sqrt{E_{\mathbf p} - m^2} (\exp(- i E_{\mathbf p} t) - \exp(i E_{\mathbf p} t)) \\ & = \frac{m}{8 \pi t} (Y_1(mt) + i J_1(mt) - Y_1(-mt) - i J_1(-mt)) ~,
        \end{aligned}
        \end{equation*}
        where have used the Bessel function of first order. notice that their behaviour at infinity is 
        \begin{equation*}
            J_1(x) \xrightarrow{x \rightarrow \infty} \sqrt{\frac{2\pi}{x}} \cos x ~, \quad Y_1(x) \xrightarrow{x \rightarrow \infty} \sqrt{\frac{2}{\pi x}} \sin x ~,
        \end{equation*}
        hence 
        \begin{equation*}
            Y_1(mt) + m J_2(mt) \xrightarrow{t \rightarrow \infty} \sqrt{\frac{2}{\pi mt}} (\sin (mt) + i \cos (mt)) = i \sqrt{\frac{2}{\pi mt}} \exp(- i mt)
        \end{equation*}
        and 
        \begin{equation*}
            \Delta (x-y) \xrightarrow{t \rightarrow \infty} \propto \exp(-imt) - \exp(i mt) \neq 0~.
        \end{equation*}

        For outside the light cone, because of the Lorentz invariance, we need to prove only a particular spacelike separations and it becomes true for all spacelike separations. We choose the one at the same t
        \begin{equation*}
        \begin{aligned}
            \Delta (x - y) & = \int \frac{d^3 p}{(2 \pi)^3} \frac{1}{2 E_{\mathbf p}} \Big ( \exp(- i \underbrace{p (x - y)}_{- \mathbf p \cdot (\mathbf x - \mathbf y)}) - \exp(i \underbrace{p (x - y)}_{- \mathbf p \cdot (\mathbf x - \mathbf y)}) \Big) \\ & = \int \frac{d^3 p}{(2 \pi)^3} \frac{1}{2 E_{\mathbf p}} \Big ( \exp( i\mathbf p \cdot (\mathbf x - \mathbf y)) - \exp( - i \mathbf p \cdot (\mathbf x - \mathbf y)) \Big) = 0 ~,
        \end{aligned}
        \end{equation*}
        where we echanged $\mathbf p$ in $-\mathbf p$ in the second term of the integrand.
    \end{proof}

    We have just proved that the Klein-Gordon theory preserves causality.
    
\section{Correlators}

    Another way to study is via the propagators. Consider a particle at a spacetime point $y$. What is the probability to find it st $x$? We define the propagator 
    \begin{equation*}
        D(x - y) = \bra{0} \hat \varphi (x) \hat \varphi (y) \ket{0} ~.
    \end{equation*}
    which is 
    \begin{equation}
        D(x-y) = \int \frac{d^3 p}{(2\pi)^3} \frac{1}{2 E_{\mathbf p}} \exp(- i p (x - y)) ~.
    \end{equation}
    \begin{proof}
        In fact 
        \begin{equation*}
        \begin{aligned}
            D(x- y) & = \bra{0} \hat \varphi (x) \hat \varphi (y) \ket{0} \\ & = \bra{0} \int \frac{d^3 p}{(2\pi)^3} \frac{1}{\sqrt{2 E_{\mathbf p}}} (\hat a^\dagger_{\mathbf p} \exp(i p x) + \hat a_{\mathbf p} \exp(-i p x)) \\ & \qquad \int \frac{d^3 q}{(2\pi)^3} \frac{1}{\sqrt{2 E_{\mathbf q}}} (\hat a^\dagger_{\mathbf q} \exp(i q y) + \hat a_{\mathbf q} \exp(-i q y)) \ket{0} \\ & = \int \frac{d^3 p ~ d^3 q}{(2\pi)^6} \frac{1}{2 \sqrt{E_{\mathbf p} E_{\mathbf p}}} \bra{0} \Big ( \hat a_{\mathbf p} \hat a_{\mathbf q} \exp (- i p x - i q y) + \hat a_{\mathbf p}^\dagger \hat a_{\mathbf q} \exp (i p x - i q y) \\ & \qquad + \hat a_{\mathbf p} \hat a_{\mathbf q}^\dagger \exp (- i p x + i q y) + \hat a_{\mathbf p}^\dagger \hat a_{\mathbf p}^\dagger \exp (i p x + i q y) \Big) \ket{0} \\ & = \int \frac{d^3 p ~ d^3 q}{(2\pi)^6} \frac{1}{2 \sqrt{E_{\mathbf p} E_{\mathbf p}}} \Big ( \bra{0} \hat a_{\mathbf p} \underbrace{\hat a_{\mathbf q} \ket{0}}_0 \exp (- i p x - i q y) + \bra{0} \hat a_{\mathbf p}^\dagger \underbrace{\hat a_{\mathbf q} \ket{0}}_0  \exp (i p x - i q y) \\ & \qquad + \bra{0} \underbrace{\hat a_{\mathbf p} \hat a_{\mathbf q}^\dagger}_{[\hat a_{\mathbf p}, \hat a_{\mathbf q}^\dagger] + \hat a_{\mathbf q}^\dagger \hat a_{\mathbf p}} \ket{0} \exp (- i p x + i q y) + \underbrace{\bra{0} \hat a_{\mathbf p}^\dagger}_0 \hat a_{\mathbf p}^\dagger \ket{0}  \exp (i p x + i q y) \Big) \\ & = \int \frac{d^3 p ~ d^3 q}{(2\pi)^6} \frac{1}{2 \sqrt{E_{\mathbf p} E_{\mathbf p}}} \Big (\bra{0} \underbrace{[\hat a_{\mathbf p}, \hat a_{\mathbf q}^\dagger]}_{(2\pi)^3 \delta (\mathbf p - \mathbf q)} \ket{0} \exp (- i p x + i q y) + \bra{0} \hat a_{\mathbf q}^\dagger \underbrace{\hat a_{\mathbf p} \ket{0}}_0 \exp (- i p x + i q y) \Big ) \\ & = \int \frac{d^3 p ~ d^3 q}{(2\pi)^3} \frac{1}{2 \sqrt{E_{\mathbf p} E_{\mathbf p}}} 
            \underbrace{\delta (\mathbf p - \mathbf q)}_{\mathbf p = \mathbf q} \underbrace{\braket{0}{0}}_1 \exp (- i p x + i q y) \\ & = \int \frac{d^3 p}{(2\pi)^3} \frac{1}{2 E_{\mathbf p}} \exp (- i p (x-y)) ~.
        \end{aligned}
        \end{equation*}
    \end{proof}

    Outside the light cone, it does not vanish.
    \begin{proof}
        For spacelike separation, like $(x-y) = (0, \mathbf r)$, we have 
        \begin{equation*}
            D(x-y) = \frac{1}{(2\pi)^2 r} \int_m^\infty dy ~ \frac{y \exp(- yr)}{\sqrt{y^2 - m^1}} ~,
        \end{equation*}
        or with asymptotic behaviour at infinity 
        \begin{equation*}
            D(x-y) \xrightarrow{r \rightarrow \infty} \propto \exp(-mr) ~,
        \end{equation*}
        which means that the propagators decays exponentially quickly outside the light cone but it does not vanish.
    \end{proof}

    However, causality is not violated because even though the propagator is non-zero, the commutator $\Delta (x-y)$ is so and it can be written as 
    \begin{equation*}
        \Delta (x-y) = D(x-y) - D(y-x) ~,
    \end{equation*}
    which means that there is a decostructive interference, with the Feynman interpretation as 
    \begin{equation*}
        \Delta (x-y) = \bra{0} \hat \varphi (x) \hat \varphi (y) \ket{0} - \bra{0} \hat \varphi (y) \hat \varphi (x) \ket{0} ~,
    \end{equation*}
    where the first term is the probability amplitude to go from $\mathbf y$ to $\mathbf x$ at a speed larger than that of light and the latter term is the probability amplitude to go from $\mathbf x$ to $\mathbf y$ at a speed larger than that of light. The physical intuition, for a complex Klein-Gordon field, is a particle and an antiparticle that travels backward in time
    \begin{equation*}
        \Delta (x-y) = \bra{0} \hat \varphi (x) \hat \varphi^* (y) \ket{0} - \bra{0} \hat \varphi^* (y) \hat \varphi (x) \ket{0} ~.
    \end{equation*} 
    
    
    

\part{Dirac's theory}

\chapter{Dirac's action and spinors}

    In this chapter, we will review some notion of classical Dirac theory: spinor representation of the Lorentz group, Dirac's Lagrangian that leads to the Dirac's equation and plane waves solutions.

\section{Spinor representation of the Lorentz group}    

    It follows from group theory that the reducible representation of a Dirac spinor is 
    \begin{equation*}
        {\psi'}_D = \exp(-\frac{i}{2} \omega_{\mu\nu} \Sigma^{\mu\nu}) \psi_D ~,
    \end{equation*}
    where $\psi_D$ is a four-components complex vector, $\Sigma^{\mu\nu} = \frac{i}{4} [\gamma^\mu, \gamma^\nu]$ and the gamma matrices satisfy the Clifford algebra
    \begin{equation*}
        \{\gamma^\mu, \gamma^\nu\} = 2 \eta^{\mu\nu} \mathbb I_4 ~.
    \end{equation*}
    There are several basis in which we can write down these matrices. The one we will use in these notes is the Weyl or chiral basis, in which gamma matrices are
    \begin{equation*}
        \gamma^0 = \begin{bmatrix}
            0 & \mathbb I_2 \\ \mathbb I_2 & 0 \\ 
        \end{bmatrix} ~, \quad \gamma^i = \begin{bmatrix}
            0 & \sigma^i \\ - \sigma^i & 0 \\ 
        \end{bmatrix} ~,
    \end{equation*}
    where $\sigma^i$ are the Pauli matrices. It is useful to define the matrix 
    \begin{equation*}
        S^{\mu\nu} = - i \Sigma^{\mu\nu} = \frac{1}{4} [\gamma^\mu, \gamma^\nu] ~,
    \end{equation*}
    such that a Dirac spinor transforms as
    \begin{equation}\label{lorspin}
        {\psi'}_D^\alpha (x) = \underbrace{\exp(\frac{1}{2} \omega_{\mu\nu} S^{\mu\nu})^\alpha_{\phantom \alpha \beta}}_{S^\alpha_{\phantom \alpha \beta} } \psi^\beta_D (x) = S^\alpha_{\phantom \alpha \beta} \psi^\beta_D (x) ~,
    \end{equation}
    where $\alpha, \beta = 1,2,3,4$ are not Minkovski indices.

    Now, we study the properties of gamma matrices and Dirac spinor in order to find invariant quantities in function of $\psi_D$ and the construct a Lorentz invariant Lagrangian. First, notice that 
    \begin{equation}\label{gammadag}
        (\gamma^\mu)^\dagger = \gamma^0 \gamma^\mu \gamma^0 ~.
    \end{equation}
    \begin{proof}
        For $\mu = 0$, we have
        \begin{equation*}
            (\gamma^0)^\dagger = \underbrace{\gamma^0 \gamma^0}_{\mathbb I} \gamma^0 = \gamma^0 ~.
        \end{equation*}
        For $\mu = i$, we have 
        \begin{equation*}
            (\gamma^i)^\dagger = \underbrace{\gamma^0 \gamma^i}_{- \gamma^i \gamma^0} \gamma^0 = - \underbrace{\gamma^0 \gamma^0}_{\mathbb I} \gamma^i = - \gamma^i ~.
        \end{equation*}
    \end{proof}
    The first guess to have a Lorentz invariant is the bilinear $\psi \psi^\dagger$. However, it is not a scalar, since the Dirac representation is not unitary, but it satisfies the relation 
    \begin{equation*}
        S^\dagger = \gamma^0 S^{-1} \gamma^0 ~.
    \end{equation*}. 
    \begin{proof}
        Consider the spinor representation
        \begin{equation*}
            S = \exp ( \frac{1}{2} \omega_{\mu\nu} S^{\mu\nu}) ~,
        \end{equation*}
        its hermitian 
        \begin{equation*}
            S^\dagger = \exp ( \frac{1}{2} \omega_{\mu\nu} (S^{\mu\nu})^\dagger)
        \end{equation*}
        and its inverse 
        \begin{equation*}
            S^{-1} = \exp ( - \frac{1}{2} \omega_{\mu\nu} S^{\mu\nu}) ~.
        \end{equation*}
        This implies that if $S^{-1} = S^{\dagger}$ then $(S^{\mu\nu})^\dagger = - S^{\mu\nu}$ and the generators are hermitian $(\Sigma^{\mu\nu})^\dagger = \Sigma^{\mu\nu}$. However, notice that
        \begin{equation*}
            (\Sigma^{\mu\nu})^\dagger = - \frac{i}{4} [\gamma^\mu, \gamma^\nu]^\dagger = - \frac{i}{4} [(\gamma^\nu)^\dagger, (\gamma^\mu)^\dagger] = - \Sigma^{\mu\nu} = - \frac{i}{4} [\gamma^\nu, \gamma^\mu]
        \end{equation*}
        which means that all gamma matrices are hermitian or antihermitian $(\gamma^\mu)^\dagger = \pm \gamma^\mu$. But this is impossible since spatial gamma matrices are anti-hermitian
        \begin{equation*}
            (\gamma^i)^2 = - \mathbb I ~, \quad \lambda^2 = - 1 ~, \quad \lambda \in \imm ~,
        \end{equation*}
        and temporal gamma matrix is hermitian 
        \begin{equation*}
            (\gamma^0)^2 = - \mathbb I ~, \quad \lambda^2 = 1 ~, \quad \lambda \in \real ~.
        \end{equation*}
        Hence, we find
        \begin{equation*}
            {\psi'}^\dagger \psi' = \psi^\dagger S^\dagger S \psi \neq \psi^\dagger \psi ~.
        \end{equation*}
        Finally, using~\eqref{gammadag}, we have
        \begin{equation*}
        \begin{aligned}
            (S^{\mu\nu})^\dagger & = - i (\Sigma^{\mu\nu})^\dagger = - \frac{1}{4} (\gamma^\mu \gamma^\nu + \gamma^\nu \gamma^\mu)^\dagger = - \frac{1}{4} ((\gamma^\nu)^\dagger (\gamma^\mu)^\dagger + (\gamma^\mu)^\dagger (\gamma^\nu)^\dagger ) \\ & = - \frac{1}{4} (\gamma^0 \gamma^\nu \underbrace{\gamma^0 \gamma^0}_{\mathbb I} \gamma^\mu \gamma^0 + \gamma^0 \gamma^\mu \underbrace{\gamma^0 \gamma^0}_{\mathbb I} \gamma^\nu \gamma^0) = - \gamma^0 \frac{1}{4} [\gamma^\mu, \gamma^\nu] \gamma^0 = - \gamma^0 S^{\mu\nu} \gamma^0 ~,
        \end{aligned}
        \end{equation*}
        which implies that
        \begin{equation*}
        \begin{aligned}
            S^\dagger & = \exp ( \frac{1}{2} \omega_{\mu\nu} (S^{\mu\nu})^\dagger) = \exp ( - \frac{1}{2} \omega_{\mu\nu} \gamma^0 S^{\mu\nu} \gamma^0) = \sum_k \frac{1}{k!} \Big (- \frac{1}{2} \omega_{\mu\nu} \gamma^0 S^{\mu\nu} \gamma^0 \Big)^k \\ & = \gamma^0 \sum_k \frac{1}{k!} \Big (- \frac{1}{2} \omega_{\mu\nu} S^{\mu\nu} \Big)^k \gamma^0 = \gamma^0 \exp ( - \frac{1}{2} \omega_{\mu\nu} S^{\mu\nu}) \gamma^0 = \gamma^0 S^{-1} \gamma^0 ~,
        \end{aligned}
        \end{equation*}
        where we have extracted $\gamma^0$ since in every multiplication we have $\gamma^0 \gamma^0 = \mathbb I$ except at the beginning and and the end.
    \end{proof}

    We need to look for another quantity. we define the adjoint Dirac spinor 
    \begin{equation*}
        \overline \psi (x) = \psi^\dagger (x) \gamma^0 
    \end{equation*}
    and with it, we can construct a Lorentz invariant bilinear spinor 
    \begin{equation*}
        \overline \psi \psi ~.
    \end{equation*}
    \begin{proof}
        In fact, we have
        \begin{equation*}
            \overline \psi' \psi' = {\psi'}^\dagger \gamma^0 \psi' = \psi^\dagger \underbrace{S^\dagger}_{\gamma^0 S^{-1} \gamma^0} \gamma^0 S \psi = = \psi^\dagger \gamma^0 S^{-1} \underbrace{\gamma^0 \gamma^0}_1 S \psi = \psi^\dagger \gamma^0 \underbrace{S^{-1} S}_1 \psi = \psi^\dagger \gamma^0 \psi = \overline \psi \psi ~.
        \end{equation*}
    \end{proof}

    Furthermore, it is possible to build a $4$-vector $\overline \psi \gamma^\mu \psi$ such that
    \begin{equation*}
        \overline \psi' \gamma^\mu \psi' = \overline \psi \Lambda^\mu_{\phantom \mu \nu} \gamma^\nu \psi ~,
    \end{equation*}
    or equivalently 
    \begin{equation*}
        S^{-1} \gamma^\mu S = \Lambda^\mu_{\phantom \mu \nu} \gamma^\nu ~.
    \end{equation*}
    \begin{proof}
        Maybe in the future.
    \end{proof}

    We obtained a Lorentz invariant scalar by contracting $\gamma^\mu$ with the first order derivative $\partial_\mu$. 

    Furthermore, $\Sigma^{\mu\nu}$ is a $2$-tensor 
    \begin{equation*}
        \overline \psi' \Sigma^{\mu\nu} \psi' = \overline \psi \Lambda^\mu_{\phantom \mu \alpha} \Lambda^\nu_{\phantom \nu \beta} \Sigma^{\alpha\beta} \psi ~.
    \end{equation*}
    \begin{proof}
        Maybe in the future.
    \end{proof}

\section{Dirac action}

    Now, we have all the tools to build a Lorentz invariant lagrangian
    \begin{equation*}
        \mathcal L = \overline \psi (x) \gamma^\mu \partial_\mu \psi(x) - m \overline \psi (x) \psi (x) = \overline \psi (x) (i \gamma^\mu \partial_\mu - m) \psi(x) ~.
    \end{equation*}
    We have added an $i$ factor to ensure that $\mathcal L \in \mathbb R$.
    \begin{proof}
        Maybe in the future.
    \end{proof}

    The dymensional analysis is 
    \begin{equation*}
        [S] = 0 ~, [d^4 x] = - ~, [\mathcal L] = 4~, [\psi] = \frac{3}{2} ~, [\partial_\mu] = 1 ~, [m] = 1 ~.
    \end{equation*}
    Notice that in the Klein Gordon theory, we had $[\varphi] = 1$. However, in a renormalisable theory, the coupling between operators must be of dimension $4$. This means that only terms like $\varphi \overline \psi \psi$ are allowed. Another difference in the Dirac theory is that the lagrangian in at first order whereas in the Klein-Gordon theory is at second order. This is possible only because the gamma matrices exists only in the Dirac theory, while in the Klein-Gordon we have to constract to partial derivatives to get a scalar.

    The equations of motion can be obtained by the Euler-Lagrange equations: the Dirac equation is 
    \begin{equation*}
        (i \gamma^\mu \partial_\mu - m) \psi(x) = 0
    \end{equation*}
    and the conjugate Dirac equation is 
    \begin{equation*}
        \overline \psi(x) (i \gamma^\mu \overleftarrow{\partial_\mu} + m) = 0 ~.
    \end{equation*}
    \begin{proof}
        Maybe in the future.
    \end{proof}

\section{Dirac and Klein-Gordon equations}

    The four-components of the Dirac spinor satisfy the Dirac equation, but each components separately satisfy the Klein-Gordon equation, because it means that particles ensures the mass-shell condition.
    \begin{proof}
        Maybe in the future.
    \end{proof}

\section{Chiral spinors}

    Recall that the Dirac representation $(\frac{1}{2}, 0) \oplus (0, \frac{1}{2})$ is reducible and it can be decomposed into $2$ irreducible Weyl representations $(\frac{1}{2}, 0)$ and $(0, \frac{1}{2})$. 

    We introduce the $\gamma^5$ matrix 
    \begin{equation*}
        \gamma^5 = i \gamma^0 \gamma^1 \gamma^2 \gamma^3
    \end{equation*}
    such that it satisfies 
    \begin{equation*}
        \{\gamma^\mu, \gamma^5\} = 0~, \quad (\gamma^5)^2 = \mathbb I~, \quad (\gamma^5)^\dagger = \gamma^5 ~.
    \end{equation*}
    
    In the Weyl basis it becomes 
    \begin{equation*}
        \gamma^5 = \begin{bmatrix}
            - \mathbb I_2 & 0 \\ 0 & \mathbb I_2 \\
        \end{bmatrix} ~.
    \end{equation*}

    With $\gamma^5$, we can define the projection operators 
    \begin{equation*}
        P_L = \frac{\mathbb I - \gamma^5}{2} ~, \quad P_R = \frac{\mathbb I + \gamma^5}{2} ~.
    \end{equation*}
    such that they satisfy 
    \begin{equation*}
        P_L^2 = P_L ~, \quad P_R^2 = P_R ~\quad P_L^\dagger = P_L ~, \quad P_R^\dagger = P_R ~, \quad P_L P_R = P_R P_L = 0 ~, \quad P_L + P_R = \mathbb I ~.
    \end{equation*}
    and they decompose the Dirac spinor into a left-handed Weyl spinor $\psi_L^{(W)}$ and a right-handed Weyl spinor $\psi_R^{(W)}$
    \begin{equation*}
        \psi_L = \begin{bmatrix}
            \psi_L^{(W)} \\ 0 \\
        \end{bmatrix} = P_L \psi = \frac{\mathbb I - \gamma^5}{2} \psi ~, \quad \psi_R = \begin{bmatrix}
            0 \\ \psi_R^{(W)} \\
        \end{bmatrix} = P_R \psi = \frac{\mathbb I + \gamma^5}{2} \psi ~.
    \end{equation*}
    Furthermore, their eigenvalues are 
    \begin{equation*}
        \gamma^5 \psi_L = (-1) \psi_L ~, \quad \gamma^5 \psi_R = (+1) \psi_R ~.
    \end{equation*}

    The Dirac lagrangian in terms of the Weyl spinors is 
    \begin{equation*}
        \mathcal L = \overline \psi_L i \gamma^\mu \partial_\mu \psi_L + \overline \psi_R i \gamma^\mu \partial_\mu \psi_R - m (\overline \psi_L \psi_R + \overline \psi_R \psi_L) ~.
    \end{equation*}
    Notice that for a massive fermions, we do not know if it is right-handed or left-handed because of the last mixed term. Instead for massless fermions, we know.
    \begin{proof}
        Maybe in the future.
    \end{proof}

    In terms of the Weyl spinors, the Dirac equation becomes 
    \begin{equation*}
        \begin{cases}
            i \pdv{}{t} \psi^{(W)}_R (x) + i \boldsymbol \sigma \cdot \boldsymbol \nabla \psi_R^{(W)} - m \psi_L^{(W)} = 0 \\
            i \pdv{}{t} \psi^{(W)}_L (x) + i \boldsymbol \sigma \cdot \boldsymbol \nabla \psi_L^{(W)} - m \psi_R^{(W)} = 0 
        \end{cases} ~.
    \end{equation*}
    \begin{proof}
        Maybe in the future.
    \end{proof}

    For massless fermions, which have a hamiltonian $\hat H = |\hat p|$, the Weyl equations become
    \begin{equation*}
        \begin{cases}
            (\hat{\mathbf S} \cdot \mathbf p) \psi^{(W)}_R (x) = (+1) \psi^{(W)}_R (x) \\
            (\hat{\mathbf S} \cdot \mathbf p) \psi^{(W)}_L (x) = (-1) \psi^{(W)}_L (x) \\
        \end{cases}
    \end{equation*}
    where $\mathbf p$ is the direction of motion and $\hat S$ is the spin operator. The quantity $\hat{\mathbf S} \cdot \mathbf p$ is called helicity and it is the projection of the spin along the direction of motion. 
    \begin{proof}
        Maybe in the future.
    \end{proof}

\section{Parity} 

    The parity operator transforms a right-handed Weyl spinor into a left-handed Weyl spinor and viceversa
    \begin{equation*}
        \begin{cases}
            {\psi'}_L^{(W)} = \psi_R^{(W)} \\
            {\psi'}_R^{(W)} = \psi_L^{(W)} \\
        \end{cases} ~.
    \end{equation*}
    \begin{proof}
        Maybe in the future.
    \end{proof}

\section{Solutions of the Dirac equation}

    Since each components of the Dirac spinor $\psi(x)$ satisfies the Klein-Gordon equation, the plane waves are solutions 
    \begin{equation*}
        \psi_\alpha (x) = u_\alpha (\mathbf p) \exp(- i p x) ~,
    \end{equation*}
    where $u_\alpha (\mathbf p)$ is the polarisation vector with $4$ components $\alpha = 1,2,3,4$ and $p_0 = E_{\mathbf p} = \sqrt{|\mathbf p|^2 + m^2}$. Furthermore, in order to satisfy the Dirac equation, $u_\alpha (\mathbf p)$ satisfies 
    \begin{equation}\label{cond}
        \begin{bmatrix}
            - m & p^\mu \sigma_\mu \\ p^\mu \overline \sigma_\mu & -m \\
        \end{bmatrix} u (\mathbf p) = 0 ~,
    \end{equation}
    where $\sigma^\mu = (\mathbb I_2, \sigma^i)$ and $\overline \sigma^\mu = (\mathbb I_2, - \sigma^i)$.
    \begin{proof}
        In fact, 
        \begin{equation*}
            0 = (i \gamma^\mu \partial_\mu - m) \psi(x) = (i \gamma^\mu (- i p_\mu ) - m) u(\mathbf p) \exp(- i p x) ~.
        \end{equation*}
        Hence 
        \begin{equation*}
            0 = (\gamma^\mu  p_\mu - m) u(\mathbf p) = \Big ( 
            \begin{bmatrix}
                0 & 1 \\ 1 & 0 \\ 
            \end{bmatrix} p_0 + \begin{bmatrix}
                0 & \sigma^i \\ - \sigma^i & 0
            \end{bmatrix} p_i  - m \begin{bmatrix}
                1 & 0 \\ 0 & 1 \\
            \end{bmatrix} \Big) u (\mathbf p) = \begin{bmatrix}
                - m & p^\mu \sigma_\mu \\ p^\mu \overline \sigma_\mu & -m \\
            \end{bmatrix} u (\mathbf p) = 0 ~. 
        \end{equation*}
    \end{proof}

    Moreover, we can split the polarisation vector $u (\mathbf p)$ into the right and the left-handed part 
    \begin{equation}\label{split}
        u (\mathbf p) = \begin{bmatrix}
            u_L (\mathbf p) \\ u_R (\mathbf p) \\
        \end{bmatrix} ~,
    \end{equation}
    which can be intepret as the positive frequency solution. 
    \begin{proof}
        In fact, putting~\eqref{split} into~\eqref{cond} 
        \begin{equation}\label{proof4}
            \begin{cases}
                (p^\mu \overline \sigma_\mu) u_L = m u_R \\
                (p^\mu \sigma_\mu) u_R = m u_L \\
            \end{cases} ~.
        \end{equation}

        Notice that 
        \begin{equation*}
        \begin{aligned}
            (p_\mu \sigma^\mu) (p_\nu \overline \sigma^\nu) & = (p_0 + p_i \sigma^i) (p_0 + p_j \overline \sigma^j) \\ & = (p_0 + p_i \sigma^i) (p_0 - p_j \sigma^j) \\ & = p_0^2 - p_i p_j \underbrace{\sigma^i \sigma^j}_{\delta^{ij} + i \epsilon^{ijk} \sigma_k} \\ & = p_0^2 - p_i p_j \underbrace{\delta^{ij}}_{i = j} + \cancel{i \underbrace{p_i p_j}_{symm} \underbrace{\epsilon^{ijk}}_{anti} \sigma_k} \\ & = p_0^2 - |\mathbf p|^2 = m^2 ~.
        \end{aligned}
        \end{equation*}

        We choose the form of $u_L$ such that 
        \begin{equation*}
            u_L (\mathbf p) = A p^\mu \sigma_\mu \chi ~,
        \end{equation*}
        where $A$ is a constant and $\chi$ is $2$-components spinor. 
        Hence, the first equation of~\eqref{proof4}
        \begin{equation*}
            m u_R = (p^\mu \overline \sigma_\mu) u_L = A \underbrace{(p^\mu \overline \sigma_\mu) (p^\nu \sigma_\nu)}_{m^2} \chi = A m^2 \chi
        \end{equation*}
        and 
        \begin{equation*}
            u_R (\mathbf p) = m A \chi.
        \end{equation*}
        In this way, the second equation of~\eqref{proof4} is automatically satisfied
        \begin{equation*}
            m u_L = p^\mu \sigma_\mu \underbrace{m A \chi }_{u_R} = p^\mu \sigma_\mu u_R ~.
        \end{equation*}
        Therefore 
        \begin{equation*}
            u (\mathbf p) = A \begin{bmatrix}
                (p^\mu \sigma_\mu) \chi \\ m \chi \\
            \end{bmatrix} ~.
        \end{equation*} 

        We choose $A = \frac{1}{m}$ and $\chi = \sqrt{p^\mu \overline \sigma_\mu} \xi$, where $\xi$ is a constant $2$-components spinor normalised such that $\xi^\dagger \xi = 1$. Hence 
        \begin{equation*}
            u (\mathbf p) = \frac{1}{m} \begin{bmatrix}
                (p^\mu \sigma_\mu) \sqrt{p^\nu \overline \sigma_\nu} \xi \\ m \sqrt{p^\mu \overline \sigma_\mu} \xi \\
            \end{bmatrix} = \frac{1}{m} \begin{bmatrix}
                \sqrt{p^\mu \sigma_\mu} \underbrace{\sqrt{p^\alpha \sigma_\alpha} \sqrt{p^\nu \overline \sigma_\nu}}_m \xi \\ m \sqrt{p^\mu \overline \sigma_\mu} \xi \\
            \end{bmatrix} = \begin{bmatrix}
                \sqrt{p^\mu \sigma_\mu} \xi \\ \sqrt{p^\mu \overline \sigma_\mu} \xi \\
            \end{bmatrix} ~.
            \end{equation*}
    \end{proof}

    Actually, there is another class of plane waves solutions, the negative frequency solutions 
    \begin{equation*}
        \psi(x) = v(\mathbf p) \exp(i p x) ~,
    \end{equation*}
    where $v (\mathbf p)$ is the polarisation vector 
    \begin{equation*}
        v (\mathbf p) = \begin{bmatrix}
            \sqrt{p_\mu \sigma^\mu} \eta \\
            - \sqrt{p_\mu \overline \sigma^\mu} \eta \\
        \end{bmatrix} ~,
    \end{equation*}
    where $\eta$ is a constant $2$-components spinor normalised such that $\eta^\dagger \eta = 1$. 

    They can be distinguished since 
    \begin{equation*}
        (\gamma^\mu p_\mu - m) u(\mathbf p) = 0 ~, \quad (\gamma^\mu p_\mu + m) v(\mathbf p) = 0 ~,
    \end{equation*}
    and 
    \begin{equation*}
        \hat H \psi(x) = i \pdv{}{t} (u (\mathbf p) \exp(- i px)) = E_{\mathbf p} \psi (x) ~, \quad \hat H \psi(x) = i \pdv{}{t} (u (\mathbf p) \exp(i px)) = - E_{\mathbf p} \psi (x) ~.
    \end{equation*}

    Consider a massive particle in the rest frame $p^\mu = (m, 0, 0,0)$. The positive frequency solutions look like 
    \begin{equation*}
        \psi(x) = \sqrt{m} \exp(- i E_{\mathbf p} t) \begin{bmatrix}
            \xi \\ \xi \\
        \end{bmatrix} ~.
    \end{equation*}
    Using~\eqref{lorspin}, we restrict to spatial rotations in which the generators are
    \begin{equation*}
        S^{ij} = - \frac{1}{2} \epsilon^{ijk} \begin{bmatrix}
            \sigma^k & 0 \\ 0 & \sigma^k \\
        \end{bmatrix} ~,
    \end{equation*}
    where $i \neq j$ and the parameters are 
    \begin{equation*}
        \omega_{ij} = - \epsilon_{ijk} \theta^k ~.
    \end{equation*}
    Therefore, the matrix rotation is 
    \begin{equation*}
        \exp(\frac{1}{2} \omega_{ij} S^{ij}) = \begin{bmatrix}
            \exp(\frac{i}{2} \theta^i \sigma_i) & 0 \\ 0 & \exp(\frac{i}{2} \theta^i \sigma_i) \\
        \end{bmatrix}
    \end{equation*}
    and the Dirac spinor transforms as 
    \begin{equation*}
        \psi' (x) = \begin{bmatrix}
            \exp(\frac{i}{2} \theta^i \sigma_i) & 0 \\ 0 & \exp(\frac{i}{2} \theta^i \sigma_i) \\
        \end{bmatrix} \psi (x) ~,
    \end{equation*}
    which induces a transformation on $\xi$ such that 
    \begin{equation*}
        \xi ' = \exp(\frac{i}{2} \theta^i \sigma_i) \xi ~.
    \end{equation*}
    This is indeed an $SU(2)$ transformation, where we can recognise tha spin operator $\hat{\mathbf S} = \frac{1}{2} \boldsymbol \sigma$ and $\xi$ is a $2$-components spinot which describes particle with spin $\frac{1}{2}$. Since $\xi^\dagger \xi = 1$, we choose, for the spin up
    \begin{equation*}
        \xi = \begin{bmatrix}
            1 \\ 0 \\
        \end{bmatrix} \colon \sigma_3 \xi = \begin{bmatrix}
            1 & 0 \\ 0 & -1 \\
        \end{bmatrix} \begin{bmatrix}
            1 \\ 0 \\
        \end{bmatrix} = (+1) \xi ~,
    \end{equation*}
    for the spin down
    \begin{equation*}
        \xi = \begin{bmatrix}
            0 \\ 1 \\
        \end{bmatrix} \colon \sigma_3 \xi = \begin{bmatrix}
            1 & 0 \\ 0 & -1 \\
        \end{bmatrix} \begin{bmatrix}
            0 \\ 1 \\
        \end{bmatrix} = (-1) \xi ~,
    \end{equation*}

    We introduce a basis of the $2$-components spinors 
    \begin{equation*}
        \xi^r ~, \eta^s ~,
    \end{equation*}
    where $r,s = 1,2$ such that they satisfy 
    \begin{equation*}
        (\xi^\dagger)^r \xi^s = \delta^{rs} ~, \quad (\eta^\dagger)^r \eta^s = \delta^{rs}  ~.
    \end{equation*}

    For example, 
    \begin{equation*}
        \xi^1 = \begin{bmatrix}
            1 \\ 0 \\
        \end{bmatrix} ~, \quad \xi^2 = \begin{bmatrix}
            0 \\ 1 \\
        \end{bmatrix} ~, \quad \eta^1 = \begin{bmatrix}
            1 \\ 0 \\
        \end{bmatrix} ~, \quad \eta^2 = \begin{bmatrix}
            0 \\ 1 \\
        \end{bmatrix} ~.
    \end{equation*}

    We define the following inner products 
    \begin{enumerate}
        \item \begin{equation*}
            (u^\dagger)^r (\mathbf p) u^s (\mathbf p) = 2 p^0 \delta^{rs} ~,
        \end{equation*}
        \item \begin{equation*}
            \overline u^r (\mathbf p) u^s (\mathbf p) = 2 m \delta^{rs} ~,
        \end{equation*} 
        \item \begin{equation*}
            (v^\dagger)^r (\mathbf p) v^s (\mathbf p) = 2 p^0 \delta^{rs} ~,
        \end{equation*}
        \item \begin{equation*}
            \overline v^r (\mathbf p) v^s (\mathbf p) = - 2 m \delta^{rs} ~,
        \end{equation*}
        \item \begin{equation*}
            \overline u^r (\mathbf p) v^s (\mathbf p) = \overline v^r (\mathbf p) u^s (\mathbf p) = 0 ~,
        \end{equation*}
        \item \begin{equation*}
            (u^\dagger)^r (\mathbf p) v^s (- \mathbf p) = (v^\dagger)^r (\mathbf p) u^s (- \mathbf p) = 0 ~.
        \end{equation*}
    \end{enumerate}
    \begin{proof}
        For the first one, 
        \begin{equation*}
        \begin{aligned}
            (u^\dagger)^r (\mathbf p) u^s (\mathbf p) & = \begin{bmatrix}
                \sqrt{p^\mu \sigma_\mu} (\xi^\dagger)^r & \sqrt{p^\mu \overline \sigma_\mu} (\xi^\dagger)^r \\
            \end{bmatrix} \begin{bmatrix}
                \sqrt{p^\mu \sigma_\mu} \xi^s \\ \sqrt{p^\mu \overline \sigma_\mu} \xi^s \\
            \end{bmatrix} \\ & = (\xi^\dagger)^r p^\mu \sigma_\mu \xi^s + (\xi^\dagger)^r p^\mu \overline \sigma_\mu \xi^s \\ & = (\xi^\dagger)^r p^0 \underbrace{\sigma_0}_{\mathbb I_2} \xi^s + (\xi^\dagger)^r p^0 \underbrace{\overline \sigma_0}_{\mathbb I_2} \xi^s + (\xi^\dagger)^r p^i \sigma_i \xi^s + (\xi^\dagger)^r p^i \underbrace{\overline \sigma_i}_{- \sigma_i} \xi^s \\ & = (\xi^\dagger)^r p^0 \xi^s + (\xi^\dagger)^r p^0 \xi^s + \cancel{(\xi^\dagger)^r p^i \sigma_i \xi^s } - \cancel{(\xi^\dagger)^r p^i \sigma_i \xi^s} \\ & = 2 p_0 \underbrace{(\xi^\dagger)^r \xi^s}_{\delta^{rs}} \\ & = 2 p_0 \delta^{rs} ~.
        \end{aligned}
        \end{equation*}

        For the second one, 
        \begin{equation*}
        \begin{aligned}
            \overline u^r (\mathbf p) u^s (\mathbf p) & = \begin{bmatrix}
                \sqrt{p^\mu \sigma_\mu} (\xi^\dagger)^r & \sqrt{p^\mu \overline \sigma_\mu} (\xi^\dagger)^r \\
            \end{bmatrix} \begin{bmatrix}
                0 & 1 \\ 1 & 0 \\
            \end{bmatrix} \begin{bmatrix}
                \sqrt{p^\mu \sigma_\mu} \xi^s \\ \sqrt{p^\mu \overline \sigma_\mu} \xi^s \\
            \end{bmatrix} \\ & = (\xi^\dagger)^r p^\mu \underbrace{\sqrt{p^\mu  \sigma_\mu} \sqrt{p^\nu \overline \sigma_\nu}}_m \xi^s + (\xi^\dagger)^r \underbrace{\sqrt{p^\mu \sigma_\mu} \sqrt{p^\nu \overline \sigma_\nu}}_m \xi^s \\ & = 2 m \underbrace{(\xi^\dagger)^r \xi^s}_{\delta^{rs}} \\ & 2 m = \delta^{rs} ~.
        \end{aligned}
        \end{equation*}

        For the third one, 
        \begin{equation*}
        \begin{aligned}
            (v^\dagger)^r (\mathbf p) v^s (\mathbf p) & = \begin{bmatrix}
                \sqrt{p^\mu \sigma_\mu} (\eta^\dagger)^r & - \sqrt{p^\mu \overline \sigma_\mu} (\eta^\dagger)^r \\
            \end{bmatrix} \begin{bmatrix}
                \sqrt{p^\mu \sigma_\mu} \eta^s \\ - \sqrt{p^\mu \overline \sigma_\mu} \eta^s \\
            \end{bmatrix} \\ & = (\eta^\dagger)^r p^\mu \sigma_\mu \eta^s + (\eta^\dagger)^r p^\mu \overline \sigma_\mu \eta^s \\ & = (\eta^\dagger)^r p^0 \underbrace{\sigma_0}_{\mathbb I_2} \eta^s + (\eta^\dagger)^r p^0 \underbrace{\overline \sigma_0}_{\mathbb I_2} \eta^s + (\eta^\dagger)^r p^i \sigma_i \eta^s + (\eta^\dagger)^r p^i \underbrace{\overline \sigma_i}_{- \sigma_i} \eta^s \\ & = (\eta^\dagger)^r p^0 \eta^s + (\eta^\dagger)^r p^0 \eta^s + \cancel{(\eta^\dagger)^r p^i \sigma_i \eta^s } - \cancel{(\eta^\dagger)^r p^i \sigma_i \eta^s} \\ & = 2 p_0 \underbrace{(\eta^\dagger)^r \eta^s}_{\delta^{rs}} \\ & = 2 p_0 \delta^{rs} ~.
        \end{aligned}
        \end{equation*}

        For the fourth one, 
        \begin{equation*}
        \begin{aligned}
            \overline v^r (\mathbf p) v^s (\mathbf p) & = \begin{bmatrix}
                \sqrt{p^\mu \sigma_\mu} (\eta^\dagger)^r & - \sqrt{p^\mu \overline \sigma_\mu} (\eta^\dagger)^r \\
            \end{bmatrix} \begin{bmatrix}
                0 & 1 \\ 1 & 0 \\
            \end{bmatrix} \begin{bmatrix}
                \sqrt{p^\mu \sigma_\mu} \eta^s \\ - \sqrt{p^\mu \overline \sigma_\mu} \eta^s \\
            \end{bmatrix} \\ & = - (\eta^\dagger)^r p^\mu \underbrace{\sqrt{p^\mu  \sigma_\mu} \sqrt{p^\nu \overline \sigma_\nu}}_m \eta^s - (\eta^\dagger)^r \underbrace{\sqrt{p^\mu \sigma_\mu} \sqrt{p^\nu \overline \sigma_\nu}}_m \eta^s \\ & = - 2 m \underbrace{(\eta^\dagger)^r \eta^s}_{\delta^{rs}} \\ & - 2 m = \delta^{rs} ~.
        \end{aligned}
        \end{equation*}

        For the fifth one, 
        \begin{equation*}
        \begin{aligned}
            \overline u^r (\mathbf p) v^s (\mathbf p) & = \begin{bmatrix}
                \sqrt{p^\mu \sigma_\mu} (\xi^\dagger)^r & \sqrt{p^\mu \overline \sigma_\mu} (\xi^\dagger)^r \\
            \end{bmatrix} \begin{bmatrix}
                0 & 1 \\ 1 & 0 \\
            \end{bmatrix} \begin{bmatrix}
                \sqrt{p^\mu \sigma_\mu} \eta^s \\ - \sqrt{p^\mu \overline \sigma_\mu} \eta^s \\
            \end{bmatrix} \\ & = - (\xi^\dagger)^r \underbrace{\sqrt{p^\mu  \sigma_\mu} \sqrt{p^\nu \overline \sigma_\nu}}_m \eta^s + (\xi^\dagger)^r \underbrace{\sqrt{p^\mu \sigma_\mu} \sqrt{p^\nu \overline \sigma_\nu}}_m \eta^s \\ & = m ( \cancel{(- \xi^\dagger)^r \eta^s} + \cancel{(\xi^\dagger)^r \eta^2}) = 0 ~.
        \end{aligned}
        \end{equation*}

        For the second in the fifth one, 
        \begin{equation*}
        \begin{aligned}
            \overline v^r (\mathbf p) u^s (\mathbf p) & = \begin{bmatrix}
                \sqrt{p^\mu \sigma_\mu} (\eta^\dagger)^r & \sqrt{p^\mu \overline \sigma_\mu} (\eta^\dagger)^r \\
            \end{bmatrix} \begin{bmatrix}
                0 & 1 \\ 1 & 0 \\
            \end{bmatrix} \begin{bmatrix}
                \sqrt{p^\mu \sigma_\mu} \xi^s \\ - \sqrt{p^\mu \overline \sigma_\mu} \xi^s \\
            \end{bmatrix} \\ & = - (\eta^\dagger)^r \underbrace{\sqrt{p^\mu  \sigma_\mu} \sqrt{p^\nu \overline \sigma_\nu}}_m \xi^s + (\eta^\dagger)^r \underbrace{\sqrt{p^\mu \sigma_\mu} \sqrt{p^\nu \overline \sigma_\nu}}_m \xi^s \\ & = m ( \cancel{(- \eta^\dagger)^r \xi^s} + \cancel{(\eta^\dagger)^r \xi^2}) = 0 ~.
        \end{aligned}
        \end{equation*}

        For the sixth one, 
        \begin{equation*}
        \begin{aligned}
            (u^\dagger)^r (\mathbf p) v^s (- \mathbf p) & = \begin{bmatrix}
                \sqrt{p^\mu \sigma_\mu} (\xi^\dagger)^r & \sqrt{p^\mu \overline \sigma_\mu} (\xi^\dagger)^r \\
            \end{bmatrix} \begin{bmatrix}
                \sqrt{\overline p^\mu \sigma_\mu} \eta^s \\ - \sqrt{\overline p^\mu \overline \sigma_\mu} \eta^s \\
            \end{bmatrix} \\ & = (\xi^\dagger)^r \underbrace{\sqrt{p^\mu \sigma_\mu} \sqrt{\overline p^\mu \sigma_\mu}}_m \eta^s - (\xi^\dagger)^r \underbrace{\sqrt{p^\mu \overline \sigma_\mu} \sqrt{\overline p^\mu \overline \sigma_\mu}}_m \eta^s \\ & = m ((\xi^\dagger)^r \eta^s - (\xi^\dagger)^r \eta^s) = 0 ~.
        \end{aligned}
        \end{equation*}

        For the second in the sixth one, 
        \begin{equation*}
        \begin{aligned}
            (v^\dagger)^r (\mathbf p) u^s (- \mathbf p) & = \begin{bmatrix}
                \sqrt{p^\mu \sigma_\mu} (\eta^\dagger)^r & - \sqrt{p^\mu \overline \sigma_\mu} (\eta^\dagger)^r \\
            \end{bmatrix} \begin{bmatrix}
                \sqrt{\overline p^\mu \sigma_\mu} \xi^s \\ \sqrt{\overline p^\mu \overline \sigma_\mu} \xi^s \\
            \end{bmatrix} \\ & = (\eta^\dagger)^r \underbrace{\sqrt{p^\mu \sigma_\mu} \sqrt{\overline p^\mu \sigma_\mu}}_m \xi^s - (\eta^\dagger)^r \underbrace{\sqrt{p^\mu \overline \sigma_\mu} \sqrt{\overline p^\mu \overline \sigma_\mu}}_m \xi^s \\ & = m ((\eta^\dagger)^r \xi^s - (\eta^\dagger)^r \xi^s) = 0 ~.
        \end{aligned}
        \end{equation*}
    \end{proof}
    
    We define the following outer products 
    \begin{enumerate}
        \item \begin{equation*}
            \sum_{s=1}^{2} u^s (\mathbf p) \overline u^s(\mathbf p) = \gamma^\mu p_\mu + m \mathbb I_4 ~,
        \end{equation*} 
        \item \begin{equation*}
            \sum_{s=1}^{2} v^s (\mathbf p) \overline v^s(\mathbf p) = \gamma^\mu p_\mu - m \mathbb I_4 ~.
        \end{equation*} 
    \end{enumerate}
    \begin{proof}
        For the first one
        \begin{equation*}
        \begin{aligned}
            \sum_{s=1}^{2} u^s (\mathbf p) \overline u^s(\mathbf p) & = \sum_{s=1}^{2} \begin{bmatrix}
                \sqrt{ p^\mu \sigma_\mu} \xi^s \\ \sqrt{ p^\mu \overline \sigma_\mu} \xi^s \\
            \end{bmatrix} \begin{bmatrix}
                \sqrt{p^\mu \sigma_\mu} (\xi^\dagger)^r & \sqrt{p^\mu \overline \sigma_\mu} (\xi^\dagger)^r \\
            \end{bmatrix} \begin{bmatrix}
                0 & 1 \\ 1 & 0 \\
            \end{bmatrix} \\ & = \sum_{s=1}^{2} \begin{bmatrix}
                \sqrt{p^\mu \sigma_\mu} \xi^s (\xi^\dagger)^s \sqrt{p^\mu \overline \sigma_\mu} & \sqrt{p^\mu \sigma_\mu} \xi^s (\xi^\dagger)^s \sqrt{p^\mu \sigma_\mu} \\ \sqrt{p^\mu \overline \sigma_\mu} \xi^s (\xi^\dagger)^s \sqrt{p^\mu \overline \sigma_\mu} & \sqrt{p^\mu \overline \sigma_\mu} \xi^s (\xi^\dagger)^s \sqrt{p^\mu \sigma_\mu} 
            \end{bmatrix} \\ & = \begin{bmatrix}
                \underbrace{\sqrt{p^\mu \sigma_\mu} \sqrt{p^\mu \overline \sigma_\mu}}_m & \sqrt{p^\mu \sigma_\mu} \sqrt{p^\mu \sigma_\mu} \\ \sqrt{p^\mu \overline \sigma_\mu} \sqrt{p^\mu \overline \sigma_\mu} & \underbrace{\sqrt{p^\mu \overline \sigma_\mu} \sqrt{p^\mu \sigma_\mu}}_m
            \end{bmatrix} \\ & = \begin{bmatrix}
                m & p^\mu \sigma_\mu \\ p^\mu \overline \sigma_\mu & m \\
            \end{bmatrix} \\ & = \gamma^\mu p_\mu + m ~,
        \end{aligned}
        \end{equation*}
        where we have used 
        \begin{equation*}
            \sum_{s=1}^{2} \xi^s (\xi^\dagger)^s = \mathbb I_2 ~.
        \end{equation*}

        For the first one
        \begin{equation*}
        \begin{aligned}
            \sum_{s=1}^{2} v^s (\mathbf p) \overline v^s(\mathbf p) & = \sum_{s=1}^{2} \begin{bmatrix}
                \sqrt{ p^\mu \sigma_\mu} \eta^s \\ - \sqrt{ p^\mu \overline \sigma_\mu} \eta^s \\
            \end{bmatrix} \begin{bmatrix}
                \sqrt{p^\mu \sigma_\mu} (\eta^\dagger)^r & - \sqrt{p^\mu \overline \sigma_\mu} (\eta^\dagger)^r \\
            \end{bmatrix} \begin{bmatrix}
                0 & 1 \\ 1 & 0 \\
            \end{bmatrix} \\ & = \sum_{s=1}^{2} \begin{bmatrix}
                - \sqrt{p^\mu \sigma_\mu} \eta^s (\eta^\dagger)^s \sqrt{p^\mu \overline \sigma_\mu} & \sqrt{p^\mu \sigma_\mu} \eta^s (\eta^\dagger)^s \sqrt{p^\mu \sigma_\mu} \\ \sqrt{p^\mu \overline \sigma_\mu} \eta^s (\eta^\dagger)^s \sqrt{p^\mu \overline \sigma_\mu} & - \sqrt{p^\mu \overline \sigma_\mu} \eta^s (\eta^\dagger)^s \sqrt{p^\mu \sigma_\mu} 
            \end{bmatrix} \\ & = \begin{bmatrix}
                - \underbrace{\sqrt{p^\mu \sigma_\mu} \sqrt{p^\mu \overline \sigma_\mu}}_m & \sqrt{p^\mu \sigma_\mu} \sqrt{p^\mu \sigma_\mu} \\ \sqrt{p^\mu \overline \sigma_\mu} \sqrt{p^\mu \overline \sigma_\mu} & -\underbrace{\sqrt{p^\mu \overline \sigma_\mu} \sqrt{p^\mu \sigma_\mu}}_m 
            \end{bmatrix} \\ & = \begin{bmatrix}
                - m & p^\mu \sigma_\mu \\ p^\mu \overline \sigma_\mu & - m \\
            \end{bmatrix} \\ & = \gamma^\mu p_\mu - m ~,
        \end{aligned}
        \end{equation*}
        where we have used 
        \begin{equation*}
            \sum_{s=1}^{2} \eta^s (\eta^\dagger)^s = \mathbb I_2 ~.
        \end{equation*}
    \end{proof}

\chapter{Quantisation}

\section{How to not quantise the Dirac theory}

    We quantise the theory with the same procedure we used for the scalar field. The conjugate field is 
    \begin{equation*}
        \pi = i \psi^\dagger ~.
    \end{equation*}

    \begin{proof}
        In fact, 
        \begin{equation*}
            \pi = \pdv{\mathcal L}{\dot \psi} = \underbrace{\overline \psi}_{\psi^\dagger \gamma_0} i \gamma_0 = \psi^\dagger \gamma_0 i \gamma_0 = \psi^\dagger i \underbrace{(\gamma_0)^2}_{1} = i \psi^\dagger ~.
        \end{equation*}
    \end{proof}

    In the Schroedinger picture, we impose the canonical commutation relations
    \begin{equation*}
        [\hat \psi_\alpha (\mathbf x), \hat \psi_\beta (\mathbf y)] = [\hat \pi_\alpha (\mathbf x), \hat \pi_\beta (\mathbf y)] = 0 ~,
    \end{equation*}
    \begin{equation*}
        [\hat \psi_\alpha (\mathbf x), \hat \pi_\beta (\mathbf y)] = [\hat \psi_\alpha (\mathbf x), \hat \psi^\dagger_\beta (\mathbf y)] = i \delta^3 (\mathbf x - \mathbf y) \delta_{\alpha\beta} ~.
    \end{equation*}

    Since the general solution of the Dirac equation can be written as a linear combination of positive and negative frequency solutions, the general form of the field operator is 
    \begin{equation*}
        \hat \psi(\mathbf x) = \sum_{s=1}^{2} \int \frac{d^3 p}{(2\pi)^3} \frac{1}{\sqrt{2 E_{\mathbf p}}} \Big ( \hat b^s_{\mathbf p} u^s (\mathbf p) \exp(i \mathbf p \cdot \mathbf x) + (\hat c_{\mathbf p}^\dagger)^s v^s(\mathbf p) \exp(- i \mathbf p \cdot \mathbf x) \Big) ~,
    \end{equation*}
    \begin{equation*}
        \hat \psi^\dagger (\mathbf x) = \sum_{s=1}^{2} \int \frac{d^3 p}{(2\pi)^3} \frac{1}{\sqrt{2 E_{\mathbf p}}} \Big ( (\hat b^\dagger_{\mathbf p})^s (u^\dagger)^s (\mathbf p) \exp(- i \mathbf p \cdot \mathbf x) + \hat c^s_{\mathbf p} (v^\dagger)^s(\mathbf p) \exp(i \mathbf p \cdot \mathbf x) \Big) ~,
    \end{equation*}
    Therefore, the commutation relation among the ladder operators are 
    \begin{equation*}
        [\hat b^s_{\mathbf p}, (\hat b^\dagger_{\mathbf q})] = (2\pi)^3 \delta^{rs} \delta^3 (\mathbf p - \mathbf q) ~, \quad [\hat c^s_{\mathbf p}, (\hat c^\dagger_{\mathbf q})] = - (2\pi)^3 \delta^{rs} \delta^3 (\mathbf p - \mathbf q) ~, 
    \end{equation*}
    and all the others vanish. Notice that there is a minus sign in the commutator of $\hat c$.
    \begin{proof}
        In fact 
        \begin{equation*}
        \begin{aligned}
            [\hat \psi(\mathbf x), \hat \psi^\dagger (\mathbf y)] & = [\sum_{s=1}^{2} \int \frac{d^3 p}{(2\pi)^3} \frac{1}{\sqrt{2 E_{\mathbf p}}} \Big ( \hat b^s_{\mathbf p} u^s (\mathbf p) \exp(i \mathbf p \cdot \mathbf x) + (\hat c_{\mathbf p}^\dagger)^s v^s(\mathbf p) \exp(- i \mathbf p \cdot \mathbf x) \Big), \\ & \qquad \sum_{r=1}^{2} \int \frac{d^3 q}{(2\pi)^3} \frac{1}{\sqrt{2 E_{\mathbf q}}} \Big ( (\hat b^\dagger_{\mathbf q})^r (u^\dagger)^r (\mathbf q) \exp(- i \mathbf q \cdot \mathbf y) + \hat c^r_{\mathbf q} (v^\dagger)^r(\mathbf q) \exp(i \mathbf q \cdot \mathbf y) \Big) ] \\ & = \sum_{r,s=1}^2 \int \frac{d^3 p ~ d^3 q}{(2 \pi)^6} \frac{1}{2 \sqrt{E_{\mathbf p} E_{\mathbf q}}} \Big ( \underbrace{[\hat b^s_{\mathbf p}, (\hat b^\dagger_{\mathbf q})^r]}_{(2\pi)^3 \delta^{rs} \delta^3 (\mathbf p - \mathbf q)} u^s (\mathbf p) (u^\dagger)^r (\mathbf q) \exp (i (\mathbf p \cdot \mathbf x - \mathbf q \cdot \mathbf y)) \\ & \qquad + \underbrace{[\hat b^s_{\mathbf p}, \hat c_{\mathbf q}^r]}_0 u^s (\mathbf p) (v^\dagger)^r (\mathbf q) \exp (i (\mathbf p \cdot \mathbf x + \mathbf q \cdot \mathbf y)) \\ & \qquad + \underbrace{[(\hat c^\dagger_{\mathbf p})^s, (\hat b^\dagger_{\mathbf q})^r]}_0 v^s (\mathbf p) (u^\dagger)^r (\mathbf q) \exp (i (- \mathbf p \cdot \mathbf x - \mathbf q \cdot \mathbf y)) \\ & \qquad + \underbrace{[(\hat c^\dagger_{\mathbf p})^s, \hat c^r_{\mathbf q}]}_{(2\pi)^3 \delta^{rs} \delta^3 (\mathbf p - \mathbf q)} v^s (\mathbf p) (v^\dagger)^r (\mathbf q) \exp (i (- \mathbf p \cdot \mathbf x + \mathbf q \cdot \mathbf y)) \Big) \\ & = \sum_{s=1}^2 \int \frac{d^3 p}{(2 \pi)^3} \frac{1}{2 E_{\mathbf p}} \Big ( \underbrace{u^s (\mathbf p) \overline u^s (\mathbf p)}_{\gamma^\mu p_\mu + m} \gamma^0 \exp (i \mathbf p \cdot (\mathbf x - \mathbf y)) + \underbrace{v^s (\mathbf p) \overline v^s (\mathbf p)}_{\gamma^\mu p_\mu - m} \gamma^0 \exp (- i \mathbf p  \cdot (\mathbf x - \mathbf y)) \Big) 
        \end{aligned}
        \end{equation*}
        \begin{equation*}
        \begin{aligned}
            \phantom{[\hat \psi(\mathbf x), \hat \psi^\dagger (\mathbf y)]} & = \int \frac{d^3 p}{(2 \pi)^3} \frac{1}{2 E_{\mathbf p}} \Big ( (\gamma^\mu p_\mu + m) \gamma^0 \exp (i \mathbf p \cdot (\mathbf x - \mathbf y)) + (\gamma^\mu p_\mu - m) \gamma^0 \exp (- i \mathbf p  \cdot (\mathbf x - \mathbf y)) \Big)  \\ & = \int \frac{d^3 p}{(2 \pi)^3} \frac{1}{2 E_{\mathbf p}} \Big ( (\gamma^\mu p_\mu + m) \gamma^0 \exp (i \mathbf p \cdot (\mathbf x - \mathbf y)) + (-\gamma^\mu p_\mu - m) \gamma^0 \exp (i \mathbf p  \cdot (\mathbf x - \mathbf y)) \Big) \\ & = \int \frac{d^3 p}{(2 \pi)^3} \frac{1}{2 E_{\mathbf p}} \Big ( (\gamma^0 p_0 + \cancel{\gamma^i p_i} + \cancel{m}) \gamma^0 \exp (i \mathbf p \cdot (\mathbf x - \mathbf y)) + (\gamma^0 p_0 - \cancel{\gamma^i p_i} - \cancel{m}) \gamma^0 \exp (i \mathbf p  \cdot (\mathbf x - \mathbf y)) \Big) \\ & = \int \frac{d^3 p}{(2 \pi)^3} \frac{1}{\cancel{2 E_{\mathbf p}}} \Big (\cancel{2 p_0} \underbrace{\gamma^0 \gamma^0}_1 \exp (i \mathbf p \cdot (\mathbf x - \mathbf y)) \Big) \\ & = \int \frac{d^3 p}{(2 \pi)^3} \exp (i \mathbf p \cdot (\mathbf x - \mathbf y)) = \delta^3 (\mathbf x - \mathbf y) ~.
        \end{aligned}
        \end{equation*} 
    \end{proof}

    The hamiltonian is 
    \begin{equation*}
        \hat H = \sum_{s=1}^{2} \int \frac{d^3 p}{(2\pi)^3} E_{\mathbf p} ((\hat b^\dagger_{\mathbf p})^s \hat b^s_{\mathbf p} - (\hat c^\dagger_{\mathbf p})^s \hat c^s_{\mathbf p}) ~.
    \end{equation*}
    \begin{proof}
        The hamiltonian density is 
        \begin{equation*}
            \mathcal H = \pi \dot \psi - L = i \psi^\dagger \psi - \overline \psi i \gamma^\mu \partial_\mu \psi + m \overline \psi \psi = \cancel{i \psi^\dagger \psi} - \cancel{\psi^\dagger \gamma^0 i \gamma^0 \partial_0 \psi} - i \gamma^i \partial_i \overline \psi \psi + m \overline \psi \psi = \overline \psi (- i \gamma^i \partial_i + m) \psi ~.
        \end{equation*}

        We compute 
        \begin{equation*}
        \begin{aligned}
            (- i \gamma^i \partial_i + m) \hat \psi (\mathbf x) & = \sum_{s=1}^{2} \int \frac{d^3 p}{(2\pi)^3} \frac{1}{\sqrt{2 E_{\mathbf p}}} (\hat b^s_{\mathbf p} (- \gamma^i \underbrace{i \partial_i}_{p_i} + m) u^s (\mathbf p) \exp(i \mathbf p \cdot x) \\ & \qquad + (\hat c^\dagger_{\mathbf p})^s (- \gamma^i \underbrace{i \partial_i}_{- p_i} + m) v^s (\mathbf p) \exp(-i \mathbf p \cdot x) ) \\ & = \sum_{s=1}^{2} \int \frac{d^3 p}{(2\pi)^3} \frac{1}{\sqrt{2 E_{\mathbf p}}} (\hat b^s_{\mathbf p} \underbrace{(- \gamma^i p_i + m) u^s (\mathbf p)}_{\gamma^0 p_0 u^s (\mathbf p)} \exp(i \mathbf p \cdot x) \\ & \qquad + (\hat c^\dagger_{\mathbf p})^s \underbrace{(\gamma^i p_i + m) v^s (\mathbf p)}_{- \gamma^0 p_0 v^s (\mathbf p)} \exp(-i \mathbf p \cdot x) ) \\ & = \sum_{s=1}^{2} \int \frac{d^3 p}{(2\pi)^3} \sqrt{\frac{E_{\mathbf p}}{2}} \gamma^0 (\hat b^s_{\mathbf p} u^s (\mathbf p) \exp(i \mathbf p \cdot x) - (\hat c^\dagger_{\mathbf p})^s v^s (\mathbf p) \exp(-i \mathbf p \cdot x) ) ~.
        \end{aligned}
        \end{equation*}

        Therefore, 
        \begin{equation*}
        \begin{aligned}
            \hat H & = \int d^3 x ~ \mathcal H \\ & = \sum_{r, s=1}^{2} \int \frac{d^3 x ~ d^3 p ~ d^3 q}{(2\pi)^6} \frac{1}{2} \sqrt{\frac{E_{\mathbf p}}{E_{\mathbf q}}} \Big ( (\hat b^\dagger_{\mathbf q})^s (u^\dagger)^s (\mathbf q) \exp(- i \mathbf q \cdot \mathbf x) + \hat c^s_{\mathbf q} (v^\dagger)^s (\mathbf q) \exp(i \mathbf q \cdot \mathbf x) \Big) \\ & \qquad \underbrace{\gamma^0 \gamma^0}_1 (\hat b^s_{\mathbf p} u^s (\mathbf p) \exp(i \mathbf p \cdot x) - (\hat c^\dagger_{\mathbf p})^s v^s (\mathbf p) \exp(-i \mathbf p \cdot x) ) \\ & = \sum_{r, s=1}^{2} \int \frac{d^3 x ~ d^3 p ~ d^3 q}{(2\pi)^6} \frac{1}{2} \sqrt{\frac{E_{\mathbf p}}{E_{\mathbf q}}} \Big ( (\hat b^\dagger_{\mathbf q})^s (\mathbf q) \hat b^s_{\mathbf p} (u^\dagger)^s u^s (\mathbf p) \underbrace{\exp(- i (\mathbf q - \mathbf p )\cdot \mathbf x)}_{(2\pi)^3 \delta^3 (\mathbf p - \mathbf q)} \\ & \qquad - (\hat b^\dagger_{\mathbf q})^s (\hat c^\dagger_{\mathbf p})^s (u^\dagger)^s (\mathbf q) v^s (\mathbf p) \underbrace{\exp(- i (\mathbf q + \mathbf p) \cdot \mathbf x) }_{(2\pi)^3 \delta^3 (\mathbf p + \mathbf q)} \\ & \qquad + \hat c^s_{\mathbf q} \hat b^s_{\mathbf p} (v^\dagger)^s (\mathbf q) u^s (\mathbf p) \underbrace{\exp(i (\mathbf p + \mathbf q) \cdot x)}_{(2\pi)^3 \delta^3 (\mathbf p + \mathbf q)}  \\ & \qquad - \hat c^s_{\mathbf q} (\hat c^\dagger_{\mathbf p})^s (v^\dagger)^s (\mathbf q) v^s (\mathbf p)\underbrace{ \exp(- i (\mathbf p - \mathbf q) \cdot x) }_{(2\pi)^3 \delta^3 (\mathbf p - \mathbf q)} \Big )
        \end{aligned}
        \end{equation*}
        \begin{equation*}
        \begin{aligned}
            \phantom{\hat H} & = \sum_{r, s=1}^{2} \int \frac{d^3 p}{(2\pi)^3} \frac{1}{2} \Big ( (\hat b^\dagger_{\mathbf p})^s (\mathbf p) \hat b^s_{\mathbf p} \underbrace{(u^\dagger)^s u^s (\mathbf p) }_{2 p_0 \delta^{rs}} \\ & \qquad - (\hat b^\dagger_{- \mathbf p})^s (\hat c^\dagger_{\mathbf p})^s \underbrace{(u^\dagger)^s (- \mathbf p) v^s (\mathbf p) }_0 \\ & \qquad + \hat c^s_{- \mathbf p} \hat b^s_{\mathbf p} \underbrace{(v^\dagger)^s (- \mathbf p) u^s (\mathbf p)}_0 \\ & \qquad - \hat c^s_{\mathbf p} (\hat c^\dagger_{\mathbf p})^s \underbrace{(v^\dagger)^s (\mathbf p) v^s (\mathbf p)}_{2 p_0 \delta^{rs}} \Big ) \\ & = \sum_{s = 1}^{2} \int \frac{d^3 p}{(2\pi)^3} E_{\mathbf p} ((\hat b^\dagger_{\mathbf p})^s \hat b^s_{\mathbf p} - \underbrace{\hat c^s_{\mathbf p} (\hat c^\dagger_{\mathbf p})^s}_{[\hat c^s_{\mathbf p}, (\hat c^\dagger_{\mathbf p})^s] + (\hat c^\dagger_{\mathbf p})^s \hat c^s_{\mathbf p}} ) \\ & = \sum_{s=1}^{2} \int \frac{d^3 p}{(2\pi)^3} E_{\mathbf p} ((\hat b^\dagger_{\mathbf p})^s \hat b^s_{\mathbf p} - (\hat c^\dagger_{\mathbf p})^s \hat c^s_{\mathbf p} + (2\pi)^3 \delta^3 (0)) \\ & = \sum_{s=1}^{2} \int \frac{d^3 p}{(2\pi)^3} E_{\mathbf p} ((\hat b^\dagger_{\mathbf p})^s \hat b^s_{\mathbf p} - (\hat c^\dagger_{\mathbf p})^s \hat c^s_{\mathbf p}) ~,
        \end{aligned}
        \end{equation*}
        where the last term is canceled due to normal ordering.
    \end{proof}

    Furthermore, they satisfy the following relations 
    \begin{equation*}
        [\hat H, (\hat b^\dagger_{\mathbf p})^s] = E_{\mathbf p} (\hat b^\dagger_{\mathbf p})^s ~,
    \end{equation*}
    \begin{equation*}
        [\hat H, (\hat c^\dagger_{\mathbf p})^s] = E_{\mathbf p} (\hat c^\dagger_{\mathbf p})^s ~.
    \end{equation*}

    Interpeting $(\hat c^\dagger_{\mathbf p})^s$ as a creation operator and $\hat c^s_{\mathbf p}$ as an annihilation operator, we find positive energies but negative norms. 
    \begin{proof}
        Computing 
        \begin{equation*}
            \hat H ((\hat c^\dagger_{\mathbf p})^s \ket{0}) = [\hat H, (\hat c^\dagger_{\mathbf p})^s] \ket{0} + (\hat c^\dagger_{\mathbf p})^s \underbrace{\hat H \ket{0}}_0 = E_\mathbf p ((\hat c^\dagger_{\mathbf p})^s \ket{0}) ~,
        \end{equation*}
        which tells us that $c$-particles have positive energies.

        Computing 
        \begin{equation*}
            \hat H ((\hat b^\dagger_{\mathbf p})^s \ket{0}) = [\hat H, (\hat b^\dagger_{\mathbf p})^s] \ket{0} + (\hat b^\dagger_{\mathbf p})^s \underbrace{\hat H \ket{0}}_0 = E_\mathbf p ((\hat b^\dagger_{\mathbf p})^s \ket{0}) ~,
        \end{equation*}
        which tells us that $b$-particles have positive energies.

        However, the norm is negative. In fact, since
        \begin{equation*}
            \braket{\mathbf q, s}{\mathbf p, s} = \bra{0} \hat c^s_{\mathbf p} (\hat c^\dagger_{\mathbf p})^s \ket{0} = \bra{0} \underbrace{[\hat c^s_{\mathbf p} , (\hat c^\dagger_{\mathbf p})^s]}_{- (2\pi)^3 \delta^3 (\mathbf p - \mathbf q)} \ket{0} + \cancel{\bra{0} (\hat c^\dagger_{\mathbf p})^s \hat c^s_{\mathbf p} \ket{0}} =- (2\pi)^3 \delta^3 (\mathbf p - \mathbf q) ~,
        \end{equation*}
        in momentum space 
        \begin{equation*}
            \ket{\psi, s} = \int \frac{d^3 p}{(2\pi)^3} \psi (\mathbf p) \ket{\mathbf p, s} ~,
        \end{equation*}
        we have 
        \begin{equation*}
            \braket{\psi, s}{\psi, s} = ||\ket{\psi, s}||^2 = \int \frac{d^3 p ~ d^3 q}{(2\pi)^6} \psi^* \mathbf(q) \psi \mathbf(p) \braket{\mathbf q, s}{\mathbf p, s} = - \int \frac{d^3 p}{(2\pi)^3} \psi^* \mathbf(q) \psi \mathbf(p) < 0 ~.
        \end{equation*}
    \end{proof}

    Interpeting $(\hat c^\dagger_{\mathbf p})^s$ as a annihilation operator and $\hat c^s_{\mathbf p}$ as an creation operator, we find positive norms but negative energies. 
    \begin{proof}
        Computing 
        \begin{equation*}
            \hat H (\hat c^s_{\mathbf p} \ket{0}) = [\hat H, \hat c^s_{\mathbf p}] \ket{0} + \hat c^s_{\mathbf p} \underbrace{\hat H \ket{0}}_0 = - E_\mathbf p (\hat c^s_{\mathbf p})^s \ket{0} ~,
        \end{equation*}
        which tells us that $c$-particles have negative energies.

        However, the norm is negative. In fact, since
        \begin{equation*}
            \braket{\mathbf q, s}{\mathbf p, s} = \bra{0} (\hat c^\dagger_{\mathbf p})^s \hat c^s_{\mathbf p} \ket{0} = \bra{0} \underbrace{[(\hat c^\dagger_{\mathbf p} , \hat c^s_{\mathbf p})^s]}_{(2\pi)^3 \delta^3 (\mathbf p - \mathbf q)} \ket{0} + \cancel{\bra{0} \hat c^s_{\mathbf p} (\hat c^\dagger_{\mathbf p})^s \ket{0}} = (2\pi)^3 \delta^3 (\mathbf p - \mathbf q) ~,
        \end{equation*}
        in momentum space 
        \begin{equation*}
            \ket{\psi, s} = \int \frac{d^3 p}{(2\pi)^3} \psi (\mathbf p) \ket{\mathbf p, s} ~,
        \end{equation*}
        we have 
        \begin{equation*}
            \braket{\psi, s}{\psi, s} = ||\ket{\psi, s}||^2 = \int \frac{d^3 p ~ d^3 q}{(2\pi)^6} \psi^* \mathbf(q) \psi \mathbf(p) \braket{\mathbf q, s}{\mathbf p, s} = \int \frac{d^3 p}{(2\pi)^3} \psi^* \mathbf(q) \psi \mathbf(p) > 0 ~.
        \end{equation*}
    \end{proof}

    To summarise, if we try to quantise Dirac theory with commutation relations, we get into troubles since $b$-particles have both positive norms and energies, but $c$-particles have either positive norm/negative energies or negative norms/positive energies. 

\section{How to rightly quantise Dirac theory}

    Since bosons follow the Bose-Eistein statistics, in which 
    \begin{equation*}
        \ket{\mathbf p, \mathbf q} = \hat a^\dagger_{\mathbf p} \hat a^\dagger_{\mathbf q} \ket{0} = \hat a^\dagger_{\mathbf q} \hat a^\dagger_{\mathbf p} \ket{0} = \ket{\mathbf q, \mathbf p} ~.
    \end{equation*}
    fermions experimentally follows Fermi-Dirac statistics. Thus, we try imposing anticommutation relations 
    \begin{equation*}
        \{\hat \psi_\alpha (\mathbf x), \hat \psi_\beta (\mathbf y)\} = \{\hat \pi_\alpha (\mathbf x), \hat \pi_\beta (\mathbf y)\} = 0 ~,
    \end{equation*}
    \begin{equation*}
        \{\hat \psi_\alpha (\mathbf x), \hat \psi^\dagger_\beta (\mathbf y)\} = \{\hat \psi_\alpha (\mathbf x), \hat \psi^\dagger_\beta (\mathbf y)\} = \delta^3 (\mathbf x - \mathbf y) \delta_{\alpha\beta} ~.
    \end{equation*}
    Therefore, the commutation relation among the ladder operators are 
    \begin{equation*}
        \{\hat b^s_{\mathbf p}, (\hat b^\dagger_{\mathbf q})\} = (2\pi)^3 \delta^{rs} \delta^3 (\mathbf p - \mathbf q) ~, \quad \{\hat c^s_{\mathbf p}, (\hat c^\dagger_{\mathbf q})\} = (2\pi)^3 \delta^{rs} \delta^3 (\mathbf p - \mathbf q) ~, 
    \end{equation*}
    and all the others vanish. Notice that now there is a plus sign in the commutator of $\hat c$.

    Similarly for the computation we have made in the previous chapter, the hamiltonian is 
    \begin{equation*}
        \hat H = \sum_{s=1}^{2} \int \frac{d^3 p}{(2\pi)^3} E_{\mathbf p} ((\hat b^\dagger_{\mathbf p})^s \hat b^s_{\mathbf p} + (\hat c^\dagger_{\mathbf p})^s \hat c^s_{\mathbf p}) ~.
    \end{equation*}
    \begin{proof}
        In fact, 
        \begin{equation*}
        \begin{aligned}
            \hat H & = \sum_{s = 1}^{2} \int \frac{d^3 p}{(2\pi)^3} E_{\mathbf p} ((\hat b^\dagger_{\mathbf p})^s \hat b^s_{\mathbf p} - \underbrace{\hat c^s_{\mathbf p} (\hat c^\dagger_{\mathbf p})^s}_{\{\hat c^s_{\mathbf p}, (\hat c^\dagger_{\mathbf p})^s\} - (\hat c^\dagger_{\mathbf p})^s \hat c^s_{\mathbf p}} ) \\ & = \sum_{s=1}^{2} \int \frac{d^3 p}{(2\pi)^3} E_{\mathbf p} ((\hat b^\dagger_{\mathbf p})^s \hat b^s_{\mathbf p} + (\hat c^\dagger_{\mathbf p})^s \hat c^s_{\mathbf p} + (2\pi)^3 \delta^3 (0)) \\ & = \sum_{s=1}^{2} \int \frac{d^3 p}{(2\pi)^3} E_{\mathbf p} ((\hat b^\dagger_{\mathbf p})^s \hat b^s_{\mathbf p} + (\hat c^\dagger_{\mathbf p})^s \hat c^s_{\mathbf p}) ~,
        \end{aligned}
        \end{equation*}
        where the last term is canceled due to normal ordering.
    \end{proof}

    Furthermore, they satisfy the following relations 
    \begin{equation*}
        \{\hat H, (\hat b^\dagger_{\mathbf p})^s\} = E_{\mathbf p} (\hat b^\dagger_{\mathbf p})^s ~,
    \end{equation*}
    \begin{equation*}
        \{\hat H, (\hat c^\dagger_{\mathbf p})^s\} = E_{\mathbf p} (\hat c^\dagger_{\mathbf p})^s ~.
    \end{equation*}

    Now, we have both positive norms and positive energies. 
    \begin{proof}
        Computing 
        \begin{equation*}
            \hat H ((\hat c^\dagger_{\mathbf p})^s \ket{0}) = \{\hat H, \hat c^s_{\mathbf p}\} \ket{0} + \hat c^s_{\mathbf p} \underbrace{\hat H \ket{0}}_0 = E_\mathbf p (\hat c^s_{\mathbf p})^s \ket{0} ~,
        \end{equation*}
        which tells us that $c$-particles have positive energies.

        Since
        \begin{equation*}
            \braket{\mathbf q, s}{\mathbf p, s} = \bra{0} (\hat c^\dagger_{\mathbf p})^s \hat c^s_{\mathbf p} \ket{0} = \bra{0} \underbrace{\{(\hat c^\dagger_{\mathbf p} , \hat c^s_{\mathbf p})^s\}}_{(2\pi)^3 \delta^3 (\mathbf p - \mathbf q)} \ket{0} + \cancel{\bra{0} \hat c^s_{\mathbf p} (\hat c^\dagger_{\mathbf p})^s \ket{0}} = (2\pi)^3 \delta^3 (\mathbf p - \mathbf q) ~,
        \end{equation*}
        in momentum space 
        \begin{equation*}
            \ket{\psi, s} = \int \frac{d^3 p}{(2\pi)^3} \psi (\mathbf p) \ket{\mathbf p, s} ~,
        \end{equation*}
        we have 
        \begin{equation*}
            \braket{\psi, s}{\psi, s} = ||\ket{\psi, s}||^2 = \int \frac{d^3 p ~ d^3 q}{(2\pi)^6} \psi^* \mathbf(q) \psi \mathbf(p) \braket{\mathbf q, s}{\mathbf p, s} = \int \frac{d^3 p}{(2\pi)^3} \psi^* \mathbf(q) \psi \mathbf(p) > 0 ~.
        \end{equation*}
    \end{proof}

\section{Electric charge}

    In order to distinguish $b$-particles and $c$-particles, we use the Noether charge associated to the global $U(1)$ symmetry. The Noether current is 
    \begin{equation*}
        j^\mu = q \overline \psi \gamma^\mu \psi ~.
    \end{equation*}
    \begin{proof}
        A global $U(1)$ symmetry is 
        \begin{equation*}
            \psi' = \exp(i q \theta) \psi ~, \quad \overline \psi' = \exp(- i q \theta) ~,
        \end{equation*}
        which infinitesimally becomes
        \begin{equation*}
            \delta \psi = i q \theta ~, \quad \delta \overline \psi = - i q \theta ~.
        \end{equation*}

        The Noether current is 
        \begin{equation*}
            j^\mu = \pdv{L}{\partial_\mu \psi} \delta \psi + \cancel{\pdv{L}{\partial_\mu \overline \psi}} \delta \overline \psi = - q \theta \overline \psi \gamma^\mu \psi \rightarrow q \overline \psi \gamma^\mu \psi ~,
        \end{equation*}
        where we have dropped the constant $- \theta$.
    \end{proof}

    The Noether charge is 
    \begin{equation*}
        \hat Q = q \sum_{s=1}^{2} \int \frac{d^3 p}{(2\pi)^3} ((\hat b^\dagger_{\mathbf p})^s \hat b^r_{\mathbf p} - (\hat c_{\mathbf p}^\dagger)^r \hat c^s_{\mathbf p}) ~.
    \end{equation*}
    \begin{proof}
        In fact, 
        \begin{equation*}
        \begin{aligned}
            Q = \int d^3 x ~ j^0 = \int d^3 x ~ q \overline \psi \gamma^0 \psi = q \int d^3 x ~  \psi^\dagger (\gamma^0)^2 \psi = q \int d^3 x ~ \psi^\dagger \psi ~.
        \end{aligned}
        \end{equation*}

        Promoting to operator 
        \begin{equation*}
        \begin{aligned}
            \hat Q & = q \int d^3 x ~ \sum_{s=1}^{2} \int \frac{d^3 p}{(2\pi)^3} \frac{1}{\sqrt{2 E_{\mathbf p}}} \Big ( (\hat b^\dagger_{\mathbf p})^s (u^\dagger)^s (\mathbf p) \exp(- i \mathbf p \cdot \mathbf x) + \hat c^s_{\mathbf p} (v^\dagger)^s(\mathbf p) \exp(i \mathbf p \cdot \mathbf x) \Big) \\ & \quad \sum_{r=1}^{2} \int \frac{d^3 q}{(2\pi)^3} \frac{1}{\sqrt{2 E_{\mathbf q}}} \Big ( \hat b^r_{\mathbf q} u^r (\mathbf q) \exp(i \mathbf q \cdot \mathbf x) + (\hat c_{\mathbf q}^\dagger)^r v^r(\mathbf q) \exp(- i \mathbf q \cdot \mathbf x) \Big) \\ & = q \sum_{s=1}^{2} \sum_{r=1}^{2} \int \frac{d^3 x ~ d^3 p ~ d^3 q}{(2\pi)^6} \frac{1}{2 \sqrt{E_{\mathbf q} E_{\mathbf q}}}  \Big ( (\hat b^\dagger_{\mathbf p})^s \hat b^r_{\mathbf q}  (u^\dagger)^s (\mathbf p) u^r (\mathbf q) \underbrace{\exp(i (- \mathbf p + \mathbf q) \cdot \mathbf x)}_{\delta^3 (\mathbf p - \mathbf q)} \\ & \quad + (\hat b^\dagger_{\mathbf p})^s (\hat c_{\mathbf q}^\dagger)^r (u^\dagger)^s (\mathbf p) v^r(\mathbf q) \underbrace{\exp( -i (\mathbf p + \mathbf q) \cdot \mathbf x)}_{\delta^3 (\mathbf p + \mathbf q)} \\ & \quad + \hat c^s_{\mathbf p} \hat b^r_{\mathbf q} (v^\dagger)^s (\mathbf p) u^r (\mathbf q) \underbrace{\exp(i (\mathbf p + \mathbf q) \cdot \mathbf x)}_{\delta^3 (\mathbf p + \mathbf q)} \\ & \quad + \hat c^s_{\mathbf p} (\hat c_{\mathbf q}^\dagger)^r (v^\dagger)^s(\mathbf p) v^r(\mathbf q) \underbrace{\exp(i (\mathbf p - \mathbf q) \cdot \mathbf x)}_{\delta^3 (\mathbf p - \mathbf q)} \Big )
        \end{aligned}
        \end{equation*} 
        \begin{equation*}
        \begin{aligned}
            & = q \sum_{s=1}^{2} \sum_{r=1}^{2} \int \frac{d^3 p}{(2\pi)^3} \frac{1}{2 \sqrt{E_{\mathbf q} E_{\mathbf q}}}  \Big ( (\hat b^\dagger_{\mathbf p})^s \hat b^r_{\mathbf p} \underbrace{(u^\dagger)^s (\mathbf p) u^r (\mathbf p)}_{2 |\mathbf p| \delta^{rs}} + (\hat b^\dagger_{\mathbf p})^s (\hat c_{ - \mathbf p}^\dagger)^r \underbrace{(u^\dagger)^s (\mathbf p) v^r( -\mathbf p)}_0 \\ & \quad + \hat c^s_{\mathbf p} \hat b^r_{ - \mathbf p} \underbrace{(v^\dagger)^s (\mathbf p) u^r ( - \mathbf p)}_0 + \hat c^s_{\mathbf p} (\hat c_{\mathbf p}^\dagger)^r \underbrace{(v^\dagger)^s(\mathbf p) v^r(\mathbf p)}_{2 |\mathbf p| \delta^{rs}}  \Big ) \\ & = q \sum_{s=1}^{2} \int \frac{d^3 p}{(2\pi)^3} \frac{1}{2 |\mathbf p|} (2 |\mathbf p|) ((\hat b^\dagger_{\mathbf p})^s \hat b^r_{\mathbf p} + \hat c^s_{\mathbf p} (\hat c_{\mathbf p}^\dagger)^r) ~,
        \end{aligned}
        \end{equation*}
        which in (anticommutating) normal ordering is 
        \begin{equation*}
            \hat Q = q \sum_{s=1}^{2} \int \frac{d^3 p}{(2\pi)^3} ((\hat b^\dagger_{\mathbf p})^s \hat b^r_{\mathbf p} - (\hat c_{\mathbf p}^\dagger)^r \hat c^s_{\mathbf p}) ~.
        \end{equation*}
    \end{proof}

    We can see that it distinguishes indeed $b$ and $c$-particles, since it associates an eigenvalue of $+q$ for $b$ and $-q$ for $c$.
    \begin{proof}
        For $b$
        \begin{equation*}
        \begin{aligned}
            \hat Q (\hat b^\dagger_{\mathbf p})^s \ket{0} & = q \sum_{r=1}^2 \int \frac{d^3 k}{(2\pi)^3} ((\hat b_{\mathbf k}^\dagger)^r \underbrace{\hat b_{\mathcal k}^r (\hat b^\dagger_{\mathbf p})^s \ket{0}}_{(2\pi)^3 \delta(\mathbf k - \mathbf p) \delta^{rs} \ket{0} - (\hat b^\dagger_{\mathbf p})^s \cancel{\hat b_{\mathcal k}^r \ket{0}}} - (\hat c_{\mathbf k}^\dagger)^r \underbrace{\hat c_{\mathcal k}^r (\hat b^\dagger_{\mathbf p})^s}_{- (\hat b^\dagger_{\mathbf p})^s \cancel{\hat c_{\mathcal k}^r \ket{0}}}  ) \\ &  = q (\hat b^\dagger_{\mathbf p})^s \ket{0} ~.
        \end{aligned}
        \end{equation*}

        For $c$
        \begin{equation*}
        \begin{aligned}
            \hat Q (\hat c^\dagger_{\mathbf p})^s \ket{0} & = q \sum_{r=1}^2 \int \frac{d^3 k}{(2\pi)^3} (- (\hat c_{\mathbf k}^\dagger)^r \underbrace{\hat c_{\mathcal k}^r (\hat c^\dagger_{\mathbf p})^s \ket{0}}_{(2\pi)^3 \delta(\mathbf k - \mathbf p) \delta^{rs} \ket{0} - (\hat c^\dagger_{\mathbf p})^s \cancel{\hat c_{\mathcal k}^r \ket{0}}} + (\hat b_{\mathbf k}^\dagger)^r \underbrace{\hat b_{\mathcal k}^r (\hat c^\dagger_{\mathbf p})^s}_{- (\hat c^\dagger_{\mathbf p})^s \cancel{\hat b_{\mathcal k}^r \ket{0}}}  ) \\ &  = q (\hat c^\dagger_{\mathbf p})^s \ket{0} ~.
        \end{aligned}
        \end{equation*}
    \end{proof}

\section{Spin-statistics}

    The wave function is antisymmetric 
    \begin{equation*}
        \ket{\mathbf p_1, s_1; \mathbf p_2, s_2} = - \ket{\mathbf p_2, s_2; \mathbf p_1, s_1} ~.
    \end{equation*}
    \begin{proof}
        In fact 
        \begin{equation*}
            \ket{\mathbf p_1, s_1; \mathbf p_2, s_2} = (\hat b^\dagger_{\mathbf p_1})^{s_1} (\hat b^\dagger_{\mathbf p_2})^{s_2} \ket{0} = (\hat b^\dagger_{\mathbf p_2})^{s_2} (\hat b^\dagger_{\mathbf p_1})^{s_1} \ket{0} = - \ket{\mathbf p_2, s_2; \mathbf p_1, s_1} ~.
        \end{equation*}
    \end{proof}

    Notice that the Pauli exclusion principle holds, since 
    \begin{equation*}
        \ket{\mathbf p, s; \mathbf p, s} = - \ket{\mathbf p, s; \mathbf p, s}  \Rightarrow \ket{\mathbf p, s; \mathbf p, s} = 0~.
    \end{equation*}

\section{Propagator} 

    Now, we move into Heisenberg picture. The Dirac spinor depends also on time $\psi(x) = \psi(t, \mathbf x)$ and it satisfies the Heisenberg equation 
    \begin{equation*}
        \pdv{}{t} \hat \psi(x) = i [\hat H, \hat \psi (x)] ~.
    \end{equation*}
    The field operators become
    \begin{equation*}
        \hat \psi(x) = \sum_{s=1}^2 \int \frac{d^3 p}{(2\pi)^3} \frac{1}{\sqrt{2 E_{\mathbf p}}} \Big (\hat b_{\mathbf p}^s u^s (\mathbf p) \exp(- i p^\mu x_\mu ) + \hat c_{\mathbf p}^{\dagger s} v^s (\mathbf p) \exp(i p^\mu x_\mu ) \Big) ~, 
    \end{equation*}
    \begin{equation*}
        \hat \psi^\dagger(x) = \sum_{s=1}^2 \int \frac{d^3 p}{(2\pi)^3} \frac{1}{\sqrt{2 E_{\mathbf p}}} \Big (\hat b_{\mathbf p}^{\dagger s} u^{\dagger s} (\mathbf p) \exp(i p^\mu x_\mu ) + \hat c_{\mathbf p}^{s} v^{\dagger s} (\mathbf p) \exp(- i p^\mu x_\mu ) \Big) ~.
    \end{equation*} 
    The fermionic propagator is defined in terms of the anticommutator
    \begin{equation*}
        i S_{\alpha\beta} = \{\hat \psi_\alpha (x), \hat{ \overline \psi_\beta} (y)\} ~,
    \end{equation*}
    or in compact notation 
    \begin{equation*}
        i S(x - y) = \{\hat \psi (x), \hat{\overline \psi}(y)\} ~.
    \end{equation*}
    We can write it in terms of bosonic propagators 
    \begin{equation*}
        i S(x - y) = (i \gamma^\mu \partial_\mu + m) (D(x-y) - D(y-x)) ~.
    \end{equation*}
    Notice that for spacelike separated points, we have seen that $D(x-y) - D(y-x) = 0$, therefore for $(x-y)^2 < 0$
    \begin{equation*}
        \begin{cases}
            [\hat \varphi (x), \hat \varphi (y)] = 0 & \textnormal{bosons} \\
            \{\hat \varphi (x), \hat \varphi (y)\} = 0 & \textnormal{fermions} \\
        \end{cases} ~.
    \end{equation*}
    This ensures that causality is not violated. This holds because all the observables are bilinears of fermions. Hence, they commute and they do not anticommute outside the lightcone. The fermionic propagator satisfies Dirac equation as a matrix 
    \begin{equation*}
        (i \gamma^\mu \partial_\mu - m) S(x-y) = 0~.
    \end{equation*}
    The probability amplitude to propagate freely for a particle from $y$ to $x$ is 
    \begin{equation*}
        \bra{0} \hat \psi_\alpha (x) \hat {\overline \psi_\beta} (y) \ket{0} = \int \frac{d^3 p}{(2\pi)^3} \frac{1}{2E_{\mathbf p}} (\gamma^\mu p_\mu + m)_{\alpha\beta} \exp(- i p (x-y)) ~,
    \end{equation*}
    whereas the probability amplitude to propagate freely for an antiparticle from $x$ to $y$ is 
    \begin{equation*}
        \bra{0} \hat {\overline \psi_\beta} (y) \hat \psi_\alpha (x) \ket{0} = \int \frac{d^3 p}{(2\pi)^3} \frac{1}{2E_{\mathbf p}} (\gamma^\mu p_\mu - m)_{\alpha\beta} \exp(i p (x-y)) ~.
    \end{equation*}

\part{Maxwell's theory}

\chapter{Maxwell's action}

    In this chapter, we will review some notion of classical electrodynamics: Maxwell's Lagrangian that leads to the Maxwell's equations and how, using gauge symmetry in vacuum, an electromagnetic wave carries only two degrees of freedom, which are the two transversal polarisations indepependent components.

\section{Maxwell's Lagrangian}

    Classical free electrodynamic fields, without sources, can be describe starting from the Maxwell's Lagrangian
    \begin{equation}\label{maxlag}
    \begin{aligned}
        \mathcal L & = - \frac{1}{4} F_{\mu\nu} F^{\mu\nu} = - \frac{1}{4} (\partial_\mu A_\nu - \partial_\nu A_\mu) (\partial^\mu A^\nu - \partial^\nu A^\mu) \\ & = - \frac{1}{4} ( \partial_\mu A_\nu \partial^\mu A^\nu - \partial_\nu A_\mu \partial^\mu A^\nu - \partial_\mu A_\nu \partial^\nu A^\mu + \partial_\nu A_\mu  \partial^\nu A^\mu ) \\ & = - \frac{1}{2} (\partial_\mu A_\nu \partial^\mu A^\nu - \partial_\nu A_\mu \partial^\mu A^\nu) ~,
    \end{aligned}
    \end{equation}
    where $F_{\mu\nu}$ is the electromagnetic tensor
    \begin{equation*}
        F_{\mu\nu} = \partial_\mu A_\nu - \partial_\nu A_\mu = \begin{bmatrix}
            0 & E_1 & E_2 & E_3 \\ 
            -E_1 & 0 & - B_3 & B_2 \\ 
            - E_2 & B_3 & 0 & - B_1 \\ 
            - E_3 & -B_2 & B_1 & 0 \\
        \end{bmatrix}
    \end{equation*}
    and the $4$-potential is $A^\mu = (\phi, \mathbf A)$. Recall that they are related to the electric and the magnetic field via 
    \begin{equation}\label{ef}
        \mathbf B = \boldsymbol \nabla \times \mathbf A ~, \quad \mathbf E = - \boldsymbol \nabla \phi - \pdv{\mathbf A}{t} 
    \end{equation}
    and we can write them in terms of the electromagnetic tensor as 
    \begin{equation*}
        E_i = F_{0i} ~, \quad B_i = \frac{1}{2} \epsilon_{ijk} F^{jk} ~.
    \end{equation*}
    From this Lagrangian, we recover the equation of motion, which are exactly Maxwell's equations in vacuum
    \begin{equation*}
        \partial_\mu F^{\mu\nu} = 0 ~.
    \end{equation*}
    \begin{proof}
        Using~\eqref{eleq} and~\eqref{maxlag}, we have
        \begin{equation*}
        \begin{aligned}
            0 & = \partial_\mu \pdv{\mathcal L}{\partial_\mu A_\nu} - \underbrace{\pdv{\mathcal L}{A_\nu}}_0 = \partial_\mu \pdv{}{\partial_\mu A_\nu} (- \frac{1}{2} (\partial_\alpha A_\beta \partial^\alpha A^\beta - \partial_\beta A_\alpha \partial^\alpha A^\beta)) \\ & = - \partial_\mu (\partial^\mu A^\nu - \partial^\nu A^\nu) = - \partial_\mu F^{\mu\nu} ~.
        \end{aligned}
        \end{equation*}
    \end{proof}
    An useful property of the electromagnetic tensor is that it satisfies the Bianchi identity 
    \begin{equation*}
        \partial_\mu F_{\nu\lambda} + \partial_\nu F_{\lambda \mu} + \partial_\lambda F_{\mu \nu} = 0 ~.
    \end{equation*}
    \begin{proof}
        In fact, we have 
        \begin{equation*}
        \begin{aligned}
            \partial_\mu F_{\nu\lambda} + \partial_\nu F_{\lambda \mu} + \partial_\lambda F_{\mu \nu} & = \partial_\mu (\partial_\nu A_\lambda - \partial_\lambda A_\nu) + \partial_\nu (\partial_\lambda A_\mu - \partial_\mu A_\lambda) + \partial_\lambda (\partial_\mu A_\nu - \partial_\nu A_\mu) \\ & = \partial_\mu \partial_\nu A_\lambda -  \partial_\mu \partial_\lambda A_\nu + \partial_\nu \partial_\lambda A_\mu - \partial_\nu  \partial_\mu A_\lambda + \partial_\lambda \partial_\mu A_\nu - \partial_\lambda \partial_\nu A_\mu \\ & = \cancel{\partial_\nu \partial_\mu A_\lambda} - \cancel{\partial_\lambda \partial_\mu A_\nu} +  \cancel{\partial_\lambda \partial_\nu A_\mu} - \cancel{\partial_\nu \partial_\mu A_\lambda} + \cancel{\partial_\lambda \partial_\mu A_\nu} - \cancel{\partial_\lambda \partial_\nu A_\mu} = 0 ~,
        \end{aligned}
        \end{equation*}
        where we have used the fact that partial derivatives commute.
    \end{proof}
    We define the dual electromagnetic tensor
    \begin{equation*}
        \tilde F^{\mu\nu} = - \frac{1}{2} \epsilon^{\mu\nu\rho\sigma} F_{\rho\sigma} ~,
    \end{equation*}
    or, explicitly, 
    \begin{equation*}
        \tilde F_{\mu\nu} = \begin{bmatrix}
            0 & - B_1 & - B_2 & - B_3 \\ 
            B_1 & 0 & E_3 & -E_2 \\ 
            B_2 & -E_3 & 0 & E_1 \\ 
            B_3 & E_2 & -E_1 & 0 \\
        \end{bmatrix} ~.
    \end{equation*}
    Notice that there is a duality symmetry, since if we perfom the exhange $\mathbf E \leftrightarrow - \mathbf B$, we find $F^{\mu\nu} \leftrightarrow \tilde F^{\mu\nu}$. We can write the Bianchi identity in terms of the dual tensor as
    \begin{equation*}
        \partial_\mu \tilde F^{\mu\nu} = - \frac{1}{2} \epsilon^{\mu\nu\rho\sigma} \partial_\mu  F_{\rho\sigma} = 0 ~.
    \end{equation*}
    \begin{proof}
        In fact, we have
        \begin{equation*}
        \begin{aligned}
            0 & = \partial_\mu F_{\nu\lambda} + \partial_\nu F_{\lambda \mu} + \partial_\lambda F_{\mu \nu} = \epsilon^{\mu\nu\lambda\sigma} (\partial_\mu F_{\nu\lambda} + \partial_\nu F_{\lambda \mu} + \partial_\lambda F_{\mu \nu}) \\ & = \partial_\mu \epsilon^{\mu\nu\lambda\sigma} F_{\nu\lambda} + \partial_\nu \epsilon^{\mu\nu\lambda\sigma} F_{\lambda \mu} + \partial_\lambda \epsilon^{\mu\nu\lambda\sigma} F_{\mu \nu} = \cancel{\partial_\mu \epsilon^{\sigma\mu\nu\lambda} F_{\nu\lambda}} - \cancel{\partial_\nu \epsilon^{\sigma\nu\lambda\mu} F_{\lambda\mu}} + \partial_\lambda \epsilon^{\lambda\sigma\mu\nu} F_{\mu \nu} \\ & = \epsilon^{\mu\nu\rho\sigma} \partial_\mu  F_{\rho\sigma} ~.
        \end{aligned}
        \end{equation*}
    \end{proof}

    It can be proved that the Maxwell's equations in covariant formalism can be written as 
    \begin{equation*}
        \boldsymbol \nabla \cdot \mathbf B = 0 ~, \quad \pdv{\mathbf B}{t} = - \boldsymbol \nabla \times \mathbf E \quad \Rightarrow \quad \partial_\mu \tilde F^{\mu\nu} = 0 ~,
    \end{equation*}
    \begin{equation*}
        \boldsymbol \nabla \cdot \mathbf E = 0 ~, \quad \pdv{\mathbf E}{t} = \boldsymbol \nabla \times \mathbf E \quad \Rightarrow \quad \partial_\mu F^{\mu\nu} = 0 ~.
    \end{equation*}
    In presence of sources, the first one remains the same whereas the second one changes because it carries information about them. 

\section{Gauge symmetry}

    It is useful to rewrite the Maxwell's Lagrangian in terms of temporal and spatial indices
    \begin{equation*}
        \mathcal L = - \frac{1}{2} (F_{0i} F^{0i} + F_{ij} F^{ij} ) ~.
    \end{equation*} 
    \begin{proof}
        In fact, we have 
        \begin{equation*}
        \begin{aligned}
            \mathcal L & = - \frac{1}{4} (\partial_\mu A_\nu - \partial_\nu A_\mu)(\partial^\mu A^\nu - \partial^\nu A^\mu) \\ & = -\frac{1}{4} (\partial_\mu A_\nu \partial^\mu A^\nu - \partial_\mu A_\nu \partial^\nu A^\mu -  \partial_\nu A_\mu \partial^\mu A^\nu + \partial_\nu A_\mu \partial^\nu A^\mu) \\ & = - \frac{1}{2} (\partial_\mu \partial^\mu A^\nu A_\nu - \partial_\nu \partial^\mu A^\nu A_\mu) \\ & = - \frac{1}{2} (\cancel{\partial_0 \partial^0 A^0 A_0} + \partial_0 \partial^0 A^i A_i + \partial_i \partial^i A^0 A_0 + \partial_i \partial^i A^j A_j \\ & \quad - \cancel{\partial_0 \partial^0 A^0 A_0} - \partial_0 \partial^i A^0 A_i - \partial_i \partial^j A^i A_j - \partial_i \partial^0 A^i A_0) \\ & = - \frac{1}{2} (\partial_0 \partial^0 A^i A_i + \partial_i \partial^i A^0 A_0 + \partial_i \partial^i A^j A_j - \partial_0 \partial^i A^0 A_i - \partial_i \partial^j A^i A_j - \partial_i \partial^0 A^i A_0) \\ & = - \frac{1}{2} 
            (\partial_0 \partial^0 A^i A_i - \partial_0 \partial^i A_i A^0 - \partial_i  \partial^0 A^0 A^i + \partial_j \partial^j A_i A^i \\ & \quad - \partial_i \partial^j A_j A^i + \partial_j \partial^i A_i A^j + \partial_i \partial^j A_j A^i - \partial_i \partial^i A_j A^j ) \\ & = - \frac{1}{2} ((\partial_0 A_i - \partial_i A_0) (\partial^0 A^i - \partial^i A^0) - (\partial_j A_i - \partial_i A_j) (\partial^j A^i - \partial^i A^j)) \\ & = - \frac{1}{2} (F_{0i} F^{0i} + F_{ij} F^{ij} ) ~.
        \end{aligned}
        \end{equation*}
    \end{proof}
    Notice that there is no dependence on the kinetic part of $A^0$, i.e. $\dot A^0$, which means that $A^0$ is fully determined by $A^i$. Therefore, we do not need to specify initial condition for $A_0$ at $t=t_0$ since the ones of $A_i$ and $\dot A_i$ are sufficient, and $A_\mu$ seems to contain only $3$ independent components. This is a consequence of the Gauss' law, which implies that
    \begin{equation*}
        A_0 (t_0, \mathbf x) = \int d^3 y ~ \frac{1}{4\pi |\mathbf x - \mathbf y} \boldsymbol \nabla \cdot \pdv{\mathbf A}{t} (t_0, \mathbf y)  ~.
    \end{equation*}
    \begin{proof}
        Using the Gauss' law, we have 
        \begin{equation*}
            0 = - \boldsymbol \nabla \cdot \mathbf E = \boldsymbol \nabla \cdot \boldsymbol \nabla A_0 + \boldsymbol \nabla \cdot \pdv{\mathbf A}{t} ~,
        \end{equation*}
        \begin{equation*}
            \nabla^2 A_0 (t_0, \mathbf x) = \boldsymbol \nabla \cdot \pdv{\mathbf A}{t} (t_0, \mathbf x) ~.
        \end{equation*}
        Now, we use the Green function to solve this differential equation. The Green operator is 
        \begin{equation*}
            \nabla^2 G (t_0, \mathbf x - \mathbf y) = \delta^3 (\mathbf x - \mathbf y) ~.
        \end{equation*}
        By a Fourier transform 
        \begin{equation*}
            G (t_0, \mathbf x - \mathbf y) = \int \frac{d^3 p}{(2\pi)^3} ~ \tilde G (p) \exp(- i \mathbf p \cdot (\mathbf x - \mathbf y)) ~,
        \end{equation*}
        we find 
        \begin{equation*}
        \begin{aligned}
            \nabla^2_x G (t_0, \mathbf x - \mathbf y) & = \nabla^2_x \int \frac{d^3 p}{(2\pi)^3} ~ \tilde G (p) \exp(- i \mathbf p \cdot (\mathbf x - \mathbf y)) \\ &= \int \frac{d^3 p}{(2\pi)^3} ~ \tilde G (p) (- p^2) \exp(- i \mathbf p \cdot (\mathbf x - \mathbf y)) \\ & = \delta^3 (\mathbf x - \mathbf y) = \int \frac{d^3 p}{(2\pi)^3} ~ \exp(- i \mathbf p \cdot (\mathbf x - \mathbf y)) ~.
        \end{aligned}
        \end{equation*}
        which implies that 
        \begin{equation*}
            \tilde G (p) = - \frac{1}{p^2} ~.
        \end{equation*}
        Putting inside the Fourier transform and using polar coordinates in momentum space, we obtain 
        \begin{equation*}
        \begin{aligned}
            G (t_0, \mathbf x - \mathbf y) & = - \int \frac{d^3 p}{(2\pi)^3} ~ \frac{\exp(- i \mathbf p \cdot (\mathbf x - \mathbf y))}{p^2} \\ & = - 2 \pi \frac{1}{(2\pi)^3} \int_0^\infty dp ~ p^2 \int_0^\pi d\theta ~ \sin \theta \frac{\exp(- i p \cos\theta |\mathbf x - \mathbf y|)}{p^2} \\ & = - \frac{1}{4 \pi^2} \int_0^\infty dp \int_{-1}^{1} d (- \cos \theta) ~ \exp(- i p \cos \theta |\mathbf x - \mathbf y|) \\ & = - \frac{1}{4 \pi^2} \int_0^\infty dp ~ \frac{\exp(-i p \cos \theta |\mathbf x - \mathbf y|)}{i p |\mathbf x - \mathbf y|} \Big \vert_{-\cos \theta = -1}^{-\cos \theta = 1} \\ & = - \frac{1}{4 \pi^2 |\mathbf x - \mathbf y| i} \int_0^\infty dp ~ \frac{\exp(i p \cos \theta |\mathbf x - \mathbf y|)}{p} - \frac{\exp(-i p \cos \theta |\mathbf x - \mathbf y|)}{p} \\ & = - \frac{1}{4 \pi^2 |\mathbf x - \mathbf y| i} \int_{-\infty}^\infty dp ~ \frac{\exp(i p \cos \theta |\mathbf x - \mathbf y|)}{p}  \\ & = - \frac{1}{4 \pi^2 |\mathbf x - \mathbf y| i} \pi i \exp (i p |\mathbf x - \mathbf y|) \Big \vert_{p = 0} = - \frac{1}{4 \pi |\mathbf x - \mathbf y|} ~,
        \end{aligned}
        \end{equation*}
        where we have integrated in the upper-part of the complex plane with one pole in $p=0$. Finally, we find 
        \begin{equation*}
            A_0 (t_0, \mathbf x) = - \int d^3 y ~ G(\mathbf x - \mathbf y) \boldsymbol \nabla \cdot \pdv{\mathbf A}{t} (t_0, \mathbf y) = \int d^3 y ~ \frac{1}{4\pi |\mathbf x - \mathbf y|} \boldsymbol \nabla \cdot \pdv{\mathbf A}{t} (t_0, \mathbf y)~.
        \end{equation*}
    \end{proof}

    This is a signal that Maxwell's theory is a gauge theory. In fact, Maxwell's Lagrangian is symmetric with respect to a gauge transformation  
    \begin{equation*}
        {A'}_\mu (x) = A_\mu (x) + \partial_\mu \alpha (x) ~,
    \end{equation*}
    where $\alpha (x)$ is an arbitrary gauge function of spacetime coordinates, such that its derivative vanishes at spatial infinity.
    \begin{proof}
        In fact, we have
        \begin{equation*}
            {F'}^{\mu\nu} = \partial_\mu {A'}_\nu - \partial_\nu {A'}_\mu = \partial_\mu A_\nu - \partial_\nu A_\mu + \cancel{\partial_\mu \partial_\nu \alpha(x)} - \cancel{\partial_\nu \partial_\mu \alpha(x)} = \partial_\mu A_\nu - \partial_\nu A_\mu = F^{\mu\nu} ~,
        \end{equation*}
        which means that
        \begin{equation*}
            \mathcal L' = - \frac{1}{4} {F'}^{\mu\nu} F'_{\mu\nu} = - \frac{1}{4} F^{\mu\nu}_{\mu\nu} = \mathcal L ~.
        \end{equation*}
    \end{proof}

    However, we experimentally know that an electromagnetic wave traveling in vacuum has only two degrees of freedom corresponding to the two transversal polarisations. This signals that there is a second residual gauge transformation 
    \begin{equation*}
        {A''}_\mu (x) = {A'}_\mu (x) + \partial_\mu \beta (x) = A_\mu (x) + \partial_\mu (\alpha (x) + \beta (x) ) ~,
    \end{equation*}
    where $\beta (x)$ is the second gauge function. Physically, $A$, $A'$ and $A''$ descibe the same physical state, since the Lagrangians are the same. Therefore, they define a class of equivalence of physical points of view. There are several choices for the representative of this class or gauge orbit by some conditions, called gauge fixing, based on the convenience to make easier the problem. See Figure~\ref{fig:gauge}. It is important to highlight the difference between a global and a local symmetry. A global symmetry does not depend on the spacetime coordinates and it is a true symmetry of the system, which leads to the Noether's theorem. A local symmetry is just a redundancy in the description of the physics and it does not have a conservation law (even because there will be infinitely many).

    \begin{figure}[h!]
        \centering
        \begin{tikzpicture}
        \draw[] (0,0) to[bend right=20] (2,2) to[bend left=20] (4,4) node[above right] {gauge orbits};
        \draw[] (2,0) to[bend right=20] (4,2) to[bend left=20] (6,4) ;
        \draw[] (4,0) to[bend right=20] (6,2) to[bend left=20] (8,4) ;

        \draw[] (0,4) to[bend right=20] (4,2) to[bend left=20] (8,0) node[below left] {gauge fixing};
        \draw[] (0,3) to[bend right=20]  (2,2) to[bend left=20] (6,0) ;

        \filldraw[black] (4,2) circle (0.05) node[above left] {$A_\mu$};
        \filldraw[black] (2,2) circle (0.05) node[below right] {$A'_\mu$};

        \end{tikzpicture}
        \caption{Pictorial representation of gauge orbits and gauge fixing.}
        \label{fig:gauge}
    \end{figure}
    
    The gauge fixing we will use in this notes is the Lorenz gauge, which is manifestly Lorentz invariant and brings down the number of degrees of freedom to $2$
    \begin{equation*}
        \partial_\mu A^\mu = 0 ~.
    \end{equation*}
    \begin{proof}
        With a gauge transformation, we have
        \begin{equation*}
            {A'}_\mu (x) = A_\mu (x) + \partial_\mu \alpha (x) ~,
        \end{equation*}
        where $\alpha (x)$ must satisfy 
        \begin{equation*}
            0 = \partial_\mu {A'}^\mu = \partial_\mu A^\mu + \Box \alpha (x) ~,
        \end{equation*}
        hence, we find
        \begin{equation*}
            \Box \alpha(x) = \partial_\mu A^\mu ~.
        \end{equation*}
        With the second gauge transformation, we have
        \begin{equation*}
            {A''}_\mu (x) = {A'}_\mu (x) + \partial_\mu \beta (x) ~,
        \end{equation*}
        where $\beta (x)$ must satisfy 
        \begin{equation*}
            \partial_\mu {A''}^\mu = \partial_\mu {A'}^\mu + \Box \beta (x) ~,
        \end{equation*}
        hence, we find
        \begin{equation*}
            \Box \beta(x) = 0 ~.
        \end{equation*}
    \end{proof}
    The equations of motion in the Lorenz gauge become 
    \begin{equation}\label{lgem}
        \Box A^\mu (x) = 0 ~.
    \end{equation}
    Notice that the Maxwell's equations in the Lorentz gauge are equal to the Klein-Gordon ones with mass equals to zero. This ensures that the mass shell condition is preserved. This means that each components of $A^\mu$ separately satisfies the mass shell condition. At quantum level, we will see taht $A_\mu$ describes particles (photons) with energy $E_{\mathbf p} = |\mathbf p|$.
    \begin{proof}
        In fact, we have
        \begin{equation*}
            0 = \partial_\mu F^{\mu\nu} = \partial_\mu (\partial^\mu A^\nu - \partial^\nu A^\mu) = \Box A^\nu - \partial^\nu \underbrace{\partial_\mu A^\mu}_0 = \Box A^\nu ~.
        \end{equation*}
    \end{proof}

    The conjugate momentum of $A_\mu$ is 
    \begin{equation*}
        \pi^\mu = (0, \mathbf E) ~.
    \end{equation*}
    \begin{proof}
        In fact, for $\mu = 0$, we have
        \begin{equation*}
            \pi^0 = \pdv{\mathcal L}{\dot A_0} = 0 ~,
        \end{equation*}
        whereas, for $\mu = i$, we have
        \begin{equation*}
        \begin{aligned}
            \pi^i = \pdv{\mathcal L}{\dot A_i} = - \frac{1}{2} \pdv{}{\dot A_i} ((\dot A_j - \partial_j A_0))(\dot A^j - \partial^j A_0) = - \frac{1}{2} \pdv{}{\dot A_i} (F_{0j} F^{0j}) = - F^{0i} = E^i ~.
        \end{aligned}
        \end{equation*}
    \end{proof}

    The Hamiltonian is 
    \begin{equation*}
        H = \int d^3 x ~\Big ( \frac{1}{2}  (|E|^2 + |B|^2) - A_0 (\boldsymbol \nabla \cdot \mathbf E) \Big) ~.
    \end{equation*}
    Notice that we have a Lagrange multiplier $A_0$ to ensure the constrain of the Gauss' law, since $A_0$ is not a physical variable. In fact, one of the Hamilton's equation is
    \begin{equation*}
        0 = \dot A_0 = \pdv{\mathcal H}{A_0} = - \boldsymbol \nabla \cdot \mathbf E ~. 
    \end{equation*}
    \begin{proof}
        In fact, by a Legendre transformation and~\eqref{ef}, we have
        \begin{equation*}
        \begin{aligned}
            \mathcal H & = \pi^\mu \dot A_\mu - \mathcal L = \pi^i \dot A_i - \mathcal L = - E^i \underbrace{\dot A_i}_{- E_i - \partial_i A_0} - \mathcal L \\ & = E^i E_i + E^i \partial_i A_0 - \frac{1}{2} (|E|^2 - |B|^2) = \frac{1}{2} (|E|^2 + |B|^2) + E^i \partial_i A_0 ~,
        \end{aligned}
        \end{equation*}
        hence, we find
        \begin{equation}
        \begin{aligned}
            H & = \int d^3 x ~ \mathcal H = \int d^3 x ~ \Big ( \frac{1}{2} (|E|^2 + |B|^2) + \underbrace{E^i \partial_i A_0}_{- A_0 \partial_i E^i + \textnormal{boundary terms}} \Big) \\ & = \int d^3 x ~ \Big ( \frac{1}{2} (|E|^2 + |B|^2) - A_0 (\boldsymbol \nabla \cdot \mathbf E) \Big) ~.
        \end{aligned}
        \end{equation}
    \end{proof}

\chapter{Quantisation}

    In this chapter, we will quantise the Maxwell's theory. We will find field operators by canonical commutation relations, we will analyse the Fock space, finding which are the physical states compatible with the Lorenz gauge, and we will find the Hamiltonian operator.

\section{Quantisation without Lorenz gauge}

    The first guess to quantise the theory is, instead of imposind by hand the Lorenz gauge, to modify the Lagrangian 
    \begin{equation*}
        \mathcal L = - \frac{1}{4} F_{\mu\nu} F^{\mu\nu} - \frac{1}{2} (\partial_\mu A^\mu)^2 ~. 
    \end{equation*}
    Therefore, the equations of motion remains the same, even if we do not impose the Lorenz gauge, and the conjugate momentum becomes
    \begin{equation*}
        \pi^\mu = (- \partial_\mu A^\mu, F^{i0}) = F^{\nu 0} - \eta^{\nu 0} \partial_\mu A^\mu ~.
    \end{equation*}
    \begin{proof}        
        For the equations of motion, using~\eqref{eleq}, we have 
        \begin{equation*}
            0 = \partial_\mu \pdv{\mathcal L}{\partial_\mu \partial_\mu A_\nu} = - \partial_\mu F^{\mu\nu} - \partial^\nu \partial_\alpha A^\alpha ~,
        \end{equation*}
        hence, we obtain 
        \begin{equation*}
            0 = \partial_\mu F^{\mu\nu} + \partial_\mu \eta^{\mu\nu} \partial_\alpha A^\alpha = \partial_\mu \partial^\mu A^\nu - \partial_\mu \partial^\nu A^\mu + \partial^\nu \partial_\alpha A^\alpha = \partial_\mu \partial^\mu A^\nu ~,
        \end{equation*}
        where we have used the fact that partial derivatives commute.
        For the conjugate momentum, for $\mu = 0$, we have
        \begin{equation*}
            \pi^0 = \pdv{\mathcal L}{\dot A_0} = - \frac{1}{2} \pdv{\mathcal L}{\dot A_0} (\partial_\mu A^\mu)^2 = - \partial_\mu A^\mu  ~,
        \end{equation*}
        whereas, for $\mu = i$, we have
        \begin{equation*}
            \pi^i = E^i = - F^{0i} = F^{i0} ~.
        \end{equation*}
        Finally, to recover the covariant formalism, we find
        \begin{equation*}
            \pi^0 = \underbrace{F^{00}}_0 - \underbrace{\eta^{00}}_1 (\partial_\mu A^\mu) = - \partial_\mu A^\mu 
        \end{equation*}
        and 
        \begin{equation*}
            \pi^i = F^{0i} - \underbrace{\eta^{i0}}_0 (\partial_\mu A^\mu) = F^{0i} ~.
        \end{equation*}
    \end{proof}

    Now, we use the machinery of second quantisation: in Schoedinger picture, we promote $A^\mu(x)$ and $\pi^\mu(x)$ to operators in the Fock space by imposing the canonical commutation, since we have an integer spin theory ($s = 1$), 
    \begin{equation*}
        [\hat A_\mu (\mathbf x), \hat A_\nu (\mathbf y)] = [\hat \pi_\mu (\mathbf x), \hat \pi_\nu (\mathbf y)] = 0 ~, \quad [\hat A_\mu (\mathbf x), \hat \pi_\nu (\mathbf y)] = i \eta_{\mu\nu} \delta^3 (\mathbf x - \mathbf y) ~.
    \end{equation*}

    Since the general solution of~\eqref{lgem} is a linear combination of plane waves, we expand the field operators in terms of ladder operators in the following way
    \begin{equation*}
        \hat A_\mu (\mathbf x) = \int \frac{d^3 p}{(2\pi)^3} \frac{1}{\sqrt{2 |\mathbf p|}} \Big ( \hat \xi_\mu (\mathbf p) \exp(i \mathbf p \cdot \mathbf x) + \hat \xi_\mu^\dagger (\mathbf p) \exp(- i \mathbf p \cdot \mathbf x)) ~,
    \end{equation*}
    \begin{equation*}
        \hat \pi_\mu (\mathbf x) = \int \frac{d^3 p}{(2\pi)^3} \Big ( i \sqrt{\frac{|\mathbf p|}{2}} \Big ) \Big ( \hat \xi_\mu (\mathbf p) \exp(i \mathbf p \cdot \mathbf x) - \hat \xi_\mu^\dagger (\mathbf p) \exp(- i \mathbf p \cdot \mathbf x)) ~,
    \end{equation*}
    where $E_{\mathbf p} = |\mathbf p|$ and $\xi_\mu (\mathbf p)$ is the polarisation $4$-vector. Notice that there is a plus sign instead of a minus sign in the conjugate field, because in Maxwell's theory we have $\pi^\mu = - \dot A^\mu$ whereas in Klein-Gordon's theory we have $\pi = \dot \varphi$. Furthermore, polarisation vectors depend on momentum.
    We introduce an orthonormal basis for the polarisation $4$-vectors $\epsilon_\mu^{(\lambda)}$ $\lambda = 0, 1, 2, 3$, such that 
    \begin{equation*}
        \epsilon_\mu^{(\lambda)} \epsilon^{\mu (\lambda')} = \eta^{\lambda \lambda'} ~, \quad \epsilon_\mu^{(\lambda)} \epsilon_\nu^{(\lambda')} \eta_{\lambda \lambda'} = \eta_{\mu\nu} ~.
    \end{equation*}
    Since in second quantisation polarisation vectors are operators, the coefficients on this expansion are the annihilation operators
    \begin{equation*}
        \hat \xi_\mu (\mathbf p) = \sum_{\lambda=0}^3 \epsilon_\mu^{(\lambda)} (\mathbf p) \hat a_{\mathbf p}^{(\lambda)} ~.
    \end{equation*}
    Therefore, the field operators become
    \begin{equation}\label{max:a}
        \hat A_\mu (\mathbf x) = \int \frac{d^3 p}{(2\pi)^3} \frac{1}{\sqrt{2 |\mathbf p|}} \sum_{\lambda=0}^{3} \epsilon_\mu^{(\lambda)} (\mathbf p) \Big ( \hat a_{\mathbf p}^{(\lambda)} \exp(i \mathbf p \cdot \mathbf x) + \hat a_{\mathbf p}^{\dagger (\lambda)}  \exp(- i \mathbf p \cdot \mathbf x) \Big)  ~,
    \end{equation}
    \begin{equation}\label{max:p}
        \hat \pi^\mu (\mathbf x) = \int \frac{d^3 p}{(2\pi)^3} \Big (i \sqrt{\frac{|\mathbf p|}{2}} \Big ) \sum_{\lambda=0}^{3} \epsilon^{\mu(\lambda)} (\mathbf p) \Big ( \hat a_{\mathbf p}^{(\lambda)} \exp(i \mathbf p \cdot \mathbf x) - \hat a_{\mathbf p}^{\dagger (\lambda)}  \exp(- i \mathbf p \cdot \mathbf x) \Big)  ~.
    \end{equation}
    To make contact with the $2$ transversal polarisation of an electromagnetic wave, we choose $\epsilon_\mu^{(1)}$ and $\epsilon_\mu^{(2)}$ to be orthogonal to the motion, such that $\epsilon_\mu^{(1)} p^\mu = \epsilon_\mu^{(2)} p^\mu = 0$. For example, if $p^\mu = (E, 0, 0, E)$ lies along the $z$-direction, we have $\epsilon_\mu^{(1)} p^\mu = \epsilon_0^{(1)} p^0 + \epsilon_3^{(1)} p^3 = E (\epsilon_0^{(1)} + \epsilon_3^{(1)}) = 0$, which means $\epsilon_0^{(1)} = -\epsilon_3^{(1)}$ and for convenience we put to zero $\epsilon_0^{(1)} = \epsilon_3^{(1)} = 0$. Similarly, we have $\epsilon_\mu^{(2)} p^\mu = \epsilon_0^{(2)} p^0 + \epsilon_3^{(2)} p^3 = E (\epsilon_0^{(2)} + \epsilon_3^{(2)}) = 0$, which means $\epsilon_0^{(2)} = -\epsilon_3^{(2)}$ and for convenience we put to zero $\epsilon_0^{(2)} = \epsilon_3^{(2)} = 0$. Therefore, we choose $\epsilon_1^{(1)} = -1$, $\epsilon_2^{(1)} = 0$, $\epsilon_1^{(2)} = 0$ and $\epsilon_2^{(2)} = -1$. In matrix notation, it becomes
    \begin{equation}\label{pol}
        \epsilon^{(0)}_\mu = \begin{bmatrix}
            1 \\ 0 \\ 0 \\ 0 \\
        \end{bmatrix} ~,  \epsilon^{(1)}_\mu = \begin{bmatrix}
            0 \\ - 1 \\ 0 \\ 0 \\
        \end{bmatrix} ~, \epsilon^{(2)}_\mu = \begin{bmatrix}
            0 \\ 0 \\ - 1 \\ 0 \\
        \end{bmatrix} ~, \epsilon^{(3)}_\mu = \begin{bmatrix}
            0 \\ 0 \\ 0 \\ - 1 \\
        \end{bmatrix} ~,
    \end{equation}
    where $\epsilon^{(0)}_\mu$ is timelike, $\epsilon^{(1)}_\mu$ and $\epsilon^{(2)}_\mu$ are spacelike and $\epsilon^{(3)}_\mu$ is the longitudinal polarisation.

    The commutation relations induced by the canonical ones on the ladder operators are 
    \begin{equation}\label{max:coml}
        [\hat a_{\mathbf p}^{(\lambda)}, \hat a_{\mathbf q}^{(\lambda')}] = [\hat a_{\mathbf p}^{\dagger (\lambda)}, \hat a_{\mathbf q}^{\dagger (\lambda')}] = 0 ~, \quad [\hat a_{\mathbf p}^{(\lambda)}, \hat a_{\mathbf q}^{\dagger(\lambda')}] = - (2\pi)^3 \eta^{\lambda \lambda'} \delta^3 (\mathbf p - \mathbf q) ~.
    \end{equation}
    or, explicitly the latter, 
    \begin{equation*}
        [\hat a_{\mathbf p}^{(0)}, \hat a_{\mathbf q}^{\dagger (0)}] = - (2\pi)^3 \delta^3 (\mathbf p - \mathbf q) ~, \quad [\hat a_{\mathbf p}^{(i)}, \hat a_{\mathbf q}^{\dagger (i)}] = (2\pi)^3 \delta^3 (\mathbf p - \mathbf q) ~.
    \end{equation*}
    \begin{proof}
        In fact, using~\eqref{max:a},~\eqref{max:p} and~\eqref{max:coml}, we have
        \begin{equation*}
        \begin{aligned}
            & [\hat A_\mu (\mathbf x), \hat \pi^\nu (\mathbf y)] \\ & = [\int \frac{d^3 p}{(2\pi)^3} \frac{1}{\sqrt{2 |\mathbf p|}} \sum_{\lambda=0}^{3} \epsilon_\mu^{(\lambda)} (\mathbf p) \Big ( \hat a_{\mathbf p}^{(\lambda)} \exp(i \mathbf p \cdot \mathbf x) + \hat a_{\mathbf p}^{\dagger (\lambda)} \exp(- i \mathbf p \cdot \mathbf x) \Big), \\ & \quad \int \frac{d^3 q}{(2\pi)^3} i \sqrt{\frac{|\mathbf q|}{2}} \sum_{\lambda'=0}^{3} \epsilon^{\nu(\lambda')} (\mathbf q) \Big ( \hat a_{\mathbf q}^{(\lambda')}  \exp(i \mathbf q \cdot \mathbf y) - \hat a_{\mathbf q}^{\dagger (\lambda')} \exp(- i \mathbf q \cdot \mathbf y) \Big)] \\ & = \sum_{\lambda=0}^{3} \sum_{\lambda'=0}^{3} \int \frac{d^3 p ~ d^3 q}{(2\pi)^6} \frac{i}{2} \sqrt{\frac{|\mathbf q|}{|\mathbf p|}} \epsilon_\mu^{(\lambda)} \epsilon^{\nu(\lambda')} \Big ( \underbrace{[\hat a_{\mathbf p}^{(\lambda)} , \hat a_{\mathbf q}^{(\lambda')}]}_0 \exp(i (\mathbf p \cdot \mathbf x + \mathbf q \cdot \mathbf y)) \\ & \quad - \underbrace{[\hat a_{\mathbf p}^{(\lambda)} , \hat a_{\mathbf q}^{\dagger (\lambda')}]}_{- (2\pi)^3 \eta^{\lambda \lambda'} \delta^3 (\mathbf p - \mathbf q)} \exp(i (\mathbf p \cdot \mathbf x - \mathbf q \cdot \mathbf y)) + \underbrace{[\hat a_{\mathbf p}^{\dagger (\lambda)} , \hat a_{\mathbf q}^{(\lambda')}]}_{(2\pi)^3 \eta^{\lambda \lambda'} \delta^3 (\mathbf p - \mathbf q)} \exp(i (- \mathbf p \cdot \mathbf x + \mathbf q \cdot \mathbf y)) \\ & \quad - \underbrace{[\hat a_{\mathbf p}^{\dagger(\lambda)} , \hat a_{\mathbf q}^{\dagger(\lambda')}]}_0 \exp(i (- \mathbf p \cdot \mathbf x - \mathbf q \cdot \mathbf y)) \Big) \\ & = \sum_{\lambda=0}^{3} \int \frac{d^3 p}{(2\pi)^3} \frac{i}{2} \underbrace{\epsilon_\mu^{(\lambda)} \epsilon^{\nu(\lambda)}}_{\eta^\nu_{\phantom \nu \mu}} \Big ( \underbrace{\exp(i \mathbf p \cdot (\mathbf x - \mathbf y))}_{\delta^3 (\mathbf x - \mathbf y)} + \underbrace{\exp(- i \mathbf p \cdot (\mathbf x - \mathbf y))}_{\delta^3 (\mathbf x - \mathbf y)} \Big) = i \eta^\nu_{\phantom \nu \mu} \delta^3 (\mathbf x - \mathbf y) ~.
        \end{aligned}
        \end{equation*}
    \end{proof}

    Notice that there is a problem, since we do not have a probabilistic intepretation for a negative norm (only for the temporal components). We cannot neither interpret $\hat a$ as a creation operator, because we would have problems for the spatial components. This kind of states are called ghosts. We could have seen it already by the Lagrangian, since we had a minus sign in the temporal component
    \begin{equation*}
        \mathcal L = \frac{1}{2} (- (\dot A_0)^2 + (\dot A_1)^2 + (\dot A_2)^2 +(\dot A_3)^2 ) + \ldots ~.
    \end{equation*}
    \begin{proof}
        In fact, given the vacuum defined as 
        \begin{equation*}
            \hat a^{(\lambda)}_{\mathbf p} \ket{0} = 0 ~,
        \end{equation*}
        a state with polarisation $\lambda$ defined as 
        \begin{equation*}
            \ket{\mathbf p, \lambda} = \hat a^{\dagger (\lambda)}_{\mathbf p} \ket{0} ~,
        \end{equation*}
        has a norm equals to
        \begin{equation*}
            \braket{\mathbf p, \lambda}{\mathbf q, \lambda'} = \bra{0} \hat a^{(\lambda)}_{\mathbf p} \hat a^{\dagger (\lambda')}_{\mathbf q} \ket{0} = \bra{0} \underbrace{[\hat a^{(\lambda)}_{\mathbf p}, \hat a^{\dagger (\lambda')}_{\mathbf q}] }_{(2\pi)^3 \eta^{\lambda \lambda'} \delta^3 (\mathbf p - \mathbf q)} \ket{0} + \bra{0} \hat a^{\dagger (\lambda')}_{\mathbf q} \underbrace{\hat a^{(\lambda)}_{\mathbf p} \ket{0}}_0 = (2\pi)^3 \eta^{\lambda \lambda'} \delta^3 (\mathbf p - \mathbf q) ~,
        \end{equation*}
        which for temporal component $\lambda = \lambda' = 0$, we find
        \begin{equation*}
            \braket{\mathbf p, \lambda=0}{\mathbf q, \lambda'=0} = - (2\pi)^3 \delta^3 (\mathbf p - \mathbf q) \leq 0 ~.
        \end{equation*}
    \end{proof}

\section{Quantisation with Lorenz gauge}

    To solve this problem, we impose the gauge fixing condition, which we have not used yet. However, since the Lorenz gauge contains a time derivative, we must go into the Heisenberg picture $\hat A^\mu (x) = \hat A^\mu (t, \mathbf x)$. There are different possible ways to impose the gauge fixing conditions:
    \begin{enumerate}
        \item on operators $\partial_\mu \hat A^\mu = 0$, 
        \item on states $(\partial_\mu \hat A^\mu) \ket{\psi} = 0$, 
        \item on matrix elements $\bra{\psi} \partial_\mu \hat A^\mu \ket{\psi} = 0$.
    \end{enumerate}

    Let us consider the first case. A contradiction arises because, on one hand, we have 
    \begin{equation*}
        \hat \pi^0 = - \partial_\mu \hat A^\mu = 0 ~,
    \end{equation*}
    on the other hand, the commutation relations at fixed time
    \begin{equation*}
        [\hat A_0 (\mathbf x), \hat \pi_0 (\mathbf y)] = i \eta_{00} \delta^3 (\mathbf x - \mathbf y) \neq 0 ~.
    \end{equation*}

    Let us consider the second case. Positive norm physical state are such that
    \begin{equation*}
        (\partial_\mu \hat A^\mu) \ket{\psi} ~,
    \end{equation*}
    but, if we decomposed the field operator,
    \begin{equation*}
    \begin{aligned}
        \hat A_\mu (x) & = \int \frac{d^3 p}{(2\pi)^3} \frac{1}{\sqrt{2 |\mathbf p|}} \sum_{\lambda=0}^{3} \epsilon_\mu^{(\lambda)} (\mathbf p) \Big ( \hat a_{\mathbf p}^{(\lambda)} (\mathbf p) \exp(i \mathbf p \cdot \mathbf x) + \hat a_{\mathbf p}^{\dagger (\lambda)} (\mathbf p) \exp(- i \mathbf p \cdot \mathbf x) \Big) \\ & = \hat A^+_\mu (x) + \hat A^-_\mu (x) ~,
    \end{aligned}
    \end{equation*}
    where $\hat A^+_\mu (x)$ contains only creation operators 
    \begin{equation*}
        \hat A^+_\mu (x) = \int \frac{d^3 p}{(2\pi)^3} \frac{1}{\sqrt{2 |\mathbf p|}} \sum_{\lambda=0}^{3} \epsilon_\mu^{(\lambda)} (\mathbf p) \hat a_{\mathbf p}^{\dagger (\lambda)} (\mathbf p) \exp(- i \mathbf p \cdot \mathbf x) 
    \end{equation*}
    and $\hat A^-_\mu (x)$ contains only annihilation operators
    \begin{equation*}
        \hat A^-_\mu (x) = \int \frac{d^3 p}{(2\pi)^3} \frac{1}{\sqrt{2 |\mathbf p|}} \sum_{\lambda=0}^{3} \epsilon_\mu^{(\lambda)} (\mathbf p) \hat a_{\mathbf p}^{(\lambda)} (\mathbf p) \exp(i \mathbf p \cdot \mathbf x) ~,
    \end{equation*}
    we find that 
    \begin{equation*}
        \partial^\mu \hat A_\mu \ket{0} = \partial^\mu \hat A_\mu^+ \ket{0} = \partial^\mu \hat A_\mu^- \ket{0} = i p^\mu \hat A_\mu^+ \ket{0} - i p^\mu \underbrace{\hat A_\mu^- \ket{0}}_0 \neq 0 ~,
    \end{equation*}
    which means that vacuum state does not satisfy the Lorenz gauge and it is not physical.

    Let us consider the third case. This condition are called Gupta-Bleuler conditions and they can be formulated in three equivalent ways
    \begin{equation}\label{GB}
        \partial^\mu \hat A^+_\mu \ket{\psi} = 0 \iff \bra{\psi} \partial^\mu \hat A^-_\mu = 0 \iff \bra{\phi} \partial_\mu \hat A^\mu \ket{\psi} = 0 ~.
    \end{equation}
    Physical states of the Hilbert space are the only one that satisfy this condition. It can be proved that it implies a constraint: the number of timelike photons are the same number of longitudinal photons for physical state with same momemtum
    \begin{equation}\label{GB2}
        \bra{\psi} \hat n^{(0)}_{\mathbf p} \ket{\psi} = \bra{\psi} \hat n^{(3)}_{\mathbf p} \ket{\psi} ~.
    \end{equation}
    \begin{proof}
        We start from
        \begin{equation*}
            \partial^\mu \hat A^-_\mu (x) = \int \frac{d^3 p}{(2\pi)^3} \frac{1}{\sqrt{2 |\mathbf p|}} \sum_{\lambda=0}^{3} p^\mu \epsilon_\mu^{(\lambda)} (\mathbf p) \hat a_{\mathbf p}^{(\lambda)} (\mathbf p) \exp(i \mathbf p \cdot \mathbf x) ~,
        \end{equation*}
        which contains the term $p^\mu \epsilon_\mu^{(\lambda)}$, but recall that we have chosed $\epsilon^{(\lambda)}_\mu = 0$ for transversal photons $\lambda = 1,2$. Hence, with $p^\mu = (E, 0, 0, E)$ and~\eqref{pol}, we obtain 
        \begin{equation*}
            0 = (\epsilon^{(0)}_\mu p^\mu \hat a^{(0)}_{\mathbf p} + \epsilon^{(3)}_\mu p^\mu \hat a^{(3)}_{\mathbf p}) \ket{\psi} = E (\underbrace{\epsilon^{(0)}_1}_1 \hat a^{(0)}_{\mathbf p} + \underbrace{\epsilon^{(3)}_3}_{-1} \hat a^{(3)}_{\mathbf p}) \ket{\psi} = E (\hat a^{(0)}_{\mathbf p} - \hat a^{(3)}_{\mathbf p}) \ket{\psi} ~,
        \end{equation*}
        \begin{equation*}
            \hat a^{(0)}_{\mathbf p} \ket{\psi } = \hat a^{(3)}_{\mathbf p} \ket{\psi} \iff \bra{\psi} \hat a^{\dagger(0)}_{\mathbf p} = \bra{\psi} \hat a^{\dagger(3)}_{\mathbf p}  ~.
        \end{equation*}
        Finally, we find
        \begin{equation*}
            \bra{\psi} \hat n^{(0)}_{\mathbf p} \ket{\psi} = \bra{\psi} \hat a^{\dagger(0)}_{\mathbf p} \hat a^{(0)}_{\mathbf p} \ket{\psi} = \bra{\psi} \hat a^{\dagger(3)}_{\mathbf p} \hat a^{(3)}_{\mathbf p} \ket{\psi} = \bra{\psi} \hat n^{(3)}_{\mathbf p} \ket{\psi} ~.
        \end{equation*}
    \end{proof}

    The latter result implies that ghost with only timelike photons cannot exist, since a negative norm state with only timelike photons is unphysical. 
    \begin{proof}
        In fact, for $\ket{\mathbf q, \lambda = 0} = \hat a^{\dagger (0)}_{\mathbf q} \ket{0}$ such that 
        \begin{equation*}
            (\hat a^{(0)}_{\mathbf p} - \hat a^{(3)}_{\mathbf p}) \ket{\mathbf q, \lambda = 0} = \underbrace{\hat a^{(0)}_{\mathbf p} \hat a^{\dagger (0)}_{\mathbf q}}_{- (2\pi)^3 \delta^3 (\mathbf p - \mathbf q)} \ket{0} - \underbrace{\hat a^{(3)}_{\mathbf p} \hat a^{\dagger (0)}_{\mathbf q}}_0 \ket{0} = - (2\pi)^3 \delta^3 (\mathbf p - \mathbf q) \ket{0} \neq 0 ~,
        \end{equation*}
        which shows that a state with only timelike photons is unphysical because it does not satisfy the Gupta-Breuler conditions. 
    \end{proof} 

\section{Fock space of photons}

    Now, we investigate the Fock space in which the Gupta-Breuler conditions are valid. We start by making a change of basis, from 
    \begin{equation*}
        \hat a^{\dagger(0)}_{\mathbf p} ~, \quad \hat a^{\dagger(1)}_{\mathbf p} ~, \quad \hat a^{\dagger(2)}_{\mathbf p} ~, \quad \hat a^{\dagger(3)}_{\mathbf p} 
    \end{equation*}
    into 
    \begin{equation*}
        \hat a^{\dagger(1)}_{\mathbf p} ~, \quad \hat a^{\dagger(2)}_{\mathbf p} ~, \quad \hat b_{\pm, \mathbf p} = \hat a^{\dagger(0)}_{\mathbf p} \pm \hat a^{\dagger(3)}_{\mathbf p} ~.
    \end{equation*}
    which inverted looks like 
    \begin{equation*}
        \hat a^{\dagger (0)} = \frac{\hat b^\dagger_{+, \mathbf p} + \hat b^\dagger_{-, \mathbf p} }{2} ~, \quad \hat a^{\dagger (3)} = \frac{\hat b^\dagger_{+, \mathbf p} - \hat b^\dagger_{-, \mathbf p} }{2}  ~.
    \end{equation*}
    This implies that 
    \begin{equation*}
        \hat a^{\dagger(1)}_{\mathbf p} \ket{0} = \ket{\mathbf p, \lambda = 1} ~, \quad \hat a^{\dagger(2)}_{\mathbf p} \ket{0} = \ket{\mathbf p, \lambda = 2} ~, \quad \hat b_{\pm, \mathbf p} \ket{0} = \ket{\mathbf p, \lambda = 0} \pm \ket{\mathbf p, \lambda = 3} ~.
    \end{equation*}
    The last state is a linear combination of one timelike photon and one longitudinal photons.

    Furthermore, the commutation relations becomes 
    \begin{equation*}
        [\hat b_{\mp, \mathbf p}, \hat b_{\mp, \mathbf q}^\dagger] = 0 ~, \quad [\hat b_{-, \mathbf p}, \hat b_{+, \mathbf q}^\dagger] = - 2 (2\pi)^3 \delta^3 (\mathbf p - \mathbf q) ~.
    \end{equation*}
    \begin{proof}
        For the first 
        \begin{equation*}
        \begin{aligned}
            [\hat b_{\mp, \mathbf p}, \hat b_{\mp, \mathbf q}^\dagger] & = [\hat a^{(0)}_{\mathbf p} \mp \hat a^{(3)}_{\mathbf p}, \hat a^{\dagger(0)}_{\mathbf q} \mp \hat a^{\dagger(3)}_{\mathbf q}] \\ & = \underbrace{[\hat a^{(0)}_{\mathbf p} , \hat a^{\dagger(0)}_{\mathbf q}]}_{-(2\pi)^3 \delta^3 (\mathbf p - \mathbf q)} \mp \underbrace{[\hat a^{(0)}_{\mathbf p} , \hat a^{\dagger(3)}_{\mathbf q}] }_0 \mp \underbrace{[\hat a^{(3)}_{\mathbf p}, \hat a^{\dagger(0)}_{\mathbf q}]}_0 + \underbrace{[\hat a^{(3)}_{\mathbf p}, \hat a^{\dagger(3)}_{\mathbf q}]}_{(2\pi)^3 \delta^3 (\mathbf p - \mathbf q)} = 0 ~.
        \end{aligned}
        \end{equation*}
        For the second 
        \begin{equation*}
        \begin{aligned}
            [\hat b_{\mp, \mathbf p}, \hat b_{\pm, \mathbf q}^\dagger] & = [\hat a^{(0)}_{\mathbf p} \mp \hat a^{(3)}_{\mathbf p}, \hat a^{\dagger(0)}_{\mathbf q} \pm \hat a^{\dagger(3)}_{\mathbf q}] \\ & = \underbrace{[\hat a^{(0)}_{\mathbf p} , \hat a^{\dagger(0)}_{\mathbf q}]}_{-(2\pi)^3 \delta^3 (\mathbf p - \mathbf q)} \pm \underbrace{[\hat a^{(0)}_{\mathbf p} , \hat a^{\dagger(3)}_{\mathbf q}] }_0 \mp \underbrace{[\hat a^{(3)}_{\mathbf p}, \hat a^{\dagger(0)}_{\mathbf q}]}_0 - \underbrace{[\hat a^{(3)}_{\mathbf p}, \hat a^{\dagger(3)}_{\mathbf q}]}_{(2\pi)^3 \delta^3 (\mathbf p - \mathbf q)} = - 2 (2\pi)^3 \delta^3 (\mathbf p - \mathbf q)  ~.
        \end{aligned}
        \end{equation*}
    \end{proof}

    Physical Gupta-Breuler conditions~\eqref{GB} for $\ket{\psi}$ becomes 
    \begin{equation*}
        \hat b_{-, \mathbf p} \ket{\psi} = 0 ~.
    \end{equation*} 
    Hence transversal photons $\ket{T}$ are physical, timelike $\ket{S}$ and longitudinal photons $\ket{L}$ are unphysical, the combination with plus of timelike and longitudinal photons $\ket{S} + \ket{L}$ are unphysical, the combination with minus of timelike and longitudinal photons $\ket{S} - \ket{L}$ are physical. However, the latter has zero-norm. To summarise, Fock space contains all states such that it is satisfied~\eqref{GB}, which brings to positive norm states (transversal $\ket{T}$ photons) and zero norm states ($\ket{S} - \ket{L}$ photons).
    \begin{proof}
        For the transverse photons $\ket{T}$
        \begin{equation*}
            \hat b_{-, \mathbf p} \ket{\mathbf q, \lambda= 1,2} = \hat b_{-, \mathbf p} \hat a_{\mathbf q}^{\dagger (1,2)} \ket{0} = 0 ~.
        \end{equation*}
        For the longitudinal photons $\ket{L}$
        \begin{equation*}
        \begin{aligned}
            \hat b_{-, \mathbf p} \ket{\mathbf q, \lambda=3} & = \hat b_{-, \mathbf p} \hat a_{\mathbf q}^{\dagger (3)} \ket{0} = \hat b_{-, \mathbf p} \frac{\hat b^\dagger_{+, \mathbf p} - \hat b^\dagger_{-, \mathbf p}}{2} \ket{0} \\ &  = \frac{1}{2} \underbrace{\hat b_{-, \mathbf p} \hat b^\dagger_{+, \mathbf p}}_{[\hat b_{-, \mathbf p} ,\hat b^\dagger_{+, \mathbf p}] + \hat b^\dagger_{+, \mathbf p} \hat b_{-, \mathbf p} } \ket{0} - \frac{1}{2} \underbrace{\hat b_{-, \mathbf p} \hat b^\dagger_{-, \mathbf p}}_{[\hat b_{-, \mathbf p}, \hat b^\dagger_{-, \mathbf p}] + \hat b^\dagger_{-, \mathbf p} \hat b_{-, \mathbf p}} \ket{0} \\ & = \frac{1}{2} \underbrace{[\hat b_{-, \mathbf p} ,\hat b^\dagger_{+, \mathbf p}]}_{- (2\pi)^3 \delta^3 (\mathbf p - \mathbf q)} \ket{0} + \frac{1}{2} \hat b^\dagger_{+, \mathbf p} \underbrace{\hat b_{-, \mathbf p} \ket{0}}_0 - \frac{1}{2} \underbrace{[\hat b_{-, \mathbf p}, \hat b^\dagger_{-, \mathbf p}]}_{0} \ket{0} - \frac{1}{2} \hat b^\dagger_{-, \mathbf p} \underbrace{\hat b_{-, \mathbf p} \ket{0}}_0 \\ & = - (2\pi)^3 \delta^3 (\mathbf p - \mathbf q) \ket{0} \neq 0 ~.
        \end{aligned}
        \end{equation*}
        For the timelike photons $\ket{S}$
        \begin{equation*}
        \begin{aligned}
            \hat b_{-, \mathbf p} \ket{\mathbf q, \lambda=0} & = \hat b_{-, \mathbf p} \hat a_{\mathbf q}^{\dagger (0)} \ket{0} = \hat b_{-, \mathbf p} \frac{\hat b^\dagger_{+, \mathbf p} + \hat b^\dagger_{-, \mathbf p}}{2} \ket{0} \\ &  = \frac{1}{2} \underbrace{\hat b_{-, \mathbf p} \hat b^\dagger_{+, \mathbf p}}_{[\hat b_{-, \mathbf p} ,\hat b^\dagger_{+, \mathbf p}] + \hat b^\dagger_{+, \mathbf p} \hat b_{-, \mathbf p} } \ket{0} + \frac{1}{2} \underbrace{\hat b_{-, \mathbf p} \hat b^\dagger_{-, \mathbf p}}_{[\hat b_{-, \mathbf p}, \hat b^\dagger_{-, \mathbf p}] + \hat b^\dagger_{-, \mathbf p} \hat b_{-, \mathbf p}} \ket{0} \\ & = \frac{1}{2} \underbrace{[\hat b_{-, \mathbf p} ,\hat b^\dagger_{+, \mathbf p}]}_{- (2\pi)^3 \delta^3 (\mathbf p - \mathbf q)} \ket{0} + \frac{1}{2} \hat b^\dagger_{+, \mathbf p} \underbrace{\hat b_{-, \mathbf p} \ket{0}}_0 + \frac{1}{2} \underbrace{[\hat b_{-, \mathbf p}, \hat b^\dagger_{-, \mathbf p}]}_{0} \ket{0} + \frac{1}{2} \hat b^\dagger_{-, \mathbf p} \underbrace{\hat b_{-, \mathbf p} \ket{0}}_0 \\ & = - (2\pi)^3 \delta^3 (\mathbf p - \mathbf q) \ket{0} \neq 0 ~.
        \end{aligned}
        \end{equation*}
        For the $\ket{S} + \ket{L}$ photons
        \begin{equation*}
        \begin{aligned}
            \hat b_{-, \mathbf p} \ket{\mathbf q, S + L} & = \underbrace{\hat b_{-, \mathbf p} \hat b_{+,\mathbf q}^{\dagger}}_{[\hat b_{-, \mathbf p}, \hat b_{+,\mathbf q}^{\dagger}] + \hat b_{+,\mathbf q}^{\dagger} \hat b_{-, \mathbf p} } \ket{0} \\ & = \underbrace{[\hat b_{-, \mathbf p}, \hat b_{+,\mathbf q}^{\dagger}]}_{- (2\pi)^3 \delta^3 (\mathbf p -\mathbf q)} \ket{0} + \hat b_{+,\mathbf q}^{\dagger} \underbrace{\hat b_{-, \mathbf p} \ket{0}}_0 \\ & = - (2\pi)^3 \delta^3 (\mathbf p -\mathbf q) \ket{0} \neq 0 ~.
        \end{aligned}
        \end{equation*}
        For the $\ket{S} - \ket{L}$ photons
        \begin{equation*}
        \begin{aligned}
            \hat b_{-, \mathbf p} \ket{\mathbf q, S - L} & = \underbrace{\hat b_{-, \mathbf p} \hat b_{-,\mathbf q}^{\dagger}}_{[\hat b_{-, \mathbf p}, \hat b_{-,\mathbf q}^{\dagger}] + \hat b_{-,\mathbf q}^{\dagger} \hat b_{-, \mathbf p} } \ket{0} \\ & = \underbrace{[\hat b_{-, \mathbf p}, \hat b_{-,\mathbf q}^{\dagger}]}_{0} \ket{0} + \hat b_{-,\mathbf q}^{\dagger} \underbrace{\hat b_{-, \mathbf p} \ket{0}}_0 = 0 ~.
        \end{aligned}
        \end{equation*}
    \end{proof}

    Notice that the photons $\ket{S} - \ket{L}$ or $\ket{S} + \ket{L}$ have zero norm. This is true even for $n$ particles state.
    \begin{proof}
        In fact, given
        \begin{equation*}
            \hat b_{-, \mathbf p}^\dagger \ket{0} = \ket{\mathbf p, S - L} ~,
        \end{equation*}
        we have 
        \begin{equation*}
        \begin{aligned}
            \braket{\mathbf p, S-L}{\mathbf p, S-L} & = \bra{0} \hat b_{-, \mathbf p} \hat b_{-, \mathbf p}^\dagger \ket{0} = \bra{0} \underbrace{\hat b_{-, \mathbf p} \hat b_{-, \mathbf p}^\dagger}_{[\hat b_{-, \mathbf p} , \hat b_{-, \mathbf p}^\dagger] + \hat b_{-, \mathbf p}^\dagger \hat b_{-, \mathbf p}} \ket{0} \\ & = \bra{0} \underbrace{[\hat b_{-, \mathbf p} , \hat b_{-, \mathbf p}^\dagger]}_0 \ket{0} + \bra{0} \hat b_{-, \mathbf p}^\dagger \underbrace{\hat b_{-, \mathbf p} \ket{0}}_0 = 0 ~.
        \end{aligned}
        \end{equation*}
        Simirly, given
        \begin{equation*}
            \hat b_{+, \mathbf p}^\dagger \ket{0} = \ket{\mathbf p, S + L} ~,
        \end{equation*}
        we have 
        \begin{equation*}
        \begin{aligned}
            \braket{\mathbf p, S+L}{\mathbf p, S+L} & = \bra{0} \hat b_{+, \mathbf p} \hat b_{+, \mathbf p}^\dagger \ket{0} = \bra{0} \underbrace{\hat b_{+, \mathbf p} \hat b_{+, \mathbf p}^\dagger}_{[\hat b_{+, \mathbf p} , \hat b_{+, \mathbf p}^\dagger] + \hat b_{+, \mathbf p}^\dagger \hat b_{+, \mathbf p}} \ket{0} \\ & = \bra{0} \underbrace{[\hat b_{+, \mathbf p} , \hat b_{+, \mathbf p}^\dagger]}_0 \ket{0} + \bra{0} \hat b_{+, \mathbf p}^\dagger \underbrace{\hat b_{+, \mathbf p} \ket{0}}_0 = 0 ~.
        \end{aligned}
        \end{equation*}
    \end{proof}

    \begin{example}
        Consider a state in which $2$ photons have polarisation $\ket{T}$ and $\ket{S} - \ket{L}$. It has zero norm. In fact, given
        \begin{equation*}
            \hat b_{-, \mathbf p}^\dagger \hat a_{\mathbf q}^{\dagger (1,2)} \ket{0} = \ket{\mathbf q, T; \mathbf p, S-L} ~,
        \end{equation*}
        we have 
        \begin{equation*}
        \begin{aligned}
            & \braket{\mathbf q, T; \mathbf p, S-L}{\mathbf q, T; \mathbf p, S-L} = \bra{0} \hat a_{\mathbf q}^{(1,2)} \hat b_{-, \mathbf p} \hat b_{-, \mathbf p}^\dagger \hat a_{\mathbf q}^{\dagger (1,2)} \ket{0} \\ & = \bra{0} \hat b_{-, \mathbf p} \hat b_{-, \mathbf p}^\dagger \underbrace{\hat a_{\mathbf q}^{\dagger (1,2)} \hat a_{\mathbf q}^{(1,2)}}_{[\hat a_{\mathbf q}^{\dagger (1,2)} , \hat a_{\mathbf q}^{(1,2)}] + \hat a_{\mathbf q}^{(1,2)} \hat a_{\mathbf q}^{\dagger (1,2)}} \ket{0} \\ & = \bra{0} \hat b_{-, \mathbf p} \hat b_{-, \mathbf p}^\dagger \underbrace{[\hat a_{\mathbf q}^{\dagger (1,2)} , \hat a_{\mathbf q}^{(1,2)}] }_{(2\pi)^3 \delta^3 (0) } \ket{0} + \bra{0} \hat b_{-, \mathbf p} \hat b_{-, \mathbf p}^\dagger \hat a_{\mathbf q}^{(1,2)} \underbrace{\hat a_{\mathbf q}^{\dagger (1,2)} \ket{0}}_0 \\ & = (2\pi)^3 \delta^3 (0) \bra{0} \underbrace{\hat b_{-, \mathbf p} \hat b_{-, \mathbf p}^\dagger}_{[\hat b_{-, \mathbf p} , \hat b_{-, \mathbf p}^\dagger] + \hat b_{-, \mathbf p}^\dagger \hat b_{-, \mathbf p}} \ket{0} \\ & = (2\pi)^3 \delta^3 (0) \bra{0} \underbrace{[\hat b_{-, \mathbf p} , \hat b_{-, \mathbf p}^\dagger]}_0 \ket{0} + (2\pi)^3 \delta^3 (0) \bra{0} \hat b_{-, \mathbf p}^\dagger \underbrace{\hat b_{-, \mathbf p} \ket{0}}_0 = 0 ~.
        \end{aligned}
        \end{equation*}
    \end{example}

    Zero-norm states can be ignored. In fact, a state with $n_T$ transversal photons is gauge equivalent to a state with $n_T$ transversal photons and $n$ pairs of timelike and longitudinal photons, for all $n$. This quantum gauge symmetry descends from the classical one. See Figure~\eqref{fig:gauge2}
    
    \begin{figure}[h!]
        \centering
        \begin{tikzpicture}
        \draw[smooth cycle, tension=0.4] plot coordinates{(2,2) (-2.5,0) (3,-2) (6,1)} node at (3,2.3) {Total Fock space};
        \draw[smooth cycle, tension=0.4] plot coordinates { (0.75, 0) (1.25, 1.5) (3.5, 1.5) (4, 0)}  node [label={[label distance=-0.3cm, xshift=-1.5cm, yshift=-0.75cm]: Physical states}] {};

        \draw[] (0, 1) to[bend right=20] (-3,2) node[left] {$\ket{S} + \ket{L}$};
        \filldraw[black] (0, 1) circle (0.05);

        \draw[] (-2, 0) to[bend right=20] (-3,0) node[left] {$\ket{S}$};
        \filldraw[black] (-2, 0) circle (0.05);

        \draw[] (0, -1) to[bend right=20] (-3,-2) node[left] {$\ket{L}$};
        \filldraw[black] (0, -1) circle (0.05);

        \draw[] (2, 0.75) to[bend right=20] (8, 1) node[right] {$\ket{T}$};
        \filldraw[black] (2, 0.75) circle (0.05);

        \draw[] (2.5, 1.1) to[bend right=20] (8, 2) node[right] {$\ket{T} + n (\ket{S} - \ket{L})$};
        \filldraw[black] (2.5,1.1) circle (0.05);

        \draw[] (1,0.25) to[bend right=20] (2, 0.75) to[bend left=20] (3.25, 1.25);

        \end{tikzpicture}
        \caption{Pictorial representation of Fock space of photons.}
        \label{fig:gauge2}
    \end{figure}

    Two states are physically equivalent if the give the same expectation value for all observables. In fact, the Hamiltonian depends only on the number of tranversal photons and the zero-norm $\ket{S} - \ket{L}$ photons are not considered at all. Therefore,given the Hamiltonian
    \begin{equation*}
        \hat H = \int \frac{d^3 p}{(2\pi)^3} |\mathbf p| ( - \hat a_{\mathbf p}^{\dagger (0)} \hat a_{\mathbf p}^{(0)} + \hat a_{\mathbf p}^{\dagger (1)} \hat a_{\mathbf p}^{(1)} + \hat a_{\mathbf p}^{\dagger (2)} \hat a_{\mathbf p}^{(2)} + \hat a_{\mathbf p}^{\dagger (3)} \hat a_{\mathbf p}^{(3)})  ~.
    \end{equation*}
    and given a physical state $\ket{\psi}$ which satisfies~\eqref{GB}, the expectation value of the hamiltonian is given only by the transversal polarisations
    \begin{equation*}
    \begin{aligned}
        \bra{\psi} \hat H \ket{\psi} & = \int \frac{d^3 p}{(2\pi)^3} |\mathbf p| ( - \cancel{\bra{\psi} \hat a_{\mathbf p}^{\dagger (0)} \hat a_{\mathbf p}^{(0)} \ket{\psi}} + \bra{\psi} \hat a_{\mathbf p}^{\dagger (1)} \hat a_{\mathbf p}^{(1)} \ket{\psi} + \bra{\psi} \hat a_{\mathbf p}^{\dagger (2)} \hat a_{\mathbf p}^{(2)} \ket{\psi} + \cancel{\bra{\psi} \hat a_{\mathbf p}^{\dagger (3)} \hat a_{\mathbf p}^{(3)} \ket{\psi}} ) \\ & = \int \frac{d^3 p}{(2\pi)^3} |\mathbf p| ( \bra{\psi} \hat a_{\mathbf p}^{\dagger (1)} \hat a_{\mathbf p}^{(1)} \ket{\psi} + \bra{\psi} \hat a_{\mathbf p}^{\dagger (2)} \hat a_{\mathbf p}^{(2)} \ket{\psi} ) = n_T |\mathbf p| ~.
    \end{aligned}
    \end{equation*}
    \begin{proof}
        The Lagrangian is
        \begin{equation*}
        \begin{aligned}
            \mathcal L & = - \frac{1}{4} F_{\mu\nu} F^{\mu\nu} - \frac{1}{2} \partial_\mu A^\mu \partial_\nu A^\nu = - \frac{1}{4} (\partial_\mu A_\nu - \partial_\nu A_\mu) (\partial^\mu A^\nu - \partial^\nu A^\mu) - \frac{1}{2} \partial_\mu A^\mu \partial_\nu A^\nu \\ & = - \frac{1}{4} \partial_\mu A_\nu \partial^\mu A^\nu + \frac{1}{4} \partial_\mu A_\nu \partial^\nu A^\mu + \frac{1}{4} \partial_\nu A_\mu \partial^\mu A^\nu - \frac{1}{4} \partial_\nu A_\mu \partial^\nu A^\mu - \frac{1}{2} \partial_\mu A^\mu \partial_\nu A^\nu \\ & = - \frac{1}{2} \partial_\mu A_\nu \partial^\mu A^\nu + \frac{1}{2} \underbrace{(\partial_\mu A_\nu \partial^\nu A^\mu - \partial_\mu A^\mu \partial_\nu A^\nu) }_{ - A_\nu  \partial_\mu \partial^\nu A^\mu - A^\mu \partial_\mu \partial_\nu A^\nu + \textnormal{boundary terms}} \\ & = - \frac{1}{2} \partial_\mu A_\nu \partial^\mu A^\nu + \frac{1}{2} (- A_\nu  \partial_\mu \partial^\nu A^\mu - A^\mu \partial_\mu \partial_\nu A^\nu) \\ & = - \frac{1}{2} \partial_\mu A_\nu \partial^\mu A^\nu + \frac{1}{2} (- \cancel{A_\mu  \partial_\nu \partial^\mu A^\nu} + \cancel{A^\mu \partial_\nu \partial_\mu A^\nu}) = - \frac{1}{2} \partial_\mu A_\nu \partial^\mu A^\nu ~,
        \end{aligned}
        \end{equation*}
        where we have integrated by parts since the Lagrangian is always integrated to obtain the action. The conjugate field is 
        \begin{equation*}
            \pi^\mu = \pdv{\mathcal L}{\dot A_\mu} = - \dot A_\mu ~.
        \end{equation*}
        The Hamiltonian is 
        \begin{equation*}
        \begin{aligned}
            \mathcal H & = \pi^\mu \underbrace{\dot A_\mu}_{- \pi_\mu} - \mathcal L = - \pi^\mu \pi_\mu + \frac{1}{2} \partial_\mu A_\nu \partial^\mu A^\nu \\ & = - \pi^\mu \pi_\mu + \frac{1}{2} \underbrace{\dot A^\mu \dot A_\mu}_{\pi^\mu \pi_\mu} + \frac{1}{2} \partial_i A_\mu \partial^i A^\mu = - \frac{1}{2} \pi^\mu \pi_\mu + \frac{1}{2} \partial_i A_\mu \partial^i A^\mu = \mathcal H_1 + \mathcal H_2 ~.
        \end{aligned}
        \end{equation*}
        Now, we promote to operator. 
        \begin{equation*}
            \hat H = \int d^3 x ~ \mathcal H ~.
        \end{equation*}
        The first part is 
        \begin{equation*}
        \begin{aligned}
            & \hat H_1 = - \frac{1}{2} \int d^3 x ~ \hat \pi^\mu \hat \pi_\mu \\ & = - \frac{1}{2} \int d^3 x \int \frac{d^3 p}{(2\pi)^3} i \sqrt{\frac{|\mathbf p|}{2}} \sum_{\lambda=0}^{3} \epsilon^{\mu(\lambda)} (\mathbf p) \Big ( \hat a_{\mathbf p}^{(\lambda)}   \exp(i \mathbf p \cdot \mathbf x) - \hat a_{\mathbf p}^{\dagger (\lambda)}   \exp(- i \mathbf p \cdot \mathbf x) \Big) \\ & \quad \int \frac{d^3 q}{(2\pi)^3} i \sqrt{\frac{|\mathbf q|}{2}} \sum_{\lambda'=0}^{3} \epsilon_{\mu(\lambda')} (\mathbf q)  \Big ( \hat a_{\mathbf q}^{(\lambda')}   \exp(i \mathbf q \cdot \mathbf x) - \hat a_{\mathbf q}^{\dagger (\lambda')}   \exp(- i \mathbf q \cdot \mathbf x) \Big) \\ & = \frac{1}{4} \int \frac{d^3 x ~ d^3 p ~ d^3 q}{(2\pi)^6} \sqrt{|\mathbf p| |\mathbf q|} \sum_{\lambda=0}^{3} \sum_{\lambda'=0}^{3} \underbrace{\epsilon^{\mu(\lambda)} (\mathbf p) \epsilon_{\mu(\lambda')} (\mathbf q)}_{\eta^{\lambda \lambda'}} ( \hat a_{\mathbf p}^{(\lambda)}   \hat a_{\mathbf q}^{(\lambda')}   \underbrace{\exp(i  \mathbf x \cdot (\mathbf p + \mathbf q)) }_{\delta(\mathbf p + \mathbf q)}  \\ & \quad - \hat a_{\mathbf p}^{(\lambda)}  \hat a_{\mathbf q}^{\dagger (\lambda')}   \underbrace{\exp(i  \mathbf x \cdot (\mathbf p - \mathbf q)) }_{\delta(\mathbf p - \mathbf q)} - \hat a_{\mathbf p}^{\dagger (\lambda)}   \hat a_{\mathbf q}^{(\lambda')}   \underbrace{\exp(i  \mathbf x \cdot (- \mathbf p + \mathbf q)) }_{\delta(\mathbf p - \mathbf q)} \\ & \quad + \hat a_{\mathbf p}^{\dagger (\lambda)}   \hat a_{\mathbf q}^{\dagger (\lambda')}   \underbrace{\exp(- i  \mathbf x \cdot (\mathbf p + \mathbf q)) }_{\delta(\mathbf p + \mathbf q)} )
        \end{aligned}
        \end{equation*}
        \begin{equation*}
        \begin{aligned}
            & = \frac{1}{4} \int \frac{d^3 x}{(2\pi)^3} |\mathbf p| \sum_{\lambda=0}^{3} \sum_{\lambda'=0}^{3} \eta^{\lambda \lambda'} ( \hat a_{\mathbf p}^{(\lambda)} \hat a_{- \mathbf p}^{(\lambda')}  - \hat a_{\mathbf p}^{(\lambda)} \hat a_{\mathbf p}^{\dagger (\lambda')}  - \hat a_{\mathbf p}^{\dagger (\lambda)} \hat a_{\mathbf p}^{(\lambda')} + \hat a_{\mathbf p}^{\dagger (\lambda)} \hat a_{- \mathbf p}^{\dagger (\lambda')} ) ~.
        \end{aligned}
        \end{equation*}
        Given 
        \begin{equation*}
            \partial_i \hat A_\mu = \int \frac{d^3 p}{(2\pi)^3} \frac{1}{\sqrt{2 |\mathbf p|}} \sum_{\lambda=0}^{3} \epsilon_\mu^{(\lambda)} (\mathbf p) \Big ( (- i p_i )\hat a_{\mathbf p}^{(\lambda)} \exp(i \mathbf p \cdot \mathbf x) + (i p_i)\hat a_{\mathbf p}^{\dagger (\lambda)}  \exp(- i \mathbf p \cdot \mathbf x) \Big) ~,
        \end{equation*}
        the second part is
        \begin{equation*}
        \begin{aligned}
            & \hat H_2 = \frac{1}{2} \int d^3 x ~ \partial_i \hat A_\mu \partial^i \hat A^\mu \\ & = \frac{1}{2} \int d^3 x \int \frac{d^3 p}{(2\pi)^3} \frac{1}{\sqrt{2 |\mathbf p|}} \sum_{\lambda=0}^{3} \epsilon_\mu^{(\lambda)} (\mathbf p) \Big ( (- i p_i ) \hat a_{\mathbf p}^{(\lambda)} \exp(i \mathbf p \cdot \mathbf x) + (i p_i)\hat a_{\mathbf p}^{\dagger (\lambda)}  \exp(- i \mathbf p \cdot \mathbf x) \Big) \\ & \quad \int \frac{d^3 q}{(2\pi)^3} \frac{1}{\sqrt{2 |\mathbf q}} \sum_{\lambda'=0}^{3} \epsilon^{\mu(\lambda')} (\mathbf q) \Big ( (- i q^i )\hat a_{\mathbf q}^{(\lambda')} \exp(i \mathbf q \cdot \mathbf x) + (i q^i)\hat a_{\mathbf q}^{\dagger (\lambda')} \exp(- i \mathbf q \cdot \mathbf x) \Big) \\ & = \frac{1}{2} \int \frac{d^3 x ~ d^3 p ~ d^3 q}{(2\pi)^6} \frac{1}{2\sqrt{|\mathbf p| |\mathbf q|}} \sum_{\lambda=0}^{3} \sum_{\lambda'=0}^{3} \underbrace{\epsilon^{\mu(\lambda)} (\mathbf p) \epsilon_{\mu}^{(\lambda')} (\mathbf q)}_{\eta^{\lambda \lambda'}} p_i q^i ( \hat a_{\mathbf p}^{(\lambda)} \hat a_{\mathbf q}^{(\lambda)} \underbrace{\exp(i  \mathbf x \cdot (\mathbf p + \mathbf q)) }_{\delta(\mathbf p + \mathbf q)} \\ & \quad - \hat a_{\mathbf p}^{(\lambda)}\hat a_{\mathbf q}^{\dagger (\lambda)} \underbrace{\exp(i \mathbf x \cdot (\mathbf p - \mathbf q)) }_{\delta(\mathbf p - \mathbf q)} - \hat a_{\mathbf p}^{\dagger (\lambda)} \hat a_{\mathbf q}^{(\lambda)} \underbrace{\exp( i  \mathbf x \cdot ( -\mathbf p + \mathbf q)) }_{\delta(\mathbf p - \mathbf q)} \\ & \quad + \hat a_{\mathbf p}^{\dagger (\lambda)} \hat a_{\mathbf q}^{\dagger (\lambda)} \underbrace{\exp(- i  \mathbf x \cdot (\mathbf p + \mathbf q)) }_{\delta(\mathbf p + \mathbf q)})
        \end{aligned}
        \end{equation*}
        \begin{equation*}
        \begin{aligned}
            & = \frac{1}{2} \int \frac{d^3 p}{(2\pi)^3} \frac{1}{2 |\mathbf p|} \sum_{\lambda=0}^{3} \sum_{\lambda'=0}^{3} \eta^{\lambda \lambda'} |\mathbf p|^2 (- \hat a_{\mathbf p}^{(\lambda)} \hat a_{- \mathbf p}^{(\lambda')} - \hat a_{\mathbf p}^{(\lambda)} \hat a_{\mathbf p}^{\dagger (\lambda')} - \hat a_{\mathbf p}^{\dagger (\lambda)} \hat a_{\mathbf p}^{(\lambda')} - \hat a_{\mathbf p}^{\dagger (\lambda)} \hat a_{-\mathbf p}^{\dagger (\lambda')} ) \\ & = \frac{1}{4} \int \frac{d^3 p}{(2\pi)^3} |\mathbf p| \sum_{\lambda=0}^{3} \sum_{\lambda'=0}^{3} \eta^{\lambda \lambda'} (- \hat a_{\mathbf p}^{(\lambda)} \hat a_{- \mathbf p}^{(\lambda')} - \hat a_{\mathbf p}^{(\lambda)} \hat a_{\mathbf p}^{\dagger (\lambda')} - \hat a_{\mathbf p}^{\dagger (\lambda)} \hat a_{\mathbf p}^{(\lambda')} - \hat a_{\mathbf p}^{\dagger (\lambda)} \hat a_{-\mathbf p}^{\dagger (\lambda')} ) ~.
        \end{aligned}
        \end{equation*}
        Putting everything together 
        \begin{equation*}
        \begin{aligned}
            \hat H & = \frac{1}{4} \int \frac{d^3 p}{(2\pi)^3} |\mathbf p| \sum_{\lambda=0}^{3} \sum_{\lambda'=0}^{3} \eta^{\lambda \lambda'} (- \cancel{\hat a_{\mathbf p}^{(\lambda)} \hat a_{- \mathbf p}^{(\lambda')}} - \hat a_{\mathbf p}^{(\lambda)} \hat a_{\mathbf p}^{\dagger (\lambda')} - \hat a_{\mathbf p}^{\dagger (\lambda)} \hat a_{\mathbf p}^{(\lambda')} - \cancel{\hat a_{\mathbf p}^{\dagger (\lambda)} \hat a_{-\mathbf p}^{\dagger (\lambda')}} ) \\ & \quad + \frac{1}{4} \int \frac{d^3 x}{(2\pi)^3} |\mathbf p| \sum_{\lambda=0}^{3} \sum_{\lambda'=0}^{3} \eta^{\lambda \lambda'} ( \cancel{\hat a_{\mathbf p}^{(\lambda)} \hat a_{- \mathbf p}^{(\lambda')}}  - \hat a_{\mathbf p}^{(\lambda)} \hat a_{\mathbf p}^{\dagger (\lambda')}  - \hat a_{\mathbf p}^{\dagger (\lambda)} \hat a_{\mathbf p}^{(\lambda')} + \cancel{\hat a_{\mathbf p}^{\dagger (\lambda)} \hat a_{- \mathbf p}^{\dagger (\lambda')}} ) \\ & = \frac{1}{2} \int \frac{d^3 p}{(2\pi)^3} |\mathbf p| \sum_{\lambda=0}^{3} \sum_{\lambda'=0}^{3} \eta^{\lambda \lambda'} (- \hat a_{\mathbf p}^{(\lambda)} \hat a_{\mathbf p}^{\dagger (\lambda')} - \hat a_{\mathbf p}^{\dagger (\lambda)} \hat a_{\mathbf p}^{(\lambda')})
        \end{aligned}
        \end{equation*}
        which in normal ordering is 
        \begin{equation*}
        \begin{aligned}
            \hat H & = \int \frac{d^3 p}{(2\pi)^3} |\mathbf p| \sum_{\lambda=0}^{3} \sum_{\lambda'=0}^{3} \eta^{\lambda \lambda'} (- \hat a_{\mathbf p}^{\dagger (\lambda)} \hat a_{\mathbf p}^{(\lambda')}) \\ & = \int \frac{d^3 p}{(2\pi)^3} |\mathbf p| \sum_{\lambda=0}^{3} (- \hat a_{\mathbf p}^{\dagger (0)} \hat a_{\mathbf p}^{(0)} + \hat a_{\mathbf p}^{\dagger (1)} \hat a_{\mathbf p}^{(1)} + \hat a_{\mathbf p}^{\dagger (2)} \hat a_{\mathbf p}^{(2)}+ \hat a_{\mathbf p}^{\dagger (3)} \hat a_{\mathbf p}^{(3)}) ~.
        \end{aligned}
        \end{equation*}
    \end{proof}

    To summarise, any observables gives only results in terms of transversal photons and gauge equivalent states are all physically equivalent, since they are indistinguishable in measurements of observables. The only truly physical degrees of freedome of a massless spin-$1$ particle described by the Maxwell's field are the two transversal polarisations. However, this is not true for massive photons, since the longitudinal polararisation is the thire degree of freedom.

\section{Massive photons}    

    Massive photons are described by the Proca lagrangian
    \begin{equation*}
        \mathcal L = - \frac{1}{4} F^{\mu\nu} F_{\mu\nu} - \frac{m^2}{2} A_\mu A^\mu ~,
    \end{equation*}
    where the equations of motion are 
    \begin{equation*}
        \partial_\mu F^{\mu\nu} + m^2 A^\nu = 0~.
    \end{equation*}
    Notice that the Lorenz gauge is automatically always satisfied and it is not imposed by hand.
    \begin{proof}
        In fact, we have
        \begin{equation}
            0 = \underbrace{\partial_\mu \partial_\nu}_{\text{symm}} \underbrace{F^{\mu\nu}}_{\text{anti}} + m^2 \partial_\nu A^\nu = m^2 \partial_\nu A^\nu ~.
        \end{equation}
    \end{proof}
    The equations of motion become
    \begin{equation*}
        (\Box + m^2) A^\mu (x) = 0 ~.
    \end{equation*}
    \begin{proof}
        In fact, we have
        \begin{equation*}
            0 = \partial_\mu F^{\mu\nu} + m^2 A^\nu = \partial_\mu \partial^\mu A^\nu - \cancel{\partial_\mu \partial^\nu A^\nu} + m^2 A^\nu = (\Box + m^2) A^\nu ~.
        \end{equation*}
    \end{proof}
    Therefore, each $A^\mu$ satisfies the Klein-Gordon equation but there are only $3$ independent degrees of freedom by the Lorenz gauge. Finally, notice that the mass term breaks the gauge symmetry, since for a gauge transformation ${A'}_\mu = A_\mu + \partial_\mu \lambda$
    \begin{equation*}
        \frac{m}{2} {A'}^\mu {A'}_\mu \neq \frac{m}{2} A^\mu A_\mu ~.
    \end{equation*}


\backmatter
\nocite{qft1lecture}

\clearpage
\phantomsection
\printbibliography

\end{document}
