\documentclass[a4paper, 12pt]{memoir}

\usepackage[a4paper, top = 4cm, bottom = 4cm, left = 3cm, right = 3cm]{geometry}

\usepackage[T1]{fontenc}
\usepackage[utf8]{inputenc}
\usepackage{pythontex} 
\usepackage{nopageno} 
\usepackage{pgf}

\usepackage{tocloft}
\newcommand{\listequationsname}{List of Equations}
\newlistof{listofequations}{equ}{\listequationsname}
\newcommand{\myequation}[1]{%
	\addcontentsline{equ}{equation}{\protect\numberline{\theequation}#1}\par
}
\makeatletter
\let\l@equation\l@figure
\makeatother

\usepackage{xcolor}
\xdefinecolor{mycolor}{RGB}{0,175,179} 
\usepackage{hyperref}
\hypersetup{colorlinks, linkcolor={mycolor}, citecolor={mycolor}, urlcolor={mycolor}}

\usepackage{lipsum}

\renewcommand{\aftertoctitle}{\afterchaptertitle\par\nobreak\hfill{\normalfont{Page}}\par\nobreak}

\usepackage{titlesec}
\titleformat{\part}[display]
  {\normalfont\HUGE\bfseries\color{mycolor}\centering}
  {Part \thepart}{20pt}{\HUGE\normalfont\color{black}}
\titleformat{\chapter}[display]
  {\normalfont\HUGE\bfseries\color{mycolor}\centering}
  {Chapter \thechapter}{20pt}{\HUGE\normalfont\color{black}}
\titleformat{\section}
  {\normalfont\Large\bfseries\color{mycolor}\centering}
  {\thesection}{1em}{}
\titleformat{\subsection}
  {\normalfont\large\bfseries\color{mycolor}\centering}
  {\thesubsection}{1em}{}

\renewcommand{\printtoctitle}[1]{\HUGE\normalfont\color{black}#1}

\usepackage[backend=bibtex, sorting=none]{biblatex}
\addbibresource{../bibliography.bib}

\usepackage{amsmath}
\usepackage{amsthm}
\usepackage{thmtools}
\usepackage{mathtools}

\newtheorem{principle}{Principle}[chapter]
\newtheorem{lemma}{Lemma}[chapter]
\theoremstyle{definition}
\newtheorem{example}{Example}[chapter]
\newtheorem{exercise}{Exercise}[chapter]
\renewcommand\qedsymbol{q.e.d.}

\theoremstyle{remark}
\newtheorem{case}{Case}

\newcommand{\dv}[2]{\frac{d#1}{d#2}}
\newcommand{\cdv}[2]{\frac{D#1}{D#2}}
\newcommand{\dvin}[3]{\frac{d#1}{d#2}\Big\vert_{#3}}
\newcommand{\dvd}[2]{\frac{d^2#1}{d#2^2}}
\newcommand{\dvf}[2]{\frac{\delta #1}{\delta #2}}
\newcommand{\pdv}[2]{\frac{\partial#1}{\partial#2}}
\newcommand{\pdvd}[3]{\frac{\partial^2 #1}{\partial#2 \partial#3}}
\newcommand{\pdvdu}[2]{\frac{\partial^2 #1}{\partial#2^2}}
\newcommand{\integ}[3]{\int_{#1}^{#2}d#3~}
\newcommand{\poi}[2]{[#1,~#2]}
\newcommand{\poiexp}[2]{\pdv{#1}{q^i} \pdv{#2}{p_i} - \pdv{#2}{q^i} \pdv{#1}{p_i}}

\newcommand{\comm}[2]{[#1,~#2]}
\newcommand{\set}[2]{\{#1\colon#2\}}
\newcommand{\inner}[2]{\langle#1,~#2\rangle}
\newcommand{\av}[1]{\langle#1\rangle}
\newcommand{\avp}[2]{\langle#1\rangle_{#2}}
\newcommand{\ket}[1]{\vert#1\rangle}
\newcommand{\bra}[1]{\langle#1\vert}
\newcommand{\braket}[2]{\langle#1\vert#2\rangle}

\newtheoremstyle{colored}{}{}{\itshape}{}{\color{mycolor}\normalfont\bfseries\indent}{}{\newline}{}

\declaretheorem[
  style=colored,
  name=Definition,
  numberwithin=chapter,
]{definition}

\declaretheorem[
  style=colored,
  name=Theorem,
  numberwithin=chapter,
]{theorem}

\declaretheorem[
  style=colored,
  name=Corollary,
  numberwithin=chapter,
]{corollary}

\declaretheorem[
  style=colored,
  name=Law,
  numberwithin=chapter,
]{law}

\declaretheorem[
  style=colored,
  name=Principle,
  numberwithin=chapter,
]{princ}

\usepackage{amsfonts}
\usepackage{dsfont}
\usepackage{yfonts}
\usepackage{amssymb}

\let\oldproof\proof
\renewcommand{\proof}{\color{darkgray}\oldproof}

\let\oldexample\example
\renewcommand{\example}{\color{darkgray}\oldexample}

\let\oldexercise\exercise
\renewcommand{\exercise}{\color{darkgray}\oldexercise}

\usepackage{cancel}
\usepackage{indentfirst}

\usepackage{tikz}
\usepackage{amssymb}
\usepackage{pgfplots}
\usepgfplotslibrary{patchplots}
\usetikzlibrary{patterns, positioning, arrows}
\pgfplotsset{compat=1.15}

\DeclareMathOperator{\tr}{tr}
\DeclareMathOperator{\str}{str}
\DeclareMathOperator{\real}{Re}
\DeclareMathOperator{\imm}{Im}
\DeclareMathOperator{\sgn}{sgn}
\DeclareMathOperator{\spann}{span}
\DeclareMathOperator{\vol}{vol}
\DeclareMathOperator{\erf}{erf}



\def\blankpage{%
      \clearpage%
      \thispagestyle{empty}%
      \addtocounter{page}{-1}%
      \null%
      \clearpage}

\usepackage{tikz-feynman}
\usepackage{feynmp}
\tikzfeynmanset{compat=1.1.0}
\DeclareGraphicsRule{*}{mps}{*}{}


\title{quantum field theory I}
\date{\today}

\newcommand{\subt}{second quantisation of free theories}

\begin{document}

\frontmatter

\pagestyle{empty}
{\raggedleft\vspace*{\baselineskip}
{\LARGE Matteo Zandi}\\[0.35\textheight]
{\HUGE \textcolor{mycolor}{\textbf{On~\thetitle:}}}\\[\baselineskip]
{\LARGE \subt }\\[\baselineskip]
{\large \thedate}\par
\vspace*{2\baselineskip}
\vfill
{\large matteo.zandi2@studio.unibo.it}\par
\vspace*{\baselineskip}}
\clearpage
\pagestyle{headings}

\blankpage

\tableofcontents

\mainmatter

\begin{pycode}
import sympy as sy
def plot1(x, f, rangex, rangey, fig, leg, negx, negy):
    rangexx = rangex
    rangeyy = rangey
    if negx == True:
        rangexx = 0
    if negy == True:
        rangeyy = 0
    x = sy.Symbol('x')
    p = sy.plot((f, (x, -rangexx, rangex)), ylim=[-rangeyy, rangey], legend= leg, show=False, line_color='#00AFB3')
    p.save(f'fig/fig{fig}.pgf')
    print(r'\input{fig/fig'+ rf'{fig}' + r'.pgf}')

def plot2(x, f, g, rangex, rangey, fig, leg, negx, negy):
    rangexx = rangex
    rangeyy = rangey
    if negx == True:
        rangexx = 0
    if negy == True:
        rangeyy = 0
    x = sy.Symbol('x')
    p = sy.plot((f, (x, -rangexx, rangex)), (g, (x, -rangexx, rangex)), ylim=[-rangeyy, rangey], legend= leg, show=False, line_color='#00AFB3')
    p[0].line_color='red'
    p[1].line_color='blue'
    p.save(f'fig/fig{fig}.pgf')
    print(r'\input{fig/fig'+ rf'{fig}' + r'.pgf}')

def plot3(x, f, g, h, rangex, rangey, fig, leg, negx, negy):
    rangexx = rangex
    rangeyy = rangey
    if negx == True:
        rangexx = 0
    if negy == True:
        rangeyy = 0
    x = sy.Symbol('x')
    p = sy.plot((f, (x, -rangexx, rangex)), (g, (x, -rangexx, rangex)), (h, (x, -rangexx, rangex)), ylim=[-rangeyy, rangey], legend= leg, show=False, line_color='#00AFB3')
    p[0].line_color='red'
    p[1].line_color='violet'
    p[2].line_color='blue'
    p.save(f'fig/fig{fig}.pgf')
    print(r'\input{fig/fig'+ rf'{fig}' + r'.pgf}')

def plot4(x, f, g, h, l, rangex, rangey, fig, leg, negx, negy):
    rangexx = rangex
    rangeyy = rangey
    if negx == True:
        rangexx = 0
    if negy == True:
        rangeyy = 0
    x = sy.Symbol('x')
    p = sy.plot((f, (x, -rangexx, rangex)), (g, (x, -rangex, rangex)), (h, (x, -rangex, rangex)), (l, (x, -rangex, rangex)), ylim=[-rangeyy, rangey], legend= leg, show=False, line_color='#00AFB3')
    p[3].line_color='black'
    p.save(f'fig/fig{fig}.pgf')
    print(r'\input{fig/fig'+ rf'{fig}' + r'.pgf}')

def der(y, x):
    x = sy.Symbol(x) 
    derivative = sy.diff(y, x)
    return sy.latex(derivative) 
 
def indint(integrand, x): 
    x = sy.Symbol(x) 
    integral = sy.integrate(integrand,x) 
    return sy.latex(integral) 

def defint(integrand, x, min, max): 
    x = sy.Symbol(x) 
    integral = sy.integrate(integrand, (x, min, max)) 
    return sy.latex(integral) 

def infint(integrand, x): 
    x = sy.Symbol(x) 
    integral = sy.integrate(integrand, (x, float('-inf'), float('inf'))) 
    return sy.latex(integral) 

def infzint(integrand, x): 
    x = sy.Symbol(x) 
    integral = sy.integrate(integrand, (x, 0, float('inf'))) 
    return sy.latex(integral) 

def ode(ode, y, x): 
    x = sy.Symbol(x) 
    y = sy.Function(y) 
    lhs, rhs = ode.split('=') 
    ode = sy.Eq(sy.S(lhs),sy.S(rhs)) 
    sol = sy.dsolve(ode,y(x)) 
    return sy.latex(sol) 

# \py{ode("Derivative(y(x),x,x) + y(x) = 0", "y", "x")} ~.
 
def odeic(ode, y, x, ic): 
    x  = sy.Symbol(x) 
    y  = sy.Function(y) 
    lhs,rhs = ode.split('=') 
    ode = sy.Eq(sy.S(lhs),sy.S(rhs)) 
    sol = sy.dsolve(ode,y(x), ics= sy.S(ic)) 
    return sy.latex(sol) 

#\py{odeic("Derivative(y(x),x,x) + y(x) = 0", "y", "x", "{y(0):1, y(x).diff(x).subs(x, 0): 0}")} ~.

def matrixmult(A, B):
    C = A*B
    return sy.latex(C)

def Taylor(x, f, point, order):
    x = sy.Symbol('x')
    ts = sy.series(f, x, point, order) 
    return sy.latex(ts)

def limit(x, f, point):
    x = sy.Symbol('x')
    lim = sy.limit(f, x, point) 
    return sy.latex(lim)

\end{pycode}

\chapter*{Abstract}

    In these notes, we will study the mathematical framework of quantum field of free theory without interactions. Quantum fields arise out from second quantisation of classical theories, like Klein-Gordon, Dirac or Maxwell. In the first part, we will introduce the passage from standard quantum mechanics to quantum field theory, with the example of the mechanical model of a string. In the second part, we will review classical field theories: Euler-Lagrange equations and Noether's theorem. In the third part, we will quantise those free theory, focusing on field operators, normal ordering, Noether's charges of quantum fields describing particles with spin $0$, $1/2$ and $1$.
    
\input{qft10.tex}
\part{Groups}

\chapter{Groups}

    The aim of this chapter is to reconcile Quantum Mechanics with Special Relativity. The mathematical language of spacetime symmetries is the Lorentz or Poincarè group and its infinite-dimensional representation is the Hilbert space of a quantum particle.

\section{}

    
\part{Klein-Gordon theory}

\chapter{Canonical or second quantisation}

    In Schoedinger picture, where states evolve in time while operators do not, recall that standard quantisation from classical mechanics to quantum mechanics works in this way: 
    \begin{enumerate}
        \item hamiltonian formalism $H \mapsto$ hamiltonian operator $\hat H$~,
        \item generalised coordinates and conjugate momenta $(q_i, p^i = \pdv{L}{\dot q_i}) \mapsto$ operators on a Hilbert space $\hat q_i$ and $\hat p^i$~,
        \item Poissons brackets $\{q_i, p^j\} = \delta_i^{\phantom i j}$ and $\{p^i, p^j\} = \{q_i, q_j\} = 0 \mapsto$ commutators $[q_i, p^j] = i \delta_i^{\phantom i j}$ and $[p^i, p^j] = [q_i, q_j] = 0$~.
    \end{enumerate}

    Similarly, the second quantisation from classical field theory to quantum field theory works in this way:
    \begin{enumerate}
        \item fields and conjugate fields $(\varphi_i(t, \mathbf x), \pi^i (t, \mathbf x) = \pdv{\mathcal L}{\dot \varphi_i}) \mapsto$ operators on a Fock space $\hat \varphi_i(t, \mathbf x)$ and $\hat \pi^i (t, \mathbf x)$~,
        \item canonical commutation relations $[\hat \varphi_i(t, \mathbf x), \hat \pi^j (t, \mathbf y)] = i \delta_i^{\phantom i j} \delta^3(\mathbf x - \mathbf y)$ and $[\hat \varphi_i(t, \mathbf x), \hat \varphi_j(t, \mathbf y)] = [\hat \pi^i (t, \mathbf x), \hat \pi^j (t, \mathbf y)] = 0$~.
    \end{enumerate}

    States which live in the Fock state $\ket{\psi}$ evolve in time via the Schoedinger equation 
    \begin{equation*}
        i \pdv{}{t} \ket{\psi} = \hat H \ket{\psi} 
    \end{equation*}
    where $\ket{\psi}$ is a wave functional such that its modulus square gives the density probability to find the field in a certain configuration and $\hat H (\varphi_i(t, \mathbf x), \pi^i (t, \mathbf x))$ is an operator, since $\varphi_i(t, \mathbf x)$ and $\pi^i (t, \mathbf x)$ are.

    In order to solve the theory, we need to find the eigenstates of $\hat H$, but it is too difficult expect in the case of a free theory, which the lagrangian is quadratic and the equations od motion are linear and solvable.

\section{Harmonic oscillator}

    Recall some feature of the harmonic oscillator.

\section{Dirac delta}

    Recall that the integral representation of the Dirac delta is 
    \begin{equation}\label{deltaint}
        \delta^3 (\mathbf x - \mathbf y) = \int \frac{d^3 p}{(2\pi)^3} \exp(i \mathbf p \cdot (\mathbf x - \mathbf y)) = \int \frac{d^3 p}{(2\pi)^3} \exp(- i \mathbf p \cdot (\mathbf x - \mathbf y)) ~.
    \end{equation}

\chapter{Single real Klein-Gordon field}

\section{Hamiltonian}

    The simplest relativistic field theory is the Klein-Gordon theory of a single real scalar field chargeless and spinless. Its lagrangian is 
    \begin{equation*}
        \mathcal L = \frac{1}{2} \partial_\mu \varphi \partial^\mu \varphi - \frac{1}{2} m^2 \varphi^2 
    \end{equation*}
    and its equations of motion are 
    \begin{equation}\label{kgeq}
        (\Box + m^2) \varphi(x) = 0 ~.
    \end{equation}
    \begin{proof}
        Infact, using~\eqref{eleq} 
        \begin{equation*}
        \begin{aligned}
            0 & = \pdv{\mathcal L}{\varphi} - \partial_\mu \pdv{\mathcal L}{\partial_\mu \varphi} \\ & = \pdv{}{\varphi} \Big ( \cancel{\frac{1}{2} \partial_\mu \varphi \partial^\mu \varphi} - \underbrace{\frac{1}{2} m^2 \varphi^2}_{m^2 \varphi} \Big) + \partial_\mu \pdv{}{\partial_\mu \varphi} \Big ( \underbrace{\frac{1}{2} \partial_\mu \varphi \partial^\mu \varphi}_{\partial_\mu \partial^\mu \varphi} - \cancel{\frac{1}{2} m^2 \varphi^2} \Big) \\ & = \underbrace{\partial_\mu \partial^\mu}_\Box \varphi + m^2 \varphi \\ & = (\Box + m^2) \varphi ~.
        \end{aligned}
        \end{equation*}
    \end{proof}

    It is a system of infinitely many degrees of freedom and to decouple them we need to perform a Fourier transform 
    \begin{equation}\label{fourkg}
        \varphi (t, \mathbf x) = \int \frac{d^3 p}{(2\pi)^3} \exp(i \mathbf p \cdot \mathbf x) \tilde \varphi(t, \mathbf p) ~,
    \end{equation}
    which in momentum space becomes 
    \begin{equation*}
        \Big ( \pdvdu{}{t} + |\mathbf p|^2 + m^2 \Big) \tilde \varphi(t, \mathbf x) = 0
    \end{equation*}
    and its solution is an harmonic oscillator for each $\mathbf p$ of frequency 
    \begin{equation}\label{kgenergy}
        \omega_{\mathbf p} = \sqrt{|\mathbf p|^2 + m^2}~.
    \end{equation}
    Hence, the most general solution of the Klein-Gordon equation~\eqref{kgeq} is a superposition of simple harmonic oscillators, each vibrating with different frequency and amplitude. To quantise the theory and $\varphi$, we need to quantise this set of infinitely decoupled harmonic oscillators.
    \begin{proof}
        We decompose~\eqref{kgeq} into time and space components
        \begin{equation*}
            0 = (\Box + m^2) \varphi = (\underbrace{\partial_0}_{\partial^0} \partial^0 + \underbrace{\partial_i}_{-\partial^i} \partial^i + m^2) \varphi = ((\partial^0)^2 - (\partial^i)^2 + m^2) \varphi = (\pdvdu{}{t} - \nabla^2 + m^2) \varphi ~,
        \end{equation*}
        and we substitute~\eqref{fourkg}
        \begin{equation*}
        \begin{aligned}
            0 & = (\pdvdu{}{t} - \nabla^2 + m^2) \int \frac{d^3 p}{(2\pi)^3} \exp(i \mathbf p \cdot \mathbf x) \tilde \varphi(t, \mathbf p) \\ & = \int \frac{d^3 p}{(2\pi)^3} (\pdvdu{}{t} - \underbrace{\nabla^2}_{- i^2 |\mathbf p|^2} + m^2) (\exp(i \mathbf p \cdot \mathbf x) \tilde \varphi(t, \mathbf p)) \\ & = \int \frac{d^3 p}{(2\pi)^3} (\pdvdu{}{t} - i^2 |\mathbf p|^2 + m^2) \exp(i \mathbf p \cdot \mathbf x) \tilde \varphi(t, \mathbf p) \\ & = \int \frac{d^3 p}{(2\pi)^3} (\pdvdu{}{t} + |\mathbf p|^2 + m^2) \exp(i \mathbf p \cdot \mathbf x) \tilde \varphi(t, \mathbf p) ~,
        \end{aligned}
        \end{equation*}
        where the integrand vanishes with the exponential. Finally, we define the energy~\eqref{kgenergy} and we obtain 
        \begin{equation*}
            (\pdvdu{}{t} + \omega_{\mathbf p})^2 \tilde \varphi(t, \mathbf p) = 0 ~,
        \end{equation*} 
        which is indeed the equation of an harmonic oscillator in the form $\ddot x + \omega^2 x = 0$.
    \end{proof}

    By analogy with the simple quantum harmonic oscillator, we define the field operator 
    \begin{equation}\label{kgfop}
        \hat \varphi (\mathbf x) = \int \frac{d^3 p}{{(2\pi)}^3} \frac{1}{\sqrt{2 \omega_{\mathbf p}}} \Big (\hat a_{\mathbf p} \exp(i \mathbf p \cdot \mathbf x) + \hat a_{\mathbf p}^\dagger \exp(- i \mathbf p \cdot \mathbf x) \Big)
    \end{equation}
    and the conjugate operator
    \begin{equation}\label{kgpop}
        \hat \pi (\mathbf x) = \int \frac{d^3 p}{{(2\pi)}^3} \Big (- i\sqrt{\frac{\omega_{\mathbf p}}{2}} \Big ) \Big (\hat a_{\mathbf p} \exp(i \mathbf p \cdot \mathbf x) - \hat a_{\mathbf p}^\dagger \exp(- i \mathbf p \cdot \mathbf x) \Big) ~,
    \end{equation}
    such that they satisfies the commutation relations for annihilation and creation operators
    \begin{equation}\label{anncrea}
        [\hat a_{\mathbf p}, \hat a_{\mathbf q}] = [\hat a_{\mathbf p}^\dagger, \hat a_{\mathbf q}^\dagger] = 0 ~, \quad [\hat a_{\mathbf p}, \hat a_{\mathbf q}^\dagger] = (2\pi)^3 \delta^3 (\mathbf p - \mathbf q) ~.
    \end{equation}
    Therefore, the canonical commutation relations become 
    \begin{equation*}
        [\hat \varphi(\mathbf x), \hat \varphi (\mathbf y)] = [\hat \pi(\mathbf x), \hat \pi (\mathbf y)]  = 0
    \end{equation*}
    and 
    \begin{equation*}
        [\hat \varphi(\mathbf x), \hat \pi (\mathbf y)] = i \delta^3 (\mathbf x - \mathbf y) ~.
    \end{equation*}
    \begin{proof}
        For the field-field commutator, using~\eqref{anncrea},~\eqref{kgfop} and~\eqref{deltaint}
        \begin{equation*}
        \begin{aligned}
            [\hat \varphi(\mathbf x), \hat \varphi (\mathbf y)] & = [\int \frac{d^3 p}{{(2\pi)}^3} \frac{1}{\sqrt{2 \omega_{\mathbf p}}} \Big (\hat a_{\mathbf p} \exp(i \mathbf p \cdot \mathbf x) + \hat a_{\mathbf p}^\dagger \exp(- i \mathbf p \cdot \mathbf x) \Big), \\ & \qquad \int \frac{d^3 q}{{(2\pi)}^3} \frac{1}{\sqrt{2 \omega_{\mathbf q}}} \Big (\hat a_{\mathbf q} \exp(i \mathbf q \cdot \mathbf y) + \hat a_{\mathbf q}^\dagger \exp(- i \mathbf q \cdot \mathbf y) \Big)] \\ &  = \int \frac{d^3 p ~ d^3 q}{{(2\pi)}^6} \frac{1}{2 \sqrt{\omega_{\mathbf p}} \omega_{\mathbf q}} [\hat a_{\mathbf p} \exp(i \mathbf p \cdot \mathbf x) + \hat a_{\mathbf p}^\dagger \exp(- i \mathbf p \cdot \mathbf x), \\ & \qquad \hat a_{\mathbf q} \exp(i \mathbf q \cdot \mathbf y) + \hat a_{\mathbf q}^\dagger \exp(- i \mathbf q \cdot \mathbf y)] \\ & = \int \frac{d^3 p ~ d^3 q}{{(2\pi)}^6} \frac{1}{2 \sqrt{\omega_{\mathbf p}} \omega_{\mathbf q}} \Big ( \underbrace{[\hat a_{\mathbf p}, \hat a_{\mathbf q}]}_0 \exp(i (\mathbf p \cdot \mathbf x + \mathbf q \cdot \mathbf y)) + \underbrace{[\hat a_{\mathbf p}, \hat a_{\mathbf q}^\dagger]}_{(2\pi)^3 \delta^3 (\mathbf p - \mathbf q)} \exp(i (\mathbf p \cdot \mathbf x - \mathbf q \cdot \mathbf y)) \\ & \qquad + \underbrace{[\hat a_{\mathbf p}^\dagger, \hat a_{\mathbf q}]}_{- (2\pi)^3 \delta^3 (\mathbf q - \mathbf p)} \exp(i (- \mathbf p \cdot \mathbf x + \mathbf q \cdot \mathbf y)) + \underbrace{[\hat a_{\mathbf p}^\dagger, \hat a_{\mathbf q}^\dagger]}_0 \exp(i (- \mathbf p \cdot \mathbf x - \mathbf q \cdot \mathbf y))\Big) \\ & = \int \frac{d^3 p ~ d^3 q}{{(2\pi)}^3} \frac{1}{2 \sqrt{\omega_{\mathbf p}} \omega_{\mathbf q}} \Big ( \underbrace{\delta^3 (\mathbf p - \mathbf q) \exp(i (\mathbf p \cdot \mathbf x - \mathbf q \cdot \mathbf y))}_{\mathbf p = \mathbf q} \\ & \qquad - \underbrace{\delta^3 (\mathbf q - \mathbf p) \exp(i (- \mathbf p \cdot \mathbf x + \mathbf q \cdot \mathbf y))}_{\mathbf p = \mathbf q} \Big) \\ & = \int \frac{d^3 p}{{(2\pi)}^3} \frac{1}{2 \omega_{\mathbf p}} \Big (\underbrace{\exp(i \mathbf p \cdot (\mathbf x - \mathbf y))}_{\delta^3 (\mathbf x - \mathbf y)} - \underbrace{\exp(i \mathbf p \cdot (- \mathbf x + \mathbf y))}_{\delta^3 (\mathbf x - \mathbf y)}\Big) = 0 ~.
        \end{aligned}
        \end{equation*}

        For the conjugate-conjugate commutator, using~\eqref{anncrea},~\eqref{kgpop} and~\eqref{deltaint}
        \begin{equation*}
        \begin{aligned}
            [\hat \pi(\mathbf x), \hat \pi (\mathbf y)] & = [\int \frac{d^3 p}{{(2\pi)}^3} \Big (- i\sqrt{\frac{\omega_{\mathbf p}}{2}} \Big )  \Big (\hat a_{\mathbf p} \exp(i \mathbf p \cdot \mathbf x) - \hat a_{\mathbf p}^\dagger \exp(- i \mathbf p \cdot \mathbf x) \Big), \\ & \qquad \int \frac{d^3 q}{{(2\pi)}^3} \Big (- i \sqrt{\frac{\omega_{\mathbf q}}{2}} \Big )  \Big (\hat a_{\mathbf q} \exp(i \mathbf q \cdot \mathbf y) - \hat a_{\mathbf q}^\dagger \exp(- i \mathbf q \cdot \mathbf y) \Big)] \\ &  = \int \frac{d^3 p ~ d^3 q}{{(2\pi)}^6} \Big (- \frac{1}{2} \sqrt{\omega_{\mathbf p}\omega_{\mathbf q}} \Big ) [\hat a_{\mathbf p} \exp(i \mathbf p \cdot \mathbf x) - \hat a_{\mathbf p}^\dagger \exp(- i \mathbf p \cdot \mathbf x), \\ & \qquad \hat a_{\mathbf q} \exp(i \mathbf q \cdot \mathbf y) - \hat a_{\mathbf q}^\dagger \exp(- i \mathbf q \cdot \mathbf y)] \\ & = \int \frac{d^3 p ~ d^3 q}{{(2\pi)}^6} \Big (- \frac{1}{2} \sqrt{\omega_{\mathbf p}\omega_{\mathbf q}} \Big ) \Big (\underbrace{[\hat a_{\mathbf p}, \hat a_{\mathbf q}]}_0 \exp(i (\mathbf p \cdot \mathbf x + \mathbf q \cdot \mathbf y)) - \underbrace{[\hat a_{\mathbf p}, \hat a_{\mathbf q}^\dagger]}_{(2\pi)^3 \delta^3 (\mathbf p - \mathbf q)} \exp(i (\mathbf p \cdot \mathbf x - \mathbf q \cdot \mathbf y)) \\ & \qquad - \underbrace{[\hat a_{\mathbf p}^\dagger, \hat a_{\mathbf q}]}_{- (2\pi)^3 \delta^3 (\mathbf q - \mathbf p)} \exp(i (- \mathbf p \cdot \mathbf x + \mathbf q \cdot \mathbf y)) + \underbrace{[\hat a_{\mathbf p}^\dagger, \hat a_{\mathbf q}^\dagger]}_0 \exp(i (- \mathbf p \cdot \mathbf x - \mathbf q \cdot \mathbf y))\Big) \\ & = \int \frac{d^3 p ~ d^3 q}{{(2\pi)}^3} \Big (- \frac{1}{2} \sqrt{\omega_{\mathbf p}\omega_{\mathbf q}} \Big ) \Big ( - \underbrace{\delta^3 (\mathbf p - \mathbf q) \exp(i (\mathbf p \cdot \mathbf x - \mathbf q \cdot \mathbf y))}_{\mathbf p = \mathbf q} \\ & \qquad + \underbrace{\delta^3 (\mathbf q - \mathbf p) \exp(i (- \mathbf p \cdot \mathbf x + \mathbf q \cdot \mathbf y))}_{\mathbf p = \mathbf q} \Big) \\ & = \int \frac{d^3 p}{{(2\pi)}^3} \Big (- \frac{\omega_{\mathbf p}}{2} \Big ) \Big (-\underbrace{\exp(i \mathbf p \cdot (\mathbf x - \mathbf y))}_{\delta^3 (\mathbf x - \mathbf y)} + \underbrace{\exp(i \mathbf p \cdot (- \mathbf x + \mathbf y))}_{\delta^3 (\mathbf x - \mathbf y)}\Big) = 0 ~.
        \end{aligned}
        \end{equation*}

        For the field-conjugate commutator, using~\eqref{anncrea},~\eqref{kgfop},~\eqref{kgpop} and~\eqref{deltaint}
        \begin{equation*}
        \begin{aligned}
            [\hat \varphi(\mathbf x), \hat \pi (\mathbf y)] & = [\int \frac{d^3 p}{{(2\pi)}^3} \frac{1}{\sqrt{2 \omega_{\mathbf p}}} \Big (\hat a_{\mathbf p} \exp(i \mathbf p \cdot \mathbf x) + \hat a_{\mathbf p}^\dagger \exp(- i \mathbf p \cdot \mathbf x) \Big), \\ & \qquad \int \frac{d^3 q}{{(2\pi)}^3} \Big (- i \sqrt{\frac{\omega_{\mathbf q}}{2}} \Big )  \Big (\hat a_{\mathbf q} \exp(i \mathbf q \cdot \mathbf y) - \hat a_{\mathbf q}^\dagger \exp(- i \mathbf q \cdot \mathbf y) \Big)] \\ &  = \int \frac{d^3 p ~ d^3 q}{{(2\pi)}^6} \Big (- \frac{i}{2}\sqrt{\frac{\omega_{\mathbf q}}{\omega_{\mathbf p}}} \Big ) [\hat a_{\mathbf p} \exp(i \mathbf p \cdot \mathbf x) + \hat a_{\mathbf p}^\dagger \exp(- i \mathbf p \cdot \mathbf x), \\ & \qquad \hat a_{\mathbf q} \exp(i \mathbf q \cdot \mathbf y) - \hat a_{\mathbf q}^\dagger \exp(- i \mathbf q \cdot \mathbf y)] \\ & = \int \frac{d^3 p ~ d^3 q}{{(2\pi)}^6} \Big (- \frac{i}{2}\sqrt{\frac{\omega_{\mathbf q}}{\omega_{\mathbf p}}} \Big ) \Big ( \underbrace{[\hat a_{\mathbf p}, \hat a_{\mathbf q}]}_0 \exp(i (\mathbf p \cdot \mathbf x + \mathbf q \cdot \mathbf y)) - \underbrace{[\hat a_{\mathbf p}, \hat a_{\mathbf q}^\dagger]}_{(2\pi)^3 \delta^3 (\mathbf p - \mathbf q)} \exp(i (\mathbf p \cdot \mathbf x - \mathbf q \cdot \mathbf y)) \\ & \qquad + \underbrace{[\hat a_{\mathbf p}^\dagger, \hat a_{\mathbf q}]}_{- (2\pi)^3 \delta^3 (\mathbf q - \mathbf p)} \exp(i (- \mathbf p \cdot \mathbf x + \mathbf q \cdot \mathbf y)) - \underbrace{[\hat a_{\mathbf p}^\dagger, \hat a_{\mathbf q}^\dagger]}_0 \exp(i (- \mathbf p \cdot \mathbf x - \mathbf q \cdot \mathbf y))\Big) \\ & = \int \frac{d^3 p ~ d^3 q}{{(2\pi)}^3} \Big (- \frac{i}{2}\sqrt{\frac{\omega_{\mathbf q}}{\omega_{\mathbf p}}} \Big ) \Big ( - \underbrace{\delta^3 (\mathbf p - \mathbf q) \exp(i (\mathbf p \cdot \mathbf x - \mathbf q \cdot \mathbf y))}_{\mathbf p = \mathbf q} \\ & \qquad - \underbrace{\delta^3 (\mathbf q - \mathbf p) \exp(i (- \mathbf p \cdot \mathbf x + \mathbf q \cdot \mathbf y))}_{\mathbf p = \mathbf q} \Big) \\ & = \int \frac{d^3 p}{{(2\pi)}^3} \Big (\frac{i}{2} \Big ) \Big (\underbrace{\exp(i \mathbf p \cdot (\mathbf x - \mathbf y))}_{\delta^3 (\mathbf x - \mathbf y)} + \underbrace{\exp(i \mathbf p \cdot (- \mathbf x + \mathbf y))}_{\delta^3 (\mathbf x - \mathbf y)}\Big) \\ & = \frac{i}{2} 2 \delta^3 (\mathbf x - \mathbf y) = i \delta^3 (\mathbf x - \mathbf y) ~.
        \end{aligned}
        \end{equation*}
    \end{proof}

    The hamiltonian is 
    \begin{equation*}
        H = \frac{1}{2} \int d^3 x ~ (\pi^2 + (\boldsymbol \nabla \varphi)^2 + m^2 \varphi^2) ~.
    \end{equation*}
    If we make a function study of the classical hamiltonian, we notice that it has quadratic terms and a minimum at $\varphi_0 (t, \mathbf x) = const$ which we could consider as the ground state with $\varphi_0 = 0$. Quantising the theory means that we consider quantum (small) fluctuations $\delta \varphi$ around this ground state such that 
    \begin{equation*}
        \varphi(t, \mathbf x) = \underbrace{\varphi(t, \mathbf x)_0}_0 + \delta \varphi(t, \mathbf x) ~.
    \end{equation*} 
    The hamiltonian operator in quantum field theory becomes
    \begin{equation}\label{hamkg}
        \hat H = \int \frac{d^3 p}{(2\pi)^3} \omega_{\mathbf p} \hat a_{\mathbf p}^\dagger \hat a_{\mathbf p} + \frac{1}{2} \int d^3 p ~ \omega_{\mathbf p} \delta^3 (0) ~.
    \end{equation}
    \begin{proof}
        Infact, the conjugate field is 
        \begin{equation}\label{conjfield}
        \begin{aligned}
            \pi = \pdv{\mathcal L}{\dot \varphi} = \pdv{\mathcal L}{\partial_0 \varphi} = \partial_0 \varphi = \dot \varphi 
        \end{aligned}
        \end{equation}
        and using~\eqref{energ} and~\eqref{kglan}
        \begin{equation*}
        \begin{aligned}
            H & = \int d^3 x ~ T^{00} \\ & = \int d^3 x ~(\pi \underbrace{\dot \varphi}_\pi - \mathcal L) \\ & = \int d^3 x ~(\pi^2 - \frac{1}{2} \partial_\mu \varphi \partial^\mu \varphi + \frac{1}{2} m^2 \varphi^2) \\ & = \int d^3 x ~(\pi^2 - \frac{1}{2} \partial_0 \varphi \partial^0 \varphi - \frac{1}{2} \partial_i \varphi \partial^i \varphi + \frac{1}{2} m^2 \varphi^2) \\ & = \int d^3 x ~(\pi^2 - \frac{1}{2} \underbrace{\partial_0 \varphi \partial^0 \varphi}_{\pi^2} - \frac{1}{2} \underbrace{\partial_i \varphi \partial^i \varphi}_{- \nabla^2 \varphi} + \frac{1}{2} m^2 \varphi^2) \\ & = \frac{1}{2} \int d^3 x ~ (\pi^2 + (\boldsymbol \nabla \varphi)^2 + m^2 \varphi^2) ~.
        \end{aligned}
        \end{equation*}

        Furthermore, using~\eqref{anncrea},~\eqref{kgfop},~\eqref{kgpop} and~\eqref{deltaint}
        \begin{equation*}
        \begin{aligned}
            \hat H & = \frac{1}{2} \int d^3 x ~ \Big (\hat \pi^2 + (\boldsymbol \nabla \hat \varphi)^2 + m^2 \hat \varphi^2) \\ & = \frac{1}{2} \int d^3 x ~ (\int \frac{d^3 p}{{(2\pi)}^3} \Big (- i\sqrt{\frac{\omega_{\mathbf p}}{2}} \Big ) \Big (\hat a_{\mathbf p} \exp(i \mathbf p \cdot \mathbf x) - \hat a_{\mathbf p}^\dagger \exp(- i \mathbf p \cdot \mathbf x) \Big) \Big ) \\ & \qquad \Big (\int \frac{d^3 q}{{(2\pi)}^3} \Big (- i\sqrt{\frac{\omega_{\mathbf q}}{2}} \Big ) \Big (\hat a_{\mathbf q} \exp(i \mathbf q \cdot \mathbf x) - \hat a_{\mathbf q}^\dagger \exp(- i \mathbf q \cdot \mathbf x) \Big) \Big ) \\ & \qquad + \nabla \Big ( \int \frac{d^3 p}{{(2\pi)}^3} \frac{1}{\sqrt{2 \omega_{\mathbf p}}} \Big (\hat a_{\mathbf p} \exp(i \mathbf p \cdot \mathbf x) + \hat a_{\mathbf p}^\dagger \exp(- i \mathbf p \cdot \mathbf x) \Big) \Big) \\ & \qquad \nabla \Big ( \int \frac{d^3 q}{{(2\pi)}^3} \frac{1}{\sqrt{2 \omega_{\mathbf q}}} \Big (\hat a_{\mathbf q} \exp(i \mathbf q \cdot \mathbf x) + \hat a_{\mathbf q}^\dagger \exp(- i \mathbf q \cdot \mathbf x) \Big) \Big) \\ & \qquad + m^2 \Big (\int \frac{d^3 p}{{(2\pi)}^3} \frac{1}{\sqrt{2 \omega_{\mathbf p}}} \Big (\hat a_{\mathbf p} \exp(i \mathbf p \cdot \mathbf x) + \hat a_{\mathbf p}^\dagger \exp(- i \mathbf p \cdot \mathbf x) \Big) \Big ) \\ & \qquad \Big ( \int \frac{d^3 q}{{(2\pi)}^3} \frac{1}{\sqrt{2 \omega_{\mathbf q}}} \Big (\hat a_{\mathbf q} \exp(i \mathbf q \cdot \mathbf x) + \hat a_{\mathbf q}^\dagger \exp(- i \mathbf q \cdot \mathbf x) \Big) \Big)
        \end{aligned}
        \end{equation*}
        \begin{equation*}
        \begin{aligned}
            \phantom{\hat H} & = \frac{1}{2} \int \frac{d^3 x ~ d^3 p ~d^3 q}{(2\pi)^6} ~ \Big (\Big (- \frac{1}{2} \sqrt{\omega_{\mathbf p} \omega_{\mathbf q}} \Big ) \Big (\hat a_{\mathbf p} \hat a_{\mathbf q} \exp(i (\mathbf p + \mathbf q) \cdot \mathbf x) - \hat a_{\mathbf p} \hat a_{\mathbf q}^\dagger \exp(i (\mathbf p - \mathbf q) \cdot \mathbf x) \\ & \qquad - \hat a_{\mathbf p}^\dagger \hat a_{\mathbf q} \exp(i (- \mathbf p + \mathbf q) \cdot \mathbf x) + \hat a_{\mathbf p}^\dagger \hat a_{\mathbf q}^\dagger \exp(i (- \mathbf p - \mathbf q) \cdot \mathbf x) \Big) \\ & \qquad + \frac{1}{2 \sqrt{\omega_{\mathbf p} \omega_{\mathbf q}}} \Big (i \mathbf p \hat a_{\mathbf p} \exp(i \mathbf p \cdot \mathbf x) - i \mathbf p \hat a_{\mathbf p}^\dagger \exp(- i \mathbf p \cdot \mathbf x) \Big) \cdot \\ & \qquad \Big ( i \mathbf q \hat a_{\mathbf q} \exp(i \mathbf q \cdot \mathbf x) - i \mathbf q \hat a_{\mathbf q}^\dagger \exp(- i \mathbf q \cdot \mathbf x) \Big) \\ & \qquad + m^2 \frac{1}{2 \sqrt{\omega_{\mathbf p} \omega_{\mathbf q}}} \Big (\hat a_{\mathbf p} \hat a_{\mathbf q} \exp(i (\mathbf p + \mathbf q) \cdot \mathbf x) + \hat a_{\mathbf p} \hat a_{\mathbf q}^\dagger \exp(i (\mathbf p - \mathbf q) \cdot \mathbf x) \\ & \qquad + \hat a_{\mathbf p}^\dagger \hat a_{\mathbf q} \exp(i (- \mathbf p + \mathbf q) \cdot \mathbf x) + \hat a_{\mathbf p}^\dagger \hat a_{\mathbf q}^\dagger \exp(i (- \mathbf p - \mathbf q) \cdot \mathbf x) \Big) \Big) 
        \end{aligned}
        \end{equation*}
        \begin{equation*}
        \begin{aligned}
            \phantom{\hat H} & = \frac{1}{2} \int \frac{d^3 x ~ d^3 p ~d^3 q}{(2\pi)^6} \Big (\Big (- \frac{1}{2} \sqrt{\omega_{\mathbf p} \omega_{\mathbf q}} \Big ) \Big (\hat a_{\mathbf p} \hat a_{\mathbf q} \underbrace{\exp(i (\mathbf p + \mathbf q) \cdot \mathbf x)}_{\delta^3 (\mathbf p + \mathbf q)} - \hat a_{\mathbf p} \hat a_{\mathbf q}^\dagger \underbrace{\exp(i (\mathbf p - \mathbf q) \cdot \mathbf x)}_{\delta^3 (\mathbf p - \mathbf q)} \\ & \qquad - \hat a_{\mathbf p}^\dagger \hat a_{\mathbf q} \underbrace{\exp(i (- \mathbf p + \mathbf q) \cdot \mathbf x)}_{\delta^3 (\mathbf p - \mathbf q)} + \hat a_{\mathbf p}^\dagger \hat a_{\mathbf q}^\dagger \underbrace{\exp(i (- \mathbf p - \mathbf q) \cdot \mathbf x)}_{\delta^3 (\mathbf p + \mathbf q)} \Big) \\ & \qquad + \frac{1}{2 \sqrt{\omega_{\mathbf p} \omega_{\mathbf q}}} \Big (- \mathbf p \cdot \mathbf q \hat a_{\mathbf p} \hat a_{\mathbf q} \underbrace{\exp(i (\mathbf p + \mathbf q) \cdot \mathbf x)}_{\delta^3 (\mathbf p + \mathbf q)} + \mathbf p \cdot \mathbf q \hat a_{\mathbf p} \hat a_{\mathbf q}^\dagger \underbrace{\exp(i (\mathbf p - \mathbf q) \cdot \mathbf x)}_{\delta^3 (\mathbf p - \mathbf q)} \\ & \qquad + \mathbf p \cdot \mathbf q \hat a_{\mathbf p}^\dagger \hat a_{\mathbf q} \underbrace{\exp(i (- \mathbf p + \mathbf q) \cdot \mathbf x)}_{\delta^3 (\mathbf p - \mathbf q)} - \mathbf p \cdot \mathbf q \hat a_{\mathbf p}^\dagger \hat a_{\mathbf q}^\dagger \underbrace{\exp(i (- \mathbf p - \mathbf q) \cdot \mathbf x)}_{\delta^3 (\mathbf p + \mathbf q)} \Big) \\ & \qquad + \frac{m^2}{2 \sqrt{\omega_{\mathbf p} \omega_{\mathbf q}}} \Big (\hat a_{\mathbf p} \hat a_{\mathbf q} \underbrace{\exp(i (\mathbf p + \mathbf q) \cdot \mathbf x)}_{\delta^3 (\mathbf p + \mathbf q)} + \hat a_{\mathbf p} \hat a_{\mathbf q}^\dagger \underbrace{\exp(i (\mathbf p - \mathbf q) \cdot \mathbf x)}_{\delta^3 (\mathbf p - \mathbf q)} \\ & \qquad + \hat a_{\mathbf p}^\dagger \hat a_{\mathbf q} \underbrace{\exp(i (- \mathbf p + \mathbf q) \cdot \mathbf x)}_{\delta^3 (\mathbf p - \mathbf q)} + \hat a_{\mathbf p}^\dagger \hat a_{\mathbf q}^\dagger \underbrace{\exp(i (- \mathbf p - \mathbf q) \cdot \mathbf x)}_{\delta^3 (\mathbf p + \mathbf q)} \Big) \Big) 
        \end{aligned}
        \end{equation*}
        \begin{equation*}
        \begin{aligned}
            \phantom{\hat H} & = \frac{1}{2} \int \frac{d^3 p ~d^3 q}{(2\pi)^3} \Big (\Big (- \frac{1}{2} \sqrt{\omega_{\mathbf p} \omega_{\mathbf q}} \Big ) \Big (\hat a_{\mathbf p} \hat a_{\mathbf q} \underbrace{\delta^3 (\mathbf p + \mathbf q)}_{\mathbf p = - \mathbf q} - \hat a_{\mathbf p} \hat a_{\mathbf q}^\dagger \underbrace{\delta^3 (\mathbf p - \mathbf q)}_{\mathbf p = \mathbf q} \\ & \qquad - \hat a_{\mathbf p}^\dagger \hat a_{\mathbf q} \underbrace{\delta^3 (\mathbf p - \mathbf q)}_{\mathbf p = \mathbf q} + \hat a_{\mathbf p}^\dagger \hat a_{\mathbf q}^\dagger \underbrace{\delta^3 (\mathbf p + \mathbf q)}_{\mathbf p = - \mathbf q} \Big) \\ & \qquad + \frac{1}{2 \sqrt{\omega_{\mathbf p} \omega_{\mathbf q}}} \Big (- \mathbf p \cdot \mathbf q \hat a_{\mathbf p} \hat a_{\mathbf q} \underbrace{\delta^3 (\mathbf p + \mathbf q)}_{\mathbf p = - \mathbf q} + \mathbf p \cdot \mathbf q \hat a_{\mathbf p} \hat a_{\mathbf q}^\dagger \underbrace{\delta^3 (\mathbf p - \mathbf q)}_{\mathbf p = \mathbf q} \\ & \qquad + \mathbf p \cdot \mathbf q \hat a_{\mathbf p}^\dagger \hat a_{\mathbf q} \underbrace{\delta^3 (\mathbf p - \mathbf q)}_{\mathbf p = \mathbf q} - \mathbf p \cdot \mathbf q \hat a_{\mathbf p}^\dagger \hat a_{\mathbf q}^\dagger \underbrace{\delta^3 (\mathbf p + \mathbf q)}_{\mathbf p = - \mathbf q} \Big) \\ & \qquad + \frac{m^2}{2 \sqrt{\omega_{\mathbf p} \omega_{\mathbf q}}} \Big (\hat a_{\mathbf p} \hat a_{\mathbf q} \underbrace{\delta^3 (\mathbf p + \mathbf q)}_{\mathbf p = - \mathbf q} + \hat a_{\mathbf p} \hat a_{\mathbf q}^\dagger \underbrace{\delta^3 (\mathbf p - \mathbf q)}_{\mathbf p = \mathbf q} \\ & \qquad + \hat a_{\mathbf p}^\dagger \hat a_{\mathbf q} \underbrace{\delta^3 (\mathbf p - \mathbf q)}_{\mathbf p = \mathbf q} + \hat a_{\mathbf p}^\dagger \hat a_{\mathbf q}^\dagger \underbrace{\delta^3 (\mathbf p + \mathbf q)}_{\mathbf p = - \mathbf q} \Big) \Big)  
        \end{aligned}
        \end{equation*}
        \begin{equation*}
        \begin{aligned}
            \phantom{\hat H} & = \frac{1}{2} \int \frac{d^3 p}{(2\pi)^3} \Big ( \Big (-\frac{\omega_{\mathbf p}}{2} \Big) \Big (\hat a_{\mathbf p} \hat a_{- \mathbf p} - \hat a_{\mathbf p} \hat a_{\mathbf p}^\dagger - \hat a_{\mathbf p}^\dagger \hat a_{\mathbf p} + \hat a_{\mathbf p}^\dagger \hat a_{- \mathbf p}^\dagger \Big) \\ & \qquad + \Big (\frac{|\mathbf p|^2}{2 \omega_{\mathbf p}} \Big) \Big (\hat a_{\mathbf p} \hat a_{- \mathbf p} + \hat a_{\mathbf p} \hat a_{\mathbf p}^\dagger + \hat a_{\mathbf p}^\dagger \hat a_{\mathbf p} + \hat a_{\mathbf p}^\dagger \hat a_{- \mathbf p}^\dagger \Big) \\ & \qquad + \Big ( \frac{m^2}{2 \omega_{\mathbf p}} \Big) \Big (\hat a_{\mathbf p} \hat a_{- \mathbf p}+ \hat a_{\mathbf p} \hat a_{\mathbf p}^\dagger  + \hat a_{\mathbf p}^\dagger \hat a_{\mathbf p} + \hat a_{\mathbf p}^\dagger \hat a_{- \mathbf p}^\dagger \Big) \Big)
        \end{aligned}
        \end{equation*}
        \begin{equation*}
        \begin{aligned}
            \phantom{\hat H} & = \frac{1}{4} \int \frac{d^3 p}{(2\pi)^3} \frac{1}{\omega_{\mathbf p}} \Big ( (\hat a_{\mathbf p} \hat a_{- \mathbf p} + \hat a_{\mathbf p}^\dagger \hat a_{- \mathbf p}^\dagger ) \underbrace{(- \omega_{\mathbf p}^2 + | \mathbf p|^2 + m^2 )}_0 \\ & \qquad + (\hat a_{\mathbf p} \hat a_{\mathbf p}^\dagger + \hat a_{\mathbf p}^\dagger \hat a_{\mathbf p} ) \underbrace{(\omega_{\mathbf p}^2 + | \mathbf p|^2 + m^2 )}_{2\omega_{\mathbf p}^2} \Big) \\ & = \frac{1}{4} \int \frac{d^3 p}{(2\pi)^3} \frac{2 \omega_{\mathbf p}^{\cancel{2}}}{\cancel{\omega_{\mathbf p}}} (\hat a_{\mathbf p} \hat a_{\mathbf p}^\dagger + \hat a_{\mathbf p}^\dagger \hat a_{\mathbf p}) \\ & = \frac{1}{2} \int \frac{d^3 p}{(2\pi)^3} \omega_{\mathbf p} (\underbrace{\hat a_{\mathbf p} \hat a_{\mathbf p}^\dagger}_{[\hat a_{\mathbf p}, \hat a_{\mathbf p}^\dagger] + \hat a_{\mathbf p}^\dagger \hat a_{\mathbf p}} + \hat a_{\mathbf p}^\dagger \hat a_{\mathbf p}) \\ & = \frac{1}{2} \int \frac{d^3 p}{(2\pi)^3} \omega_{\mathbf p} (\underbrace{[\hat a_{\mathbf p}, \hat a_{\mathbf p}^\dagger]}_{(2\pi)^3 \delta^3 (\mathbf p - \mathbf p)} + 2 \hat a_{\mathbf p}^\dagger \hat a_{\mathbf p}) \\ & = \frac{1}{2} \int d^3 p ~ \omega_{\mathbf p} \delta^3 (0) + \int \frac{d^3 p}{(2\pi)^3} ~ \omega_{\mathbf p}\hat a_{\mathbf p}^\dagger \hat a_{\mathbf p} ~,
        \end{aligned}
        \end{equation*}
        where we have used the fact that $\omega_{- \mathbf p} = \sqrt{| - \mathbf p|^2 + m^2} = \sqrt{|\mathbf p|^2 + m^2} = \omega_{\mathbf p}$. 

    \end{proof}

    The first term of~\eqref{hamkg} counts simply how what is the relativistic energy of each particle $\omega_{\mathbf p}$ and through the number operator $\hat N_{\mathbf p} = \hat a_{\mathbf p}^\dagger \hat a_{\mathbf p}$ and the integral, we sum all over the possible value of $\mathbf p$. However, most of them may be zero and we do not have to worry about divergences. 

\section{Vacuum energy}
    Things are different if we look at the second term of~\eqref{hamkg}, beacuse, in analogy with the energy of the single harmonic oscillator, we interpret it as the energy of the vacuum and it diverges for two reasons
    \begin{enumerate}
        \item infrared divergence, i.e. 
            \begin{equation*}
                \delta^3 (0) \rightarrow \infty~,
            \end{equation*}
        \item ultraviolet divergence, i.e. for $|\mathbf p| \rightarrow \infty$
            \begin{equation*}
            \int d^3 p ~ \omega_{\mathbf p} \rightarrow \infty ~,
        \end{equation*} 
            since for $|\mathbf p| \rightarrow \infty$
            \begin{equation*}
                \omega_{\mathbf p} = \sqrt{|\mathbf p|^2 + m^2} \simeq |\mathbf p| ~.
            \end{equation*}
    \end{enumerate}

    This can be better understood by applying the hamiltonian operator to the vacuum state $\ket{0}$, i.e.~the state such that it is annihilated by all the annihilation operators is for all $\mathbf p$
    \begin{equation*}
        \hat a_{\mathbf p} \ket{0} = 0 \quad \forall \mathbf p ~.
    \end{equation*}
    Therefore 
    \begin{equation*}
        \hat H \ket{0} = E_0 \ket{0} = \infty \ket{0}
    \end{equation*}
    and the vaccum energy is infinite.
    \begin{proof}
        Infact, using~\eqref{hamkg}
        \begin{equation*}
            \hat H \ket{0} = \int \frac{d^3 p}{(2\pi)^3} \omega_{\mathbf p} \hat a_{\mathbf p}^\dagger \underbrace{\hat a_{\mathbf p} \ket{0}}_0 + \Big (\underbrace{\frac{1}{2} \int d^3 p ~ \omega_{\mathbf p} \delta^3 (0)}_\infty \Big ) \ket{0} = \infty \ket{0} = E_0 \ket{0} ~.
        \end{equation*}
    \end{proof}

\subsection{IR divergence}

    The infrared divergence is due to the fact that space is infinitely large. This means that in every point of spacetime there is an harmonic oscilators. To prove this, consider a box of sides $L$ and periodic boundary conditions for the fields. The volume of the box is just the Dirac delta inside the integrand of the energy vacuum. Infact 
    \begin{equation*}
        (2\pi)^3 \delta^3 (0) = \lim_{L \rightarrow \infty} \int_{-\frac{L}{2}}^{\frac{L}{2}} \int_{-\frac{L}{2}}^{\frac{L}{2}} \int_{-\frac{L}{2}}^{\frac{L}{2}} d^3 x ~ \exp(- i \mathbf p \cdot \mathbf x) \Big \vert_{\mathbf p = 0} = \lim_{L \rightarrow \infty} \int_{-\frac{L}{2}}^{\frac{L}{2}} \int_{-\frac{L}{2}}^{\frac{L}{2}} \int_{-\frac{L}{2}}^{\frac{L}{2}} d^3 x = L^3 = V ~.
    \end{equation*}
    This divergence can be removed by studying energy densities instead of pure energies. 
    \begin{equation*}
        \mathcal E_0 = \frac{E_0}{V} = \int \frac{d^3}{(2\pi)^3} \frac{\omega_{\mathbf p}}{2} ~.
    \end{equation*}

\subsection{UV divergence}

    However, still the energy density is infinite because of the ultraviolet divergence, since for $|\mathbf p| \rightarrow \infty$
    \begin{equation*}
        \mathcal E_0 \rightarrow \infty ~.
    \end{equation*}
    
    The reason is the following: we made a strong assumption considering the theory valid for any large value of energy and now we have found where the theory breaks, since this divergence arises indeed from the fact that our theory is not valid for arbitrarily high energies. What we need to do id to introduce a cut-off, i.e. a maximum energy after which the theory is not anymore valid. Since gravity cannot be neglected and becomes strongly coupled at Planck mass $M_P \simeq 10^{19} GeV$, we therefore set the cut-off at this energy. 

    Computationally, we measure only energy differences between excited estates, which are particles, and the vacuum energy, which becomes irrelevant and it can be set to zero. This procedure is called \textit{normal ordering}. We define a new hamiltonian operator 
    \begin{equation*}
        \colon \hat H \colon = \hat H - E_0 = \hat H - \bra{0} \hat H \ket{0} ~,
    \end{equation*}
    such that 
    \begin{equation*}
        \colon \hat H \colon \ket{0} = \underbrace{\hat H \ket{0}}_{E_0 \ket{0}} - E_0 \ket{0} = 0~.
    \end{equation*}
    The difference between $\hat H$ and $\colon \hat H \colon$ is due to an ambiguity in going from classical to quantum theory. Infact, normal ordering means to set a rule to order annihilation and creation operators: all annihilation operators are pleced to the right and, consequently, creation operatore to the left (dagger always first). We emphasise that in the interaction theory, vaccume energy cannot be anymore set to zero.

    As we said, different ordering in the classical hamiltonians bring different hamiltonian operators. Infact, if we rewrite the hamiltonian of the classical harmonic oscillator
    \begin{equation*}
        H = \frac{p^2}{2m} + \frac{1}{2} \omega^2 q^2 = \frac{1}{2} (\omega q - i p) (\omega q + i p) ~,
    \end{equation*}
    we notice that the first one leads us to
    \begin{equation*}
        \hat H = \omega (\hat a^\dagger \hat a + \frac{\mathbb I}{2}) ~,
    \end{equation*}
    while the second one to 
    \begin{equation*}
        \hat H = \omega a^\dagger \hat a ~.
    \end{equation*}
    \begin{proof}
        For the first hamiltonian 
        \begin{equation*}
        \begin{aligned}
            \hat H & = \frac{1}{2} (-i \sqrt{\frac{\omega}{2}} (\hat a - \hat a^\dagger))^2 + \frac{1}{2} \omega^2 (\frac{1}{\sqrt{2 \omega}} (\hat a + \hat a^\dagger))^2 \\ & = - \frac{\omega}{4} (\cancel{\hat a^2} - \hat a \hat a^\dagger - \hat a^\dagger \hat a + \cancel{(\hat a^\dagger)^2}) + \frac{\omega}{4} (\cancel{\hat a^2} + \hat a \hat a^\dagger + \hat a^\dagger \hat a + \cancel{(\hat a^\dagger)^2}) \\ & = \frac{\omega}{4} (\hat a \hat a^\dagger + \hat a^\dagger \hat a + \hat a \hat a^\dagger + \hat a^\dagger \hat a) \\ & = \frac{\omega}{2} (\underbrace{\hat a \hat a^\dagger}_{[\hat a, \hat a^\dagger] + \hat a^\dagger \hat a} + \hat a^\dagger \hat a) \\ & = \frac{\omega}{2} (\underbrace{[\hat a, \hat a^\dagger]}_{\mathbb I} + 2 \hat a^\dagger \hat a) \\ & = \omega (\frac{\mathbb I}{2}+ \hat a^\dagger \hat a) ~,
        \end{aligned}
        \end{equation*}
        while for the second hamiltonian
        \begin{equation*}
        \begin{aligned}
            \hat H & = \frac{1}{2} \Big (\omega \frac{1}{\sqrt{2 \omega}} (\hat a + \hat a^\dagger) - i (-i \sqrt{\frac{\omega}{2}} (\hat a - \hat a^\dagger)) \Big ) \Big (\omega \frac{1}{\sqrt{2 \omega}} (\hat a + \hat a^\dagger) + i (-i \sqrt{\frac{\omega}{2}} (\hat a - \hat a^\dagger)) \Big) \\ & = \frac{\omega}{4} ( \cancel{\hat a} + \hat a^\dagger - \cancel{\hat a} + \hat a^\dagger ) (\hat a + \cancel{\hat a^\dagger} + \hat a - \cancel{\hat a^\dagger}) \\ & = \omega \hat a^\dagger \hat a ~.
        \end{aligned}
        \end{equation*}
    \end{proof}

    Finally, the normal ordered hamiltonian of the Klein-Gordon theory is 
    \begin{equation} \label{hamop}
        \colon \hat H \colon = \int \frac{d^3 p}{(2\pi)^3} \omega_{\mathbf p} \hat a_{\mathbf p}^\dagger \hat a_{\mathbf p} ~.
    \end{equation}
    \begin{proof}
        Infact, since
        \begin{equation*}
            \hat H = \frac{1}{2} \int \frac{d^3 p}{(2\pi)^3} \omega_{\mathbf p} (\hat a_{\mathbf p} \hat a_{\mathbf p}^\dagger + \hat a_{\mathbf p}^\dagger \hat a_{\mathbf p}) ~,
        \end{equation*}
        we have 
        \begin{equation*}
            \colon \hat H \colon = \frac{1}{2} \int \frac{d^3 p}{(2\pi)^3} \omega_{\mathbf p} (\hat a_{\mathbf p}^\dagger \hat a_{\mathbf p} + \hat a_{\mathbf p}^\dagger \hat a_{\mathbf p}) = \int \frac{d^3 p}{(2\pi)^3} \omega_{\mathbf p} \hat a_{\mathbf p}^\dagger \hat a_{\mathbf p} ~.
        \end{equation*}
    \end{proof}

    Furthermore, by analogy of the harmonic oscillator, the hamiltonian~\eqref{hamkg} and the annihilation and creation operators satisfies the commutation relations 
    \begin{equation*}
        [\hat H, \hat a_{\mathbf p}] = - \omega_{\mathbf p} \hat a_{\mathbf p} ~, \quad [\hat H, \hat a_{\mathbf p}^\dagger] = \omega_{\mathbf p} \hat a_{\mathbf p}^\dagger ~.
    \end{equation*}
    \begin{proof}
        For the first commutator
        \begin{equation*}
        \begin{aligned}
            [\hat H, \hat a_{\mathbf p}] & = \int \frac{d^3 q}{(2\pi)^3} \omega_{\mathbf q} [\hat a_{\mathbf q}^\dagger \hat a_{\mathbf q}, \hat a_{\mathbf p}] \\ & = \int \frac{d^3 q}{(2\pi)^3} \omega_{\mathbf q} (\hat a_{\mathbf q}^\dagger \underbrace{[\hat a_{\mathbf q}, \hat a_{\mathbf p}]}_0 + \underbrace{[\hat a_{\mathbf q}^\dagger, \hat a_{\mathbf p}]}_{- (2\pi)^3 \delta^3 (\mathbf p - \mathbf q)} \hat a_{\mathbf q}) \\ & = - \int \frac{d^3 q}{\cancel{(2\pi)^3}} \omega_{\mathbf q} \cancel{(2\pi)^3} \delta^3 (\mathbf p - \mathbf q) \hat a_{\mathbf q} \\ & = - \omega_{\mathbf p} \hat a_{\mathbf p} ~.
        \end{aligned}
        \end{equation*}

        For the second commutator
        \begin{equation*}
        \begin{aligned}
            [\hat H, \hat a_{\mathbf p}^\dagger] & = \int \frac{d^3 q}{(2\pi)^3} \omega_{\mathbf q} [\hat a_{\mathbf q}^\dagger \hat a_{\mathbf q}, \hat a_{\mathbf p}^\dagger] \\ & = \int \frac{d^3 q}{(2\pi)^3} \omega_{\mathbf q} (\hat a_{\mathbf q}^\dagger \underbrace{[\hat a_{\mathbf q}, \hat a_{\mathbf p}^\dagger]}_ {(2\pi)^3 \delta^3 (\mathbf p - \mathbf q)} + \underbrace{[\hat a_{\mathbf q}^\dagger, \hat a_{\mathbf p}^\dagger]}_{0} \hat a_{\mathbf q}) \\ & = \int \frac{d^3 q}{\cancel{(2\pi)^3}} \omega_{\mathbf q} \cancel{(2\pi)^3} \delta^3 (\mathbf p - \mathbf q) \hat a_{\mathbf q}^\dagger \\ & = \omega_{\mathbf p} \hat a_{\mathbf p}^\dagger ~.
        \end{aligned}
        \end{equation*}
    \end{proof}

    The momentum operator is defined as 
    \begin{equation}\label{momop}
        \hat{\mathbf P} = - \int d^3 x ~ \hat \pi \boldsymbol \nabla \hat \varphi = \int \frac{d^3 p}{(2\pi)^3} \mathbf p \hat a_{\mathbf p}^\dagger \hat a_{\mathbf p}~.
    \end{equation}
    \begin{proof}
        Infact, using~\eqref{momen}
        \begin{equation*}
        \begin{aligned}
            \hat{\mathbf P} & = \int d^3 x ~ T^{0i} \\ & = \int d^3 x ~ \hat \pi \boldsymbol \nabla \hat \varphi ~.
        \end{aligned}
        \end{equation*}

        Furthermore, using~\eqref{anncrea},~\eqref{kgfop},~\eqref{kgpop} and~\eqref{deltaint}
        \begin{equation*}
        \begin{aligned}
            \hat{\mathbf P} & = - \int d^3 x ~ \Big ( \int \frac{d^3 p}{{(2\pi)}^3} \Big (- i\sqrt{\frac{\omega_{\mathbf p}}{2}} \Big ) \Big (\hat a_{\mathbf p} \exp(i \mathbf p \cdot \mathbf x) - \hat a_{\mathbf p}^\dagger \exp(- i \mathbf p \cdot \mathbf x) \Big) \\ & \qquad \nabla \int \frac{d^3 q}{{(2\pi)}^3} \frac{1}{\sqrt{2 \omega_{\mathbf q}}} \Big (\hat a_{\mathbf q} \exp(i \mathbf q \cdot \mathbf x) + \hat a_{\mathbf q}^\dagger \exp(- i \mathbf q \cdot \mathbf x) \Big) \Big ) \\ & = - \int d^3 x ~ \Big ( \int \frac{d^3 p}{{(2\pi)}^3} \Big (- i\sqrt{\frac{\omega_{\mathbf p}}{2}} \Big ) \Big (\hat a_{\mathbf p} \exp(i \mathbf p \cdot \mathbf x) - \hat a_{\mathbf p}^\dagger \exp(- i \mathbf p \cdot \mathbf x) \Big) \\ & \qquad \int \frac{d^3 q}{{(2\pi)}^3} \frac{1}{\sqrt{2 \omega_{\mathbf q}}} \Big (i \mathbf q \hat a_{\mathbf q} \exp(i \mathbf q \cdot \mathbf x) - i \mathbf q \hat a_{\mathbf q}^\dagger \exp(- i \mathbf q \cdot \mathbf x) \Big) \Big ) \\ & = - \int \frac{d^3 x ~ d^3 p ~ d^3 q}{{(2\pi)}^6} \Big (- \frac{i}{2} \sqrt{\frac{\omega_{\mathbf p}}{\omega_{\mathbf q}}} \Big ) (i \mathbf q \hat a_{\mathbf p} \hat a_{\mathbf q} \exp(i (\mathbf p + \mathbf q) \cdot \mathbf x) - i \mathbf q  \hat a_{\mathbf p} \hat a_{\mathbf q}^\dagger \exp(i (\mathbf p - \mathbf q) \cdot \mathbf x) \\ & \qquad - i \mathbf q \hat a_{\mathbf p}^\dagger \hat a_{\mathbf q} \exp(i (- \mathbf p + \mathbf q) \cdot \mathbf x) + i \mathbf q \hat a_{\mathbf p}^\dagger \hat a_{\mathbf q}^\dagger \exp(i (- \mathbf p - \mathbf q) \cdot \mathbf x) )
        \end{aligned}
        \end{equation*}
        \begin{equation*}
        \begin{aligned}
            \phantom{\hat{\mathbf P}} & = - \int \frac{d^3 x ~ d^3 p ~ d^3 q}{{(2\pi)}^6} \Big (\frac{\mathbf q}{2} \sqrt{\frac{\omega_{\mathbf p}}{\omega_{\mathbf q}}} \Big ) (\hat a_{\mathbf p} \hat a_{\mathbf q} \underbrace{\exp(i (\mathbf p + \mathbf q) \cdot \mathbf x)}_{\delta^3 (\mathbf p + \mathbf q)} - \hat a_{\mathbf p} \hat a_{\mathbf q}^\dagger \underbrace{\exp(i (\mathbf p - \mathbf q) \cdot \mathbf x)}_{\delta^3 (\mathbf p - \mathbf q)} \\ & \qquad - \hat a_{\mathbf p}^\dagger \hat a_{\mathbf q} \underbrace{\exp(i (-\mathbf p + \mathbf q) \cdot \mathbf x)}_{\delta^3 (\mathbf p - \mathbf q)} + \hat a_{\mathbf p}^\dagger \hat a_{\mathbf q}^\dagger \underbrace{\exp(i (-\mathbf p - \mathbf q) \cdot \mathbf x)}_{\delta^3 (\mathbf p + \mathbf q)} ) \\ & = - \int \frac{d^3 p ~ d^3 q}{{(2\pi)}^3} \Big (\frac{\mathbf q}{2} \sqrt{\frac{\omega_{\mathbf p}}{\omega_{\mathbf q}}} \Big ) (\hat a_{\mathbf p} \hat a_{\mathbf q} \underbrace{\delta^3 (\mathbf p + \mathbf q) }_{\mathbf q = - \mathbf p} - \hat a_{\mathbf p} \hat a_{\mathbf q}^\dagger \underbrace{\delta^3 (\mathbf p - \mathbf q) }_{\mathbf q = \mathbf p} \\ & \qquad - \hat a_{\mathbf p}^\dagger \hat a_{\mathbf q} \underbrace{\delta^3 (\mathbf p - \mathbf q) }_{\mathbf q = \mathbf p} + \hat a_{\mathbf p}^\dagger \hat a_{\mathbf q}^\dagger \underbrace{\delta^3 (\mathbf p + \mathbf q) }_{\mathbf q = - \mathbf p}) \\ & = - \int \frac{d^3 p}{{(2\pi)}^3} \Big (\frac{\mathbf p}{2} (- \hat a_{\mathbf p}^\dagger \hat a_{\mathbf p} - \hat a_{\mathbf p} \hat a_{\mathbf p}^\dagger) - \frac{\mathbf p}{2} (\hat a_{\mathbf p} \hat a_{- \mathbf p} + \hat a_{\mathbf p}^\dagger \hat a_{- \mathbf p}^\dagger) \Big) \\ & = \int \frac{d^3 p}{{(2\pi)}^3} \Big (\frac{\mathbf p}{2} (\hat a_{\mathbf p}^\dagger \hat a_{\mathbf p} + \hat a_{\mathbf p} \hat a_{\mathbf p}^\dagger) + \frac{\mathbf p}{2} (\hat a_{\mathbf p} \hat a_{- \mathbf p} + \hat a_{\mathbf p}^\dagger \hat a_{- \mathbf p}^\dagger) \Big)  ~,
        \end{aligned}
        \end{equation*}
        which in normal ordering becomes 
        \begin{equation*}
            \hat{\mathbf P} = \int \frac{d^3 p}{{(2\pi)}^3} \Big (\frac{\mathbf p}{2} (\hat a_{\mathbf p}^\dagger \hat a_{\mathbf p} + \hat a_{\mathbf p}^\dagger \hat a_{\mathbf p}^\dagger) \Big) = \int \frac{d^3 p}{{(2\pi)}^3} \mathbf p \hat a_{\mathbf p}^\dagger \hat a_{\mathbf p} ~.
        \end{equation*}
    \end{proof}

\section{$1$-particle states}

    Now, we build the energy eigenstates of a $1$-particle state. In analogy with the harmonic oscillator, we require the following properties:
    \begin{enumerate}
        \item the vacuum state is annihilated by all the annihilation operators for all $\mathbf p$ 
            \begin{equation*}
                \hat a_{\mathbf p} \ket{0} = 0 \quad \forall \mathbf p ~,
            \end{equation*}
        \item a generic state can be defined by the creation operators actiong on the vacuum
            \begin{equation*}
                \ket{\mathbf p} = \hat a_{\mathbf p}^\dagger \ket{0} ~.
            \end{equation*}
    \end{enumerate}

    The state $\ket{\mathbf p}$ is the momentum eigentstate of a single scalar (spinless) particle with mass $m$. Infact, it is the momentum eigenstate 
    \begin{equation*}
        \hat{\mathbf P} \ket{\mathbf p} = \mathbf p \ket{\mathbf p} ~,
    \end{equation*}
    \begin{proof}
        Infact, using~\eqref{momop}
        \begin{equation*}
        \begin{aligned}
            \hat{\mathbf P} \ket{\mathbf p} & = \hat{\mathbf P} \hat a_{\mathbf p}^\dagger \ket{0} \\ & = \int \frac{d^3 q}{(2\pi)^3} \mathbf q \hat a_{\mathbf q}^\dagger \underbrace{\hat a_{\mathbf q} \hat a_{\mathbf p}^\dagger}_{[\hat a_{\mathbf q}, \hat a_{\mathbf p}^\dagger] + \hat a_{\mathbf q}^\dagger \hat a_{\mathbf p}} \ket{0} \\ & = \int \frac{d^3 q}{(2\pi)^3} \mathbf q \hat a_{\mathbf q}^\dagger (\underbrace{[\hat a_{\mathbf q}, \hat a_{\mathbf p}^\dagger]}_{(2\pi)^3 \delta^3 (\mathbf p - \mathbf q)} + \hat a_{\mathbf q}^\dagger \underbrace{\hat a_{\mathbf p}) \ket{0}}_0 \\ & = \int \frac{d^3 q}{\cancel{(2\pi)^3}} \mathbf q \hat a_{\mathbf q}^\dagger  \cancel{(2\pi)^3} \underbrace{\delta^3 (\mathbf p - \mathbf q)}_{\mathbf q = \mathbf p} \ket{0} \\ & = \mathbf p \hat a_{\mathbf p}^\dagger \ket{0} \\ & = \mathbf p \ket{\mathbf p}  ~.
        \end{aligned}
        \end{equation*}
    \end{proof}
    Furthermore, this states is also the energy eigenstate, since it is a function of the momentum 
    \begin{equation*}
        \hat H \ket{\mathbf p} = E_{\mathbf p} \ket{\mathbf p} = \omega_{\mathbf p} \ket{\mathbf p} ~.
    \end{equation*}
    \begin{proof}
        Infact, using~\eqref{hamop}
        \begin{equation*}
        \begin{aligned}
            \hat H \ket{\mathbf p} & = \hat H \hat a_{\mathbf p}^\dagger \ket{0} \\ & = \int \frac{d^3 q}{(2\pi)^3} \omega_{\mathbf q} \hat a_{\mathbf q}^\dagger \underbrace{\hat a_{\mathbf q} \hat a_{\mathbf p}^\dagger}_{[\hat a_{\mathbf q}, \hat a_{\mathbf p}^\dagger] + \hat a_{\mathbf q}^\dagger \hat a_{\mathbf p}} \ket{0} \\ & = \int \frac{d^3 q}{(2\pi)^3} \omega_{\mathbf q} \hat a_{\mathbf q}^\dagger (\underbrace{[\hat a_{\mathbf q}, \hat a_{\mathbf p}^\dagger]}_{(2\pi)^3 \delta^3 (\mathbf p - \mathbf q)} + \hat a_{\mathbf q}^\dagger \underbrace{\hat a_{\mathbf p}) \ket{0}}_0 \\ & = \int \frac{d^3 q}{\cancel{(2\pi)^3}} \omega_{\mathbf q} \hat a_{\mathbf q}^\dagger  \cancel{(2\pi)^3} \underbrace{\delta^3 (\mathbf p - \mathbf q)}_{\mathbf q = \mathbf p} \ket{0} \\ & = \omega_{\mathbf p} \hat a_{\mathbf p}^\dagger \ket{0} \\ & = \omega_{\mathbf p} \ket{\mathbf p}  ~.
        \end{aligned}
        \end{equation*}
    \end{proof}

\section{$n$-particle states}

    We can generalise for a system composed by $n$ particles. The state becomes 
    \begin{equation*}
        \ket{\mathbf p_1, \ldots \mathbf p_n} = \hat a_{\mathbf p_1}^\dagger \ldots \hat a_{\mathbf p_n}^\dagger \ket{0} ~.
    \end{equation*}
    
    Notice that the state is symmetric under exchange of any two particles, since 
    \begin{equation*}
        [\hat a_{\mathbf p_i}^\dagger, \hat a_{\mathbf p_j}^\dagger ] = 0 ~.
    \end{equation*}
    \begin{proof}
        For instance, given two particles of momenta $\mathbf p$ and $\mathbf q$, we have 
        \begin{equation*}
            \ket{\mathbf p, \mathbf q} = \hat a_{\mathbf p}^\dagger \hat a_{\mathbf q}^\dagger \ket{0} = \hat a_{\mathbf q}^\dagger  \hat a_{\mathbf p}^\dagger \ket{0} = \ket{\mathbf q, \mathbf p} ~.
        \end{equation*}
    \end{proof}
    This means that the Klein-Gordon theory describes bosons. It is indeed spin-statistics relation and it is a consequence of quantum field theory and the commutation relations imposed to quantise (not quantum mechanics).

    A basis of the Fock space is built by all the possible combination of creation operators acting on the vacuum state 
    \begin{equation*}
        \{\ket{0}, \hat a_{\mathbf p_1}^\dagger \ket{0}, \hat a_{\mathbf p_1}^\dagger \hat a_{\mathbf p_2}^\dagger \ket{0}, \ldots \} 
    \end{equation*},
    where $\ket{0}$ is the vacuum state, $\hat a_{\mathbf p_1}^\dagger \ket{0}$ is the $1$-particle state, $\hat a_{\mathbf p_1}^\dagger \hat a_{\mathbf p_2}^\dagger \ket{0}$ is the $2$-particles state, etc. The total Fock space is 
    \begin{equation*}
        \mathcal F = \bigoplus_n \mathcal H_n
    \end{equation*}
    where $\mathcal H_n$ is the Hilbert space for $n$ particles.

    We can define the number operator wich counts the number of particle in a given state 
    \begin{equation*}
        \hat N = \int \frac{d^3 p}{(2\pi)^3} \hat a_{\mathbf p}^\dagger \hat a_{\mathbf p} ~,
    \end{equation*}
    such that 
    \begin{equation*}
        \hat N \ket{\mathbf p_1, \ldots \mathbf p_n} = \hat N \hat a_{\mathbf p_1}^\dagger \ldots \hat a_{\mathbf p_n}^\dagger \ket{0} =  n \ket{\mathbf p_1, \ldots \mathbf p_n} ~.
    \end{equation*}

    Notice that the particle number is conserved, since
    \begin{equation*}
        [\hat H, \hat N] = 0 ~.
    \end{equation*}
    This means that if the system has initially $n$ particles, this number will remain the same. This happens only in a free theory, because interactions move the system between different sectors of the Fock space.
    \begin{proof}
        Infact 
        \begin{equation*}
        \begin{aligned}
            [\hat H, \hat N] & = \hat H \hat N - \hat N \hat H \\ & =\int \frac{d^3 p}{(2\pi)^3} \omega_{\mathbf p} \hat a_{\mathbf p}^\dagger \hat a_{\mathbf p} \int \frac{d^3 q}{(2\pi)^3} \hat a_{\mathbf q}^\dagger \hat a_{\mathbf q} - \int \frac{d^3 q}{(2\pi)^3} \hat a_{\mathbf q}^\dagger \hat a_{\mathbf q} \int \frac{d^3 p}{(2\pi)^3} \omega_{\mathbf p} \hat a_{\mathbf p}^\dagger \hat a_{\mathbf p} \\ & = \int \frac{d^3 p ~ d^3 q}{(2\pi)^6} \omega_{\mathbf p} (\hat a_{\mathbf p}^\dagger \underbrace{\hat a_{\mathbf p} \hat a_{\mathbf q}^\dagger}_{[\hat a_{\mathbf p}, \hat a_{\mathbf q}^\dagger] + \hat a_{\mathbf p}^\dagger \hat a_{\mathbf q}} \hat a_{\mathbf q} - \hat a_{\mathbf q}^\dagger \underbrace{\hat a_{\mathbf q} \hat a_{\mathbf p}^\dagger}_{[\hat a_{\mathbf q}, \hat a_{\mathbf p}^\dagger] + \hat a_{\mathbf q}^\dagger \hat a_{\mathbf p}} \hat a_{\mathbf p}) \\ & = \int \frac{d^3 p ~ d^3 q}{(2\pi)^6} \omega_{\mathbf p} (\hat a_{\mathbf p}^\dagger (\underbrace{[\hat a_{\mathbf p}, \hat a_{\mathbf q}^\dagger]}_{(2\pi)^3 \delta^3 (\mathbf p - \mathbf q)} + \hat a_{\mathbf p}^\dagger \hat a_{\mathbf q}) \hat a_{\mathbf q} - \hat a_{\mathbf q}^\dagger (\underbrace{[\hat a_{\mathbf q}, \hat a_{\mathbf p}^\dagger]}_{(2\pi)^3 \delta^3 (\mathbf p - \mathbf q)} + \hat a_{\mathbf q}^\dagger \hat a_{\mathbf p}) \hat a_{\mathbf p}) \\ & = \int \frac{d^3 p ~ d^3 q}{(2\pi)^6} \omega_{\mathbf p} (\hat a_{\mathbf p}^\dagger ((2\pi)^3 \delta^3 (\mathbf p - \mathbf q) + \hat a_{\mathbf p}^\dagger \hat a_{\mathbf q}) \hat a_{\mathbf q} - \hat a_{\mathbf q}^\dagger ((2\pi)^3 \delta^3 (\mathbf p - \mathbf q) + \hat a_{\mathbf q}^\dagger \hat a_{\mathbf p}) \hat a_{\mathbf p}) \\ & = \int \frac{d^3 p ~ d^3 q}{(2\pi)^{\cancel{6}}} \omega_{\mathbf p} (\hat a_{\mathbf p}^\dagger \cancel{(2\pi)^3} \underbrace{\delta^3 (\mathbf p - \mathbf q)}_{\mathbf q = \mathbf p} \hat a_{\mathbf q} - \hat a_{\mathbf q}^\dagger \cancel{(2\pi)^3} \underbrace{\delta^3 (\mathbf p - \mathbf q)}_{\mathbf q = \mathbf p} \hat a_{\mathbf p}) \\ & \qquad + \int \frac{d^3 p ~ d^3 q}{(2\pi)^6} \omega_{\mathbf p} (\hat a_{\mathbf p}^\dagger \hat a_{\mathbf p}^\dagger \hat a_{\mathbf q} \hat a_{\mathbf q} - \hat a_{\mathbf q}^\dagger \hat a_{\mathbf q}^\dagger \hat a_{\mathbf p} \hat a_{\mathbf p} ) \\ & = \int \frac{d^3 p}{(2\pi)^3} \omega_{\mathbf p} (\cancel{\hat a_{\mathbf p}^\dagger \hat a_{\mathbf p}} - \cancel{\hat a_{\mathbf p}^\dagger \hat a_{\mathbf p}}) + \int \frac{d^3 p ~ d^3 q}{(2\pi)^6} \omega_{\mathbf p} (\hat a_{\mathbf p}^\dagger \hat a_{\mathbf p}^\dagger \hat a_{\mathbf q} \hat a_{\mathbf q} - \hat a_{\mathbf q}^\dagger \hat a_{\mathbf q}^\dagger \hat a_{\mathbf p} \hat a_{\mathbf p} )
        \end{aligned}
        \end{equation*}
        \begin{equation*}
        \begin{aligned}
            \phantom{[\hat H, \hat N]} & = \int \frac{d^3 p ~ d^3 q}{(2\pi)^6} \omega_{\mathbf p} (\hat a_{\mathbf p}^\dagger \hat a_{\mathbf p}^\dagger \hat a_{\mathbf q} \hat a_{\mathbf q} - \hat a_{\mathbf q}^\dagger \hat a_{\mathbf q}^\dagger \hat a_{\mathbf p} \hat a_{\mathbf p} ) \\ & = \int \frac{d^3 p ~ d^3 q}{(2\pi)^6} \omega_{\mathbf p} \hat a_{\mathbf p}^\dagger \hat a_{\mathbf p}^\dagger \hat a_{\mathbf q} \hat a_{\mathbf q} - \int \frac{d^3 p ~ d^3 q}{(2\pi)^6} \omega_{\mathbf p} \hat a_{\mathbf q}^\dagger \hat a_{\mathbf q}^\dagger \hat a_{\mathbf p} \hat a_{\mathbf p} \\ & = \int \frac{d^3 p ~ d^3 q}{(2\pi)^6} \omega_{\mathbf p} \hat a_{\mathbf p}^\dagger \hat a_{\mathbf p}^\dagger \hat a_{\mathbf q} \hat a_{\mathbf q} - \int \frac{d^3 p ~ d^3 q}{(2\pi)^6} \omega_{\mathbf q} \hat a_{\mathbf p}^\dagger \hat a_{\mathbf p}^\dagger \hat a_{\mathbf q} \hat a_{\mathbf q} \\ & = \int \frac{d^3 p ~ d^3 q}{(2\pi)^6} (\omega_{\mathbf p} - \omega_{\mathbf q}) \hat a_{\mathbf p}^\dagger \hat a_{\mathbf p}^\dagger \hat a_{\mathbf q} \hat a_{\mathbf q} \\ & = 
        \end{aligned}
        \end{equation*}
        where we have exchanged $\mathbf p \leftrightarrow \mathbf q$, since they are integral variables.

        TO COMPLETE!

    \end{proof}

\section{Lorentz covariance}

    The vacuum state is normalised 
    \begin{equation*}
        \braket{0}{0} = 1 ~,
    \end{equation*}
    while $1$-particle states satisfiy the orthogonality relation 
    \begin{equation*}
        \braket{\mathbf p}{\mathbf q} = (2\pi)^3 \delta^3 (\mathbf p - \mathbf q) 
    \end{equation*}
    and the completeness relation 
    \begin{equation*}
        \mathbb I = \int \frac{d^3 p}{(2 \pi)^3} \ket{\mathbf p} \bra{\mathbf p} ~,
    \end{equation*}
    where $\mathbb I$ is the identity operator.
    \begin{proof}
        Maybe in the future.
    \end{proof}

    However, we want Lorentz covariance, since the identity operator is so but the right side of the completeness relation is not, given that the measure $\int d^3 p$ and the projector $\ket{\mathbf p} \bra{\mathbf p}$ are not separately so. We know that $\in d^4 p$ is Lorentz covariant, because 
    \begin{equation*}
        d^4 p' = \underbrace{|\det \Lambda|}_1 d^4 p = d^4 p ~.
    \end{equation*}
    Therefore, we change the orthogonality relation into 
    \begin{equation*}
        \braket{p}{q} = (2\pi)^3 2 \sqrt{E_{\mathbf p} E_{\mathbf q}} \delta^3 (\mathbf p - \mathbf q) 
    \end{equation*}
    and the completeness relation into 
    \begin{equation*}
        \mathbb I = \int \frac{d^4 p}{(2\pi)^3} \delta (p^2_0 - |\mathbf p|^2 - m^2) \theta(p_0) \ket{\mathbf p} \bra{\mathbf p} ~,
    \end{equation*}
    where $p_0 = E_{\mathbf p} = \sqrt{|\mathbf p|^2 + m^2}$ and the manifestly invariant states are 
    \begin{equation}
        \ket{p} = \sqrt{2E_{\mathbf p}} \ket{\mathbf p} ~.
    \end{equation}
    \begin{proof}
        Maybe in the future.
    \end{proof}

\chapter{Two real (or complex) Klein-Gordon field}
\part{Dirac theory}

\chapter{Dirac action}

\section{Spinor representation of the Lorentz group}    

    The reducible representation of a Dirac spinor is 
    \begin{equation*}
        {\psi'}_D = \exp(-\frac{i}{2} \omega_{\mu\nu} \Sigma^{\mu\nu}) \psi_D ~,
    \end{equation*}
    where $\psi_D$ is a four-components complex vector, $\Sigma^{\mu\nu} = \frac{i}{4} [\gamma^\mu, \gamma^\nu]$ and the gamma matrices satisfy 
    \begin{equation*}
        \{\gamma^\mu, \gamma^\nu\} = 2 \eta^{\mu\nu} \mathbb I_4 ~.
    \end{equation*}
    In the Weyl basis, the Dirac matrices become 
    \begin{equation*}
        \gamma^0 = \begin{bmatrix}
            0 & \mathbb I_2 \\ \mathbb I_2 & 0 \\ 
        \end{bmatrix} ~.
    \end{equation*}
    It is useful to redefine the matrix 
    \begin{equation*}
        S^{\mu\nu} = - i \Sigma^{\mu\nu} = \frac{i}{4} [\gamma^\mu, \gamma^\nu] ~,
    \end{equation*}
    such that 
    \begin{equation}\label{lorspin}
        {\psi'}_D^\alpha (x) = \exp(\frac{1}{2} \omega_{\mu\nu} S^{\mu\nu})^\alpha_{\phantom \alpha \beta} \psi^\beta_D (x) = S^\alpha_\beta \psi^\beta_D (x) ~,
    \end{equation}
    where $\alpha, \beta = 1,2,3,4$.

    Notice that 
    \begin{equation}\label{gammadag}
        (\gamma^\mu)^\dagger = \gamma^0 \gamma^\mu \gamma^0 ~.
    \end{equation}
    \begin{proof}
        Maybe in the future.
    \end{proof}

\section{Invariant quantities}

    In order to have a Lorentz invariant action, we need to built Lorentz invariant quantities in function of $\psi$. Observe that the quantity $\psi \psi^\dagger$ is not a scalar. 
    \begin{proof}
        In fact, 
        \begin{equation*}
            {\psi'}^\dagger \psi' = \psi^\dagger S^\dagger S \psi \neq \psi^\dagger \psi ~,
        \end{equation*}
        since $S^\dagger \neq S^{-1}$.
    \end{proof}

    However, $S$ satisfies $S^\dagger = \gamma^0 S^{-1} \gamma^0$.
    \begin{proof}
        In fact, consider the spinor representation
        \begin{equation*}
            S = \exp ( \frac{1}{2} \omega_{\mu\nu} S^{\mu\nu}) ~,
        \end{equation*}
        its hermitian 
        \begin{equation*}
            S = \exp ( \frac{1}{2} \omega_{\mu\nu} (S^\dagger)^{\mu\nu})
        \end{equation*}
        and its inverse 
        \begin{equation*}
            S = \exp ( - \frac{1}{2} \omega_{\mu\nu} S^{\mu\nu}) ~.
        \end{equation*}
    \end{proof}

    We define the adjoint Dirac spinor 
    \begin{equation*}
        \overline \psi (x) = \psi^\dagger (x) \gamma^0 ~.
    \end{equation*}
    With this, we can construct a Lorentz invariant bilinear spinor 
    \begin{equation*}
        \overline \psi \psi ~.
    \end{equation*}
    \begin{proof}
        In fact, 
        \begin{equation*}
            \overline \psi' \psi' = {\psi'}^\dagger \gamma^0 \psi' = \psi^\dagger \underbrace{S^\dagger}_{\gamma^0 S^{-1} \gamma^0} \gamma^0 S \psi = = \psi^\dagger \gamma^0 S^{-1} \underbrace{\gamma^0 \gamma^0}_1 S \psi = \psi^\dagger \gamma^0 \underbrace{S^{-1} S}_1 \psi = \psi^\dagger \gamma^0 \psi = \overline \psi \psi ~.
        \end{equation*}
    \end{proof}

    Now, we want to build a $4$-vector $\overline \psi \gamma^\mu \psi$ such that
    \begin{equation*}
        \overline psi' \gamma^\mu \psi' = \overline \psi \Lambda^\mu_{\phantom \mu \nu} \gamma^\nu \psi ~,
    \end{equation*}
    or equivalently 
    \begin{equation*}
        S^{-1} \gamma^\mu S = \Lambda^\mu_{\phantom \mu \nu} \gamma^\nu ~.
    \end{equation*}
    \begin{proof}
        Maybe in the future.
    \end{proof}

    We obtained a Lorentz invariant scalar by contracting $\gamma^\mu$ with the first order derivative $\partial_\mu$. 

    Furthermore, $\Sigma^{\mu\nu}$ is a $2$-tensor 
    \begin{equation*}
        \overline psi' \Sigma^{\mu\nu} \psi' = \overline \psi \Lambda^\mu_{\phantom \mu \alpha} \Lambda^\nu_{\phantom \nu \beta} \Sigma^{\alpha\beta} \psi ~.
    \end{equation*}
    \begin{proof}
        Maybe in the future.
    \end{proof}

\section{Dirac action}

    Now, we have all the tools to build a Lorentz invariant lagrangian
    \begin{equation*}
        \mathcal L = \overline \psi (x) \gamma^\mu \partial_\mu \psi(x) - m \overline \psi (x) \psi (x) = \overline \psi (x) (i \gamma^\mu \partial_\mu - m) \psi(x) ~.
    \end{equation*}
    We have added an $i$ factor to ensure that $\mathcal L \in \mathbb R$.
    \begin{proof}
        Maybe in the future.
    \end{proof}

    The dymensional analysis is 
    \begin{equation*}
        [S] = 0 ~, [d^4 x] = - ~, [\mathcal L] = 4~, [\psi] = \frac{3}{2} ~, [\partial_\mu] = 1 ~, [m] = 1 ~.
    \end{equation*}
    Notice that in the Klein Gordon theory, we had $[\varphi] = 1$. However, in a renormalisable theory, the coupling between operators must be of dimension $4$. This means that only terms like $\varphi \overline \psi \psi$ are allowed. Another difference in the Dirac theory is that the lagrangian in at first order whereas in the Klein-Gordon theory is at second order. This is possible only because the gamma matrices exists only in the Dirac theory, while in the Klein-Gordon we have to constract to partial derivatives to get a scalar.

    The equations of motion can be obtained by the Euler-Lagrange equations: the Dirac equation is 
    \begin{equation*}
        (i \gamma^\mu \partial_\mu - m) \psi(x) = 0
    \end{equation*}
    and the conjugate Dirac equation is 
    \begin{equation*}
        \overline \psi(x) (i \gamma^\mu \overleftarrow{\partial_\mu} + m) = 0 ~.
    \end{equation*}
    \begin{proof}
        Maybe in the future.
    \end{proof}

\section{Dirac and Klein-Gordon equations}

    The four-components of the Dirac spinor satisfy the Dirac equation, but each components separately satisfy the Klein-Gordon equation, because it means that particles ensures the mass-shell condition.
    \begin{proof}
        Maybe in the future.
    \end{proof}

\chapter{Chiral spinors}

    Recall that the Dirac representation $(\frac{1}{2}, 0) \oplus (0, \frac{1}{2})$ is reducible and it can be decomposed into $2$ irreducible Weyl representations $(\frac{1}{2}, 0)$ and $(0, \frac{1}{2})$. 

    We introduce the $\gamma^5$ matrix 
    \begin{equation*}
        \gamma^5 = i \gamma^0 \gamma^1 \gamma^2 \gamma^3
    \end{equation*}
    such that it satisfies 
    \begin{equation*}
        \{\gamma^\mu, \gamma^5\} = 0~, \quad (\gamma^5)^2 = \mathbb I~, \quad (\gamma^5)^\dagger = \gamma^5 ~.
    \end{equation*}
    
    In the Weyl basis it becomes 
    \begin{equation*}
        \gamma^5 = \begin{bmatrix}
            - \mathbb I_2 & 0 \\ 0 & \mathbb I_2 \\
        \end{bmatrix} ~.
    \end{equation*}

    With $\gamma^5$, we can define the projection operators 
    \begin{equation*}
        P_L = \frac{\mathbb I - \gamma^5}{2} ~, \quad P_R = \frac{\mathbb I + \gamma^5}{2} ~.
    \end{equation*}
    such that they satisfy 
    \begin{equation*}
        P_L^2 = P_L ~, \quad P_R^2 = P_R ~\quad P_L^\dagger = P_L ~, \quad P_R^\dagger = P_R ~, \quad P_L P_R = P_R P_L = 0 ~, \quad P_L + P_R = \mathbb I ~.
    \end{equation*}
    and they decompose the Dirac spinor into a left-handed Weyl spinor $\psi_L^{(W)}$ and a right-handed Weyl spinor $\psi_R^{(W)}$
    \begin{equation*}
        \psi_L = \begin{bmatrix}
            \psi_L^{(W)} \\ 0 \\
        \end{bmatrix} = P_L \psi = \frac{\mathbb I - \gamma^5}{2} \psi ~, \quad \psi_R = \begin{bmatrix}
            0 \\ \psi_R^{(W)} \\
        \end{bmatrix} = P_R \psi = \frac{\mathbb I + \gamma^5}{2} \psi ~.
    \end{equation*}
    Furthermore, their eigenvalues are 
    \begin{equation*}
        \gamma^5 \psi_L = (-1) \psi_L ~, \quad \gamma^5 \psi_R = (+1) \psi_R ~.
    \end{equation*}

    The Dirac lagrangian in terms of the Weyl spinors is 
    \begin{equation*}
        \mathcal L = \overline \psi_L i \gamma^\mu \partial_\mu \psi_L + \overline \psi_R i \gamma^\mu \partial_\mu \psi_R - m (\overline \psi_L \psi_R + \overline \psi_R \psi_L) ~.
    \end{equation*}
    Notice that for a massive fermions, we do not know if it is right-handed or left-handed because of the last mixed term. Instead for massless fermions, we know.
    \begin{proof}
        Maybe in the future.
    \end{proof}

    In terms of the Weyl spinors, the Dirac equation becomes 
    \begin{equation*}
        \begin{cases}
            i \pdv{}{t} \psi^{(W)}_R (x) + i \boldsymbol \sigma \cdot \boldsymbol \nabla \psi_R^{(W)} - m \psi_L^{(W)} = 0 \\
            i \pdv{}{t} \psi^{(W)}_L (x) + i \boldsymbol \sigma \cdot \boldsymbol \nabla \psi_L^{(W)} - m \psi_R^{(W)} = 0 
        \end{cases} ~.
    \end{equation*}
    \begin{proof}
        Maybe in the future.
    \end{proof}

    For massless fermions, which have a hamiltonian $\hat H = |\hat p|$, the Weyl equations become
    \begin{equation*}
        \begin{cases}
            (\hat{\mathbf S} \cdot \mathbf p) \psi^{(W)}_R (x) = (+1) \psi^{(W)}_R (x) \\
            (\hat{\mathbf S} \cdot \mathbf p) \psi^{(W)}_L (x) = (-1) \psi^{(W)}_L (x) \\
        \end{cases}
    \end{equation*}
    where $\mathbf p$ is the direction of motion and $\hat S$ is the spin operator. The quantity $\hat{\mathbf S} \cdot \mathbf p$ is called helicity and it is the projection of the spin along the direction of motion. 
    \begin{proof}
        Maybe in the future.
    \end{proof}

\section{Parity} 

    The parity operator transforms a right-handed Weyl spinor into a left-handed Weyl spinor and viceversa
    \begin{equation*}
        \begin{cases}
            {\psi'}_L^{(W)} = \psi_R^{(W)} \\
            {\psi'}_R^{(W)} = \psi_L^{(W)} \\
        \end{cases} ~.
    \end{equation*}
    \begin{proof}
        Maybe in the future.
    \end{proof}

\chapter{Solutions of the Dirac equation}

    Since each components of the Dirac spinor $\psi(x)$ satisfies the Klein-Gordon equation, the plane waves are solutions 
    \begin{equation*}
        \psi_\alpha (x) = u_\alpha (\mathbf p) \exp(- i p x) ~,
    \end{equation*}
    where $u_\alpha (\mathbf p)$ is the polarisation vector with $4$ components $\alpha = 1,2,3,4$ and $p_0 = E_{\mathbf p} = \sqrt{|\mathbf p|^2 + m^2}$. Furthermore, in order to satisfy the Dirac equation, $u_\alpha (\mathbf p)$ satisfies 
    \begin{equation}\label{cond}
        \begin{bmatrix}
            - m & p^\mu \sigma_\mu \\ p^\mu \overline \sigma_\mu & -m \\
        \end{bmatrix} u (\mathbf p) = 0 ~,
    \end{equation}
    where $\sigma^\mu = (\mathbb I_2, \sigma^i)$ and $\overline \sigma^\mu = (\mathbb I_2, - \sigma^i)$.
    \begin{proof}
        In fact, 
        \begin{equation*}
            0 = (i \gamma^\mu \partial_\mu - m) \psi(x) = (i \gamma^\mu (- i p_\mu ) - m) u(\mathbf p) \exp(- i p x) ~.
        \end{equation*}
        Hence 
        \begin{equation*}
            0 = (\gamma^\mu  p_\mu - m) u(\mathbf p) = \Big ( 
            \begin{bmatrix}
                0 & 1 \\ 1 & 0 \\ 
            \end{bmatrix} p_0 + \begin{bmatrix}
                0 & \sigma^i \\ - \sigma^i & 0
            \end{bmatrix} p_i  - m \begin{bmatrix}
                1 & 0 \\ 0 & 1 \\
            \end{bmatrix} \Big) u (\mathbf p) = \begin{bmatrix}
                - m & p^\mu \sigma_\mu \\ p^\mu \overline \sigma_\mu & -m \\
            \end{bmatrix} u (\mathbf p) = 0 ~. 
        \end{equation*}
    \end{proof}

    Moreover, we can split the polarisation vector $u (\mathbf p)$ into the right and the left-handed part 
    \begin{equation}\label{split}
        u (\mathbf p) = \begin{bmatrix}
            u_L (\mathbf p) \\ u_R (\mathbf p) \\
        \end{bmatrix} ~,
    \end{equation}
    which can be intepret as the positive frequency solution. 
    \begin{proof}
        In fact, putting~\eqref{split} into~\eqref{cond} 
        \begin{equation}\label{proof4}
            \begin{cases}
                (p^\mu \overline \sigma_\mu) u_L = m u_R \\
                (p^\mu \sigma_\mu) u_R = m u_L \\
            \end{cases} ~.
        \end{equation}

        Notice that 
        \begin{equation*}
        \begin{aligned}
            (p_\mu \sigma^\mu) (p_\nu \overline \sigma^\nu) & = (p_0 + p_i \sigma^i) (p_0 + p_j \overline \sigma^j) \\ & = (p_0 + p_i \sigma^i) (p_0 - p_j \sigma^j) \\ & = p_0^2 - p_i p_j \underbrace{\sigma^i \sigma^j}_{\delta^{ij} + i \epsilon^{ijk} \sigma_k} \\ & = p_0^2 - p_i p_j \underbrace{\delta^{ij}}_{i = j} + \cancel{i \underbrace{p_i p_j}_{symm} \underbrace{\epsilon^{ijk}}_{anti} \sigma_k} \\ & = p_0^2 - |\mathbf p|^2 = m^2 ~.
        \end{aligned}
        \end{equation*}

        We choose the form of $u_L$ such that 
        \begin{equation*}
            u_L (\mathbf p) = A p^\mu \sigma_\mu \chi ~,
        \end{equation*}
        where $A$ is a constant and $\chi$ is $2$-components spinor. 
        Hence, the first equation of~\eqref{proof4}
        \begin{equation*}
            m u_R = (p^\mu \overline \sigma_\mu) u_L = A \underbrace{(p^\mu \overline \sigma_\mu) (p^\nu \sigma_\nu)}_{m^2} \chi = A m^2 \chi
        \end{equation*}
        and 
        \begin{equation*}
            u_R (\mathbf p) = m A \chi.
        \end{equation*}
        In this way, the second equation of~\eqref{proof4} is automatically satisfied
        \begin{equation*}
            m u_L = p^\mu \sigma_\mu \underbrace{m A \chi }_{u_R} = p^\mu \sigma_\mu u_R ~.
        \end{equation*}
        Therefore 
        \begin{equation*}
            u (\mathbf p) = A \begin{bmatrix}
                (p^\mu \sigma_\mu) \chi \\ m \chi \\
            \end{bmatrix} ~.
        \end{equation*} 

        We choose $A = \frac{1}{m}$ and $\chi = \sqrt{p^\mu \overline \sigma_\mu} \xi$, where $\xi$ is a constant $2$-components spinor normalised such that $\xi^\dagger \xi = 1$. Hence 
        \begin{equation*}
            u (\mathbf p) = \frac{1}{m} \begin{bmatrix}
                (p^\mu \sigma_\mu) \sqrt{p^\nu \overline \sigma_\nu} \xi \\ m \sqrt{p^\mu \overline \sigma_\mu} \xi \\
            \end{bmatrix} = \frac{1}{m} \begin{bmatrix}
                \sqrt{p^\mu \sigma_\mu} \underbrace{\sqrt{p^\alpha \sigma_\alpha} \sqrt{p^\nu \overline \sigma_\nu}}_m \xi \\ m \sqrt{p^\mu \overline \sigma_\mu} \xi \\
            \end{bmatrix} = \begin{bmatrix}
                \sqrt{p^\mu \sigma_\mu} \xi \\ \sqrt{p^\mu \overline \sigma_\mu} \xi \\
            \end{bmatrix} ~.
            \end{equation*}
    \end{proof}

    Actually, there is another class of plane waves solutions, the negative frequency solutions 
    \begin{equation*}
        \psi(x) = v(\mathbf p) \exp(i p x) ~,
    \end{equation*}
    where $v (\mathbf p)$ is the polarisation vector 
    \begin{equation*}
        v (\mathbf p) = \begin{bmatrix}
            \sqrt{p_\mu \sigma^\mu} \eta \\
            - \sqrt{p_\mu \overline \sigma^\mu} \eta \\
        \end{bmatrix} ~,
    \end{equation*}
    where $\eta$ is a constant $2$-components spinor normalised such that $\eta^\dagger \eta = 1$. 

    They can be distinguished since 
    \begin{equation*}
        (\gamma^\mu p_\mu - m) u(\mathbf p) = 0 ~, \quad (\gamma^\mu p_\mu + m) v(\mathbf p) = 0 ~,
    \end{equation*}
    and 
    \begin{equation*}
        \hat H \psi(x) = i \pdv{}{t} (u (\mathbf p) \exp(- i px)) = E_{\mathbf p} \psi (x) ~, \quad \hat H \psi(x) = i \pdv{}{t} (u (\mathbf p) \exp(i px)) = - E_{\mathbf p} \psi (x) ~.
    \end{equation*}

    Consider a massive particle in the rest frame $p^\mu = (m, 0, 0,0)$. The positive frequency solutions look like 
    \begin{equation*}
        \psi(x) = \sqrt{m} \exp(- i E_{\mathbf p} t) \begin{bmatrix}
            \xi \\ \xi \\
        \end{bmatrix} ~.
    \end{equation*}
    Using~\eqref{lorspin}, we restrict to spatial rotations in which the generators are
    \begin{equation*}
        S^{ij} = - \frac{1}{2} \epsilon^{ijk} \begin{bmatrix}
            \sigma^k & 0 \\ 0 & \sigma^k \\
        \end{bmatrix} ~,
    \end{equation*}
    where $i \neq j$ and the parameters are 
    \begin{equation*}
        \omega_{ij} = - \epsilon_{ijk} \theta^k ~.
    \end{equation*}
    Therefore, the matrix rotation is 
    \begin{equation*}
        \exp(\frac{1}{2} \omega_{ij} S^{ij}) = \begin{bmatrix}
            \exp(\frac{i}{2} \theta^i \sigma_i) & 0 \\ 0 & \exp(\frac{i}{2} \theta^i \sigma_i) \\
        \end{bmatrix}
    \end{equation*}
    and the Dirac spinor transforms as 
    \begin{equation*}
        \psi' (x) = \begin{bmatrix}
            \exp(\frac{i}{2} \theta^i \sigma_i) & 0 \\ 0 & \exp(\frac{i}{2} \theta^i \sigma_i) \\
        \end{bmatrix} \psi (x) ~,
    \end{equation*}
    which induces a transformation on $\xi$ such that 
    \begin{equation*}
        \xi ' = \exp(\frac{i}{2} \theta^i \sigma_i) \xi ~.
    \end{equation*}
    This is indeed an $SU(2)$ transformation, where we can recognise tha spin operator $\hat{\mathbf S} = \frac{1}{2} \boldsymbol \sigma$ and $\xi$ is a $2$-components spinot which describes particle with spin $\frac{1}{2}$. Since $\xi^\dagger \xi = 1$, we choose, for the spin up
    \begin{equation*}
        \xi = \begin{bmatrix}
            1 \\ 0 \\
        \end{bmatrix} \colon \sigma_3 \xi = \begin{bmatrix}
            1 & 0 \\ 0 & -1 \\
        \end{bmatrix} \begin{bmatrix}
            1 \\ 0 \\
        \end{bmatrix} = (+1) \xi ~,
    \end{equation*}
    for the spin down
    \begin{equation*}
        \xi = \begin{bmatrix}
            0 \\ 1 \\
        \end{bmatrix} \colon \sigma_3 \xi = \begin{bmatrix}
            1 & 0 \\ 0 & -1 \\
        \end{bmatrix} \begin{bmatrix}
            0 \\ 1 \\
        \end{bmatrix} = (-1) \xi ~,
    \end{equation*}

\chapter{Useful formulas} 

\section{Inner product}

    We introduce a basis of the $2$-components spinors 
    \begin{equation*}
        \xi^r ~, \eta^s ~,
    \end{equation*}
    where $r,s = 1,2$ such that they satisfy 
    \begin{equation*}
        (\xi^\dagger)^r \xi^s = \delta^{rs} ~, \quad (\eta^\dagger)^r \eta^s = \delta^{rs}  ~.
    \end{equation*}

    For example, 
    \begin{equation*}
        \xi^1 = \begin{bmatrix}
            1 \\ 0 \\
        \end{bmatrix} ~, \quad \xi^2 = \begin{bmatrix}
            0 \\ 1 \\
        \end{bmatrix} ~, \quad \eta^1 = \begin{bmatrix}
            1 \\ 0 \\
        \end{bmatrix} ~, \quad \eta^2 = \begin{bmatrix}
            0 \\ 1 \\
        \end{bmatrix} ~.
    \end{equation*}

    We define the following inner products 
    \begin{enumerate}
        \item \begin{equation*}
            (u^\dagger)^r (\mathbf p) u^s (\mathbf p) = 2 p^0 \delta^{rs} ~,
        \end{equation*}
        \item \begin{equation*}
            \overline u^r (\mathbf p) u^s (\mathbf p) = 2 m \delta^{rs} ~,
        \end{equation*} 
        \item \begin{equation*}
            (v^\dagger)^r (\mathbf p) v^s (\mathbf p) = 2 p^0 \delta^{rs} ~,
        \end{equation*}
        \item \begin{equation*}
            \overline v^r (\mathbf p) v^s (\mathbf p) = - 2 m \delta^{rs} ~,
        \end{equation*}
        \item \begin{equation*}
            \overline u^r (\mathbf p) v^s (\mathbf p) = \overline v^r (\mathbf p) u^s (\mathbf p) = 0 ~,
        \end{equation*}
        \item \begin{equation*}
            (u^\dagger)^r (\mathbf p) v^s (- \mathbf p) = (v^\dagger)^r (\mathbf p) u^s (- \mathbf p) = 0 ~.
        \end{equation*}
    \end{enumerate}
    \begin{proof}
        For the first one, 
        \begin{equation*}
        \begin{aligned}
            (u^\dagger)^r (\mathbf p) u^s (\mathbf p) & = \begin{bmatrix}
                \sqrt{p^\mu \sigma_\mu} (\xi^\dagger)^r & \sqrt{p^\mu \overline \sigma_\mu} (\xi^\dagger)^r \\
            \end{bmatrix} \begin{bmatrix}
                \sqrt{p^\mu \sigma_\mu} \xi^s \\ \sqrt{p^\mu \overline \sigma_\mu} \xi^s \\
            \end{bmatrix} \\ & = (\xi^\dagger)^r p^\mu \sigma_\mu \xi^s + (\xi^\dagger)^r p^\mu \overline \sigma_\mu \xi^s \\ & = (\xi^\dagger)^r p^0 \underbrace{\sigma_0}_{\mathbb I_2} \xi^s + (\xi^\dagger)^r p^0 \underbrace{\overline \sigma_0}_{\mathbb I_2} \xi^s + (\xi^\dagger)^r p^i \sigma_i \xi^s + (\xi^\dagger)^r p^i \underbrace{\overline \sigma_i}_{- \sigma_i} \xi^s \\ & = (\xi^\dagger)^r p^0 \xi^s + (\xi^\dagger)^r p^0 \xi^s + \cancel{(\xi^\dagger)^r p^i \sigma_i \xi^s } - \cancel{(\xi^\dagger)^r p^i \sigma_i \xi^s} \\ & = 2 p_0 \underbrace{(\xi^\dagger)^r \xi^s}_{\delta^{rs}} \\ & = 2 p_0 \delta^{rs} ~.
        \end{aligned}
        \end{equation*}

        For the second one, 
        \begin{equation*}
        \begin{aligned}
            \overline u^r (\mathbf p) u^s (\mathbf p) & = \begin{bmatrix}
                \sqrt{p^\mu \sigma_\mu} (\xi^\dagger)^r & \sqrt{p^\mu \overline \sigma_\mu} (\xi^\dagger)^r \\
            \end{bmatrix} \begin{bmatrix}
                0 & 1 \\ 1 & 0 \\
            \end{bmatrix} \begin{bmatrix}
                \sqrt{p^\mu \sigma_\mu} \xi^s \\ \sqrt{p^\mu \overline \sigma_\mu} \xi^s \\
            \end{bmatrix} \\ & = (\xi^\dagger)^r p^\mu \underbrace{\sqrt{p^\mu  \sigma_\mu} \sqrt{p^\nu \overline \sigma_\nu}}_m \xi^s + (\xi^\dagger)^r \underbrace{\sqrt{p^\mu \sigma_\mu} \sqrt{p^\nu \overline \sigma_\nu}}_m \xi^s \\ & = 2 m \underbrace{(\xi^\dagger)^r \xi^s}_{\delta^{rs}} \\ & 2 m = \delta^{rs} ~.
        \end{aligned}
        \end{equation*}

        For the third one, 
        \begin{equation*}
        \begin{aligned}
            (v^\dagger)^r (\mathbf p) v^s (\mathbf p) & = \begin{bmatrix}
                \sqrt{p^\mu \sigma_\mu} (\eta^\dagger)^r & - \sqrt{p^\mu \overline \sigma_\mu} (\eta^\dagger)^r \\
            \end{bmatrix} \begin{bmatrix}
                \sqrt{p^\mu \sigma_\mu} \eta^s \\ - \sqrt{p^\mu \overline \sigma_\mu} \eta^s \\
            \end{bmatrix} \\ & = (\eta^\dagger)^r p^\mu \sigma_\mu \eta^s + (\eta^\dagger)^r p^\mu \overline \sigma_\mu \eta^s \\ & = (\eta^\dagger)^r p^0 \underbrace{\sigma_0}_{\mathbb I_2} \eta^s + (\eta^\dagger)^r p^0 \underbrace{\overline \sigma_0}_{\mathbb I_2} \eta^s + (\eta^\dagger)^r p^i \sigma_i \eta^s + (\eta^\dagger)^r p^i \underbrace{\overline \sigma_i}_{- \sigma_i} \eta^s \\ & = (\eta^\dagger)^r p^0 \eta^s + (\eta^\dagger)^r p^0 \eta^s + \cancel{(\eta^\dagger)^r p^i \sigma_i \eta^s } - \cancel{(\eta^\dagger)^r p^i \sigma_i \eta^s} \\ & = 2 p_0 \underbrace{(\eta^\dagger)^r \eta^s}_{\delta^{rs}} \\ & = 2 p_0 \delta^{rs} ~.
        \end{aligned}
        \end{equation*}

        For the fourth one, 
        \begin{equation*}
        \begin{aligned}
            \overline v^r (\mathbf p) v^s (\mathbf p) & = \begin{bmatrix}
                \sqrt{p^\mu \sigma_\mu} (\eta^\dagger)^r & - \sqrt{p^\mu \overline \sigma_\mu} (\eta^\dagger)^r \\
            \end{bmatrix} \begin{bmatrix}
                0 & 1 \\ 1 & 0 \\
            \end{bmatrix} \begin{bmatrix}
                \sqrt{p^\mu \sigma_\mu} \eta^s \\ - \sqrt{p^\mu \overline \sigma_\mu} \eta^s \\
            \end{bmatrix} \\ & = - (\eta^\dagger)^r p^\mu \underbrace{\sqrt{p^\mu  \sigma_\mu} \sqrt{p^\nu \overline \sigma_\nu}}_m \eta^s - (\eta^\dagger)^r \underbrace{\sqrt{p^\mu \sigma_\mu} \sqrt{p^\nu \overline \sigma_\nu}}_m \eta^s \\ & = - 2 m \underbrace{(\eta^\dagger)^r \eta^s}_{\delta^{rs}} \\ & - 2 m = \delta^{rs} ~.
        \end{aligned}
        \end{equation*}

        For the fifth one, 
        \begin{equation*}
        \begin{aligned}
            \overline u^r (\mathbf p) v^s (\mathbf p) & = \begin{bmatrix}
                \sqrt{p^\mu \sigma_\mu} (\xi^\dagger)^r & \sqrt{p^\mu \overline \sigma_\mu} (\xi^\dagger)^r \\
            \end{bmatrix} \begin{bmatrix}
                0 & 1 \\ 1 & 0 \\
            \end{bmatrix} \begin{bmatrix}
                \sqrt{p^\mu \sigma_\mu} \eta^s \\ - \sqrt{p^\mu \overline \sigma_\mu} \eta^s \\
            \end{bmatrix} \\ & = - (\xi^\dagger)^r \underbrace{\sqrt{p^\mu  \sigma_\mu} \sqrt{p^\nu \overline \sigma_\nu}}_m \eta^s + (\xi^\dagger)^r \underbrace{\sqrt{p^\mu \sigma_\mu} \sqrt{p^\nu \overline \sigma_\nu}}_m \eta^s \\ & = m ( \cancel{(- \xi^\dagger)^r \eta^s} + \cancel{(\xi^\dagger)^r \eta^2}) = 0 ~.
        \end{aligned}
        \end{equation*}

        For the second in the fifth one, 
        \begin{equation*}
        \begin{aligned}
            \overline v^r (\mathbf p) u^s (\mathbf p) & = \begin{bmatrix}
                \sqrt{p^\mu \sigma_\mu} (\eta^\dagger)^r & \sqrt{p^\mu \overline \sigma_\mu} (\eta^\dagger)^r \\
            \end{bmatrix} \begin{bmatrix}
                0 & 1 \\ 1 & 0 \\
            \end{bmatrix} \begin{bmatrix}
                \sqrt{p^\mu \sigma_\mu} \xi^s \\ - \sqrt{p^\mu \overline \sigma_\mu} \xi^s \\
            \end{bmatrix} \\ & = - (\eta^\dagger)^r \underbrace{\sqrt{p^\mu  \sigma_\mu} \sqrt{p^\nu \overline \sigma_\nu}}_m \xi^s + (\eta^\dagger)^r \underbrace{\sqrt{p^\mu \sigma_\mu} \sqrt{p^\nu \overline \sigma_\nu}}_m \xi^s \\ & = m ( \cancel{(- \eta^\dagger)^r \xi^s} + \cancel{(\eta^\dagger)^r \xi^2}) = 0 ~.
        \end{aligned}
        \end{equation*}

        For the sixth one, 
        \begin{equation*}
        \begin{aligned}
            (u^\dagger)^r (\mathbf p) v^s (- \mathbf p) & = \begin{bmatrix}
                \sqrt{p^\mu \sigma_\mu} (\xi^\dagger)^r & \sqrt{p^\mu \overline \sigma_\mu} (\xi^\dagger)^r \\
            \end{bmatrix} \begin{bmatrix}
                \sqrt{\overline p^\mu \sigma_\mu} \eta^s \\ - \sqrt{\overline p^\mu \overline \sigma_\mu} \eta^s \\
            \end{bmatrix} \\ & = (\xi^\dagger)^r \underbrace{\sqrt{p^\mu \sigma_\mu} \sqrt{\overline p^\mu \sigma_\mu}}_m \eta^s - (\xi^\dagger)^r \underbrace{\sqrt{p^\mu \overline \sigma_\mu} \sqrt{\overline p^\mu \overline \sigma_\mu}}_m \eta^s \\ & = m ((\xi^\dagger)^r \eta^s - (\xi^\dagger)^r \eta^s) = 0 ~.
        \end{aligned}
        \end{equation*}

        For the second in the sixth one, 
        \begin{equation*}
        \begin{aligned}
            (v^\dagger)^r (\mathbf p) u^s (- \mathbf p) & = \begin{bmatrix}
                \sqrt{p^\mu \sigma_\mu} (\eta^\dagger)^r & - \sqrt{p^\mu \overline \sigma_\mu} (\eta^\dagger)^r \\
            \end{bmatrix} \begin{bmatrix}
                \sqrt{\overline p^\mu \sigma_\mu} \xi^s \\ \sqrt{\overline p^\mu \overline \sigma_\mu} \xi^s \\
            \end{bmatrix} \\ & = (\eta^\dagger)^r \underbrace{\sqrt{p^\mu \sigma_\mu} \sqrt{\overline p^\mu \sigma_\mu}}_m \xi^s - (\eta^\dagger)^r \underbrace{\sqrt{p^\mu \overline \sigma_\mu} \sqrt{\overline p^\mu \overline \sigma_\mu}}_m \xi^s \\ & = m ((\eta^\dagger)^r \xi^s - (\eta^\dagger)^r \xi^s) = 0 ~.
        \end{aligned}
        \end{equation*}
    \end{proof}

\section{Outer product}

    We define the following outer products 
    \begin{enumerate}
        \item \begin{equation*}
            \sum_{s=1}^{2} u^s (\mathbf p) \overline u^s(\mathbf p) = \gamma^\mu p_\mu + m \mathbb I_4 ~,
        \end{equation*} 
        \item \begin{equation*}
            \sum_{s=1}^{2} v^s (\mathbf p) \overline v^s(\mathbf p) = \gamma^\mu p_\mu - m \mathbb I_4 ~.
        \end{equation*} 
    \end{enumerate}
    \begin{proof}
        For the first one
        \begin{equation*}
        \begin{aligned}
            \sum_{s=1}^{2} u^s (\mathbf p) \overline u^s(\mathbf p) & = \sum_{s=1}^{2} \begin{bmatrix}
                \sqrt{ p^\mu \sigma_\mu} \xi^s \\ \sqrt{ p^\mu \overline \sigma_\mu} \xi^s \\
            \end{bmatrix} \begin{bmatrix}
                \sqrt{p^\mu \sigma_\mu} (\xi^\dagger)^r & \sqrt{p^\mu \overline \sigma_\mu} (\xi^\dagger)^r \\
            \end{bmatrix} \begin{bmatrix}
                0 & 1 \\ 1 & 0 \\
            \end{bmatrix} \\ & = \sum_{s=1}^{2} \begin{bmatrix}
                \sqrt{p^\mu \sigma_\mu} \xi^s (\xi^\dagger)^s \sqrt{p^\mu \overline \sigma_\mu} & \sqrt{p^\mu \sigma_\mu} \xi^s (\xi^\dagger)^s \sqrt{p^\mu \sigma_\mu} \\ \sqrt{p^\mu \overline \sigma_\mu} \xi^s (\xi^\dagger)^s \sqrt{p^\mu \overline \sigma_\mu} & \sqrt{p^\mu \overline \sigma_\mu} \xi^s (\xi^\dagger)^s \sqrt{p^\mu \sigma_\mu} 
            \end{bmatrix} \\ & = \begin{bmatrix}
                \underbrace{\sqrt{p^\mu \sigma_\mu} \sqrt{p^\mu \overline \sigma_\mu}}_m & \sqrt{p^\mu \sigma_\mu} \sqrt{p^\mu \sigma_\mu} \\ \sqrt{p^\mu \overline \sigma_\mu} \sqrt{p^\mu \overline \sigma_\mu} & \underbrace{\sqrt{p^\mu \overline \sigma_\mu} \sqrt{p^\mu \sigma_\mu}}_m
            \end{bmatrix} \\ & = \begin{bmatrix}
                m & p^\mu \sigma_\mu \\ p^\mu \overline \sigma_\mu & m \\
            \end{bmatrix} \\ & = \gamma^\mu p_\mu + m ~,
        \end{aligned}
        \end{equation*}
        where we have used 
        \begin{equation*}
            \sum_{s=1}^{2} \xi^s (\xi^\dagger)^s = \mathbb I_2 ~.
        \end{equation*}

        For the first one
        \begin{equation*}
        \begin{aligned}
            \sum_{s=1}^{2} v^s (\mathbf p) \overline v^s(\mathbf p) & = \sum_{s=1}^{2} \begin{bmatrix}
                \sqrt{ p^\mu \sigma_\mu} \eta^s \\ - \sqrt{ p^\mu \overline \sigma_\mu} \eta^s \\
            \end{bmatrix} \begin{bmatrix}
                \sqrt{p^\mu \sigma_\mu} (\eta^\dagger)^r & - \sqrt{p^\mu \overline \sigma_\mu} (\eta^\dagger)^r \\
            \end{bmatrix} \begin{bmatrix}
                0 & 1 \\ 1 & 0 \\
            \end{bmatrix} \\ & = \sum_{s=1}^{2} \begin{bmatrix}
                - \sqrt{p^\mu \sigma_\mu} \eta^s (\eta^\dagger)^s \sqrt{p^\mu \overline \sigma_\mu} & \sqrt{p^\mu \sigma_\mu} \eta^s (\eta^\dagger)^s \sqrt{p^\mu \sigma_\mu} \\ \sqrt{p^\mu \overline \sigma_\mu} \eta^s (\eta^\dagger)^s \sqrt{p^\mu \overline \sigma_\mu} & - \sqrt{p^\mu \overline \sigma_\mu} \eta^s (\eta^\dagger)^s \sqrt{p^\mu \sigma_\mu} 
            \end{bmatrix} \\ & = \begin{bmatrix}
                - \underbrace{\sqrt{p^\mu \sigma_\mu} \sqrt{p^\mu \overline \sigma_\mu}}_m & \sqrt{p^\mu \sigma_\mu} \sqrt{p^\mu \sigma_\mu} \\ \sqrt{p^\mu \overline \sigma_\mu} \sqrt{p^\mu \overline \sigma_\mu} & -\underbrace{\sqrt{p^\mu \overline \sigma_\mu} \sqrt{p^\mu \sigma_\mu}}_m 
            \end{bmatrix} \\ & = \begin{bmatrix}
                - m & p^\mu \sigma_\mu \\ p^\mu \overline \sigma_\mu & - m \\
            \end{bmatrix} \\ & = \gamma^\mu p_\mu - m ~,
        \end{aligned}
        \end{equation*}
        where we have used 
        \begin{equation*}
            \sum_{s=1}^{2} \eta^s (\eta^\dagger)^s = \mathbb I_2 ~.
        \end{equation*}
    \end{proof}

\chapter{How to not quantise the Dirac theory}
\part{Maxwell's theory}

\chapter{Maxwell's action}

    In this chapter, we will review some notion of classical electrodynamics: Maxwell's Lagrangian that leads to the Maxwell's equations and how, using gauge symmetry in vacuum, an electromagnetic wave carries only two degrees of freedom, which are the two transversal polarisations indepependent components.

\section{Maxwell's Lagrangian}

    Classical free electrodynamic fields, without sources, can be describe starting from the Maxwell's Lagrangian
    \begin{equation}\label{maxlag}
    \begin{aligned}
        \mathcal L & = - \frac{1}{4} F_{\mu\nu} F^{\mu\nu} = - \frac{1}{4} (\partial_\mu A_\nu - \partial_\nu A_\mu) (\partial^\mu A^\nu - \partial^\nu A^\mu) \\ & = - \frac{1}{4} ( \partial_\mu A_\nu \partial^\mu A^\nu - \partial_\nu A_\mu \partial^\mu A^\nu - \partial_\mu A_\nu \partial^\nu A^\mu + \partial_\nu A_\mu  \partial^\nu A^\mu ) \\ & = - \frac{1}{2} (\partial_\mu A_\nu \partial^\mu A^\nu - \partial_\nu A_\mu \partial^\mu A^\nu) ~,
    \end{aligned}
    \end{equation}
    where $F_{\mu\nu}$ is the electromagnetic tensor
    \begin{equation*}
        F_{\mu\nu} = \partial_\mu A_\nu - \partial_\nu A_\mu = \begin{bmatrix}
            0 & E_1 & E_2 & E_3 \\ 
            -E_1 & 0 & - B_3 & B_2 \\ 
            - E_2 & B_3 & 0 & - B_1 \\ 
            - E_3 & -B_2 & B_1 & 0 \\
        \end{bmatrix}
    \end{equation*}
    and the $4$-potential is $A^\mu = (\phi, \mathbf A)$. Recall that they are related to the electric and the magnetic field via 
    \begin{equation}\label{ef}
        \mathbf B = \boldsymbol \nabla \times \mathbf A ~, \quad \mathbf E = - \boldsymbol \nabla \phi - \pdv{\mathbf A}{t} 
    \end{equation}
    and we can write them in terms of the electromagnetic tensor as 
    \begin{equation*}
        E_i = F_{0i} ~, \quad B_i = \frac{1}{2} \epsilon_{ijk} F^{jk} ~.
    \end{equation*}
    From this Lagrangian, we recover the equation of motion, which are exactly Maxwell's equations in vacuum
    \begin{equation*}
        \partial_\mu F^{\mu\nu} = 0 ~.
    \end{equation*}
    \begin{proof}
        Using~\eqref{eleq} and~\eqref{maxlag}, we have
        \begin{equation*}
        \begin{aligned}
            0 & = \partial_\mu \pdv{\mathcal L}{\partial_\mu A_\nu} - \underbrace{\pdv{\mathcal L}{A_\nu}}_0 = \partial_\mu \pdv{}{\partial_\mu A_\nu} (- \frac{1}{2} (\partial_\alpha A_\beta \partial^\alpha A^\beta - \partial_\beta A_\alpha \partial^\alpha A^\beta)) \\ & = - \partial_\mu (\partial^\mu A^\nu - \partial^\nu A^\nu) = - \partial_\mu F^{\mu\nu} ~.
        \end{aligned}
        \end{equation*}
    \end{proof}
    An useful property of the electromagnetic tensor is that it satisfies the Bianchi identity 
    \begin{equation*}
        \partial_\mu F_{\nu\lambda} + \partial_\nu F_{\lambda \mu} + \partial_\lambda F_{\mu \nu} = 0 ~.
    \end{equation*}
    \begin{proof}
        In fact, we have 
        \begin{equation*}
        \begin{aligned}
            \partial_\mu F_{\nu\lambda} + \partial_\nu F_{\lambda \mu} + \partial_\lambda F_{\mu \nu} & = \partial_\mu (\partial_\nu A_\lambda - \partial_\lambda A_\nu) + \partial_\nu (\partial_\lambda A_\mu - \partial_\mu A_\lambda) + \partial_\lambda (\partial_\mu A_\nu - \partial_\nu A_\mu) \\ & = \partial_\mu \partial_\nu A_\lambda -  \partial_\mu \partial_\lambda A_\nu + \partial_\nu \partial_\lambda A_\mu - \partial_\nu  \partial_\mu A_\lambda + \partial_\lambda \partial_\mu A_\nu - \partial_\lambda \partial_\nu A_\mu \\ & = \cancel{\partial_\nu \partial_\mu A_\lambda} - \cancel{\partial_\lambda \partial_\mu A_\nu} +  \cancel{\partial_\lambda \partial_\nu A_\mu} - \cancel{\partial_\nu \partial_\mu A_\lambda} + \cancel{\partial_\lambda \partial_\mu A_\nu} - \cancel{\partial_\lambda \partial_\nu A_\mu} = 0 ~,
        \end{aligned}
        \end{equation*}
        where we have used the fact that partial derivatives commute.
    \end{proof}
    We define the dual electromagnetic tensor
    \begin{equation*}
        \tilde F^{\mu\nu} = - \frac{1}{2} \epsilon^{\mu\nu\rho\sigma} F_{\rho\sigma} ~,
    \end{equation*}
    or, explicitly, 
    \begin{equation*}
        \tilde F_{\mu\nu} = \begin{bmatrix}
            0 & - B_1 & - B_2 & - B_3 \\ 
            B_1 & 0 & E_3 & -E_2 \\ 
            B_2 & -E_3 & 0 & E_1 \\ 
            B_3 & E_2 & -E_1 & 0 \\
        \end{bmatrix} ~.
    \end{equation*}
    Notice that there is a duality symmetry, since if we perfom the exhange $\mathbf E \leftrightarrow - \mathbf B$, we find $F^{\mu\nu} \leftrightarrow \tilde F^{\mu\nu}$. We can write the Bianchi identity in terms of the dual tensor as
    \begin{equation*}
        \partial_\mu \tilde F^{\mu\nu} = - \frac{1}{2} \epsilon^{\mu\nu\rho\sigma} \partial_\mu  F_{\rho\sigma} = 0 ~.
    \end{equation*}
    \begin{proof}
        In fact, we have
        \begin{equation*}
        \begin{aligned}
            0 & = \partial_\mu F_{\nu\lambda} + \partial_\nu F_{\lambda \mu} + \partial_\lambda F_{\mu \nu} = \epsilon^{\mu\nu\lambda\sigma} (\partial_\mu F_{\nu\lambda} + \partial_\nu F_{\lambda \mu} + \partial_\lambda F_{\mu \nu}) \\ & = \partial_\mu \epsilon^{\mu\nu\lambda\sigma} F_{\nu\lambda} + \partial_\nu \epsilon^{\mu\nu\lambda\sigma} F_{\lambda \mu} + \partial_\lambda \epsilon^{\mu\nu\lambda\sigma} F_{\mu \nu} = \cancel{\partial_\mu \epsilon^{\sigma\mu\nu\lambda} F_{\nu\lambda}} - \cancel{\partial_\nu \epsilon^{\sigma\nu\lambda\mu} F_{\lambda\mu}} + \partial_\lambda \epsilon^{\lambda\sigma\mu\nu} F_{\mu \nu} \\ & = \epsilon^{\mu\nu\rho\sigma} \partial_\mu  F_{\rho\sigma} ~.
        \end{aligned}
        \end{equation*}
    \end{proof}

    It can be proved that the Maxwell's equations in covariant formalism can be written as 
    \begin{equation*}
        \boldsymbol \nabla \cdot \mathbf B = 0 ~, \quad \pdv{\mathbf B}{t} = - \boldsymbol \nabla \times \mathbf E \quad \Rightarrow \quad \partial_\mu \tilde F^{\mu\nu} = 0 ~,
    \end{equation*}
    \begin{equation*}
        \boldsymbol \nabla \cdot \mathbf E = 0 ~, \quad \pdv{\mathbf E}{t} = \boldsymbol \nabla \times \mathbf E \quad \Rightarrow \quad \partial_\mu F^{\mu\nu} = 0 ~.
    \end{equation*}
    In presence of sources, the first one remains the same whereas the second one changes because it carries information about them. 

\section{Gauge symmetry}

    It is useful to rewrite the Maxwell's Lagrangian in terms of temporal and spatial indices
    \begin{equation*}
        \mathcal L = - \frac{1}{2} (F_{0i} F^{0i} + F_{ij} F^{ij} ) ~.
    \end{equation*} 
    \begin{proof}
        In fact, we have 
        \begin{equation*}
        \begin{aligned}
            \mathcal L & = - \frac{1}{4} (\partial_\mu A_\nu - \partial_\nu A_\mu)(\partial^\mu A^\nu - \partial^\nu A^\mu) \\ & = -\frac{1}{4} (\partial_\mu A_\nu \partial^\mu A^\nu - \partial_\mu A_\nu \partial^\nu A^\mu -  \partial_\nu A_\mu \partial^\mu A^\nu + \partial_\nu A_\mu \partial^\nu A^\mu) \\ & = - \frac{1}{2} (\partial_\mu \partial^\mu A^\nu A_\nu - \partial_\nu \partial^\mu A^\nu A_\mu) \\ & = - \frac{1}{2} (\cancel{\partial_0 \partial^0 A^0 A_0} + \partial_0 \partial^0 A^i A_i + \partial_i \partial^i A^0 A_0 + \partial_i \partial^i A^j A_j \\ & \quad - \cancel{\partial_0 \partial^0 A^0 A_0} - \partial_0 \partial^i A^0 A_i - \partial_i \partial^j A^i A_j - \partial_i \partial^0 A^i A_0) \\ & = - \frac{1}{2} (\partial_0 \partial^0 A^i A_i + \partial_i \partial^i A^0 A_0 + \partial_i \partial^i A^j A_j - \partial_0 \partial^i A^0 A_i - \partial_i \partial^j A^i A_j - \partial_i \partial^0 A^i A_0) \\ & = - \frac{1}{2} 
            (\partial_0 \partial^0 A^i A_i - \partial_0 \partial^i A_i A^0 - \partial_i  \partial^0 A^0 A^i + \partial_j \partial^j A_i A^i \\ & \quad - \partial_i \partial^j A_j A^i + \partial_j \partial^i A_i A^j + \partial_i \partial^j A_j A^i - \partial_i \partial^i A_j A^j ) \\ & = - \frac{1}{2} ((\partial_0 A_i - \partial_i A_0) (\partial^0 A^i - \partial^i A^0) - (\partial_j A_i - \partial_i A_j) (\partial^j A^i - \partial^i A^j)) \\ & = - \frac{1}{2} (F_{0i} F^{0i} + F_{ij} F^{ij} ) ~.
        \end{aligned}
        \end{equation*}
    \end{proof}
    Notice that there is no dependence on the kinetic part of $A^0$, i.e. $\dot A^0$, which means that $A^0$ is fully determined by $A^i$. Therefore, we do not need to specify initial condition for $A_0$ at $t=t_0$ since the ones of $A_i$ and $\dot A_i$ are sufficient, and $A_\mu$ seems to contain only $3$ independent components. This is a consequence of the Gauss' law, which implies that
    \begin{equation*}
        A_0 (t_0, \mathbf x) = \int d^3 y ~ \frac{1}{4\pi |\mathbf x - \mathbf y} \boldsymbol \nabla \cdot \pdv{\mathbf A}{t} (t_0, \mathbf y)  ~.
    \end{equation*}
    \begin{proof}
        Using the Gauss' law, we have 
        \begin{equation*}
            0 = - \boldsymbol \nabla \cdot \mathbf E = \boldsymbol \nabla \cdot \boldsymbol \nabla A_0 + \boldsymbol \nabla \cdot \pdv{\mathbf A}{t} ~,
        \end{equation*}
        \begin{equation*}
            \nabla^2 A_0 (t_0, \mathbf x) = \boldsymbol \nabla \cdot \pdv{\mathbf A}{t} (t_0, \mathbf x) ~.
        \end{equation*}
        Now, we use the Green function to solve this differential equation. The Green operator is 
        \begin{equation*}
            \nabla^2 G (t_0, \mathbf x - \mathbf y) = \delta^3 (\mathbf x - \mathbf y) ~.
        \end{equation*}
        By a Fourier transform 
        \begin{equation*}
            G (t_0, \mathbf x - \mathbf y) = \int \frac{d^3 p}{(2\pi)^3} ~ \tilde G (p) \exp(- i \mathbf p \cdot (\mathbf x - \mathbf y)) ~,
        \end{equation*}
        we find 
        \begin{equation*}
        \begin{aligned}
            \nabla^2_x G (t_0, \mathbf x - \mathbf y) & = \nabla^2_x \int \frac{d^3 p}{(2\pi)^3} ~ \tilde G (p) \exp(- i \mathbf p \cdot (\mathbf x - \mathbf y)) \\ &= \int \frac{d^3 p}{(2\pi)^3} ~ \tilde G (p) (- p^2) \exp(- i \mathbf p \cdot (\mathbf x - \mathbf y)) \\ & = \delta^3 (\mathbf x - \mathbf y) = \int \frac{d^3 p}{(2\pi)^3} ~ \exp(- i \mathbf p \cdot (\mathbf x - \mathbf y)) ~.
        \end{aligned}
        \end{equation*}
        which implies that 
        \begin{equation*}
            \tilde G (p) = - \frac{1}{p^2} ~.
        \end{equation*}
        Putting inside the Fourier transform and using polar coordinates in momentum space, we obtain 
        \begin{equation*}
        \begin{aligned}
            G (t_0, \mathbf x - \mathbf y) & = - \int \frac{d^3 p}{(2\pi)^3} ~ \frac{\exp(- i \mathbf p \cdot (\mathbf x - \mathbf y))}{p^2} \\ & = - 2 \pi \frac{1}{(2\pi)^3} \int_0^\infty dp ~ p^2 \int_0^\pi d\theta ~ \sin \theta \frac{\exp(- i p \cos\theta |\mathbf x - \mathbf y|)}{p^2} \\ & = - \frac{1}{4 \pi^2} \int_0^\infty dp \int_{-1}^{1} d (- \cos \theta) ~ \exp(- i p \cos \theta |\mathbf x - \mathbf y|) \\ & = - \frac{1}{4 \pi^2} \int_0^\infty dp ~ \frac{\exp(-i p \cos \theta |\mathbf x - \mathbf y|)}{i p |\mathbf x - \mathbf y|} \Big \vert_{-\cos \theta = -1}^{-\cos \theta = 1} \\ & = - \frac{1}{4 \pi^2 |\mathbf x - \mathbf y| i} \int_0^\infty dp ~ \frac{\exp(i p \cos \theta |\mathbf x - \mathbf y|)}{p} - \frac{\exp(-i p \cos \theta |\mathbf x - \mathbf y|)}{p} \\ & = - \frac{1}{4 \pi^2 |\mathbf x - \mathbf y| i} \int_{-\infty}^\infty dp ~ \frac{\exp(i p \cos \theta |\mathbf x - \mathbf y|)}{p}  \\ & = - \frac{1}{4 \pi^2 |\mathbf x - \mathbf y| i} \pi i \exp (i p |\mathbf x - \mathbf y|) \Big \vert_{p = 0} = - \frac{1}{4 \pi |\mathbf x - \mathbf y|} ~,
        \end{aligned}
        \end{equation*}
        where we have integrated in the upper-part of the complex plane with one pole in $p=0$. Finally, we find 
        \begin{equation*}
            A_0 (t_0, \mathbf x) = - \int d^3 y ~ G(\mathbf x - \mathbf y) \boldsymbol \nabla \cdot \pdv{\mathbf A}{t} (t_0, \mathbf y) = \int d^3 y ~ \frac{1}{4\pi |\mathbf x - \mathbf y|} \boldsymbol \nabla \cdot \pdv{\mathbf A}{t} (t_0, \mathbf y)~.
        \end{equation*}
    \end{proof}

    This is a signal that Maxwell's theory is a gauge theory. In fact, Maxwell's Lagrangian is symmetric with respect to a gauge transformation  
    \begin{equation*}
        {A'}_\mu (x) = A_\mu (x) + \partial_\mu \alpha (x) ~,
    \end{equation*}
    where $\alpha (x)$ is an arbitrary gauge function of spacetime coordinates, such that its derivative vanishes at spatial infinity.
    \begin{proof}
        In fact, we have
        \begin{equation*}
            {F'}^{\mu\nu} = \partial_\mu {A'}_\nu - \partial_\nu {A'}_\mu = \partial_\mu A_\nu - \partial_\nu A_\mu + \cancel{\partial_\mu \partial_\nu \alpha(x)} - \cancel{\partial_\nu \partial_\mu \alpha(x)} = \partial_\mu A_\nu - \partial_\nu A_\mu = F^{\mu\nu} ~,
        \end{equation*}
        which means that
        \begin{equation*}
            \mathcal L' = - \frac{1}{4} {F'}^{\mu\nu} F'_{\mu\nu} = - \frac{1}{4} F^{\mu\nu}_{\mu\nu} = \mathcal L ~.
        \end{equation*}
    \end{proof}

    However, we experimentally know that an electromagnetic wave traveling in vacuum has only two degrees of freedom corresponding to the two transversal polarisations. This signals that there is a second residual gauge transformation 
    \begin{equation*}
        {A''}_\mu (x) = {A'}_\mu (x) + \partial_\mu \beta (x) = A_\mu (x) + \partial_\mu (\alpha (x) + \beta (x) ) ~,
    \end{equation*}
    where $\beta (x)$ is the second gauge function. Physically, $A$, $A'$ and $A''$ descibe the same physical state, since the Lagrangians are the same. Therefore, they define a class of equivalence of physical points of view. There are several choices for the representative of this class or gauge orbit by some conditions, called gauge fixing, based on the convenience to make easier the problem. See Figure~\ref{fig:gauge}. It is important to highlight the difference between a global and a local symmetry. A global symmetry does not depend on the spacetime coordinates and it is a true symmetry of the system, which leads to the Noether's theorem. A local symmetry is just a redundancy in the description of the physics and it does not have a conservation law (even because there will be infinitely many).

    \begin{figure}[h!]
        \centering
        \begin{tikzpicture}
        \draw[] (0,0) to[bend right=20] (2,2) to[bend left=20] (4,4) node[above right] {gauge orbits};
        \draw[] (2,0) to[bend right=20] (4,2) to[bend left=20] (6,4) ;
        \draw[] (4,0) to[bend right=20] (6,2) to[bend left=20] (8,4) ;

        \draw[] (0,4) to[bend right=20] (4,2) to[bend left=20] (8,0) node[below left] {gauge fixing};
        \draw[] (0,3) to[bend right=20]  (2,2) to[bend left=20] (6,0) ;

        \filldraw[black] (4,2) circle (0.05) node[above left] {$A_\mu$};
        \filldraw[black] (2,2) circle (0.05) node[below right] {$A'_\mu$};

        \end{tikzpicture}
        \caption{Pictorial representation of gauge orbits and gauge fixing.}
        \label{fig:gauge}
    \end{figure}
    
    The gauge fixing we will use in this notes is the Lorenz gauge, which is manifestly Lorentz invariant and brings down the number of degrees of freedom to $2$
    \begin{equation*}
        \partial_\mu A^\mu = 0 ~.
    \end{equation*}
    \begin{proof}
        With a gauge transformation, we have
        \begin{equation*}
            {A'}_\mu (x) = A_\mu (x) + \partial_\mu \alpha (x) ~,
        \end{equation*}
        where $\alpha (x)$ must satisfy 
        \begin{equation*}
            0 = \partial_\mu {A'}^\mu = \partial_\mu A^\mu + \Box \alpha (x) ~,
        \end{equation*}
        hence, we find
        \begin{equation*}
            \Box \alpha(x) = \partial_\mu A^\mu ~.
        \end{equation*}
        With the second gauge transformation, we have
        \begin{equation*}
            {A''}_\mu (x) = {A'}_\mu (x) + \partial_\mu \beta (x) ~,
        \end{equation*}
        where $\beta (x)$ must satisfy 
        \begin{equation*}
            \partial_\mu {A''}^\mu = \partial_\mu {A'}^\mu + \Box \beta (x) ~,
        \end{equation*}
        hence, we find
        \begin{equation*}
            \Box \beta(x) = 0 ~.
        \end{equation*}
    \end{proof}
    The equations of motion in the Lorenz gauge become 
    \begin{equation}\label{lgem}
        \Box A^\mu (x) = 0 ~.
    \end{equation}
    Notice that the Maxwell's equations in the Lorentz gauge are equal to the Klein-Gordon ones with mass equals to zero. This ensures that the mass shell condition is preserved. This means that each components of $A^\mu$ separately satisfies the mass shell condition. At quantum level, we will see taht $A_\mu$ describes particles (photons) with energy $E_{\mathbf p} = |\mathbf p|$.
    \begin{proof}
        In fact, we have
        \begin{equation*}
            0 = \partial_\mu F^{\mu\nu} = \partial_\mu (\partial^\mu A^\nu - \partial^\nu A^\mu) = \Box A^\nu - \partial^\nu \underbrace{\partial_\mu A^\mu}_0 = \Box A^\nu ~.
        \end{equation*}
    \end{proof}

    The conjugate momentum of $A_\mu$ is 
    \begin{equation*}
        \pi^\mu = (0, \mathbf E) ~.
    \end{equation*}
    \begin{proof}
        In fact, for $\mu = 0$, we have
        \begin{equation*}
            \pi^0 = \pdv{\mathcal L}{\dot A_0} = 0 ~,
        \end{equation*}
        whereas, for $\mu = i$, we have
        \begin{equation*}
        \begin{aligned}
            \pi^i = \pdv{\mathcal L}{\dot A_i} = - \frac{1}{2} \pdv{}{\dot A_i} ((\dot A_j - \partial_j A_0))(\dot A^j - \partial^j A_0) = - \frac{1}{2} \pdv{}{\dot A_i} (F_{0j} F^{0j}) = - F^{0i} = E^i ~.
        \end{aligned}
        \end{equation*}
    \end{proof}

    The Hamiltonian is 
    \begin{equation*}
        H = \int d^3 x ~\Big ( \frac{1}{2}  (|E|^2 + |B|^2) - A_0 (\boldsymbol \nabla \cdot \mathbf E) \Big) ~.
    \end{equation*}
    Notice that we have a Lagrange multiplier $A_0$ to ensure the constrain of the Gauss' law, since $A_0$ is not a physical variable. In fact, one of the Hamilton's equation is
    \begin{equation*}
        0 = \dot A_0 = \pdv{\mathcal H}{A_0} = - \boldsymbol \nabla \cdot \mathbf E ~. 
    \end{equation*}
    \begin{proof}
        In fact, by a Legendre transformation and~\eqref{ef}, we have
        \begin{equation*}
        \begin{aligned}
            \mathcal H & = \pi^\mu \dot A_\mu - \mathcal L = \pi^i \dot A_i - \mathcal L = - E^i \underbrace{\dot A_i}_{- E_i - \partial_i A_0} - \mathcal L \\ & = E^i E_i + E^i \partial_i A_0 - \frac{1}{2} (|E|^2 - |B|^2) = \frac{1}{2} (|E|^2 + |B|^2) + E^i \partial_i A_0 ~,
        \end{aligned}
        \end{equation*}
        hence, we find
        \begin{equation}
        \begin{aligned}
            H & = \int d^3 x ~ \mathcal H = \int d^3 x ~ \Big ( \frac{1}{2} (|E|^2 + |B|^2) + \underbrace{E^i \partial_i A_0}_{- A_0 \partial_i E^i + \textnormal{boundary terms}} \Big) \\ & = \int d^3 x ~ \Big ( \frac{1}{2} (|E|^2 + |B|^2) - A_0 (\boldsymbol \nabla \cdot \mathbf E) \Big) ~.
        \end{aligned}
        \end{equation}
    \end{proof}

\chapter{Quantisation}

    In this chapter, we will quantise the Maxwell's theory. We will find field operators by canonical commutation relations, we will analyse the Fock space, finding which are the physical states compatible with the Lorenz gauge, and we will find the Hamiltonian operator.

\section{Quantisation without Lorenz gauge}

    The first guess to quantise the theory is, instead of imposind by hand the Lorenz gauge, to modify the Lagrangian 
    \begin{equation*}
        \mathcal L = - \frac{1}{4} F_{\mu\nu} F^{\mu\nu} - \frac{1}{2} (\partial_\mu A^\mu)^2 ~. 
    \end{equation*}
    Therefore, the equations of motion remains the same, even if we do not impose the Lorenz gauge, and the conjugate momentum becomes
    \begin{equation*}
        \pi^\mu = (- \partial_\mu A^\mu, F^{i0}) = F^{\nu 0} - \eta^{\nu 0} \partial_\mu A^\mu ~.
    \end{equation*}
    \begin{proof}        
        For the equations of motion, using~\eqref{eleq}, we have 
        \begin{equation*}
            0 = \partial_\mu \pdv{\mathcal L}{\partial_\mu \partial_\mu A_\nu} = - \partial_\mu F^{\mu\nu} - \partial^\nu \partial_\alpha A^\alpha ~,
        \end{equation*}
        hence, we obtain 
        \begin{equation*}
            0 = \partial_\mu F^{\mu\nu} + \partial_\mu \eta^{\mu\nu} \partial_\alpha A^\alpha = \partial_\mu \partial^\mu A^\nu - \partial_\mu \partial^\nu A^\mu + \partial^\nu \partial_\alpha A^\alpha = \partial_\mu \partial^\mu A^\nu ~,
        \end{equation*}
        where we have used the fact that partial derivatives commute.
        For the conjugate momentum, for $\mu = 0$, we have
        \begin{equation*}
            \pi^0 = \pdv{\mathcal L}{\dot A_0} = - \frac{1}{2} \pdv{\mathcal L}{\dot A_0} (\partial_\mu A^\mu)^2 = - \partial_\mu A^\mu  ~,
        \end{equation*}
        whereas, for $\mu = i$, we have
        \begin{equation*}
            \pi^i = E^i = - F^{0i} = F^{i0} ~.
        \end{equation*}
        Finally, to recover the covariant formalism, we find
        \begin{equation*}
            \pi^0 = \underbrace{F^{00}}_0 - \underbrace{\eta^{00}}_1 (\partial_\mu A^\mu) = - \partial_\mu A^\mu 
        \end{equation*}
        and 
        \begin{equation*}
            \pi^i = F^{0i} - \underbrace{\eta^{i0}}_0 (\partial_\mu A^\mu) = F^{0i} ~.
        \end{equation*}
    \end{proof}

    Now, we use the machinery of second quantisation: in Schoedinger picture, we promote $A^\mu(x)$ and $\pi^\mu(x)$ to operators in the Fock space by imposing the canonical commutation, since we have an integer spin theory ($s = 1$), 
    \begin{equation*}
        [\hat A_\mu (\mathbf x), \hat A_\nu (\mathbf y)] = [\hat \pi_\mu (\mathbf x), \hat \pi_\nu (\mathbf y)] = 0 ~, \quad [\hat A_\mu (\mathbf x), \hat \pi_\nu (\mathbf y)] = i \eta_{\mu\nu} \delta^3 (\mathbf x - \mathbf y) ~.
    \end{equation*}

    Since the general solution of~\eqref{lgem} is a linear combination of plane waves, we expand the field operators in terms of ladder operators in the following way
    \begin{equation*}
        \hat A_\mu (\mathbf x) = \int \frac{d^3 p}{(2\pi)^3} \frac{1}{\sqrt{2 |\mathbf p|}} \Big ( \hat \xi_\mu (\mathbf p) \exp(i \mathbf p \cdot \mathbf x) + \hat \xi_\mu^\dagger (\mathbf p) \exp(- i \mathbf p \cdot \mathbf x)) ~,
    \end{equation*}
    \begin{equation*}
        \hat \pi_\mu (\mathbf x) = \int \frac{d^3 p}{(2\pi)^3} \Big ( i \sqrt{\frac{|\mathbf p|}{2}} \Big ) \Big ( \hat \xi_\mu (\mathbf p) \exp(i \mathbf p \cdot \mathbf x) - \hat \xi_\mu^\dagger (\mathbf p) \exp(- i \mathbf p \cdot \mathbf x)) ~,
    \end{equation*}
    where $E_{\mathbf p} = |\mathbf p|$ and $\xi_\mu (\mathbf p)$ is the polarisation $4$-vector. Notice that there is a plus sign instead of a minus sign in the conjugate field, because in Maxwell's theory we have $\pi^\mu = - \dot A^\mu$ whereas in Klein-Gordon's theory we have $\pi = \dot \varphi$. Furthermore, polarisation vectors depend on momentum.
    We introduce an orthonormal basis for the polarisation $4$-vectors $\epsilon_\mu^{(\lambda)}$ $\lambda = 0, 1, 2, 3$, such that 
    \begin{equation*}
        \epsilon_\mu^{(\lambda)} \epsilon^{\mu (\lambda')} = \eta^{\lambda \lambda'} ~, \quad \epsilon_\mu^{(\lambda)} \epsilon_\nu^{(\lambda')} \eta_{\lambda \lambda'} = \eta_{\mu\nu} ~.
    \end{equation*}
    Since in second quantisation polarisation vectors are operators, the coefficients on this expansion are the annihilation operators
    \begin{equation*}
        \hat \xi_\mu (\mathbf p) = \sum_{\lambda=0}^3 \epsilon_\mu^{(\lambda)} (\mathbf p) \hat a_{\mathbf p}^{(\lambda)} ~.
    \end{equation*}
    Therefore, the field operators become
    \begin{equation}\label{max:a}
        \hat A_\mu (\mathbf x) = \int \frac{d^3 p}{(2\pi)^3} \frac{1}{\sqrt{2 |\mathbf p|}} \sum_{\lambda=0}^{3} \epsilon_\mu^{(\lambda)} (\mathbf p) \Big ( \hat a_{\mathbf p}^{(\lambda)} \exp(i \mathbf p \cdot \mathbf x) + \hat a_{\mathbf p}^{\dagger (\lambda)}  \exp(- i \mathbf p \cdot \mathbf x) \Big)  ~,
    \end{equation}
    \begin{equation}\label{max:p}
        \hat \pi^\mu (\mathbf x) = \int \frac{d^3 p}{(2\pi)^3} \Big (i \sqrt{\frac{|\mathbf p|}{2}} \Big ) \sum_{\lambda=0}^{3} \epsilon^{\mu(\lambda)} (\mathbf p) \Big ( \hat a_{\mathbf p}^{(\lambda)} \exp(i \mathbf p \cdot \mathbf x) - \hat a_{\mathbf p}^{\dagger (\lambda)}  \exp(- i \mathbf p \cdot \mathbf x) \Big)  ~.
    \end{equation}
    To make contact with the $2$ transversal polarisation of an electromagnetic wave, we choose $\epsilon_\mu^{(1)}$ and $\epsilon_\mu^{(2)}$ to be orthogonal to the motion, such that $\epsilon_\mu^{(1)} p^\mu = \epsilon_\mu^{(2)} p^\mu = 0$. For example, if $p^\mu = (E, 0, 0, E)$ lies along the $z$-direction, we have $\epsilon_\mu^{(1)} p^\mu = \epsilon_0^{(1)} p^0 + \epsilon_3^{(1)} p^3 = E (\epsilon_0^{(1)} + \epsilon_3^{(1)}) = 0$, which means $\epsilon_0^{(1)} = -\epsilon_3^{(1)}$ and for convenience we put to zero $\epsilon_0^{(1)} = \epsilon_3^{(1)} = 0$. Similarly, we have $\epsilon_\mu^{(2)} p^\mu = \epsilon_0^{(2)} p^0 + \epsilon_3^{(2)} p^3 = E (\epsilon_0^{(2)} + \epsilon_3^{(2)}) = 0$, which means $\epsilon_0^{(2)} = -\epsilon_3^{(2)}$ and for convenience we put to zero $\epsilon_0^{(2)} = \epsilon_3^{(2)} = 0$. Therefore, we choose $\epsilon_1^{(1)} = -1$, $\epsilon_2^{(1)} = 0$, $\epsilon_1^{(2)} = 0$ and $\epsilon_2^{(2)} = -1$. In matrix notation, it becomes
    \begin{equation}\label{pol}
        \epsilon^{(0)}_\mu = \begin{bmatrix}
            1 \\ 0 \\ 0 \\ 0 \\
        \end{bmatrix} ~,  \epsilon^{(1)}_\mu = \begin{bmatrix}
            0 \\ - 1 \\ 0 \\ 0 \\
        \end{bmatrix} ~, \epsilon^{(2)}_\mu = \begin{bmatrix}
            0 \\ 0 \\ - 1 \\ 0 \\
        \end{bmatrix} ~, \epsilon^{(3)}_\mu = \begin{bmatrix}
            0 \\ 0 \\ 0 \\ - 1 \\
        \end{bmatrix} ~,
    \end{equation}
    where $\epsilon^{(0)}_\mu$ is timelike, $\epsilon^{(1)}_\mu$ and $\epsilon^{(2)}_\mu$ are spacelike and $\epsilon^{(3)}_\mu$ is the longitudinal polarisation.

    The commutation relations induced by the canonical ones on the ladder operators are 
    \begin{equation}\label{max:coml}
        [\hat a_{\mathbf p}^{(\lambda)}, \hat a_{\mathbf q}^{(\lambda')}] = [\hat a_{\mathbf p}^{\dagger (\lambda)}, \hat a_{\mathbf q}^{\dagger (\lambda')}] = 0 ~, \quad [\hat a_{\mathbf p}^{(\lambda)}, \hat a_{\mathbf q}^{\dagger(\lambda')}] = - (2\pi)^3 \eta^{\lambda \lambda'} \delta^3 (\mathbf p - \mathbf q) ~.
    \end{equation}
    or, explicitly the latter, 
    \begin{equation*}
        [\hat a_{\mathbf p}^{(0)}, \hat a_{\mathbf q}^{\dagger (0)}] = - (2\pi)^3 \delta^3 (\mathbf p - \mathbf q) ~, \quad [\hat a_{\mathbf p}^{(i)}, \hat a_{\mathbf q}^{\dagger (i)}] = (2\pi)^3 \delta^3 (\mathbf p - \mathbf q) ~.
    \end{equation*}
    \begin{proof}
        In fact, using~\eqref{max:a},~\eqref{max:p} and~\eqref{max:coml}, we have
        \begin{equation*}
        \begin{aligned}
            & [\hat A_\mu (\mathbf x), \hat \pi^\nu (\mathbf y)] \\ & = [\int \frac{d^3 p}{(2\pi)^3} \frac{1}{\sqrt{2 |\mathbf p|}} \sum_{\lambda=0}^{3} \epsilon_\mu^{(\lambda)} (\mathbf p) \Big ( \hat a_{\mathbf p}^{(\lambda)} \exp(i \mathbf p \cdot \mathbf x) + \hat a_{\mathbf p}^{\dagger (\lambda)} \exp(- i \mathbf p \cdot \mathbf x) \Big), \\ & \quad \int \frac{d^3 q}{(2\pi)^3} i \sqrt{\frac{|\mathbf q|}{2}} \sum_{\lambda'=0}^{3} \epsilon^{\nu(\lambda')} (\mathbf q) \Big ( \hat a_{\mathbf q}^{(\lambda')}  \exp(i \mathbf q \cdot \mathbf y) - \hat a_{\mathbf q}^{\dagger (\lambda')} \exp(- i \mathbf q \cdot \mathbf y) \Big)] \\ & = \sum_{\lambda=0}^{3} \sum_{\lambda'=0}^{3} \int \frac{d^3 p ~ d^3 q}{(2\pi)^6} \frac{i}{2} \sqrt{\frac{|\mathbf q|}{|\mathbf p|}} \epsilon_\mu^{(\lambda)} \epsilon^{\nu(\lambda')} \Big ( \underbrace{[\hat a_{\mathbf p}^{(\lambda)} , \hat a_{\mathbf q}^{(\lambda')}]}_0 \exp(i (\mathbf p \cdot \mathbf x + \mathbf q \cdot \mathbf y)) \\ & \quad - \underbrace{[\hat a_{\mathbf p}^{(\lambda)} , \hat a_{\mathbf q}^{\dagger (\lambda')}]}_{- (2\pi)^3 \eta^{\lambda \lambda'} \delta^3 (\mathbf p - \mathbf q)} \exp(i (\mathbf p \cdot \mathbf x - \mathbf q \cdot \mathbf y)) + \underbrace{[\hat a_{\mathbf p}^{\dagger (\lambda)} , \hat a_{\mathbf q}^{(\lambda')}]}_{(2\pi)^3 \eta^{\lambda \lambda'} \delta^3 (\mathbf p - \mathbf q)} \exp(i (- \mathbf p \cdot \mathbf x + \mathbf q \cdot \mathbf y)) \\ & \quad - \underbrace{[\hat a_{\mathbf p}^{\dagger(\lambda)} , \hat a_{\mathbf q}^{\dagger(\lambda')}]}_0 \exp(i (- \mathbf p \cdot \mathbf x - \mathbf q \cdot \mathbf y)) \Big) \\ & = \sum_{\lambda=0}^{3} \int \frac{d^3 p}{(2\pi)^3} \frac{i}{2} \underbrace{\epsilon_\mu^{(\lambda)} \epsilon^{\nu(\lambda)}}_{\eta^\nu_{\phantom \nu \mu}} \Big ( \underbrace{\exp(i \mathbf p \cdot (\mathbf x - \mathbf y))}_{\delta^3 (\mathbf x - \mathbf y)} + \underbrace{\exp(- i \mathbf p \cdot (\mathbf x - \mathbf y))}_{\delta^3 (\mathbf x - \mathbf y)} \Big) = i \eta^\nu_{\phantom \nu \mu} \delta^3 (\mathbf x - \mathbf y) ~.
        \end{aligned}
        \end{equation*}
    \end{proof}

    Notice that there is a problem, since we do not have a probabilistic intepretation for a negative norm (only for the temporal components). We cannot neither interpret $\hat a$ as a creation operator, because we would have problems for the spatial components. This kind of states are called ghosts. We could have seen it already by the Lagrangian, since we had a minus sign in the temporal component
    \begin{equation*}
        \mathcal L = \frac{1}{2} (- (\dot A_0)^2 + (\dot A_1)^2 + (\dot A_2)^2 +(\dot A_3)^2 ) + \ldots ~.
    \end{equation*}
    \begin{proof}
        In fact, given the vacuum defined as 
        \begin{equation*}
            \hat a^{(\lambda)}_{\mathbf p} \ket{0} = 0 ~,
        \end{equation*}
        a state with polarisation $\lambda$ defined as 
        \begin{equation*}
            \ket{\mathbf p, \lambda} = \hat a^{\dagger (\lambda)}_{\mathbf p} \ket{0} ~,
        \end{equation*}
        has a norm equals to
        \begin{equation*}
            \braket{\mathbf p, \lambda}{\mathbf q, \lambda'} = \bra{0} \hat a^{(\lambda)}_{\mathbf p} \hat a^{\dagger (\lambda')}_{\mathbf q} \ket{0} = \bra{0} \underbrace{[\hat a^{(\lambda)}_{\mathbf p}, \hat a^{\dagger (\lambda')}_{\mathbf q}] }_{(2\pi)^3 \eta^{\lambda \lambda'} \delta^3 (\mathbf p - \mathbf q)} \ket{0} + \bra{0} \hat a^{\dagger (\lambda')}_{\mathbf q} \underbrace{\hat a^{(\lambda)}_{\mathbf p} \ket{0}}_0 = (2\pi)^3 \eta^{\lambda \lambda'} \delta^3 (\mathbf p - \mathbf q) ~,
        \end{equation*}
        which for temporal component $\lambda = \lambda' = 0$, we find
        \begin{equation*}
            \braket{\mathbf p, \lambda=0}{\mathbf q, \lambda'=0} = - (2\pi)^3 \delta^3 (\mathbf p - \mathbf q) \leq 0 ~.
        \end{equation*}
    \end{proof}

\section{Quantisation with Lorenz gauge}

    To solve this problem, we impose the gauge fixing condition, which we have not used yet. However, since the Lorenz gauge contains a time derivative, we must go into the Heisenberg picture $\hat A^\mu (x) = \hat A^\mu (t, \mathbf x)$. There are different possible ways to impose the gauge fixing conditions:
    \begin{enumerate}
        \item on operators $\partial_\mu \hat A^\mu = 0$, 
        \item on states $(\partial_\mu \hat A^\mu) \ket{\psi} = 0$, 
        \item on matrix elements $\bra{\psi} \partial_\mu \hat A^\mu \ket{\psi} = 0$.
    \end{enumerate}

    Let us consider the first case. A contradiction arises because, on one hand, we have 
    \begin{equation*}
        \hat \pi^0 = - \partial_\mu \hat A^\mu = 0 ~,
    \end{equation*}
    on the other hand, the commutation relations at fixed time
    \begin{equation*}
        [\hat A_0 (\mathbf x), \hat \pi_0 (\mathbf y)] = i \eta_{00} \delta^3 (\mathbf x - \mathbf y) \neq 0 ~.
    \end{equation*}

    Let us consider the second case. Positive norm physical state are such that
    \begin{equation*}
        (\partial_\mu \hat A^\mu) \ket{\psi} ~,
    \end{equation*}
    but, if we decomposed the field operator,
    \begin{equation*}
    \begin{aligned}
        \hat A_\mu (x) & = \int \frac{d^3 p}{(2\pi)^3} \frac{1}{\sqrt{2 |\mathbf p|}} \sum_{\lambda=0}^{3} \epsilon_\mu^{(\lambda)} (\mathbf p) \Big ( \hat a_{\mathbf p}^{(\lambda)} (\mathbf p) \exp(i \mathbf p \cdot \mathbf x) + \hat a_{\mathbf p}^{\dagger (\lambda)} (\mathbf p) \exp(- i \mathbf p \cdot \mathbf x) \Big) \\ & = \hat A^+_\mu (x) + \hat A^-_\mu (x) ~,
    \end{aligned}
    \end{equation*}
    where $\hat A^+_\mu (x)$ contains only creation operators 
    \begin{equation*}
        \hat A^+_\mu (x) = \int \frac{d^3 p}{(2\pi)^3} \frac{1}{\sqrt{2 |\mathbf p|}} \sum_{\lambda=0}^{3} \epsilon_\mu^{(\lambda)} (\mathbf p) \hat a_{\mathbf p}^{\dagger (\lambda)} (\mathbf p) \exp(- i \mathbf p \cdot \mathbf x) 
    \end{equation*}
    and $\hat A^-_\mu (x)$ contains only annihilation operators
    \begin{equation*}
        \hat A^-_\mu (x) = \int \frac{d^3 p}{(2\pi)^3} \frac{1}{\sqrt{2 |\mathbf p|}} \sum_{\lambda=0}^{3} \epsilon_\mu^{(\lambda)} (\mathbf p) \hat a_{\mathbf p}^{(\lambda)} (\mathbf p) \exp(i \mathbf p \cdot \mathbf x) ~,
    \end{equation*}
    we find that 
    \begin{equation*}
        \partial^\mu \hat A_\mu \ket{0} = \partial^\mu \hat A_\mu^+ \ket{0} = \partial^\mu \hat A_\mu^- \ket{0} = i p^\mu \hat A_\mu^+ \ket{0} - i p^\mu \underbrace{\hat A_\mu^- \ket{0}}_0 \neq 0 ~,
    \end{equation*}
    which means that vacuum state does not satisfy the Lorenz gauge and it is not physical.

    Let us consider the third case. This condition are called Gupta-Bleuler conditions and they can be formulated in three equivalent ways
    \begin{equation}\label{GB}
        \partial^\mu \hat A^+_\mu \ket{\psi} = 0 \iff \bra{\psi} \partial^\mu \hat A^-_\mu = 0 \iff \bra{\phi} \partial_\mu \hat A^\mu \ket{\psi} = 0 ~.
    \end{equation}
    Physical states of the Hilbert space are the only one that satisfy this condition. It can be proved that it implies a constraint: the number of timelike photons are the same number of longitudinal photons for physical state with same momemtum
    \begin{equation}\label{GB2}
        \bra{\psi} \hat n^{(0)}_{\mathbf p} \ket{\psi} = \bra{\psi} \hat n^{(3)}_{\mathbf p} \ket{\psi} ~.
    \end{equation}
    \begin{proof}
        We start from
        \begin{equation*}
            \partial^\mu \hat A^-_\mu (x) = \int \frac{d^3 p}{(2\pi)^3} \frac{1}{\sqrt{2 |\mathbf p|}} \sum_{\lambda=0}^{3} p^\mu \epsilon_\mu^{(\lambda)} (\mathbf p) \hat a_{\mathbf p}^{(\lambda)} (\mathbf p) \exp(i \mathbf p \cdot \mathbf x) ~,
        \end{equation*}
        which contains the term $p^\mu \epsilon_\mu^{(\lambda)}$, but recall that we have chosed $\epsilon^{(\lambda)}_\mu = 0$ for transversal photons $\lambda = 1,2$. Hence, with $p^\mu = (E, 0, 0, E)$ and~\eqref{pol}, we obtain 
        \begin{equation*}
            0 = (\epsilon^{(0)}_\mu p^\mu \hat a^{(0)}_{\mathbf p} + \epsilon^{(3)}_\mu p^\mu \hat a^{(3)}_{\mathbf p}) \ket{\psi} = E (\underbrace{\epsilon^{(0)}_1}_1 \hat a^{(0)}_{\mathbf p} + \underbrace{\epsilon^{(3)}_3}_{-1} \hat a^{(3)}_{\mathbf p}) \ket{\psi} = E (\hat a^{(0)}_{\mathbf p} - \hat a^{(3)}_{\mathbf p}) \ket{\psi} ~,
        \end{equation*}
        \begin{equation*}
            \hat a^{(0)}_{\mathbf p} \ket{\psi } = \hat a^{(3)}_{\mathbf p} \ket{\psi} \iff \bra{\psi} \hat a^{\dagger(0)}_{\mathbf p} = \bra{\psi} \hat a^{\dagger(3)}_{\mathbf p}  ~.
        \end{equation*}
        Finally, we find
        \begin{equation*}
            \bra{\psi} \hat n^{(0)}_{\mathbf p} \ket{\psi} = \bra{\psi} \hat a^{\dagger(0)}_{\mathbf p} \hat a^{(0)}_{\mathbf p} \ket{\psi} = \bra{\psi} \hat a^{\dagger(3)}_{\mathbf p} \hat a^{(3)}_{\mathbf p} \ket{\psi} = \bra{\psi} \hat n^{(3)}_{\mathbf p} \ket{\psi} ~.
        \end{equation*}
    \end{proof}

    The latter result implies that ghost with only timelike photons cannot exist, since a negative norm state with only timelike photons is unphysical. 
    \begin{proof}
        In fact, for $\ket{\mathbf q, \lambda = 0} = \hat a^{\dagger (0)}_{\mathbf q} \ket{0}$ such that 
        \begin{equation*}
            (\hat a^{(0)}_{\mathbf p} - \hat a^{(3)}_{\mathbf p}) \ket{\mathbf q, \lambda = 0} = \underbrace{\hat a^{(0)}_{\mathbf p} \hat a^{\dagger (0)}_{\mathbf q}}_{- (2\pi)^3 \delta^3 (\mathbf p - \mathbf q)} \ket{0} - \underbrace{\hat a^{(3)}_{\mathbf p} \hat a^{\dagger (0)}_{\mathbf q}}_0 \ket{0} = - (2\pi)^3 \delta^3 (\mathbf p - \mathbf q) \ket{0} \neq 0 ~,
        \end{equation*}
        which shows that a state with only timelike photons is unphysical because it does not satisfy the Gupta-Breuler conditions. 
    \end{proof} 

\section{Fock space of photons}

    Now, we investigate the Fock space in which the Gupta-Breuler conditions are valid. We start by making a change of basis, from 
    \begin{equation*}
        \hat a^{\dagger(0)}_{\mathbf p} ~, \quad \hat a^{\dagger(1)}_{\mathbf p} ~, \quad \hat a^{\dagger(2)}_{\mathbf p} ~, \quad \hat a^{\dagger(3)}_{\mathbf p} 
    \end{equation*}
    into 
    \begin{equation*}
        \hat a^{\dagger(1)}_{\mathbf p} ~, \quad \hat a^{\dagger(2)}_{\mathbf p} ~, \quad \hat b_{\pm, \mathbf p} = \hat a^{\dagger(0)}_{\mathbf p} \pm \hat a^{\dagger(3)}_{\mathbf p} ~.
    \end{equation*}
    which inverted looks like 
    \begin{equation*}
        \hat a^{\dagger (0)} = \frac{\hat b^\dagger_{+, \mathbf p} + \hat b^\dagger_{-, \mathbf p} }{2} ~, \quad \hat a^{\dagger (3)} = \frac{\hat b^\dagger_{+, \mathbf p} - \hat b^\dagger_{-, \mathbf p} }{2}  ~.
    \end{equation*}
    This implies that 
    \begin{equation*}
        \hat a^{\dagger(1)}_{\mathbf p} \ket{0} = \ket{\mathbf p, \lambda = 1} ~, \quad \hat a^{\dagger(2)}_{\mathbf p} \ket{0} = \ket{\mathbf p, \lambda = 2} ~, \quad \hat b_{\pm, \mathbf p} \ket{0} = \ket{\mathbf p, \lambda = 0} \pm \ket{\mathbf p, \lambda = 3} ~.
    \end{equation*}
    The last state is a linear combination of one timelike photon and one longitudinal photons.

    Furthermore, the commutation relations becomes 
    \begin{equation*}
        [\hat b_{\mp, \mathbf p}, \hat b_{\mp, \mathbf q}^\dagger] = 0 ~, \quad [\hat b_{-, \mathbf p}, \hat b_{+, \mathbf q}^\dagger] = - 2 (2\pi)^3 \delta^3 (\mathbf p - \mathbf q) ~.
    \end{equation*}
    \begin{proof}
        For the first 
        \begin{equation*}
        \begin{aligned}
            [\hat b_{\mp, \mathbf p}, \hat b_{\mp, \mathbf q}^\dagger] & = [\hat a^{(0)}_{\mathbf p} \mp \hat a^{(3)}_{\mathbf p}, \hat a^{\dagger(0)}_{\mathbf q} \mp \hat a^{\dagger(3)}_{\mathbf q}] \\ & = \underbrace{[\hat a^{(0)}_{\mathbf p} , \hat a^{\dagger(0)}_{\mathbf q}]}_{-(2\pi)^3 \delta^3 (\mathbf p - \mathbf q)} \mp \underbrace{[\hat a^{(0)}_{\mathbf p} , \hat a^{\dagger(3)}_{\mathbf q}] }_0 \mp \underbrace{[\hat a^{(3)}_{\mathbf p}, \hat a^{\dagger(0)}_{\mathbf q}]}_0 + \underbrace{[\hat a^{(3)}_{\mathbf p}, \hat a^{\dagger(3)}_{\mathbf q}]}_{(2\pi)^3 \delta^3 (\mathbf p - \mathbf q)} = 0 ~.
        \end{aligned}
        \end{equation*}
        For the second 
        \begin{equation*}
        \begin{aligned}
            [\hat b_{\mp, \mathbf p}, \hat b_{\pm, \mathbf q}^\dagger] & = [\hat a^{(0)}_{\mathbf p} \mp \hat a^{(3)}_{\mathbf p}, \hat a^{\dagger(0)}_{\mathbf q} \pm \hat a^{\dagger(3)}_{\mathbf q}] \\ & = \underbrace{[\hat a^{(0)}_{\mathbf p} , \hat a^{\dagger(0)}_{\mathbf q}]}_{-(2\pi)^3 \delta^3 (\mathbf p - \mathbf q)} \pm \underbrace{[\hat a^{(0)}_{\mathbf p} , \hat a^{\dagger(3)}_{\mathbf q}] }_0 \mp \underbrace{[\hat a^{(3)}_{\mathbf p}, \hat a^{\dagger(0)}_{\mathbf q}]}_0 - \underbrace{[\hat a^{(3)}_{\mathbf p}, \hat a^{\dagger(3)}_{\mathbf q}]}_{(2\pi)^3 \delta^3 (\mathbf p - \mathbf q)} = - 2 (2\pi)^3 \delta^3 (\mathbf p - \mathbf q)  ~.
        \end{aligned}
        \end{equation*}
    \end{proof}

    Physical Gupta-Breuler conditions~\eqref{GB} for $\ket{\psi}$ becomes 
    \begin{equation*}
        \hat b_{-, \mathbf p} \ket{\psi} = 0 ~.
    \end{equation*} 
    Hence transversal photons $\ket{T}$ are physical, timelike $\ket{S}$ and longitudinal photons $\ket{L}$ are unphysical, the combination with plus of timelike and longitudinal photons $\ket{S} + \ket{L}$ are unphysical, the combination with minus of timelike and longitudinal photons $\ket{S} - \ket{L}$ are physical. However, the latter has zero-norm. To summarise, Fock space contains all states such that it is satisfied~\eqref{GB}, which brings to positive norm states (transversal $\ket{T}$ photons) and zero norm states ($\ket{S} - \ket{L}$ photons).
    \begin{proof}
        For the transverse photons $\ket{T}$
        \begin{equation*}
            \hat b_{-, \mathbf p} \ket{\mathbf q, \lambda= 1,2} = \hat b_{-, \mathbf p} \hat a_{\mathbf q}^{\dagger (1,2)} \ket{0} = 0 ~.
        \end{equation*}
        For the longitudinal photons $\ket{L}$
        \begin{equation*}
        \begin{aligned}
            \hat b_{-, \mathbf p} \ket{\mathbf q, \lambda=3} & = \hat b_{-, \mathbf p} \hat a_{\mathbf q}^{\dagger (3)} \ket{0} = \hat b_{-, \mathbf p} \frac{\hat b^\dagger_{+, \mathbf p} - \hat b^\dagger_{-, \mathbf p}}{2} \ket{0} \\ &  = \frac{1}{2} \underbrace{\hat b_{-, \mathbf p} \hat b^\dagger_{+, \mathbf p}}_{[\hat b_{-, \mathbf p} ,\hat b^\dagger_{+, \mathbf p}] + \hat b^\dagger_{+, \mathbf p} \hat b_{-, \mathbf p} } \ket{0} - \frac{1}{2} \underbrace{\hat b_{-, \mathbf p} \hat b^\dagger_{-, \mathbf p}}_{[\hat b_{-, \mathbf p}, \hat b^\dagger_{-, \mathbf p}] + \hat b^\dagger_{-, \mathbf p} \hat b_{-, \mathbf p}} \ket{0} \\ & = \frac{1}{2} \underbrace{[\hat b_{-, \mathbf p} ,\hat b^\dagger_{+, \mathbf p}]}_{- (2\pi)^3 \delta^3 (\mathbf p - \mathbf q)} \ket{0} + \frac{1}{2} \hat b^\dagger_{+, \mathbf p} \underbrace{\hat b_{-, \mathbf p} \ket{0}}_0 - \frac{1}{2} \underbrace{[\hat b_{-, \mathbf p}, \hat b^\dagger_{-, \mathbf p}]}_{0} \ket{0} - \frac{1}{2} \hat b^\dagger_{-, \mathbf p} \underbrace{\hat b_{-, \mathbf p} \ket{0}}_0 \\ & = - (2\pi)^3 \delta^3 (\mathbf p - \mathbf q) \ket{0} \neq 0 ~.
        \end{aligned}
        \end{equation*}
        For the timelike photons $\ket{S}$
        \begin{equation*}
        \begin{aligned}
            \hat b_{-, \mathbf p} \ket{\mathbf q, \lambda=0} & = \hat b_{-, \mathbf p} \hat a_{\mathbf q}^{\dagger (0)} \ket{0} = \hat b_{-, \mathbf p} \frac{\hat b^\dagger_{+, \mathbf p} + \hat b^\dagger_{-, \mathbf p}}{2} \ket{0} \\ &  = \frac{1}{2} \underbrace{\hat b_{-, \mathbf p} \hat b^\dagger_{+, \mathbf p}}_{[\hat b_{-, \mathbf p} ,\hat b^\dagger_{+, \mathbf p}] + \hat b^\dagger_{+, \mathbf p} \hat b_{-, \mathbf p} } \ket{0} + \frac{1}{2} \underbrace{\hat b_{-, \mathbf p} \hat b^\dagger_{-, \mathbf p}}_{[\hat b_{-, \mathbf p}, \hat b^\dagger_{-, \mathbf p}] + \hat b^\dagger_{-, \mathbf p} \hat b_{-, \mathbf p}} \ket{0} \\ & = \frac{1}{2} \underbrace{[\hat b_{-, \mathbf p} ,\hat b^\dagger_{+, \mathbf p}]}_{- (2\pi)^3 \delta^3 (\mathbf p - \mathbf q)} \ket{0} + \frac{1}{2} \hat b^\dagger_{+, \mathbf p} \underbrace{\hat b_{-, \mathbf p} \ket{0}}_0 + \frac{1}{2} \underbrace{[\hat b_{-, \mathbf p}, \hat b^\dagger_{-, \mathbf p}]}_{0} \ket{0} + \frac{1}{2} \hat b^\dagger_{-, \mathbf p} \underbrace{\hat b_{-, \mathbf p} \ket{0}}_0 \\ & = - (2\pi)^3 \delta^3 (\mathbf p - \mathbf q) \ket{0} \neq 0 ~.
        \end{aligned}
        \end{equation*}
        For the $\ket{S} + \ket{L}$ photons
        \begin{equation*}
        \begin{aligned}
            \hat b_{-, \mathbf p} \ket{\mathbf q, S + L} & = \underbrace{\hat b_{-, \mathbf p} \hat b_{+,\mathbf q}^{\dagger}}_{[\hat b_{-, \mathbf p}, \hat b_{+,\mathbf q}^{\dagger}] + \hat b_{+,\mathbf q}^{\dagger} \hat b_{-, \mathbf p} } \ket{0} \\ & = \underbrace{[\hat b_{-, \mathbf p}, \hat b_{+,\mathbf q}^{\dagger}]}_{- (2\pi)^3 \delta^3 (\mathbf p -\mathbf q)} \ket{0} + \hat b_{+,\mathbf q}^{\dagger} \underbrace{\hat b_{-, \mathbf p} \ket{0}}_0 \\ & = - (2\pi)^3 \delta^3 (\mathbf p -\mathbf q) \ket{0} \neq 0 ~.
        \end{aligned}
        \end{equation*}
        For the $\ket{S} - \ket{L}$ photons
        \begin{equation*}
        \begin{aligned}
            \hat b_{-, \mathbf p} \ket{\mathbf q, S - L} & = \underbrace{\hat b_{-, \mathbf p} \hat b_{-,\mathbf q}^{\dagger}}_{[\hat b_{-, \mathbf p}, \hat b_{-,\mathbf q}^{\dagger}] + \hat b_{-,\mathbf q}^{\dagger} \hat b_{-, \mathbf p} } \ket{0} \\ & = \underbrace{[\hat b_{-, \mathbf p}, \hat b_{-,\mathbf q}^{\dagger}]}_{0} \ket{0} + \hat b_{-,\mathbf q}^{\dagger} \underbrace{\hat b_{-, \mathbf p} \ket{0}}_0 = 0 ~.
        \end{aligned}
        \end{equation*}
    \end{proof}

    Notice that the photons $\ket{S} - \ket{L}$ or $\ket{S} + \ket{L}$ have zero norm. This is true even for $n$ particles state.
    \begin{proof}
        In fact, given
        \begin{equation*}
            \hat b_{-, \mathbf p}^\dagger \ket{0} = \ket{\mathbf p, S - L} ~,
        \end{equation*}
        we have 
        \begin{equation*}
        \begin{aligned}
            \braket{\mathbf p, S-L}{\mathbf p, S-L} & = \bra{0} \hat b_{-, \mathbf p} \hat b_{-, \mathbf p}^\dagger \ket{0} = \bra{0} \underbrace{\hat b_{-, \mathbf p} \hat b_{-, \mathbf p}^\dagger}_{[\hat b_{-, \mathbf p} , \hat b_{-, \mathbf p}^\dagger] + \hat b_{-, \mathbf p}^\dagger \hat b_{-, \mathbf p}} \ket{0} \\ & = \bra{0} \underbrace{[\hat b_{-, \mathbf p} , \hat b_{-, \mathbf p}^\dagger]}_0 \ket{0} + \bra{0} \hat b_{-, \mathbf p}^\dagger \underbrace{\hat b_{-, \mathbf p} \ket{0}}_0 = 0 ~.
        \end{aligned}
        \end{equation*}
        Simirly, given
        \begin{equation*}
            \hat b_{+, \mathbf p}^\dagger \ket{0} = \ket{\mathbf p, S + L} ~,
        \end{equation*}
        we have 
        \begin{equation*}
        \begin{aligned}
            \braket{\mathbf p, S+L}{\mathbf p, S+L} & = \bra{0} \hat b_{+, \mathbf p} \hat b_{+, \mathbf p}^\dagger \ket{0} = \bra{0} \underbrace{\hat b_{+, \mathbf p} \hat b_{+, \mathbf p}^\dagger}_{[\hat b_{+, \mathbf p} , \hat b_{+, \mathbf p}^\dagger] + \hat b_{+, \mathbf p}^\dagger \hat b_{+, \mathbf p}} \ket{0} \\ & = \bra{0} \underbrace{[\hat b_{+, \mathbf p} , \hat b_{+, \mathbf p}^\dagger]}_0 \ket{0} + \bra{0} \hat b_{+, \mathbf p}^\dagger \underbrace{\hat b_{+, \mathbf p} \ket{0}}_0 = 0 ~.
        \end{aligned}
        \end{equation*}
    \end{proof}

    \begin{example}
        Consider a state in which $2$ photons have polarisation $\ket{T}$ and $\ket{S} - \ket{L}$. It has zero norm. In fact, given
        \begin{equation*}
            \hat b_{-, \mathbf p}^\dagger \hat a_{\mathbf q}^{\dagger (1,2)} \ket{0} = \ket{\mathbf q, T; \mathbf p, S-L} ~,
        \end{equation*}
        we have 
        \begin{equation*}
        \begin{aligned}
            & \braket{\mathbf q, T; \mathbf p, S-L}{\mathbf q, T; \mathbf p, S-L} = \bra{0} \hat a_{\mathbf q}^{(1,2)} \hat b_{-, \mathbf p} \hat b_{-, \mathbf p}^\dagger \hat a_{\mathbf q}^{\dagger (1,2)} \ket{0} \\ & = \bra{0} \hat b_{-, \mathbf p} \hat b_{-, \mathbf p}^\dagger \underbrace{\hat a_{\mathbf q}^{\dagger (1,2)} \hat a_{\mathbf q}^{(1,2)}}_{[\hat a_{\mathbf q}^{\dagger (1,2)} , \hat a_{\mathbf q}^{(1,2)}] + \hat a_{\mathbf q}^{(1,2)} \hat a_{\mathbf q}^{\dagger (1,2)}} \ket{0} \\ & = \bra{0} \hat b_{-, \mathbf p} \hat b_{-, \mathbf p}^\dagger \underbrace{[\hat a_{\mathbf q}^{\dagger (1,2)} , \hat a_{\mathbf q}^{(1,2)}] }_{(2\pi)^3 \delta^3 (0) } \ket{0} + \bra{0} \hat b_{-, \mathbf p} \hat b_{-, \mathbf p}^\dagger \hat a_{\mathbf q}^{(1,2)} \underbrace{\hat a_{\mathbf q}^{\dagger (1,2)} \ket{0}}_0 \\ & = (2\pi)^3 \delta^3 (0) \bra{0} \underbrace{\hat b_{-, \mathbf p} \hat b_{-, \mathbf p}^\dagger}_{[\hat b_{-, \mathbf p} , \hat b_{-, \mathbf p}^\dagger] + \hat b_{-, \mathbf p}^\dagger \hat b_{-, \mathbf p}} \ket{0} \\ & = (2\pi)^3 \delta^3 (0) \bra{0} \underbrace{[\hat b_{-, \mathbf p} , \hat b_{-, \mathbf p}^\dagger]}_0 \ket{0} + (2\pi)^3 \delta^3 (0) \bra{0} \hat b_{-, \mathbf p}^\dagger \underbrace{\hat b_{-, \mathbf p} \ket{0}}_0 = 0 ~.
        \end{aligned}
        \end{equation*}
    \end{example}

    Zero-norm states can be ignored. In fact, a state with $n_T$ transversal photons is gauge equivalent to a state with $n_T$ transversal photons and $n$ pairs of timelike and longitudinal photons, for all $n$. This quantum gauge symmetry descends from the classical one. See Figure~\eqref{fig:gauge2}
    
    \begin{figure}[h!]
        \centering
        \begin{tikzpicture}
        \draw[smooth cycle, tension=0.4] plot coordinates{(2,2) (-2.5,0) (3,-2) (6,1)} node at (3,2.3) {Total Fock space};
        \draw[smooth cycle, tension=0.4] plot coordinates { (0.75, 0) (1.25, 1.5) (3.5, 1.5) (4, 0)}  node [label={[label distance=-0.3cm, xshift=-1.5cm, yshift=-0.75cm]: Physical states}] {};

        \draw[] (0, 1) to[bend right=20] (-3,2) node[left] {$\ket{S} + \ket{L}$};
        \filldraw[black] (0, 1) circle (0.05);

        \draw[] (-2, 0) to[bend right=20] (-3,0) node[left] {$\ket{S}$};
        \filldraw[black] (-2, 0) circle (0.05);

        \draw[] (0, -1) to[bend right=20] (-3,-2) node[left] {$\ket{L}$};
        \filldraw[black] (0, -1) circle (0.05);

        \draw[] (2, 0.75) to[bend right=20] (8, 1) node[right] {$\ket{T}$};
        \filldraw[black] (2, 0.75) circle (0.05);

        \draw[] (2.5, 1.1) to[bend right=20] (8, 2) node[right] {$\ket{T} + n (\ket{S} - \ket{L})$};
        \filldraw[black] (2.5,1.1) circle (0.05);

        \draw[] (1,0.25) to[bend right=20] (2, 0.75) to[bend left=20] (3.25, 1.25);

        \end{tikzpicture}
        \caption{Pictorial representation of Fock space of photons.}
        \label{fig:gauge2}
    \end{figure}

    Two states are physically equivalent if the give the same expectation value for all observables. In fact, the Hamiltonian depends only on the number of tranversal photons and the zero-norm $\ket{S} - \ket{L}$ photons are not considered at all. Therefore,given the Hamiltonian
    \begin{equation*}
        \hat H = \int \frac{d^3 p}{(2\pi)^3} |\mathbf p| ( - \hat a_{\mathbf p}^{\dagger (0)} \hat a_{\mathbf p}^{(0)} + \hat a_{\mathbf p}^{\dagger (1)} \hat a_{\mathbf p}^{(1)} + \hat a_{\mathbf p}^{\dagger (2)} \hat a_{\mathbf p}^{(2)} + \hat a_{\mathbf p}^{\dagger (3)} \hat a_{\mathbf p}^{(3)})  ~.
    \end{equation*}
    and given a physical state $\ket{\psi}$ which satisfies~\eqref{GB}, the expectation value of the hamiltonian is given only by the transversal polarisations
    \begin{equation*}
    \begin{aligned}
        \bra{\psi} \hat H \ket{\psi} & = \int \frac{d^3 p}{(2\pi)^3} |\mathbf p| ( - \cancel{\bra{\psi} \hat a_{\mathbf p}^{\dagger (0)} \hat a_{\mathbf p}^{(0)} \ket{\psi}} + \bra{\psi} \hat a_{\mathbf p}^{\dagger (1)} \hat a_{\mathbf p}^{(1)} \ket{\psi} + \bra{\psi} \hat a_{\mathbf p}^{\dagger (2)} \hat a_{\mathbf p}^{(2)} \ket{\psi} + \cancel{\bra{\psi} \hat a_{\mathbf p}^{\dagger (3)} \hat a_{\mathbf p}^{(3)} \ket{\psi}} ) \\ & = \int \frac{d^3 p}{(2\pi)^3} |\mathbf p| ( \bra{\psi} \hat a_{\mathbf p}^{\dagger (1)} \hat a_{\mathbf p}^{(1)} \ket{\psi} + \bra{\psi} \hat a_{\mathbf p}^{\dagger (2)} \hat a_{\mathbf p}^{(2)} \ket{\psi} ) = n_T |\mathbf p| ~.
    \end{aligned}
    \end{equation*}
    \begin{proof}
        The Lagrangian is
        \begin{equation*}
        \begin{aligned}
            \mathcal L & = - \frac{1}{4} F_{\mu\nu} F^{\mu\nu} - \frac{1}{2} \partial_\mu A^\mu \partial_\nu A^\nu = - \frac{1}{4} (\partial_\mu A_\nu - \partial_\nu A_\mu) (\partial^\mu A^\nu - \partial^\nu A^\mu) - \frac{1}{2} \partial_\mu A^\mu \partial_\nu A^\nu \\ & = - \frac{1}{4} \partial_\mu A_\nu \partial^\mu A^\nu + \frac{1}{4} \partial_\mu A_\nu \partial^\nu A^\mu + \frac{1}{4} \partial_\nu A_\mu \partial^\mu A^\nu - \frac{1}{4} \partial_\nu A_\mu \partial^\nu A^\mu - \frac{1}{2} \partial_\mu A^\mu \partial_\nu A^\nu \\ & = - \frac{1}{2} \partial_\mu A_\nu \partial^\mu A^\nu + \frac{1}{2} \underbrace{(\partial_\mu A_\nu \partial^\nu A^\mu - \partial_\mu A^\mu \partial_\nu A^\nu) }_{ - A_\nu  \partial_\mu \partial^\nu A^\mu - A^\mu \partial_\mu \partial_\nu A^\nu + \textnormal{boundary terms}} \\ & = - \frac{1}{2} \partial_\mu A_\nu \partial^\mu A^\nu + \frac{1}{2} (- A_\nu  \partial_\mu \partial^\nu A^\mu - A^\mu \partial_\mu \partial_\nu A^\nu) \\ & = - \frac{1}{2} \partial_\mu A_\nu \partial^\mu A^\nu + \frac{1}{2} (- \cancel{A_\mu  \partial_\nu \partial^\mu A^\nu} + \cancel{A^\mu \partial_\nu \partial_\mu A^\nu}) = - \frac{1}{2} \partial_\mu A_\nu \partial^\mu A^\nu ~,
        \end{aligned}
        \end{equation*}
        where we have integrated by parts since the Lagrangian is always integrated to obtain the action. The conjugate field is 
        \begin{equation*}
            \pi^\mu = \pdv{\mathcal L}{\dot A_\mu} = - \dot A_\mu ~.
        \end{equation*}
        The Hamiltonian is 
        \begin{equation*}
        \begin{aligned}
            \mathcal H & = \pi^\mu \underbrace{\dot A_\mu}_{- \pi_\mu} - \mathcal L = - \pi^\mu \pi_\mu + \frac{1}{2} \partial_\mu A_\nu \partial^\mu A^\nu \\ & = - \pi^\mu \pi_\mu + \frac{1}{2} \underbrace{\dot A^\mu \dot A_\mu}_{\pi^\mu \pi_\mu} + \frac{1}{2} \partial_i A_\mu \partial^i A^\mu = - \frac{1}{2} \pi^\mu \pi_\mu + \frac{1}{2} \partial_i A_\mu \partial^i A^\mu = \mathcal H_1 + \mathcal H_2 ~.
        \end{aligned}
        \end{equation*}
        Now, we promote to operator. 
        \begin{equation*}
            \hat H = \int d^3 x ~ \mathcal H ~.
        \end{equation*}
        The first part is 
        \begin{equation*}
        \begin{aligned}
            & \hat H_1 = - \frac{1}{2} \int d^3 x ~ \hat \pi^\mu \hat \pi_\mu \\ & = - \frac{1}{2} \int d^3 x \int \frac{d^3 p}{(2\pi)^3} i \sqrt{\frac{|\mathbf p|}{2}} \sum_{\lambda=0}^{3} \epsilon^{\mu(\lambda)} (\mathbf p) \Big ( \hat a_{\mathbf p}^{(\lambda)}   \exp(i \mathbf p \cdot \mathbf x) - \hat a_{\mathbf p}^{\dagger (\lambda)}   \exp(- i \mathbf p \cdot \mathbf x) \Big) \\ & \quad \int \frac{d^3 q}{(2\pi)^3} i \sqrt{\frac{|\mathbf q|}{2}} \sum_{\lambda'=0}^{3} \epsilon_{\mu(\lambda')} (\mathbf q)  \Big ( \hat a_{\mathbf q}^{(\lambda')}   \exp(i \mathbf q \cdot \mathbf x) - \hat a_{\mathbf q}^{\dagger (\lambda')}   \exp(- i \mathbf q \cdot \mathbf x) \Big) \\ & = \frac{1}{4} \int \frac{d^3 x ~ d^3 p ~ d^3 q}{(2\pi)^6} \sqrt{|\mathbf p| |\mathbf q|} \sum_{\lambda=0}^{3} \sum_{\lambda'=0}^{3} \underbrace{\epsilon^{\mu(\lambda)} (\mathbf p) \epsilon_{\mu(\lambda')} (\mathbf q)}_{\eta^{\lambda \lambda'}} ( \hat a_{\mathbf p}^{(\lambda)}   \hat a_{\mathbf q}^{(\lambda')}   \underbrace{\exp(i  \mathbf x \cdot (\mathbf p + \mathbf q)) }_{\delta(\mathbf p + \mathbf q)}  \\ & \quad - \hat a_{\mathbf p}^{(\lambda)}  \hat a_{\mathbf q}^{\dagger (\lambda')}   \underbrace{\exp(i  \mathbf x \cdot (\mathbf p - \mathbf q)) }_{\delta(\mathbf p - \mathbf q)} - \hat a_{\mathbf p}^{\dagger (\lambda)}   \hat a_{\mathbf q}^{(\lambda')}   \underbrace{\exp(i  \mathbf x \cdot (- \mathbf p + \mathbf q)) }_{\delta(\mathbf p - \mathbf q)} \\ & \quad + \hat a_{\mathbf p}^{\dagger (\lambda)}   \hat a_{\mathbf q}^{\dagger (\lambda')}   \underbrace{\exp(- i  \mathbf x \cdot (\mathbf p + \mathbf q)) }_{\delta(\mathbf p + \mathbf q)} )
        \end{aligned}
        \end{equation*}
        \begin{equation*}
        \begin{aligned}
            & = \frac{1}{4} \int \frac{d^3 x}{(2\pi)^3} |\mathbf p| \sum_{\lambda=0}^{3} \sum_{\lambda'=0}^{3} \eta^{\lambda \lambda'} ( \hat a_{\mathbf p}^{(\lambda)} \hat a_{- \mathbf p}^{(\lambda')}  - \hat a_{\mathbf p}^{(\lambda)} \hat a_{\mathbf p}^{\dagger (\lambda')}  - \hat a_{\mathbf p}^{\dagger (\lambda)} \hat a_{\mathbf p}^{(\lambda')} + \hat a_{\mathbf p}^{\dagger (\lambda)} \hat a_{- \mathbf p}^{\dagger (\lambda')} ) ~.
        \end{aligned}
        \end{equation*}
        Given 
        \begin{equation*}
            \partial_i \hat A_\mu = \int \frac{d^3 p}{(2\pi)^3} \frac{1}{\sqrt{2 |\mathbf p|}} \sum_{\lambda=0}^{3} \epsilon_\mu^{(\lambda)} (\mathbf p) \Big ( (- i p_i )\hat a_{\mathbf p}^{(\lambda)} \exp(i \mathbf p \cdot \mathbf x) + (i p_i)\hat a_{\mathbf p}^{\dagger (\lambda)}  \exp(- i \mathbf p \cdot \mathbf x) \Big) ~,
        \end{equation*}
        the second part is
        \begin{equation*}
        \begin{aligned}
            & \hat H_2 = \frac{1}{2} \int d^3 x ~ \partial_i \hat A_\mu \partial^i \hat A^\mu \\ & = \frac{1}{2} \int d^3 x \int \frac{d^3 p}{(2\pi)^3} \frac{1}{\sqrt{2 |\mathbf p|}} \sum_{\lambda=0}^{3} \epsilon_\mu^{(\lambda)} (\mathbf p) \Big ( (- i p_i ) \hat a_{\mathbf p}^{(\lambda)} \exp(i \mathbf p \cdot \mathbf x) + (i p_i)\hat a_{\mathbf p}^{\dagger (\lambda)}  \exp(- i \mathbf p \cdot \mathbf x) \Big) \\ & \quad \int \frac{d^3 q}{(2\pi)^3} \frac{1}{\sqrt{2 |\mathbf q}} \sum_{\lambda'=0}^{3} \epsilon^{\mu(\lambda')} (\mathbf q) \Big ( (- i q^i )\hat a_{\mathbf q}^{(\lambda')} \exp(i \mathbf q \cdot \mathbf x) + (i q^i)\hat a_{\mathbf q}^{\dagger (\lambda')} \exp(- i \mathbf q \cdot \mathbf x) \Big) \\ & = \frac{1}{2} \int \frac{d^3 x ~ d^3 p ~ d^3 q}{(2\pi)^6} \frac{1}{2\sqrt{|\mathbf p| |\mathbf q|}} \sum_{\lambda=0}^{3} \sum_{\lambda'=0}^{3} \underbrace{\epsilon^{\mu(\lambda)} (\mathbf p) \epsilon_{\mu}^{(\lambda')} (\mathbf q)}_{\eta^{\lambda \lambda'}} p_i q^i ( \hat a_{\mathbf p}^{(\lambda)} \hat a_{\mathbf q}^{(\lambda)} \underbrace{\exp(i  \mathbf x \cdot (\mathbf p + \mathbf q)) }_{\delta(\mathbf p + \mathbf q)} \\ & \quad - \hat a_{\mathbf p}^{(\lambda)}\hat a_{\mathbf q}^{\dagger (\lambda)} \underbrace{\exp(i \mathbf x \cdot (\mathbf p - \mathbf q)) }_{\delta(\mathbf p - \mathbf q)} - \hat a_{\mathbf p}^{\dagger (\lambda)} \hat a_{\mathbf q}^{(\lambda)} \underbrace{\exp( i  \mathbf x \cdot ( -\mathbf p + \mathbf q)) }_{\delta(\mathbf p - \mathbf q)} \\ & \quad + \hat a_{\mathbf p}^{\dagger (\lambda)} \hat a_{\mathbf q}^{\dagger (\lambda)} \underbrace{\exp(- i  \mathbf x \cdot (\mathbf p + \mathbf q)) }_{\delta(\mathbf p + \mathbf q)})
        \end{aligned}
        \end{equation*}
        \begin{equation*}
        \begin{aligned}
            & = \frac{1}{2} \int \frac{d^3 p}{(2\pi)^3} \frac{1}{2 |\mathbf p|} \sum_{\lambda=0}^{3} \sum_{\lambda'=0}^{3} \eta^{\lambda \lambda'} |\mathbf p|^2 (- \hat a_{\mathbf p}^{(\lambda)} \hat a_{- \mathbf p}^{(\lambda')} - \hat a_{\mathbf p}^{(\lambda)} \hat a_{\mathbf p}^{\dagger (\lambda')} - \hat a_{\mathbf p}^{\dagger (\lambda)} \hat a_{\mathbf p}^{(\lambda')} - \hat a_{\mathbf p}^{\dagger (\lambda)} \hat a_{-\mathbf p}^{\dagger (\lambda')} ) \\ & = \frac{1}{4} \int \frac{d^3 p}{(2\pi)^3} |\mathbf p| \sum_{\lambda=0}^{3} \sum_{\lambda'=0}^{3} \eta^{\lambda \lambda'} (- \hat a_{\mathbf p}^{(\lambda)} \hat a_{- \mathbf p}^{(\lambda')} - \hat a_{\mathbf p}^{(\lambda)} \hat a_{\mathbf p}^{\dagger (\lambda')} - \hat a_{\mathbf p}^{\dagger (\lambda)} \hat a_{\mathbf p}^{(\lambda')} - \hat a_{\mathbf p}^{\dagger (\lambda)} \hat a_{-\mathbf p}^{\dagger (\lambda')} ) ~.
        \end{aligned}
        \end{equation*}
        Putting everything together 
        \begin{equation*}
        \begin{aligned}
            \hat H & = \frac{1}{4} \int \frac{d^3 p}{(2\pi)^3} |\mathbf p| \sum_{\lambda=0}^{3} \sum_{\lambda'=0}^{3} \eta^{\lambda \lambda'} (- \cancel{\hat a_{\mathbf p}^{(\lambda)} \hat a_{- \mathbf p}^{(\lambda')}} - \hat a_{\mathbf p}^{(\lambda)} \hat a_{\mathbf p}^{\dagger (\lambda')} - \hat a_{\mathbf p}^{\dagger (\lambda)} \hat a_{\mathbf p}^{(\lambda')} - \cancel{\hat a_{\mathbf p}^{\dagger (\lambda)} \hat a_{-\mathbf p}^{\dagger (\lambda')}} ) \\ & \quad + \frac{1}{4} \int \frac{d^3 x}{(2\pi)^3} |\mathbf p| \sum_{\lambda=0}^{3} \sum_{\lambda'=0}^{3} \eta^{\lambda \lambda'} ( \cancel{\hat a_{\mathbf p}^{(\lambda)} \hat a_{- \mathbf p}^{(\lambda')}}  - \hat a_{\mathbf p}^{(\lambda)} \hat a_{\mathbf p}^{\dagger (\lambda')}  - \hat a_{\mathbf p}^{\dagger (\lambda)} \hat a_{\mathbf p}^{(\lambda')} + \cancel{\hat a_{\mathbf p}^{\dagger (\lambda)} \hat a_{- \mathbf p}^{\dagger (\lambda')}} ) \\ & = \frac{1}{2} \int \frac{d^3 p}{(2\pi)^3} |\mathbf p| \sum_{\lambda=0}^{3} \sum_{\lambda'=0}^{3} \eta^{\lambda \lambda'} (- \hat a_{\mathbf p}^{(\lambda)} \hat a_{\mathbf p}^{\dagger (\lambda')} - \hat a_{\mathbf p}^{\dagger (\lambda)} \hat a_{\mathbf p}^{(\lambda')})
        \end{aligned}
        \end{equation*}
        which in normal ordering is 
        \begin{equation*}
        \begin{aligned}
            \hat H & = \int \frac{d^3 p}{(2\pi)^3} |\mathbf p| \sum_{\lambda=0}^{3} \sum_{\lambda'=0}^{3} \eta^{\lambda \lambda'} (- \hat a_{\mathbf p}^{\dagger (\lambda)} \hat a_{\mathbf p}^{(\lambda')}) \\ & = \int \frac{d^3 p}{(2\pi)^3} |\mathbf p| \sum_{\lambda=0}^{3} (- \hat a_{\mathbf p}^{\dagger (0)} \hat a_{\mathbf p}^{(0)} + \hat a_{\mathbf p}^{\dagger (1)} \hat a_{\mathbf p}^{(1)} + \hat a_{\mathbf p}^{\dagger (2)} \hat a_{\mathbf p}^{(2)}+ \hat a_{\mathbf p}^{\dagger (3)} \hat a_{\mathbf p}^{(3)}) ~.
        \end{aligned}
        \end{equation*}
    \end{proof}

    To summarise, any observables gives only results in terms of transversal photons and gauge equivalent states are all physically equivalent, since they are indistinguishable in measurements of observables. The only truly physical degrees of freedome of a massless spin-$1$ particle described by the Maxwell's field are the two transversal polarisations. However, this is not true for massive photons, since the longitudinal polararisation is the thire degree of freedom.

\section{Massive photons}    

    Massive photons are described by the Proca lagrangian
    \begin{equation*}
        \mathcal L = - \frac{1}{4} F^{\mu\nu} F_{\mu\nu} - \frac{m^2}{2} A_\mu A^\mu ~,
    \end{equation*}
    where the equations of motion are 
    \begin{equation*}
        \partial_\mu F^{\mu\nu} + m^2 A^\nu = 0~.
    \end{equation*}
    Notice that the Lorenz gauge is automatically always satisfied and it is not imposed by hand.
    \begin{proof}
        In fact, we have
        \begin{equation}
            0 = \underbrace{\partial_\mu \partial_\nu}_{\text{symm}} \underbrace{F^{\mu\nu}}_{\text{anti}} + m^2 \partial_\nu A^\nu = m^2 \partial_\nu A^\nu ~.
        \end{equation}
    \end{proof}
    The equations of motion become
    \begin{equation*}
        (\Box + m^2) A^\mu (x) = 0 ~.
    \end{equation*}
    \begin{proof}
        In fact, we have
        \begin{equation*}
            0 = \partial_\mu F^{\mu\nu} + m^2 A^\nu = \partial_\mu \partial^\mu A^\nu - \cancel{\partial_\mu \partial^\nu A^\nu} + m^2 A^\nu = (\Box + m^2) A^\nu ~.
        \end{equation*}
    \end{proof}
    Therefore, each $A^\mu$ satisfies the Klein-Gordon equation but there are only $3$ independent degrees of freedom by the Lorenz gauge. Finally, notice that the mass term breaks the gauge symmetry, since for a gauge transformation ${A'}_\mu = A_\mu + \partial_\mu \lambda$
    \begin{equation*}
        \frac{m}{2} {A'}^\mu {A'}_\mu \neq \frac{m}{2} A^\mu A_\mu ~.
    \end{equation*}


\backmatter

\nocite{qft1lecture}

\clearpage
\phantomsection
\printbibliography

\end{document}
