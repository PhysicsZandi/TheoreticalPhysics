\part{Maxwell theory}

\chapter{Classical electrodynamics}

\section{Maxwell lagrangian}

    The lagrangian of the electromagnetic field is 
    \begin{equation*}
        \mathcal L = - \frac{1}{4} F_{\mu\nu} F^{\mu\nu} ~,
    \end{equation*}
    where the electromagnetic tensor is 
    \begin{equation*}
        F_{\mu\nu} = \partial_\mu A_\nu - \partial_\nu A_\mu = \begin{bmatrix}
            0 & E_1 & E_2 & E_3 \\ 
            -E_1 & 0 & - B_3 & B_2 \\ 
            - E_2 & B_3 & 0 & - B_1 \\ 
            - E_3 & -B_2 & B_1 & 0 \\
        \end{bmatrix}
    \end{equation*}
    and the $4$-potential is $A^\mu = (\phi, \mathbf A)$.

    The equation of motion (Maxwell's equations) are 
    \begin{equation*}
        \partial_\mu F^{\mu\nu} = 0 ~.
    \end{equation*}
    \begin{proof}
        Using~\eqref{eleq}
        \begin{equation*}
            0 = \partial_\mu \pdv{\mathcal L}{\partial_\mu A_\nu} = - \partial_\mu F^{\mu\nu} ~.
        \end{equation*}
    \end{proof}

    The electromagnetic tensor satisfies the Bianchi identity 
    \begin{equation*}
        \partial_\mu F_{\nu\lambda} + \partial_\nu F_{\lambda \mu} + \partial_\lambda F_{\mu \nu} = 0 ~.
    \end{equation*}
    \begin{proof}
        In fact, by the symmetry of partial derivatives, 
        \begin{equation*}
        \begin{aligned}
            \partial_\mu F_{\nu\lambda} + \partial_\nu F_{\lambda \mu} + \partial_\lambda F_{\mu \nu} & = \partial_\mu (\partial_\nu A_\lambda - \partial_\lambda A_\nu) + \partial_\nu (\partial_\lambda A_\mu - \partial_\mu A_\lambda) + \partial_\lambda (\partial_\mu A_\nu - \partial_\nu A_\mu) \\ & = \partial_\mu \partial_\nu A_\lambda -  \partial_\mu \partial_\lambda A_\nu + \partial_\nu \partial_\lambda A_\mu - \partial_\nu  \partial_\mu A_\lambda + \partial_\lambda \partial_\mu A_\nu - \partial_\lambda \partial_\nu A_\mu \\ & = \cancel{\partial_\nu \partial_\mu A_\lambda} - \cancel{\partial_\lambda \partial_\mu A_\nu} +  \cancel{\partial_\lambda \partial_\nu A_\mu} - \cancel{\partial_\nu \partial_\mu A_\lambda} + \cancel{\partial_\lambda \partial_\mu A_\nu} - \cancel{\partial_\lambda \partial_\nu A_\mu} = 0 ~.
        \end{aligned}
        \end{equation*}
    \end{proof}
    The Bianchi identity can be written in terms of the dual tensor
    \begin{equation*}
        \tilde F^{\mu\nu} = - \frac{1}{2} \epsilon^{\mu\nu\rho\sigma} F_{\rho\sigma} ~,
    \end{equation*}
    explicitly 
    \begin{equation*}
        \tilde F_{\mu\nu} = \begin{bmatrix}
            0 & - B_1 & - B_2 & - B_3 \\ 
            B_1 & 0 & E_3 & -E_2 \\ 
            B_2 & -E_3 & 0 & E_1 \\ 
            B_3 & E_2 & -E_1 & 0 \\
        \end{bmatrix} ~,
    \end{equation*}
    as 
    \begin{equation*}
        \partial_\mu \tilde F^{\mu\nu} = - \frac{1}{2} \epsilon^{\mu\nu\rho\sigma} \partial_\mu  F_{\rho\sigma} = 0 ~.
    \end{equation*}
    \begin{proof}
        In fact
        \begin{equation*}
        \begin{aligned}
            0 & = \partial_\mu F_{\nu\lambda} + \partial_\nu F_{\lambda \mu} + \partial_\lambda F_{\mu \nu} \\ & = \epsilon^{\mu\nu\lambda\sigma} (\partial_\mu F_{\nu\lambda} + \partial_\nu F_{\lambda \mu} + \partial_\lambda F_{\mu \nu}) \\ & = \partial_\mu \epsilon^{\mu\nu\lambda\sigma} F_{\nu\lambda} + \partial_\nu \epsilon^{\mu\nu\lambda\sigma} F_{\lambda \mu} + \partial_\lambda \epsilon^{\mu\nu\lambda\sigma} F_{\mu \nu} \\ & = \cancel{\partial_\mu \epsilon^{\sigma\mu\nu\lambda} F_{\nu\lambda}} - \cancel{\partial_\nu \epsilon^{\sigma\nu\lambda\mu} F_{\lambda\mu}} + \partial_\lambda \epsilon^{\lambda\sigma\mu\nu} F_{\mu \nu} \\ & = \epsilon^{\mu\nu\rho\sigma} \partial_\mu  F_{\rho\sigma} ~.
        \end{aligned}
        \end{equation*}
    \end{proof}

    It can be proved that the Maxwell's equations in covariant formalism can be written as 
    \begin{equation*}
        \boldsymbol \nabla \cdot \mathbf B = 0 ~, \quad \pdv{\mathbf B}{t} = - \boldsymbol \nabla \times \mathbf E \quad \Rightarrow \quad \partial_\mu \tilde F^{\mu\nu} = 0 ~,
    \end{equation*}
    \begin{equation*}
        \boldsymbol \nabla \cdot \mathbf E = 0 ~, \quad \pdv{\mathbf E}{t} = \boldsymbol \nabla \times \mathbf E \quad \Rightarrow \quad \partial_\mu F^{\mu\nu} = 0 ~.
    \end{equation*}
    Notice that there is a duality symmetry, since if we perfom the exhange $\mathbf E \leftrightarrow - \mathbf B$, we find $F^{\mu\nu} \leftrightarrow \tilde F^{\mu\nu}$.

\section{Gauge symmetry}

    We rewrite the lagrangian in terms of temporal and spatial indices
    \begin{equation*}
        \mathcal L = - \frac{1}{2} (F_{0i} F^{0i} + F_{ij} F^{ij} ) ~.
    \end{equation*} 
    \begin{equation*}
    \begin{aligned}
        \mathcal L & = - \frac{1}{4} (\partial_\mu A_\nu - \partial_\nu A_\mu)(\partial^\mu A^\nu - \partial^\nu A^\mu) \\ & = -\frac{1}{4} (\partial_\mu A_\nu \partial^\mu A^\nu - \partial_\mu A_\nu \partial^\nu A^\mu -  \partial_\nu A_\mu \partial^\mu A^\nu + \partial_\nu A_\mu \partial^\nu A^\mu) \\ & = - \frac{1}{2} (\partial_\mu \partial^\mu A^\nu A_\nu - \partial_\nu \partial^\mu A^\nu A_\mu) \\ & = - \frac{1}{2} (\cancel{\partial_0 \partial^0 A^0 A_0} + \partial_0 \partial^0 A^i A_i + \partial_i \partial^i A^0 A_0 + \partial_i \partial^i A^j A_j - \cancel{\partial_0 \partial^0 A^0 A_0} - \partial_0 \partial^i A^0 A_i - \partial_i \partial^j A^i A_j - \partial_i \partial^0 A^i A_0) \\ & = - \frac{1}{2} (\partial_0 \partial^0 A^i A_i + \partial_i \partial^i A^0 A_0 + \partial_i \partial^i A^j A_j - \partial_0 \partial^i A^0 A_i - \partial_i \partial^j A^i A_j - \partial_i \partial^0 A^i A_0) \\ & = - \frac{1}{2} 
        (\partial_0 \partial^0 A^i A_i - \partial_0 \partial^i A_i A^0 - \partial_i  \partial^0 A^0 A^i + \partial_j \partial^j A_i A^i - \partial_i \partial^j A_j A^i + \partial_j \partial^i A_i A^j + \partial_i \partial^j A_j A^i - \partial_i \partial^i A_j A^j ) \\ & = - \frac{1}{2} ((\partial_0 A_i - \partial_i A_0) (\partial^0 A^i - \partial^i A^0) - (\partial_j A_i - \partial_i A_j) (\partial^j A^i - \partial^i A^j)) \\ & = - \frac{1}{2} (F_{0i} F^{0i} + F_{ij} F^{ij} ) ~.
    \end{aligned}
    \end{equation*}
    Notice that there is no dependence on the kinetic part of $A^0$, i.e. $\dot A^0$, which means that $A^0$ is fully determined by $A^i$. In fact, by the Gauss' law 
    \begin{equation*}
        \boldsymbol \nabla \boldsymbol \nabla A_0 = \boldsymbol \nabla \cdot \pdv{\mathbf A}{t} ~,
    \end{equation*}

    This implies that there is a gauge symmetry 
    \begin{equation*}
        {A'}_\mu (x) = A_\mu (x) + \partial_\mu \alpha (x) ~.
    \end{equation*}
    \begin{proof}
        In fact 
        \begin{equation*}
            {F'}^{\mu\nu} = \partial_\mu {A'}_\nu - \partial_\nu {A'}_\mu = \partial_\mu A_\nu - \partial_\nu A_\mu + \cancel{\partial_\mu \partial_\nu \alpha(x)} - \cancel{\partial_\nu \partial_\mu \alpha(x)} = \partial_\mu A_\nu - \partial_\nu A_\mu = F^{\mu\nu} ~,
        \end{equation*}
        which means 
        \begin{equation*}
            \mathcal L' = \mathcal L ~.
        \end{equation*}
    \end{proof}

    Actually, there is a second gauge transformation 
    \begin{equation*}
        {A''}_\mu (x) = {A'}_\mu (x) + \partial_\mu \beta (x) = A_\mu (x) + \partial_\mu (\alpha (x) + \beta (x) ) ~.
    \end{equation*}
    Physically, $A$, $A'$ and $A''$ descibes the same state, therefore they define a class of equivalence of physical states. There are several choices for the representative of this class. The one we will use is the Lorenz gauge, which brings down the number of degrees of freedom to $2$ (the $2$ orthogonal polarisation direction ot the motion)
    \begin{equation*}
        \partial_\mu A^\mu = 0 ~.
    \end{equation*}
    \begin{proof}
        With a gauge transformation 
        \begin{equation*}
            {A'}_\mu (x) = A_\mu (x) + \partial_\mu \alpha (x) ~,
        \end{equation*}
        $\alpha (x)$ must satisfy 
        \begin{equation*}
            0 = \partial_\mu {A'}^\mu = \partial_\mu A^\mu + \Box \alpha (x)
        \end{equation*}
        hence 
        \begin{equation*}
            \Box \alpha(x) = \partial_\mu A^\mu ~.
        \end{equation*}

        With the second gauge transformation 
        \begin{equation*}
            {A''}_\mu (x) = {A'}_\mu (x) + \partial_\mu \beta (x) ~,
        \end{equation*}
        $\beta (x)$ must satisfy 
        \begin{equation*}
            \partial_\mu {A''}^\mu = \partial_\mu {A'}^\mu + \Box \beta (x)
        \end{equation*}
        hence 
        \begin{equation*}
            \Box \beta(x) = 0 ~.
        \end{equation*}
    \end{proof}

    The equations of motion in the Lorenz gauge are 
    \begin{equation*}
        \Box A^\mu (x) = 0 ~.
    \end{equation*}
    \begin{proof}
        In fact 
        \begin{equation*}
            0 = \partial_\mu F^{\mu\nu} = \partial_\mu (\partial^\mu A^\nu - \partial^\nu A^\mu) = \Box A^\nu - \partial^\nu \underbrace{\partial_\mu A^\mu}_0 = \Box A^\nu ~.
        \end{equation*}
    \end{proof}

\chapter{Quantisation}

\section{Hamiltonian}

    The conjugate momentum of $A$ is 
    \begin{equation*}
        \pi^\mu = (0, \mathbf E) ~.
    \end{equation*}
    \begin{proof}
        In fact 
        \begin{equation*}
            \pi^0 = \pdv{\mathcal L}{\dot A_0} = 0
        \end{equation*}
        and 
        \begin{equation*}
            \pi^i = \pdv{\mathcal L}{\dot A_i} = - \frac{1}{2} \pdv{}{\dot A_i} ((\dot A_j - \partial_j A_0))(\dot A^j - \partial^j A_0) = - \frac{1}{2} \pdv{}{\dot A_i} (F_{0j} F^{0j}) = - F^{0i} = E^i ~.
        \end{equation*}
    \end{proof}

    The hamiltonian is 
    \begin{equation*}
        H = \int d^3 x ~\Big ( \frac{1}{2}  (|E|^2 + |B|^2) - A_0 (\boldsymbol \nabla \cdot \mathbf E) \Big) ~.
    \end{equation*}
    Notice that we have a Lagrange multiplier $A_0$ to ensure the constrain of the Gauss' law, since $A_0$ is not a physical variable. In fact 
    \begin{equation*}
        0 = \dot A_0 = \pdv{\mathcal H}{A_0} = - \boldsymbol \nabla \cdot \mathbf E ~. 
    \end{equation*}
    \begin{proof}
        In fact 
        \begin{equation*}
            \mathcal H = \pi^\mu \dot A_\mu - \mathcal L = \pi^i \dot A_i - \mathcal L = - E^i \underbrace{\dot A_i}_{- E_i - \partial_i A_0} - \mathcal L = E^i E_i + E^i \partial_i A_0 - \frac{1}{2} (|E|^2 - |B|^2) = \frac{1}{2} (|E|^2 + |B|^2) + E^i \partial_i A_0 ~,
        \end{equation*}
        hence 
        \begin{equation}
            H = \int d^3 x ~ \mathcal H = \int d^3 x ~ \Big ( \frac{1}{2} (|E|^2 + |B|^2) + \underbrace{E^i \partial_i A_0}_{- A_0 \partial_i E^i + \textnormal{boundary terms}} \Big) = \int d^3 x ~ \Big ( \frac{1}{2} (|E|^2 + |B|^2) - A_0 (\boldsymbol \nabla \cdot \mathbf E) \Big) ~.
        \end{equation}
    \end{proof}

\section{Quantisation without Lorenz gauge}

    Notice that the Maxwell's equations in the Lorentz gauge are equal to the Klein-Gordon ones with mass equals to zero. This ensures that the mass shell condition is preserved. This means that each components of $A^\mu$ separately satisfies the mass shell condition. At quantum level, $A_\mu$ describes particles (photons) with energy $E_{\mathbf p} = |\mathbf p|$.

    Instead of imposind by hand the Lorenz gauge, we modify the lagrangian 
    \begin{equation*}
        \mathcal L = - \frac{1}{4} F_{\mu\nu} F^{\mu\nu} - \frac{1}{2} (\partial_\mu A^\mu)^2 ~. 
    \end{equation*}
    Therefore, the conjugate momentum is 
    \begin{equation*}
        \pi^\mu = (- \partial_\mu A^\mu, F^{i0}) = F^{\nu 0} - \eta^{\nu 0} \partial_\mu A^\mu ~.
    \end{equation*}
    \begin{proof}
        In fact 
        \begin{equation*}
            \pi^0 = \pdv{\mathcal L}{\dot A_0} = - \frac{1}{2} \pdv{\mathcal L}{\dot A_0} (\partial_\mu A^\mu)^2 = - \partial_\mu A^\mu 
        \end{equation*}
        and 
        \begin{equation*}
            \pi^i = E^i = - F^{0i} = F^{i0} ~.
        \end{equation*}

        In covariant formalism 
        \begin{equation*}
            \pi^0 = \underbrace{F^{00}}_0 - \underbrace{\eta^{00}}_1 (\partial_\mu A^\mu) = - \partial_\mu A^\mu 
        \end{equation*}
        and 
        \begin{equation*}
            \pi^i = F^{0i} - \underbrace{\eta^{i0}}_0 (\partial_\mu A^\mu) = F^{0i} ~.
        \end{equation*}
    \end{proof}

    Now, in Schoedinger picture, we promote $A^\mu(x)$ and $\pi^\mu(x)$ to operators in the Fock space by imposing the canoncial commutation (since an half-integer spin theory) relations 
    \begin{equation*}
        [\hat A_\mu (\mathbf x), \hat A_\nu (\mathbf y)] = [\hat \pi_\mu (\mathbf x), \hat \pi_\nu (\mathbf y)] = 0 ~, \quad [\hat A_\mu (\mathbf x), \hat \pi_\nu (\mathbf y)] = i \eta_{\mu\nu} \delta^3 (\mathbf x - \mathbf y) ~.
    \end{equation*}

\section{Plane waves} 

    Since the general solution of $\Box A^\mu = 0$ are a linear combinations of plane waves, we write 
    \begin{equation*}
        \hat A_\mu (\mathbf x) = \int \frac{d^3 p}{(2\pi)^3} \frac{1}{\sqrt{2 |\mathbf p|}} \Big ( \hat \xi_\mu (\mathbf p) \exp(i \mathbf p \cdot \mathbf x) + \hat \xi_\mu^\dagger (\mathbf p) \exp(- i \mathbf p \cdot \mathbf x)) ~,
    \end{equation*}
    where $\xi_\mu (\mathbf p)$ is the polarisation $4$-vector. We introduce an orthonormal basis for the polarisation $4$-vectors $\epsilon_\mu^{(\lambda)}$, with $\lambda = 0, 1, 2, 3$, such that $\epsilon_\mu^{(\lambda)} \epsilon^{\mu (\lambda')} = \eta^{\lambda \lambda'}$ or $\epsilon_\mu^{(\lambda)} \epsilon_\nu^{(\lambda')} \eta_{\lambda \lambda'} = \eta_{\mu\nu}$. The coefficients on this expansion are the annihilation operators
    \begin{equation*}
        \hat \xi_\mu (\mathbf p) = \sum_{\lambda=0}^3 \epsilon_\mu^{(\lambda)} (\mathbf p) \hat a_{\mathbf p}^{(\lambda)} ~.
    \end{equation*}

    Therefore 
    \begin{equation*}
        \hat A_\mu (\mathbf x) = \int \frac{d^3 p}{(2\pi)^3} \frac{1}{\sqrt{2 |\mathbf p|}} \sum_{\lambda=0}^{3} \epsilon_\mu^{(\lambda)} (\mathbf p) \Big ( \hat a_{\mathbf p}^{(\lambda)} (\mathbf p) \exp(i \mathbf p \cdot \mathbf x) + \hat a_{\mathbf p}^{\dagger (\lambda)} (\mathbf p) \exp(- i \mathbf p \cdot \mathbf x) \Big)  ~,
    \end{equation*}
    \begin{equation*}
        \hat \pi^\mu (\mathbf x) = \int \frac{d^3 p}{(2\pi)^3} i \sqrt{\frac{|\mathbf p|}{2}} \sum_{\lambda=0}^{3} \epsilon^{\mu(\lambda)} (\mathbf p) \Big ( \hat a_{\mathbf p}^{(\lambda)} (\mathbf p) \exp(i \mathbf p \cdot \mathbf x) - \hat a_{\mathbf p}^{\dagger (\lambda)} (\mathbf p) \exp(- i \mathbf p \cdot \mathbf x) \Big)  ~.
    \end{equation*}

    To make contact with the $2$ transversal polarisation of an electromagnetic wave, we chiise $\epsilon_\mu^{(1)}$ and $\epsilon_\mu^{(2)}$ to be orthogonal to the motion, such that $\epsilon_\mu^{(1)} p^\mu = \epsilon_\mu^{(2)} p^\mu = 0$. For example, if $p^\mu = (E, 0, 0, E)$ lies along the $z$-direction, we have $\epsilon_\mu^{(1)} p^\mu = \epsilon_0^{(1)} p^0 + \epsilon_3^{(1)} p^3 = E (\epsilon_0^{(1)} + \epsilon_3^{(1)}) = 0$, which means $\epsilon_0^{(1)} = -\epsilon_3^{(1)}$ that for convenience we put to zero $\epsilon_0^{(1)} = \epsilon_3^{(1)} = 0$. Similarly, we have $\epsilon_\mu^{(2)} p^\mu = \epsilon_0^{(2)} p^0 + \epsilon_3^{(2)} p^3 = E (\epsilon_0^{(2)} + \epsilon_3^{(2)}) = 0$, which means $\epsilon_0^{(2)} = -\epsilon_3^{(2)}$ that for convenience we put to zero $\epsilon_0^{(2)} = \epsilon_3^{(2)} = 0$. Therefore, if we choose $\epsilon_1^{(1)} = -1$, $\epsilon_2^{(1)} = 0$, $\epsilon_1^{(2)} = 0$ and $\epsilon_2^{(2)} = -1$. In matrix notation, 
    \begin{equation*}
        \epsilon^{\mu (0)} = \begin{bmatrix}
            1 \\ 0 \\ 0 \\ 0 \\
        \end{bmatrix} ~,  \epsilon^{\mu (1)} = \begin{bmatrix}
            0 \\ 1 \\ 0 \\ 0 \\
        \end{bmatrix} ~, \epsilon^{\mu (2)} = \begin{bmatrix}
            0 \\ 0 \\ 1 \\ 0 \\
        \end{bmatrix} ~, \epsilon^{\mu (3)} = \begin{bmatrix}
            0 \\ 0 \\ 0 \\ 1 \\
        \end{bmatrix} ~,
    \end{equation*}
    where $\epsilon^{\mu (0)}$ is timelike, $\epsilon^{\mu (1)}$ and $\epsilon^{\mu (2)}$ are spacelike and $\epsilon^{\mu (3)}$ is the longitudinal polarisation.

    The commutation relations induced by the canonical ones on the ladder operators are 
    \begin{equation*}
        [\hat a_{\mathbf p}^{(\lambda)}, \hat a_{\mathbf q}^{(\lambda')}] = [\hat a_{\mathbf p}^{\dagger (\lambda)}, \hat a_{\mathbf q}^{\dagger (\lambda')}] = 0 ~, \quad [\hat a_{\mathbf p}^{(\lambda)}, \hat a_{\mathbf q}^{\dagger(\lambda')}] = - (2\pi)^3 \eta^{\lambda \lambda'} \delta^3 (\mathbf p - \mathbf q) ~.
    \end{equation*}
    or, explicitly the latter, 
    \begin{equation*}
        [\hat a_{\mathbf p}^{(0)}, \hat a_{\mathbf q}^{\dagger (0)}] = - (2\pi)^3 \delta^3 (\mathbf p - \mathbf q) ~, \quad [\hat a_{\mathbf p}^{(i)}, \hat a_{\mathbf q}^{\dagger (i)}] = (2\pi)^3 \delta^3 (\mathbf p - \mathbf q) ~.
    \end{equation*}
    Notice that there is a problem, since we do not have a probabilistic intepretation for a negative norm (only for the temporal components). We cannot neither interpret $\hat a$ as a creation operator, because we would have problems for the spatial components. This kind of states are called ghosts. We could have seen it already by the lagrangian, since we had a minus sign in the temporal component
    \begin{equation*}
        \mathcal L = \frac{1}{2} (- (\dot A_0)^2 + (\dot A_1)^2 + (\dot A_2)^2 +(\dot A_3)^2 ) + \ldots ~.
    \end{equation*}
    \begin{proof}
        In fact, 
        \begin{equation*}
        \begin{aligned}
            [\hat A_\mu (\mathbf x), \hat \pi^\nu (\mathbf y)] & = [\int \frac{d^3 p}{(2\pi)^3} \frac{1}{\sqrt{2 |\mathbf p|}} \sum_{\lambda=0}^{3} \epsilon_\mu^{(\lambda)} (\mathbf p) \Big ( \hat a_{\mathbf p}^{(\lambda)} \exp(i \mathbf p \cdot \mathbf x) + \hat a_{\mathbf p}^{\dagger (\lambda)} \exp(- i \mathbf p \cdot \mathbf x) \Big), \\ & \quad \int \frac{d^3 q}{(2\pi)^3} i \sqrt{\frac{|\mathbf q|}{2}} \sum_{\lambda'=0}^{3} \epsilon^{\nu(\lambda')} (\mathbf q) \Big ( \hat a_{\mathbf q}^{(\lambda')}  \exp(i \mathbf q \cdot \mathbf y) - \hat a_{\mathbf q}^{\dagger (\lambda')} \exp(- i \mathbf q \cdot \mathbf y) \Big)] \\ & = \sum_{\lambda=0}^{3} \sum_{\lambda'=0}^{3} \int \frac{d^3 p ~ d^3 q}{(2\pi)^6} \frac{i}{2} \sqrt{\frac{|\mathbf q|}{|\mathbf p|}} \epsilon_\mu^{(\lambda)} \epsilon^{\nu(\lambda')} \Big ( \underbrace{[\hat a_{\mathbf p}^{(\lambda)} , \hat a_{\mathbf q}^{(\lambda')}]}_0 \exp(i (\mathbf p \cdot \mathbf x + \mathbf q \cdot \mathbf y)) \\ & \quad - \underbrace{[\hat a_{\mathbf p}^{(\lambda)} , \hat a_{\mathbf q}^{\dagger (\lambda')}]}_{- (2\pi)^3 \eta^{\lambda \lambda'} \delta^3 (\mathbf p - \mathbf q)} \exp(i (\mathbf p \cdot \mathbf x - \mathbf q \cdot \mathbf y)) \\ & \quad + \underbrace{[\hat a_{\mathbf p}^{\dagger (\lambda)} , \hat a_{\mathbf q}^{(\lambda')}]}_{(2\pi)^3 \eta^{\lambda \lambda'} \delta^3 (\mathbf p - \mathbf q)} \exp(i (- \mathbf p \cdot \mathbf x + \mathbf q \cdot \mathbf y)) \\ & \quad - \underbrace{[\hat a_{\mathbf p}^{\dagger(\lambda)} , \hat a_{\mathbf q}^{\dagger(\lambda')}]}_0 \exp(i (- \mathbf p \cdot \mathbf x - \mathbf q \cdot \mathbf y)) \Big)
        \end{aligned}
        \end{equation*}
        \begin{equation*}
        \begin{aligned}
            & = \sum_{\lambda=0}^{3} \int \frac{d^3 p}{(2\pi)^3} \frac{i}{2} \underbrace{\epsilon_\mu^{(\lambda)} \epsilon^{\nu(\lambda)}}_{\eta^\nu_{\phantom \nu \mu}} \Big ( \underbrace{\exp(i \mathbf p \cdot (\mathbf x - \mathbf y))}_{\delta^3 (\mathbf x - \mathbf y)} + \underbrace{\exp(- i \mathbf p \cdot (\mathbf x - \mathbf y))}_{\delta^3 (\mathbf x - \mathbf y)} \Big) \\ & = i \eta^\nu_{\phantom \nu \mu} \delta^3 (\mathbf x - \mathbf y) ~.
        \end{aligned}
        \end{equation*}
    \end{proof}

\section{Quantisation with Lorenz gauge}

    Since the Lorenz gauge contains a time derivative, we go into the Heisenberg picture. However, there are different possible ways to impose the gauge fixing conditions. 
    \begin{enumerate}
        \item $\partial_\mu \hat A^\mu = 0$, 
        \item $(\partial_\mu \hat A^\mu) \ket{\psi} = 0$, 
        \item $\bra{\psi} \partial_\mu \hat A^\mu \ket{\psi} = 0$.
    \end{enumerate}

    Let us consider the first case. A contradiction arises because, on one hande we have 
    \begin{equation*}
        \hat \pi^0 = - \partial_\mu \hat A^\mu = 0 ~,
    \end{equation*}
    on the other hand, the commutation relations at fixed time
    \begin{equation*}
        [\hat A_0 (\mathbf x), \hat \pi_0 (\mathbf y)] = i \eta_{00} \delta^3 (\mathbf x - \mathbf y) \neq 0 ~.
    \end{equation*}

    Let us consider the second case. The vacuum state is not physical anymore, since positive norm physical state are such that
    \begin{equation*}
        (\partial_\mu \hat A^\mu) \ket{\psi} ~,
    \end{equation*}
    but for the vacuum state, defining
    \begin{equation*}
        \hat A_\mu (x) = \int \frac{d^3 p}{(2\pi)^3} \frac{1}{\sqrt{2 |\mathbf p|}} \sum_{\lambda=0}^{3} \epsilon_\mu^{(\lambda)} (\mathbf p) \Big ( \hat a_{\mathbf p}^{(\lambda)} (\mathbf p) \exp(i \mathbf p \cdot \mathbf x) + \hat a_{\mathbf p}^{\dagger (\lambda)} (\mathbf p) \exp(- i \mathbf p \cdot \mathbf x) \Big) = \hat A^+_\mu (x) + \hat A^-_\mu (x) ~,
    \end{equation*}
    where $\hat A^+_\mu (x)$ contains only creation operators and $\hat A^-_\mu (x)$ contains only annihilation operators. Hence 
    \begin{equation*}
        \partial^\mu \hat A_\mu \ket{0} = \partial^\mu \hat A_\mu^+ \ket{0} = \partial^\mu \hat A_\mu^- \ket{0} = i p^\mu \hat A_\mu^+ \ket{0} - i p^\mu \cancel{\hat A_\mu^- \ket{0}} \neq 0 ~,
    \end{equation*}
    which means that vacuum state does not satisfy the Lorenz gauge and it is not physical.

    Let us consider the third case. We impose the Gupta-Bleuler conditions which defines what is the phisical Fock space
    \begin{equation}\label{GB}
        \partial^\mu \hat A^+_\mu \ket{\psi} = 0 \iff \bra{\psi} \partial^\mu \hat A^-_\mu = 0 \iff \bra{\psi} \partial_\mu \hat A^\mu \ket{\psi} = 0~.
    \end{equation}
    This condition translates into the constrain that the number of timelike photons are the same number of longitudinal photons, which implies that ghost with only timelike photons cannot exist.
    \begin{proof}
        In fact 
        \begin{equation*}
            \hat A^-_\mu (x) = \int \frac{d^3 p}{(2\pi)^3} \frac{1}{\sqrt{2 |\mathbf p|}} \sum_{\lambda=0}^{3} \epsilon_\mu^{(\lambda)} (\mathbf p) \hat a_{\mathbf p}^{(\lambda)} (\mathbf p) \exp(i \mathbf p \cdot \mathbf x) ~,
        \end{equation*}
        \begin{equation*}
            \partial^\mu \hat A^-_\mu (x) = \int \frac{d^3 p}{(2\pi)^3} \frac{1}{\sqrt{2 |\mathbf p|}} \sum_{\lambda=0}^{3} p^\mu \epsilon_\mu^{(\lambda)} (\mathbf p) \hat a_{\mathbf p}^{(\lambda)} (\mathbf p) \exp(i \mathbf p \cdot \mathbf x) ~,
        \end{equation*}
        which contains the term $p^\mu \epsilon_\mu^{(\lambda)}$. Notice that it vanishes for transversal photons $\lambda = 1,2$. We choose $p^\mu = (E, 0, 0, E)$ and we obtain 
        \begin{equation*}
            0 = (\epsilon^{(0)}_\mu p^\mu \hat a^{(0)}_{\mathbf p} + \epsilon^{(3)}_\mu p^\mu \hat a^{(3)}_{\mathbf p}) \ket{\psi} = E (\underbrace{\epsilon^{(0)}_1}_1 \hat a^{(0)}_{\mathbf p} + \underbrace{\epsilon^{(3)}_3}_{-1} \hat a^{(3)}_{\mathbf p}) \ket{\psi} = E (\hat a^{(0)}_{\mathbf p} - \hat a^{(3)}_{\mathbf p}) \ket{\psi} ~,
        \end{equation*}
        hence 
        \begin{equation*}
            \hat a^{(0)}_{\mathbf p} \ket{\psi } = \hat a^{(3)}_{\mathbf p} \ket{\psi} \iff \bra{\psi} \hat a^{\dagger(0)}_{\mathbf p} = \bra{\psi} \hat a^{\dagger(3)}_{\mathbf p}  ~.
        \end{equation*}
        Finally 
        \begin{equation*}
            \bra{\psi} \hat a^{\dagger(0)}_{\mathbf p} \hat a^{(0)}_{\mathbf p} \ket{\psi} = \bra{\psi} \hat a^{\dagger(3)}_{\mathbf p} \hat a^{(3)}_{\mathbf p} \ket{\psi} ~,
        \end{equation*}
        \begin{equation*}
            \bra{\psi} \hat n^{(0)}_{\mathbf p} \ket{\psi} = \bra{\psi} \hat n^{(3)}_{\mathbf p} \ket{\psi} ~.
        \end{equation*}
    \end{proof}

\section{Fock space}

    A negative norm state with only timelike photons is unphysical, since $\ket{\mathbf q, \lambda = 0} = \hat a^{\dagger (0)}_{\mathbf q} \ket{0}$ such that 
    \begin{equation*}
        (\hat a^{(0)}_{\mathbf p} - \hat a^{(3)}_{\mathbf p}) \ket{\mathbf q, \lambda = 0} = \underbrace{\hat a^{(0)}_{\mathbf p} \hat a^{\dagger (0)}_{\mathbf q}}_{- (2\pi)^3 \delta^3 (\mathbf p - \mathbf q)} \ket{0} - \cancel{\hat a^{(3)}_{\mathbf p} \hat a^{\dagger (0)}_{\mathbf q}} \ket{0} = - (2\pi)^3 \delta^3 (\mathbf p - \mathbf q) \ket{0} \neq 0 ~.
    \end{equation*}

    We make a change of basis in the Fock space, from 
    \begin{equation*}
        \hat a^{\dagger(0)}_{\mathbf p} ~, \quad \hat a^{\dagger(1)}_{\mathbf p} ~, \quad \hat a^{\dagger(2)}_{\mathbf p} ~, \quad \hat a^{\dagger(3)}_{\mathbf p} ~,
    \end{equation*}
    to 
    \begin{equation*}
        \hat a^{\dagger(1)}_{\mathbf p} ~, \quad \hat a^{\dagger(2)}_{\mathbf p} ~, \quad \hat b_{\pm, \mathbf p} = \hat a^{\dagger(0)}_{\mathbf p} \pm \hat a^{\dagger(3)}_{\mathbf p} ~.
    \end{equation*}
    This implies that 
    \begin{equation*}
        \hat a^{\dagger(1)}_{\mathbf p} \ket{0} = \ket{\mathbf p, \lambda = 1} ~, \quad \hat a^{\dagger(2)}_{\mathbf p} \ket{0} = \ket{\mathbf p, \lambda = 2} ~, \quad \hat b_{\pm, \mathbf p} \ket{0} = \ket{\mathbf p, \lambda = 0} \pm \ket{\mathbf p, \lambda = 3} ~.
    \end{equation*}

    Furthermore, the commutation relations becomes 
    \begin{equation*}
        [\hat b_{-, \mathbf p}, \hat b_{-, \mathbf q}^\dagger] = 0 ~, \quad [\hat b_{-, \mathbf p}, \hat b_{+, \mathbf q}^\dagger] = - (2\pi)^3 \delta^3 (\mathbf p - \mathbf q) ~.
    \end{equation*}
    \begin{proof}
        For the first 
        \begin{equation*}
        \begin{aligned}
            [\hat b_{-, \mathbf p}, \hat b_{-, \mathbf q}^\dagger] & = [\hat a^{(0)}_{\mathbf p} - \hat a^{(3)}_{\mathbf p}, \hat a^{\dagger(0)}_{\mathbf q} - \hat a^{\dagger(3)}_{\mathbf q}] \\ & = \underbrace{[\hat a^{(0)}_{\mathbf p} , \hat a^{\dagger(0)}_{\mathbf q}]}_{-(2\pi)^3 \delta^3 (\mathbf p - \mathbf q)} - \underbrace{[\hat a^{(0)}_{\mathbf p} , \hat a^{\dagger(3)}_{\mathbf q}] }_0 - \underbrace{[\hat a^{(3)}_{\mathbf p}, \hat a^{\dagger(0)}_{\mathbf q}]}_0 + \underbrace{[\hat a^{(3)}_{\mathbf p}, \hat a^{\dagger(3)}_{\mathbf q}]}_{(2\pi)^3 \delta^3 (\mathbf p - \mathbf q)} = 0 ~.
        \end{aligned}
        \end{equation*}
    \end{proof}

    The condition for physical states $\ket{\psi}$ is that $\hat b_{\pm, \mathbf p} \ket{\psi} = 0$. Hence transverse photons are physical, timelike and longitudinal photons are unphysical, the combination with plus of timelike and longitudinal photons are unphysical,the combination with minus of timelike and longitudinal photons are physical. However, the latter has zero-norm. To summarise, Fock space contains all states such that it is satisfied~\eqref{GB}, which brings to positive norm states (transversal photons) and zero norm states ($\ket{S} - \ket{L}$ photons).
    \begin{proof}
        For the transverse photons $\ket{T}$
        \begin{equation*}
            \hat b_{-, \mathbf p} \ket{\mathbf q, \lambda= 1,2} = \hat b_{-, \mathbf p} \hat a_{\mathbf q}^{\dagger (1,2)} \ket{0} = 0 ~.
        \end{equation*}

        For the longitudinal photons $\ket{L}$
        \begin{equation*}
        \begin{aligned}
            \hat b_{-, \mathbf p} \ket{\mathbf q, \lambda=3} & = \hat b_{-, \mathbf p} \hat a_{\mathbf q}^{\dagger (3)} \ket{0} \\ & = \hat b_{-, \mathbf p} \frac{\hat b^\dagger_{+, \mathbf p} - \hat b^\dagger_{-, \mathbf p}}{2} \ket{0} \\ &  = \frac{1}{2} \underbrace{\hat b_{-, \mathbf p} \hat b^\dagger_{+, \mathbf p}}_{[\hat b_{-, \mathbf p} ,\hat b^\dagger_{+, \mathbf p}] + \hat b^\dagger_{+, \mathbf p} \hat b_{-, \mathbf p} } \ket{0} - \frac{1}{2} \underbrace{\hat b_{-, \mathbf p} \hat b^\dagger_{-, \mathbf p}}_{[\hat b_{-, \mathbf p}, \hat b^\dagger_{-, \mathbf p}] + \hat b^\dagger_{-, \mathbf p} \hat b_{-, \mathbf p}} \ket{0} \\ & = \frac{1}{2} \underbrace{[\hat b_{-, \mathbf p} ,\hat b^\dagger_{+, \mathbf p}]}_{- (2\pi)^3 \delta^3 (\mathbf p - \mathbf q)} \ket{0} + \frac{1}{2} \hat b^\dagger_{+, \mathbf p} \underbrace{\hat b_{-, \mathbf p} \ket{0}}_0 - \frac{1}{2} \underbrace{[\hat b_{-, \mathbf p}, \hat b^\dagger_{-, \mathbf p}]}_{0} \ket{0} - \frac{1}{2} \hat b^\dagger_{-, \mathbf p} \underbrace{\hat b_{-, \mathbf p} \ket{0}}_0 \\ & = - (2\pi)^3 \delta^3 (\mathbf p - \mathbf q) \ket{0} \neq 0 ~,
        \end{aligned}
        \end{equation*}
        where we have used 
        \begin{equation*}
            \hat a^{\dagger (3)}_{\mathbf p} = \frac{\hat b^\dagger_{+, \mathbf p} - \hat b^\dagger_{-, \mathbf p}}{2} ~.
        \end{equation*}

        For the timelike photons $\ket{S}$
        \begin{equation*}
        \begin{aligned}
            \hat b_{-, \mathbf p} \ket{\mathbf q, \lambda=0} & = \hat b_{-, \mathbf p} \hat a_{\mathbf q}^{\dagger (0)} \ket{0} \\ & = \hat b_{-, \mathbf p} \frac{\hat b^\dagger_{+, \mathbf p} + \hat b^\dagger_{-, \mathbf p}}{2} \ket{0} \\ &  = \frac{1}{2} \underbrace{\hat b_{-, \mathbf p} \hat b^\dagger_{+, \mathbf p}}_{[\hat b_{-, \mathbf p} ,\hat b^\dagger_{+, \mathbf p}] + \hat b^\dagger_{+, \mathbf p} \hat b_{-, \mathbf p} } \ket{0} + \frac{1}{2} \underbrace{\hat b_{-, \mathbf p} \hat b^\dagger_{-, \mathbf p}}_{[\hat b_{-, \mathbf p}, \hat b^\dagger_{-, \mathbf p}] + \hat b^\dagger_{-, \mathbf p} \hat b_{-, \mathbf p}} \ket{0} \\ & = \frac{1}{2} \underbrace{[\hat b_{-, \mathbf p} ,\hat b^\dagger_{+, \mathbf p}]}_{- (2\pi)^3 \delta^3 (\mathbf p - \mathbf q)} \ket{0} + \frac{1}{2} \hat b^\dagger_{+, \mathbf p} \underbrace{\hat b_{-, \mathbf p} \ket{0}}_0 + \frac{1}{2} \underbrace{[\hat b_{-, \mathbf p}, \hat b^\dagger_{-, \mathbf p}]}_{0} \ket{0} + \frac{1}{2} \hat b^\dagger_{-, \mathbf p} \underbrace{\hat b_{-, \mathbf p} \ket{0}}_0 \\ & = - (2\pi)^3 \delta^3 (\mathbf p - \mathbf q) \ket{0} \neq 0 ~,
        \end{aligned}
        \end{equation*}
        where we have used 
        \begin{equation*}
            \hat a^{\dagger (0)}_{\mathbf p} = \frac{\hat b^\dagger_{+, \mathbf p} + \hat b^\dagger_{-, \mathbf p}}{2} ~.
        \end{equation*}

        For the $\ket{S} + \ket{L}$ photons
        \begin{equation*}
        \begin{aligned}
            \hat b_{-, \mathbf p} \ket{\mathbf q, S + L} & = \underbrace{\hat b_{-, \mathbf p} \hat b_{+,\mathbf q}^{\dagger}}_{[\hat b_{-, \mathbf p}, \hat b_{+,\mathbf q}^{\dagger}] + \hat b_{+,\mathbf q}^{\dagger} \hat b_{-, \mathbf p} } \ket{0} \\ & = \underbrace{[\hat b_{-, \mathbf p}, \hat b_{+,\mathbf q}^{\dagger}]}_{- (2\pi)^3 \delta^3 (\mathbf p -\mathbf q)} \ket{0} + \hat b_{+,\mathbf q}^{\dagger} \underbrace{\hat b_{-, \mathbf p} \ket{0}}_0 \\ & = - (2\pi)^3 \delta^3 (\mathbf p -\mathbf q) \ket{0} \neq 0 ~.
        \end{aligned}
        \end{equation*}

        For the $\ket{S} - \ket{L}$ photons
        \begin{equation*}
        \begin{aligned}
            \hat b_{-, \mathbf p} \ket{\mathbf q, S - L} & = \underbrace{\hat b_{-, \mathbf p} \hat b_{-,\mathbf q}^{\dagger}}_{[\hat b_{-, \mathbf p}, \hat b_{-,\mathbf q}^{\dagger}] + \hat b_{-,\mathbf q}^{\dagger} \hat b_{-, \mathbf p} } \ket{0} \\ & = \underbrace{[\hat b_{-, \mathbf p}, \hat b_{-,\mathbf q}^{\dagger}]}_{0} \ket{0} + \hat b_{-,\mathbf q}^{\dagger} \underbrace{\hat b_{-, \mathbf p} \ket{0}}_0 = 0 ~.
        \end{aligned}
        \end{equation*}
    \end{proof}

    Notice that the photons $\ket{S} - \ket{L}$ or $\ket{S} + \ket{L}$ have zero norm. 
    \begin{proof}
        In fact, given
        \begin{equation*}
            \hat b_{-, \mathbf p}^\dagger \ket{0} = \ket{\mathbf p, S - L} ~,
        \end{equation*}
        we have 
        \begin{equation*}
        \begin{aligned}
            \braket{\mathbf p, S-L}{\mathbf p, S-L} & = \bra{0} \hat b_{-, \mathbf p} \hat b_{-, \mathbf p}^\dagger \ket{0} \\ & = \bra{0} \underbrace{\hat b_{-, \mathbf p} \hat b_{-, \mathbf p}^\dagger}_{[\hat b_{-, \mathbf p} , \hat b_{-, \mathbf p}^\dagger] + \hat b_{-, \mathbf p}^\dagger \hat b_{-, \mathbf p}} \ket{0} \\ & = \bra{0} \underbrace{[\hat b_{-, \mathbf p} , \hat b_{-, \mathbf p}^\dagger]}_0 \ket{0} + \bra{0} \hat b_{-, \mathbf p}^\dagger \underbrace{\hat b_{-, \mathbf p} \ket{0}}_0 = 0 ~.
        \end{aligned}
        \end{equation*}
        Simirly, given
        \begin{equation*}
            \hat b_{+, \mathbf p}^\dagger \ket{0} = \ket{\mathbf p, S + L} ~,
        \end{equation*}
        we have 
        \begin{equation*}
        \begin{aligned}
            \braket{\mathbf p, S+L}{\mathbf p, S+L} & = \bra{0} \hat b_{+, \mathbf p} \hat b_{+, \mathbf p}^\dagger \ket{0} \\ & = \bra{0} \underbrace{\hat b_{+, \mathbf p} \hat b_{+, \mathbf p}^\dagger}_{[\hat b_{+, \mathbf p} , \hat b_{+, \mathbf p}^\dagger] + \hat b_{+, \mathbf p}^\dagger \hat b_{+, \mathbf p}} \ket{0} \\ & = \bra{0} \underbrace{[\hat b_{+, \mathbf p} , \hat b_{+, \mathbf p}^\dagger]}_0 \ket{0} + \bra{0} \hat b_{+, \mathbf p}^\dagger \underbrace{\hat b_{+, \mathbf p} \ket{0}}_0 = 0 ~.
        \end{aligned}
        \end{equation*}
    \end{proof}

    \begin{example}
        Consider a state in which $2$ photons have polarisation $\ket{T}$ and $\ket{S} - \ket{L}$. It has zero norm. In fact, given
        \begin{equation*}
            \hat b_{-, \mathbf p}^\dagger \hat a_{\mathbf q}^{\dagger (1,2)} \ket{0} = \ket{\mathbf q, T; \mathbf p, S-L} ~,
        \end{equation*}
        we have 
        \begin{equation*}
        \begin{aligned}
            \braket{\mathbf q, T; \mathbf p, S-L}{\mathbf q, T; \mathbf p, S-L} & = \bra{0} \hat a_{\mathbf q}^{(1,2)} \hat b_{-, \mathbf p} \hat b_{-, \mathbf p}^\dagger \hat a_{\mathbf q}^{\dagger (1,2)} \ket{0} \\ & = \bra{0} \hat b_{-, \mathbf p} \hat b_{-, \mathbf p}^\dagger \underbrace{\hat a_{\mathbf q}^{\dagger (1,2)} \hat a_{\mathbf q}^{(1,2)}}_{[\hat a_{\mathbf q}^{\dagger (1,2)} , \hat a_{\mathbf q}^{(1,2)}] + \hat a_{\mathbf q}^{(1,2)} \hat a_{\mathbf q}^{\dagger (1,2)}} \ket{0} \\ & = \bra{0} \hat b_{-, \mathbf p} \hat b_{-, \mathbf p}^\dagger \underbrace{[\hat a_{\mathbf q}^{\dagger (1,2)} , \hat a_{\mathbf q}^{(1,2)}] }_{(2\pi)^3 \delta^3 (0) } \ket{0} + \bra{0} \hat b_{-, \mathbf p} \hat b_{-, \mathbf p}^\dagger \hat a_{\mathbf q}^{(1,2)} \underbrace{\hat a_{\mathbf q}^{\dagger (1,2)} \ket{0}}_0 \\ & = (2\pi)^3 \delta^3 (0) \bra{0} \underbrace{\hat b_{-, \mathbf p} \hat b_{-, \mathbf p}^\dagger}_{[\hat b_{-, \mathbf p} , \hat b_{-, \mathbf p}^\dagger] + \hat b_{-, \mathbf p}^\dagger \hat b_{-, \mathbf p}} \ket{0} \\ & = (2\pi)^3 \delta^3 (0) \bra{0} \underbrace{[\hat b_{-, \mathbf p} , \hat b_{-, \mathbf p}^\dagger]}_0 \ket{0} + (2\pi)^3 \delta^3 (0) \bra{0} \hat b_{-, \mathbf p}^\dagger \underbrace{\hat b_{-, \mathbf p} \ket{0}}_0 = 0 ~.
        \end{aligned}
        \end{equation*}
    \end{example}
    