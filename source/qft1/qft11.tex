\part{Classical field theory}

\chapter{Action}

    A field is a physical quantity $\phi(t, \mathbf x)$ which is defined ad every point in spacetime. The dynamics of a field is governed by an action, which is a functional that associates a real number to each field configuration for a fixed time interval $[t_1, t_2]$
    \begin{equation}\label{action}
        S[\phi_i(x), \partial_\mu \phi_i(x)] = \int_{t_1}^{t_2} dt ~ L = \int_{t_1}^{t_2} dt ~ \int d^3x ~ \mathcal L = \int d^4 x \mathcal L (\phi_i, \partial_\mu \phi_i) ~,
    \end{equation}
    where $\mathcal L$ is the lagrangian density, defined by 
    \begin{equation*}
        L = \int d^3x ~ \mathcal L ~.
    \end{equation*}

    In natural units, the dimensional analysis is 
    \begin{equation*}
        [S] = 0 ~ [d^4 x]=-4 ~ [\mathcal L] = 4 ~.
    \end{equation*}

\section{The principle of stationary action}

    The dynamics of the system can be determined by the principle of stationary action. 

    \begin{princ}
        The system evolve from an initial configuration at time $t_1$ to a final configuration at time $t_2$ along a path in configuration space which extremises the action~\eqref{action}, i.e.
        \begin{equation}\label{statact}
            \delta S = 0 ~.
        \end{equation}
        with the additional conditions 
        \begin{enumerate}
            \item fields vanish at spatial infinity
                \begin{equation*}
                    \phi_i(t, \mathbf x) \rightarrow 0 \quad |\mathbf x| \rightarrow \infty ~,
                \end{equation*}
                hence
                \begin{equation}\label{space}
                    \delta \phi_i (t, \infty) = 0 ~,
                \end{equation}
            \item fields vanish at time extremes
                \begin{equation}\label{time}
                    \delta \phi_i (t_1, \mathbf x) = \delta \phi_i (t_2, \mathbf x) = 0 ~.
                \end{equation}
        \end{enumerate}
    \end{princ}

    The equation of motion of the system are the Euler-Lagrange equations
    \begin{equation}\label{eleq}
        \pdv{\mathcal L}{\phi_i} - \partial_\mu \pdv{\mathcal L}{\partial_\mu \phi_i} = 0 ~.
    \end{equation}

    \begin{proof}
        The variation of the action is 
        \begin{equation*}
            \delta S = \int d^4 x ~ \Big ( \pdv{\mathcal L}{\phi_i} \delta \phi_i + \pdv{\mathcal L}{\partial_\mu \phi_i} \delta \partial_\mu \phi_i \Big) = \int d^4 x ~ \Big ( \pdv{\mathcal L}{\phi_i} \delta \phi_i + \pdv{\mathcal L}{\partial_\mu \phi_i} \partial_\mu \delta \phi_i \Big) ~,
        \end{equation*}
        where
        \begin{equation*}
            \delta \phi_i = {\phi'}_i(x) - \phi_i(x) ~,
        \end{equation*} 
        and 
        \begin{equation*}
            \delta \partial_\mu \phi_i(x) = \partial_\mu {\phi'}_i - \partial_\mu \phi(x) = \partial_\mu ({\phi'}_i(x) - \phi_i(x)) = \partial_\mu \delta \phi(x) ~.
        \end{equation*}

        By integration by parts, we obtain
        \begin{equation*}
            \delta S = \int d^4 x ~ \Big ( \pdv{\mathcal L}{\phi_i} \delta \phi_i + \pdv{\mathcal L}{\partial_\mu \phi_i} \partial_\mu \delta \phi_i \Big) = \int d^4 x ~ \Big( \pdv{\mathcal L}{\phi_i} \delta \phi_i - \partial_\mu \pdv{\mathcal L}{\partial_\mu \phi_i} \delta \phi_i \Big) + \int d^4x \partial_\mu \Big( \pdv{\mathcal L }{\partial_\mu \phi_i} \delta \phi_i \Big) ~.
        \end{equation*}

        Notice that the last term is a total derivative and it vanishes at the boundary by the condition~\eqref{time} and~\eqref{space}
        \begin{equation*}
            \int d^4x \partial_\mu \Big( \pdv{\mathcal L }{\partial_\mu \phi_i} \delta \phi_i \Big) = 0 ~.
        \end{equation*}

        Hence, we find 
        \begin{equation*}
            \delta S = \int d^4 x ~ \Big( \pdv{\mathcal L}{\phi_i} \delta \phi_i - \partial_\mu \pdv{\mathcal L}{\partial_\mu \phi_i} \delta \phi_i \Big) ~,
        \end{equation*}
        and, by the principle of stationary action~\eqref{statact} 
        \begin{equation*}
            \int d^4 x ~ \Big( \pdv{\mathcal L}{\phi_i} \delta \phi_i - \partial_\mu \pdv{\mathcal L}{\partial_\mu \phi_i} \delta \phi_i \Big) = 0 ~.
        \end{equation*}

        Finally, since $\delta_i \phi$ is arbitrary, we obtain~\eqref{eleq}
        \begin{equation*}
            \pdv{\mathcal L}{\phi_i} - \partial_\mu \pdv{\mathcal L}{\partial_\mu \phi_i} = 0 ~.
        \end{equation*}
    \end{proof}

    In order to quantise the theory, we need the hamiltonian formalism.

    \begin{definition}
        The conjugate field $\phi^i(x)$ associated to the field $\phi_i$ is 
        \begin{equation*}
            \phi^i(x) = \pdv{\mathcal L}{\dot \phi_i(x)}
        \end{equation*}
    \end{definition}

    The hamiltonian density is given by the Legendre transformation 
    \begin{equation*}
        \mathcal H = \Phi^i \dot \phi_i - \mathcal L
    \end{equation*}
    where the hamiltonian is 
    \begin{equation*}
        H = \int d^3 x ~ \mathcal H
    \end{equation*}

\chapter{Noether's theorem}

    Symmetries are fundamental in quantum field theory and they can be classified into 
    \begin{enumerate}
        \item spacetime
        \begin{enumerate}
            \item global
            \begin{enumerate}
                \item continuous (Poincarè)
                \item discrete (Parity, time reversal)
            \end{enumerate}
            \item local
            \begin{enumerate}
                \item continuous (General relativity)
                \item discrete (Parity coordinate dependent)
            \end{enumerate}
        \end{enumerate}
        \item internal
        \begin{enumerate}
            \item global
            \begin{enumerate}
                \item continuous (Flavour)
                \item discrete ($\mathbb Z_2$)
            \end{enumerate}
            \item local
            \begin{enumerate}
                \item continuous ($SU(3) \times SU(2) \times U(1)$)
                \item discrete ($\mathbb Z_2 (x)$)
            \end{enumerate}
        \end{enumerate}
    \end{enumerate}

    Through the Noether's theorem, we can associate conserved quantities to continuous symmetries.

    \begin{theorem}[Noether's]
        Every continuous symmetry $\delta \phi_i$ of the action~\eqref{action} give rise to a conserved current 
        \begin{equation}\label{conscurr}
            J^\mu = \pdv{\mathcal L}{\partial_\mu \phi_i} \delta \phi - K^\mu
        \end{equation}
        such that it satisfies a continuity equation 
        \begin{equation}\label{cont}
            \partial_\mu J^\mu = 0
        \end{equation}
    \end{theorem}

    \begin{proof}
        We consider an infinitesimal transformation for a continuous symmetry of the system
        \begin{equation*}
            {\phi'}_i = \phi_i + \delta \phi_i
        \end{equation*}
        which induces a transformation of the lagrangian 
        \begin{equation*}
            \mathcal L' = \mathcal L + \delta \mathcal L
        \end{equation*}
        
        In order to be a symmetry of the system, we require that the action is not invariant, but we allow to be up to a boundary term $K^\mu(\phi_i)$, because the dynamics of the system, i.e.~the equations of tmotion, do not change with a boundary term. Hence 
        \begin{equation*}
            S' = S + \int \partial_\mu K^\mu(\phi_i)
        \end{equation*}
        but 
        \begin{equation}\label{symm}
            \delta S = \int \partial_\mu K^\mu(\phi_i)
        \end{equation}

        Explicitly, we obtain 
        \begin{equation*}
        \begin{aligned}
            \delta S & = \delta \int d^4 x ~ \mathcal L \\ & = \int d^4 x ~ \Big ( \pdv{\mathcal L}{\phi_i} \delta \phi_i + \pdv{\mathcal L}{\partial_\mu \phi_i} \delta \partial_\mu \phi_i\Big) \\ & =  \int d^4 x ~ \Big ( \pdv{\mathcal L}{\phi_i} \delta \phi_i + \pdv{\mathcal L}{\partial_\mu \phi_i} \partial_\mu \delta \phi_i \Big) \\ & =  \int d^4 x ~ \Big ( \pdv{\mathcal L}{\phi_i} \delta \phi_i - \partial_\mu \pdv{\mathcal L}{\partial_\mu \phi_i} \delta \phi_i \Big) + \int d^4 x ~ \partial_\mu \Big ( \pdv{\mathcal L}{\phi} \delta \phi_i \Big)\\ & =  \int d^4 x ~ \delta \phi_i \Big ( \underbrace{\pdv{\mathcal L}{\phi_i} - \partial_\mu \pdv{\mathcal L}{\partial_\mu \phi_i}}_0 \Big) + \int d^4 x ~ \partial_\mu \Big ( \pdv{\mathcal L}{\phi} \delta \phi_i \Big) \\ & = \int d^4 x ~ \partial_\mu \Big ( \pdv{\mathcal L}{\phi} \delta \phi_i \Big)
        \end{aligned}
        \end{equation*}
        where we used the fact that partial derivatives and symmetries commute, the equation of motions~\eqref{eleq} and we integrated by parts. Hence, by requiring that it is a symmetry
        \begin{equation*}
            \delta S = \int d^4 x ~ \partial_\mu \Big ( \pdv{\mathcal L}{\phi} \delta \phi_i \Big) = \int d^4 x \partial_\mu K^\mu
        \end{equation*}
        or equivalently 
        \begin{equation*}
            \int d^4 x ~ \partial_\mu \Big ( \pdv{\mathcal L}{\phi} \delta \phi_i - K^\mu \Big) = 0
        \end{equation*}
        Since it is for arbitrary integration, the integrand vanishes and 
        \begin{equation*}
            \partial_\mu J^\mu = 0
        \end{equation*}
        with 
        \begin{equation*}
            J^\mu = \pdv{\mathcal L}{\partial_\mu \phi_i} \delta \phi_i - K^\mu
        \end{equation*}
    \end{proof}

    Notice that every conserved current can be related to a conserved quantity $Q$ by 
    \begin{equation*}
        Q = \int_{\mathbb R^3} d^3 x ~J^0
    \end{equation*}
    This means that $Q$ is conserved locally, i.e.~any charge carrier leaving a finite volume $V$ is associated to a flow of current $\mathbf J$ out of the volume.

    \begin{proof}
        Infact, by using~\eqref{cont}
        \begin{equation*}
        \begin{aligned}
            \dv{Q}{t} & = \dv{}{t} \int_{\mathbb R^3} d^3 x ~ J^0 \\ & = \int_{\mathbb R^3} d^3 x ~ \pdv{J^0}{t} \\ & = - \int_{\mathbb R^3} d^3 x ~ \nabla \cdot \mathbf J = 0 = -  \int_{\partial \mathbb R^3} d \mathbf S \cdot \mathbf J = 0
        \end{aligned}
        \end{equation*}
        where we used the Stoke's theorem and the fact that $\mathbf J \rightarrow 0$ for $|\mathbf x| \rightarrow 0$.
    \end{proof}

\chapter{Energy-momentum tensor}

    Spacetime translations give rise to $4$ conserved currents, which corresponds to the conservation of energy and momentum. Infact, we consider an infinitesimal spacetime translation 
    \begin{equation*}
        {x'}^\mu = x^\mu - \epsilon^\mu
    \end{equation*}
    such that fields change by 
    \begin{equation*}
        {\phi'}_i = \phi_i (x + \epsilon) = \phi(x) + \epsilon^\mu \partial_\mu \phi_i (x)
    \end{equation*}
    We considered an active transformation, where there is not a change of frame but fields themselves are indeed translated into new fields such that 
    \begin{equation*}
        {\phi'}_i (x') = \phi(x) = \phi(x' + \epsilon)
    \end{equation*}

    A passive transformation would have acted as 
    \begin{equation*}
        {\phi'}_i = \phi_i (x - \epsilon) 
    \end{equation*}

    Since the lagrangian is a function of the coordinates via fields, we have the following transformation
    \begin{equation*}
        \delta \mathcal L = \mathcal L' - \mathcal L = \epsilon^\mu \partial_\mu \mathcal L = \epsilon^\mu \partial_\nu(\delta^\nu_{\phantom \nu \mu} \mathcal L)
    \end{equation*}
    Hence, the boundary term is 
    \begin{equation*}
        K^\mu = \delta^\mu_{\phantom \mu \nu} \mathcal L
    \end{equation*}

    We apply the Noether's theorem~\eqref{conscurr} and find $4$ different conserved currents labelled by $\nu$
    \begin{equation*}
        {(J^\mu)}_\nu = \pdv{\mathcal L}{\partial_\mu \phi_i} \partial_\nu \phi_i - \delta^\mu_{\phantom \mu \nu} \mathcal L
    \end{equation*}
    and we define the energy-momentum tensor, or stress-energy tensor, 
    \begin{equation*}\label{emten}
        T^\mu_{\phantom \mu \nu} = {(J^\mu)}_\nu
    \end{equation*}
    such that 
    \begin{equation*}
        \partial_\mu T^\mu_{\phantom \mu \nu} = 0
    \end{equation*}

    In natural units, the dimensional analysis is 
    \begin{equation*}
        T^\mu_{\phantom \mu \nu} = [\mathcal L] = 4
    \end{equation*}

    The $4$ conserved charges are 
    \begin{equation*}
        Q_\nu = \int_{\mathbb R^3} d^3 x ~ (J^0)_\nu = \int_{\mathbb R^3} d^3 x ~ T^0_{\phantom 0 \nu} 
    \end{equation*}
    which correspond to the $4$-momentum
    \begin{equation*}
        P^\mu = \int_{\mathbb R^3} d^3 x ~ T^{0\mu}
    \end{equation*}

    In particular, the $0$-th component is the energy 
    \begin{equation}\label{energ}
    \begin{aligned}
        P^0 & = \int d^3 x ~ T^{00} \\ & = \int d^3 x ~ \Big ( \pdv{\mathcal L}{\partial_0 \phi_i} \partial^0 \phi_i - \delta^{0 0} \mathcal L \Big) \\ & = \int d^3 x ~ \Big ( \underbrace{\pdv{\mathcal L}{\dot \phi_i}}_{\pi^i} \dot \phi_i - \mathcal L \Big ) = \int d^3 x ~ (\underbrace{\pi^i \dot \phi_i - \mathcal L}_{\mathcal L}) = \int d^3 x ~ \mathcal H = H
    \end{aligned}
    \end{equation}
    such that 
    \begin{equation*}
        \dv{H}{t} = 0
    \end{equation*} 
    and the $j$-th components are the momentum 
    \begin{equation}\label{momen}
    \begin{aligned}
        P^j & = \int d^3 x ~ T^{0j} \\ & = \int d^3 x ~ \Big ( \pdv{\mathcal L}{\partial_0 \phi_i} \underbrace{\partial^j \phi_i}_{-\partial_j \phi_i} - \underbrace{\delta^{0j}}_0 \mathcal L \Big) \\ & = \int d^3 x ~ \Big ( - \underbrace{\pdv{\mathcal L}{\dot \phi_i}}_{\pi^i} \partial_j \phi_i \Big) \\ & = - \int d^3 x ~ \pi^i \partial_j \phi_i
    \end{aligned}
    \end{equation}
    such that 
    \begin{equation*}
        \dv{P^i}{t} = 0
    \end{equation*}

\chapter{An example: electrodynamics}

    Maxwell's equations 
    \begin{equation}\label{eqmax}
        \nabla \cdot \mathbf B = 0 \qquad \nabla \times \mathbf E + \pdv{\mathbf B}{t} = 0 \qquad \nabla \cdot E = \rho \qquad \nabla \times \mathbf B - \pdv{\mathbf E}{t} = \mathbf J
    \end{equation}
    can be written in covariant form 
    \begin{equation*}
        \partial_\mu F^{\mu\nu} = J^\nu \quad \partial_\mu {F^*}^{\mu\nu} = 0
    \end{equation*}
    where $F^{\mu\nu}$ is the electromagnetic tensor and ${F^*}^{\mu\nu} = \frac{1}{2} \epsilon^{\mu\nu\sigma\rho} F_{\sigma\rho}$ is its dual. 

    Furthermore, they can be written in terms of the scalar $\phi$ and the vector potentials $\mathbf A$, defined by
    \begin{equation*}
        \mathbf E = - \nabla \phi - \pdv{\mathbf A}{t} \quad \mathbf B = \nabla \times \mathbf A
    \end{equation*}
    Maxwell's equations do not change under this transformation. 

    \begin{proof}
        Maybe in the future.
    \end{proof}

    In covariant form, we can write the electromagnetic tensor as 
    \begin{equation*}
        F^{\mu\nu} = \partial^\mu A^\nu - \partial^\nu A^\mu
    \end{equation*}

    Maxwell's equations can be seen as the equations of motion of the electromagnetic lagrangian 
    \begin{equation*}
        \mathcal L = - \frac{1}{4} F^{\mu\nu} F_{\mu\nu} - J_\mu A^\mu
    \end{equation*}
    or, equivalenty written in terms of the $4$-potential, 
    \begin{equation*}
        \mathcal L = - \frac{1}{2} \partial_\mu A_\nu \partial^\mu A^\nu + \frac{1}{2} (\partial_\mu A^\mu) ^2 - A_\mu J^\mu
    \end{equation*}

    \begin{proof}
        First, we prove that they are equivalent.
        Maybe in the future. 

        Second, we prove that it leads to the Maxwell's equations.
        Maybe in the future.
    \end{proof}

    In natural units, the dimensional analysis is 
    \begin{equation*}
        [F^{\mu\nu}] = 2 \quad [A_\mu] = 1 \quad [J^\mu] = 3
    \end{equation*}

    The minus sign garanties that the kinetic energy has a positive one 
    \begin{equation*}
        - \frac{1}{2} \partial_0 A_i \underbrace{\partial^0}_{\partial_0} \underbrace{A^i}_{-A_i} = \frac{1}{2} \dot A_i^2
    \end{equation*}

    The fourth field $A_0$ is not a dynamical quantity, since there is no kinetic energy in terms of $\dot A_0^2$, because the first $- \frac{1}{2} \partial_0 A_0 \partial^0 A^0$ cancels out with $\frac{1}{2} (\partial_0 A_0)^2$. Therefore, there are only $3$ degrees of freedom. However, since electrodynamics is a gauge theory, it is possible to restrict to only $2$ degrees of freedom, which correspond to the $2$ transversal polarisations direction of an electromagnetic wave.

    The energy-momentum tensor is 
    \begin{equation}\label{emtem}
        T^{\mu\nu} = \partial^\nu A^\mu \partial_\rho A^\rho - \partial^\mu A^\rho \partial^\nu A_\rho + \frac{1}{4} \eta^{\mu\nu} F_{\rho \sigma} F^{\rho\sigma}
    \end{equation}

    \begin{proof}
        Maybe in the future.
    \end{proof}

    However, the first term in~\eqref{emtem} is not symmetric under change $\mu \leftrightarrow \nu$, but in order to take into account general relativity, this tensor must be symmetric, since $R_{\mu\nu}$ and $g_{\mu\nu}$ are so in
    \begin{equation*}
        R_{\mu\nu} - \frac{1}{2} R g_{\mu\nu} + \Lambda g_{\mu\nu} = \frac{8 \pi G}{c^4} T_{\mu\nu}
    \end{equation*}

    To do it, we defined a new energy-momentum tensor starting from the old one with the addition of an extra term: the partial derivative of a $3$ indices anti-symmetric in the first $2$ indices tensor $K^{\lambda\mu\nu} = - K^{\mu\lambda\nu}$
    \begin{equation*}
        \tilde T^{\mu\nu} = T^{\mu\nu} + \partial_\lambda K^{\lambda\mu\nu}
    \end{equation*} 
    This garanties that it is conserved as well 
    \begin{equation*}
        \partial_\mu \tilde T^{\mu\nu} = \partial_\mu T^{\mu\nu} + \underbrace{\partial_\mu \partial_\lambda}_{symm} \underbrace{K^{\lambda\mu\nu}}_{anti} = \partial_\mu T^{\mu\nu} = 0
    \end{equation*}

    In the electromagnetic case, we choose $K$ to be 
    \begin{equation*}
        K^{\lambda\mu\nu} = F^{\mu\lambda} A^\nu
    \end{equation*}
    and the symmetric energy-momentum tensor becomes 
    \begin{equation*}
        \tilde T^{\mu\nu} = F^{\mu\lambda} F_{\lambda}^{\phantom \lambda \nu} + \frac{1}{4} \eta^{\mu\nu} F^{\rho\sigma} F_{\rho\sigma}
    \end{equation*}
    which is called the Belifante-Rosenfeld tensor.

    \begin{proof}
        Maybe in the future.
    \end{proof}

    The energy density is 
    \begin{equation*}
        \mathcal E = \frac{1}{2} (|\mathbf E|^2 + |\mathbf B|^2)
    \end{equation*}

    \begin{proof}
        Maybe in the future.
    \end{proof}

    The momentum density is 
    \begin{equation*}
        \mathcal P^i = (\mathbf E \times \mathbf B)^i
    \end{equation*}

    \begin{proof}
        Maybe in the future.
    \end{proof}

