\part{Dirac theory}

\chapter{Dirac action}

\section{Spinor representation of the Lorentz group}    

    The reducible representation of a Dirac spinor is 
    \begin{equation*}
        {\psi'}_D = \exp(-\frac{i}{2} \omega_{\mu\nu} \Sigma^{\mu\nu}) \psi_D ~,
    \end{equation*}
    where $\psi_D$ is a four-components complex vector, $\Sigma^{\mu\nu} = \frac{i}{4} [\gamma^\mu, \gamma^\nu]$ and the gamma matrices satisfy 
    \begin{equation*}
        \{\gamma^\mu, \gamma^\nu\} = 2 \eta^{\mu\nu} \mathbb I_4 ~.
    \end{equation*}
    In the Weyl basis, the Dirac matrices become 
    \begin{equation*}
        \gamma^0 = \begin{bmatrix}
            0 & \mathbb I_2 \\ \mathbb I_2 & 0 \\ 
        \end{bmatrix} ~.
    \end{equation*}
    It is useful to redefine the matrix 
    \begin{equation*}
        S^{\mu\nu} = - i \Sigma^{\mu\nu} = \frac{i}{4} [\gamma^\mu, \gamma^\nu] ~,
    \end{equation*}
    such that 
    \begin{equation}\label{lorspin}
        {\psi'}_D^\alpha (x) = \exp(\frac{1}{2} \omega_{\mu\nu} S^{\mu\nu})^\alpha_{\phantom \alpha \beta} \psi^\beta_D (x) = S^\alpha_\beta \psi^\beta_D (x) ~,
    \end{equation}
    where $\alpha, \beta = 1,2,3,4$.

    Notice that 
    \begin{equation}\label{gammadag}
        (\gamma^\mu)^\dagger = \gamma^0 \gamma^\mu \gamma^0 ~.
    \end{equation}
    \begin{proof}
        Maybe in the future.
    \end{proof}

\section{Invariant quantities}

    In order to have a Lorentz invariant action, we need to built Lorentz invariant quantities in function of $\psi$. Observe that the quantity $\psi \psi^\dagger$ is not a scalar. 
    \begin{proof}
        In fact, 
        \begin{equation*}
            {\psi'}^\dagger \psi' = \psi^\dagger S^\dagger S \psi \neq \psi^\dagger \psi ~,
        \end{equation*}
        since $S^\dagger \neq S^{-1}$.
    \end{proof}

    However, $S$ satisfies $S^\dagger = \gamma^0 S^{-1} \gamma^0$.
    \begin{proof}
        In fact, consider the spinor representation
        \begin{equation*}
            S = \exp ( \frac{1}{2} \omega_{\mu\nu} S^{\mu\nu}) ~,
        \end{equation*}
        its hermitian 
        \begin{equation*}
            S = \exp ( \frac{1}{2} \omega_{\mu\nu} (S^\dagger)^{\mu\nu})
        \end{equation*}
        and its inverse 
        \begin{equation*}
            S = \exp ( - \frac{1}{2} \omega_{\mu\nu} S^{\mu\nu}) ~.
        \end{equation*}
    \end{proof}

    We define the adjoint Dirac spinor 
    \begin{equation*}
        \overline \psi (x) = \psi^\dagger (x) \gamma^0 ~.
    \end{equation*}
    With this, we can construct a Lorentz invariant bilinear spinor 
    \begin{equation*}
        \overline \psi \psi ~.
    \end{equation*}
    \begin{proof}
        In fact, 
        \begin{equation*}
            \overline \psi' \psi' = {\psi'}^\dagger \gamma^0 \psi' = \psi^\dagger \underbrace{S^\dagger}_{\gamma^0 S^{-1} \gamma^0} \gamma^0 S \psi = = \psi^\dagger \gamma^0 S^{-1} \underbrace{\gamma^0 \gamma^0}_1 S \psi = \psi^\dagger \gamma^0 \underbrace{S^{-1} S}_1 \psi = \psi^\dagger \gamma^0 \psi = \overline \psi \psi ~.
        \end{equation*}
    \end{proof}

    Now, we want to build a $4$-vector $\overline \psi \gamma^\mu \psi$ such that
    \begin{equation*}
        \overline psi' \gamma^\mu \psi' = \overline \psi \Lambda^\mu_{\phantom \mu \nu} \gamma^\nu \psi ~,
    \end{equation*}
    or equivalently 
    \begin{equation*}
        S^{-1} \gamma^\mu S = \Lambda^\mu_{\phantom \mu \nu} \gamma^\nu ~.
    \end{equation*}
    \begin{proof}
        Maybe in the future.
    \end{proof}

    We obtained a Lorentz invariant scalar by contracting $\gamma^\mu$ with the first order derivative $\partial_\mu$. 

    Furthermore, $\Sigma^{\mu\nu}$ is a $2$-tensor 
    \begin{equation*}
        \overline psi' \Sigma^{\mu\nu} \psi' = \overline \psi \Lambda^\mu_{\phantom \mu \alpha} \Lambda^\nu_{\phantom \nu \beta} \Sigma^{\alpha\beta} \psi ~.
    \end{equation*}
    \begin{proof}
        Maybe in the future.
    \end{proof}

\section{Dirac action}

    Now, we have all the tools to build a Lorentz invariant lagrangian
    \begin{equation*}
        \mathcal L = \overline \psi (x) \gamma^\mu \partial_\mu \psi(x) - m \overline \psi (x) \psi (x) = \overline \psi (x) (i \gamma^\mu \partial_\mu - m) \psi(x) ~.
    \end{equation*}
    We have added an $i$ factor to ensure that $\mathcal L \in \mathbb R$.
    \begin{proof}
        Maybe in the future.
    \end{proof}

    The dymensional analysis is 
    \begin{equation*}
        [S] = 0 ~, [d^4 x] = - ~, [\mathcal L] = 4~, [\psi] = \frac{3}{2} ~, [\partial_\mu] = 1 ~, [m] = 1 ~.
    \end{equation*}
    Notice that in the Klein Gordon theory, we had $[\varphi] = 1$. However, in a renormalisable theory, the coupling between operators must be of dimension $4$. This means that only terms like $\varphi \overline \psi \psi$ are allowed. Another difference in the Dirac theory is that the lagrangian in at first order whereas in the Klein-Gordon theory is at second order. This is possible only because the gamma matrices exists only in the Dirac theory, while in the Klein-Gordon we have to constract to partial derivatives to get a scalar.

    The equations of motion can be obtained by the Euler-Lagrange equations: the Dirac equation is 
    \begin{equation*}
        (i \gamma^\mu \partial_\mu - m) \psi(x) = 0
    \end{equation*}
    and the conjugate Dirac equation is 
    \begin{equation*}
        \overline \psi(x) (i \gamma^\mu \overleftarrow{\partial_\mu} + m) = 0 ~.
    \end{equation*}
    \begin{proof}
        Maybe in the future.
    \end{proof}

\section{Dirac and Klein-Gordon equations}

    The four-components of the Dirac spinor satisfy the Dirac equation, but each components separately satisfy the Klein-Gordon equation, because it means that particles ensures the mass-shell condition.
    \begin{proof}
        Maybe in the future.
    \end{proof}

\chapter{Chiral spinors}

    Recall that the Dirac representation $(\frac{1}{2}, 0) \oplus (0, \frac{1}{2})$ is reducible and it can be decomposed into $2$ irreducible Weyl representations $(\frac{1}{2}, 0)$ and $(0, \frac{1}{2})$. 

    We introduce the $\gamma^5$ matrix 
    \begin{equation*}
        \gamma^5 = i \gamma^0 \gamma^1 \gamma^2 \gamma^3
    \end{equation*}
    such that it satisfies 
    \begin{equation*}
        \{\gamma^\mu, \gamma^5\} = 0~, \quad (\gamma^5)^2 = \mathbb I~, \quad (\gamma^5)^\dagger = \gamma^5 ~.
    \end{equation*}
    
    In the Weyl basis it becomes 
    \begin{equation*}
        \gamma^5 = \begin{bmatrix}
            - \mathbb I_2 & 0 \\ 0 & \mathbb I_2 \\
        \end{bmatrix} ~.
    \end{equation*}

    With $\gamma^5$, we can define the projection operators 
    \begin{equation*}
        P_L = \frac{\mathbb I - \gamma^5}{2} ~, \quad P_R = \frac{\mathbb I + \gamma^5}{2} ~.
    \end{equation*}
    such that they satisfy 
    \begin{equation*}
        P_L^2 = P_L ~, \quad P_R^2 = P_R ~\quad P_L^\dagger = P_L ~, \quad P_R^\dagger = P_R ~, \quad P_L P_R = P_R P_L = 0 ~, \quad P_L + P_R = \mathbb I ~.
    \end{equation*}
    and they decompose the Dirac spinor into a left-handed Weyl spinor $\psi_L^{(W)}$ and a right-handed Weyl spinor $\psi_R^{(W)}$
    \begin{equation*}
        \psi_L = \begin{bmatrix}
            \psi_L^{(W)} \\ 0 \\
        \end{bmatrix} = P_L \psi = \frac{\mathbb I - \gamma^5}{2} \psi ~, \quad \psi_R = \begin{bmatrix}
            0 \\ \psi_R^{(W)} \\
        \end{bmatrix} = P_R \psi = \frac{\mathbb I + \gamma^5}{2} \psi ~.
    \end{equation*}
    Furthermore, their eigenvalues are 
    \begin{equation*}
        \gamma^5 \psi_L = (-1) \psi_L ~, \quad \gamma^5 \psi_R = (+1) \psi_R ~.
    \end{equation*}

    The Dirac lagrangian in terms of the Weyl spinors is 
    \begin{equation*}
        \mathcal L = \overline \psi_L i \gamma^\mu \partial_\mu \psi_L + \overline \psi_R i \gamma^\mu \partial_\mu \psi_R - m (\overline \psi_L \psi_R + \overline \psi_R \psi_L) ~.
    \end{equation*}
    Notice that for a massive fermions, we do not know if it is right-handed or left-handed because of the last mixed term. Instead for massless fermions, we know.
    \begin{proof}
        Maybe in the future.
    \end{proof}

    In terms of the Weyl spinors, the Dirac equation becomes 
    \begin{equation*}
        \begin{cases}
            i \pdv{}{t} \psi^{(W)}_R (x) + i \boldsymbol \sigma \cdot \boldsymbol \nabla \psi_R^{(W)} - m \psi_L^{(W)} = 0 \\
            i \pdv{}{t} \psi^{(W)}_L (x) + i \boldsymbol \sigma \cdot \boldsymbol \nabla \psi_L^{(W)} - m \psi_R^{(W)} = 0 
        \end{cases} ~.
    \end{equation*}
    \begin{proof}
        Maybe in the future.
    \end{proof}

    For massless fermions, which have a hamiltonian $\hat H = |\hat p|$, the Weyl equations become
    \begin{equation*}
        \begin{cases}
            (\hat{\mathbf S} \cdot \mathbf p) \psi^{(W)}_R (x) = (+1) \psi^{(W)}_R (x) \\
            (\hat{\mathbf S} \cdot \mathbf p) \psi^{(W)}_L (x) = (-1) \psi^{(W)}_L (x) \\
        \end{cases}
    \end{equation*}
    where $\mathbf p$ is the direction of motion and $\hat S$ is the spin operator. The quantity $\hat{\mathbf S} \cdot \mathbf p$ is called helicity and it is the projection of the spin along the direction of motion. 
    \begin{proof}
        Maybe in the future.
    \end{proof}

\section{Parity} 

    The parity operator transforms a right-handed Weyl spinor into a left-handed Weyl spinor and viceversa
    \begin{equation*}
        \begin{cases}
            {\psi'}_L^{(W)} = \psi_R^{(W)} \\
            {\psi'}_R^{(W)} = \psi_L^{(W)} \\
        \end{cases} ~.
    \end{equation*}
    \begin{proof}
        Maybe in the future.
    \end{proof}

\chapter{Solutions of the Dirac equation}

    Since each components of the Dirac spinor $\psi(x)$ satisfies the Klein-Gordon equation, the plane waves are solutions 
    \begin{equation*}
        \psi_\alpha (x) = u_\alpha (\mathbf p) \exp(- i p x) ~,
    \end{equation*}
    where $u_\alpha (\mathbf p)$ is the polarisation vector with $4$ components $\alpha = 1,2,3,4$ and $p_0 = E_{\mathbf p} = \sqrt{|\mathbf p|^2 + m^2}$. Furthermore, in order to satisfy the Dirac equation, $u_\alpha (\mathbf p)$ satisfies 
    \begin{equation}\label{cond}
        \begin{bmatrix}
            - m & p^\mu \sigma_\mu \\ p^\mu \overline \sigma_\mu & -m \\
        \end{bmatrix} u (\mathbf p) = 0 ~,
    \end{equation}
    where $\sigma^\mu = (\mathbb I_2, \sigma^i)$ and $\overline \sigma^\mu = (\mathbb I_2, - \sigma^i)$.
    \begin{proof}
        In fact, 
        \begin{equation*}
            0 = (i \gamma^\mu \partial_\mu - m) \psi(x) = (i \gamma^\mu (- i p_\mu ) - m) u(\mathbf p) \exp(- i p x) ~.
        \end{equation*}
        Hence 
        \begin{equation*}
            0 = (\gamma^\mu  p_\mu - m) u(\mathbf p) = \Big ( 
            \begin{bmatrix}
                0 & 1 \\ 1 & 0 \\ 
            \end{bmatrix} p_0 + \begin{bmatrix}
                0 & \sigma^i \\ - \sigma^i & 0
            \end{bmatrix} p_i  - m \begin{bmatrix}
                1 & 0 \\ 0 & 1 \\
            \end{bmatrix} \Big) u (\mathbf p) = \begin{bmatrix}
                - m & p^\mu \sigma_\mu \\ p^\mu \overline \sigma_\mu & -m \\
            \end{bmatrix} u (\mathbf p) = 0 ~. 
        \end{equation*}
    \end{proof}

    Moreover, we can split the polarisation vector $u (\mathbf p)$ into the right and the left-handed part 
    \begin{equation}\label{split}
        u (\mathbf p) = \begin{bmatrix}
            u_L (\mathbf p) \\ u_R (\mathbf p) \\
        \end{bmatrix} ~,
    \end{equation}
    which can be intepret as the positive frequency solution. 
    \begin{proof}
        In fact, putting~\eqref{split} into~\eqref{cond} 
        \begin{equation}\label{proof4}
            \begin{cases}
                (p^\mu \overline \sigma_\mu) u_L = m u_R \\
                (p^\mu \sigma_\mu) u_R = m u_L \\
            \end{cases} ~.
        \end{equation}

        Notice that 
        \begin{equation*}
        \begin{aligned}
            (p_\mu \sigma^\mu) (p_\nu \overline \sigma^\nu) & = (p_0 + p_i \sigma^i) (p_0 + p_j \overline \sigma^j) \\ & = (p_0 + p_i \sigma^i) (p_0 - p_j \sigma^j) \\ & = p_0^2 - p_i p_j \underbrace{\sigma^i \sigma^j}_{\delta^{ij} + i \epsilon^{ijk} \sigma_k} \\ & = p_0^2 - p_i p_j \underbrace{\delta^{ij}}_{i = j} + \cancel{i \underbrace{p_i p_j}_{symm} \underbrace{\epsilon^{ijk}}_{anti} \sigma_k} \\ & = p_0^2 - |\mathbf p|^2 = m^2 ~.
        \end{aligned}
        \end{equation*}

        We choose the form of $u_L$ such that 
        \begin{equation*}
            u_L (\mathbf p) = A p^\mu \sigma_\mu \chi ~,
        \end{equation*}
        where $A$ is a constant and $\chi$ is $2$-components spinor. 
        Hence, the first equation of~\eqref{proof4}
        \begin{equation*}
            m u_R = (p^\mu \overline \sigma_\mu) u_L = A \underbrace{(p^\mu \overline \sigma_\mu) (p^\nu \sigma_\nu)}_{m^2} \chi = A m^2 \chi
        \end{equation*}
        and 
        \begin{equation*}
            u_R (\mathbf p) = m A \chi.
        \end{equation*}
        In this way, the second equation of~\eqref{proof4} is automatically satisfied
        \begin{equation*}
            m u_L = p^\mu \sigma_\mu \underbrace{m A \chi }_{u_R} = p^\mu \sigma_\mu u_R ~.
        \end{equation*}
        Therefore 
        \begin{equation*}
            u (\mathbf p) = A \begin{bmatrix}
                (p^\mu \sigma_\mu) \chi \\ m \chi \\
            \end{bmatrix} ~.
        \end{equation*} 

        We choose $A = \frac{1}{m}$ and $\chi = \sqrt{p^\mu \overline \sigma_\mu} \xi$, where $\xi$ is a constant $2$-components spinor normalised such that $\xi^\dagger \xi = 1$. Hence 
        \begin{equation*}
            u (\mathbf p) = \frac{1}{m} \begin{bmatrix}
                (p^\mu \sigma_\mu) \sqrt{p^\nu \overline \sigma_\nu} \xi \\ m \sqrt{p^\mu \overline \sigma_\mu} \xi \\
            \end{bmatrix} = \frac{1}{m} \begin{bmatrix}
                \sqrt{p^\mu \sigma_\mu} \underbrace{\sqrt{p^\alpha \sigma_\alpha} \sqrt{p^\nu \overline \sigma_\nu}}_m \xi \\ m \sqrt{p^\mu \overline \sigma_\mu} \xi \\
            \end{bmatrix} = \begin{bmatrix}
                \sqrt{p^\mu \sigma_\mu} \xi \\ \sqrt{p^\mu \overline \sigma_\mu} \xi \\
            \end{bmatrix} ~.
            \end{equation*}
    \end{proof}

    Actually, there is another class of plane waves solutions, the negative frequency solutions 
    \begin{equation*}
        \psi(x) = v(\mathbf p) \exp(i p x) ~,
    \end{equation*}
    where $v (\mathbf p)$ is the polarisation vector 
    \begin{equation*}
        v (\mathbf p) = \begin{bmatrix}
            \sqrt{p_\mu \sigma^\mu} \eta \\
            - \sqrt{p_\mu \overline \sigma^\mu} \eta \\
        \end{bmatrix} ~,
    \end{equation*}
    where $\eta$ is a constant $2$-components spinor normalised such that $\eta^\dagger \eta = 1$. 

    They can be distinguished since 
    \begin{equation*}
        (\gamma^\mu p_\mu - m) u(\mathbf p) = 0 ~, \quad (\gamma^\mu p_\mu + m) v(\mathbf p) = 0 ~,
    \end{equation*}
    and 
    \begin{equation*}
        \hat H \psi(x) = i \pdv{}{t} (u (\mathbf p) \exp(- i px)) = E_{\mathbf p} \psi (x) ~, \quad \hat H \psi(x) = i \pdv{}{t} (u (\mathbf p) \exp(i px)) = - E_{\mathbf p} \psi (x) ~.
    \end{equation*}

    Consider a massive particle in the rest frame $p^\mu = (m, 0, 0,0)$. The positive frequency solutions look like 
    \begin{equation*}
        \psi(x) = \sqrt{m} \exp(- i E_{\mathbf p} t) \begin{bmatrix}
            \xi \\ \xi \\
        \end{bmatrix} ~.
    \end{equation*}
    Using~\eqref{lorspin}, we restrict to spatial rotations in which the generators are
    \begin{equation*}
        S^{ij} = - \frac{1}{2} \epsilon^{ijk} \begin{bmatrix}
            \sigma^k & 0 \\ 0 & \sigma^k \\
        \end{bmatrix} ~,
    \end{equation*}
    where $i \neq j$ and the parameters are 
    \begin{equation*}
        \omega_{ij} = - \epsilon_{ijk} \theta^k ~.
    \end{equation*}
    Therefore, the matrix rotation is 
    \begin{equation*}
        \exp(\frac{1}{2} \omega_{ij} S^{ij}) = \begin{bmatrix}
            \exp(\frac{i}{2} \theta^i \sigma_i) & 0 \\ 0 & \exp(\frac{i}{2} \theta^i \sigma_i) \\
        \end{bmatrix}
    \end{equation*}
    and the Dirac spinor transforms as 
    \begin{equation*}
        \psi' (x) = \begin{bmatrix}
            \exp(\frac{i}{2} \theta^i \sigma_i) & 0 \\ 0 & \exp(\frac{i}{2} \theta^i \sigma_i) \\
        \end{bmatrix} \psi (x) ~,
    \end{equation*}
    which induces a transformation on $\xi$ such that 
    \begin{equation*}
        \xi ' = \exp(\frac{i}{2} \theta^i \sigma_i) \xi ~.
    \end{equation*}
    This is indeed an $SU(2)$ transformation, where we can recognise tha spin operator $\hat{\mathbf S} = \frac{1}{2} \boldsymbol \sigma$ and $\xi$ is a $2$-components spinot which describes particle with spin $\frac{1}{2}$. Since $\xi^\dagger \xi = 1$, we choose, for the spin up
    \begin{equation*}
        \xi = \begin{bmatrix}
            1 \\ 0 \\
        \end{bmatrix} \colon \sigma_3 \xi = \begin{bmatrix}
            1 & 0 \\ 0 & -1 \\
        \end{bmatrix} \begin{bmatrix}
            1 \\ 0 \\
        \end{bmatrix} = (+1) \xi ~,
    \end{equation*}
    for the spin down
    \begin{equation*}
        \xi = \begin{bmatrix}
            0 \\ 1 \\
        \end{bmatrix} \colon \sigma_3 \xi = \begin{bmatrix}
            1 & 0 \\ 0 & -1 \\
        \end{bmatrix} \begin{bmatrix}
            0 \\ 1 \\
        \end{bmatrix} = (-1) \xi ~,
    \end{equation*}

\chapter{Useful formulas} 

\section{Inner product}

    We introduce a basis of the $2$-components spinors 
    \begin{equation*}
        \xi^r ~, \eta^s ~,
    \end{equation*}
    where $r,s = 1,2$ such that they satisfy 
    \begin{equation*}
        (\xi^\dagger)^r \xi^s = \delta^{rs} ~, \quad (\eta^\dagger)^r \eta^s = \delta^{rs}  ~.
    \end{equation*}

    For example, 
    \begin{equation*}
        \xi^1 = \begin{bmatrix}
            1 \\ 0 \\
        \end{bmatrix} ~, \quad \xi^2 = \begin{bmatrix}
            0 \\ 1 \\
        \end{bmatrix} ~, \quad \eta^1 = \begin{bmatrix}
            1 \\ 0 \\
        \end{bmatrix} ~, \quad \eta^2 = \begin{bmatrix}
            0 \\ 1 \\
        \end{bmatrix} ~.
    \end{equation*}

    We define the following inner products 
    \begin{enumerate}
        \item \begin{equation*}
            (u^\dagger)^r (\mathbf p) u^s (\mathbf p) = 2 p^0 \delta^{rs} ~,
        \end{equation*}
        \item \begin{equation*}
            \overline u^r (\mathbf p) u^s (\mathbf p) = 2 m \delta^{rs} ~,
        \end{equation*} 
        \item \begin{equation*}
            (v^\dagger)^r (\mathbf p) v^s (\mathbf p) = 2 p^0 \delta^{rs} ~,
        \end{equation*}
        \item \begin{equation*}
            \overline v^r (\mathbf p) v^s (\mathbf p) = - 2 m \delta^{rs} ~,
        \end{equation*}
        \item \begin{equation*}
            \overline u^r (\mathbf p) v^s (\mathbf p) = \overline v^r (\mathbf p) u^s (\mathbf p) = 0 ~,
        \end{equation*}
        \item \begin{equation*}
            (u^\dagger)^r (\mathbf p) v^s (- \mathbf p) = (v^\dagger)^r (\mathbf p) u^s (- \mathbf p) = 0 ~.
        \end{equation*}
    \end{enumerate}
    \begin{proof}
        For the first one, 
        \begin{equation*}
        \begin{aligned}
            (u^\dagger)^r (\mathbf p) u^s (\mathbf p) & = \begin{bmatrix}
                \sqrt{p^\mu \sigma_\mu} (\xi^\dagger)^r & \sqrt{p^\mu \overline \sigma_\mu} (\xi^\dagger)^r \\
            \end{bmatrix} \begin{bmatrix}
                \sqrt{p^\mu \sigma_\mu} \xi^s \\ \sqrt{p^\mu \overline \sigma_\mu} \xi^s \\
            \end{bmatrix} \\ & = (\xi^\dagger)^r p^\mu \sigma_\mu \xi^s + (\xi^\dagger)^r p^\mu \overline \sigma_\mu \xi^s \\ & = (\xi^\dagger)^r p^0 \underbrace{\sigma_0}_{\mathbb I_2} \xi^s + (\xi^\dagger)^r p^0 \underbrace{\overline \sigma_0}_{\mathbb I_2} \xi^s + (\xi^\dagger)^r p^i \sigma_i \xi^s + (\xi^\dagger)^r p^i \underbrace{\overline \sigma_i}_{- \sigma_i} \xi^s \\ & = (\xi^\dagger)^r p^0 \xi^s + (\xi^\dagger)^r p^0 \xi^s + \cancel{(\xi^\dagger)^r p^i \sigma_i \xi^s } - \cancel{(\xi^\dagger)^r p^i \sigma_i \xi^s} \\ & = 2 p_0 \underbrace{(\xi^\dagger)^r \xi^s}_{\delta^{rs}} \\ & = 2 p_0 \delta^{rs} ~.
        \end{aligned}
        \end{equation*}

        For the second one, 
        \begin{equation*}
        \begin{aligned}
            \overline u^r (\mathbf p) u^s (\mathbf p) & = \begin{bmatrix}
                \sqrt{p^\mu \sigma_\mu} (\xi^\dagger)^r & \sqrt{p^\mu \overline \sigma_\mu} (\xi^\dagger)^r \\
            \end{bmatrix} \begin{bmatrix}
                0 & 1 \\ 1 & 0 \\
            \end{bmatrix} \begin{bmatrix}
                \sqrt{p^\mu \sigma_\mu} \xi^s \\ \sqrt{p^\mu \overline \sigma_\mu} \xi^s \\
            \end{bmatrix} \\ & = (\xi^\dagger)^r p^\mu \underbrace{\sqrt{p^\mu  \sigma_\mu} \sqrt{p^\nu \overline \sigma_\nu}}_m \xi^s + (\xi^\dagger)^r \underbrace{\sqrt{p^\mu \sigma_\mu} \sqrt{p^\nu \overline \sigma_\nu}}_m \xi^s \\ & = 2 m \underbrace{(\xi^\dagger)^r \xi^s}_{\delta^{rs}} \\ & 2 m = \delta^{rs} ~.
        \end{aligned}
        \end{equation*}

        For the third one, 
        \begin{equation*}
        \begin{aligned}
            (v^\dagger)^r (\mathbf p) v^s (\mathbf p) & = \begin{bmatrix}
                \sqrt{p^\mu \sigma_\mu} (\eta^\dagger)^r & - \sqrt{p^\mu \overline \sigma_\mu} (\eta^\dagger)^r \\
            \end{bmatrix} \begin{bmatrix}
                \sqrt{p^\mu \sigma_\mu} \eta^s \\ - \sqrt{p^\mu \overline \sigma_\mu} \eta^s \\
            \end{bmatrix} \\ & = (\eta^\dagger)^r p^\mu \sigma_\mu \eta^s + (\eta^\dagger)^r p^\mu \overline \sigma_\mu \eta^s \\ & = (\eta^\dagger)^r p^0 \underbrace{\sigma_0}_{\mathbb I_2} \eta^s + (\eta^\dagger)^r p^0 \underbrace{\overline \sigma_0}_{\mathbb I_2} \eta^s + (\eta^\dagger)^r p^i \sigma_i \eta^s + (\eta^\dagger)^r p^i \underbrace{\overline \sigma_i}_{- \sigma_i} \eta^s \\ & = (\eta^\dagger)^r p^0 \eta^s + (\eta^\dagger)^r p^0 \eta^s + \cancel{(\eta^\dagger)^r p^i \sigma_i \eta^s } - \cancel{(\eta^\dagger)^r p^i \sigma_i \eta^s} \\ & = 2 p_0 \underbrace{(\eta^\dagger)^r \eta^s}_{\delta^{rs}} \\ & = 2 p_0 \delta^{rs} ~.
        \end{aligned}
        \end{equation*}

        For the fourth one, 
        \begin{equation*}
        \begin{aligned}
            \overline v^r (\mathbf p) v^s (\mathbf p) & = \begin{bmatrix}
                \sqrt{p^\mu \sigma_\mu} (\eta^\dagger)^r & - \sqrt{p^\mu \overline \sigma_\mu} (\eta^\dagger)^r \\
            \end{bmatrix} \begin{bmatrix}
                0 & 1 \\ 1 & 0 \\
            \end{bmatrix} \begin{bmatrix}
                \sqrt{p^\mu \sigma_\mu} \eta^s \\ - \sqrt{p^\mu \overline \sigma_\mu} \eta^s \\
            \end{bmatrix} \\ & = - (\eta^\dagger)^r p^\mu \underbrace{\sqrt{p^\mu  \sigma_\mu} \sqrt{p^\nu \overline \sigma_\nu}}_m \eta^s - (\eta^\dagger)^r \underbrace{\sqrt{p^\mu \sigma_\mu} \sqrt{p^\nu \overline \sigma_\nu}}_m \eta^s \\ & = - 2 m \underbrace{(\eta^\dagger)^r \eta^s}_{\delta^{rs}} \\ & - 2 m = \delta^{rs} ~.
        \end{aligned}
        \end{equation*}

        For the fifth one, 
        \begin{equation*}
        \begin{aligned}
            \overline u^r (\mathbf p) v^s (\mathbf p) & = \begin{bmatrix}
                \sqrt{p^\mu \sigma_\mu} (\xi^\dagger)^r & \sqrt{p^\mu \overline \sigma_\mu} (\xi^\dagger)^r \\
            \end{bmatrix} \begin{bmatrix}
                0 & 1 \\ 1 & 0 \\
            \end{bmatrix} \begin{bmatrix}
                \sqrt{p^\mu \sigma_\mu} \eta^s \\ - \sqrt{p^\mu \overline \sigma_\mu} \eta^s \\
            \end{bmatrix} \\ & = - (\xi^\dagger)^r \underbrace{\sqrt{p^\mu  \sigma_\mu} \sqrt{p^\nu \overline \sigma_\nu}}_m \eta^s + (\xi^\dagger)^r \underbrace{\sqrt{p^\mu \sigma_\mu} \sqrt{p^\nu \overline \sigma_\nu}}_m \eta^s \\ & = m ( \cancel{(- \xi^\dagger)^r \eta^s} + \cancel{(\xi^\dagger)^r \eta^2}) = 0 ~.
        \end{aligned}
        \end{equation*}

        For the second in the fifth one, 
        \begin{equation*}
        \begin{aligned}
            \overline v^r (\mathbf p) u^s (\mathbf p) & = \begin{bmatrix}
                \sqrt{p^\mu \sigma_\mu} (\eta^\dagger)^r & \sqrt{p^\mu \overline \sigma_\mu} (\eta^\dagger)^r \\
            \end{bmatrix} \begin{bmatrix}
                0 & 1 \\ 1 & 0 \\
            \end{bmatrix} \begin{bmatrix}
                \sqrt{p^\mu \sigma_\mu} \xi^s \\ - \sqrt{p^\mu \overline \sigma_\mu} \xi^s \\
            \end{bmatrix} \\ & = - (\eta^\dagger)^r \underbrace{\sqrt{p^\mu  \sigma_\mu} \sqrt{p^\nu \overline \sigma_\nu}}_m \xi^s + (\eta^\dagger)^r \underbrace{\sqrt{p^\mu \sigma_\mu} \sqrt{p^\nu \overline \sigma_\nu}}_m \xi^s \\ & = m ( \cancel{(- \eta^\dagger)^r \xi^s} + \cancel{(\eta^\dagger)^r \xi^2}) = 0 ~.
        \end{aligned}
        \end{equation*}

        For the sixth one, 
        \begin{equation*}
        \begin{aligned}
            (u^\dagger)^r (\mathbf p) v^s (- \mathbf p) & = \begin{bmatrix}
                \sqrt{p^\mu \sigma_\mu} (\xi^\dagger)^r & \sqrt{p^\mu \overline \sigma_\mu} (\xi^\dagger)^r \\
            \end{bmatrix} \begin{bmatrix}
                \sqrt{\overline p^\mu \sigma_\mu} \eta^s \\ - \sqrt{\overline p^\mu \overline \sigma_\mu} \eta^s \\
            \end{bmatrix} \\ & = (\xi^\dagger)^r \underbrace{\sqrt{p^\mu \sigma_\mu} \sqrt{\overline p^\mu \sigma_\mu}}_m \eta^s - (\xi^\dagger)^r \underbrace{\sqrt{p^\mu \overline \sigma_\mu} \sqrt{\overline p^\mu \overline \sigma_\mu}}_m \eta^s \\ & = m ((\xi^\dagger)^r \eta^s - (\xi^\dagger)^r \eta^s) = 0 ~.
        \end{aligned}
        \end{equation*}

        For the second in the sixth one, 
        \begin{equation*}
        \begin{aligned}
            (v^\dagger)^r (\mathbf p) u^s (- \mathbf p) & = \begin{bmatrix}
                \sqrt{p^\mu \sigma_\mu} (\eta^\dagger)^r & - \sqrt{p^\mu \overline \sigma_\mu} (\eta^\dagger)^r \\
            \end{bmatrix} \begin{bmatrix}
                \sqrt{\overline p^\mu \sigma_\mu} \xi^s \\ \sqrt{\overline p^\mu \overline \sigma_\mu} \xi^s \\
            \end{bmatrix} \\ & = (\eta^\dagger)^r \underbrace{\sqrt{p^\mu \sigma_\mu} \sqrt{\overline p^\mu \sigma_\mu}}_m \xi^s - (\eta^\dagger)^r \underbrace{\sqrt{p^\mu \overline \sigma_\mu} \sqrt{\overline p^\mu \overline \sigma_\mu}}_m \xi^s \\ & = m ((\eta^\dagger)^r \xi^s - (\eta^\dagger)^r \xi^s) = 0 ~.
        \end{aligned}
        \end{equation*}
    \end{proof}

\section{Outer product}

    We define the following outer products 
    \begin{enumerate}
        \item \begin{equation*}
            \sum_{s=1}^{2} u^s (\mathbf p) \overline u^s(\mathbf p) = \gamma^\mu p_\mu + m \mathbb I_4 ~,
        \end{equation*} 
        \item \begin{equation*}
            \sum_{s=1}^{2} v^s (\mathbf p) \overline v^s(\mathbf p) = \gamma^\mu p_\mu - m \mathbb I_4 ~.
        \end{equation*} 
    \end{enumerate}
    \begin{proof}
        For the first one
        \begin{equation*}
        \begin{aligned}
            \sum_{s=1}^{2} u^s (\mathbf p) \overline u^s(\mathbf p) & = \sum_{s=1}^{2} \begin{bmatrix}
                \sqrt{ p^\mu \sigma_\mu} \xi^s \\ \sqrt{ p^\mu \overline \sigma_\mu} \xi^s \\
            \end{bmatrix} \begin{bmatrix}
                \sqrt{p^\mu \sigma_\mu} (\xi^\dagger)^r & \sqrt{p^\mu \overline \sigma_\mu} (\xi^\dagger)^r \\
            \end{bmatrix} \begin{bmatrix}
                0 & 1 \\ 1 & 0 \\
            \end{bmatrix} \\ & = \sum_{s=1}^{2} \begin{bmatrix}
                \sqrt{p^\mu \sigma_\mu} \xi^s (\xi^\dagger)^s \sqrt{p^\mu \overline \sigma_\mu} & \sqrt{p^\mu \sigma_\mu} \xi^s (\xi^\dagger)^s \sqrt{p^\mu \sigma_\mu} \\ \sqrt{p^\mu \overline \sigma_\mu} \xi^s (\xi^\dagger)^s \sqrt{p^\mu \overline \sigma_\mu} & \sqrt{p^\mu \overline \sigma_\mu} \xi^s (\xi^\dagger)^s \sqrt{p^\mu \sigma_\mu} 
            \end{bmatrix} \\ & = \begin{bmatrix}
                \underbrace{\sqrt{p^\mu \sigma_\mu} \sqrt{p^\mu \overline \sigma_\mu}}_m & \sqrt{p^\mu \sigma_\mu} \sqrt{p^\mu \sigma_\mu} \\ \sqrt{p^\mu \overline \sigma_\mu} \sqrt{p^\mu \overline \sigma_\mu} & \underbrace{\sqrt{p^\mu \overline \sigma_\mu} \sqrt{p^\mu \sigma_\mu}}_m
            \end{bmatrix} \\ & = \begin{bmatrix}
                m & p^\mu \sigma_\mu \\ p^\mu \overline \sigma_\mu & m \\
            \end{bmatrix} \\ & = \gamma^\mu p_\mu + m ~,
        \end{aligned}
        \end{equation*}
        where we have used 
        \begin{equation*}
            \sum_{s=1}^{2} \xi^s (\xi^\dagger)^s = \mathbb I_2 ~.
        \end{equation*}

        For the first one
        \begin{equation*}
        \begin{aligned}
            \sum_{s=1}^{2} v^s (\mathbf p) \overline v^s(\mathbf p) & = \sum_{s=1}^{2} \begin{bmatrix}
                \sqrt{ p^\mu \sigma_\mu} \eta^s \\ - \sqrt{ p^\mu \overline \sigma_\mu} \eta^s \\
            \end{bmatrix} \begin{bmatrix}
                \sqrt{p^\mu \sigma_\mu} (\eta^\dagger)^r & - \sqrt{p^\mu \overline \sigma_\mu} (\eta^\dagger)^r \\
            \end{bmatrix} \begin{bmatrix}
                0 & 1 \\ 1 & 0 \\
            \end{bmatrix} \\ & = \sum_{s=1}^{2} \begin{bmatrix}
                - \sqrt{p^\mu \sigma_\mu} \eta^s (\eta^\dagger)^s \sqrt{p^\mu \overline \sigma_\mu} & \sqrt{p^\mu \sigma_\mu} \eta^s (\eta^\dagger)^s \sqrt{p^\mu \sigma_\mu} \\ \sqrt{p^\mu \overline \sigma_\mu} \eta^s (\eta^\dagger)^s \sqrt{p^\mu \overline \sigma_\mu} & - \sqrt{p^\mu \overline \sigma_\mu} \eta^s (\eta^\dagger)^s \sqrt{p^\mu \sigma_\mu} 
            \end{bmatrix} \\ & = \begin{bmatrix}
                - \underbrace{\sqrt{p^\mu \sigma_\mu} \sqrt{p^\mu \overline \sigma_\mu}}_m & \sqrt{p^\mu \sigma_\mu} \sqrt{p^\mu \sigma_\mu} \\ \sqrt{p^\mu \overline \sigma_\mu} \sqrt{p^\mu \overline \sigma_\mu} & -\underbrace{\sqrt{p^\mu \overline \sigma_\mu} \sqrt{p^\mu \sigma_\mu}}_m 
            \end{bmatrix} \\ & = \begin{bmatrix}
                - m & p^\mu \sigma_\mu \\ p^\mu \overline \sigma_\mu & - m \\
            \end{bmatrix} \\ & = \gamma^\mu p_\mu - m ~,
        \end{aligned}
        \end{equation*}
        where we have used 
        \begin{equation*}
            \sum_{s=1}^{2} \eta^s (\eta^\dagger)^s = \mathbb I_2 ~.
        \end{equation*}
    \end{proof}

\chapter{How to not quantise the Dirac theory}