\part{Dirac theory}

\chapter{Dirac equation}

\section{Derivation}

    We are looking for a quantum equation that describes $\frac{1}{2}$-spin particles, but unlike the Klein-Gordon equation, it allows a probabilistic interpretation. The problem with the Klein-Gordon equation is the presence of second-order terms in time, therefore we need an hamiltonian which is linear but at the same time recover the energy-momentum relation~\eqref{enmom}. The first guess is
    \begin{equation}\label{guess}
        E = c \mathbf p \cdot \boldsymbol \alpha + m c^2 \beta ~,
    \end{equation}
    where $\alpha$ and $\beta$ are hermitian matrices such that satisfies the Clifford algebra 
    \begin{equation}\label{clifford}
        \{\alpha^i, \alpha^j\} = 2 \delta^{ij} \mathbb I ~, \quad \{\beta, \beta\} = 2 \mathbb I ~, \quad \{\alpha^i, \beta\} = 0~.
    \end{equation}
    \begin{proof}
        Infact, we compute the square of~\eqref{guess}
        \begin{equation*}
        \begin{aligned}
            E^2 & = (c p^i \alpha^i + m c^2 \beta)^2 \\ & = (c p^i \alpha^i + m c^2 \beta) (c p^j \alpha^j + m c^2 \beta) \\ & = c^2 \alpha^i p^i \alpha^j p^j + \beta^2 m^2 c^4 + m c^3 p^i \alpha^i \beta + m c^3 p^j \beta \alpha^j \\ & = c^2 p^i p^j \underbrace{\alpha^i \alpha^j}_{\frac{\alpha^i \alpha^j + \alpha^j \alpha^i}{2}} + \beta^2 m^2 c^4 + m c^3 p^i (\alpha^i \beta + \beta \alpha^i) \\ & = c^2 p^i p^j \frac{\alpha^i \alpha^j + \alpha^j \alpha^i}{2} + \beta^2 m^2 c^4 + m c^3 p^i (\alpha^i \beta + \beta \alpha^i) ~,
        \end{aligned}
        \end{equation*}
        where in the fourth row, we exploit the symmetry of $p^i p^j$ to symmetrise $\alpha^i \alpha^j$. We compare it with~\eqref{enmom}
        \begin{equation*}
            E^2 = c^2 p^i p^j \underbrace{\frac{\alpha^i \alpha^j + \alpha^j \alpha^i}{2}}_{\delta^{ij}} + \underbrace{\beta^2}_{1} m^2 c^4 + m c^3 p^i \underbrace{(\alpha^i \beta + \beta \alpha^i)}_{0} = p^2 c^2 + m^2 c^4 ~.
        \end{equation*}
        Hence
        \begin{equation*}
            \{\alpha^i, \alpha^j\} = \alpha^i \alpha^j + \alpha^j \alpha^i = 2 \delta^{ij} ~,
        \end{equation*}
        \begin{equation*}
            \{\beta, \beta\} = \beta^2 + \beta^2 = 2 \beta^2 = 2 ~,
        \end{equation*}
        \begin{equation*}
            \{\alpha^i, \beta\} = \alpha^i \beta + \beta \alpha^i = 0 ~.
        \end{equation*}
    \end{proof}

    The minimal solutions for the set of algebraic equation~\eqref{clifford} are $4 \times 4$ traceless matrices $\boldsymbol \alpha$ and $\beta$ such that
    \begin{equation*}
        \alpha^i = \begin{bmatrix}
            0 & \sigma^i \\ 
            \sigma^i & 0 \\
        \end{bmatrix} ~, \quad \beta = 
        \begin{bmatrix}
            \mathbb I_2 & 0 \\
            0 & - \mathbb I_2 \\
        \end{bmatrix} ~, 
    \end{equation*}
    where $\sigma^i$ are the Pauli matrices 
    \begin{equation*}
        \sigma^1 = \begin{bmatrix} 0 & 1 \\ 1 & 0 \\ \end{bmatrix} ~,
        \quad \sigma^2 = \begin{bmatrix} 0 & -i \\ i & 0 \\ \end{bmatrix} ~,
        \quad \sigma^3 = \begin{bmatrix} 1 & 0 \\ 0 & -1 \\ \end{bmatrix} ~,
    \end{equation*}
    and they satisfy the relation 
    \begin{equation} \label{sigma}
        \sigma^i \sigma^j = \delta^{ij} \mathbb I + i \epsilon^{ijk} \sigma^k ~.
    \end{equation}
    It is called the Dirac representation and it is the only irreducible representation of the Clifford algebra up to others that are unitarily equivalent (by a change of basis) to the Dirac one or that are higher dimensional and thus reducible.

    The hamiltonian form of the Dirac equation becomes 
    \begin{equation}\label{hamdirac}
        i \hbar \pdv{}{t} \psi (t, \mathbf x) = (- i \hbar c \boldsymbol \alpha \cdot \boldsymbol \nabla + \beta m c^2) \psi (t, \mathbf x)  = H_D \psi (t, \mathbf x) ~,
    \end{equation}
    while in covariant form it becomes 
    \begin{equation}\label{covdirac}
        (\gamma^\mu \partial_\mu + m) \psi(x) = (\cancel \partial + m ) \psi(x) = 0 ~,
    \end{equation}
    where $\psi(t, \mathbf x)$ is a matrix 
    \begin{equation*}
        \psi(x) = \begin{bmatrix} \psi_1(x) \\ \psi_2(x) \\ \psi_3(x) \\ \psi_4(x) \\ \end{bmatrix} ~,
    \end{equation*}
    and $\gamma^\mu$ are the matrices 
    \begin{equation*}
        \gamma^0 = - i \beta ~, \quad \gamma^i = - i \beta \alpha^i ~,
    \end{equation*}
    such that they satisfy the Clifford algebra
    \begin{equation}\label{cliffordrel}
        \{\gamma^\mu, \gamma^\nu\} = 2 \eta^{\mu\nu} ~.
    \end{equation}
    Explicitly, they are 
    \begin{equation*}
        \gamma^0 = - i \begin{bmatrix} \mathbb I_2 & 0 \\ 0 & \mathbb I_2 \end{bmatrix} ~, \quad \gamma^i = \begin{bmatrix} 0 & - i \sigma^i \\ i \sigma^i & 0 \\ \end{bmatrix} ~.
    \end{equation*}
    \begin{proof}
        Infact, we compute operator substitution~\eqref{sub} on~\eqref{guess} 
        \begin{equation*}
            \underbrace{E}_{i\hbar \pdv{}{t}} \psi(t, \mathbf x) = (c \underbrace{\mathbf p}_{ - i \hbar \boldsymbol \nabla} \cdot \boldsymbol \alpha + m c^2 \beta) \psi(t, \mathbf x) ~.
        \end{equation*}
        Hence
        \begin{equation*}
            i \hbar \pdv{}{t} \psi (t, \mathbf x) = (- i \hbar c \boldsymbol \alpha \cdot \boldsymbol \nabla + \beta m c^2) \psi (t, \mathbf x) = H_D \psi(t, \mathbf x) ~.
        \end{equation*} 
        In order to write it in covariant form, we compute 
        \begin{equation*}
        \begin{aligned}
            \frac{\beta}{\cancel{\hbar} c} i \cancel{\hbar} \pdv{}{t} \psi (t, \mathbf x) & = \frac{\beta}{\hbar c}(- i \hbar c \boldsymbol \alpha \cdot \boldsymbol \nabla + \beta m c^2) \psi (t, \mathbf x) \\ & = \Big ( - \frac{\beta}{\cancel{\hbar} \cancel{c}} i \cancel{\hbar} \cancel{c} \boldsymbol \alpha \cdot \boldsymbol \nabla + \frac{m c}{\hbar} \underbrace{\beta^2}_1 \Big ) \psi (t, \mathbf x) ~.
        \end{aligned}
        \end{equation*}
        Hence 
        \begin{equation*}
            i \frac{\beta}{c} \pdv{}{t} \psi (t, \mathbf x) = \Big ( - i \beta \boldsymbol \alpha \cdot \boldsymbol \nabla + \frac{m c}{\hbar} \Big ) \psi (t, \mathbf x) ~,
        \end{equation*}
        \begin{equation*}
            \Big ( \underbrace{- i \beta}_{\gamma^0} \frac{1}{c} \pdv{}{t} \underbrace{- i \beta \boldsymbol \alpha}_{\boldsymbol \gamma} \cdot \boldsymbol \nabla + \frac{m c}{\hbar} \Big ) \psi (t, \mathbf x) = 0 ~,
        \end{equation*}
        \begin{equation*}
            \Big ( \gamma^0 \frac{1}{c} \pdv{}{t} + \boldsymbol \gamma \cdot \boldsymbol \nabla + \frac{m c}{\hbar} \Big ) \psi (t, \mathbf x) = 0 ~, 
        \end{equation*}
        and in covariant form, we obtain 
        \begin{equation*}
            (\gamma^\mu \partial_\mu + \mu ) \psi(x) = 0 ~,
        \end{equation*}
        where $\mu = \frac{m c}{\hbar}$ is the inverse reduced Compton wavelength. In natural units, it becomes 
        \begin{equation*}
            (\gamma^\mu \partial_\mu + m) \psi(x) = 0 ~.
        \end{equation*}

        Finally, they satisfy the Clifford algebra 
        \begin{equation*}
            \{\gamma^0, \gamma^0\} = 2 \eta^{00} = 2 ~,
        \end{equation*}
        \begin{equation*}
            \{\gamma^i, \gamma^j\} = 2 \eta^{ij} = 2 \delta^{ij} 
        \end{equation*}
        and 
        \begin{equation*}
            \{\gamma^0, \gamma^i\} = 2 \eta^{0i} = 0 ~.
        \end{equation*}
    \end{proof}

\section{Continuity equation}
    
    The continuity equation associated to the Dirac equation is 
    \begin{equation*}
        \pdv{}{t} (\psi^\dagger \psi) + \boldsymbol \nabla \cdot (c \psi^\dagger \boldsymbol \alpha \psi) = 0
    \end{equation*}
    where the density charge is positive defined $\psi^\dagger \psi > 0$ and it's compatible with the probabilistic intepretation.
    \begin{proof}
        Infact, we multiply by $\psi^\dagger$ and subtract the hermitian conjugate on~\eqref{hamdirac} 
        \begin{equation*}
        \begin{aligned}
            0 & = \psi^\dagger (i \hbar \pdv{}{t} \psi - (- i \hbar c \boldsymbol \alpha \cdot \boldsymbol \nabla + m c^2 \beta) \psi) - (\psi^\dagger (i \hbar \pdv{}{t} \psi - (- i \hbar c \boldsymbol \alpha \cdot \boldsymbol \nabla + m c^2 \beta) \psi))^\dagger \\ & = \psi^\dagger (i \hbar \pdv{}{t} \psi - (- i \hbar c \boldsymbol \alpha \cdot \boldsymbol \nabla + m c^2 \beta) \psi) - (- i \hbar \pdv{}{t} \psi^\dagger - (i \hbar c \underbrace{\boldsymbol \alpha^\dagger}_{\boldsymbol \alpha} \cdot \boldsymbol \nabla + m c^2 \underbrace{\beta^\dagger}_\beta) \psi^\dagger) \psi \\ & = i \hbar \psi^\dagger \pdv{}{t} \psi + i \hbar c \psi^\dagger \boldsymbol \alpha \cdot \boldsymbol \nabla \psi - \cancel{m c^2 \beta \psi^\dagger \psi }+ i \hbar \psi \pdv{}{t} \psi^\dagger + i \hbar c \psi \boldsymbol \alpha \cdot \boldsymbol \nabla \psi^\dagger + \cancel{m c^2 \beta \psi^\dagger \psi} \\ & = \cancel{i \hbar} \underbrace{\Big( \psi^\dagger \pdv{}{t} \psi + \psi \pdv{}{t} \psi^\dagger \Big)}_{\pdv{}{t} (\psi^\dagger \psi)} + \cancel{i \hbar} \underbrace{(c \psi^\dagger \boldsymbol \alpha \cdot \boldsymbol \nabla \psi  + \psi \boldsymbol \alpha \cdot \boldsymbol \nabla \psi^\dagger)}_{\boldsymbol \nabla \cdot (c \psi^\dagger \boldsymbol \alpha \psi)} \\ & = \pdv{}{t} (\psi^\dagger \psi) + \boldsymbol \nabla \cdot (c \psi^\dagger \boldsymbol \alpha \psi)  ~.
        \end{aligned}
        \end{equation*}
    \end{proof}

\section{Gamma matrices}

    As we said before, $\alpha^i$ and $\beta$ are hermitian whicle $\gamma^0$ is antihermitian and $\gamma^i$ is hermitian 
    \begin{equation*}
        (\gamma^0)^\dagger = - \gamma^0 ~, \quad (\gamma^i)^\dagger = \gamma^i ~.
    \end{equation*}
    which can be written in the following way 
    \begin{equation}\label{betagamma}
        (\gamma^\mu)^\dagger = \gamma^0 \gamma^\mu \gamma^0 = - \beta \gamma^\mu \beta ~.
    \end{equation}
    This means that the Clifford algebra is valid for $(\gamma^\mu)^\dagger$, intepreted as a change of basis 
    \begin{equation*}
        \{- (\gamma^\mu)^\dagger, - (\gamma^\nu)^\dagger \} = 2 \eta^{\mu\nu} ~.
    \end{equation*}
    \begin{proof}
        Infact, by the hermiticity of $\alpha^i$ and $\beta$
        \begin{equation*}
            (\gamma^0)^\dagger = (- i \beta)^\dagger = i \beta = - \gamma^0
        \end{equation*}
        and 
        \begin{equation*}
        \begin{aligned}
            (\gamma^i)^\dagger & = (- i \beta \alpha^i)^\dagger  = (\gamma^0 \alpha^i)^\dagger = \alpha^i \underbrace{(\gamma^0)^\dagger}_{\gamma^0} \\ & = - \alpha^i \gamma^0 = - i \underbrace{\alpha^i \beta}_{\beta \alpha^i} = - i \beta \alpha^i = \gamma^i ~.
        \end{aligned}
        \end{equation*}
        
        Furthermore,
        \begin{equation*}
            (\gamma^0)^\dagger = \underbrace{\gamma^0 \gamma^0}_{-1} \gamma^0 = - \gamma^0
        \end{equation*}
        and 
        \begin{equation*}
            (\gamma^i)^\dagger = \gamma^0 \underbrace{\gamma^i \gamma^0}_{-\gamma^0 \gamma^i} = - \gamma^0 \gamma^0 \gamma^i = - \underbrace{(\gamma^0)^2}_{-1} \gamma^i = \gamma^i ~.
        \end{equation*}

        Finally, using~\eqref{cliffordrel}
        \begin{equation*}
        \begin{aligned}
            \{- (\gamma^\mu)^\dagger, - (\gamma^\nu)^\dagger \} & = (\gamma^\mu)^\dagger (\gamma^\nu)^\dagger + (\gamma^\nu)^\dagger (\gamma^\mu)^\dagger \\ & = \gamma^0 \gamma^\mu \underbrace{\gamma^0 \gamma^0}_{-1} \gamma^\nu \gamma^0 + \gamma^0 \gamma^\nu \underbrace{\gamma^0 \gamma^0}_{-1} \gamma^\mu \gamma^0 \\ & = - \underbrace{\gamma^0 \gamma^\mu}_{- \gamma^\mu \gamma^0} \underbrace{\gamma^\nu \gamma^0}_{- \gamma^0 \gamma^\nu} - \underbrace{\gamma^0 \gamma^\nu}_{- \gamma^\nu \gamma^0} \underbrace{\gamma^\mu \gamma^0}_{- \gamma^0 \gamma^\mu} \\ & = - \gamma^\mu \underbrace{\gamma^0 \gamma^0}_{-1} \gamma^\nu - \gamma^\nu \underbrace{\gamma^0 \gamma^0}_{-1} \gamma^\mu = \gamma^\mu \gamma^\nu + \gamma^\nu \gamma^\mu = 2 \eta^{\mu\nu} ~.
        \end{aligned}
        \end{equation*}
    \end{proof}

    As we said before, $\alpha^i$ and $\beta$ are traceless and so $\gamma^\mu$ are
    \begin{equation*}
        \tr \gamma^\mu = 0 ~.
    \end{equation*}
    \begin{proof}
        Infact, by the linearity and cyclic property of the trace and~\eqref{clifford}
        \begin{equation*}
            \tr \gamma^0 = \tr (- i \beta) = - i \underbrace{\tr \beta}_0 = 0
        \end{equation*}
        and 
        \begin{equation*}
        \begin{aligned}
            \tr (\gamma^i) & = \tr (\mathbb I \gamma^i) = \tr ((\gamma^j)^2 \gamma^i) = \tr (\gamma^j \underbrace{\gamma^j \gamma^i}_{- \gamma^i \gamma^j}) \\ & = - \tr (\gamma^j \gamma^i \gamma^j) = - \tr (\gamma^i \gamma^j \gamma^j) = - \tr (\gamma^i (\gamma^j)^2) = - \tr (\gamma^i) ~.
        \end{aligned}
        \end{equation*}
    \end{proof}

\subsection{$\gamma^5$}

    We introduce another matrix $\gamma^5$, called the chirality matrix
    \begin{equation*}
        \gamma^5 = - i \gamma^0 \gamma^1 \gamma^2 \gamma^3
    \end{equation*}
    such that it satisfies the gamma-matrix properties
    \begin{enumerate}
        \item anticommutator, i.e.
            \begin{equation*}
                \{\gamma^5, \gamma^\mu\} = 0 ~,
            \end{equation*}
        \item the square is the identity, i.e.
            \begin{equation}\label{sqgamma}
                (\gamma^5)^2 = \mathbb I ~,
            \end{equation}
        \item hermiticity, i.e.
            \begin{equation*}
                (\gamma^5)^\dagger = \gamma^5 ~,
            \end{equation*}
        \item traceless, i.e.
            \begin{equation*}
                \tr(\gamma^5) = 0 ~.
            \end{equation*}
    \end{enumerate}
    \begin{proof}
        For the anticommutator property 
        \begin{equation*}
        \begin{aligned}
            \{\gamma^5, \gamma^\mu\} = \gamma^5 \gamma^\mu + \gamma^\mu \gamma^5 = - i \gamma^0 \gamma^1 \gamma^2 \gamma^3 \gamma^\mu - i \gamma^\mu \gamma^0 \gamma^1 \gamma^2 \gamma^3  ~.
        \end{aligned}
        \end{equation*}
        Now, consider $\mu=0$ 
        \begin{equation*}
        \begin{aligned}
            \{\gamma^5, \gamma^0\} & = - i \gamma^0 \gamma^1 \gamma^2 \underbrace{\gamma^3 \gamma^0}_{-\gamma^0 \gamma^3} - i \underbrace{\gamma^0 \gamma^0}_{-1} \gamma^1 \gamma^2 \gamma^3 \\ & = i \gamma^0 \gamma^1 \underbrace{\gamma^2 \gamma^0}_{- \gamma^0 \gamma^2} \gamma^3 + i \gamma^1 \gamma^2 \gamma^3 \\ & = - i \gamma^0 \underbrace{\gamma^1 \gamma^0}_{- \gamma^0 \gamma^1} \gamma^2 \gamma^3 + i \gamma^1 \gamma^2 \gamma^3 \\ & = i \underbrace{\gamma^0 \gamma^0 }_{-1} \gamma^1 \gamma^2 \gamma^3 + i \gamma^1 \gamma^2 \gamma^3 \\ & = - i \gamma^1 \gamma^2 \gamma^3 + i \gamma^1 \gamma^2 \gamma^3 = 0
        \end{aligned}
        \end{equation*}
        and similarly for $\mu = 1,2,3$.

        For the square property 
        \begin{equation*}
            (\gamma^5)^2 = (-i\gamma^0 \gamma^1 \gamma^2 \gamma^3)^2 = - \underbrace{(\gamma^0)^2}_{- \mathbb I} \underbrace{(\gamma^1)^2}_{\mathbb I} \underbrace{(\gamma^2)^2}_{\mathbb I} \underbrace{(\gamma^3)^2}_{\mathbb I} = \mathbb I ~.
        \end{equation*}

        For the hermiticity property 
        \begin{equation*}
        \begin{aligned}
            (\gamma^5)^\dagger & = (-i \gamma^0 \gamma^1 \gamma^2 \gamma^3)^\dagger \\ & = i (\gamma^3)^\dagger (\gamma^2)^\dagger (\gamma^1)^\dagger (\gamma^0)^\dagger \\ & = i \gamma^0 \gamma^3 \underbrace{\gamma^0 \gamma^0}_{-1}\gamma^2 \underbrace{\gamma^0 \gamma^0}_{-1} \gamma^1 \underbrace{\gamma^0 \gamma^0}_{-1} \underbrace{\gamma^0 \gamma^0}_{-1} \\ & = i \gamma^0 \gamma^3 \underbrace{\gamma^2 \gamma^1}_{\gamma^1 \gamma^2} \\ & = - i \gamma^0 \underbrace{\gamma^3 \gamma^1}_{-\gamma^3 \gamma^1} \gamma^2 \\ & = i \gamma^0 \gamma^1 \underbrace{\gamma^3 \gamma^2}_{\gamma^2 \gamma^3} \\ & = - i \gamma^0 \gamma^1 \gamma^2 \gamma^3 = \gamma^5 ~.
        \end{aligned}
        \end{equation*}

        For the traceless property 
        \begin{equation*}
        \begin{aligned}
            \tr (\gamma^5) & = \tr(-i \gamma^0 \gamma^1 \gamma^2 \gamma^3) \\ & = - i \tr (\underbrace{\gamma^0 \gamma^1}_{- \gamma^1 \gamma^0} \gamma^2 \gamma^3)\\ &  = i \tr(\gamma^1 \underbrace{\gamma^0 \gamma^2}_{- \gamma^2 \gamma^0} \gamma^3)\\ &  = - i \tr (\gamma^1 \gamma^2 \underbrace{\gamma^0 \gamma^3}_{- \gamma^3 \gamma^0}) \\ & = i \tr (\gamma^1 \gamma^2 \gamma^3 \gamma^0) \\ & = i \tr ( \gamma^0 \gamma^1 \gamma^2 \gamma^3) \\ & = - \tr (\gamma^5) ~,
        \end{aligned}        
        \end{equation*}
        where we have used the cyclic property of the trace.
    \end{proof}

    Explicitly, in the Dirac representation, it becomes
    \begin{equation*}
        \gamma^5 = \begin{bmatrix}
            0 & - \mathbb I_2 \\ - \mathbb I_2 & 0 \\
        \end{bmatrix} ~.
    \end{equation*}
    \begin{proof}
        Infact 
        \begin{equation*}
        \begin{aligned}
            \gamma^5 & = - i \gamma^0 \gamma^1 \gamma^2 \gamma^3 = (-i)^5 \begin{bmatrix}
                \mathbb I_2 & 0 \\ 0 & - \mathbb I_2 \\
            \end{bmatrix} \begin{bmatrix}
                0 & \sigma^1 \\ - \sigma^1 & 0 \\
            \end{bmatrix} \begin{bmatrix}
                0 & \sigma^2 \\ - \sigma^2 & 0 \\
            \end{bmatrix}\begin{bmatrix}
                0 & \sigma^3 \\ - \sigma^3 & 0 \\
            \end{bmatrix} \\ & = -i \begin{bmatrix}
                \mathbb I_2 & 0 \\ 0 & - \mathbb I_2 \\
            \end{bmatrix} \begin{bmatrix}
                0 & - \sigma^1 \sigma^2 \\ - \sigma^1 \sigma^2 & 0 \\
            \end{bmatrix} \begin{bmatrix}
                0 & \sigma^3 \\ - \sigma^3 & 0 \\
            \end{bmatrix} \\ & = -i \begin{bmatrix}
                \mathbb I_2 & 0 \\ 0 & - \mathbb I_2 \\
            \end{bmatrix} \begin{bmatrix}
                0 & - \underbrace{\sigma^1 \sigma^2}_{i \sigma^3} \sigma^3 \\ \underbrace{\sigma^1 \sigma^2}_{i \sigma^3} \sigma^3 & 0 \\
            \end{bmatrix} = \begin{bmatrix}
                \mathbb I_2 & 0 \\ 0 & - \mathbb I_2 \\
            \end{bmatrix} \begin{bmatrix}
                0 & - \underbrace{(\sigma^3)^2 }_{\mathbb I_2}\\ \underbrace{(\sigma^3)^2 }_{\mathbb I_2} & 0 \\
            \end{bmatrix} \\ & = \begin{bmatrix}
                \mathbb I_2 & 0 \\ 0 & - \mathbb I_2 \\
            \end{bmatrix} \begin{bmatrix}
                0 & \mathbb I_2 \\ - \mathbb I_2 & 0 \\
            \end{bmatrix} = \begin{bmatrix}
                0 & -\mathbb I_2 \\ - \mathbb I_2 & 0 \\
            \end{bmatrix} ~.
        \end{aligned}
        \end{equation*}
    \end{proof}

    It adds another dimension to the $4$-dimensional Clifford algebra. Infact, we can define a $5$-dimensional Clifford algebra by the anticommutator relations
    \begin{equation*}
        \{\gamma^M, \gamma^N\} = 2 \eta^{MN} ~,
    \end{equation*}
    where $M, N = 0,1,2,3,5$ with Minkovski metric $\eta^{MN} = diag(-++++)$.

    It can be used to define the projection operators on chiral (Weyl) spinors
    \begin{equation}\label{proj}
        P_L = \frac{\mathbb I - \gamma^5}{2} ~, \quad P_L = \frac{\mathbb I - \gamma^5}{2} ~,
    \end{equation}
    such that they satisfy the following properties 
    \begin{enumerate}
        \item nilpotent, i.e. 
            \begin{equation*}
                P_L^2 = P_L ~, \quad P_R^2 = P_R ~,
            \end{equation*}
        \item orthogonality, i.e. 
            \begin{equation*}
                P_L P_R = 0 ~, \quad P_L + P_R = \mathbb I~.
            \end{equation*}
    \end{enumerate}
    \begin{proof}
        For the nilpotent property, using~\eqref{sqgamma}
        \begin{equation*}
            P_L^2 = \Big ( \frac{\mathbb I - \gamma^5}{2} \Big)^2 = \frac{1}{4} (\mathbb I^2 - 2 \gamma^5 + \underbrace{(\gamma^5)^2}_{\mathbb I} ) = \frac{2 \mathbb I - 2 \gamma^5}{4} = \frac{\mathbb I - \gamma^5}{2} = P_L
        \end{equation*}
        and 
        \begin{equation*}
            P_R^2 = \Big ( \frac{\mathbb I + \gamma^5}{2} \Big)^2 = \frac{1}{4} (\mathbb I^2 + 2 \gamma^5 + \underbrace{(\gamma^5)^2}_{\mathbb I} ) = \frac{2 \mathbb I + 2 \gamma^5}{4} = \frac{\mathbb I + \gamma^5}{2} = P_R ~.
        \end{equation*}

        For the orthogonality property, using~\eqref{sqgamma}
        \begin{equation*}
            P_L P_R = \Big ( \frac{\mathbb I - \gamma^5}{2} \Big)  \Big ( \frac{\mathbb I + \gamma^5}{2} \Big) = \frac{1}{4} (\mathbb I^2 - \underbrace{(\gamma^5)^2}_{\mathbb I} ) = \frac{\mathbb I - \mathbb I}{4} = 0
        \end{equation*}
        and
        \begin{equation*}
            P_L + P_R = \Big ( \frac{\mathbb I - \cancel{\gamma^5}}{2} \Big) + \Big ( \frac{\mathbb I + \cancel{\gamma^5}}{2} \Big) = \frac{\mathbb I + \mathbb I}{2} = \mathbb I ~.
        \end{equation*}
    \end{proof}
    They allow to divide a Dirac spinor into two components $\psi = \psi_L + \psi_R$, where $\psi_L = P_L \psi$ is the left-handed one and $\psi_R = P_R \psi$ is the right-handed one. Infact a Dirac spinor can be written as $(\frac{1}{2}, 0) \oplus (0, \frac{1}{2})$. Spinors live in a $4$-dimensional complex linear space $\psi(x) \in \mathbb C^4$. Therefore, the gamma matrices are an example of $4$-dimensional matrices act on this space. A complete basis of linear operators must have $16$ of them and we can choose $(\mathbb I, \gamma^5, \Sigma^{\mu\nu}, \gamma^\mu \gamma^5, \gamma^5)$ where $\Sigma^{\mu\nu} = - \frac{i}{4} [\gamma^\mu, \gamma^\nu]$ with $\mu > \nu$. They are indeed respectively $1 + 4 + 6 + 4 + 1 = 16$ linearly independent matrices. 

\chapter{Non-relativistic limit} 

    In order to study the non-relativistic limit for $c \rightarrow \infty$, we insert back the $\hbar$ and $c$. 

\section{Free Pauli equation}
    
    Starting from~\eqref{hamdirac}, we decompose the Dirac spinor wave function into two spinor wave functions 
    \begin{equation}\label{nonrel}
        \psi(t, \mathbf x) = \begin{bmatrix}
            \varphi(t, \mathbf x) \\ \chi (t, \mathbf x) \\
        \end{bmatrix} \exp(- \frac{i}{\hbar} m c^2 t)
    \end{equation}
    where we have brought out the time dependence from the energy-mass (similarly to the rest-frame wave plane).

    In the non-relativistic limit, we recove the free Pauli equation 
    \begin{equation*}
        i \hbar \pdv{}{t} \varphi = \frac{p^2}{2m} \varphi
    \end{equation*}
    where $\varphi$ is a $2$-dimensional spinor, bringing information about the spin of the particle.
    \begin{proof}
        We insert~\eqref{nonrel} into~\eqref{hamdirac}
        \begin{equation*}
            i \hbar \pdv{}{t} (\begin{bmatrix}
                \varphi \\ \chi \\
            \end{bmatrix} \exp(- \frac{i}{\hbar} m c^2 t)) = (c \boldsymbol \alpha \cdot \mathbf p + m c^2 \beta) \begin{bmatrix}
                \varphi \\ \chi \\
            \end{bmatrix} \exp(- \frac{i}{\hbar} m c^2 t) ~,
        \end{equation*}
        \begin{equation*}
            i \hbar (\begin{bmatrix}
                \dot \varphi \\ \dot \chi \\
            \end{bmatrix} - \frac{i}{\hbar} m c^2 \begin{bmatrix}
                \varphi \\ \chi \\
            \end{bmatrix} ) \cancel{\exp(- \frac{i}{\hbar} m c^2 t)} = (c \begin{bmatrix}
                0 & \boldsymbol \sigma \cdot \mathbf p \\
                \boldsymbol \sigma \cdot \mathbf p & 0 \\
            \end{bmatrix} + m c^2 \begin{bmatrix}
                \mathbb I_2 & 0 \\ 0 & -\mathbb I_2 \\
            \end{bmatrix}) \begin{bmatrix}
                \varphi \\ \chi \\
            \end{bmatrix} \cancel{\exp(- \frac{i}{\hbar} m c^2 t)} ~,
        \end{equation*}
        \begin{equation*}
            \begin{bmatrix}
                i \hbar \dot \varphi + m c^2 \varphi \\ i \hbar \dot \chi + m c^2 \chi \\
            \end{bmatrix} = c \begin{bmatrix}
                \boldsymbol \sigma \cdot \mathbf p \chi\\
                \boldsymbol \sigma \cdot \mathbf p \varphi \\
            \end{bmatrix} + \begin{bmatrix}
                m c^2 \varphi \\ - m c^2 \chi \\
            \end{bmatrix} ~.
        \end{equation*}
        This is a system of $2$ equations 
        \begin{equation*}
            \begin{cases}
                i \hbar \dot \varphi + \cancel{m c^2 \varphi} = c \boldsymbol \sigma \cdot \mathbf p \chi + \cancel{m c^2 \varphi} \\
                i \hbar \dot \chi + m c^2 \chi = c \boldsymbol \sigma \cdot \mathbf p \varphi - m c^2 \chi \\
            \end{cases} ~,
        \end{equation*}
        \begin{equation*}
            \begin{cases}
                i \hbar \dot \varphi = c \boldsymbol \sigma \cdot \mathbf p \chi \\
                i \hbar \dot \chi + 2m c^2 \chi = c \boldsymbol \sigma \cdot \mathbf p \varphi \\
            \end{cases} ~.
        \end{equation*}

        Now, we go into the non-relativistic limit for $c \rightarrow \infty$ which means that $i \hbar \dot \chi \ll 2 m c^2 \chi$ and we obtain 
        \begin{equation*}
            \begin{cases}
                i \hbar \dot \varphi = c \boldsymbol \sigma \cdot \mathbf p \chi \\
                \underbrace{\cancel{i \hbar \dot \chi}}_{c \rightarrow \infty} + 2m c^2 \chi = c \boldsymbol \sigma \cdot \mathbf p \varphi \\
            \end{cases} \quad \Rightarrow \quad \begin{cases}
                i \hbar \dot \varphi = c \boldsymbol \sigma \cdot \mathbf p \chi \\
                2m c^2 \chi = c \boldsymbol \sigma \cdot \mathbf p \varphi \\
            \end{cases} ~.
        \end{equation*}
        We solve this algebraic equation, starting with the second 
        \begin{equation*}
            \chi = \frac{\boldsymbol \sigma \cdot \mathbf p}{2 m c} \varphi
        \end{equation*}
        and putting into the first
        \begin{equation*}
            i \hbar \dot \varphi = c \boldsymbol \sigma \cdot \mathbf p \chi = \cancel{c} \boldsymbol \sigma \cdot \mathbf p \frac{\boldsymbol \sigma \cdot \mathbf p}{2 m \cancel{c}} \varphi = \frac{(\boldsymbol \sigma \cdot \mathbf p)^2}{2m} \varphi ~.
        \end{equation*}
        We notice that, using~\eqref{sigma}
        \begin{equation*}
            (\boldsymbol \sigma \cdot \mathbf p)^2 = \sigma^i p^i \sigma^j p^j = p^i p^j \sigma^i \sigma^j = p^i p^j (\delta^{ij} + i \epsilon^{ijk} \sigma^k) = p^i p^j \delta^{ij} + i \cancel{\underbrace{p^i p^j}_{\textnormal{symm}}\underbrace{\epsilon^{ijk}}_{\textnormal{anti}}} \sigma^k = p^i p^i ~,
        \end{equation*}
        and we obtain the free Pauli equation 
        \begin{equation*}
            i \hbar \dot \varphi = \frac{(\boldsymbol \sigma \cdot \mathbf p)^2}{2m} \varphi = \frac{p^2}{2m} \varphi ~. 
        \end{equation*}
    \end{proof}

\section{External electromagnetic field}

    We study how particles, in this case an electron with charge $e < 0$, couple with the electromagnetic field through the minimal substitution 
    \begin{equation*}
        E \mapsto E - q \phi ~, \quad \mathbf p \mapsto \boldsymbol \pi = \mathbf p - \frac{e}{c} \mathbf A
    \end{equation*}
    or in covariant form 
    \begin{equation}\label{minsub}
        p_\mu \mapsto \pi_\mu = p_\mu - \frac{e}{c} A_\mu  ~,
    \end{equation}
    where the $4$-momentum is $p^\mu = (\frac{E}{c}, \mathbf p)$ and the $4$-potential is $A^\mu = (\phi, \mathbf A)$. In this way we are also able to study the effects due to the spin.

    The non-relativistic Dirac equation for an electron is
    \begin{equation*}
        i \hbar \dot \varphi = \Big ( \frac{\pi^2}{2m} - \frac{e}{mc} \mathbf B \cdot \mathbf S + e \phi \Big) \varphi
    \end{equation*}
    where $\mathbf S = \frac{\hbar}{2} \boldsymbol \sigma$ is the spin of the electron.
    \begin{proof}
        We use the minimal substitution in~\eqref{hamdirac}
        \begin{equation*}
            i \hbar \pdv{}{t} \psi(t, \mathbf x) = (c \boldsymbol \alpha \cdot \mathbf p + \beta m c^2) \psi(t, \mathbf x) \mapsto (c \boldsymbol \alpha \cdot \boldsymbol \pi + \beta m c^2 + e \phi) \psi(t, \mathbf x) ~,
        \end{equation*}
        where 
        \begin{equation*}
            \boldsymbol \pi = - i \hbar \boldsymbol \nabla - \frac{e}{c} \mathbf A = - i \hbar \Big ( \boldsymbol \nabla - \frac{ie}{\hbar c} \mathbf A \Big)
        \end{equation*}
        is the covariant derivative, covariant because of the gauge transformation.

        Using the same procedure as before, we decompose $\psi$ with~\eqref{nonrel} 
        \begin{equation*}
            i \hbar \pdv{}{t} (\begin{bmatrix}
                \varphi \\ \chi \\
            \end{bmatrix} \exp(- \frac{i}{\hbar} m c^2 t)) = (c \boldsymbol \alpha \cdot \boldsymbol \pi + m c^2 \beta + e \phi) \begin{bmatrix}
                \varphi \\ \chi \\
            \end{bmatrix} \exp(- \frac{i}{\hbar} m c^2 t) ~,
        \end{equation*}
        \begin{equation*}
            i \hbar (\begin{bmatrix}
                \dot \varphi \\ \dot \chi \\
            \end{bmatrix} - \frac{i}{\hbar} m c^2 \begin{bmatrix}
                \varphi \\ \chi \\
            \end{bmatrix} ) \cancel{\exp(- \frac{i}{\hbar} m c^2 t)} = (c \begin{bmatrix}
                0 & \boldsymbol \sigma \cdot \boldsymbol \pi \\
                \boldsymbol \sigma \cdot \boldsymbol \pi & 0 \\
            \end{bmatrix} + m c^2 \begin{bmatrix}
                \mathbb I_2 & 0 \\ 0 & -\mathbb I_2 \\
            \end{bmatrix} + e \phi) \begin{bmatrix}
                \varphi \\ \chi \\
            \end{bmatrix} \cancel{\exp(- \frac{i}{\hbar} m c^2 t)} ~,
        \end{equation*}
        \begin{equation*}
            \begin{bmatrix}
                i \hbar \dot \varphi + m c^2 \varphi \\ i \hbar \dot \chi + m c^2 \chi \\
            \end{bmatrix} = c \begin{bmatrix}
                \boldsymbol \sigma \cdot \boldsymbol \pi \chi\\
                \boldsymbol \sigma \cdot \boldsymbol \pi \varphi \\
            \end{bmatrix} + \begin{bmatrix}
                (m c^2 + e \phi) \varphi\\ (-m c^2 + e \phi) \chi \\
            \end{bmatrix} ~.
        \end{equation*}
        This is a system of $2$ equations 
        \begin{equation*}
            \begin{cases}
                i \hbar \dot \varphi + \cancel{m c^2 \varphi} = c \boldsymbol \sigma \cdot \boldsymbol \pi \chi + \cancel{m c^2 \varphi} + e \phi \varphi \\
                i \hbar \dot \chi + m c^2 \chi = c \boldsymbol \sigma \cdot \boldsymbol \pi \varphi - m c^2 \chi + e \phi \chi \\
            \end{cases} ~,
        \end{equation*}
        \begin{equation*}
            \begin{cases}
                i \hbar \dot \varphi = c \boldsymbol \sigma \cdot \boldsymbol \pi \chi + e \phi \varphi\\
                i \hbar \dot \chi + 2m c^2 \chi = c \boldsymbol \sigma \cdot \boldsymbol \pi \varphi + e \phi \chi\\
            \end{cases} ~.
        \end{equation*}

        Now, we go into the non-relativistic limit for $c \rightarrow \infty$ which means that $i \hbar \dot \chi \ll 2 m c^2 \chi$ and $e \phi \chi \ll 2 m c^2 \chi$, thus we obtain 
        \begin{equation*}
            \begin{cases}
                i \hbar \dot \varphi = c \boldsymbol \sigma \cdot \boldsymbol \pi \chi + e \phi \varphi\\
                \underbrace{\cancel{i \hbar \dot \chi}}_{c \rightarrow \infty} + 2m c^2 \chi = c \boldsymbol \sigma \cdot \boldsymbol \pi \varphi + \underbrace{\cancel{e \phi \chi}}_{c \rightarrow \infty}\\
            \end{cases} \quad \Rightarrow \quad \begin{cases}
                i \hbar \dot \varphi = c \boldsymbol \sigma \cdot \boldsymbol \pi \chi + e \phi \varphi\\
                2m c^2 \chi = c \boldsymbol \sigma \cdot \boldsymbol \pi \varphi\\
            \end{cases} ~.
        \end{equation*}
        We solve this algebraic equation, starting with the second 
        \begin{equation*}
            \chi = \frac{\boldsymbol \sigma \cdot \boldsymbol \pi}{2 m c} \varphi
        \end{equation*}
        and putting into the first
        \begin{equation*}
            i \hbar \dot \varphi = c \boldsymbol \sigma \cdot \boldsymbol \pi \chi + e \phi \varphi = \cancel{c} \boldsymbol \sigma \cdot \boldsymbol \pi \frac{\boldsymbol \sigma \cdot \boldsymbol \pi}{2 m \cancel{c}} \varphi + e \phi \varphi = \frac{(\boldsymbol \sigma \cdot \boldsymbol \pi)^2}{2m} \varphi + e \phi \varphi ~.
        \end{equation*}
        We notice that, using~\eqref{sigma}
        \begin{equation*}
        \begin{aligned}
            (\boldsymbol \sigma \cdot \boldsymbol \pi)^2 & = \sigma^i \pi^i \sigma^j \pi^j \\ & = \pi^i \pi^j \sigma^i \sigma^j \\ & = \pi^i \pi^j (\delta^{ij} + i \epsilon^{ijk} \sigma^k) \\ & = \pi^i \pi^j \delta^{ij} + i \epsilon^{ijk} \sigma^k \underbrace{\pi^i \pi^j}_{\frac{\pi^i \pi^j - \pi^j \pi^i}{2}} \\ & = \pi^2 + i \epsilon^{ijk} \sigma^k \frac{\pi^i \pi^j - \pi^j \pi^i}{2} \\ & = \pi^2 + \frac{i}{2} \epsilon^{ijk} \sigma^k [\pi^i, \pi^j]  ~,
        \end{aligned}
        \end{equation*}
        where we exploit the antisymmetry of $\epsilon^{ijk}$ to antisymmetrise $\pi^i \pi^j$. We compute the commutator
        \begin{equation*}
        \begin{aligned}
            [\pi^i, \pi^j] & = [p^i - \frac{e}{c} A^i, p^j - \frac{e}{c} A^j] \\ & = \underbrace{[p^i, p^j]}_0 - \frac{e}{c} [p^i, A^j] - \frac{e}{c} \underbrace{[A^i, p^j]}_{- [p^j, A^i]} + \frac{e^2}{c^2} \underbrace{[A^i(x), A^j(x)]}_0 \\ & = - \frac{e}{c} ([p^i, A^j] - [p^j, A^i]) \\ & = - \frac{e}{c} (p^i A^j - A^j p^i - p^j A^i + p^i A^j) \\ & = \frac{i \hbar e}{c} (\partial^i A^j - A^j \partial^i - \partial^j A^i + A^i \partial^j ) \\ & =\frac{i \hbar e}{c} (\partial^i A^j - \partial^j A^i) \\ & = \frac{2 i \hbar e}{c} \partial^i A^j
        \end{aligned}
        \end{equation*}
        where we used the canonical commutation relation, $[A^i(x), A^j(x)] = 0$ because $A(x)$ is a function of $x$,~\eqref{sub} and the fact that second partial derivatives commute. Hence 
        \begin{equation*}
        \begin{aligned}
            (\boldsymbol \sigma \cdot \boldsymbol \pi)^2 & = \pi^2 + \frac{i \hbar e}{2 c} i \epsilon^{ijk} \underbrace{(\partial^i A^j - \partial^j A^i)}_{2 \partial^i A^j} \\ & = \pi^2 + \frac{i \hbar e}{\cancel{2} c} \cancel{2} i \underbrace{\epsilon^{ijk} \partial^i A^j}_{B^k} \sigma^k \\ & = \pi^2 - \frac{\hbar e}{c} B^k \sigma^k \\ & = \pi^2 - \frac{2e}{c} \mathbf B \cdot \underbrace{\frac{1}{2} \boldsymbol \sigma}_{\mathbf S} \\ & = \pi^2 - \frac{2e}{c} \mathbf B \cdot \mathbf S
        \end{aligned}
        \end{equation*}
        where we exploit the antisymmetry of $\epsilon^{ijk}$ to antisymmetrise $\partial^i A^j$. Finally we obtain the non-relativistic Dirac equation for the electron
        \begin{equation*}
            i \hbar \dot \varphi = \Big ( \frac{1}{2m} (\pi^2 - \frac{2e}{mc} \mathbf B \cdot \mathbf S) + e \phi \Big) \varphi = \Big ( \frac{\pi^2}{2m} - \frac{e}{mc} \mathbf B \cdot \mathbf S + e \phi \Big) \varphi ~. 
        \end{equation*}
    \end{proof}

\subsection{Angular momentum and spin}

    Recall that a magnetic dipole with magnetic moment $\boldsymbol \mu$ couples with an external magnetic field $\mathbf B$ through the hamiltonian 
    \begin{equation*}
        H = - \boldsymbol \mu \cdot \mathbf B ~.
    \end{equation*}
    In particular, a charge $e$ in motion with angular momentum $\mathbf L$ has magnetic moment proportional to it 
    \begin{equation*}
        |\boldsymbol \mu| = \frac{e}{2mc} g |\mathbf L|
    \end{equation*}
    where the gyromagnetic factor $g = 1$ for the angular momentum of the electron and $g = 2$ for its spin.
    \begin{example}
        Consider a constant magnetic field $\mathbf B$ and its associated vector potential $\mathbf A = \frac{1}{2} \mathbf B \times \mathbf r$. Hence 
        \begin{equation*}
        \begin{aligned}
            \pi^2 & = (\mathbf p - \frac{e}{c} \mathbf A) (\mathbf p - \frac{e}{c} \mathbf A) \\ & = p^2 - \frac{e}{c} (\mathbf A \cdot \mathbf p + \mathbf p \cdot \mathbf A) + \frac{e^2}{c^2} A^2 \\ & = p^2 - \frac{2e}{c} (\mathbf A \cdot \mathbf p) + \frac{e^2}{c^2} A^2 \\ & = p^2 - \frac{2e}{c} (\frac{1}{2} \mathbf B \times \mathbf r \cdot \mathbf p) + \frac{e^2}{c^2} A^2 \\ & = p^2 - \frac{2e}{c} (\frac{1}{2} \mathbf B \cdot \underbrace{\mathbf r \times \mathbf p}_{\mathbf L}) + \frac{e^2}{c^2} A^2 \\ & = p^2 - \frac{2e}{c} (\frac{1}{2} \mathbf B \cdot \mathbf L) + \frac{e^2}{c^2} A^2 \\ & = p^2 - \frac{e}{c} \mathbf B \cdot \mathbf L + \frac{e^2}{c^2} A^2 ~,
        \end{aligned}
        \end{equation*}
        where we have used the fact that $\mathbf p \cdot \mathbf A$ commutes for constant $\mathbf B$ and the cyclic property for the mixed product. This means that the gyromagnetic factors associated to angular momentum and spin are
        \begin{equation*}
            i \hbar \dot \varphi = \Big ( \frac{\pi^2}{2m} - \frac{e}{2mc} \mathbf B \cdot (\underbrace{1}_{g=1}\mathbf L + \underbrace{2}_{g=2}\mathbf S) + \frac{e^2}{2 m c^2} A^2 + e \phi \Big) \varphi ~.
        \end{equation*}
    \end{example}

    By similarity with the non-relativistic limit $\mathbf S = \frac{1}{2} \boldsymbol \sigma$ and $\chi \propto \varphi$, the spin operator of the Dirac spinor in the Dirac representation is \begin{equation*}
        \mathbf S = \frac{1}{2} \boldsymbol \Sigma = \frac{1}{2} \begin{bmatrix}
            \boldsymbol \sigma & 0 \\ 
            0 & \boldsymbol \sigma \\ 
        \end{bmatrix} ~.
    \end{equation*}
    In index notation
    \begin{equation*}
        \Sigma^i = - \frac{i}{2} \epsilon^{ijk} \alpha^j \alpha^k ~.
    \end{equation*}

    Introducing the angular momentum $\mathbf L = \mathbf r \times \mathbf p$ and the total angular momentum $\mathbf J = \mathbf L + \mathbf S$, we have the following commutation relations
    \begin{equation*}
        [H_D, L^i] = - i \epsilon^{ijk} \alpha^j p^k ~,
    \end{equation*}
    \begin{equation*}
        [H_D, S^i] = - i \epsilon^{ijk} p^j \alpha^k
    \end{equation*}
    and
    \begin{equation*}
        [H_D, J^i] = 0 ~.
    \end{equation*}
    \begin{proof}
        For the angular momentum commutation relation,~\eqref{clifford}
        \begin{equation*}
        \begin{aligned}
            [H_D, L^i] & = [\alpha^l p_l + \beta m, \epsilon^{ijk} x^j p^k] \\ & = [\alpha^l p_l, \epsilon^{ijk} x^j p^k] + \underbrace{[\beta m, \epsilon^{ijk} x^j p^k]}_0 \\ & = [\alpha^l p_l, \epsilon^{ijk} x^j p^k] \\ & = \alpha^l \underbrace{[p_l,  x^j]}_{-i \delta_l^{\phantom l j}} \epsilon^{ijk} p^k + \alpha^l \underbrace{[p_l, p^k]}_0 \epsilon^{ijk} x^j \\ & = - i \alpha^l \delta_l^{\phantom l j} \epsilon^{ijk} p^k \\ & = - i \epsilon^{ijk} \alpha^j p^k ~.
        \end{aligned}
        \end{equation*}

        For the spin commutation relation,~\eqref{clifford}
        \begin{equation*}
        \begin{aligned}
            [H_D, S^i] & = [\alpha^l p_l + \beta m, -\frac{i}{4} \epsilon^{ijk} \alpha^j \alpha^k] \\ & = [\alpha^l p_l, -\frac{i}{4} \epsilon^{ijk} \alpha^j \alpha^k] + [\beta m, -\frac{i}{4} \epsilon^{ijk} \alpha^j \alpha^k] \\ & =  -\frac{i}{4} p_l \epsilon^{ijk} [\alpha^l, \alpha^j \alpha^k] -\frac{i}{4} m \epsilon^{ijk}  [\beta, \alpha^j \alpha^k] \\ & = -\frac{i}{4} p_l \epsilon^{ijk} (\underbrace{\{\alpha^l, \alpha^j\}}_{2\eta^{lj}} \alpha^k - \alpha^j \underbrace{\{\alpha^l, \alpha^k\}}_{2\eta^{lk}}) - \frac{i}{4} m \epsilon^{ijk} (\underbrace{\{\beta, \alpha^j\}}_0 \alpha^k - \alpha^j \underbrace{\{\beta, \alpha^k\}}_0) \\ & = -\frac{i}{4} p_l \epsilon^{ijk} (2\eta^{lj} \alpha^k - \alpha^j 2\eta^{lk}) \\ & = - i \epsilon^{ijk} p^j \alpha^k ~,
        \end{aligned}
        \end{equation*}
        where we exploit the antysymmetry of $\epsilon^{ijk}$ to antisymmetrise $\alpha^j \eta^{lk}$ and we have used the identity 
        \begin{equation*}
            [AB,C] = ABC - CAB = ABC + ACB - CAB - ACB = A \{B, C\} - \{A, C\} B ~.
        \end{equation*}

        For the total momentum commutation relation 
        \begin{equation*}
        \begin{aligned}
            [H_D, J^i] & = [H_D, L^i + S^i] = [H_D, L_i] + [H_D, S^i] = - i \epsilon^{ijk} \alpha^j p^k - i \epsilon^{ijk} p^j \alpha^k \\ & = - i \epsilon^{ijk} \alpha^j p^k + i \epsilon^{ikj} p^j \alpha^k = 0 ~.
        \end{aligned}
        \end{equation*}
    \end{proof}

    An experimental test for the Dirac theory is the prediction of the energy levels of the hydrogen atom, via perturbation solutions. Schoredinger solution is 
    \begin{equation*}
        E_{nl} = - \frac{m_e \alpha^2}{2 n^2} ~,
    \end{equation*}
    Klein-Gordon solution is 
    \begin{equation*}
        E_{nl} = m_e \Big( 1 - \frac{\alpha^2}{2n^2} - \frac{\alpha^4}{n^4} \Big ( \frac{n}{2l+1} - \frac{3}{8} \Big) + O(\alpha^6)\Big)
    \end{equation*}
    and Dirac solution is
    \begin{equation*}
        E_{nl} = m_e \Big( 1 - \frac{\alpha^2}{2n^2} - \frac{\alpha^4}{n^4} \Big ( \frac{n}{2j+1} - \frac{3}{8} \Big) + O(\alpha^6)\Big) ~.
    \end{equation*}
    where $\alpha = \frac{e^2}{4\pi}$ is the fine structure constant. Notice that the Schoredinger solution is degenerate in $l$, Klein-Gordon solution lifts this degeneracy but it is not in agree with experiments, since $2l+1$ is odd and it must be even, and finally Dirac solution is the right one, since $2j+1$ is even.

\chapter{Covariance} 

\section{Dirac spinor representation}

    Now, we verify that the Dirac equation is Lorentz invariant, i.e~it is covariant under a generic transformation of $SO^+(1,3)$. Recall that given a Lorentz transformation $\Lambda \in SO^+(1,3)$, the coordinates transform as 
    \begin{equation*}
        (x')^\mu = \Lambda^\mu_{\phantom \mu \nu} x^\nu
    \end{equation*}
    and the partial derivatives transform as 
    \begin{equation*}
        {\partial'}_\mu = \Lambda_\mu^{\phantom \mu \nu} \partial_\nu ~.
    \end{equation*}
    Therefore, the Dirac spinor transform as 
    \begin{equation*}
        \psi'(x') = S(\Lambda) \psi(x)
    \end{equation*}
    where $S(\Lambda)$ is linear representation of the proper orthochronous group of spinors and the Dirac equation is covariant 
    \begin{equation*}
        (\gamma^\mu \partial'_\mu + m) \psi' (x') = 0 ~.
    \end{equation*}

    The infinitesimal Lorentz transformation $S(\Lambda)$ is 
    \begin{equation*}
        S = \mathbb I + \frac{i}{2} \omega_{\mu\nu} \Sigma^{\mu\nu}
    \end{equation*}
    where $\Sigma^{\mu\nu}$ are a set of $6$ antisymmetric $4 \times 4$ matrices that act on spinors
    \begin{equation}\label{sigmas}
        \Sigma^{\mu\nu} = - \frac{i}{4} [\gamma^\mu, \gamma^\nu] ~,
    \end{equation}
    such that they satisfy the commutator relations 
    \begin{equation*}
        [\Sigma^{\mu\nu}, \gamma^\rho] = i (\eta^{\mu\rho} \gamma^\nu - \eta^{\nu\rho} \gamma^\mu) ~.
    \end{equation*}
    \begin{proof}
        We transform with a Lorentz transformation every components in the Dirac equation
        \begin{equation*}
            0 = (\gamma^\mu \partial'_\mu + m) \psi'(x') = (\gamma^\mu \Lambda_\mu^{\phantom \mu \nu} \partial_\nu + m) S(\Lambda) \psi(x) ~.
        \end{equation*}
        Hence 
        \begin{equation*}
        \begin{aligned}
            0 & = S^{-1}(\Lambda) (\gamma^\mu \Lambda_\mu^{\phantom \mu \nu} \partial_\nu + m) S(\Lambda) \psi(x) \\ & = (S^{-1}(\Lambda) \gamma^\mu \Lambda_\mu^{\phantom \mu \nu} S(\Lambda) \partial_\nu + m \underbrace{S^{-1} (\Lambda) S(\Lambda)}_1) \psi(x) \\ & = (S^{-1}(\Lambda) \gamma^\mu \Lambda_\mu^{\phantom \mu \nu} S(\Lambda) \partial_\nu + m) \psi(x) ~.
        \end{aligned}
        \end{equation*}
        We compare it with~\eqref{covdirac} 
        \begin{equation*}
            0 = (S^{-1}(\Lambda) \gamma^\mu \Lambda_\mu^{\phantom \mu \nu} S(\Lambda) \partial_\nu + m) \psi(x) = (\gamma^\nu \partial_\nu + m) \psi(x) 
        \end{equation*}
        and we find 
        \begin{equation*}
            S^{-1}(\Lambda) \gamma^\mu \Lambda_\mu^{\phantom \mu \nu} S(\Lambda) = \gamma^\nu
        \end{equation*}
        or, equivalently,
        \begin{equation*}
            S^{-1}(\Lambda) \gamma^\mu \underbrace{\Lambda^\rho_{\phantom \rho \nu} \Lambda_\mu^{\phantom \mu \nu}}_{\delta^\rho_{\phantom \rho \mu} } S(\Lambda) = \Lambda^\rho_{\phantom \rho \nu} \gamma^\nu ~,
        \end{equation*}
        \begin{equation*}
            S^{-1}(\Lambda) \underbrace{\gamma^\mu \delta^\rho_{\phantom \rho \mu}}_{\gamma^\rho}  S(\Lambda) = \Lambda^\rho_{\phantom \rho \nu} \gamma^\nu ~,
        \end{equation*}
        \begin{equation}\label{proof1}
            S^{-1}(\Lambda) \gamma^\rho  S(\Lambda) = \Lambda^\rho_{\phantom \rho \nu} \gamma^\nu ~.
        \end{equation}

        Now, we consider an infinitesimal Lorentz transformation 
        \begin{equation*}
            \Lambda^\mu_{\phantom \mu \nu} = \delta^\mu_{\phantom \mu \nu} + \omega^\mu_{\phantom \mu \nu} ~,
        \end{equation*}
        where $\omega_{\mu\nu} = - \omega_{\nu\mu}$, which induces an infinitesimal Lorentz transformation on the spinor 
        \begin{equation*}
            S(\Lambda) = \mathbb I + \frac{i}{2} \omega_{\mu\nu} \Sigma^{\mu\nu} ~,
        \end{equation*}
        where $\Sigma^{\mu\nu} = - \Sigma^{\mu\nu}$. Substituting in~\eqref{proof1}, we find 
        \begin{equation*}
            \Big (\mathbb I - \frac{i}{2} \omega_{\alpha\beta} \Sigma^{\alpha\beta} \Big) \gamma^\rho \Big (\mathbb I + \frac{i}{2} \omega_{\sigma\lambda} \Sigma^{\sigma\lambda} \Big) = (\delta^\rho_{\phantom \rho \nu} + \omega^\rho_{\phantom \rho \nu}) \gamma^\nu ~.
        \end{equation*}
        and we only keep first order terms in $\omega$ 
        \begin{equation*}
            \cancel{\gamma^\rho} - \frac{i}{2} \omega_{\alpha\beta} \Sigma^{\alpha\beta} \gamma^\rho + \frac{i}{2} \gamma^\rho \omega_{\sigma\lambda} \Sigma^{\sigma\lambda} = \cancel{\gamma^\rho} + \omega^\rho_{\phantom \rho \nu} \gamma^\nu ~,
        \end{equation*}
        \begin{equation*}
            - \frac{i}{2} \omega_{\alpha\beta} \underbrace{(\Sigma^{\alpha\beta} \gamma^\rho - \gamma^\rho \Sigma^{\alpha\beta})}_{[\Sigma^{\alpha\beta}, \gamma^\rho]} = \omega^\rho_{\phantom \rho \nu} \gamma^\nu ~,
        \end{equation*}
        \begin{equation*}
            - \frac{i}{2} \omega_{\alpha\beta} [\Sigma^{\alpha\beta}, \gamma^\rho] = \omega^\rho_{\phantom \rho \nu} \gamma^\nu ~,
        \end{equation*}
        where we have exchanged $\sigma = \alpha$ and $\lambda = \beta$. Hence 
        \begin{equation*}
            \cancel{\omega_{\alpha\beta}} [\Sigma^{\alpha\beta}, \gamma^\rho] = 2 i \omega^\rho_{\phantom \rho \beta} \gamma^\beta = \omega_{\alpha\beta} 2i \underbrace{\eta^{\rho\alpha} \gamma^{\beta}}_{\frac{\eta^{\rho\alpha} \gamma^{\beta} - \eta^{\rho\beta} \gamma^{\alpha}}{2}} = \cancel{\omega_{\alpha\beta}} i ( \eta^{\rho\alpha} \gamma^{\beta} - \eta^{\rho\beta} \gamma^{\alpha}) ~,
        \end{equation*}
        where we have exchanged $\nu = \beta$ and we exploit the antysymmetry of $\omega_{\alpha\beta}$ to antisymmetrise $\eta^{\rho\alpha}\gamma^{\beta}$. Thus 
        \begin{equation*}
            [\Sigma^{\alpha\beta}, \gamma^\rho] = i (\eta^{\rho\alpha} \gamma^{\beta} - \eta^{\rho\beta} \gamma^{\alpha}) ~.
        \end{equation*}
        The solution of this algebraic commutation equation is 
        \begin{equation*}
            \Sigma^{\alpha\beta} = - \frac{i}{4} [\gamma^\alpha, \gamma^\beta] ~.
        \end{equation*}
        Infact, using~\eqref{cliffordrel}
        \begin{equation*}
        \begin{aligned}
            [\Sigma^{\alpha\beta}, \gamma^\mu] & = - \frac{i}{4} [\gamma^\alpha \gamma^\beta - \gamma^\beta \gamma^\alpha, \gamma^\mu] \\ & = - \frac{i}{4} [\gamma^\alpha \gamma^\beta, \gamma^\mu] - \frac{i}{4} [\gamma^\beta \gamma^\alpha, \gamma^\mu] \\ & = - \frac{i}{4} (\gamma^\alpha \{\gamma^\beta, \gamma^\mu\} - \{\gamma^\alpha, \gamma^\mu \} \gamma^\beta - \gamma^\beta \{\gamma^\alpha, \gamma^\mu\} + \{\gamma^\beta, \gamma^\mu \} \gamma^\alpha ) \\ & = - \frac{i}{4} (\gamma^\alpha \underbrace{\{\gamma^\beta, \gamma^\mu\}}_{2 \eta^{\beta\mu}} - \underbrace{\{\gamma^\alpha, \gamma^\mu \}}_{2 \eta^{\alpha\mu}} \gamma^\beta - \gamma^\beta \underbrace{\{\gamma^\alpha, \gamma^\mu\}}_{2\eta^{\alpha\mu}} + \underbrace{\{\gamma^\beta, \gamma^\mu \}}_{2 \eta^{\beta \mu}} \gamma^\alpha ) \\ & = - \frac{i}{2} (\gamma^\alpha \eta^{\beta\mu} - \eta^{\alpha\mu}\gamma^\beta - \gamma^\beta \eta^{\alpha\mu} + \eta^{\beta \mu} \gamma^\alpha ) \\ & = - \frac{i}{2} (\eta^{\beta\mu} \gamma^\alpha  - \eta^{\alpha\mu} \gamma^\beta - \eta^{\alpha\mu} \gamma^\beta + \eta^{\beta \mu} \gamma^\alpha ) \\ & = - i (\eta^{\mu\beta} \gamma^\alpha  - \eta^{\mu\alpha} \gamma^\beta) ~,
        \end{aligned}
        \end{equation*}
        where we have used the fact that the $\eta$ is symmetric, it commutes with the $\gamma$'s and the identity
        \begin{equation*}
            [AB,C] = ABC - CAB = ABC + ACB - CAB - ACB = A \{B, C\} - \{A, C\} B ~.
        \end{equation*}
    \end{proof}

    A generic Lorentz transformation is obtained by iterating intinitesimal ones via exponential map
    \begin{equation}\label{lorspin}
        S(\Lambda) = \exp(\frac{i}{2} \omega_{\mu\nu} \Sigma^{\mu\nu}) = \exp(\frac{1}{4} \omega_{\mu\nu} \gamma^\mu \gamma^\nu)
    \end{equation}
    \begin{proof}
        Infact, using~\eqref{sigmas}
        \begin{equation*}
        \begin{aligned}
            S(\Lambda) & = \exp(\frac{i}{2} \omega_{\mu\nu} \Sigma^{\mu\nu}) \\ & =  \exp(\frac{i}{2} \omega_{\mu\nu} (-\frac{i}{4} [\gamma^\mu,\gamma^\nu])) \\ & = \exp(\frac{1}{8} \omega_{\mu\nu} (\gamma^\mu \gamma^\nu - \underbrace{\gamma^\nu \gamma^\mu}_{-\gamma^\mu \gamma^\nu})) \\ & = \exp(\frac{1}{8} \omega_{\mu\nu} (\gamma^\mu \gamma^\nu + \gamma^\mu \gamma^\nu )) \\ & = \exp(\frac{1}{4} \omega_{\mu\nu} \gamma^\mu \gamma^\nu) ~.
        \end{aligned}
        \end{equation*}
    \end{proof}

\section{Application on rotations and boosts}

    \begin{example}[Rotation around the z-axis]
        Consider a rotation around the z-axis. The infinitesimal Lorentz transformation is parametrised by 
        \begin{equation*}
            \omega_{\mu\nu} = \begin{cases}
                \varphi & (\mu, \nu) = (1,2) \\
                - \varphi & (\mu, \nu) = (2,1) \\
                0 & otherwise \\
            \end{cases} ~.
        \end{equation*}
        Therefore, the infinitesimal $\omega$ matrix is 
        \begin{equation*}
            \omega^\mu_{\phantom \mu \nu} = \begin{bmatrix}
                0 & 0 & 0 & 0 \\
                0 & 0 & \varphi & 0 \\
                0 & -\varphi & 0 & 0 \\
                0 & 0 & 0 & 0 \\
            \end{bmatrix}
        \end{equation*}
        and a finite Lorentz transformation can be found by the exponential map
        \begin{equation*}
            \Lambda^\mu_{\phantom \mu \nu} = (\exp(\omega))^\mu_{\phantom \mu \nu} = \begin{bmatrix}
                1 & 0 & 0 & 0 \\
                0 & \cos \varphi & \sin \varphi & 0 \\
                0 & -\sin \varphi & \cos \varphi & 0 \\
                0 & 0 & 0 & 1 \\
            \end{bmatrix} ~.
        \end{equation*}
        \begin{proof}
            Recall the Taylor expansions of the sine and cosine functions
            \begin{equation*}
                \cos \varphi = 1 - \frac{\varphi^2}{2} + \ldots ~, \quad \sin \varphi = \varphi - \frac{\varphi^3}{3!} + \ldots ~.
            \end{equation*}
            We Taylor expand the exponential and find
            \begin{equation*}
            \begin{aligned}
                \exp(\omega) & = \sum_{k = 0}^\infty \frac{\omega^k}{k!} = \sum_{k=0}^{\infty} \frac{1}{k!} \begin{bmatrix}
                    0 & 0 & 0 & 0 \\
                    0 & 0 & \varphi & 0 \\
                    0 & -\varphi & 0 & 0 \\
                    0 & 0 & 0 & 0 \\
                \end{bmatrix} \\ & = \mathbb I_4 + \begin{bmatrix}
                    0 & 0 & 0 & 0 \\
                    0 & 0 & \varphi & 0 \\
                    0 & -\varphi & 0 & 0 \\
                    0 & 0 & 0 & 0 \\
                \end{bmatrix} + \frac{1}{2}\begin{bmatrix}
                    0 & 0 & 0 & 0 \\
                    0 & -\varphi^2 & 0 & 0 \\
                    0 & 0 & -\varphi^2 & 0 \\
                    0 & 0 & 0 & 0 \\
                \end{bmatrix} + \frac{1}{3!} \begin{bmatrix}
                    0 & 0 & 0 & 0 \\
                    0 & 0 & - \varphi^3 & 0 \\
                    0 & \varphi^3 & 0 & 0 \\
                    0 & 0 & 0 & 0 \\
                \end{bmatrix} + \ldots \\ & = \begin{bmatrix}
                    1 & 0 & 0 & 0 \\
                    0 & 1 - \frac{\varphi^2}{2} + \ldots & \varphi - \frac{\varphi^3}{3!} + \ldots & 0 \\
                    0 & - \varphi + \frac{\varphi^3}{3!} + \ldots & 1 - \frac{\varphi^2}{2} + \ldots & 0 \\
                    0 & 0 & 0 & 1 \\
                \end{bmatrix} = \begin{bmatrix}
                    1 & 0 & 0 & 0 \\
                    0 & \cos \varphi & \sin \varphi & 0 \\
                    0 & - \sin \varphi & \cos \varphi & 0 \\
                    0 & 0 & 0 & 1 \\
                \end{bmatrix} ~.
            \end{aligned}
            \end{equation*}
        \end{proof}

        Moreover, a generic Lorents transformation on a Dirac spinor is 
        \begin{equation*}
            S(\Lambda) = \exp(\frac{i \varphi}{2} \begin{bmatrix}
                \sigma^3 & 0 \\ 0 & \sigma^3 \\
            \end{bmatrix}) = \begin{bmatrix}
                \exp(\frac{i \varphi}{2}) & 0 & 0 & 0 \\ 
                0 & \exp(\frac{-i \varphi}{2}) & 0 & 0 \\ 
                0 & 0 & \exp(\frac{i \varphi}{2}) & 0 \\ 
                0 & 0 & 0 & \exp(\frac{-i \varphi}{2}) \\
            \end{bmatrix} ~.
        \end{equation*}
        \begin{proof}
            Infact, using~\eqref{lorspin}
            \begin{equation*}
            \begin{aligned}
                S(\Lambda) & = \exp(\frac{1}{4} \omega_{\mu\nu} \gamma^\mu \gamma^\nu) = \exp(\frac{1}{4} (\underbrace{\omega_{12}}_\varphi \gamma^1 \gamma^2 + \underbrace{\omega_{21}}_{-\varphi} \gamma^2 \gamma^1 )) = \exp(\frac{\varphi}{4} (\gamma^1 \gamma^2 - \underbrace{\gamma^2 \gamma^1}_{- \gamma^1 \gamma^2} )) \\ & = \exp(\frac{\varphi}{2} \underbrace{\gamma^1}_{-i \beta \alpha^1} \underbrace{\gamma^2}_{- i \beta \alpha^3}) = \exp(\frac{\varphi}{2} \underbrace{\beta^2}_1 \alpha^1 \alpha^2) = \exp(\frac{\varphi}{2} \begin{bmatrix}
                    0 & \sigma^1 \\ \sigma^1 & 0 \\
                \end{bmatrix} \begin{bmatrix}
                    0 & \sigma^2 \\ \sigma^2 & 0 \\
                \end{bmatrix}) \\ & = \exp(\frac{\varphi}{2} \begin{bmatrix}
                    \underbrace{\sigma^1 \sigma^2}_{i \sigma^3} & 0 \\ 0 & \underbrace{\sigma^1 \sigma^2}_{i \sigma^3} \\
                \end{bmatrix})  = \exp(\frac{i \varphi}{2} \begin{bmatrix}
                    \sigma^3 & 0 \\ 0 & \sigma^3 \\
                \end{bmatrix}) \\ & = \exp(\begin{bmatrix}
                    \frac{i \varphi}{2} & 0 & 0 & 0 \\
                    0 & - \frac{i \varphi}{2} & 0 & 0 \\
                    0 & 0 & \frac{i \varphi}{2} & 0 \\
                    0 & 0 & 0 & -\frac{i \varphi}{2} \\
                \end{bmatrix}) = \begin{bmatrix}
                    \exp(\frac{i \varphi}{2}) & 0 & 0 & 0 \\ 
                    0 & \exp(\frac{-i \varphi}{2}) & 0 & 0 \\ 
                    0 & 0 & \exp(\frac{i \varphi}{2}) & 0 \\ 
                    0 & 0 & 0 & \exp(\frac{-i \varphi}{2}) \\
                \end{bmatrix} ~,
            \end{aligned}
            \end{equation*}
            where we used the property of the exponential of a diagonal matrix.
        \end{proof}
        Notice that it is a unitary representation $S^\dagger(\Lambda) = S^{-1}(\Lambda)$. It is also a double-valued representation, since a rotation of $\varphi = 2 \pi$ is represented by $S(\Lambda) = - \mathbb I$. Only with a rotation of $\varphi = 4 \pi$, we find the identity again.
    \end{example}

    \begin{example}[Generic rotation]
        A generic rotation of an angle $\varphi$ around an axis $\mathbb n$ is represented by 
        \begin{equation*}
            S(\Lambda) = \begin{bmatrix}
                \exp(\frac{i \varphi}{2} \mathbf n \cdot \boldsymbol \sigma) & 0 \\
                0 & \exp(\frac{i \varphi}{2} \mathbf n \cdot \boldsymbol \sigma) \\
            \end{bmatrix} ~.
        \end{equation*}
        This means that a decomposition like the non-relativistic limit make a rotation on both the wave functions indipendently.
    \end{example}

    \begin{example}[Boost along the x-axis]
        Consider a boost around the x-axis. The infinitesimal Lorentz transformation is parametrised by 
        \begin{equation*}
            \omega_{\mu\nu} = \begin{cases}
                - w & (\mu, \nu) = (0,1) \\
                - w & (\mu, \nu) = (1,0) \\
                0 & otherwise \\
            \end{cases} ~.
        \end{equation*}
        Therefore, the infinitesimal $\omega$ matrix is 
        \begin{equation*}
            \omega^\mu_{\phantom \mu \nu} = \begin{bmatrix}
                0 & - w & 0 & 0 \\
                - w & 0 & 0 & 0 \\
                0 & 0 & 0 & 0 \\
                0 & 0 & 0 & 0 \\
            \end{bmatrix}
        \end{equation*}
        and a finite Lorentz transformation can be found by the exponential map
        \begin{equation*}
            \Lambda^\mu_{\phantom \mu \nu} = (\exp(\omega))^\mu_{\phantom \mu \nu} = \begin{bmatrix}
                \cosh w & - \sinh w & 0 & 0 \\
                - \sinh w & \cosh w & 0 & 0 \\
                0 & 0 & 1 & 0 \\
                0 & 0 & 0 & 1 \\
            \end{bmatrix} = \begin{bmatrix}
                \gamma & - \beta \gamma & 0 & 0 \\
                - \beta \gamma & \gamma & 0 & 0 \\
                0 & 0 & 1 & 0 \\
                0 & 0 & 0 & 1 \\
            \end{bmatrix}~,
        \end{equation*}
        where we defined the rapidity $w$ in terms of the Lorentz factors 
        \begin{equation}\label{rap}
            \gamma = \frac{1}{\sqrt{1 - v^2}} = \cosh w ~, \quad \beta = v = \tanh w ~, \quad \beta \gamma = \sinh w ~.
        \end{equation}
        \begin{proof}
            Recall the Taylor expansions of the hyperbolic sine and hyperbolic cosine functions
            \begin{equation*}
                \cosh w = 1 + \frac{w^2}{2} + \ldots ~, \quad \sinh w = w + \frac{w^3}{3!} + \ldots ~.
            \end{equation*}
            We Taylor expand the exponential and find
            \begin{equation*}
            \begin{aligned}
                \exp(\omega) & = \sum_{k = 0}^\infty \frac{\omega^k}{k!} = \sum_{k=0}^{\infty} \frac{1}{k!} \begin{bmatrix}
                    0 & - w & 0 & 0 \\
                    - w & 0 & 0 & 0 \\
                    0 & 0 & 0 & 0 \\
                    0 & 0 & 0 & 0 \\
                \end{bmatrix} \\ & = \mathbb I_4 + \begin{bmatrix}
                    0 & - w & 0 & 0 \\
                    - w & 0 & 0 & 0 \\
                    0 & 0 & 0 & 0 \\
                    0 & 0 & 0 & 0 \\
                \end{bmatrix} + \frac{1}{2}\begin{bmatrix}
                    w^2 & 0 & 0 & 0 \\
                    0 & w^2 & 0 & 0 \\
                    0 & 0 & 0 & 0 \\
                    0 & 0 & 0 & 0 \\
                \end{bmatrix} + \frac{1}{3!} \begin{bmatrix}
                    0 & - w^3 & 0 & 0 \\
                    - w^3 & 0 & 0 & 0 \\
                    0 & 0 & 0 & 0 \\
                    0 & 0 & 0 & 0 \\
                \end{bmatrix} + \ldots \\ & = \begin{bmatrix}
                    1 + \frac{w^2}{2} + \ldots & - w - \frac{w^3}{3!} + \ldots & 0 & 0 \\
                    - w - \frac{w^3}{3!} + \ldots & 1 + \frac{w^2}{2} + \ldots & 0 & 0 \\
                    0 & 0 & 1 & 0 \\
                    0 & 0 & 0 & 1 \\
                \end{bmatrix} = \begin{bmatrix}
                    \cosh w & - \sinh w & 0 & 0 \\
                    - \sinh w & \cosh w & 0 & 0 \\
                    0 & 0 & 1 & 0 \\
                    0 & 0 & 0 & 1 \\    
                \end{bmatrix} ~.
            \end{aligned}
            \end{equation*}
        \end{proof}

        Moreover, a generic Lorents transformation on a Dirac spinor is 
        \begin{equation}\label{loridrac}
            S(\Lambda) = \cosh \frac{w}{2} \mathbb I - \sinh \frac{w}{2} \alpha^1 ~.
        \end{equation}
        \begin{proof}
            Infact, using~\eqref{lorspin}
            \begin{equation*}
            \begin{aligned}
                S(\Lambda) & = \exp(\frac{1}{4} \omega_{\mu\nu} \gamma^\mu \gamma^\nu) = \exp(\frac{1}{4} (\underbrace{\omega_{01}}_w \gamma^0 \gamma^1 + \underbrace{\omega_{10}}_w \gamma^1 \gamma^0 )) = \exp(\frac{w}{4} (\gamma^0 \gamma^1 + \underbrace{\gamma^0 \gamma^1}_{\gamma^1 \gamma^0} )) \\ & = \exp(\frac{w}{2} \underbrace{\gamma^0}_{-i \beta} \underbrace{\gamma^1}_{- i \beta \alpha^1}) = \exp(- \frac{w}{2} \underbrace{\beta^2}_1 \alpha^1) = \exp(- \frac{w}{2} \begin{bmatrix}
                    0 & \sigma^1 \\ \sigma^1 & 0 \\
                \end{bmatrix}) \\ & = \exp(\begin{bmatrix}
                    0 & 0 & 0 & - \frac{w}{2} \\
                    0 & 0 & - \frac{w}{2} & 0 \\
                    0 & - \frac{w}{2} & 0 & 0 \\
                    - \frac{w}{2} & 0 & 0 & 0 \\
                \end{bmatrix}) = \sum_{k = 0}^{\infty} \frac{1}{k!} \begin{bmatrix}
                    0 & 0 & 0 & - \frac{w}{2} \\
                    0 & 0 & - \frac{w}{2} & 0 \\
                    0 & - \frac{w}{2} & 0 & 0 \\
                    - \frac{w}{2} & 0 & 0 & 0 \\
                \end{bmatrix}^k \\ & = \mathbb I + \begin{bmatrix}
                    0 & 0 & 0 & - \frac{w}{2} \\
                    0 & 0 & - \frac{w}{2} & 0 \\
                    0 & - \frac{w}{2} & 0 & 0 \\
                    - \frac{w}{2} & 0 & 0 & 0 \\
                \end{bmatrix} + \frac{1}{2} \begin{bmatrix}
                    \frac{w^2}{4} & 0 & 0 & 0\\
                    0 & \frac{w^2}{4} & 0 & 0 \\
                    0 & 0 & \frac{w^2}{4} & 0 \\
                    0 & 0 & 0 & \frac{w^2}{4} \\
                \end{bmatrix} + \frac{1}{3!} \begin{bmatrix}
                    0 & 0 & 0 & - \frac{w^3}{8} \\
                    0 & 0 & - \frac{w^3}{8} & 0 \\
                    0 & - \frac{w^3}{8} & 0 & 0 \\
                    - \frac{w^3}{8} & 0 & 0 & 0 \\
                \end{bmatrix} + \ldots \\ & = \begin{bmatrix}
                    1 + \frac{w^2}{8} + \ldots & 0 & 0 & - \frac{w}{2} - \frac{w^3}{48} + \ldots \\
                    0 & 1 + \frac{w^2}{8} + \ldots & - \frac{w}{2} - \frac{w^3}{48} + \ldots & 0 \\ 
                    0 & - \frac{w}{2} - \frac{w^3}{48} + \ldots & 1 + \frac{w^2}{8} + \ldots & 0 \\ 
                    - \frac{w}{2} - \frac{w^3}{48} + \ldots & 0 & 0 & 1 + \frac{w^2}{8} + \ldots \\ 
                \end{bmatrix} \\ & = \begin{bmatrix}
                    \cosh \frac{w}{2} & 0 & 0 & - \sinh \frac{w}{2} \\
                    0 & \cosh \frac{w}{2} & - \sinh \frac{w}{2} & 0 \\
                    0 & - \sinh \frac{w}{2} & \cosh \frac{w}{2} & 0 \\
                    - \sinh \frac{w}{2} & 0 & 0 & \cosh \frac{w}{2} \\
                \end{bmatrix} \\ & = \cosh \frac{w}{2} \mathbb I - \sinh \frac{w}{2} \alpha^1 ~,
            \end{aligned}
            \end{equation*}
            where we used the Taylor expansion of the exponential.
        \end{proof}
        Notice that it is a not unitary representation $S^\dagger(\Lambda) \neq S^{-1}(\Lambda)$, since there is a theorem that states that in a non-compact group, like the boost because they are not upper-bounded in velocity, the only irreducible representations are infinite-dimensional. However, it satisfies $S^\dagger (\Lambda) = S(\Lambda)$.

        In terms of mass, momentum and energy, a finite Lorentz transformation on a Dirac spinor becomes 
        \begin{equation*}
            S(\Lambda) = \sqrt{\frac{m + E}{2m}} \Big( \mathbb I - \frac{\alpha^1 |\mathbf p|}{m + E} \Big) ~.
        \end{equation*}
        \begin{proof}
            Infact, using the hyberbolic trigonometry identities
            \begin{equation*}
                \tanh \frac{w}{2} = \frac{\sinh w}{1 + \cosh w}
            \end{equation*}
            and 
            \begin{equation*}
                \cosh \frac{w}{2} = \sqrt{\frac{1 + \cosh w}{2}}
            \end{equation*}

            Using the rapidity relations~\eqref{rap}, we can rewrite~\eqref{loridrac} as 
            \begin{equation*}
            \begin{aligned}
                S(\Lambda) & = \cosh \frac{w}{2} \Big( \mathbb I - \alpha^1 \tanh \frac{w}{2} \Big) \\ & = \sqrt{\frac{1}{2}} (1 + \underbrace{\cosh w}_{\gamma})^{\frac{1}{2}} \Big (\mathbb I + \alpha^1 \frac{\overbrace{\sinh w}^{\beta\gamma}}{1 + \underbrace{\cosh w}_\gamma} \Big) \\ & = \sqrt{\frac{1 + \gamma}{2}} \Big ( \mathbb I + \alpha^1 \frac{\beta \gamma}{1 + \gamma}\Big) ~.
            \end{aligned}
            \end{equation*}

            Now, we use the $4$-momentum $(E, p) = (m \gamma, m \gamma\beta)$ and we reverse to find 
            \begin{equation*}
                \gamma = \frac{E}{m} ~, \quad \beta \gamma = \frac{|\mathbf p|}{m} ~.
            \end{equation*}

            Putting together, we obtain 
            \begin{equation*}
                S(\Lambda) = \sqrt{\frac{1 + \gamma}{2}} \Big ( \mathbb I + \alpha^1 \frac{\beta \gamma}{1 + \gamma}\Big) = \sqrt{\frac{1 + \frac{E}{m}}{2}} \Big ( \mathbb I + \alpha^1  \frac{\frac{\mathbf p}{m}}{1 + \frac{E}{m}}\Big) = \sqrt{\frac{m + E}{2m}} \Big ( \mathbb I + \frac{\alpha^1 |\mathbf p|}{m + E}\Big) ~.
            \end{equation*}
        \end{proof}
    \end{example}

    \begin{example}[Generic boost]
        A generic boost of rapidity $w$ along an axis $\mathbb n$ is represented by 
        \begin{equation*}\label{genlorspi}
            S(\Lambda) = \sqrt{\frac{m + E}{2m}} \Big ( \mathbb I + \frac{\boldsymbol \alpha \cdot \mathbf p}{m + E}\Big)
        \end{equation*}
    \end{example}

\subsection{Pseudo-unitary}

    Dirac spinorial representation is pseudo-unitary 
    \begin{equation*}
        S^\dagger (\Lambda) = \beta S^{-1} (\Lambda) \beta ~.
    \end{equation*}
    \begin{proof}
        Infact, using~\eqref{betagamma}
        \begin{equation*}
        \begin{aligned}
            (\Sigma^{\mu\nu})^\dagger & = (- \frac{i}{4} [\gamma^\mu, \gamma^\nu])^\dagger \\ & = \frac{i}{4} [\underbrace{(\gamma^\nu)^\dagger}_{-\beta \gamma^\nu \beta}, \underbrace{(\gamma^\mu)^\dagger}_{- \beta \gamma^\mu \beta}] \\ & = \frac{i}{4} [\beta \gamma^\nu \beta, \beta \gamma^\mu \beta] \\ & = \frac{i}{4} (\beta \gamma^\nu \underbrace{\beta \beta}_1 \gamma^\mu \beta - \beta \gamma^\mu \underbrace{\beta \beta}_1 \gamma^\nu \beta ) \\ & = \beta \frac{i}{4} [\gamma^\nu, \gamma^\mu] \beta \\ & = - \beta \underbrace{(-\frac{i}{4} [\gamma^\mu, \gamma^\nu] )}_{\Sigma^{\mu\nu}} \beta \\ & = - \beta \Sigma^{\mu\nu} \beta ~.
        \end{aligned}
        \end{equation*}
        Therefore,
        \begin{equation*}
        \begin{aligned}
            S^\dagger (\Lambda) & = \exp(\frac{i}{2} \omega_{\mu\nu} \Sigma^{\mu\nu})^\dagger \\ & = \exp(-\frac{i}{2} \omega_{\mu\nu} (\Sigma^{\mu\nu})^\dagger ) \\ & = \exp(-\frac{i}{2} \omega_{\mu\nu} \beta \Sigma^{\mu\nu} \beta ) \\ & = \sum_{k=0}^\infty \frac{1}{k!} (-\frac{i}{2} \omega_{\mu\nu} \beta \Sigma^{\mu\nu} \beta)^k \\ & = \beta \sum_{k=0}^\infty \frac{1}{k!} (-\frac{i}{2} \omega_{\mu\nu} \Sigma^{\mu\nu})^k \beta \\ & = \beta \exp(-\frac{i}{2} \omega_{\mu\nu} \Sigma^{\mu\nu}) \beta \\ & = \beta S^{-1} (\Lambda) \beta~.
        \end{aligned} 
        \end{equation*}
    \end{proof}

\section{Fermionic bilinears}

    Since the Dirac spinor representatio is pseudo-unitary, we define the Dirac conjugate by
    \begin{equation*}
        \overline \psi (x) = \psi^\dagger (x) \beta ~,
    \end{equation*}
    such that it transforms as 
    \begin{equation*}
        \overline \psi' (x') = \overline \psi (x) S^{-1} (\Lambda) ~.
    \end{equation*}
    \begin{proof}
        Infact 
        \begin{equation*}
            \overline \psi' (x') = (\underbrace{\psi'}_{S(\Lambda) \psi})^\dagger (x') \beta = (S^(\Lambda) \psi (x) )^\dagger = \psi^\dagger (x) \underbrace{S^\dagger (\Lambda)}_{\beta S^{-1} (\Lambda) \beta} \beta = \underbrace{\psi^\dagger (x) \beta}_{\overline \psi (x)} S^{-1} (\Lambda) \underbrace{\beta \beta}_1 = \overline \psi (x) S^{-1} (\Lambda) ~.
        \end{equation*}
    \end{proof}

    With the Dirac spinor and its conjugate we can build a scalar, which is
    \begin{equation*}
        \overline \psi (x) \psi (x) ~,
    \end{equation*}
    but not this
    \begin{equation*}
        \psi^\dagger (x) \psi(x) ~.
    \end{equation*}
    \begin{proof}
        For the scalar
        \begin{equation*}
            \overline \psi' (x') \psi'(x') = \overline \psi (x) \underbrace{S^{-1} (\Lambda) S (\Lambda)}_1 \psi (x) = \overline \psi (x) \psi (x) ~.
        \end{equation*}

        For the non-scalar
        \begin{equation*}
            (\psi' )^\dagger (x') \psi'(x') = (S(\Lambda) \psi)^\dagger (x) S(\Lambda) \psi(x) = \psi^\dagger (x) \underbrace{S^\dagger (\Lambda)}_{\beta S^{-1} \beta} S (\Lambda) \psi(x) = \psi^\dagger (x) \beta S^{-1} \beta S (\Lambda) \psi(x) \neq \psi^\dagger (x) \psi (x) ~.
        \end{equation*}
    \end{proof}
    
    Actually, it is the $0$-th component of a four-vector 
    \begin{equation*}
        J^\mu = (J^0, \mathbf J) = (\psi^\dagger \psi, \psi^\dagger \boldsymbol \alpha \psi) = i \overline \psi (x) \gamma^\mu \psi (x)
    \end{equation*}
    which is the current that appears in the continuity equation.
    \begin{proof}
        Infact 
        \begin{equation*}
            (J')^\mu (x') = i \overline \psi' (x') \gamma^\mu \psi' (x') = i \overline \psi (x) \underbrace{S^{-1} (\Lambda) \gamma^\mu S(\Lambda)}_{\Lambda^\mu_{\phantom \mu \nu} \gamma^\nu} \psi (x) = \Lambda^\mu_{\phantom \mu \nu} \underbrace{i \overline \psi (x) \gamma^\nu \psi (x)}_{J^\mu (x)} = \Lambda^\mu_{\phantom \mu \nu} J^\mu (x) ~.
        \end{equation*}
    \end{proof}

    A fermionic bilinear is a useful quantity to describe physical properties. We can costruct them starting with the basis 
    \begin{equation}\label{basis}
        \Gamma^A = (\mathbb I, \gamma^5, \Sigma^{\mu\nu}, \gamma^\mu \gamma^5, \gamma^5) ~,
    \end{equation} such that
    \begin{equation*}
        \overline \psi \Gamma^a \psi ~.
    \end{equation*}
    Notice that under Lorentz transformation, they all five transform in a good way: $\mathbb I$ and $\gamma^5$ are scalars, $\gamma^5$ and $\gamma^\mu \gamma^5$ are $4$-vectors, $\Sigma^{\mu\nu}$ is a $2$-tensor. However, under parity transformation, the last two of~\eqref{basis} are an axial-vector and pseudo-scalar.
    \begin{proof}
        Maybe in the future.
    \end{proof}

    Recall from group theory, the Dirac spinor representation of the Lorentz group satisfies the Lie algebra 
    \begin{equation*}
        [\Sigma, \Sigma] 
    \end{equation*}
    and the gamma matrices $\gamma^\mu$ are invariant tensors (Clebsh-Gordan coefficients). 
    \begin{proof}
        Maybe in the future.
    \end{proof}
    This means that the covariance of the Dirac equation can be understood in the usual way: contraction of tensor indices that makes the spinor $\xi (x) = (\gamma^\mu \partial_\mu + m) \psi(x)$ manifestly covariant and equals zero in any inertial frame.

\chapter{Wave plane solutions}

    In order to find a solution of the Dirac equation~\eqref{covdirac}, we propose a plane wave ansatz 
    \begin{equation*}
        \psi_P(x) = w (p) \exp(i p_\mu x^\mu) ~,
    \end{equation*}
    where $\exp(i p_\mu x^\mu)$ is the propagation in space-time, $p^\mu$ is arbitrary and $w(p)$ is the polarisation 
    \begin{equation*}
        w(p) = \begin{bmatrix}
            w_1(p) \\ w_2(p) \\ w_3(p) \\ w_4(p) \\
        \end{bmatrix} ~.
    \end{equation*}
    such that the polarisation satisfies the algebraic equation 
    \begin{equation}\label{pol}
        (i \gamma^\mu p_\mu + m ) w(p) = 0 ~,
    \end{equation}
    and $p^\mu$ satisfies the mass-shell relation for a relativistic particle
    \begin{equation*}
        p^\mu p_\mu + m^2 = 0~.
    \end{equation*}
    \begin{proof}
        Infact, inserting the ansatz in~\eqref{covdirac}
        \begin{equation*}
            0 = (\gamma^\mu \partial_\mu + m) \psi(x) = (\gamma^\mu \partial_\mu + m) w(p) \exp(i p_\mu x^\mu) = \underbrace{(i \gamma^\mu p_\mu + m) w(p)}_0 \underbrace{\exp(i p_\mu x^\mu)}_{\neq 0} ~.
        \end{equation*}
        Hence
        \begin{equation*}
            (i \gamma^\mu p_\mu + m) w(p) = (i \cancel p + m) w(p) = 0 ~.
        \end{equation*}

        Furthermore, we have 
        \begin{equation*}
            0 = (- i \cancel p + m)(i \cancel p + m) w(p) = \underbrace{(\cancel p^2 + m^2)}_0 \underbrace{w(p)}_{\neq 0} ~,
        \end{equation*}
        and we notice, using~\eqref{cliffordrel}
        \begin{equation*}
            \cancel p^2 = \gamma^\mu p_\mu \gamma^\nu p_\nu = p_\mu p_\nu \underbrace{\gamma^\mu \gamma^\nu}_{\frac{\gamma^\mu \gamma^\nu + \gamma^\nu \gamma^\mu}{2}} = p_\mu p_\nu \frac{1}{2} \underbrace{(\gamma^\mu \gamma^\nu + \gamma^\nu \gamma^\mu)}_{\{\gamma^\mu, \gamma^\nu\} = 2 \eta^{\mu\nu}}= p_\mu p_\nu \eta^{\mu\nu} = p_\mu p^\mu = p^2 ~,
        \end{equation*}
        where we exploit the symmetry of $p^\mu p^\nu$ to symmetrise $\gamma^\mu \gamma^\nu$. Hence
        \begin{equation*}
            p^2 + m^2 = 0 ~.
        \end{equation*}
    \end{proof}

\section{Plane wave at rest}

    \begin{example}[Rest-frame]
        Consider a particle at rest, which means with $p^\mu = (E,0,0,0)$. Then, we substitute in~\eqref{pol}
        \begin{equation*}
            0  = (i \cancel p + m)w(p) = (i \gamma^0 p_0) w(p) = (- \underbrace{i \gamma^0}_{\beta} \underbrace{p^0}_E + m) w(p) = (- \beta E + m) w(p) ~.
        \end{equation*}
        Hence 
        \begin{equation*}
            0 = \beta (- \beta E + m) w(P) = (- \beta^2 E + \beta m) w(p) = (- E + \beta m) w(P)
        \end{equation*}
        and 
        \begin{equation*}
            E w(p) = \beta m w(p)
        \end{equation*}
        Recalling the matrix representation of $\beta$, we obtain 
        \begin{equation*}
            \begin{bmatrix}
                E & 0 & 0 & 0 \\
                0 & E & 0 & 0 \\
                0 & 0 & E & 0 \\
                0 & 0 & 0 & E \\
            \end{bmatrix} 
            \begin{bmatrix}
                w_1(p) \\ w_2(p) \\ w_3(p) \\ w_4(p) \\
            \end{bmatrix} = 
            \begin{bmatrix}
                m & 0 & 0 & 0 \\
                0 & m & 0 & 0 \\
                0 & 0 & -m & 0 \\
                0 & 0 & 0 & -m \\
            \end{bmatrix}
            \begin{bmatrix}
                w_1(p) \\ w_2(p) \\ w_3(p) \\ w_4(p) \\ 
            \end{bmatrix} ~.
        \end{equation*}
        This means that we have four different solutions: two are with positive energy $E = m$ which can be intepreted electrons with spin-up and spin-down
        \begin{equation*}
            \psi_1(x) = \begin{bmatrix}
                1 \\ 0 \\ 0 \\ 0 \\
            \end{bmatrix} \exp(-imt) ~, \quad \psi_2(x) = \begin{bmatrix}
                0 \\ 1 \\ 0 \\ 0 \\
            \end{bmatrix} \exp(-imt) ~,
        \end{equation*}
        and two with negative energy $E = - m$ which can be intepreted positrons with spin-up and spin-down
        \begin{equation*}
            \psi_3(x) = \begin{bmatrix}
                0 \\ 0 \\ 1 \\ 0 \\
            \end{bmatrix} \exp(imt) ~, \quad \psi_4(x) = \begin{bmatrix}
                0 \\ 0 \\ 0 \\ 1 \\
            \end{bmatrix} \exp(imt) ~.
        \end{equation*}
    \end{example}

\section{Moving plane wave}

    In order to find general solutions with arbitrary momentum, we apply a Lorentz transformation to the rest-frame solutions. A generic boost trasforms the rest-frame plane wave into 
    \begin{equation*}
    \begin{aligned}
        & \psi_1(x) = \sqrt{\frac{m + E}{2m}} \begin{bmatrix}
            1 \\ 0 \\ \frac{p_3}{m+E} \\ \frac{p_+}{m+E} \\
        \end{bmatrix} \exp(ip_\mu x^\mu t) ~, \quad \psi_2(x) = \sqrt{\frac{m + E}{2m}} \begin{bmatrix}
            0 \\ 1 \\ \frac{p_-}{m + E} \\ - \frac{p_3}{m + E} \\
        \end{bmatrix} \exp(ip_\mu x^\mu t) ~, \\ & 
        \psi_3(x) = \sqrt{\frac{m + E}{2m}} \begin{bmatrix}
            \frac{p_3 }{m+E}\\ \frac{p_+}{m+E} \\ 1 \\ 0 \\
        \end{bmatrix} \exp(-ip_\mu x^\mu t) ~, \quad \psi_4(x) = \sqrt{\frac{m + E}{2m}} \begin{bmatrix}
            \frac{p_- }{m+E}\\ - \frac{p_3}{m+E} \\ 0 \\ 1 \\
        \end{bmatrix} \exp(-ip_\mu x^\mu t) ~,
    \end{aligned}
    \end{equation*}
    where $p_\pm = p_1 \pm i p_2$. 
    \begin{proof}
        Firsly, we compute  
        \begin{equation*}
            \alpha^1 = \begin{bmatrix}
                0 & \sigma^1 \\ \sigma^1 & 0 \\
            \end{bmatrix} = \begin{bmatrix}
                0 & 0 & 0 & 1 \\ 
                0 & 0 & 1 & 0 \\ 
                0 & 1 & 0 & 0 \\ 
                1 & 0 & 0 & 0 \\ 
            \end{bmatrix} ~,
        \end{equation*}
        \begin{equation*}
            \alpha^2 = \begin{bmatrix}
                0 & \sigma^2 \\ \sigma^2 & 0 \\
            \end{bmatrix} = \begin{bmatrix}
                0 & 0 & 0 & -i \\ 
                0 & 0 & i & 0 \\ 
                0 & -i & 0 & 0 \\ 
                i & 0 & 0 & 0 \\ 
            \end{bmatrix} ~, 
        \end{equation*}
        \begin{equation*}
            \alpha^3 = \begin{bmatrix}
                0 & \sigma^3 \\ \sigma^3 & 0 \\
            \end{bmatrix} = \begin{bmatrix}
                0 & 0 & 1 & 0 \\ 
                0 & 0 & 0 & -1 \\ 
                1 & 0 & 0 & 0 \\ 
                0 & -1 & 0 & 0 \\ 
            \end{bmatrix} ~.
        \end{equation*}
        Hence
        \begin{equation*}
            \boldsymbol \alpha \cdot \mathbf p = \alpha^1 p_1 + \alpha^2 p_2 + \alpha^3 p_3 = \begin{bmatrix}
            0 & 0 & p_3 & p_1 - i p_2 \\
            0 & 0 & p_1 + i p_2 & - p_3 \\
            p_3 & p_1 - i p_2 & 0 & 0 \\
            p_1 + i p_2& -p_3 & 0 & 0 \\
            \end{bmatrix} = \begin{bmatrix}
                0 & 0 & p_3 & p_- \\
                0 & 0 & p_+ & - p_3 \\
                p_3 & p_- & 0 & 0 \\
                p_+ & -p_3 & 0 & 0 \\
            \end{bmatrix} ~.
        \end{equation*}
    
        Now, we compute $\psi_1$ 
        \begin{equation*}
        \begin{aligned}
            S(\Lambda) \psi_1 (x) & = \sqrt{\frac{m + E}{2m}} \Big ( \mathbb I + \frac{\boldsymbol \alpha \cdot \mathbf p}{m + E}\Big) \psi_1 (x) \\ & = \sqrt{\frac{m + E}{2m}} \Big ( \mathbb I + \frac{1}{m + E} \begin{bmatrix}
                0 & 0 & p_3 & p_- \\
                0 & 0 & p_+ & - p_3 \\
                p_3 & p_- & 0 & 0 \\
                p_+ & -p_3 & 0 & 0 \\
            \end{bmatrix} \Big) \begin{bmatrix}
                1 \\ 0 \\ 0 \\ 0 \\
            \end{bmatrix} \exp(i p_\mu x^\mu t) \\ & = \sqrt{\frac{m + E}{2m}} \Big ( \begin{bmatrix}
                1 \\ 0 \\ 0 \\ 0 \\
            \end{bmatrix} + \frac{1}{m + E} \begin{bmatrix}
                0 \\ 0 \\ p_3 \\ p_+ \\
            \end{bmatrix} \Big) \exp(i p_\mu x^\mu t) \\ & = \sqrt{\frac{m + E}{2m}} \begin{bmatrix}
                1 \\ 0 \\ \frac{p_3}{m+E} \\ \frac{p_+}{m+E} \\
            \end{bmatrix} \exp(ip_\mu x^\mu t)  ~,
        \end{aligned}
        \end{equation*}
        we compute $\psi_2$ 
        \begin{equation*}
        \begin{aligned}
            S(\Lambda) \psi_2 (x) & = \sqrt{\frac{m + E}{2m}} \Big ( \mathbb I + \frac{\boldsymbol \alpha \cdot \mathbf p}{m + E}\Big) \psi_2 (x) \\ & = \sqrt{\frac{m + E}{2m}} \Big ( \mathbb I + \frac{1}{m + E} \begin{bmatrix}
                0 & 0 & p_3 & p_- \\
                0 & 0 & p_+ & - p_3 \\
                p_3 & p_- & 0 & 0 \\
                p_+ & -p_3 & 0 & 0 \\
            \end{bmatrix} \Big) \begin{bmatrix}
                0 \\ 1 \\ 0 \\ 0 \\
            \end{bmatrix} \exp(i p_\mu x^\mu t) \\ & = \sqrt{\frac{m + E}{2m}} \Big ( \begin{bmatrix}
                0 \\ 1 \\ 0 \\ 0 \\
            \end{bmatrix} + \frac{1}{m + E} \begin{bmatrix}
                0 \\ 0 \\ p_- \\ - p_3 \\
            \end{bmatrix} \Big) \exp(i p_\mu x^\mu t) \\ & = \sqrt{\frac{m + E}{2m}} \begin{bmatrix}
                0 \\ 1 \\ \frac{p_-}{m + E} \\ - \frac{p_3}{m + E} \\
            \end{bmatrix} \exp(ip_\mu x^\mu t) ~,
        \end{aligned}
        \end{equation*}
        we compute $\psi_3$ 
        \begin{equation*}
        \begin{aligned}
            S(\Lambda) \psi_3 (x) & = \sqrt{\frac{m + E}{2m}} \Big ( \mathbb I + \frac{\boldsymbol \alpha \cdot \mathbf p}{m + E}\Big) \psi_3 (x) \\ & = \sqrt{\frac{m + E}{2m}} \Big ( \mathbb I + \frac{1}{m + E} \begin{bmatrix}
                0 & 0 & p_3 & p_- \\
                0 & 0 & p_+ & - p_3 \\
                p_3 & p_- & 0 & 0 \\
                p_+ & -p_3 & 0 & 0 \\
            \end{bmatrix} \Big) \begin{bmatrix}
                0 \\ 0 \\ 1 \\ 0 \\
            \end{bmatrix} \exp(- i p_\mu x^\mu t) \\ & = \sqrt{\frac{m + E}{2m}} \Big ( \begin{bmatrix}
                0 \\ 0 \\ 1 \\ 0 \\
            \end{bmatrix} + \frac{1}{m + E} \begin{bmatrix}
                p_3 \\ p_+ \\ 0 \\ 0 \\
            \end{bmatrix} \Big) \exp(- i p_\mu x^\mu t) \\ & = \sqrt{\frac{m + E}{2m}} \begin{bmatrix}
                \frac{p_3 }{m+E}\\ \frac{p_+}{m+E} \\ 1 \\ 0 \\
            \end{bmatrix} \exp(-ip_\mu x^\mu t)
        \end{aligned}
        \end{equation*}
        and we compute $\psi_4$ 
        \begin{equation*}
        \begin{aligned}
            S(\Lambda) \psi_4 (x) & = \sqrt{\frac{m + E}{2m}} \Big ( \mathbb I + \frac{\boldsymbol \alpha \cdot \mathbf p}{m + E}\Big) \psi_4 (x) \\ & = \sqrt{\frac{m + E}{2m}} \Big ( \mathbb I + \frac{1}{m + E} \begin{bmatrix}
                0 & 0 & p_3 & p_- \\
                0 & 0 & p_+ & - p_3 \\
                p_3 & p_- & 0 & 0 \\
                p_+ & -p_3 & 0 & 0 \\
            \end{bmatrix} \Big) \begin{bmatrix}
                0 \\ 0 \\ 0 \\ 1 \\
            \end{bmatrix} \exp(- i p_\mu x^\mu t) \\ & = \sqrt{\frac{m + E}{2m}} \Big ( \begin{bmatrix}
                0 \\ 0 \\ 0 \\ 1 \\
            \end{bmatrix} + \frac{1}{m + E} \begin{bmatrix}
                p_- \\ - p_3 \\ 0 \\ 0 \\
            \end{bmatrix} \Big) \exp(- i p_\mu x^\mu t) \\ & = \sqrt{\frac{m + E}{2m}} \begin{bmatrix}
                \frac{p_- }{m+E}\\ - \frac{p_3}{m+E} \\ 0 \\ 1 \\
            \end{bmatrix} \exp(-ip_\mu x^\mu t) ~.
        \end{aligned}
        \end{equation*}
    \end{proof}

\chapter{Discrete symmetries} 

    In order to recover the whole Lorentz group $O(1,3)$, it is necessary to take into account also discrete transformations, like parity and time reversal, which allows us to go into the disconnected to the identity parts of the group. 

\section{Parity} 

    The parity transformation is defined by the revesal of the orientation of the spatial axes
    \begin{equation*}
        (t', \mathbf x') = (t, - \mathbf x) ~, \quad (x')^\mu = P^\mu_{\phantom \mu \nu} x^\nu ~,
    \end{equation*}
    where the parity matrix $P$ is 
    \begin{equation*}
        P^\mu_{\phantom \mu \nu} = \begin{bmatrix}
            1 & 0 & 0 & 0 \\
            0 & -1 & 0 & 0 \\
            0 & 0 & -1 & 0 \\
            0 & 0 & 0 & -1 \\
        \end{bmatrix} ~.
    \end{equation*}
    It has determinant equals to $-1$, so it is not connected to the identity and it forms a subgroup with the identity, isomorphic to $\mathbb Z_2 = \{\mathbb I, P\}$.

    What is the analoguous parity transformation for a Dirac spinor? We conjecture that exists a linear transformation 
    \begin{equation*}
        \psi' (x') = \mathcal P \psi(x) ~.
    \end{equation*}
    It turns out that such a matrix exists and it is 
    \begin{equation*}
        \mathcal P = \beta ~.
    \end{equation*}
    \begin{proof}
        Infact
        \begin{equation*}
            0 = (\gamma^\mu \underbrace{{\partial'}_\mu}_{P_\mu^{\phantom \mu \nu} \partial_\nu} + m) \underbrace{\psi'(x')}_{\mathcal P \psi(x)} = (\gamma^\mu P_\mu^{\phantom \mu \nu} \partial_\nu + m) \mathcal P \psi(x) ~.
        \end{equation*}
        Hence 
        \begin{equation*}
            0 = \mathcal P^{-1}(\gamma^\mu P_\mu^{\phantom \mu \nu} \partial_\nu + m)\mathcal P \psi(x) = (\mathcal P^{-1}\gamma^\mu P_\mu^{\phantom \mu \nu} \mathcal P \partial_\nu + m) \psi(x)
        \end{equation*}
        and to be Lorentz invariant, we have the condition 
        \begin{equation*}
            \mathcal P^{-1}\gamma^\mu P_\mu^{\phantom \mu \nu} \mathcal P = \gamma^\nu
        \end{equation*}
        or equivalently
        \begin{equation*}
            \mathcal P^{-1}\gamma^\mu \mathcal P = P^\mu_{\phantom \mu \nu} \gamma^\nu ~.
        \end{equation*}

        Since 
        \begin{equation*}
            P^\mu_{\phantom \mu \nu} \gamma^\mu = \begin{bmatrix}
                1 & 0 & 0 & 0 \\
                0 & -1 & 0 & 0 \\
                0 & 0 & -1 & 0 \\
                0 & 0 & 0 & -1 \\
            \end{bmatrix} \begin{bmatrix}
                \gamma^0 \\ \gamma^1 \\ \gamma^2 \\ \gamma^3 \\
            \end{bmatrix} = \begin{bmatrix}
                \gamma^0 \\ - \gamma^1 \\ - \gamma^2 \\ - \gamma^3 \\
            \end{bmatrix} ~,
        \end{equation*}
            we need a matrix that commutes with $\gamma^0$ and anticommutes with $\gamma^i$, which is $\gamma^0$ itself  and 
        \begin{equation*}
            \mathcal P = \eta \gamma^0 ~,
        \end{equation*}
        where $\eta$ is an arbitrary phase factor, that we choose to be $1$, because it is one of the four solutions of $\mathcal P^4 = 1$, which are $\eta = \{\pm 1, \pm i\}$.
    \end{proof}

    Hence, we have under parity these transformations
    \begin{equation*}
        \begin{cases}
            (x')^\mu = P^\mu_{\phantom \mu \nu} x^\nu \\
            \psi' (x') = \beta \psi(x) \\
            \overline \psi'(x') = \overline \psi (x) \beta \\
        \end{cases} ~.
    \end{equation*}
    \begin{proof}
        Infact 
        \begin{equation*}
            \overline \psi' (x') = (\psi')^\dagger (x') \beta = \beta \psi^\dagger (x) \beta = \psi^\dagger \underbrace{\beta^2}_1 = \psi^\dagger = \overline \psi \beta ~.
        \end{equation*}
    \end{proof}

    From these transformations, we can deduce which are the true scalar or vectors and which are only pseudo, because it appears a minus sign when transformed under parity
    \begin{equation*}
        \begin{cases}
            \overline \psi'(x') \psi'(x') = \overline \psi(x) \psi(x) \\
            \overline \psi'(x') \gamma^5 \psi'(x') = - \overline \psi(x) \gamma^5 \psi(x) \\
            \overline \psi'(x')\gamma^\mu \psi'(x') = \overline \psi(x) P^\mu_{\phantom \mu \nu} \gamma^\nu \psi(x) \\
            \overline \psi'(x') \gamma^\mu \gamma^5 \psi'(x') = -\overline \psi(x) P^\mu_{\phantom \mu \nu} \gamma^\nu \gamma^5 \psi(x) \\
            \overline \psi'(x') \Sigma^{\mu\nu} \psi'(x') = \overline \psi(x) P^\mu_{\phantom \mu \alpha} P^\nu_{\phantom \nu \beta} \Sigma^{\alpha \beta} \psi(x) \\
        \end{cases} ~,
    \end{equation*}
    which shows that $\gamma^5$ is indeed a pseudoscalar and $\gamma^\mu \gamma^5$ is indeed an axial vector.
    \begin{proof}
        For $\mathbb I$, 
        \begin{equation*}
            \overline \psi' (x') \psi' (x') = \overline \psi (x) \underbrace{\beta \beta}_1 \psi (x) = \overline \psi (x) \psi (x)  ~.
        \end{equation*}

        For $\gamma^5$, 
        \begin{equation*}
            \overline \psi' (x') \gamma^5 \psi' (x') = \overline \psi (x) \beta \underbrace{\gamma^5 \beta}_{-\beta \gamma^5} \psi (x) = - \overline \psi (x) \underbrace{\beta \beta}_1 \gamma^5 \psi (x) = - \overline \psi (x) \gamma^5 \psi (x)~.
        \end{equation*}

        For $\gamma^\mu$, 
        \begin{equation*}
            \overline \psi' (x') \gamma^\mu \psi' (x') = \overline \psi (x) \underbrace{\beta \gamma^\mu \beta }_{P^\mu_{\phantom \mu \nu} \gamma^\nu} \psi (x) = \overline \psi (x) P^\mu_{\phantom \mu \nu} \gamma^\nu \psi (x) ~.
        \end{equation*}

        For $\gamma^\mu \gamma^5$, 
        \begin{equation*}
            \overline \psi' (x') \gamma^5 \psi' (x') = \overline \psi (x) \beta \gamma^\mu \underbrace{\gamma^5 \beta}_{-\beta \gamma^5} \psi (x) = - \overline \psi (x) \underbrace{\beta \gamma^\mu \beta }_{P^\mu_{\phantom \mu \nu} \gamma^\nu} \gamma^5 \psi (x) = - \overline \psi (x) P^\mu_{\phantom \mu \nu} \gamma^\nu \gamma^5 \psi (x)  ~.
        \end{equation*}
        
        For $\Sigma^{\mu\nu}$, 
        \begin{equation*}
        \begin{aligned}
            \overline \psi' (x') \Sigma^{\mu\nu} \psi' (x') & = \overline \psi (x) \beta \underbrace{\Sigma^{\mu\nu}}_{-\frac{i}{4} [\gamma^\mu, \gamma^\nu]} \beta \psi (x) \\ & = - \frac{i}{4} \overline \psi (x) \beta [\gamma^\mu,  \gamma^\nu] \beta \psi (x) \\ & = - \frac{i}{4} \overline \psi (x) \beta (\gamma^\mu \gamma^\nu - \gamma^\nu \gamma^\mu) \beta \psi (x) \\ & = - \frac{i}{4} \overline \psi (x) \beta \gamma^\mu \underbrace{\mathbb I }_{\beta \beta} \gamma^\nu \psi (x) + \frac{i}{4} \overline \psi (x) \beta \gamma^\nu \underbrace{\mathbb I }_{\beta \beta} \gamma^\mu \beta \psi (x) \\ & = - \frac{i}{4} \overline \psi (x) \underbrace{\beta \gamma^\mu \beta}_{P^\mu_{\phantom \mu \alpha} \gamma^\alpha} \underbrace{\beta \gamma^\nu \beta}_{P^\nu_{\phantom \nu \beta} \gamma^\beta} \psi (x) + \frac{i}{4} \overline \psi (x) \underbrace{\beta \gamma^\nu \beta}_{P^\nu_{\phantom \nu \beta} \gamma^\beta} \underbrace{\beta \gamma^\mu \beta}_{P^\mu_{\phantom \mu \alpha} \gamma^\alpha} \psi (x) \\ & = - \frac{i}{4} \overline \psi (x) P^\mu_{\phantom \mu \alpha} \gamma^\alpha P^\nu_{\phantom \nu \beta} \gamma^\beta \psi (x) + \frac{i}{4} \overline \psi (x) P^\nu_{\phantom \nu \beta} \gamma^\beta P^\mu_{\phantom \mu \alpha} \gamma^\alpha \psi (x) \\ & = \overline \psi (x) P^\mu_{\phantom \mu \alpha} P^\nu_{\phantom \nu \beta} \Big (-\frac{i}{4} (\gamma^\alpha \gamma^\beta - \gamma^\beta \gamma^\alpha) \Big ) \psi (x) \\ & = \overline \psi (x) P^\mu_{\phantom \mu \alpha} P^\nu_{\phantom \nu \beta} \underbrace{\Big (-\frac{i}{4} [\gamma^\alpha, \gamma^\beta] \Big )}_{\Sigma^{\alpha \beta}} \psi (x) \\ & = \overline \psi (x) P^\mu_{\phantom \mu \alpha} P^\nu_{\phantom \nu \beta} \Sigma^{\alpha \beta} \psi (x) ~.
        \end{aligned}
        \end{equation*}
    \end{proof}

\subsection{Weyl spinors}   

    The projectors~\eqref{proj} allow us to show that the Dirac representation is reducible into two spin $\frac{1}{2}$ representations of the Lorentz group. It follows from the fact that $\Sigma^{\mu\nu}$ commutes with $P_L$ and $P_R$
    \begin{equation*}
        [\Sigma^{\mu\nu}, P_L] = 0 ~, \quad [\Sigma^{\mu\nu}, P_R] = 0 ~,
    \end{equation*}
    which means that an infinitesimal Lorentz transformation acts on a Weyl spinor and it does not change its chirality. 
    \begin{proof}
        Infact 
        \begin{equation*}
        \begin{aligned}
            [P_L, \Sigma^{\mu\nu}] & = [\frac{\mathbb I - \gamma^5}{2}, - \frac{i}{4} [\gamma^\mu, \gamma^\nu]] \\ & = - \frac{i}{8} [\mathbb I - \gamma^5, \gamma^\mu \gamma^\nu - \gamma^\nu \gamma^\mu] \\ & = - \frac{i}{8} \underbrace{[\mathbb I, \gamma^\mu \gamma^\nu - \gamma^\nu \gamma^\mu]}_0 + \frac{i}{8} [\gamma^5, \gamma^\mu \gamma^\nu - \gamma^\nu \gamma^\mu] \\ & = \frac{i}{8} [\gamma^5, \gamma^\mu \gamma^\nu] - \frac{i}{8}  [\gamma^5, \gamma^\nu \gamma^\mu] \\ & = \frac{i}{8} \gamma^\mu [\gamma^5, \gamma^\nu] + \frac{i}{8} [\gamma^5, \gamma^\mu ] \gamma^\nu - \frac{i}{8}  \gamma^\nu [\gamma^5, \gamma^\mu] - \frac{i}{8}  [\gamma^5, \gamma^\nu] \gamma^\mu \\ & = \frac{i}{8} \gamma^\mu \underbrace{\{\gamma^5, \gamma^\nu\}}_0 - 2 \frac{i}{8} \gamma^\mu \gamma^\nu \gamma^5 + \frac{i}{8} \underbrace{\{\gamma^5, \gamma^\mu \}}_0 \gamma^\nu - 2 \frac{i}{8} \gamma^\mu \underbrace{\gamma^5 \gamma^\nu}_{- \gamma^\nu \gamma^5} \\ & \qquad - \frac{i}{8}  \gamma^\nu \underbrace{\{\gamma^5, \gamma^\mu\}}_0 + 2 \frac{i}{8} \gamma^\nu \gamma^\mu \gamma^5 - \frac{i}{8} \underbrace{\{\gamma^5, \gamma^\nu\}}_0 \gamma^\mu + 2 \frac{i}{8} \gamma^\nu \underbrace{\gamma^5 \gamma^\mu}_{- \gamma^\mu \gamma^5} \\ & = - \cancel{2 \frac{i}{8} \gamma^\mu \gamma^\nu \gamma^5} + 2 \cancel{\frac{i}{8} \gamma^\mu \gamma^\nu \gamma^5} + 2 \cancel{\frac{i}{8}  \gamma^\nu \gamma^\mu \gamma^5} - \cancel{2 \frac{i}{8} \gamma^\nu \gamma^\mu \gamma^5} \\ & = 0 ~,
        \end{aligned}
        \end{equation*}
        where we have used the identity
        \begin{equation*}
            [A, B] = AB - BA = AB + BA - BA - BA = \{A, B\} - 2 B A ~.
        \end{equation*}
    \end{proof}

    Knowing that the Lie algebra of the Lorentz group can be written as $\mathfrak{so}(1,3) = \mathfrak{su}(2) + \mathfrak{su}(2)$, we can label each $SU(2)$ subalgebra with an half-integer $(j, j')$ associated to an irreducible representations. Dirac representation $(0, \frac{1}{2}) \oplus (\frac{1}{2}, 0)$ is invariant under parity, while Weyl spinor one is not. Infact under parity, a Weyl spinor get exchanged into the other one
    \begin{equation*}
        (\psi_L)' = (\psi')_R ~, (\psi_R)' = (\psi')_L ~.
    \end{equation*}
    \begin{proof}
        Infact, Lorentz trasformation does not mix
        \begin{equation*}
            (\psi_L)' = S (\Lambda) \psi_L = \exp(\frac{i}{2} \omega_{\mu\nu} \Sigma^{\mu\nu}) P_L \psi = P_L \exp(\frac{i}{2} \omega_{\mu\nu} \Sigma^{\mu\nu}) \psi = P_L S(\Lambda) \psi = (\psi')_L ~,
        \end{equation*}
        while parity transformation does
        \begin{equation*}
            (\psi_L)' = \beta \psi_L = \beta P_L \psi = \beta \frac{\mathbb I - \gamma^5}{2} \psi = \frac{\mathbb I + \gamma^5}{2} \beta \psi_L = P_R \beta \psi = (\psi')_R ~.
        \end{equation*}
    \end{proof}

\section{Time reversal} 

    The time reversal transformation is defined by the reversal of the orientation of the time axis
    \begin{equation*}
        (t', \mathbf x') = (- t, \mathbf x) ~, \quad (x')^\mu = T^\mu_{\phantom \mu \nu} x^\nu ~,
    \end{equation*}
    where the time reversal matrix $T$ is 
    \begin{equation*}
        T^\mu_{\phantom \mu \nu} = \begin{bmatrix}
            -1 & 0 & 0 & 0 \\
            0 & 1 & 0 & 0 \\
            0 & 0 & 1 & 0 \\
            0 & 0 & 0 & 1 \\
        \end{bmatrix} ~.
    \end{equation*}
    It has determinant equals to $-1$, so it is not connected to the identity and it forms a subgroup with the identity, isomorphic to $\mathbb Z_2 = \{\mathbb I, T\}$.

    What is the analoguous parity transformation for a Dirac spinor? We conjecture that exists a linear transformation 
    \begin{equation*}
        \psi' (x') = \mathcal T \psi(x) ~.
    \end{equation*}
    It turns out that such a matrix does not exist. However, by analogy with the Schoredinger equation in which time reversal is the same as complex conjugate, we can defied as 
    \begin{equation*}
        \mathcal T = \gamma^1 \gamma^3 ~,
    \end{equation*}
    with the difference that it transforms its complex conjugate $\psi'(x') = \mathcal T \psi^* (x)$.
    \begin{proof}
        Infact
        \begin{equation*}
            0 = (\gamma^\mu \underbrace{{\partial'}_\mu}_{T_\mu^{\phantom \mu \nu} \partial_\nu} + m) \underbrace{\psi'(x')}_{\mathcal T \psi^* (x)} = (\gamma^\mu T_\mu^{\phantom \mu \nu} \partial_\nu + m) \mathcal T \psi^*(x) ~.
        \end{equation*}
        Hence 
        \begin{equation*}
            0 = \mathcal T^{-1}(\gamma^\mu T_\mu^{\phantom \mu \nu} \partial_\nu + m)\mathcal T \psi^*(x) = (\mathcal T^{-1}\gamma^\mu T_\mu^{\phantom \mu \nu} \mathcal T \partial_\nu + m) \psi^*(x)
        \end{equation*}
        and to be Lorentz invariant, we have the condition 
        \begin{equation*}
            \mathcal T^{-1}\gamma^\mu T_\mu^{\phantom \mu \nu} \mathcal T = (\gamma^*)^\nu
        \end{equation*}
        or equivalently
        \begin{equation*}
            \mathcal T^{-1}\gamma^\mu \mathcal T = T^\mu_{\phantom \mu \nu} (\gamma^*)^\nu ~.
        \end{equation*}

        Since 
        \begin{equation*}
            T^\mu_{\phantom \mu \nu} (\gamma^*)^\mu = \begin{bmatrix}
                - 1 & 0 & 0 & 0 \\
                0 & 1 & 0 & 0 \\
                0 & 0 & 1 & 0 \\
                0 & 0 & 0 & 1 \\
            \end{bmatrix} \begin{bmatrix}
                (\gamma^*)^0 \\ (\gamma^*)^1 \\ (\gamma^*)^2 \\ (\gamma^*)^3 \\
            \end{bmatrix} = \begin{bmatrix}
                - (\gamma^*)^0 \\ (\gamma^*)^1 \\ (\gamma^*)^2 \\ (\gamma^*)^3 \\ = 
            \end{bmatrix} = \begin{bmatrix}
                \gamma^0 \\ -\gamma^1 \\ \gamma^2 \\ - \gamma^3 \\ 
            \end{bmatrix} ~,
        \end{equation*}
            we need a matrix that commutes with $\gamma^0$ and $\gamma^2$ and anticommutes with $\gamma^1$ and $\gamma^3$, which is $\gamma^1 \gamma^3$ and 
        \begin{equation*}
            \mathcal T = \eta \gamma^1 \gamma^3~,
        \end{equation*}
        where $\eta$ is an arbitrary phase factor, that we choose to be $1$, because it is one of the four solutions of $\mathcal T^4 = 1$, which are $\eta = \{\pm 1, \pm i\}$.
    \end{proof}

\subsection{Hole theory}

    In order to solve the problem of negative energy solutions, we propose the hole theory. Consider the vacuum state, which is the minimum energy level, defined as configuration in which all negeative energy are occupied by electrons, called the Dirac sea. Pauli's exclusion principle guarantees the stability in which all holes all filled. By definition, the vacuum energy $E_{vac} = 0$ and charge $Q_{vac} = 0$ are zero. An electron state means that a positive energy level is occupied with energy energy $E_{e^-} = E_p > 0$ and charge $Q_{e^-} = e$. Pauli's exclusion principle ensures that it cannot jump in the Dirac sea. If there is a hole, we can interpet it as an antiparticle. Infact, it is equivalent to a configuration with energy $E_{hol} = E_p > 0$ and charge $Q_{hol} = - e$. 
    \begin{proof}
        Infact 
        \begin{equation*}
            E_{hol} + (- E_p) = E_{vac} = 0 
        \end{equation*}
        and 
        \begin{equation*}
            Q_{hol} + e = Q_{vac} = 0 ~.
        \end{equation*}
    \end{proof}
    It predicts both the positron and the pair creation, i.e. if a photon interacts with the vacuum, it creates an electron and a positron because it tranfers energy to a state of the Dirac sea and it occupates a positive energy one.

\section{Charge conjugation}

    The couple with electromagnetism is made through minimal substitution~\eqref{minsub}. For a particle of charge $e$ and mass $m$, it can be written as 
    \begin{equation*}
        (\gamma^\mu (\partial_\mu - i e A_\mu) + m) \psi = 0 ~,
    \end{equation*}
    while for an antiparticle of charge $-e$ and mass $m$ is 
    \begin{equation*}
        (\gamma^\mu (\partial_\mu + i e A_\mu) + m) \psi_c = 0 ~.
    \end{equation*}
    It is not a symmetry, but they are related by a transformation
    \begin{equation*}
        \mathcal A (\gamma^*)^\mu \mathcal A = \gamma^\mu ~,
    \end{equation*}
    where $A$ can be identify as 
    \begin{equation*}
        \mathcal A = \mathcal C \beta
    \end{equation*}
    and $\mathcal C$ is the charge conjugation matrix, defined as 
    \begin{equation*}
        \mathcal C = \gamma^0 \gamma^2 ~.
    \end{equation*}
    It is a background symmetry. However, in QED for dynamical $A_\mu$, we have a true symmetry. Notice that $\mathcal C$ is antysymmetric and it coincides with its inverse.
    \begin{proof}
        Infact, from
        \begin{equation*}
            0 = (\gamma^\mu ( \partial_\mu - i e A_\mu ) + m) \psi (x) 
        \end{equation*}
        we have 
        \begin{equation*}
            0 = ((\gamma^*)^\mu ( \partial_\mu + i e A_\mu ) + m) \psi^* (x) ~,
        \end{equation*}
        and comparing with 
        \begin{equation*}
            (\gamma^\mu (\partial_\mu + i e A_\mu) + m) \psi_c = 0 ~,
        \end{equation*}
        to be Lorentz invariant, we have the condition 
        \begin{equation*}
            \mathcal A (\gamma^*)^\mu \mathcal A = \gamma^\mu ~,
        \end{equation*}
        such that 
        \begin{equation*}
            \psi_c = \mathcal A \psi^* ~.
        \end{equation*}

        We choose in the form
        \begin{equation*}
            \mathcal = \mathcal C \beta
        \end{equation*}
        and in terms of the Dirac conjugate
        \begin{equation*}
            \psi_c = \mathcal A \psi^* = \mathcal C \beta \psi^* = \mathcal C \overline \psi^T ~.
        \end{equation*}

        Hence, using
        \begin{equation*}
            (\gamma^*)^\mu = (- \beta \gamma^\mu \beta)^T = - \beta (\gamma^T)^\mu \beta ~,
        \end{equation*}
        we have 
        \begin{equation*}
            \gamma^\mu = \mathcal A (\gamma^*)^\mu \mathcal A = \mathcal C \beta (\gamma^*)^\mu \mathcal C \beta = -\mathcal C \underbrace{\beta \beta }_ 1(\gamma^T)^\mu \beta \mathcal C \beta = ~,
        \end{equation*}
        or equivalently
        \begin{equation*}
            \mathcal C^{-1} \gamma^\mu \mathcal C = - (\gamma^T)^\mu = \begin{bmatrix}
                - \gamma^0 \\ \gamma^1 \\ - \gamma^2 \\ \gamma^3 \\
            \end{bmatrix} ~,
        \end{equation*}
        we need a matrix that commutes with $\gamma^1$ and $\gamma^2$ and anticommutes with $\gamma^0$ and $\gamma^2$, which is $\gamma^0 \gamma^2$ and 
        \begin{equation*}
            \mathcal C = \eta \gamma^0 \gamma^2~,
        \end{equation*}
        where $\eta$ is an arbitrary phase factor, that we choose to be $1$, because it is one of the four solutions of $\mathcal C^4 = 1$, which are $\eta = \{\pm 1, \pm i\}$.
    \end{proof}

\subsection{Weyl spinors}   

    For Weyl spinors, we have 
    \begin{equation*}
        (\psi_L)_c = P_R C (\psi_L)_c ~.
    \end{equation*}
    \begin{proof}
        Infact 
        \begin{equation*}
            \psi_{L,c} = \mathcal C (\overline \psi_L)^T = \mathcal (\overline \psi_L P_R)^T = \mathcal C P_R (\overline \psi_L)^T = P_R \mathcal C (\overline \psi_L)^T = P_R \psi_{L,c} ~.
        \end{equation*}
    \end{proof}

\subsection{CPT}

    Even though singular discrete symmetries are broken by interaction, the union of the three is always valid for Lorentz invariance theories. It acts as 
    \begin{equation*}
        \begin{cases}
            (x')^\mu = - x^\mu \\
            \psi'(x') = \gamma^5 \psi(x) \\
        \end{cases} ~,
    \end{equation*}
    which is a symmetry for the Dirac equation.
    \begin{proof}
        Infact 
        \begin{equation*}
            (\gamma^\mu {\partial'}_\mu + m) \psi'(x') = (- \gamma^\mu \partial_\mu + m) \gamma^5 \psi(x) = \cancel{\gamma^5} (\gamma^\mu \partial_\mu + m) \psi (x) = 0 ~.
        \end{equation*}
    \end{proof}


\chapter{Dirac action} 