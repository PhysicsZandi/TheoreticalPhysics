\part{Klein-Gordon equation}

\chapter{Derivation}

    To find the equation of motion which governs a relativistic quantum particle, we start with the relation between energy and momentum. We define the $4$-momentum as 
    \begin{equation*}
        p^\mu = (E, \vec p)
    \end{equation*}
    such that its norm is constant
    \begin{equation*}
        p^\mu p_\mu = - E^2 + |\vec p|^2 = - m^2
    \end{equation*}
    which is called the mass-shell condition. The energy-momentum relation is then
    \begin{equation*}
        E^2 = p^2 + m^2
    \end{equation*}

    Now, we can use this energy and promote it to an operator
    \begin{equation*}
        E = \sqrt{p^2 + m^2}
    \end{equation*}
    and then put it inside the Schroedinger equation 
    \begin{equation*}
        i \pdv{}{t} \phi(t, \vec x) = E \phi(t, \vec x) = \sqrt{p^2 + m^2} \phi(t, \vec x) 
    \end{equation*}

    The square root of an operator could lead us to a non-local theory and this approach was abandoned. 

    Klein and Gordon used instead the square relation between energy and momentum, but uses the operator substitution 
    \begin{equation*}
        E \rightarrow i \pdv{}{t} \quad \vec p \rightarrow - i \vec \nabla
    \end{equation*}
    and the energy-momentum relation becomes 
    \begin{equation*}
        (- \pdvdu{}{t} + \nabla^2 - m^2) \phi(t, \vec x) = 0
    \end{equation*}
    which in covariant notation is 
    \begin{equation}\label{kgeq}
        (\partial^\mu \partial_\mu - m^2) \phi(x) = (\Box - m^2) \phi(x) = 0
    \end{equation}

\chapter{Solutions}

\section{Plane waves}

    Plane waves are solutions. Let us consider a plane waves ansatz 
    \begin{equation*}
        \phi(x) = \exp(i p^\mu x_\mu)
    \end{equation*}
    where $p^\mu$ is arbitrary. Using~\eqref{kgeq}
    \begin{equation*}
        - (p^\mu p_\mu + m^2) \exp(i p_\mu x^\mu) = 0
    \end{equation*}
    Hence, the plane wave is a solution only if $p^\mu$ satistfies the mass-shell condition 
    \begin{equation*}
        (p^0)^2 = \vec p^2 + m^2 \Rightarrow p^0 = \pm \sqrt{\vec p^2 + m^2} = \pm E_p
    \end{equation*}
    Since $E_p geq 0$, notice that negative energy are allowed as solutions. This shows that the theory is not stable, since there is no energy limitation from below.

    Positive energy solutions are  
    \begin{equation*}
        \phi^+_p = \exp(i(-E_p t + \vec p \cdot \vec x))
    \end{equation*}
    while the negative energy solutions are 
    \begin{equation*}
        \phi^-_p = \exp(i(E_p t - \vec p \cdot \vec x))
    \end{equation*}
    Hence, a general solution is a linear combination of plane waves 
    \begin{equation*}
        \phi (x) = \int \frac{d^3 p}{(2 \pi)^3} \frac{1}{2 E_p} (a(\vec p) \exp(i(-E_p t + \vec p \cdot \vec x)) + b^*(\vec p) \exp(i(E_p t - \vec p \cdot \vec x)))
    \end{equation*}
    and its complex conjugate
    \begin{equation*}
        \phi^* (x) = \int \frac{d^3 p}{(2 \pi)^3} \frac{1}{2 E_p} (a^* (\vec p) \exp(i(-E_p t + \vec p \cdot \vec x)) + b(\vec p) \exp(i(E_p t - \vec p \cdot \vec x)))
    \end{equation*}
    where $a(\vec p)$ and $b(\vec p)$ are the Fourier coefficients. If the field is real, i.e. $\phi 0 \phi^*$, then $a(\vec p) = b(\vec p)$.

\section{Continuity equation}

    Starting from~\eqref{kgeq} and multiplying by the complex conjugation 
    \begin{equation*}
        0 = \phi^* (\Box - m^2) \phi - \phi (\Box - m^2) \phi^* = \partial_\mu (\phi^* \partial^\mu \phi - \phi \partial^\mu \phi^*)
    \end{equation*}
    Hence, the current is 
    \begin{equation*}
        J^\mu = \frac{1}{2im} (\phi^* \partial^\mu \phi - \phi \partial^\mu \phi^*)
    \end{equation*}
    such that it satisfies the continuity equation 
    \begin{equation*}
        \partial_\mu J^\mu = 0
    \end{equation*}

    The time component is 
    \begin{equation*}
        J^0 = \frac{1}{2im} (\phi^* \partial^t \phi - \phi \partial^t \phi^*) = \frac{i}{2m} (\phi^* \partial_t \phi - \phi \partial_t \phi^*)
    \end{equation*}
    which is not positive, even though it is real. This is because it can be negative or positive by the initial condition choice, e.g. for plane waves 
    \begin{equation*}
        J^0 (\phi^\pm_{\vec p}) = \pm \frac{E_p}{m}
    \end{equation*}

    Hence, there is no probability intepretation (what is a negative probability?). Dirac suggested a particle intepretation, lead to a electromagnetism comparison.

\section{Green functions}

    The Green function is a solution of the Klein-Gordon equation in presence of a point-like istantaneous source. For convenience, we put it at the origin. Mathematically, it satisfies the equation 
    \begin{equation*}
        (- \Box + m^2) G(x) = \delta^4 (x)
    \end{equation*}
    
    It is useful to determine the solution of a general source $J(x)$ 
    \begin{equation*}
        (- \Box + m^2) \phi(x) = J (x)
    \end{equation*}
    which are related by 
    \begin{equation*}
        \phi (x) = \phi_0 (x) + \int d^4y ~ G(x-y) J (y)
    \end{equation*}
    wherw $\phi_0 (x)$ is a solution of the associated inhomogeneous equation.

    \begin{proof}
        \begin{equation*}
        \begin{aligned}
            (- \Box + m^2) \phi(x) & =  (- \Box + m^2) \Big (\phi_0 (x) + \int d^4y ~ G(x-y) J (y) \Big ) \\ & = \underbrace{(- \Box + m^2) \phi_0 (x)}_0 + (- \Box + m^2) \int d^4y ~ G(x-y) J (y) \\ & =  \int d^4y ~ \underbrace{(- \Box + m^2) G(x-y)}_{\delta^4(x-y)} J (y) \\ & = \int d^4 y ~ \delta^4(x-y) J(y) \\ & = J(x)
        \end{aligned}
        \end{equation*}
    \end{proof}

    In general, the Green function is not unique, but it depends on boundary conditions chosen at the infinity. We use the Feynman-Stueckelberg which gives the correct quantum intepretation, i.e. negative energies are antiparticles. This means that positive energy particle are propagated forward in time and negative energy particle are propagated backward in time. It describes also real particle, i.e. whose satisfy the mass-shell condition, and virtal particle, i.e. whose do not. The latter ones are not visible at large distances.  

    Using the Fourier transform 
    \begin{equation*}
        G(x) = \int \frac{d^4 p}{(2\pi)^4} \exp(i p_\mu x^\mu) \tilde G(p)
    \end{equation*}
    which could be written as 
    \begin{equation*}
        G(x) = \int \frac{d^4 p}{(2\pi)^4} {\exp(i p_\mu x^\mu)}{p^2 + m^2 - i \epsilon}
    \end{equation*}
    where $\epsilon \ll 1$ is a positive infinitesimal number.

    The propagator is the Green function which is the amplitude of propagation from a point y to another point x
    \begin{equation*}
        \Delta (x - y) = - i G(x - y)
    \end{equation*}

    In our case, it becomes 
    \begin{equation}
        \Delta (x - y) = - i G(x - y) = 
    \end{equation}




    \section{Yukawa potential}
    
    Yukawa suggested from the Klein-Gordon equation, a theory of nuclear interactions.



\chapter{Action}


