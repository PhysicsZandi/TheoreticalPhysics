\part{Schroedinger equation}

\chapter{Derivation}

    To find the equation of motion which governs a relativistic quantum particle, we start with the relation between energy and momentum. We define the $4$-momentum as 
    \begin{equation*}
        p^\mu = (E, \vec p)
    \end{equation*}
    such that its norm is constant
    \begin{equation*}
        p^\mu p_\mu = - E^2 + |\vec p|^2 = - m^2
    \end{equation*}
    which is called the mass-shell condition. The energy-momentum relation is then
    \begin{equation}\label{enmom}
        E^2 = p^2 + m^2
    \end{equation}

    Now, we can use this energy and promote it to an operator
    \begin{equation*}
        E = \sqrt{p^2 + m^2}
    \end{equation*}
    and then put it inside the Schroedinger equation 
    \begin{equation*}
        i \pdv{}{t} \phi(t, \vec x) = E \phi(t, \vec x) = \sqrt{p^2 + m^2} \phi(t, \vec x) 
    \end{equation*}

    The square root of an operator could lead us to a non-local theory and this approach was abandoned. 

    Klein and Gordon used instead the square relation between energy and momentum, but uses the operator substitution 
    \begin{equation}\label{sub}
        E \rightarrow i \pdv{}{t} \quad \vec p \rightarrow - i \vec \nabla
    \end{equation}
    and the energy-momentum relation becomes 
    \begin{equation*}
        (- \pdvdu{}{t} + \nabla^2 - m^2) \phi(t, \vec x) = 0
    \end{equation*}
    which in covariant notation is 
    \begin{equation}\label{kgeq}
        (\partial^\mu \partial_\mu - m^2) \phi(x) = (\Box - m^2) \phi(x) = 0
    \end{equation}
