\documentclass[a4paper, 12pt, openany]{memoir}

\usepackage[a4paper, top = 4cm, bottom = 4cm, left = 3cm, right = 3cm]{geometry}

\usepackage[T1]{fontenc}
\usepackage[utf8]{inputenc}
\usepackage{pythontex} 
\usepackage{nopageno} 
\usepackage{pgf}

\usepackage{tocloft}
\newcommand{\listequationsname}{List of Equations}
\newlistof{listofequations}{equ}{\listequationsname}
\newcommand{\myequation}[1]{%
	\addcontentsline{equ}{equation}{\protect\numberline{\theequation}#1}\par
}
\makeatletter
\let\l@equation\l@figure
\makeatother

\usepackage{xcolor}
\xdefinecolor{mycolor}{RGB}{0,175,179} 
\usepackage{hyperref}
\hypersetup{colorlinks, linkcolor={mycolor}, citecolor={mycolor}, urlcolor={mycolor}}

\usepackage{lipsum}

\renewcommand{\aftertoctitle}{\afterchaptertitle\par\nobreak\hfill{\normalfont{Page}}\par\nobreak}

\usepackage{titlesec}
\titleformat{\part}[display]
  {\normalfont\HUGE\bfseries\color{mycolor}\centering}
  {Part \thepart}{20pt}{\HUGE\normalfont\color{black}}
\titleformat{\chapter}[display]
  {\normalfont\HUGE\bfseries\color{mycolor}\centering}
  {Chapter \thechapter}{20pt}{\HUGE\normalfont\color{black}}
\titleformat{\section}
  {\normalfont\Large\bfseries\color{mycolor}\centering}
  {\thesection}{1em}{}
\titleformat{\subsection}
  {\normalfont\large\bfseries\color{mycolor}\centering}
  {\thesubsection}{1em}{}

\renewcommand{\printtoctitle}[1]{\HUGE\normalfont\color{black}#1}

\usepackage[backend=bibtex, sorting=none]{biblatex}
\addbibresource{../bibliography.bib}

\usepackage{amsmath}
\usepackage{amsthm}
\usepackage{thmtools}
\usepackage{mathtools}

\newtheorem{principle}{Principle}[chapter]
\newtheorem{lemma}{Lemma}[chapter]
\theoremstyle{definition}
\newtheorem{example}{Example}[chapter]
\newtheorem{exercise}{Exercise}[chapter]
\renewcommand\qedsymbol{q.e.d.}

\theoremstyle{remark}
\newtheorem{case}{Case}

\newcommand{\dv}[2]{\frac{d#1}{d#2}}
\newcommand{\cdv}[2]{\frac{D#1}{D#2}}
\newcommand{\dvin}[3]{\frac{d#1}{d#2}\Big\vert_{#3}}
\newcommand{\dvd}[2]{\frac{d^2#1}{d#2^2}}
\newcommand{\dvf}[2]{\frac{\delta #1}{\delta #2}}
\newcommand{\pdv}[2]{\frac{\partial#1}{\partial#2}}
\newcommand{\pdvd}[3]{\frac{\partial^2 #1}{\partial#2 \partial#3}}
\newcommand{\pdvdu}[2]{\frac{\partial^2 #1}{\partial#2^2}}
\newcommand{\integ}[3]{\int_{#1}^{#2}d#3~}
\newcommand{\poi}[2]{[#1,~#2]}
\newcommand{\poiexp}[2]{\pdv{#1}{q^i} \pdv{#2}{p_i} - \pdv{#2}{q^i} \pdv{#1}{p_i}}

\newcommand{\comm}[2]{[#1,~#2]}
\newcommand{\set}[2]{\{#1\colon#2\}}
\newcommand{\inner}[2]{\langle#1,~#2\rangle}
\newcommand{\av}[1]{\langle#1\rangle}
\newcommand{\avp}[2]{\langle#1\rangle_{#2}}
\newcommand{\ket}[1]{\vert#1\rangle}
\newcommand{\bra}[1]{\langle#1\vert}
\newcommand{\braket}[2]{\langle#1\vert#2\rangle}

\newtheoremstyle{colored}{}{}{\itshape}{}{\color{mycolor}\normalfont\bfseries\indent}{}{\newline}{}

\declaretheorem[
  style=colored,
  name=Definition,
  numberwithin=chapter,
]{definition}

\declaretheorem[
  style=colored,
  name=Theorem,
  numberwithin=chapter,
]{theorem}

\declaretheorem[
  style=colored,
  name=Corollary,
  numberwithin=chapter,
]{corollary}

\declaretheorem[
  style=colored,
  name=Law,
  numberwithin=chapter,
]{law}

\declaretheorem[
  style=colored,
  name=Principle,
  numberwithin=chapter,
]{princ}

\usepackage{amsfonts}
\usepackage{dsfont}
\usepackage{yfonts}
\usepackage{amssymb}

\let\oldproof\proof
\renewcommand{\proof}{\color{darkgray}\oldproof}

\let\oldexample\example
\renewcommand{\example}{\color{darkgray}\oldexample}

\let\oldexercise\exercise
\renewcommand{\exercise}{\color{darkgray}\oldexercise}

\usepackage{cancel}
\usepackage{indentfirst}

\usepackage{tikz}
\usepackage{amssymb}
\usepackage{pgfplots}
\usepgfplotslibrary{patchplots}
\usetikzlibrary{patterns, positioning, arrows}
\pgfplotsset{compat=1.15}

\DeclareMathOperator{\tr}{tr}
\DeclareMathOperator{\str}{str}
\DeclareMathOperator{\real}{Re}
\DeclareMathOperator{\imm}{Im}
\DeclareMathOperator{\sgn}{sgn}
\DeclareMathOperator{\spann}{span}
\DeclareMathOperator{\vol}{vol}
\DeclareMathOperator{\erf}{erf}



\def\blankpage{%
      \clearpage%
      \thispagestyle{empty}%
      \addtocounter{page}{-1}%
      \null%
      \clearpage}

\usepackage{tikz-feynman}
\usepackage{feynmp}
\tikzfeynmanset{compat=1.1.0}
\DeclareGraphicsRule{*}{mps}{*}{}


\title{statistical mechanics}
\author{Matteo Zandi \\ ~ \\ (matteo.zandi2@studio.unibo.it)}
\date{\today}

\newcommand{\subt}{what happens when there are too many particles?}

\begin{document}

\frontmatter

\pagestyle{empty}
{\raggedleft\vspace*{\baselineskip}
{\LARGE Matteo Zandi}\\[0.35\textheight]
{\HUGE \textcolor{mycolor}{\textbf{On~\thetitle:}}}\\[\baselineskip]
{\LARGE \subt }\\[\baselineskip]
{\large \thedate}\par
\vspace*{2\baselineskip}
\vfill
{\large matteo.zandi2@studio.unibo.it}\par
\vspace*{\baselineskip}}
\clearpage
\pagestyle{headings}

\blankpage

\tableofcontents

\mainmatter

\begin{pycode}
import sympy as sy
def plot1(x, f, rangex, rangey, fig, leg, negx, negy):
    rangexx = rangex
    rangeyy = rangey
    if negx == True:
        rangexx = 0
    if negy == True:
        rangeyy = 0
    x = sy.Symbol('x')
    p = sy.plot((f, (x, -rangexx, rangex)), ylim=[-rangeyy, rangey], legend= leg, show=False, line_color='#00AFB3')
    p.save(f'fig/fig{fig}.pgf')
    print(r'\input{fig/fig'+ rf'{fig}' + r'.pgf}')

def plot2(x, f, g, rangex, rangey, fig, leg, negx, negy):
    rangexx = rangex
    rangeyy = rangey
    if negx == True:
        rangexx = 0
    if negy == True:
        rangeyy = 0
    x = sy.Symbol('x')
    p = sy.plot((f, (x, -rangexx, rangex)), (g, (x, -rangexx, rangex)), ylim=[-rangeyy, rangey], legend= leg, show=False, line_color='#00AFB3')
    p[0].line_color='red'
    p[1].line_color='blue'
    p.save(f'fig/fig{fig}.pgf')
    print(r'\input{fig/fig'+ rf'{fig}' + r'.pgf}')

def plot3(x, f, g, h, rangex, rangey, fig, leg, negx, negy):
    rangexx = rangex
    rangeyy = rangey
    if negx == True:
        rangexx = 0
    if negy == True:
        rangeyy = 0
    x = sy.Symbol('x')
    p = sy.plot((f, (x, -rangexx, rangex)), (g, (x, -rangexx, rangex)), (h, (x, -rangexx, rangex)), ylim=[-rangeyy, rangey], legend= leg, show=False, line_color='#00AFB3')
    p[0].line_color='red'
    p[1].line_color='violet'
    p[2].line_color='blue'
    p.save(f'fig/fig{fig}.pgf')
    print(r'\input{fig/fig'+ rf'{fig}' + r'.pgf}')

def plot4(x, f, g, h, l, rangex, rangey, fig, leg, negx, negy):
    rangexx = rangex
    rangeyy = rangey
    if negx == True:
        rangexx = 0
    if negy == True:
        rangeyy = 0
    x = sy.Symbol('x')
    p = sy.plot((f, (x, -rangexx, rangex)), (g, (x, -rangex, rangex)), (h, (x, -rangex, rangex)), (l, (x, -rangex, rangex)), ylim=[-rangeyy, rangey], legend= leg, show=False, line_color='#00AFB3')
    p[3].line_color='black'
    p.save(f'fig/fig{fig}.pgf')
    print(r'\input{fig/fig'+ rf'{fig}' + r'.pgf}')

def der(y, x):
    x = sy.Symbol(x) 
    derivative = sy.diff(y, x)
    return sy.latex(derivative) 
 
def indint(integrand, x): 
    x = sy.Symbol(x) 
    integral = sy.integrate(integrand,x) 
    return sy.latex(integral) 

def defint(integrand, x, min, max): 
    x = sy.Symbol(x) 
    integral = sy.integrate(integrand, (x, min, max)) 
    return sy.latex(integral) 

def infint(integrand, x): 
    x = sy.Symbol(x) 
    integral = sy.integrate(integrand, (x, float('-inf'), float('inf'))) 
    return sy.latex(integral) 

def infzint(integrand, x): 
    x = sy.Symbol(x) 
    integral = sy.integrate(integrand, (x, 0, float('inf'))) 
    return sy.latex(integral) 

def ode(ode, y, x): 
    x = sy.Symbol(x) 
    y = sy.Function(y) 
    lhs, rhs = ode.split('=') 
    ode = sy.Eq(sy.S(lhs),sy.S(rhs)) 
    sol = sy.dsolve(ode,y(x)) 
    return sy.latex(sol) 

# \py{ode("Derivative(y(x),x,x) + y(x) = 0", "y", "x")} ~.
 
def odeic(ode, y, x, ic): 
    x  = sy.Symbol(x) 
    y  = sy.Function(y) 
    lhs,rhs = ode.split('=') 
    ode = sy.Eq(sy.S(lhs),sy.S(rhs)) 
    sol = sy.dsolve(ode,y(x), ics= sy.S(ic)) 
    return sy.latex(sol) 

#\py{odeic("Derivative(y(x),x,x) + y(x) = 0", "y", "x", "{y(0):1, y(x).diff(x).subs(x, 0): 0}")} ~.

def matrixmult(A, B):
    C = A*B
    return sy.latex(C)

def Taylor(x, f, point, order):
    x = sy.Symbol('x')
    ts = sy.series(f, x, point, order) 
    return sy.latex(ts)

def limit(x, f, point):
    x = sy.Symbol('x')
    lim = sy.limit(f, x, point) 
    return sy.latex(lim)

\end{pycode}

\chapter*{Introduction}

    In these lectures notes, we will cover the part of physics which studies systems composed by a large amount of constituents, like particles: thermodynamics and statistical mehcanics. In the first part, we will recall some notions of thermodynamics, in the language of differential geometry: the laws of thermodynamics and thermodynamics potentials. In the second part, we will study classical statistical mechanics, starting from the basic notions of classical (hamiltonian) mechanics till the $3$ ensembles: microcanonical, canonical and grancanonical. In the third part, we will study quantum statistical mechanics, starting from the basic notions of quantum mechanics, in the language of canonical and second quantisation, till the properties of bosons and fermions. At the end of each of these last $2$ parts, there will be some applications and exercises. In the last part, we will superficially introduce classical phase transitions and the classical Ising model. 

\part{Thermodynamics}

\chapter{Laws of Thermodynamics}

\section{The laws of Thermodynamics}

    In this chapter, we will recall some notions of Thermodynamics.

    In Thermodynamics, a state is defined by a set of macroscopic quantities, called thermodynamical variables. They can be divided into two groups, one conjugate to the other, according to their behaviour when the physical system is rescaled, i.e.~when the volume and the number of particles change: extensive variables do scale with it whereas intensive ones do not. See Table~\ref{table:1}. An equation of state is a functional relation among them.

    \begin{table}[h!]
        \centering
        \begin{tabular}{c | c }
            Extensive & Intensive \\
            \hline
            Energy $E$ & - \\ 
            Entropy $S$ & Temperature $T$ \\ 
            Volume $V$ & Pression $p$\\ 
            Number of particles $E$ & Chemical potential $\mathbf \mu$ \\ 
            Polarization $\mathbf P$ & Electric field $\mathbf E$ \\ 
            Magnetization $\mathbf M$ & Magnetic field $\mathbf B$ \\ 
        \end{tabular}
        \caption{Extensive and intensive thermodynamical variables.}
        \label{table:1}
    \end{table}

    Thermodynamics is described by four laws.

    \begin{law}[0th]
        Two systems in thermal contact have the same empirical temperature $T$ at equilibrium
        \begin{equation*}
            T_1 = T_2
        \end{equation*}.
    \end{law}

    \begin{law}[1st]
        The (generalised) principle of conservation of energy states that
        \begin{equation}\label{first}
            d E = \delta Q - \delta L + \mu dN
        \end{equation}
        where $E$ is the internal energy, $Q$ is the heat, $L$ is the work, $\mu$ is the chemical potential (the necessary energy to add or remove a particle) and $N$ is the number of particles.
    \end{law}

    Recall that $E$ is a exact differential, i.e $\oint d E = 0$, whereas heat and word are not, i.e $\oint \delta Q \neq 0$ and $\oint \delta H \neq 0$.

    \begin{law}[2nd]
        A system naturally evolves in order to maximize its entropy $S$. For reversible processes
        \begin{equation}\label{second}
            dS = \frac{\delta Q}{T}
        \end{equation}
        whereas for irreversible processes
        \begin{equation*}
           dS \ge \frac{\delta Q}{T}
        \end{equation*}
    \end{law}

    \begin{law}[3rd]
        For any reversible isothermal process
        \begin{equation*}
        \Delta S \rightarrow 0 ~\textnormal{as}~ T \rightarrow 0
        \end{equation*}
    \end{law}


\section{Thermodynamical potentials}

    For reversible processes, using~\eqref{second} and $\delta L = p dV$,~\eqref{first} can be expressed as 
    \begin{equation}\label{energy}
        dE = T dS - pdV + \mu dN
    \end{equation}
    Notice that the left variables are intensive and the right variables (those with the differential) are extensive. 

    $E$ is a function of $S$, $V$, $N$, hence it must be extensive and a homogeneous function of degree one, satisfying the property
        \begin{equation*}
            E(\lambda S, ~\lambda V, ~\lambda N) = \lambda E(S, ~V, ~N)
        \end{equation*}
    where $\lambda > 0$ is the scale factor. It can be proved that the only function is 
        \begin{equation*}
            E(S, ~V, ~N) = TS - pV + \mu N
        \end{equation*}

    Similar expression can be found for other thermodynamical quantities, simply exchanging the role of conjugate functions. See Table~\ref{table:2}.

    \begin{table}[h!]
        \centering
        \begin{tabular}{c | c}
        Potentials & Differential \\
        \hline
        Internal energy $E(S, ~V, ~N) = TS - pV + \mu N$ & $dE = TdS - pdV + \mu dN$ \\ 
        Helmotz free energy $F(T, ~V, ~N) = E - TS = -pV + \mu N$ & $dF = -SdT - pdV + \mu dN$ \\ 
        Entalpy $H(S, ~p, ~N) = E + pV = St + \mu N$ & $dH = TdS + Vdp + \mu dN$ \\ 
        Gibbs free energy $G(T, ~p, ~N) = E - TS + pV = \mu N$ & $dG = -SdT + V dp + \mu dN$ \\ 
        Granpotential $\Omega (T, ~V, ~\mu) = E - TS - \mu N = -pV$ & $d\Omega = -SdT - pdV - N d \mu$ \\ 
        \end{tabular}
    \caption{Thermodynamical potentials.}
    \label{table:2}
    \end{table}

    \begin{proof}
        Maybe in the future.
    \end{proof}

    Fixing three of the thermodynamical variables to be constant, a system evolves in order to minimises the corresponding thermodynamical potential until it reaches its minimum, i.e the equilibrium state. Mathematically, it means that the first derivative must be vaninshing and the hessian must be positive defined. See Table~\ref{table:3}.

    \begin{table}[h!]
        \centering
        \begin{tabular}{c | c | c}
        Inequality & Constant quantities & \\
        \hline
        $d E \leq 0$ & $S, V, N$\\ 
        $d F \leq 0$ & $T, V, N$\\ 
        $d H \leq 0$ & $S, p, N$\\ 
        $d G \leq 0$ & $T, p, N$\\ 
        $d \Omega \leq 0$ & $T, V, \mu$\\ 
        \end{tabular}
    \caption{Thermodynamical variation principles.}
    \label{table:3}
    \end{table}

    \begin{proof}
        Maybe in the future.
    \end{proof}
\part{Classical statistical mechanics}

\chapter{Classical mechanics}

    \begin{equation}\label{norm}
        \int_{\mathcal M^N} \rho = 1
    \end{equation}
    \begin{equation}\label{T}
        \pdv{S}{E} = \frac{1}{T}
    \end{equation}
    \begin{equation}\label{F}
        \pdv{F}{T} = - S
    \end{equation}

\chapter{Microcanonical ensemble}

    A microcanonical ensemble is a system which is isolated from the environment, i.e. it cannot exchange neither energy nor matter, so $E$, $N$ and $V$ are fixed. Since energy is conserved and the hamiltonian is time-independent, the trajectory of motion is restricted on the surface $S_E$ and not on all the phase space.
    
    Assume an a-priory uniform probability 
    \begin{equation*}
        \rho_{mc}(q^i, ~p_i) = C \delta (\mathcal H(q^i, p_i) - E)
    \end{equation*}
    where $C$ is a normalisation constant, which can be evaluated by~\eqref{norm}
    \begin{equation*}
        1 = \int_{\mathcal M^N} d\Omega \rho_{mc} = \int_{\mathcal M^N} d\Omega C \delta(\mathcal H - E) = C \int_{\mathcal M^N} d\Omega \delta(\mathcal H - E) = C \omega(E)
    \end{equation*}

    Hence
    \begin{equation*}
        \rho_{mc}(q^i, ~p_i) = \frac{1}{\omega(E)} \delta (\mathcal H(q^i, p_i) - E)
    \end{equation*}

    Consider a displacement on an infinitesimal displacement of energy $\Delta E \ll 1$, then 
    \begin{equation*}
        \Gamma (E) = \integ{E}{E+dE}{E'} \omega(E') \simeq \omega(E) \Delta E
    \end{equation*}
    and the distribution is 
    \begin{equation*}
        \rho_{mc}(q^i, p_i) = \begin{cases}
            \frac{1}{\Gamma(E)} & \mathcal H \in [E, E + \Delta E] \\
            0 & otherwise
        \end{cases}
    \end{equation*}

    Let $f(q^i, p_i)$ be an observable, then its microcanonical average is 
    \begin{equation}\label{obs}
        \avp{f(q^i, p_i)}{mc} = \int_{\mathcal M} d\Omega ~ \rho_{mc} f = \int_{\mathcal M} d\Omega ~ \frac{1}{\omega(E)} \delta (\mathcal H - E) f = \frac{1}{\omega(E)} \int_{S_E} dS_E ~ f = \avp{f}{E} 
    \end{equation}

\section{Thermodynamics potentials}

    The microcanonical entropy $S_{mc}$ is defined by 
    \begin{equation*}
        S_{mc} (E, V, N) = k_B \ln \omega(E)
    \end{equation*}

    The logarithm is justified by the fact that the volume of a N-particle phase space is $(W_1)^N$, where $W_1$ is the volume of a single particle phase space. According to the properties of the logarithm, entropy becomes extensive.

    In the thermodynamic limit, the following equations hold 
    \begin{equation*}
        s_{mc} = \lim_{td} \frac{S_{mc}}{N} = k_B \lim_{td} \frac{\log \omega(E)}{N} = \underbrace{k_B \lim_{td} \frac{\log \Sigma(E)}{N}}_{\mathcal H \in [0, E]} = \underbrace{k_B \lim_{td} \frac{\log \Gamma(E)}{N}}_{\mathcal H \in [E, E + \Delta E]}
    \end{equation*}

    Entropy is additive, so given two sistems $1$ and $2$
    \begin{equation*}
        s_{mc}^{tot} = s_{mc}^{(1)} + s_{mc}^{(2)}
    \end{equation*}

    \begin{proof}
        Consider two isolated systems in contact at equilibrium with the same temperature $T = T_1 = T_2$. The total energy is $E = E_1 + E_2 + E_{surface}$ but, in the thermodynamic limit, the energy exchanged by the surface is a subleading term ($E_1$ and $E_2$ go as $L^3$ whereas $E_{surface}$ goes as $L^2$) and can be neglected. The energy density is 
        \begin{equation*}
        \begin{aligned}
            \omega(E) & = \int_{\mathcal M^N} d\Gamma_1 d\Gamma_2 \delta(\mathcal H - E) \\ & = \int dE_1 \int dS_{E_1} \int dE_2 \int dS_{E_2} \delta (E - E_1 - E_2) \\ & = \int dE_1 \int dE_2 \omega_1(E_1) \omega_2(E_2) \delta (E - E_1 - E_2) \\ & = \integ{0}{E}{E_1} \omega_1(E_1) \omega_2(E_2 = E - E_1)
        \end{aligned}
        \end{equation*}
        Since the integrand is a positive function with a maximum in $_1 \in [0, E]$
        \begin{equation}\label{proof1}
        \begin{aligned}
            \integ{0}{E}{E_1} \omega_1(E_1) \omega_2(E_2 = E - E_1) & \leq \omega_1(E^*_1) \omega_2(E^*_2 = E - E^*_1) \integ{0}{E}{E_1} \\ & = \omega_1(E^*_1) \omega_2(E^*_2 = E - E^*_1) E
        \end{aligned}
        \end{equation}

        On the other hand, it is always possible to find a value for $\Delta E$ in order to have 
        \begin{equation} \label{proof2}
            \Delta E \omega_1(E^*_1) \omega_2(E^*_2) \leq \omega(E)
        \end{equation}

        Putting together~\eqref{proof1} and~\eqref{proof2}
        \begin{equation*}
            \Delta E \omega_1(E^*_1) \omega_2(E^*_2) \leq \omega(E) \leq \omega_1(E^*_1) \omega_2(E^*_2) E
        \end{equation*}
        \begin{equation*}
            \omega_1(E^*_1) \Delta E \omega_2(E^*_2) \Delta E \leq \omega(E) \Delta E \leq \frac{E}{\Delta E} \omega_1(E^*_1) \Delta E \omega_2(E^*_2) \Delta E
        \end{equation*}
        \begin{equation*}
            \Gamma_1(E^*_1) \Gamma(E^*_2) \leq \Gamma(E) \leq \frac{E}{\Delta E}\Gamma(E^*_1) \Gamma(E^*_2)
        \end{equation*}
        Since the logarithm is a monotomic function
        \begin{equation*}
            \log \Big ( \Gamma_1(E^*_1) \Gamma(E^*_2) \Big ) \leq \log \Gamma(E) \leq \log \Big ( \frac{E}{\Delta E}\Gamma(E^*_1) \Gamma(E^*_2) \Big )
        \end{equation*}
        \begin{equation*}
            k_B \log \Big ( \Gamma_1(E^*_1) \Gamma(E^*_2) \Big ) \leq k_B \log \Gamma(E) \leq k_B \log \Big ( \frac{E}{\Delta E}\Gamma(E^*_1) \Gamma(E^*_2) \Big )
        \end{equation*}
        \begin{equation*}
            k_B \log \Gamma_1(E^*_1) + k_B \log \Gamma(E^*_2) \leq k_B \log \Gamma(E) \leq k_B \log \frac{E}{\Delta E} + k_B \log \Gamma(E^*_1) + k_B \log \Gamma(E^*_2)
        \end{equation*}
        \begin{equation*}
            \frac{k_B \log \Gamma_1(E^*_1) + k_B \log \Gamma(E^*_2)}{N} \leq \frac{k_B \log \Gamma(E)}{N} \leq \frac{k_B \log \frac{E}{\Delta E} + k_B \log \Gamma(E^*_1) + k_B \log \Gamma(E^*_2)}{N}
        \end{equation*}

        In the thermodynamic limit, the last term vanishes, since $\lim_{td} \frac{1}{N} \log \frac{N}{\Delta N} = 0$. Hence 
        \begin{equation*}
            s_{mc}(E) = s_{mc}^{(1)} + s_{mc}^{(2)}
        \end{equation*}
 
    \end{proof}

    The last result tells also that at equilibrium entropy is maximum.

    In the thermodynamic limit, microcanonical entropy coincides with the thermodynamical one 
    \begin{equation*}
        s_{mc} = s_{td}
    \end{equation*}

    \begin{proof}
        Since entropy is maximum at equilibrium, also $\Gamma_1(E_1) \Gamma_2(E_2)$ is so and
        \begin{equation*}
        \begin{aligned}
            0 & = \delta (\Gamma_1(E^*_1) \Gamma_2(E^*_2 = E - E^*_1)) \\ & = \delta \Gamma_1(E^*_1) \Gamma_2 (E^*_2) + \Gamma_1(E^*_1) \delta \Gamma_2 (E^*_2) \\ & = \pdv{\Gamma_1}{E_1} \Big\vert_{E^*_1} \delta E_1 \Gamma_2 (E^*_2) + \Gamma_1(E^*_1) \pdv{\Gamma_2}{E_2} \Big\vert_{E^*_2} \delta E_2 
        \end{aligned}
        \end{equation*}

        Since $E = const$, $0 = \delta E = \delta E_1 + \delta E_2$, $\delta E_2 = \delta E_1$ and
        \begin{equation*}
            0 = \pdv{\Gamma_1}{E_1} \Big\vert_{E^*_1} \delta E_1 \Gamma_2 (E^*_2) - \Gamma_1(E^*_1) \pdv{\Gamma_2}{E_2} \Big\vert_{E^*_2} \delta E_1 
        \end{equation*}
        \begin{equation*}
            0 = \pdv{\Gamma_1}{E_1} \Big\vert_{E^*_1} \Gamma_2 (E^*_2) - \Gamma_1(E^*_1) \pdv{\Gamma_2}{E_2} \Big\vert_{E^*_2} 
        \end{equation*}
        \begin{equation*}
            \pdv{\Gamma_1}{E_1} \Big\vert_{E^*_1} \Gamma_2 (E^*_2) = \Gamma_1(E^*_1) \pdv{\Gamma_2}{E_2} \Big\vert_{E^*_2} 
        \end{equation*}
        \begin{equation*}
            \frac{1}{\Gamma_1 (E^*_1)} \pdv{\Gamma_1}{E_1} \Big\vert_{E^*_1} = \frac{1}{\Gamma_2 (E^*_2)} \pdv{\Gamma_2}{E_2} \Big\vert_{E^*_2} 
        \end{equation*}
        \begin{equation*}
            \pdv{\log \Gamma_1}{E_1} \Big\vert_{E^*_1} = \pdv{\log \Gamma_2}{E_2} \Big\vert_{E^*_2} 
        \end{equation*}

        Using the thermodynamical relation~\eqref{T}
        \begin{equation*}
            S_{mc} (E) = S_{td} (E) \times const
        \end{equation*}
        where the constant can be chosen in order to have $k_B$ in the same unit.
    \end{proof}

    The universal Boltzmann's formula is 
    \begin{equation*}
        s_{mc} = s_{td} = k_B \log \omega(E) = - k_B \avp{\log \rho_{mc}}{mc}
    \end{equation*}

    \begin{proof}
        
    Using~\eqref{obs}, 
    \begin{equation*}
    \begin{aligned}
        \avp{\log \rho_{mc}}{mc} & = \int d\Gamma \rho_{mc} \log \rho_{mc} \\ & = \int d\Gamma \frac{1}{\omega(E)} \delta (\mathcal H - E) \log \Big ( \frac{1}{\omega(E)} \delta (\mathcal H - E) \Big) \\ & = \int dS_E \frac{1}{\omega(E)} \log \frac{1}{\omega(E)} \\ & = - \frac{1}{\omega(E)} \log \omega(E) \int dS_E \\ & = - \log \omega (E)
    \end{aligned}
    \end{equation*}
    \end{proof}



\chapter{Canonical ensemble}

    A canionical ensemble is a system which is immersed in a bigger environment or reservoir, which can exchange energy but not matter, so $T$, $N$ and $V$ are fixed. Globally, energy is conserved, since the universe composed by the union of the system and the environment can be considered as a microcanonical ensemble. 

    The canonical probability density distribution is 
    \begin{equation*}
        \rho_c (q^i, p_i) = \frac{1}{Z_N} \exp (-\beta \mathcal H(q^i, p_i))
    \end{equation*}
    where $\beta$ is 
    \begin{equation*}
        \beta = \frac{1}{k_B T}
    \end{equation*}
    and $Z_N$ is the partition function 
    \begin{equation}
        Z_N[V, T] = \int_{\mathcal M^N} d\Omega ~\exp (-\beta \mathcal H(q^i, p_i))
    \end{equation}
    which depends on the temperature through $\beta$ and volume and temperature due to the integration domain $\mathcal M^N = V \otimes \mathbb R^d$.

    Notice that the probability is a function of the hamiltonian, like Liouville's theorem said.

    \begin{proof}
        Consider the universe as a microcanonical ensemble. Its probability density distribution is 
        \begin{equation*}
            \rho_{mc} (q_i^{(1)}, p_i^{(1)}, q_i^{(2)}, p_i^{(2)}) = \frac{1}{\omega(E)} \delta (\mathcal H (q_i^{(1)}, p_i^{(1)}, q_i^{(2)}, p_i^{(2)}) - E)
        \end{equation*}
        where the total hamiltonian is 
        \begin{equation*}
            \mathcal H (q_i^{(1)}, p_i^{(1)}, q_i^{(2)}, p_i^{(2)}) = \mathcal H_1 (q_i^{(1)}, p_i^{(1)}) + \mathcal H_2 (q_i^{(2)}, p_i^{(2)})
        \end{equation*}

        Integrating it to all the possible state in the environment
        \begin{equation*}
            \rho^{(1)} = \int d\Omega_2  \rho_{mc} = \int d\Omega_2 \frac{1}{\omega(E)} \delta(\mathcal H - E) = \frac{1}{\omega(E)} \int dS_{E_2} = \frac{1}{\omega(E)} \omega(E_2 = E - E_1)
        \end{equation*}
        and the corresponding entropy is 
        \begin{equation*}
            S_2 (E_2) = k_B \ln \omega_2 (E_2)
        \end{equation*}

        Applying small variation $\delta E_1$ to $E_1$ to preserve equilibrium, the entropy trasforms, using~\eqref{T}
        \begin{equation*}
            k_B \ln \omega_2 (E_2) = S_{mc}(E) - E_1 \pdv{S_{mc}}{E} \Big \vert_{E_2} = S_{mc}(E) - E_1 \frac{1}{T} 
        \end{equation*}
        \begin{equation*}
            \ln \omega_2 (E_2) = \frac{S_{mc}(E)}{k_B} - E_1 \frac{1}{k_B T} 
        \end{equation*}
        \begin{equation*}
            \omega_2 (E_2) = \exp (\frac{S_{mc}(E)}{k_B} - E_1 \frac{1}{k_B T}) = \exp (\frac{S_{mc}(E)}{k_B}) \exp (- \frac{E_1}{k_B T}) 
        \end{equation*}

        Putting together, dropping the indices
        \begin{equation}
            \rho_c = \frac{\omega(2)(E_2)}{\omega(E)} = \frac{1}{\omega(E)} \exp (\frac{S_{mc}(E)}{k_B}) \exp (- \frac{E_1}{k_B T}) = C \exp (- \frac{E_1}{k_B T})
        \end{equation}
        where $C$ is a normalisation constant, which can be evaluated by~\eqref{norm}
        \begin{equation*}
            1 = \int_{\mathcal M^N} d\Omega \rho = \int_{\mathcal M^N} d\Omega C \exp (- \frac{E_1}{k_B T}) = C \int_{\mathcal M^N} d\Omega \exp (- \frac{E_1}{k_B T}) 
        \end{equation*}
    \end{proof}

    The partition function can also be written as 
    \begin{equation*}
        Z_N[T, V] = \integ{0}{\infty}{E} \omega(E) \exp (-\beta E)
    \end{equation*}

    \begin{proof}
        Foliating the phase space in energy hyper-surfaces 
        \begin{equation*}
            Z_N = \int_{\mathcal M^N} d\Omega \exp (- \beta \mathcal H) = \integ{0}{\infty}{E} \int dS_E ~ \exp (-\beta \mathcal H) = \integ{0}{\infty}{E} \omega(E) \exp (-\beta E)
        \end{equation*}
    \end{proof}

    Taking also in consideration indistinguishable particles, the partition function 
    \begin{equation*}
        Z_N = \int \frac{\prod_{i=1}^N d^d q^i d^d p^i}{h^{dN} \zeta_N} \exp (- \beta \mathcal H) 
    \end{equation*}
    where $\zeta_N$ is 
    \begin{equation*}
        \zeta_N = \begin{cases}
            1 & \textnormal{distinguishable} \\
            N! & \textnormal{indistinguishable}
        \end{cases}
    \end{equation*}

    The partition function of two systems is the multiplication of the single system ones
    \begin{equation}
        Z_N = Z_{N_1} Z_{N_2}
    \end{equation}

    \begin{proof}
        Since $\mathcal H = \mathcal H_1 + \mathcal H_2$, 
        \begin{equation*}
        \begin{aligned}
        \end{aligned}
        \end{equation*}
    \end{proof}

    If the hamiltonian is the sum of $N$ identical ones, like $N$ non-interacting particles
    \begin{equation*}
        \mathcal H = \sum_{i = 1}^{N} \mathcal H_i
    \end{equation*} 
    the partition function becomes 
    \begin{equation*}
        Z_N = \frac{(Z_1)^N}{\zeta_N}
    \end{equation*}

    \begin{proof}
        Denominating $Z_1$ the single-particle partition function
        \begin{equation*}
        \begin{aligned}
            Z_N & = \int_{\mathcal M^N = \mathcal M^{(1)} \otimes \ldots \otimes \mathcal M^{(1)}} \prod_{i=1}^N \frac{d^d q^i d^d p^i}{h^{dN} \zeta_N} \exp (-\beta \mathcal H) \\ & = \int_{\mathcal M^N = \mathcal M^{(1)} \otimes \ldots \otimes \mathcal M^{(1)}} \prod_{i=1}^N \frac{d^d q^i d^d p^i}{h^{dN} \zeta_N} \exp (-\beta \sum_{i = 1}^{N} \mathcal H_i) \\ & = \int_{\mathcal M^N = \mathcal M^{(1)} \otimes \ldots \otimes \mathcal M^{(1)}} \prod_{i=1}^N \frac{d^d q^i d^d p^i}{h^{dN} \zeta_N} \prod_{i=1}^{N}\exp (-\beta \mathcal H_i) \\ & = \int_{\mathcal M^N = \mathcal M^{(1)} \otimes \ldots \otimes \mathcal M^{(1)}} \prod_{i=1}^N \frac{d^d q^i d^d p^i}{h^{dN} \zeta_N} \exp (-\beta \mathcal H_i) \\ & = \frac{Z_1 Z_1 \ldots Z_1}{\zeta_N} = \frac{(Z_1)^N}{\zeta_N}
        \end{aligned}
        \end{equation*}
    \end{proof}

    Let $f(q^i, p_i)$ be an observable, then its canonical average is 
    \begin{equation*}\label{obsc}
        \avp{f(q^i, p_i)}{c} = \int_{\mathcal M} d\Omega ~ \rho_{c} f = \int_{\mathcal M} d\Omega ~ \frac{\exp (-\beta \mathcal H)}{Z_N} f
    \end{equation*}

\section{Thermodynamics variable}

    The canonical Helmotz free energy $F$ is defined by 
    \begin{equation*}
        Z_[V, T] = \exp(-\beta F[N, V, T])
    \end{equation*}
    or, equivalently,
    \begin{equation}\label{Fc}
        F[V, N ,T] = -\frac{1}{\beta} \ln Z_N
    \end{equation}

    Furthermore, the canonical internal energy is 
    \begin{equation}\label{Ec}
        E = \avp{\mathcal H}{c} = \int d\Omega \frac{\exp(-\beta (\mathcal H))}{Z_N} \mathcal H
    \end{equation}

    \begin{proof}
        By normalisation condition 
        \begin{equation*}
            1 = \int d\Omega \frac{\exp(-\beta \mathcal H)}{Z_N} = \int d\Omega \frac{\exp(-\beta \mathcal H)}{\exp(-\beta F)} = \int d\Omega \exp (- \beta (\mathcal H - F))
        \end{equation*}

        Since $F$ depends on the temperature, it is possible to derive with respect to $\beta$
        \begin{equation*}
        \begin{aligned}
            0 & = \pdv{}{\beta} \Big ( \int d\Omega \exp (- \beta (\mathcal H - F)) \Big) \\ & = \int d\Omega \exp (-\beta (\mathcal H - F)) \Big (-(\mathcal H - F) + \beta \pdv{F}{\beta}) \\ & = - \underbrace{\int d\Omega \frac{\exp(-\beta \mathcal H)}{Z_N} \mathcal H}_{E} + F \underbrace{\int d\Omega \frac{\exp(-\beta \mathcal H)}{Z_N}}_{1} + \beta \pdv{F}{\beta} \underbrace{\int d\Omega \frac{\exp(-\beta \mathcal H)}{Z_N}}_{1} \\ & = - E + F + \beta \pdv{F}{\beta}
        \end{aligned}
        \end{equation*}
        Hence, using~\eqref{F}
        \begin{equation*}
            F = E + \beta \pdv{F}{\beta} = E + T \pdv{F}{T} = E - TS
        \end{equation*}
        showing that is indeed the Helmotz free energy.
    \end{proof}

    Notice that in the last result, the entropy can be also written as 
    \begin{equation} \label{Sc}
        S_c = \frac{E - F}{T}
    \end{equation}
    
    The internal energy can also be written as 
    \begin{equation*}
        E = - \pdv{}{\beta} \ln Z_N
    \end{equation*}

    \begin{proof}
        Using~\eqref{Ec},
        \begin{equation*}
            - \pdv{}{\beta} \ln Z_N = - \frac{1}{Z_N} \pdv{Z_N}{\beta} = - \frac{1}{Z_N} \pdv{}{\beta} \int d\Omega \exp (-\beta \mathcal H) = \int d\Omega \frac{\exp(-\beta \mathcal H)}{Z_N}  \mathcal H = \avp{\mathcal H}{c} = E
        \end{equation*}
    \end{proof}

    The universal Boltzmann's formula is still valid
    \begin{equation*}
        S_c = -k_B \avp{\ln \rho_c}{c} 
    \end{equation*}


    \begin{proof}
        Using~\eqref{Ec} and~\eqref{Fc}
        \begin{equation*}
            \begin{aligned}
            -k_B \avp{\ln \rho_c}{c} & = -k_B \int d\Omega \rho_c \ln \rho_c \\ & = -k_B \int d\Omega \rho_c \ln \frac{\exp(-\beta \mathcal H)}{Z_N} \\ & = -k_B \int d\Omega \rho_c \ln \exp(-\beta \mathcal H) - k_B \int d\Omega \rho_c \ln Z_N \\ & = k_B \int d\Omega \beta \mathcal H - k_B \underbrace{\ln Z_N}_{\beta F} \underbrace{\int d\Omega \rho_c}_{1} \\ & =  \frac{E - F}{T} = S_c
        \end{aligned}
        \end{equation*}
    \end{proof}

\section{Equipartition theorem}

    \begin{theorem}[Generalised equipartition theorem]
        Let $\xi \in [a,b]$ and $\xi_j$ with $j \neq 1$ all the other coordinates or momenta. Suppose also 
        \begin{equation}\label{cond}
            \int \prod_{j \neq 1} d \xi_j [\xi_1 \exp(-\beta \mathcal H)]_a^b = 0
        \end{equation}
        Then 
        \begin{equation*}
            \avp{\xi_1 \pdv{\mathcal H}{\xi_1}}{c} = k_B T
        \end{equation*}
    \end{theorem}

    \begin{proof}
        By normalisation condition 
        \begin{equation*}
            1 = \int d\Omega \frac{\exp(-\beta \mathcal H)}{Z_N} = \frac{1}{Z_N} \int \prod_{j \neq 1} d \xi_j \exp(-\beta \mathcal H)
        \end{equation*}
        Using
        \begin{equation*}
            d\xi_1 (\xi_1 \exp(-\beta \mathcal H)) = d\xi_1 \exp(-\beta \mathcal H) + \xi \exp(-\beta \mathcal H) (-\beta) \pdv{\mathcal H}{\xi_1} d\xi_1
        \end{equation*}
        and integrating per parts
        \begin{equation*}
        \begin{aligned}        
            1 & = \frac{1}{Z_N} \underbrace{\int \prod_{j \neq 1} d \xi_j [\xi_1 \exp(-\beta \mathcal H)]_a^b}_{0} + \frac{\beta}{Z_N} \int \prod_{j \neq 1} d \xi_j d\xi_1 \xi_1 \pdv{\mathcal H}{\xi_1} \exp (-\beta \mathcal H) \\ & = \beta \int d\Omega \xi_1 \pdv{\mathcal H}{\xi_1} \frac{\exp (- \beta \mathcal H)}{Z_N} \\ & = \beta \avp{\xi_1 \pdv{\mathcal H}{\xi_1}}{c}
        \end{aligned}
        \end{equation*}
        Hence
        \begin{equation*}
            \avp{\xi_1 \pdv{\mathcal H}{\xi_1}}{c} = \frac{1}{\beta} = k_B T
        \end{equation*}
    \end{proof}

    Examples of system that satisfies the condition~\eqref{cond} are hamiltonians which depend on the square of momentum or confining potentials which go to infinity on the extremes $a$ and $b$. 

\part{Applications of classical statistical mechanics}

\chapter{Microcanonical ensemble}
\section{Non-relativistic ideal gas}
\section{Gas of harmonic oscillators}
\section{$2$-levels system}

\chapter{Canonical ensemble}
\section{Non-relativistic ideal gas}
\section{Gas of harmonic oscillators}
\section{Ultra-relativistic ideal gas}
\section{Maxwell-Boltzmann velocity distribution}
\section{Magnetic solid}

\chapter{Grancanonical ensemble}
\section{Non-relativistic ideal gas}
\section{Gas of harmonic oscillators}
\section{Solid-vapor equilibrium phase}

\chapter{Entropy}
\section{Maxwell-Boltzmann distribution}
\section{Fermi-Dirac distribution}
\section{Bose-Einstein distribution}

    
\part{Quantum statistical mechanics}

\chapter{Quantum mechanics}

    In quantum mechanics, a pure state is described by a normalised vector in a Hilbert space $\ket{\psi} \in \mathcal H$, which is a vector space on $\mathbb C$, i.e.~in which a linear superposition is still in the space $\lambda_1 \ket{\psi_1} + \lambda_2 \ket{\psi_2}$, endowed with a scalar product $\braket{\psi}{\phi}$ through which it is possible to associate a probability and the normalisation condition $||\psi||^2 = \braket{\psi}{\psi} = 1$. Furthermore, the normalisatin condition ensures that a state is not only a vector, but a ray in a Hilbert space, since two states are physically equivalent if $\ket{\psi} \sim \exp(i \phi) \ket{\psi}$. It can be seen as an equivalence class of states.

\section{Projectors}

    It is possible to uniquely determine the state via a projection operator or projector 
    \begin{equation*}
        P_\psi = \frac{\ket{\psi} \bra{\psi}}{\braket{\psi}{\psi}} ~,
    \end{equation*}
    which for normalisation states becomes 
    \begin{equation*}
        P_\psi = \ket{\psi} \bra{\psi} ~.
    \end{equation*}
    \begin{proof}
        If $\ket{\psi} \mapsto \exp(i \phi) \ket{\psi}$ and $\bra{\psi} \mapsto \exp(- i \phi) \bra{\psi}$, we have 
        \begin{equation*}
            P_\psi \mapsto \cancel{\exp(i \phi)} \ket{\psi} \cancel{\exp(- i \phi)} \bra{\psi} = \ket{\psi} \bra{\psi} = P_\psi ~.
        \end{equation*}
    \end{proof}
    It projects onto the $1$-dimensional subspace generated by the state $\ket{\psi}$
    \begin{equation*}
        P_\psi \colon \mathcal H \rightarrow \mathcal H_\psi ~,
    \end{equation*}
    where $\mathcal H_\psi = \{\lambda \ket{\psi} \colon \lambda \in \mathbb C\}$.
    \begin{proof}
        In fact, $\forall \ket{\psi} \in \mathcal H$, we decomposed it into 
        \begin{equation*}
            \ket{\psi} = \alpha \ket{\psi} + \beta \ket{\psi^\perp}
        \end{equation*}
        and the action of the projector
        \begin{equation*}
            P_\psi \ket{\psi} = \alpha \underbrace{P_\psi}_{\ket{\psi} \bra{\psi}} \ket{\psi} + \beta \underbrace{P_\psi \ket{\psi^\perp}}_0 = \alpha \ket{\psi} \underbrace{\braket{\psi}{\psi}}_1 = \alpha \ket{\psi} ~.
        \end{equation*}
    \end{proof}

    The projector is also called density matrix $\rho_\psi$. It satisfies the following properties 
    \begin{enumerate}
        \item boundness, i.e. 
            \begin{equation*}
                ||\rho_\psi|| < C~,
            \end{equation*}
        \item hermiticity, i.e. 
            \begin{equation*}
                \rho_\psi^\dagger = \rho_\psi ~,
            \end{equation*}
        \item idempotence, i.e. 
            \begin{equation*}
                \rho_\psi^2 = \rho_\psi ~,
            \end{equation*}
        \item positive defined, i.e. $\forall \ket{\phi} \in \mathcal H$
            \begin{equation*}
                \bra{\phi} \rho_\psi \ket{\phi} \geq 0 ~,
            \end{equation*}
        \item unit trace, i.e. 
            \begin{equation*}
                \tr \rho_\psi = 1 ~.
            \end{equation*}
    \end{enumerate}
    Actually, there is a theorem which ensures that an operators such that it satifies these 5 conditions is indeed the projector.
    \begin{proof}
        For the hermiticity
        \begin{equation*}
            \rho_\psi^\dagger = (\ket{\psi} \bra{\psi})^\dagger = \bra{\psi}^\dagger \ket{\psi}^\dagger = \ket{\psi} \bra{\psi} = \rho_\psi ~.
        \end{equation*}

        For the idempotence
        \begin{equation*}
            \rho_\psi^2 = (\ket{\psi} \bra{\psi})^2 = \ket{\psi} \underbrace{\braket{\psi}{\psi}}_1 \bra{\psi} = \ket{\psi} \bra{\psi} = \rho_\psi ~.
        \end{equation*}
    \end{proof}

    Given an orthonormal basis $\{\ket{e_n}\}_{n=1}^\infty$ of a separable Hilbert space, the trace is defined as 
    \begin{equation*}
        \tr A = \sum_{n=1}^\infty A_{nn} = \sum_{n=1}^\infty \bra{e_n} A \ket{e_n} ~.
    \end{equation*}
    If it convergent, it is called a trace-class operator. Furthermore, if it is absolute convergent, the trace is independent on the choice of the basis.

\section{Observable}

    An observable is a linear hermitian operator acting on the Hilbert space. We require the self-adjointness because its eigenvalues are real and it always admit an eigenbasis, such that every state can be expanded in this basis 
    \begin{equation*}
        A \ket{\psi_n} = \lambda_n \ket{\psi_n} ~,
    \end{equation*}
    such that 
    \begin{equation*}
        \lambda_n \in \mathbb R
    \end{equation*}
    and $\forall \ket{\phi} \in \mathcal H$
    \begin{equation*}
        \ket{\phi} = \sum_{n=1}^{\infty} c_n \ket{\psi_n} ~.
    \end{equation*}

    The projectors on the eigenstates are orthogonal 
    \begin{equation*}
        P_n P_m = P_n P_m = 0 ~.
    \end{equation*}

    If we prepare the system in a state $\ket{\psi}$, a measurement of an observable $A$ can have outcomes $\lambda_n$ with probability $P_n = |c_n|^2$ where we have defined 
    \begin{equation*}
        \ket{\phi} = \sum_n c_n \ket{\psi_n}~, \quad A \ket{\psi_n} = \lambda_n \ket{\psi_n} ~.
    \end{equation*}
    Its average value is 
    \begin{equation*}
        \av{A} = \sum_{n} \lambda_n P_n = \sum_{n} \lambda_n |c_n|^2 = \tr(A \rho_\psi) ~. 
    \end{equation*}

\section{Composite system}

    For $2$ particles, the total Hilbert space is the tensor product between the single particle Hilbert spaces 
    \begin{equation*}
        \mathcal H_{tot} = \mathcal H_1 \otimes \mathcal H_2 ~. 
    \end{equation*}
    Given an orthonormal basis for each Hilbert space $\{\ket{\psi_n}\} \in \mathcal H_1$ and $\{\ket{\phi_m}\} \in \mathcal H_2$, the total orthonormal basis is 
    \begin{equation*}
        \{\ket{\psi_n}_1 \ket{\phi_m}_2 = \ket{\psi_n \phi_m}\} 
    \end{equation*}
    such that a generic state can be expanded into this basis, $\forall \ket{\phi} \in \mathcal H_{tot}$
    \begin{equation*}
        \ket{\phi} = \sum_n \sum_m a_{nm} \ket{\psi_n \phi_m} ~,
    \end{equation*}
    where the normalisation condition is $\sum_{nm} |a_{nm}|^2 = 1$.

    If the $2$ particle are identical, we have $\mathcal H_1 = \mathcal H_2 = \mathcal H$. Therefore $\mathcal H_{tot} = \mathcal H^{\otimes 2}$.

    The scalar product is 
    \begin{equation*}
        \braket{\psi_n \phi_m}{\psi_{n'} \phi_{m'}} = \braket{\psi_n}{\psi_{n'}} \braket{\phi_m}{\phi_{m'}} ~.
    \end{equation*}

\section{N distinguishable particles}

    A single particle lives in $\mathbb R^3$ and its Hilbert space is $\mathcal H = L^2 (\mathbb R^3) \ni \psi(x)$. The scalar product is 
    \begin{equation*}
        \braket{\psi}{\phi} = \int d^3 x ~ \psi^*(x) \phi(x) ~,
    \end{equation*}
    where the normalisation condition is 
    \begin{equation*}
        ||\psi||^2 = \braket{\psi}{\psi} = \int_{\mathbb R^3} d^3 x ~ |\psi(x)|^2 < infty ~.
    \end{equation*}

    For $N$ distinguishable particles, the total Hilbert space is $\mathcal H_{tot} = \mathcal H \otimes \ldots \otimes \mathcal H$ and an orthonormal basis is $\ket{\psi_{n_1} \ldots \psi_{n_N}}$ where $\ket{\psi_{n_j}}$ is a single particle orthonormal basis. Hence, $N$ distinguishable particle live in $\mathbb R^{3N}$ and their Hilbert space is $\mathcal H_N = L^2(\mathbb R^3) \otimes \ldots \otimes L^2(\mathbb R^3) = L^2 (\mathbb R^{3N}) \ni \psi(x_1, \ldots x_N)$. Therefore, an orthonormal basis is $\{u_{\alpha_1 (x_1)} \ldots u_{\alpha_N (x_N)} = u_{\alpha_1 \ldots \alpha_N} (x_1, \ldots x_N)\}$ where $\{u_\alpha (x)\}$ is the single particle orthonormal basis.

    A generic state can be expanded in this basis as 
    \begin{equation*}
        \psi(x_1, \ldots x_N) = \sum_{\alpha_1 \ldots \alpha_N} c_{\alpha_1 \ldots \alpha_N} u_{\alpha_1 \ldots \alpha_N} (x_1, \ldots x_N) ~.
    \end{equation*}
    For instance, choosing $\alpha_1 = a$ and $\alpha_2 = b$ or viceversa
    \begin{equation*}
        u_{\alpha_1 = a} (x_1) u_{\alpha_2 = b} (x_2) \neq u_{\alpha_1 = b} (x_1) u_{\alpha_2 = a} (x_2) ~.
    \end{equation*}

    If the particle are indistinguishable, they are invariant under permutations
    \begin{equation*}\label{perm}
        \psi(x_1, \ldots x_N) \mapsto \psi(P(x_1, \ldots x_N)) = \exp(i \alpha_P) \psi (x_1, \ldots x_N) ~,
    \end{equation*}
    where $P$ belongs to the permutation group.

\chapter{Permutation group}

    The premutation of $N$ elements form a group $P_N$. This group is generated by transposition $\{\sigma_i\}_{i=1}^N$. In fact, any permutation can be defined by consecutive transposition, where a transposition is defined as 
    \begin{equation*}
        \sigma_i \colon (1,2,\ldots, i, i+1, \ldots N) \mapsto (1,2,\ldots, i+1, i, \ldots N) ~.
    \end{equation*}
    However, this decomposition is not unique but the number of transposition in its decomposition is always even or odd. Therefore, we can define the sign of permutation $\forall P \in P_N$
    \begin{equation*}
        sign(P) = \begin{cases}
            + 1 & \textnormal{even number of transposition in its decomposition } \\
            - 1 & \textnormal{odd number of transposition in its decomposition } \\
        \end{cases} ~.
    \end{equation*}

    Transpositions follow the properties 
    \begin{enumerate}
        \item if $|i - j| > 2$ \begin{equation}\label{prop1}
            \sigma_i \sigma_j = \sigma_j \sigma_i ~,
        \end{equation} 
        \item \begin{equation}\label{prop2}
            \sigma_i \sigma_{i+1} \sigma_i = \sigma_{i+1} \sigma_i \sigma_{i+1} ~,
        \end{equation}
        \item \begin{equation}\label{prop3}
            (\sigma_i)^2 = \mathbb I ~.
        \end{equation}
    \end{enumerate}

    Hence, we can calculate explicitly~\eqref{perm}, which is 
    \begin{equation}
        \alpha_P = \alpha_1 + \ldots \alpha_N~.
    \end{equation}
    \begin{proof}
        In fact
        \begin{equation*}
        \begin{aligned}
        \psi(P(x_1,\ldots x_N)) & = \psi((\sigma_{\alpha_1} \ldots \sigma_{\alpha_N}) (x_1,\ldots x_N)) \\ & = \exp (i \alpha_1) \psi((\sigma_{\alpha_2} \ldots \sigma_{\alpha_N}) ) \\ & ~~ \vdots \\ & = \exp (i \alpha_1) \ldots \exp (i \alpha_N) \psi(x_1,\ldots x_N) \\ & = \exp (i (\alpha_1 + \ldots \alpha_N)) \psi(x_1,\ldots x_N) \\ &  = \exp (i \alpha_P) \psi(x_1,\ldots x_N)~,
        \end{aligned}
        \end{equation*}
        where $P = \sigma_{\alpha_1} \ldots \sigma_{\alpha_N}$.
    \end{proof}

    Furthermore, $\alpha_P = 0, m\pi$, which correspond respectively to a bosonic totally symmetric wavefunction, i.e. under $P$
    \begin{equation*}
        \psi(x_1, \ldots x_N) \xmapsto{P} + 1 \psi(x_1, \ldots x_N) ~,
    \end{equation*}
    or to a fermionic totally antisymmetric wavefunction, i.e. under $P$ 
    \begin{equation*}
        \psi(x_1, \ldots x_N) \xmapsto{P} (-1)^m \psi(x_1, \ldots x_N) = \begin{cases}
            + & \textnormal{sign(P) = +1} \\
            - & \textnormal{sign(P) = -1} \\
        \end{cases}~.
    \end{equation*}

    \begin{proof}
        For~\eqref{prop1}
        \begin{equation*}
            \psi(x_1, \ldots x_N) \xmapsto{\sigma_i} \exp(i \alpha_i) \psi(x_1, \ldots x_N)\xmapsto{\sigma_i \sigma_j} \exp(i \alpha_i) \exp(i \alpha_j) \psi(x_1, \ldots x_N) ~,
        \end{equation*}
        \begin{equation*}
            \psi(x_1, \ldots x_N) \xmapsto{\sigma_j} \exp(i \alpha_j) \psi(x_1, \ldots x_N)\xmapsto{\sigma_j \sigma_i} \exp(i \alpha_j) \exp(i \alpha_i) \psi(x_1, \ldots x_N) ~,
        \end{equation*}
        which means that 
        \begin{equation*}
            \exp(i \alpha_i) \exp(i \alpha_j) = \exp(i \alpha_j) \exp(i \alpha_i) ~.
        \end{equation*}

        For~\eqref{prop2}
        \begin{equation*}
            \psi(x_1, \ldots x_N) \xmapsto{\sigma_i} \exp(i \alpha_i) \psi(x_1, \ldots x_N)\xmapsto{\sigma_i \sigma_{i+1}} \exp(i \alpha_i) \exp(i \alpha_{i+1}) \psi(x_1, \ldots x_N) \xmapsto{\sigma_i \sigma_{i+1} \sigma_i} \exp(i \alpha_i) \exp(i \alpha_{i+1}) \exp(i \alpha_i) \psi(x_1, \ldots x_N) ~,
        \end{equation*}
        \begin{equation*}
            \psi(x_1, \ldots x_N) \xmapsto{\sigma_{i+1}} \exp(i \alpha_{i+1}) \psi(x_1, \ldots x_N)\xmapsto{\sigma_{i+1} \sigma_i} \exp(i \alpha_{i+1}) \exp(i \alpha_i) \psi(x_1, \ldots x_N) \xmapsto{\sigma_{i+1} \sigma_i \sigma_{i+1} } \exp(i \alpha_{i+1}) \exp(i \alpha_i) \exp(i \alpha_{i+1}) \psi(x_1, \ldots x_N) ~,
        \end{equation*}
        which means that 
        \begin{equation*}
            \exp(i \alpha_i) \exp(i \alpha_{i+1}) \exp(i \alpha_i) = \exp(i \alpha_{i+1}) \exp(i \alpha_i) \exp(i \alpha_{i+1}) ~,
        \end{equation*}
        where we have used the fact that $\exp(i \alpha_i) \exp(i \alpha_j)$ commutes. Therefore, $\forall i= 1, \ldots N-1$ and $\alpha_i \in [0, 2\pi[$ we have $\alpha_i = \alpha_{i+1} = \alpha$.

        For~\eqref{prop3}
        \begin{equation*}
            \exp(i \alpha)^2 = \exp (2 i \alpha) = \mathbb I = \exp(0) ~,
        \end{equation*}
        which means that 
        \begin{equation*}
            \alpha = 0, 2\pi ~.
        \end{equation*}

        Finally, there are only two possibilities 
        \begin{equation*}
            \psi(x_1, \ldots x_N) \xmapsto{\sigma_i} \underbrace{\exp(i 0)}_{+1} \psi(x_1, \ldots x_N)
        \end{equation*}
        and 
        \begin{equation*}
            \psi(x_1, \ldots x_N) \xmapsto{\sigma_i} \underbrace{\exp(i \pi)}_{-1} \psi(x_1, \ldots x_N) ~.
        \end{equation*}
    \end{proof}

\section{Symmetric/antisymmetric Hilbert space} 

    Consider $2$ distinguishable particles. In general, the Hilbert space is $\mathcal H_{tot} = \mathcal H \otimes \mathcal H$. If the particle are indistinguishable, we can decomposed the Hilbert space into $\mathcal H_{tot} = \mathcal H_S \otimes_\perp  \mathcal H_A$. In fact, given two states $\ket{a}_1 \in \mathcal H_1$ and $\ket{b}_2 \in \mathcal H_2$, we have 
    \begin{equation*}
    \begin{aligned}
        \ket{a}_1\ket{b}_2 & = \frac{2}{2} \ket{a}_1\ket{b}_2 + \frac{1}{2} \ket{b}_1\ket{a}_2 - \frac{1}{2} \ket{b}_1\ket{a}_2 \\ & = \underbrace{\frac{\ket{a}_1\ket{b}_2 + \ket{b}_1\ket{a}_2}{2}}_{\ket{\psi_S}} + \underbrace{\frac{\ket{a}_1\ket{b}_2 - \ket{b}_1\ket{a}_2}{2}}_{\ket{\psi_A}} \\ & = \ket{\psi_S} + \ket{\psi_A} ~.
    \end{aligned}
    \end{equation*}
    Notice that Pauli's exclusion principle is encoded into the antysymmetric part, because if $a = b$ we have $\ket{\psi_A} = 0$. It is also an orthogonal decomposition. In fact 
    \begin{equation*}
    \begin{aligned}
        \braket{\psi_S}{\psi_A} & = \frac{\bra{a}_1\bra{b}_2 + \bra{b}_1\bra{a}_2}{2} \frac{\ket{a}_1 \ket{b}_2 - \ket{b}_1 \ket{a}_2}{2} \\ & = \frac{1}{4} (\underbrace{\braket{a}{a}_1}_1 \underbrace{\braket{b}{b}_2}_1 - \cancel{\braket{a}{b}_1 \braket{b}{a}_2} + \cancel{\braket{b}{a}_1 \braket{a}{b}_2} - \underbrace{\braket{b}{b}_1}_1 \underbrace{\braket{a}{a}_2}_1 ) = 0 ~. 
    \end{aligned}
    \end{equation*}

    The decomposition is equivalent to define two orthogonal projectors: the symmetriser 
    \begin{equation*}
        \hat S \colon \mathcal H \rightarrow mathcal H_S
    \end{equation*}
    and the antisymmetriser 
    \begin{equation*}
        \hat A \colon \mathcal H \rightarrow mathcal H_A ~,
    \end{equation*}
    such that they satisfy the properties 
    \begin{equation*}
        \hat S^\dagger = \hat S~, \quad \hat A^\dagger = \hat A~, \quad \hat S^2 = \hat S~, \quad \hat A^2 = \hat A~, \quad \hat S \hat A = \hat A \hat S = 0 ~.
    \end{equation*}

    Generalising for $N$ particles, we have $\mathcal H_{tot} = \mathcal H \otimes \ldots \mathcal H$ and a state is $\ket{a_1}_1 \ldots \ket{a_N}_N$ where $\ket{a_j} \in \mathcal H$. The symmetriser is 
    \begin{equation*}
        \hat S \colon \ket{\psi} \mapsto \frac{1}{N!} \sum_{P \in P_N} \ket{a_{P(1)}}_1 \ldots \ket{a_P(N)}_N
    \end{equation*}
    and the antisymmetriser is 
    \begin{equation*}
        \hat A \colon \ket{\psi} \mapsto \frac{1}{N!} \sum_{P \in P_N} sgn(P) \ket{a_{P(1)}}_1 \ldots \ket{a_P(N)}_N
    \end{equation*}
    where $P(1, \ldots N) \mapsto (P(1), \ldots P(N))$. They satisfy the orthogonal projector properties. Notice that for $N > 2$ particles, the total Hilbert space is $\mathcal H_{tot} = \mathcal H_S \otimes \mathcal H_A \otimes \mathcal H'$, where bosons work only in $\mathcal H_S$, fermions work only in $\mathcal H_A$ and $\mathcal H'$ is not physical.

    For distinguishable particles, we have $\mathcal H_{tot} = \mathcal H^{\otimes N}$ with orthonormal basis $\{u_{\alpha_1}(x_1) \ldots u_{\alpha_N}(x_N)\}_{\alpha_1, \ldots \alpha_N=0}^\infty$ labelled by the ordered set $(\alpha_1, \ldots \alpha_N)$. In this case, we are specifying which particle is in which states. However, for indistinguishable particles, we lose information because we know only how many particle are in each state. We label the states with $n_1, \ldots n_j$ with $j=1, \ldots \infty$, which are the occupation number. For bosons, we have $n_k = 0, 1, \ldots, \infty$, whereas for fermions, we have $n_k = 0, 1$. For both cases, there is the constrain $N = \sum_k n_k$, which is an infinite sum but mostly are zero occupied.

\chapter{Second quantisation} 

\section{Bosonic case}

    We define creation and annihilation operators such that they satisfies the properties 
    \begin{equation*}
        [\hat a, \hat a^\dagger]_- = \hat a \hat a^\dagger - \hat a^\dagger \hat a = \mathbb I~.
    \end{equation*}
    Furthermore, the number operator $\hat N = \hat a^\dagger \hat a$ such that 
    \begin{equation*}
        [\hat N, \hat a] = - \hat a~, \quad [\hat N, \hat a^\dagger] = \hat a^\dagger ~.
    \end{equation*} 

    By analogy with the harmonic oscillator, the ground state is the vacuum 
    \begin{equation*}
        \hat a \ket{0} = 0 ~,
    \end{equation*}
    and a generic state is defined by the ladder operators
    \begin{equation*}
        \ket{\psi} = \frac{1}{\sqrt{n!}} (\hat a^\dagger)^N \ket{0} ~.
    \end{equation*}

\section{Fermionic case}

    We define creation and annihilation operators such that they satisfies the properties 
    \begin{equation*}
        [\hat a, \hat a^\dagger]_+ = \hat a \hat a^\dagger + \hat a^\dagger \hat a = \mathbb I~.
    \end{equation*}
    Furthermore, the number operator $\hat N = \hat a^\dagger \hat a$ such that 
    \begin{equation*}
        [\hat N, \hat a] = - \hat a~, \quad [\hat N, \hat a^\dagger] = \hat a^\dagger ~.
    \end{equation*} 

    The properties can be obtained from the Pauli matrices 
    \begin{equation*}
        \sigma_\pm = \sigma_1 \pm i \sigma_2 ~,
    \end{equation*}
    such that 
    \begin{equation*}
        (\sigma_+)^\dagger = \sigma_- ~, \quad (\sigma_-)^\dagger = \sigma_+ ~, \quad (\sigma_+)^2 = (\sigma_-)^2 = 0 ~, \quad [\sigma_-, \sigma_+]_+ = \mathbb I ~.
    \end{equation*}

    By analogy with the harmonic oscillator, the ground state is the vacuum 
    \begin{equation*}
        \hat a \ket{0} = 0 ~,
    \end{equation*}
    and a generic state is defined by the ladder operators
    \begin{equation*}
        \ket{\psi} = \frac{1}{\sqrt{n!}} (\hat a^\dagger)^N \ket{0} ~.
    \end{equation*}

    However, the anticommutator relation ensures the validity of the Pauli's exclusion principle. In fact, we have 
    \begin{equation*}
        a^2 = (\hat a^\dagger)^2 = 0 ~.
    \end{equation*}

\section{Fock space}

    Consider a single particle Hilbert space $\mathcal H$ with an orthonormal basis $\{\ket{e_n}\}_{n=1}^\infty$. To each $\ket{e_n}$, we associate an annihilation and a creation operators 
    \begin{equation*}
        \ket{e_n} \mapsto \{\hat a_n, \hat a_n^\dagger \}_{n=1}^\infty ~,
    \end{equation*}
    such that they satisfy 
    \begin{equation*}
        [\hat a_n, \hat a_m]_\pm = [\hat a_n^\dagger, \hat a_m^\dagger]_\pm = 0 ~, \quad [\hat a_n, \hat a_m^\dagger]_\pm = \delta_{nm} ~,
    \end{equation*}
    where the minus sign correponds to the commutator (bosons) and the plus sign to the anticommutator (fermions). 

    The vacuum state is defined as 
    \begin{equation*}
        \hat a_n \ket{0} = 0 \quad \forall n~.
    \end{equation*}

    For each $\ket{e_n}$, we associate a number operator $\hat n_k = \hat a_k^\dagger \hat a_k$ such that 
    \begin{equation*}
        \hat n_k \hat a_k^\dagger \ket{0} = 1 \hat a_k^\dagger \ket{0} ~, \quad \hat n_{k'} \hat a_k^\dagger \ket{0} = 0 \quad k' \neq k ~.
    \end{equation*}
    For a $n$ particle state, we have 
    \begin{equation*}
        \hat a_k^\dagger \ket{0} = \ket{n_1=0, \ldots n_k=1, \ldots n_N=0} = \ket{e_k} ~.
    \end{equation*}

    However, for 
    \begin{equation*}
        \hat a_{k_1}^\dagger \hat a_{k_2}^\dagger \ket{0} = \ket{e_{k_1}} \ket{e_{k_2}} 
    \end{equation*}
    we have for fermions, if $k_1 = k_2 = k$
    \begin{equation*}
    (\hat a^\dagger_k)^2 \ket{0} = 0 ~,
    \end{equation*}
    whereas for bosons 
    \begin{equation*}
        (\hat a^\dagger_k)^2 \ket{0} \neq 0 ~.
    \end{equation*}
    Furthermore, if $k_1 \neq  k_2$, we have for fermions
    \begin{equation*}
        \hat a^\dagger_{k_1} \hat a^\dagger_{k_2} \ket{0} = - \hat a^\dagger_{k_2} \hat a^\dagger_{k_1} \ket{0} ~,
    \end{equation*}
    whereas for bosons 
    \begin{equation*}
        \hat a^\dagger_{k_1} \hat a^\dagger_{k_2} \ket{0} = \hat a^\dagger_{k_2} \hat a^\dagger_{k_1} \ket{0} ~.
    \end{equation*}

\section{Alternative way}

    There is a $1-1$ correspondence between the orthonormal basis  $\{\ket{e_n}\}_{n=1}^\infty$ of $\mathcal H$ and the orthonormal basis $\{\hat a_k \ket{0}\}_{k=1}^\infty$ of $\mathcal H_{S/A}$. Hence for $N$ particles, we have 
    \begin{equation*}
        \mathcal H_{S/A}^{(N)} = \{\ket{n_1, \ldots n_k, \ldots} = \frac{1}{\sqrt{ \prod_j n_j}} (\hat a_1^\dagger)^{n_1} \ldots (\hat a_k^\dagger)^{n_k} \ldots \ket{0} \} ~.
    \end{equation*} 

    If $N$ is not fixed, like the passage from canonical to grancanonicl ensemble, the total Fock space is 
    \begin{equation*}
        \mathcal F = \bigoplus_{N=0}^\infty \mathcal H^{(N)}_{S/A} ~.
    \end{equation*} 

    It satisfies the following properties 
    \begin{enumerate}
        \item orthonormality, i.e. 
            \begin{equation*}
                \braket{{n'}_1, \ldots {n'}_k, \ldots}{n_1, \ldots n_k, \ldots} = \delta_{{n'}_1, n_1} \ldots \delta_{{n'}_k, n_k} \ldots  ~,
            \end{equation*}
        \item annihilation $\hat a_k \colon \mathcal H^{(N)}_{S/A} \rightarrow \mathcal H^{(N-1)}_{S/A}$, i.e.
            \begin{equation*}
                \hat a_k \ket{n_1, \ldots n_k, \ldots} = \eta_k \sqrt{n_k} \ket{n_1, \ldots (n_k - 1), \ldots} ~,
            \end{equation*}
            where for bosons $\eta_k = 1$ and for fermions $\eta_k = (-1)^{\sum_{j < k} n_j}$,
        \item creation $\hat a_k^\dagger \colon \mathcal H^{(N)}_{S/A} \rightarrow \mathcal H^{(N+1)}_{S/A}$, i.e. for bosons
            \begin{equation*}
                \hat a^\dagger_k \ket{n_1, \ldots n_k, \ldots} = \sqrt{n_k + 1} \ket{n_1, \ldots (n_k + 1), \ldots} ~,
            \end{equation*}
            and for fermions
            \begin{equation*}
                \hat a^\dagger_k \ket{n_1, \ldots n_k, \ldots} = \eta_k \sqrt{1 - n_k} \ket{n_1, \ldots (n_k + 1), \ldots} ~,
            \end{equation*}
        \item number operator $\hat n_k = \hat a_k^\dagger \hat a_k$ such that 
            \begin{equation*}
                \hat n_k \ket{n_1, \ldots n_k, \ldots} = n_k \ket{n_1, \ldots n_k, \ldots}
            \end{equation*}
        and the total number operator $\hat N = \sum_k \hat n_k \sum_k \hat a^\dagger_k \hat a_k$ such that 
        \begin{equation*}
            \hat N \ket{n_1, \ldots n_k, \ldots} = \Big (\sum_k n_k \Big ) \ket{n_1, \ldots n_k, \ldots} ~.
        \end{equation*}
    \end{enumerate}

\section{Field operators} 

    In the first quantisation, we quantise observables to operators, while, in the second quantisation, we quantise fields to operators. Now, a generic particle state is represented by $\ket{f} = \sum_k f_k \ket{e_k} \in \mathcal H$, which is equivalent to $sum_k f_k \hat a_k^\dagger \ket{0}$. Hence, we define the field operators
    \begin{equation*}
        \hat \psi^\dagger (f) = \sum_k f_k \hat a^\dagger_k ~, \quad \hat \psi (f) = \sum_k f_k^* \hat a_k ~,
    \end{equation*}
    in order to get a state $\hat \psi (f) \ket{0}$. The related commutator relations become
    \begin{equation*}
        [\hat \psi (f), \hat \psi^\dagger (g)]_\pm = \braket{f}{g}\mathbb I ~.
    \end{equation*}
    \begin{proof}
        In fact,
        \begin{equation*}
            [\hat \psi (f), \hat \psi^\dagger (g)]_\pm = [\sum_k f^*_k \hat a_k, \sum_m g_m \hat a^\dagger]_\pm = \sum_k \sum_m f^*_k g_m \underbrace{[\hat a_k, \hat a^\dagger_m]}_{\delta_{km} \mathbb I} = \sum_k \sum_m f^*_k g_m \underbrace{\delta_{km}}_{k= m} \mathbb I = \sum_k f^*_k g_k \mathbb I = \braket{f}{g} \mathbb I ~.
        \end{equation*}
        where we have used $\ket{f} \sum_k f_k \ket{e_k}$, $\ket{g} = \sum_m g_m \ket{e_m}$ and $\braket{f}{g} = \sum_k \sum_m f^*_k g_m \underbrace{\braket{e_k}{e_m}}_{\delta_{km}} = \sum_k f^*_k g_k$.
    \end{proof}

    Consider a single particle state in $\mathcal H = L^2(\mathbb R^d) \ni \psi(x)$ with an orthonormal basis $u_k(x)$ such that to each ket there are ladder operators $\hat a_k$ and $\hat a_k^\dagger$. Hence $L^2(\mathbb R^d) \ni f(x) = \sum_k f_k u_k(x)$ and we define field operators
    \begin{equation*}
        \hat \psi(x) = \sum_k u_k^* (x) \hat a_k ~, \quad \hat \psi^\dagger (x) = \sum_k u_k (x) \hat a_k^\dagger ~,
    \end{equation*}
    which is alinear superposition of annihilation and creation operators. Actually, it is called an operator-valued function because its output is an operator. In fact 
    \begin{equation*}
        \int_{\mathbb R^d} d^d x ~ \psi^\dagger (x) \sum_k u_k^* (x) \hat a_k^\dagger = \sum_k \hat a_k^\dagger \int_{\mathbb R^d} d^d x ~ u^*_k(x) f(x) = \sum_k \hat a_k^\dagger f_k~,
    \end{equation*}
    where we have exchanged sum and integral because they are convergent. 

    The commutation relations are 
    \begin{equation*}
        [\psi(x), \psi^\dagger (y)] = \mathbb I \delta (x - y) ~.
    \end{equation*}
    \begin{proof}
        In fact,
        \begin{equation*}
            [\hat \psi (f), \hat \psi^\dagger (g)]_\pm = [\int d^d x ~ f^* (x) \hat \psi(x), \int d^d y ~ g(y) \hat \psi^\dagger (y)]_\pm = \int d^d x \int d^d y ~ f^*(x) g(y) [\psi(x), \psi^\dagger (y)]  ~,
        \end{equation*}
        which must be equal to 
        \begin{equation*}
            \braket{f}{g} = \int d^d x ~ f^* (x) g(x) ~.
        \end{equation*}
        Hence 
        \begin{equation*}
            [\psi(x), \psi^\dagger (y)] = \mathbb I \delta (x - y) ~.
        \end{equation*}
    \end{proof}

    For instance, a plane wave $u(x) = \exp (i \mathbf k \cdot \mathbf x) $ and $\hat \psi(x) = \sum_k \hat a_k^\dagger \exp(i \mathbf k \cdot \mathbf x)$.

    Notice that field operators are basis independent

\section{Operators}

    Consider a Fock space $\mathcal F = \bigoplus_{N=0}^\infty \mathcal H^{(N)}_{B/F}$ with orthonormal basis $\ket{n_1, \ldots n_k, \ldots} = \frac{1}{\sqrt{\prod_j n_j !}} (\hat a^\dagger_1)^{n_1} \ldots (\hat a_k^\dagger)^{n_k} \ldots \ket{0}$, which is in $1-1$ correspondence to the orthonormal basis $\psi_{n_1 \ldots n_k \ldots} (x_1, \ldots x_k, \ldots) = c_N \begin{bmatrix} \hat S \\ \hat A \\ \end{bmatrix} u_{\alpha_1} (x_1) \ldots u_{\alpha_k} (x_k) \ldots$, where $hat S$ is the symmetriser and $\hat A$ is the antisymmetriser.

    We define a one-body operator, associated to a system in which all the particles are the same, as 
    \begin{equation*}
        \hat O^{(1)} = \sum_{j=1}^{N} \hat O(\hat p_j, \hat x_j) ~.
    \end{equation*}
    Since it is self-adjoint, it exists an orthonormal basis of eigenvalues $\{u_\alpha (x)\}$, such that 
    \begin{equation*}
        \hat O(\hat p, \hat x) u_\alpha (x) = \epsilon_\alpha u_\alpha (x) ~.
    \end{equation*}

    Since 
    \begin{equation*}
    \begin{aligned}
        \hat O^{(1)} \psi_{n_1 \ldots n_k \ldots} (x_1, \ldots x_k, \ldots) & = \Big ( \sum_{j=1}^{\infty} \hat O(\hat p_j, \hat x_j) \Big) \psi_{n_1 \ldots n_k \ldots} (x_1, \ldots x_k, \ldots) \\ & = \Big ( \sum_{j=1}^{\infty} \hat O(\hat p_j, \hat x_j) \Big) c_N \begin{bmatrix} \hat S \\ \hat A \\ \end{bmatrix} u_{\alpha_1} (x_1) \ldots u_{\alpha_k} (x_k) \ldots \\ & = c_N \begin{bmatrix} \hat S \\ \hat A \\ \end{bmatrix} \Big ( \sum_{j=1}^{\infty} \hat O(\hat p_j, \hat x_j) u_{\alpha_1} (x_1) \ldots u_{\alpha_k} (x_k) \ldots \Big) \\ & = c_N \begin{bmatrix} \hat S \\ \hat A \\ \end{bmatrix} \Big ( \sum_{j=1}^{\infty}  u_{\alpha_1} (x_1) \ldots \underbrace{\hat O(\hat p_j, \hat x_j) u_{\alpha_j} (x_j)}_{\epsilon_{\alpha_j} u_{\alpha_j} (x_j) } \ldots \Big) \\ & = \Big (\sum_{j=1}^{\infty} \epsilon_j n_j \Big ) \psi_{n_1 \ldots n_k \ldots} (x_1, \ldots x_k, \ldots) ~.
    \end{aligned}
    \end{equation*}

    For the Fock space, we have 
    \begin{equation*}
        \hat O^{(1)}_F = \sum_{j=1}^{\infty} \epsilon_j \hat n_j = \sum_{j=1}^{\infty} \epsilon_j \hat a_j^\dagger \hat a_j ~,
    \end{equation*}
    where 
    \begin{equation*}
        \epsilon_j = \bra{u_j (x)} \hat O (\hat p_j, \hat x_j) \ket{u_j(x)} ~.
    \end{equation*}
    Hence 
    \begin{equation*}
        \hat O^{(1)}_F = \sum_{j=1}^{\infty} \bra{u_j (x)} \hat O (\hat p_j, \hat x_j) \ket{u_j(x)} \hat a_j^\dagger \hat a_j ~.
    \end{equation*}

    Since it is dependent of the basis, because we choose the eigenbasis, we choose a different arbitrary basis 
    \begin{equation*}
        \psi^\dagger (x) = \sum_k u_k (x) \hat a^\dagger_k = \sum_m v_m (x) b^\dagger_m ~,
    \end{equation*}
    and we define the one-body operator
    \begin{equation*}
        \hat O^{(1)}_F = \int d^d x ~ \hat \varphi^\dagger (x) \hat O (\hat p, \hat x) \hat \varphi (x) ~,
    \end{equation*}
    which this time is basis independent.
    \begin{proof}
        In fact 
        \begin{equation*}
        \begin{aligned}
            \int d^d x ~ \hat \varphi^\dagger (x) \hat O (\hat p, \hat x) \hat \varphi (x) & = \int d^d x ~ \Big ( \sum_k u_k(x) \hat a^\dagger (x) \Big ) \hat O (\hat p, \hat x) \Big ( \sum_m u_m^* (x) \hat a_m (x) \Big ) \\ & = \sum_k \sum_m \hat a_k^\dagger \hat a_m \int d^d x ~ u_k (x) \underbrace{\hat O(\hat p, \hat x) u^*_m (x)}_{\epsilon_m u^*_m (x)} \\ & = \sum_k \sum_m \hat a_k^\dagger \hat a_m \epsilon_m \underbrace{\int d^d x ~ u_k (x) u^*_m (x)}_{\delta_{km}} \\ & = \sum_k \sum_m \hat a_k^\dagger \hat a_m \epsilon_m \underbrace{\delta_{km}}_{k = m} \\ & = \sum_k\hat a_k^\dagger \hat a_k \epsilon_k = \hat O^{(1)}_F ~.
        \end{aligned}
        \end{equation*}
    \end{proof}
    It can be written as 
    \begin{equation*}
        \hat O^{(1)}_F = \sum_k \sum_m t_{km} \hat b_k^\dagger \hat h_m ~,
    \end{equation*}
    where the transition amplitude is
    \begin{equation*}
        t_{km} = \bra{v_k} \hat O (\hat p, \hat x) \ket{v_m} ~.
    \end{equation*}




\part{Quantum statistical mechanics}

\chapter{Microcanonical ensemble}

    The microcanonical ensemble is characterised by constant volume, energy and number of particle. Since $N$ is fixed, we can work in the Hilbert space $\mathcal H_{tot}$. Given a time-independent hamiltonian operator $\hat H$, we find the energy eigenbasis $\ket{\psi_j} \in \mathcal H_{tot}$ 
    \begin{equation*}
        \hat H \ket{\psi_j} = E_j \ket{\psi_j} ~.
    \end{equation*}
    However, there could be some degeneracy we want to consider, i.e. $E_{j,\alpha} = E_{j, \beta}$ for $\ket{\psi_{j, \alpha}} \neq \ket{\psi_{j, \beta}}$. Therefore, we have 
    \begin{equation}\label{eneigen}
        \hat H \ket{\psi_{j,\alpha}} = E_j \ket{\psi_{j,\alpha}} ~,
    \end{equation}
    where $\alpha = 1, \ldots n_j$.

    The density operator for mixed states is~\eqref{mix}
    \begin{equation*}
        \rho_{mc} = \sum_{\alpha=1}^{n_j} p_\alpha \ket{\psi_{j, \beta}} \bra{\psi_{j, \beta}} ~,
    \end{equation*}
    where $p_\alpha$ is the probability for the eigenstate $\ket{\psi_{j, \beta}}$. Since $E = E_j$ is fixed, all the eigenstates have the same probability to occur. Therefore $p_\alpha = \frac{1}{n_j}$ and 
    \begin{equation*}
        \rho_{mc} = \frac{1}{n_j} \sum_{\alpha=1}^{n_j} \ket{\psi_{j, \alpha}} \bra{\psi_{j, \alpha}} = \frac{1}{n_j} \hat P_j ~,
    \end{equation*}
    where 
    \begin{equation*}
        P_j = \sum_{\alpha=1}^{n_j} \ket{\psi_{j, \alpha}} \bra{\psi_{j, \alpha}}
    \end{equation*} 
    is the projector onto the energy eigenspace. Notice that we can expand the hamiltonian using~\eqref{spec}
    \begin{equation}\label{endec}
        \hat H = \sum_j E_j \hat P_j ~.
    \end{equation}

    The average of an observable $\hat A$ in the microcanonical ensemble is 
    \begin{equation*}
        \av{A}_{mc} = \frac{1}{n_j} \sum_{\alpha=1}^{n_j} \bra{\psi_{n,\alpha}} \hat A \ket{\psi_{n,\alpha}} ~.
    \end{equation*}
    \begin{proof}
        In fact, choosing an orthonormal basis $\ket{e_j}$, the trace is 
        \begin{equation*}
            \tr_{\mathcal H_{tot}} \hat A = \sum_j \bra{e_j} \hat A \ket{e_j} ~.
        \end{equation*}
        Therefore, using~\eqref{avobs}
        \begin{equation*}
        \begin{aligned}
            \av{A}_{mc} & = \tr_{\mathcal H_{tot}} (\hat A \rho_{mc}) \\ & = \tr_{\mathcal H_{tot}} \Big ( \hat A \frac{1}{n_j} \sum_{\alpha=1}^{n_j} \ket{\psi_{j, \alpha}} \bra{\psi_{j, \alpha}} \Big) \\ & = \frac{1}{n_j} \sum_{\alpha=1}^{n_j} \tr_{\mathcal H_{tot}} \Big ( \hat A \ket{\psi_{j, \alpha}} \bra{\psi_{j, \alpha}} \Big) \\ & = \frac{1}{n_j} \sum_{\alpha=1}^{n_j} \bra{\psi_{j, \alpha}} \hat A \ket{\psi_{j, \alpha}} ~.
        \end{aligned}
        \end{equation*}
    \end{proof}

    The entropy in the microcanonical ensemble is 
    \begin{equation*}
        S_{mc} = k_B \log n_j ~,
    \end{equation*}
    where $n_j$ is the number of states with $E = E_j$. Notice that it is similar to the classical case~\eqref{entropymc}.
    \begin{proof}
        In fact, using~\eqref{unboltz}
        \begin{equation*}
        \begin{aligned}
            S_{mc} = - k_B \av{\log \rho_{mc}}_{mc} = - k_B \tr_{\mathcal H_{tot}} ( \rho_{mc} \log \rho_{mc}) ~.
        \end{aligned}
        \end{equation*}

        In matrix notation, the density operator is 
        \begin{equation*}
        \begin{aligned}
            \rho_{mc} & = \begin{bmatrix}
                \begin{bmatrix}
                    \frac{1}{n_1} & 0 & \ldots & 0 \\
                    0 & \frac{1}{n_1} & \ldots & 0 \\
                    \ldots & \ldots & \ldots & \ldots \\
                    0 & 0 & \ldots & \frac{1}{n_1} \\
                \end{bmatrix} & 0 & \ldots & 0 & \ldots & \ldots \\ 0 & 
                \begin{bmatrix}
                    \frac{1}{n_2} & 0 & \ldots & 0 \\
                    0 & \frac{1}{n_2} & \ldots & 0 \\
                    \ldots & \ldots & \ldots & \ldots \\
                    0 & 0 & \ldots & \frac{1}{n_2} \\
                \end{bmatrix} & \ldots & 0 & \ldots & \ldots \\ 
                \ldots & \ldots & \ldots & \ldots & \ldots & \ldots \\
                0 & 0 & \ldots & \begin{bmatrix}
                    \frac{1}{n_j} & 0 & \ldots & 0 \\
                    0 & \frac{1}{n_j} & \ldots & 0 \\
                    \ldots & \ldots & \ldots & \ldots \\
                    0 & 0 & \ldots & \frac{1}{n_j} \\
                \end{bmatrix} & \ldots & \ldots \\
                \ldots & \ldots & \ldots & \ldots & \ldots & \ldots \\
                \ldots & \ldots & \ldots & \ldots & \ldots & \ldots \\
            \end{bmatrix} \\ & = \sum_j \begin{bmatrix}
                0 & 0 & \ldots & 0 & \ldots & \ldots \\ 
                0 & 0 & \ldots & 0 & \ldots & \ldots \\ 
                \ldots & \ldots & \ldots & \ldots & \ldots & \ldots \\
                0 & 0 & \ldots & \begin{bmatrix}
                    \frac{1}{n_j} & 0 & \ldots & 0 \\
                    0 & \frac{1}{n_j} & \ldots & 0 \\
                    \ldots & \ldots & \ldots & \ldots \\
                    0 & 0 & \ldots & \frac{1}{n_j} \\
                \end{bmatrix} & \ldots & \ldots \\
                \ldots & \ldots & \ldots & \ldots & \ldots & \ldots \\
            \end{bmatrix}
        \end{aligned} ~.
        \end{equation*}

        In order to compute the logarithm of $0$, we use a trick: we define a small parameter $\epsilon$ and we make it go to zero. In this way, the limit becomes $\epsilon \log \epsilon \xrightarrow{\epsilon \rightarrow 0} = 0$. Finally, we compute the trace 
        \begin{equation*}
        \begin{aligned}
            \tr_{\mathcal H_{tot}} ( \rho_{mc} \log \rho_{mc}) & = \tr \begin{bmatrix}
                0 & 0 & \ldots & 0 & \ldots & \ldots \\ 
                0 & 0 & \ldots & 0 & \ldots & \ldots \\ 
                \ldots & \ldots & \ldots & \ldots & \ldots & \ldots \\
                0 & 0 & \ldots & \begin{bmatrix}
                    \frac{1}{n_j} \log \frac{1}{n_j} & 0 & \ldots & 0 \\
                    0 & \frac{1}{n_j} \log \frac{1}{n_j} & \ldots & 0 \\
                    \ldots & \ldots & \ldots & \ldots \\
                    0 & 0 & \ldots & \frac{1}{n_j} \log \frac{1}{n_j} \\
                \end{bmatrix} & \ldots & \ldots \\
                \ldots & \ldots & \ldots & \ldots & \ldots & \ldots \\
            \end{bmatrix} \\ & = \sum_j \frac{1}{n_j} \log \frac{1}{n_j} = n_j \frac{1}{n_j} \log \frac{1}{n_j} = - \log n_j ~.
        \end{aligned}
        \end{equation*}
        Hence, 
        \begin{equation*}
            S_{mc} = - k_B \tr_{\mathcal H_{tot}} ( \rho_{mc} \log \rho_{mc}) = k_B \log n_j ~.
        \end{equation*}
    \end{proof}

    Notice that entropy is always a positive function, since there is at least one state occupied $n_j \geq 1$, which implies $S \geq 0$.

\chapter{Canonical ensemble}

    The canonical ensemble is characterised by constant volume, temperature and number of particle. Energy, which can be exchange in an external reservoir, can be in one of the eigenstates~\eqref{eneigen} with probability 
    \begin{equation}\label{prob}
        p_j \propto \exp(- \beta E_j) ~.
    \end{equation}

    Consider a family of projectors $\{\hat P_j\}$, the density matrix of a mixed states is 
    \begin{equation*}
        \rho_c = \frac{1}{Z_N } \sum_j \exp(- \beta E_j) \hat P_j = \frac{\exp(- \beta \hat H)}{Z_N} ~,
    \end{equation*}
    where the quantum canonical partition function is 
    \begin{equation*}
        Z_N(T,V) = \tr_{\mathcal H_{tot}} \Big ( \frac{\exp(- \beta \hat H)}{Z_N} \Big) ~.
    \end{equation*}
    \begin{proof}
        For a mixed state, the density matrix is~\eqref{mix}
        \begin{equation*}
            \rho_c = \sum_j p_j \hat P_j = C \sum_j \exp(- \beta E_j) \hat P_J ~,
        \end{equation*}
        where the probability is given by~\eqref{prob} and $C$ is a normalisation function.

        Moreover, using~\eqref{endec}
        \begin{equation*}
        \begin{aligned}
            \rho_c & = C \sum_j \exp(- \beta E_j) \hat P_J \\ & = C \sum_j \sum_k \frac{1}{k!} (-\beta E_j)^k \underbrace{\hat P_j}_{(P_j)^k} \\ & = C \sum_j \sum_k \frac{1}{k!} (-\beta E_j \hat P_j)^k \\ & = C \sum_k \frac{1}{k!} (-\beta \sum_j E_j \hat P_j)^k \\ & = C \exp(- \beta \underbrace{\sum_j E_j \hat P_j}_{\hat H}) \\ & = C \exp(- \beta \hat H) ~,
        \end{aligned}
        \end{equation*}
        where we have used the Taylor expansion of the exponential, one of the properties of the projectors~\eqref{idem} and we have exchanged the two series.

        Finally, We set $C = \frac{1}{Z_N}$, where $Z_N$ is the quantum canonical partition function, and by the normalisation condition 
        \begin{equation*}
            1 = \tr_{\mathcal H_{tot}} \rho_c = \frac{1}{Z_N} \tr_{\mathcal H_{tot}} \exp(- \beta \hat H) ~,
        \end{equation*}
        hence 
        \begin{equation*}
            Z_N = \tr_{\mathcal H_{tot}} \exp(- \beta \hat H) ~.
        \end{equation*}
    \end{proof}

    We define the Helmoltz free energy
    \begin{equation*}
        Z_N = \exp(- \beta F) ~,
    \end{equation*}
    or equivalently 
    \begin{equation*}
        F = - \frac{1}{\beta} \log Z_N ~.
    \end{equation*}
    The average energy is 
    \begin{equation*}
        E = \av{\hat H}_c = - \pdv{}{\beta} \log Z_N ~.
    \end{equation*}
    \begin{proof}
        In fact, 
        \begin{equation*}
        \begin{aligned}
            E & = \av{\hat H}_c \\ & = \tr_{\mathcal H_{tot}} (\hat H \rho_c) \\ & = \tr_{\mathcal H_{tot}} \Big ( \hat H \frac{\exp(- \beta \hat H)}{Z_N} \Big ) \\ & = \frac{1}{Z_N} \tr_{\mathcal H_{tot}} \Big (- \pdv{}{\beta} \exp(- \beta \hat H) \Big) \\ & = - \frac{1}{Z_N} \pdv{}{\beta} \underbrace{\tr_{\mathcal H_{tot}} \exp(- \beta \hat H)}_{Z_N} \\ & = - \frac{1}{Z_N} \pdv{}{\beta} Z_N \\ & = - \pdv{}{\beta} \log Z_N ~.
        \end{aligned}
        \end{equation*}
    \end{proof}

    The entropy is 
    \begin{equation*}
        S = \frac{E - F}{T} = \pdv{F}{T} ~.
    \end{equation*}
    \begin{proof}
        In fact, using~\eqref{unboltz}
        \begin{equation*}
        \begin{aligned}
            S_c & = - k_B \av{\log \rho_c}_c \\ & = - k_B \tr_{\mathcal H_{tot}} (\rho_c \log \rho_c) \\ & = - k_B \tr_{\mathcal H_{tot}} (\frac{\exp(- \beta \hat H)}{Z_N} \log \frac{\exp(- \beta \hat H)}{Z_N}) \\ & = - k_B \tr_{\mathcal H_{tot}} \Big (\frac{\exp(- \beta \hat H)}{Z_N} (\log \exp(- \beta \hat H) - \log Z_N) \Big ) \\ & = - k_B \tr_{\mathcal H_{tot}} (\frac{\exp(- \beta \hat H)}{Z_N} (- \beta \hat H - \log Z_N)) \\ & = k_B \beta ~ \underbrace{\tr_{\mathcal H_{tot}} (\frac{\exp(- \beta \hat H)}{Z_N} \hat H )}_E + k_B \tr_{\mathcal H_{tot}} (\frac{\exp(- \beta \hat H)}{Z_N} \underbrace{\log Z_N}_{- \beta F} ) \\ & = \frac{E}{T} - k_B \beta F ~ \frac{1}{Z_N} \underbrace{\tr_{\mathcal H_{tot}} (\exp(- \beta \hat H))}_{Z_N} \\ & = \frac{E-F}{T} ~.
        \end{aligned}
        \end{equation*}
    \end{proof}
    Notice that the entropy is well defined because the trace of the exponential of the energy eigenvalues diverges only if they are negative. Thus, we assume that $E_j \geq \min E_j = 0$.

\chapter{Grancanonical ensemble}

    The grancanonical ensemble is characterised by constant volume, temperature and chemical potential. Since $N$ is not fixed, we work in the full Fock space $\mathcal F_N$. However, we restrict the hamiltonian operator in the Fock space to the condition that it conserves the number of particles, i.e. $[\hat H, \hat N] = 0$ 
    \begin{equation*}
        \hat H \Big \vert_{\mathcal F_N} = \hat H_N ~.
    \end{equation*}
    An example of physical system which does not satisfy this condition is a photons absorbed by an electron. Energy can be in one of the eigenstates, each for a fixed $N$
    \begin{equation*}
        \hat H^{(N)} \ket{\psi_{j, \alpha}^{(N)}} = E_j^{(N)} \ket{\psi_{j, \alpha}^{(N)}} ~,
    \end{equation*}
    with probability 
    \begin{equation}\label{prob2}
        p_j^{(N)} \propto \exp(- \beta (E_j - \mu N)) ~.
    \end{equation}

    Consider a family of projectors $\{\hat P_j^{(N)}\}$
    \begin{equation*}
        \hat P_j^{N} = \sum_\alpha \ket{\psi_{j, \alpha}^{(N)}} \bra{\psi_{j, \alpha}^{(N)}} ~,
    \end{equation*}  
    the density matrix of a mixed states is 
    \begin{equation*}
        \rho_{gc} = \frac{1}{\mathcal Z} \sum_N \sum_j \exp(- \beta (E_j - \mu N)) \hat P_j^{(N)} = \frac{\exp(- \beta (\hat H - \mu \hat N))}{\mathcal Z} ~,
    \end{equation*}
    where $z = \exp(\beta \mu)$ is the fugacity and the quantum grancanonical partition function is 
    \begin{equation*}
        \mathcal Z = \sum_{N=0}^{\infty} \tr_{\mathcal H_{tot}} \Big ( \exp(- \beta (\hat H - \mu \hat N)) \Big) = \sum_{N=0}^\infty z^N Z_N ~.
    \end{equation*}
    \begin{proof}
        For a mixed state, the density matrix is~\eqref{mix}
        \begin{equation*}
            \rho_{gc} = \sum_N \sum_j p_j \hat P_j^{(N)} = C \sum_N \sum_j \exp(- \beta (E_j^{(N)} - \mu N)) \hat P_j^{(N)} ~,
        \end{equation*}
        where the probability is given by~\eqref{prob2} and $C$ is a normalisation function.

        Moreover, using~\eqref{endec} and~\eqref{numb}
        \begin{equation*}
        \begin{aligned}
            \rho_{gc} & = C \sum_N \sum_j \exp(- \beta (E_j - \mu N)) \hat P_j^{(N)} \\ & = C \sum_N \sum_j \sum_k \frac{1}{k!} (-\beta (E_j^{(N)} - \mu N))^k \underbrace{\hat P_j^{(N)}}_{(P_j^{(N)})^k} \\ & = C \sum_j \sum_k \frac{1}{k!} (-\beta (E_j^{(N)} \hat P_j^{(N)} - \nu N P_j^{(N)}))^k \\ & = C \sum_k \frac{1}{k!} (-\beta \sum_N \sum_j (E_j^{(N)} \hat P_j^{(N)} - \mu N P_j^{(N)}))^k \\ & = C \exp(- \beta (\underbrace{\sum_j \sum_N E_j^{(N)} \hat P_j^{(N)}}_{\hat H}) - \mu \underbrace{\sum_j \sum_N N \hat P_j^{(N)}}_{\hat N}) \\ & = C \exp(- \beta (\hat H - \mu \hat N)) ~,
        \end{aligned}
        \end{equation*}
        where we have used the Taylor expansion of the exponential, one of the properties of the projectors~\eqref{idem} and we have exchanged the two series.

        Finally, We set $C = \frac{1}{\mathcal Z}$, where $\mathcal Z$ is the quantum canonical partition function, and by the normalisation condition 
        \begin{equation*}
            1 = \tr_{\mathcal F} \rho_{gc} = \sum_N \frac{1}{\mathcal H_{tot}} \tr_{\mathcal F} \exp(- \beta (\hat H - \mu \hat N)) ~,
        \end{equation*}
        hence 
        \begin{equation*}
            \mathcal Z = \tr_{\mathcal F} \exp(- \beta (\hat H - \mu \hat N)) = \sum_{N=0}^{\infty} \tr_{\mathcal H_{tot}} \exp(- \beta (\hat H - \mu \hat N)) = \sum_{N=0}^{\infty} z^N \underbrace{\tr_{\mathcal H_{tot}} \exp(- \beta \hat H)}_{Z_N} = \sum_N z^N Z_N ~.
        \end{equation*}
    \end{proof}

    Consider an observable $\hat A$ such that it conserves the number of particles, i.e. $[\hat A, \hat N]$, the average value is 
    \begin{equation*}
        \av{\hat A}_{gc} = \tr_{\mathcal F} (\hat A \rho_{gc}) = \frac{1}{\mathcal Z} \sum_{N=0}^{\infty} z^N Z_N \av{\hat A}_c ~.
    \end{equation*}
    \begin{proof}
        In fact, 
        \begin{equation*}
        \begin{aligned}
            \av{\hat A}_{gc} & = \tr_{\mathcal F} (\hat A \rho_{gc}) \\ & = \sum_{N=0}^{\infty} \tr_{\mathcal H_{tot}} \Big (\hat A \frac{z^N \exp(- \beta \hat H)}{\mathcal Z}) = \frac{1}{\mathcal Z} \sum_{N=0}^{\infty} z^N \tr_{\mathcal H_{tot}} (\hat A \exp(- \beta \hat H)) \\ & = \frac{1}{\mathcal Z} \sum_{N=0}^{\infty} z^N Z_N \underbrace{\frac{\tr_{\mathcal H_{tot}} (\hat A \exp(- \beta \hat H))}{Z_N}}_{\av{\hat A}_c} \\ & = \frac{1}{\mathcal Z} \sum_{N=0}^{\infty} z^N Z_N \av{\hat A}_c ~.
        \end{aligned}
        \end{equation*}
    \end{proof}
    
    We define the granpotential 
    \begin{equation*}
        \Omega = - \frac{1}{\beta} \log \mathcal Z ~,
    \end{equation*}
    the energy in the grancanonical is 
    \begin{equation*}
        E - \mu N = \av{\hat H - \mu \hat N} = - \pdv{}{\beta} \log \mathcal Z ~.
    \end{equation*}
    \begin{proof}
        In fact 
        \begin{equation*}
        \begin{aligned}
            E - \mu N & = \av{\hat H - \mu \hat N} \\ & = \tr_{\mathcal F} \Big ( (\hat H - \mu \hat N) \frac{\exp( - \beta (\hat H - \mu \hat N))}{\mathcal Z} \Big) \\ & = - \frac{1}{\mathcal Z} \pdv{}{\beta} \underbrace{\tr_{\mathcal F} (\exp(- \beta (\hat H - \mu \hat N)))}_{\mathcal Z} \\ & = - \frac{1}{\mathcal Z} \pdv{}{\beta} \mathcal Z \\ & = - \pdv{}{\beta} \log \mathcal Z ~.
        \end{aligned}
        \end{equation*}
    \end{proof}

    The entropy in the grancanonical ensemble is 
    \begin{equation*}
        S = \frac{E - \mu N - \Omega}{T} ~.
    \end{equation*}
    \begin{proof}
        In fact 
        \begin{equation*}
        \begin{aligned}
            S & = - k_B \av{\log \rho_{gc}}_{gc} \\ & = - k_B \tr_{\mathcal F} ( \rho_{gc} \log \rho_{gc}) \\ & = - k_B \tr_{\mathcal F} \Big ( \frac{\exp(- \beta (\hat H - \mu \hat N))}{\mathcal Z} \log \frac{\exp(- \beta (\hat H - \mu \hat N))}{\mathcal Z} \Big) \\ & = - k_B \tr_{\mathcal F} \Big ( \frac{\exp(- \beta (\hat H - \mu \hat N))}{\mathcal Z} (\log \exp(- \beta (\hat H - \mu \hat N)) - \log \mathcal Z) \Big) \\ & = k_B \beta \underbrace{\tr_{\mathcal F} \frac{\exp(- \beta (\hat H - \mu \hat N))}{\mathcal Z} (\hat H - \mu \hat N)}_{E - \mu N} + k_B \underbrace{\tr_{\mathcal F} \log \mathcal Z }_{- \beta \Omega} \\ & = \frac{E - \mu N - \Omega}{T} ~.
        \end{aligned}
        \end{equation*}
    \end{proof}

\chapter{Quantum gas}

\section{Generic quantum gas}

    Consider a quantum gas. The hamiltonian operator of one particle, labelled by $k$ is 
    \begin{equation*}
        \hat H_k = \epsilon_k \hat n_k = \epsilon_k \hat a^\dagger_k \hat a_k ~,
    \end{equation*}
    where $\hat n_k = \hat a^\dagger_k \hat a_k$ is the number operator and $\epsilon_k$ is the energy eigenvalue associated to the eigenbasis $\ket{u_k(x)}$ by the eigenvalue relation
    \begin{equation*}
        \hat H_k \ket{u_k(x)} = \epsilon_k \ket{u_k(x)} ~.
    \end{equation*} 
    Therefore, the hamiltonian one-body operator in the Fock space $\mathcal F$, created by the ladder operators $\hat a^\dagger_k$ each associated to the element of the eigenbasis $\ket{u_k(x)}$, is 
    \begin{equation*}
        \hat H = \sum_k \hat H_k =  \sum_k \epsilon_k \hat n_k =  \sum_k \epsilon_k \hat a^\dagger_k \hat a_k ~.
    \end{equation*}
    In $\mathcal F$, the total number onebody operator is 
    \begin{equation*}
        \hat N = \sum_k \hat n_k ~,
    \end{equation*}
    where their eignevalues are given with respect to an orthonormal basis $\ket{n_1, \ldots n_k, \ldots}$ by the eigenvalue relation
    \begin{equation*}
        \hat n_k \ket{n_1, \ldots n_k, \ldots} = n_k \ket{n_1, \ldots n_k, \ldots} ~.
    \end{equation*}
    In particular, we distinguish the bosonic and the fermionic case
    \begin{equation*}
        n_k = \begin{cases}
            0,1,2,\ldots & \textnormal{bosons} \\
            0,1 & \textnormal{fermions} \\
        \end{cases} ~.
    \end{equation*}

    We exploit the grancanonical ensemble. The grancanonical partition function is 
    \begin{equation*}
        \mathcal Z = \tr_{\mathcal F} \exp(- \beta (\hat H - \mu \hat N)) = \prod_k \sum_{n_1, \ldots n_k, \ldots} \exp(-\beta(\epsilon_k - \mu)n_k) ~.
    \end{equation*}
    \begin{proof}
        In fact, 
        \begin{equation*}
        \begin{aligned}
            \mathcal Z & = \tr_{\mathcal F} \exp(- \beta (\hat H - \mu \hat N)) \\ & = \sum_{n_1, \ldots n_k, \ldots} \bra{n_1, \ldots n_k, \ldots} \exp(- \beta \sum_k (\epsilon - \mu) \underbrace{\hat n_k) \ket{n_1, \ldots n_k, \ldots}}_{n_k \ket{n_1, \ldots n_k, \ldots}} \\ & = \sum_{n_1, \ldots n_k, \ldots} \bra{n_1, \ldots n_k, \ldots} \underbrace{\exp(- \beta \sum_k}_{\prod_k \exp} (\epsilon - \mu) n_k) \ket{n_1, \ldots n_k, \ldots} \\ & = \sum_{n_1, \ldots n_k, \ldots} \prod_k \exp(\beta (\epsilon - \mu) n_k) \braket{n_1, \ldots n_k, \ldots}{n_1, \ldots n_k, \ldots} \\ & = \prod_k \sum_{n_1, \ldots n_k, \ldots} \exp(-\beta(\epsilon_k - \mu)n_k) ~,
        \end{aligned}
        \end{equation*}
        where in the last passage, we have switched the product with the sum because $n_1, \ldots, n_k, \ldots$ are independent.
    \end{proof}

    Furthermore, for bosons and fermions, it becomes
    \begin{equation*}
        \mathcal Z = \begin{cases}
            \prod_k \frac{1}{1 - \exp (- \beta (\epsilon_k - \mu))} & \textnormal{bosons} \\
            \prod_k \Big (1 + \exp (- \beta (\epsilon_k - \mu)) \Big ) & \textnormal{fermions} \\
        \end{cases} ~,
    \end{equation*}
    or, in compact notation, 
    \begin{equation*}
        \mathcal Z_\mp = \prod_k \Big ( 1 \mp \exp(- \beta (\epsilon_k - \mu) ) \Big)^\mp ~,
    \end{equation*}
    where the minus is associated to bosons and the plus to fermions.
    \begin{proof}
        For fermions, $n_k = 0, 1$
        \begin{equation*}
            \mathcal Z_+ = \prod_k \sum_{n_1, \ldots n_k, \ldots = 0}^1 \exp(-\beta(\epsilon_k - \mu)n_k) = \prod_k \Big (1 + \exp (- \beta (\epsilon_k - \mu)) \Big ) ~.
        \end{equation*}

        For bosons, $n_k = 0, 1, 2, \ldots$
        \begin{equation*}
        \begin{aligned}
            \mathcal Z_- & = \prod_k \sum_{n_1, \ldots n_k, \ldots = 0}^\infty \exp(-\beta(\epsilon_k - \mu)n_k) \\ & = \prod_k \underbrace{\sum_{n_1, \ldots n_k, \ldots = 0}^\infty \exp(-\beta(\epsilon_k - \mu))^{n_k}}_{\textnormal{geometrical series}} \\ & = \prod_k \frac{1}{1 - \exp (- \beta (\epsilon_k - \mu))} ~.
        \end{aligned}
        \end{equation*}
        Notice that the condition of convergence of the geometrical series is $\mu < \min \epsilon_k = 0$, which we have set to zero for convenience.
    \end{proof}

    The grancanonical potential is 
    \begin{equation*}
        \Omega_\mp = -\frac{1}{\beta} \log \mathcal Z_\mp = \pm \frac{1}{\beta} \sum_k \log \Big (1 \mp \exp (-\beta (\epsilon_k - \mu)) \Big) ~.
    \end{equation*}
    \begin{proof}
        In fact 
        \begin{equation*}
        \begin{aligned}
            \Omega_\mp & = -\frac{1}{\beta} \log \mathcal Z_\mp \\ & = - \frac{1}{\beta} \underbrace{\log \Big (\prod_k}_{\sum_k \log} ( 1 \mp \exp(- \beta (\epsilon_k - \mu) ))^\mp \Big ) \\ & = - (\mp) \sum_k \log \Big (1 \mp \exp (-\beta (\epsilon_k - \mu))) \\ & = \pm \frac{1}{\beta} \sum_k \log \Big (1 \mp \exp (-\beta (\epsilon_k - \mu)) \Big) ~.
        \end{aligned}
        \end{equation*}
    \end{proof}

    The grancanonical average number of particle in an energy level state $\overline k$ is 
    \begin{equation*}
        \av{\hat n_{\overline k}}_{gc} = \tr_{\mathcal F} \Big (\hat n_{\overline k} \frac{\exp (-\beta \sum_k (\epsilon_k - \mu) \hat n_k)}{\mathcal Z} \Big) = \pdv{\Omega}{\epsilon_{\overline k}} = \frac{1}{\exp(\beta(\epsilon_{\overline k} \mp 1))} ~.
    \end{equation*}
    \begin{proof}
        In fact
        \begin{equation*}
        \begin{aligned}
            \av{\hat n_{\overline k}}_{gc} & = \tr_{\mathcal F} \Big (\hat n_{\overline k} \frac{\exp (-\beta \sum_k (\epsilon_k - \mu) \hat n_k)}{\mathcal Z} \Big) \\ & = \frac{1}{\mathcal Z} \tr_{\mathcal F} \Big (- \frac{1}{\beta} \pdv{}{\epsilon_{\overline k}} \exp (-\beta \sum_k (\epsilon_k - \mu) \hat n_k) \Big) \\ & = - \frac{1}{\beta \mathcal Z} \pdv{}{\epsilon_{\overline k}} \underbrace{\tr_{\mathcal F} \Big ( \exp (-\beta \sum_k (\epsilon_k - \mu) \hat n_k) \Big)}_{\mathcal Z} \\ & = - \frac{1}{\beta \mathcal Z} \pdv{}{\epsilon_{\overline k}} \mathcal Z = \\ & = \pdv{}{\epsilon_{\overline k}} \underbrace{\Big ( - \frac{\log \mathcal Z}{\beta} \Big)}_\Omega \\ & = \pdv{}{\epsilon_{\overline k}} \Omega ~.
        \end{aligned}
        \end{equation*}

        Therefore, 
        \begin{equation*}
        \begin{aligned}
            \pdv{}{\epsilon_{\overline k}} \Omega & = \pdv{}{\epsilon_{\overline k}}  \Big (\pm \frac{1}{\beta} \sum_k \log (1 \mp \exp (-\beta (\epsilon_k - \mu)) ) \Big ) \\ & = \pm \frac{1}{\beta} (- \beta) \frac{\exp(- \beta(\epsilon_k - \mu))}{1 \mp \exp(- \beta(\epsilon_k - \mu))} \\ & = \mp \frac{1}{1 \mp \exp(\beta(\epsilon_k - \mu))} \\ & = \frac{1}{\exp(\beta(\epsilon_k - \mu))\mp 1} ~.
        \end{aligned}
        \end{equation*}
    \end{proof}

    The average total number of particle is 
    \begin{equation*}
        N = \av{\hat N}_{gc} = \av{\sum_k \hat n_k}_{gc} = \sum_k \frac{1}{\exp(\beta(\epsilon_k - \mu))\mp 1} ~.
    \end{equation*}

    The average energy is 
    \begin{equation*}
        E = \av{\hat H}_{gc} = \tr_{\mathcal F} \Big (\hat H \frac{\exp (-\beta (\hat H - \mu \hat N))}{\mathcal Z} \Big) = \sum_k \epsilon_k \av{\hat n_k}
    \end{equation*}
    \begin{proof}
        In fact 
        \begin{equation*}
        \begin{aligned}
            E & = \av{\hat H}_{gc} \\ & = \tr_{\mathcal F} \Big (\hat H \frac{\exp (-\beta (\hat H - \mu \hat N))}{\mathcal Z} \Big) \\ & = \frac{1}{\mathcal Z} \tr_{\mathcal F} \Big (- \pdv{}{\beta} \exp (-\beta (\hat H - \mu \hat N)) \Big) \\ & = - \frac{1}{\mathcal Z} \pdv{}{\beta}\underbrace{ \tr_{\mathcal F} \Big (\exp (-\beta (\hat H - \mu \hat N)) \Big)}_{\mathcal Z} \\ & = - \pdv{}{\beta} \Big \vert_z \log \mathcal Z \\ & =  - \pdv{}{\beta} \Big \vert_z (\mp \sum_k \log \Big (1 \mp \exp (-\beta (\epsilon_k - \mu)) \Big)) \\ & = \mp \sum_k \frac{\epsilon_k \exp (-\beta (\epsilon_k - \mu))}{1 \mp \exp (-\beta (\epsilon_k - \mu))} \\ & = \sum_k \frac{\epsilon_k}{\exp (\beta (\epsilon_k - \mu)) \mp 1} \\ & = \sum_k \epsilon_l \av{\hat n_k}
        \end{aligned}
        \end{equation*}
        where we have kept the fugacity $z$ constant.
    \end{proof}

\section{Non-relativistic non-interacting quantum gas}

    So far, we have made computations for a generic quantum gas. From now on, we will deal with non-relativistic non-interacting quantum gas. The finite-volume energy eigenvalues are 
    \begin{equation*}
        \epsilon_k = \frac{\hbar^2 k^2}{2m} ~ \quad \mathbf k = \frac{2\pi}{L} \mathbf n ~,
    \end{equation*}
    where $\mathbf n = (n_1, n_2, n_3) \in \mathbb Z^3$. In the thermodynamic limit, the spectrum $\mathbf k$ becomes continuous, but $\mathbf n$ not, because
    \begin{equation*}
        \Delta K_i = \frac{2\pi}{L} (n_i + 1 - n_i) = \frac{2\pi}{L} ~.
    \end{equation*}

    Therefore, sums in $k$ becomes integrals in $dk$ 
    \begin{equation*}
        \sum_k = \sum_{n_1, n_2, n_2 = -\infty}^\infty \rightarrow \frac{V}{2\pi^2} \int dk~k^2 ~.
    \end{equation*}
    \begin{proof}
        In fact, in $1$-dimensional
        \begin{equation*}
        \begin{aligned}
            \sum_{n_1} \underbrace{\Delta n_1}_1 = \sum_{k_1} \frac{L}{2\pi} \Delta k_1 \rightarrow \frac{L}{2\pi} \int dk_1 ~.
        \end{aligned}
        \end{equation*}

        Similarly, in the $3$-dimensional case
        \begin{equation*}
        \begin{aligned}
            \sum_{n_1, n_2, n_3=- \infty}^\infty \underbrace{\Delta n_1 \Delta n_2 \Delta n_3}_1 \rightarrow \Big ( \frac{L}{2\pi} \Big)^3 \int dk_1 dk_2 dk_3 = \Big ( \frac{L}{2\pi} \Big)^3 \int dk_1 dk_2 dk_3 = \Big ( \frac{L}{2\pi} \Big)^3 \int dk^3 = \Big ( \frac{L}{2\pi} \Big)^3 4 \pi \int dk ~ k^2 = \frac{V}{2\pi^2} \int dk~k^2 ~. 
        \end{aligned}
        \end{equation*}
    \end{proof}

    The grandcanonical potential is 
    \begin{equation*}
        \Omega_\mp = \mp \frac{2}{3}AV \int_0^\infty d \epsilon^{\frac{3}{2}} \frac{1}{\exp(\beta(\epsilon - \mu)) \mp 1}  ~.
    \end{equation*}
    \begin{proof}
        In fact 
        \begin{equation*}
        \begin{aligned}
            \Omega_\mp & = \pm \frac{1}{\beta} \sum_k \log \Big (1 \mp \exp (-\beta (\epsilon_k - \mu)) \Big) \\ & \rightarrow \pm \frac{1}{\beta} \frac{V}{2\pi^2} \int_{-\infty}^\infty dk ~ k^2 \log \Big (1 \mp \exp (-\beta (\epsilon_k - \mu)) \Big) ~.
        \end{aligned}
        \end{equation*}

        Under a change of variable 
        \begin{equation*}
            \epsilon = \frac{\hbar^2 k^2}{2m} ~, \quad k^2 dk = \frac{1}{2} \Big (\frac{2m}{\hbar^2}\Big)^{\frac{3}{2}} \sqrt{\epsilon} d\epsilon ~,
        \end{equation*}
        we obtain 
        \begin{equation*}
        \begin{aligned}
            \Omega_\mp & = \pm \frac{AV}{\beta} \int_0^\infty \underbrace{d\epsilon \sqrt{\epsilon}}_{\frac{2}{3} d \epsilon^{\frac{3}{2}}} \log  (1 \mp \exp (-\beta (\epsilon_k - \mu)) ) \\ & = \pm \frac{2}{3} \frac{AV}{\beta} \int_0^\infty d \epsilon^{\frac{3}{2}} \log  (1 \mp \exp (-\beta (\epsilon_k - \mu)) ) \\ & = \pm \frac{2}{3} \frac{AV}{\beta} \underbrace{\epsilon^{\frac{3}{2}}}_{0 \textnormal{ for } \epsilon = 0} \underbrace{\log  (1 \mp \exp (-\beta (\epsilon_k - \mu)))}_{0 \textnormal{ for } \epsilon = \infty} \Big \vert_0^\infty \mp \frac{2}{3} \frac{AV}{\beta} \beta \int_0^\infty d \epsilon^{\frac{3}{2}} \frac{1}{\exp(\beta(\epsilon - \mu)) \mp 1} \\ & = \mp \frac{2}{3}AV \int_0^\infty d \epsilon^{\frac{3}{2}} \frac{1}{\exp(\beta(\epsilon - \mu)) \mp 1} \\ & = \mp \frac{2}{3} AV \int_0^\infty d \epsilon^{\frac{3}{2}} \frac{1}{\exp(\beta(\epsilon - \mu)) \mp 1} ~.
        \end{aligned}
        \end{equation*}
        where we have integrated by parts and, introducing the degeneracy ($g = 2s+1$ for spin particles), we have called 
        \begin{equation*}
            A = \frac{g}{4\pi^2} \Big (\frac{2m}{\hbar^2}\Big)^{\frac{3}{2}} ~. 
        \end{equation*}
    \end{proof}

    The equation of state reads as 
    \begin{equation*}
        \Omega = - pV = - \frac{2}{3} E ~.
    \end{equation*}

    Furthermore, we have the formulas 
    \begin{equation*}
        N = AV \int_0^\infty d\epsilon \epsilon^{\frac{1}{2}} n(\epsilon) ~, 
    \end{equation*}
    \begin{equation*}
        P = \frac{2}{3} \frac{E}{V} = \frac{2}{3} A \int_0^\infty d\epsilon \epsilon^{\frac{3}{2}} n(\epsilon) ~.
    \end{equation*}

\section{Expanding with respect to fugacity $z$}

    We can expand the density with respect to the fugacity $z = \exp(\beta \mu) \geq 0$
    \begin{equation*}
        n = \frac{g}{\lambda_T^3} f^\mp_{\frac{3}{2}} ~,
    \end{equation*}
    where 
    \begin{equation*}
        f^\mp_l = \begin{cases}
            \sum_{n=0}^\infty \frac{2^{n+1}}{(n+1)^l} & f^- \textnormal{ for bosons} \\
            \sum_{n=0}^\infty \frac{(-1)^n 2^{n+1}}{(n+1)^l} & f^+ \textnormal{ for fermions}
        \end{cases} ~.
    \end{equation*}
    \begin{proof}
        Under a change of variable 
        \begin{equation*}
            x^2 = \beta \epsilon ~, \quad \beta d \epsilon = 2 x dx ~,
        \end{equation*}
        we obtain 
        \begin{equation*}
        \begin{aligned}
            n & = A \int_0^\infty dx ~\frac{2x}{\beta} \frac{x}{\sqrt(\beta) (\exp(x^2) z^{-1}) \mp 1} \\ & = \frac{4 g}{\sqrt{\pi} \lambda_T^3} \int_0^\infty dx ~ \frac{x^2 z}{\exp(x^2) \mp 2} \\ & = \frac{4g}{\sqrt{\pi} \lambda_T^3} \int_0^\infty dx ~ x^2 z \exp(- x^2) \sum_{n=0}^\infty (\pm 1) z^n \exp(- n x^2) \\ & = \frac{4g}{\sqrt{\pi} \lambda_T^3} \sum_{n=0}^\infty (\pm 1)^n z^{n+1} \underbrace{\int_0^\infty dx ~ x^2 \exp(- x^2 (n+1)) }_{\frac{\sqrt{\pi}}{4 (n+1)^{\frac{3}{2}}}} \\ & = \frac{g}{\lambda_T^3} \sum_{n=0}^\infty (\pm 1)^n \frac{z^{n+1}}{(n+1)^{\frac{3}{2}}} \\ & = \frac{g}{\lambda_T^3} f^\mp_{\frac{3}{2}} ~.
        \end{aligned}
        \end{equation*}
    \end{proof}

    Notice that for bosons, the convergence of the series implies $z < 1$, which means $\mu > 0$.

\section{Classical limit}

\section{Semiclassical limit}

\chapter{Fermions}

    In this chapter, we restrict ourselves with the fermionic case. The equations of state are 
    \begin{equation*}
        n = \frac{g}{\lambda_T^3} f_{\frac{3}{2}}^+ (z) ~, \quad \beta p = \frac{g}{\lambda_T^3} f_{\frac{5}{2}}^+ (z) ~,
    \end{equation*}
    where 
    \begin{equation*}
        f_l^+ (z) = \sum_{n=0}^\infty \frac{(-1)^n z^{n+1}}{(n+1)^l}
    \end{equation*}
    which is an alternate-sign power series in $z = \exp(\beta\mu) > 0$, always positive. It absolutely converges for $z < 1$ and pointwisely converges for $z > 1$. Moreover, it is a monotonic function in $z$. 

    It is interesting to study its behaviour for $z \ll 1$. In fact, in the classical limit
    \begin{equation*}
        f_{\frac{3}{2}}(z) \sim f_{\frac{5}{2}}(z) \sim z  ~,
    \end{equation*}
    and in the semiclassical limit 
    \begin{equation*}
        f_{\frac{3}{2}}(z) \sim z - \frac{z^2}{2^{\frac{3}{2}}} ~, \quad f_{\frac{5}{2}}(z) \sim z - \frac{z^2}{2^{\frac{5}{2}}}  ~.
    \end{equation*}

\section{Low temperature limit}

    For the zero temperature limit $T = 0$, the Fermi-Dirac distribution becomes 
    \begin{equation*}
        n(\epsilon) = \frac{1}{\exp(\beta(\epsilon - \mu)) + 1} \xrightarrow{T \rightarrow 0} \begin{cases}
            0 & \epsilon > \mu \\
            \frac{1}{2} & \epsilon = \mu \\
            1 & \epsilon < \mu \\
        \end{cases} ~.
    \end{equation*} 
    It is a step function in $\epsilon = \mu$. This energy value is called Fermi energy $\epsilon_F$. Physically, it means that all the states below this energy level are occupied. Hence, for $\epsilon < \epsilon_F$, we have as many states as particles. If we add a particle, we increase $\epsilon_F$, whereas if we remove a particle, we decrease $\epsilon_F$. This is the procedure to dope a material.

    For small $T$, it is no longer a step function, but it can be accurately approximate to it for a certain range $\Delta \epsilon$. Physically, more energetic particle are transfered over $\epsilon_F$. We define Fermi temperature $T_F$
    \begin{equation*}
        \epsilon_F = \lim_{T \rightarrow 0} \mu (T) = k_B T_F ~.
    \end{equation*}
    In fact, if $\Delta \epsilon \ll \epsilon_F$, which means $T \ll T_F$, we can approximate $n(\epsilon)$ with a step function without making a big error. 

\section{Fermi Energy for a non-relativistic non-interacting quantum gas} 

    In the $3$-dimensional case, the density is
    \begin{equation*}
        n = A \frac{2}{3} \epsilon_F^{\frac{3}{2}} ~.
    \end{equation*}
    \begin{proof}
        In fact, using $n (\epsilon) = \theta (- \epsilon_F)$
        \begin{equation*}
        \begin{aligned}
            n & = A \int_0^\infty d\epsilon \epsilon^{\frac{1}{2}} n(\epsilon) \\ & = A \int_0^{\epsilon_F} d\epsilon \epsilon^{\frac{1}{2}} \\ & = A \frac{2}{3} \epsilon_F^{\frac{3}{2}} ~.
        \end{aligned}
        \end{equation*}
    \end{proof}
    The energy is 
    \begin{equation*}
        E = A V \frac{2}{5} \epsilon_F^{\frac{5}{2}} ~.
    \end{equation*}
    \begin{proof}
        In fact, using $n (\epsilon) = \theta (- \epsilon_F)$
        \begin{equation*}
        \begin{aligned}
            n & = A \int_0^\infty d\epsilon \epsilon^{\frac{3}{2}} n(\epsilon) \\ & = A \int_0^{\epsilon_F} d\epsilon \epsilon^{\frac{3}{2}} \\ & = A \frac{2}{5} \epsilon_F^{\frac{5}{2}} ~.
        \end{aligned}
        \end{equation*}
    \end{proof}

    Notice that at $T = 0$, there is a positive pressure 
    \begin{equation*}
        p = \frac{2}{5} n \epsilon_F > 0~.
    \end{equation*}
    This can be seen visually, because at $T=0$, there are particle with energy $\epsilon \neq 0$, unlikely the classical case, in which $p = 0$.
    \begin{proof}
        In fact 
        \begin{equation*}
            p = \frac{2}{3} \frac{E}{V} = \frac{2}{3} \frac{E}{N} \frac{N}{V} = \frac{2}{5} n \epsilon_F ~. 
        \end{equation*}
    \end{proof}

\chapter{Bosons}

    In this chapter, we restrict ourselves with the bosonic case. The equations of state are 
    \begin{equation*}
        n = \frac{g}{\lambda^3_T} f^-_{\frac{3}{2}} (z) ~, \quad \beta p = \frac{g}{\lambda^3_T} f^-_{\frac{5}{2}} (z)
    \end{equation*}
    where 
    \begin{equation*}
        f_l^- (z) = \sum_{n=0}^\infty \frac{z^{n+1}}{(n+1)^l}
    \end{equation*}
    which is a positive-terms power series in $z = \exp(\beta\mu) > 0$, always positive. It absolutely converges for $z < 1$ and converges for $z > 1$ only if $l < 2$. Moreover, it is a monotonic function in $z$. At $z = 1$, it becomes the Riemann zeta 
    \begin{equation*}
        g_{\frac{3}{2}} (z=1) = \sum_{n=0}^\infty \frac{1}{n+1}^{\frac{3}{2}} = \zeta(\frac{3}{2}) ~.
    \end{equation*}

    Notice that in $z = 1$, it has a vertical derivative, and for $z > 1$, it is not defined according to the physical $\mu > 0$ in the grandcanonical ensemble. 

    We can study the behaviour of the chemical potential $\mu$. It goes to $-\infty$ for $T \rightarrow \infty$ but the equilibrium condition implies that $\pdv{\mu}{T} < 0$, therefore, it cannot increase. 

\section{Low temperature limit}
\part{Application of quantum statistical mechanics}

\part{Phase transition}

\chapter{Classical phase transitions}

    Consider the phase diagram of the water. Microscopically, they all have the same hamiltonian, however, the macroscopical variables changes. There are three phases 
    \begin{enumerate}
        \item solid, i.e. it has its own shape and volume, 
        \item liquid, i.e. it has its own volume but it has the shape of the container,
        \item gas, i.e. it has the shape and volume of the container. 
    \end{enumerate}
    There are lines, called coexistence lines, along which $2$ phases are in equilibrium. They are lines because, other than $T_1 = T_2$ and $p_1 = p_2$, we have a costrain 
    \begin{equation*}
        \mu_1(p, T) = \mu_2 (p, T) ~.
    \end{equation*}
    This reduce to a line. 

    Furthermore, there are points, called coexistence points or triple point, in which $3$ phases are in equailibrium. They are points because, other than $T_1 = T_2 = T_3$ and $p_1 = p_2 = p_3$, we have the costrains
    \begin{equation*}
        \mu_1(p, T) = \mu_2 (p, T) = \mu_3 (p, T) ~.
    \end{equation*}
    This reduce to a point.
    
\section{Symmetries} 

    We can use symmetries of the system to study it. We can distinguish solid from fluid by the translation or rotations invariance. In fact, solid has only discrete invariance, whereas fluid has continuous invariance. However, we cannot distinguish with symmetries between gas and liquid.

\section{Clausius-Clapeyron equation}

    On the coexistence line, the Clausius-Clapeyron equation is 
    \begin{equation*}
        \dv{p}{T} = \frac{s_2 - s_1}{v_1 - v_2} = \frac{\Delta q}{T \Delta v}  ~.
    \end{equation*}
    \begin{proof}
        In order to remain on the coexistence line, the costrain relation holds
        \begin{equation*}
            \mu_1(p, T) = \mu_2 (p, T) ~.
        \end{equation*}
        We differentiate it, keeping in mind that $p = p(T)$
        \begin{equation*}
            \pdv{\mu_1}{T} \Big \vert_p + \dv{\mu_1}{p} \Big \vert_T \dv{p}{T} = \pdv{\mu_2}{T} \Big \vert_p + \dv{\mu_2}{p} \Big \vert_T \dv{p}{T} ~,
        \end{equation*}
        hence 
        \begin{equation*}
            \dv{p}{T} = \frac{\pdv{\mu_1}{T} \vert_p - \pdv{\mu_2}{T} \vert_p}{\pdv{\mu_2}{p} \vert_T - \pdv{\mu_2}{p} \vert_T} ~.
        \end{equation*}

        At fixed number of particle, we can use the Gibbs free energy $G(p, T, N) = \mu (p, T) N$ or the Gibbs free energy per particle 
        \begin{equation*}
            g = \frac{G}{N} = \mu(p,T) ~.
        \end{equation*}
        Using the relations~\eqref{ges}
        \begin{equation*}
            \pdv{\mu}{p} \Big \vert_T = \pdv{g}{p} \Big \vert_T = \frac{1}{N} \pdv{G}{p} \Big \vert_T = \frac{V}{N} = v ~,
        \end{equation*}
        \begin{equation*}
            \pdv{\mu}{T} \Big \vert_p = \pdv{g}{T} \Big \vert_p = \frac{1}{N} \pdv{G}{T} \Big \vert_p = - \frac{S}{N} = - s ~,
        \end{equation*}
        we obtain
        \begin{equation*}
            \dv{p}{T} = \frac{\pdv{\mu_1}{T} \vert_p - \pdv{\mu_2}{T} \vert_p}{\pdv{\mu_2}{p} \vert_T - \pdv{\mu_2}{p} \vert_T} = \frac{s_2 - s_1}{v_2 - v_1} ~.
        \end{equation*}

        Furthermore, when there is a phase change, temperature remains constant whereas the thermal energy put in the system is transformed into latent heat
        \begin{equation*}
            \Delta s = \frac{\Delta q}{T} ~.
        \end{equation*}
        Therefore
        \begin{equation*}
            \dv{p}{T} = \frac{\Delta q}{T \Delta v} ~.
        \end{equation*}
    \end{proof}


    There could be $2$ different kind of phase transitions 
    \begin{enumerate}
        \item $1$st order phase transitions, i.e. those in which the $1$st derivatives of thermodynamic potentials are discontinuous;
        \item continuous phase transitions, i.e. those in which the higher derivatives of thermodynamic potentials are discontinuous.
    \end{enumerate}

    In our case, the former are those in which there is a jump $v_2 \neq v_1$ and $s_2 \neq s_1$ and the latter are those in which $v_2 = v_1$ and $s_2 = s_1$. 

\chapter{Theorems of Lee and Young}

    Consider a classical fluid in a volume $V \subset \mathbb R^3$. We treat it in the grancanonical ensemble. The grancanonical partition function is 
    \begin{equation*}
        \mathcal Z [V, T, z] = \sum_{N=0}^\infty z^n Z_N[T, V] ~.
    \end{equation*}

    The canonical partition function is 
    \begin{equation*}
    \begin{aligned}
        Z_N & = \frac{1}{N! h^{3N}} \int_{V^N} \prod_i d^3 q^i \underbrace{\int_{\mathbb R^{3N}} \prod_i d^3 p^i  \exp (- \beta \sum_j \frac{p_j}{2m}}_{\frac{1}{\lambda_T^{3N}}} + U_N(q^i)) \\ & = \frac{1}{N! \lambda_T^{3N}} \underbrace{\int_{V^N}\prod_i d^3 q^i \exp (- \beta U_N(q^i))}_{Q_N (T, V)} = \frac{Q_N(T, V)}{N! \lambda_T^{3N}} ~.
    \end{aligned}
    \end{equation*}
    Notice that $Q_N(T,V) > 0$.

    Therefore 
    \begin{equation*}
        \mathcal Z = \sum_{N=0}^\infty z^n \frac{Q_N}{N! \lambda_T^{3N}} ~.
    \end{equation*}
    which is a power series in $z$. Now we promote $z$ into a complex variables, keeping in mind that the physical states are only the ones for $z \in \mathbb R^+$

    We make the assumption that $U_N \geq - BN$ with $B > 0$, which means that it grows no more than $N$ order. This implies that 
    \begin{equation*}
        \exp(- \beta U_N) \leq \exp(\beta B N) ~,
    \end{equation*}
    hence 
    \begin{equation*}
        Q_N = \int_{V^N} \prod_i d^3 q^i \exp (- \beta U_N(q^i)) \leq \exp(\beta B N) \int_{V^N}\prod_i d^3 q^i = \exp(\beta B N) V^N 
    \end{equation*}
    and 
    \begin{equation*}
        Z_N = \frac{Q_N}{N! \lambda_T^{3N}} \leq \frac{V^N}{N! \lambda^{3N}_T} \exp(\beta B N) ~.
    \end{equation*}
    Therefore 
    \begin{equation*}
        |\mathcal Z| \leq \sum_{N=0}^\infty \frac{|z|^N}{N! \lambda_T^{3N}} V^N \exp(\beta B N) = \exp(\frac{V \exp(\beta B) |z|}{\lambda_T^3}) ~,
    \end{equation*}
    which, given the fact that it is an exponential, has an infinite convergence radius. We have proved that it is analytical $\forall z \in \mathbb C$, in particular for $z \in \mathbb R^+$. Furthermore, $\mathcal Z$ cannot become vanishing since it is convergent and it is a sum of positive terms. We introduce the granpotential 
    \begin{equation*}
        \Omega = - \frac{1}{\beta} \ln \mathcal Z ~,
    \end{equation*}
    which is well defined, since $\mathcal Z \neq 0$ and analytical $\forall z \in \mathbb R^+$. 

    Finally, all thermodynamic functions are analytical for $z \in \mathbb R^+$ and there are no phase transitions. How it it possible? We have not yet computed the thermodynamic limit.

    Consider a system composed of hard sphere occupying a finite volume $v$. The maximum number of particles is $M = \frac{V}{v}$. $\mathcal Z$ is a polynomial function in $z$ of degree $M$ and, by the fundamental theorem of algebra, it has $M$ zeroes but none in $\mathbb R^+$. If we go into the thermodynamic limit, $V \rightarrow \infty$, $M \rightarrow \infty$ and the number of zeroes increases. However, it holds that 
    \begin{enumerate}
        \item $\forall V, M$, there exists an open subset of $\mathbb R^+$ which does not contain zeroes, i.e. $\mathcal Z (V \rightarrow \infty, T, \mu)$ has no zeroes on $\mathbf R^+$,
        \item if zeroes accumulate towards a certain $z = z_c$, then  $\mathcal Z (V \rightarrow \infty, T, \mu)$ has a zero in $z = z_c$.
    \end{enumerate}
    This means that $\Omega$ is no longer analytical at $z = z_c$. Now, there is no more equilbrium and phase transitions come up from the dark. 

    This statements can be written down in terms of 
    \begin{equation*}
        \psi = \lim_{td} \frac{\ln \mathcal Z}{V} ~.
    \end{equation*}
    Hence 
    \begin{equation*}
        p \beta = \psi ~, \quad n = z \pdv{}{z} \psi ~.
    \end{equation*}
    \begin{proof}
        For the first 
        \begin{equation*}
            \Omega = - p V = - \frac{1}{\beta} \ln \mathcal Z ~,
        \end{equation*}
        hence
        \begin{equation*}
            p \beta = \frac{\ln \mathcal Z}{V} = \psi ~.
        \end{equation*}

        For the second,
        \begin{equation*}
            N = z \pdv{}{z} \ln \Omega = - \frac{z}{\beta} \pdv{}{z} \ln \Omega ~,
        \end{equation*}
        hence
        \begin{equation*}
            n = \frac{N}{V} = - \frac{z}{\beta} \pdv{}{z} \frac{\ln \Omega}{V} =  - \frac{z}{\beta} \pdv{}{z} \psi ~.
        \end{equation*}
    \end{proof}

    \begin{theorem}
        If $U_N \geq - BN$ with $B > 0$, if the boundary of the volume does not increases fastes than $V^{2/3}$, in order to neglect surface terms, then $\psi$ exists, it is continuous and monotonically increasing.
    \end{theorem}
    \begin{theorem}
        Given an open subset of an interval of $\mathbb R^+$ such that it doesn not contain zeroes, then $\psi$ exists and it is analytic.
    \end{theorem}
    \begin{corollary}
        A phase transitions may appear at $z = z_c$ if it is an accumulation point of zeroes. This point divides $\mathbb R^+$ into $2$ regions corresponding to $2$ different phases. Furthermore, $\psi$ is continuous but it is not analytic: $1st$ order phase transitions or continuous phase transitions.
    \end{corollary}

\chapter{Ising model}

    Consider a system composed by a discrete lattice, for example an hypercubic lattice of dimension $d$. For each vertex, there is a degree of freedom, characterised by the approximation of a spin that could have only two values $\sigma = \pm 1$. Two adjacent verteces are called nearest neighborhood. Each site has therefore $z$ nearest neighborhood, called the coordination number. For a dimension $d$ hybercube, $z = 2 d$. A possible configuration stae is defined as $\{\sigma_i\}_{i \in \mathcal L}$. The phase space is a discrete space composed by $2^N$ states $\{\{\sigma_i\}_{i \in \mathcal L}, \sigma_i = \pm 1\}$. 

    The hamiltonian of the system is 
    \begin{equation*}
        H(\sigma_i) = H_{int} + H_{field} ~,
    \end{equation*}
    where
    \begin{equation*}
        H_{int} = - J \sum_{i \text{near} j} \sigma_i \sigma_j 
    \end{equation*}
    and 
    \begin{equation*}
        H_{field} = - B \sum_{i=1}^{N} \sigma_i ~.
    \end{equation*}
    $B$ is an external magnetic field and $J$ is the interaction constant. Notice that in order to have a phase transitions, we have to allow interactions. For $J > 0$, the minimum energy configuration is the one in which all the spins are aligned $\sigma_i = \sigma_j ~, \quad \forall i, j$. This model is called ferromagnetic model. For $J < 0$, the minimum energy configuration is the one in which all the spins are antialigned $\sigma_i = - \sigma_j ~, \quad \forall i, j$. This model is called antiferromagnetic model. For $B > 0$, the minimum energy configuration is the one in which all the spins are aligned upwards $\sigma_i = + 1$. For $B < 0$, the minimum energy configuration is the one in which all the spins are aligned upwards $\sigma_i = - 1$. 

    In the canonical ensemble, the partition function is 
    \begin{equation*}
        Z_N = \sum_{\sigma_i = \pm 1} \exp(- \beta H(\sigma_i)) ~,
    \end{equation*}
    where the sum is made over all the $2^N$ states. The Helmoltz free energy is 
    \begin{equation*}
        F = E - TS = - \frac{1}{\beta} \ln Z_N ~.
    \end{equation*}
    where $E = \av{H}_c$. The thermodynamic equilibrium correspond to the configuration of minimum free energy.

    Suppose the external magnetic field is shut down. The ground state is the one with minimum energy and the entropy is small, because there are only $2$ states possible with all aligned spins. The excited state is the one with minimum energy and the entropy is big, because all spins point in all direction. Recall that entropy is $S = k_B \ln \Gamma(E)$. The minimal configuration of free energy is therefore at low $T$ with minimum $E$, i.e. all aligned, and at high $T$ with large $S$, i.e. random alignment.

    To stimate the alignment, we introduce the magnetisation 
    \begin{equation*}
        M = \av{\sum_{i=1}^N \sigma_i}_c = \sum_{i=1}^N \av{\sigma_i}_c ~,
    \end{equation*}
    where we have used the translation invariance. 

    Computing the phase diagram, we find that along the $T$-axis at $B=0$, $m \neq 0$ for $T < T_c$ and $m = 0$ for $T > T_c$, where $T_c$ is the critical temperature. The former is called the ferromagnetic phase and the latter is called the paramagnetic phase. We can use the magnetisation as an order parameter, since when it is zero there is disorder and when it is different from zero, there is order. In the neighborhood of $T_c$, we have the behaviour for $T < T_c$
    \begin{equation*}
        M \sim (T - T_c)^\beta ~,
    \end{equation*}
    where $\beta$ is a parameter. It characterise the phase transition, since it tells which speed $M \rightarrow 0$ when approacing $T \rightarrow T_c$.

\section{Correlation}

\section{Symmetry breaking}

    Consider the Ising model hamiltonian. The first term is invariant under 
    \begin{equation*}
        \sigma_i \rightarrow - \sigma_i ~, \quad \sigma_i \rightarrow \sigma_i ~.
    \end{equation*}
    Therefore, it is invariant under the global symmetry group $\mathbb Z_2$. If it were the only term, i.e. with $B=0$, the whole hamiltonian would be invariant under this group. Howver, the second term breaks explicitly the symmetry. Moreover, noticing that under this symmetry $m \rightarrow - m$, the only possible value of $m$ would be zero. Hence, for $T > T_c$ there is indeed $m=0$. But for $T < T_c$, the equilibrium state is no longer invariant under this symmetry. The hamiltonian remains the same but states are not invariant. There is a spontaneous symmetry breaking, spontaneous because $H$ is stil invariant under $\mathbb Z_2$.

\chapter{Mean-field treatment} 

    Consider the Ising model. In general, it is diffucult to compute the canonical partition function,
    \begin{equation*}
        Z_N = \sum_{\{\sigma_i = \pm 1\}} \exp(- \beta H) \neq (Z_1)^N ~,
    \end{equation*}
    since it is interacting. However, we can make an useful approximation 
    \begin{equation*}
        \sigma_i \sigma_j = ((\sigma_i - m) + m)((\sigma_j - m) + m) = m^2 + m(\sigma_i - m) + m (\sigma_j - m) + (\sigma_i - m)(\sigma_j - m) ~,
    \end{equation*}
    in which we keep only the first constant term and the second linear, but we neglect the last quadratic fluctuation term. Therefore 
    \begin{equation*}
        \sigma_i \sigma_j =  m^2 + m(\sigma_i - m) + m (\sigma_j - m) = m^2 + m\sigma_i - m^2 + m \sigma_j - m^2 = - m^2 + m (\sigma_i + \sigma_j) ~.
    \end{equation*}
    The hamiltonian in the mean-field approximation becomes 
    \begin{equation*}
        H_{mf} = - J \sum_{i \text{nn} j} (- m^2 + m(\sigma_i + \sigma_j)) - B \sum_i \sigma_i = m^2 J \sum_{i \text{nn} j} 1 - J m \sum_{i \text{nn} j} (\sigma_i + \sigma_j) - B \sum_i \sigma_i ~.
    \end{equation*}
    The number of links, given the coordination number $z$ which tells how many neighboring sites, is $NZ/2$. Hence 
    \begin{equation*}
        H_{mf} = \frac{m^2 z N J}{2} - Jmz \sum_i \sigma_i - B \sum_i \sigma_i = frac{m^2 z N J}{2} - (J m z + B) \sum_i \sigma_i ~.
    \end{equation*}
    The physical intepretation of the mean-field treatment is that we do not have to compute every links with respect to each others but only with respect to the mean field $m$. It is valid only if fluctuations are smaller than the mean-field. The partition function is 
    \begin{equation*}
    \begin{aligned}
        Z_N^{mf} & \sum_{\{\sigma_i = \pm 1\}} \exp(- \beta H_{mf}) \\ & = \exp(- \beta \frac{J z n m^2}{2}) \sum_{\{\sigma_i = \pm 1\}} \exp(\beta (B + Jmz) \sum_i \sigma_i) \\ & = \exp(- \beta \frac{J z n m^2}{2}) (\sum_{\{\sigma_i = \pm 1\}} \exp(\beta (B + Jmx) \sigma_i))^N \\ & = \exp(- \beta \frac{J z n m^2}{2}) (\exp(\beta(B + Jmz)) + \exp(- \beta (B + Jmz)))^N \\ & = \exp(- \beta \frac{J z n m^2}{2}) (2 \cosh (\beta (B + Jmz))) ~.
    \end{aligned}
    \end{equation*}
    The Helmoltz free energy is 
    \begin{equation*}
    \begin{aligned}
        F & = - \frac{1}{\beta} \ln Z_N^{mf} \\ & = - \frac{1}{\beta} (- \beta \frac{J z N m^2}{2}) N \ln (2 \cosh (\beta (B + Jmz))) \\ & = \frac{J z N m^2}{2} N \ln (2 \cosh (\beta (B + Jmz))) ~.
    \end{aligned}
    \end{equation*}
    The magnetisation is 
    \begin{equation*}
    \begin{aligned}
         m & = \frac{1}{N} \av{\sum_i \sigma_i}_c \\ & = \frac{1}{N} \sum_{\{\sigma_i = \pm 1\}} \sum_i \sigma_i \exp(- \beta H) \\ & = - \frac{1}{\beta N} \sum_{\{\sigma_i = \pm 1\}} \frac{1}{Z_N} \pdv{}{\beta} \exp(- \beta H) \\ & = - \frac{1}{\beta N} \pdv{\ln Z_N}{\beta} ~.
    \end{aligned}
    \end{equation*}
    Hence 
    \begin{equation*}
        m = \tanh (\beta (B + J m z)) ~.
    \end{equation*}
    We have a self-contistet equation for m to solve. The condition for solution is $B >0$ then $m > 0$ and $B < 0$ then $m < 0$. Particular attention we can study for $B = 0$ then 
    \begin{equation*}
        m = \tanh \frac{J m z}{k_B T} = \tanh \frac{T_c m}{T} ~,
    \end{equation*}
    where $T_c = J z / k_B$ is the critical temperature. Notice that it depends on $z$. Calling $\tilde m = T_c m / T$, we have 
    \begin{equation*}
        \frac{T \tilde m}{T_c} = \tanh \tilde m ~.
    \end{equation*}
    The solutions are points that intersect a straigh line and an hyperbolic tangent. If $T > T_c$, there is only one solution $m = 0$. If $T > T_c$, there are two solutions $\pm m_0$, one positive and one negative. See Figure~\ref{mf:m}.
    \begin{figure}
        \centering
        \scalebox{0.7}{\pyc{plot4('x', '2* x', 'x / 2', 'x', 'tanh(x)', 5, 5, 20, True, False, False)}}
        \caption{A plot of the graphical solution of $T \tilde m / T_c = \tanh \tilde m$.}
        \label{mf:m}
    \end{figure}
    We have proven that $m$ is indeed an order parameter. 
    
\appendix

\part{Appendix}

\chapter{Volume of an N-dimensional sphere}

    In this appendix chapter, we will prove that the volume of an $N$-dimensional sphere of radius $R$ is 
    \begin{equation}\label{app:volumen}
        V_n (R) = \frac{\pi^{n/2} R^n}{\Gamma(n/2 + 1)} ~.
    \end{equation}
    \begin{proof}
        Consider the rotationally invariant function $f$ 
        \begin{equation*}
            f(x_1, \ldots x_n) = \exp(- \frac{1}{2} \sum_{i=1}^{n} x_i^2 ) = \prod_{i=1}^{n} \exp(- \frac{1}{2} x_i^2 ) =~.
        \end{equation*}
        Using the Gaussian integral, this function can be integrated over all $\mathbb R^n$, with volume element $dV = dx_1 \ldots dx_n$, and it gives
        \begin{equation*}
        \begin{aligned}
            \int_{\mathbb R^n} dV ~ f & = \int_{\mathbb R^n} \prod_{i=1}^n dx_i ~ f = \int_{\mathbb R^n} \prod_{i=1}^n dx_i ~ \exp(- \frac{1}{2} \sum_{i=1}^{n} x_i^2 ) \\ & = \prod_{i=1}^{n} \underbrace{( \int_{\mathbb R} dx_i ~ \exp(- \frac{1}{2} x_i^2 ))}_{(2 \pi)^{1/2}} = \prod_{i=1}^{n} (2 \pi)^{1/2} = (2 \pi)^{n/2} ~.
        \end{aligned}
        \end{equation*}
        Exploiting the rotational invariant property, we can decomposed the volume element into a surface element $dA$, which integrated gives an $(n-1)$-dimensional sphere $S^{n-1} (r)$ of radius $r$, multiplied by a length element $dr$, i.e.
        \begin{equation*}
            \int_{\mathbb R^n} dV ~ f = \int_0^\infty dr \int_{S^{n-1} (r)} dA ~ f ~.
        \end{equation*}
        Since the area is proportial to the radius, e.g.~for $n=3$ the area is $A \propto r^2$, the radius-dependence of the area is given by $A_{n-1}(r) = r^{n-1} A_{n-1} (1)$. Therefore, putting it inside the integral, we obtain 
        \begin{equation*}
            A_{n-1} (1) \int_0^\infty dr r^{n-1} \exp(- \frac{1}{2} r^2) ~.
        \end{equation*}
        Now, we make a change of variables into 
        \begin{equation*}
            t = \frac{r^2}{2} ~, \quad r = (2t)^{1/2} ~, \quad dr = 2^{-1/2} t^{-1/2} dt
        \end{equation*}
        to have the integral of the gamma function
        \begin{equation*}
        \begin{aligned}
            \int_0^\infty dr ~ r^{n-1} \exp(- \frac{1}{2} r^2) & = 2^{(n-1)/2} 2^{-1/2}\int_0^\infty dt ~ t^{(n-1)/2} t^{-1/2} \exp(-t) \\ & = 2^{n/2 - 1} \underbrace{\int_0^\infty dt ~ t^{n/2 - 1} \exp(-t)}_{\Gamma(n/2)} = 2^{n/2 - 1} \Gamma(n/2) ~.
        \end{aligned}
        \end{equation*}
        Now, we combine the two results together to obtain the surface
        \begin{equation*}
            (2 \pi)^{n/2} = A_{n-1} (1) 2^{n/2 - 1} \Gamma(n/2) ~,
        \end{equation*}
        hence 
        \begin{equation*}
            A_{n-1} (1) = \frac{2 \pi^{n/2}}{\Gamma(n/2)} ~.
        \end{equation*}
        Finally, in order to find the volume we need to integrate from $0$ to $R$ 
        \begin{equation*}
        \begin{aligned}
            V_n(R) & = \int_0^R dr A_{n-1} (r) = \int_0^R dr ~ A_{n-1} (1) r^{n-1} = \frac{2 \pi^{n/2}}{\Gamma(n/2)} \int_0^R dr ~ r^{n-1} \\ & = \frac{2 \pi^{n/2}}{\Gamma(n/2)} \frac{r^n}{n} \Big \vert_0^R = \frac{2 \pi^{n/2}}{n\Gamma(n/2)} R^n = \frac{\pi^{n/2} R^n}{\Gamma(n/2 + 1)} ~.
        \end{aligned}
        \end{equation*}
    \end{proof}

\chapter{Stirling approximation}

    In this appendix chapter, we will prove the Stirling approximation 
    \begin{equation}\label{app:stirl}
        \ln n! \simeq n \ln n - n ~.
    \end{equation}
    \begin{proof}
        The factorial can be expressed in integral form via the gamma function 
        \begin{equation*}
            \Gamma (n + 1) = n! = \int_0^\infty dt ~ t^n \exp(-t) ~.
        \end{equation*}
        Now, we make a change of variables into 
        \begin{equation*}
            t = n x ~, \quad x = \frac{t}{n} ~, \quad dx = \frac{dt}{n} ~,
        \end{equation*}
        to have 
        \begin{equation*}
        \begin{aligned}
            \int_0^\infty dt ~ t^n \exp(-t) & = int_0^\infty dt ~ \exp(\ln t^n) \exp(-t) \\ & = \int_0^\infty dt ~ \exp(n\ln t - t) \\ & = n \int_0^\infty dx ~ \exp(n \ln (nx) - nx) \\ & = n \int_0^\infty dx ~ \exp(n \ln x + n \ln n - nx) \\ & = n \exp(n \ln n) \int_0^\infty dx ~ \exp(n (\ln x - x)) ~.
        \end{aligned}
        \end{equation*}
        In the limit for which $n$ is large, we can use the Laplace approximation method 
        \begin{equation*}
            \int_a^b dx ~ \exp(n f(x)) \simeq \exp(n f(x_0)) \sqrt{\frac{2\pi}{n |f'' (x_0)|}} ~.
        \end{equation*}
        where $x_0 \in [a, b]$ is a stationary point of $f(x)$. A simple sketch of the proof is given by means of the Taylor expansion around $x_0$
        \begin{equation*}
            f(x) \simeq f(x_0) - \frac{1}{2} |f''(x_0)| (x - x_0)^2 ~,
        \end{equation*}
        hence, integrating the Gaussian integral,
        \begin{equation*}
        \begin{aligned}
            \int_a^b dx ~ \exp(n f(x)) & \simeq \exp(n f(x_0)) \int_a^b dx ~ \exp(- \frac{n}{2} |f''(x_0)| (x - x_0)^2) \\ & = \sqrt{\frac{2\pi}{n |f'' (x_0)|}} ~.
        \end{aligned}
        \end{equation*}
        In our case, $a=0$, $b=\infty$ and $f(x) = \ln x - x$, which has a maximum in $x_0 = 1$ and second derivatives equals to $|f''(x)| = 1 / x^2$. Therefore
        \begin{equation*}
            \int_0^\infty dx ~ \exp(n (\ln x - x)) \simeq \exp(n (\ln x_0 - x_0)) \sqrt{\frac{2\pi x_0^2}{n}} \Big \vert_{x_0 = 1} = \exp(- n) \sqrt{\frac{2\pi}{n}} ~.
        \end{equation*}
        Now, we combine the two results together
        \begin{equation*}
            n! \simeq n \exp(n \ln n) \exp(- n) \sqrt{\frac{2\pi}{n}} = \exp(n \ln n - n) \sqrt{2 \pi n} = n^n \exp(-n) \sqrt{\frac{2\pi}{n}} ~,
        \end{equation*}
        which can be rewritten in terms of logarithms rather than exponentials 
        \begin{equation*}
            \ln n! \simeq \ln (n^n \exp(-n) \sqrt{\frac{2\pi}{n}} ) = n \ln n - n + O(\ln n) ~.
        \end{equation*}
    \end{proof}

\chapter{Gaussian integral}

    In this appendix chapter, we will prove that the Gaussian integral is
    \begin{equation}\label{app:gauss}
        \int_{-\infty}^\infty dx ~ \exp(- x^2) = \sqrt{\pi} ~.
    \end{equation}
    \begin{proof}
        We start from the square Gaussian integral, which it is the square same integral for the mute properties of the integration variables 
        \begin{equation*}
        \begin{aligned}
            \Big (\int_{-\infty}^\infty dx ~ \exp(- x^2) \Big)^2 & = \int_{-\infty}^\infty dx ~ \exp(- x^2) \int_{-\infty}^\infty dy ~ \exp(- y^2) \\ & = \int_{-\infty}^\infty dx \int_{-\infty}^\infty dy ~ \exp(- (x^2 + y^2)) ~.
        \end{aligned}
        \end{equation*}
        Now, we make a change of variables and we use polar coordinates $(r, \theta)$
        \begin{equation*}
            r^2 = x^2 + y^2 ~, \quad \theta = \arctan \frac{y}{x} ~, \quad dx ~ dy = r ~ dr ~ d\theta ~, \quad (r, \theta) \in [0, \infty) \times [0, 2\pi] ~,
        \end{equation*} 
        to obtain
        \begin{equation*}
        \begin{aligned}
            \int_{-\infty}^\infty dx \int_{-\infty}^\infty dy ~ \exp(- (x^2 + y^2)) & = \underbrace{\int_0^{2\pi} d\theta}_{2\pi} \int_0^\infty dr ~ r \exp(- r^2) \\ & = 2 \pi \int_0^\infty dr ~ r \exp(- r^2) \\ & = \pi \int_0^\infty dr ~ 2 r \exp(- r^2) \\ & = \pi \exp(- r^2) \Big \vert_0^{\cancel \infty} = \pi ~.
        \end{aligned}
        \end{equation*}
        Now, we combine the two results together
        \begin{equation*}
            \Big (\int_{-\infty}^\infty dx ~ \exp(- x^2) \Big)^2 = \pi ~,
        \end{equation*}
        hence
        \begin{equation*}
            \int_{-\infty}^\infty dx ~ \exp(- x^2) = \sqrt{\pi} ~.
        \end{equation*}
    \end{proof}
    

\backmatter

\nocite{smlecture}
\nocite{ercolessi}

\clearpage
\phantomsection
\printbibliography

\end{document}
