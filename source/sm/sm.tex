\documentclass[a4paper, 12pt, openany]{memoir}

\usepackage[a4paper, top = 4cm, bottom = 4cm, left = 3cm, right = 3cm]{geometry}

\usepackage[T1]{fontenc}
\usepackage[utf8]{inputenc}
\usepackage{pythontex} 
\usepackage{nopageno} 
\usepackage{pgf}

\usepackage{tocloft}
\newcommand{\listequationsname}{List of Equations}
\newlistof{listofequations}{equ}{\listequationsname}
\newcommand{\myequation}[1]{%
	\addcontentsline{equ}{equation}{\protect\numberline{\theequation}#1}\par
}
\makeatletter
\let\l@equation\l@figure
\makeatother

\usepackage{xcolor}
\xdefinecolor{mycolor}{RGB}{0,175,179} 
\usepackage{hyperref}
\hypersetup{colorlinks, linkcolor={mycolor}, citecolor={mycolor}, urlcolor={mycolor}}

\usepackage{lipsum}

\renewcommand{\aftertoctitle}{\afterchaptertitle\par\nobreak\hfill{\normalfont{Page}}\par\nobreak}

\usepackage{titlesec}
\titleformat{\part}[display]
  {\normalfont\HUGE\bfseries\color{mycolor}\centering}
  {Part \thepart}{20pt}{\HUGE\normalfont\color{black}}
\titleformat{\chapter}[display]
  {\normalfont\HUGE\bfseries\color{mycolor}\centering}
  {Chapter \thechapter}{20pt}{\HUGE\normalfont\color{black}}
\titleformat{\section}
  {\normalfont\Large\bfseries\color{mycolor}\centering}
  {\thesection}{1em}{}
\titleformat{\subsection}
  {\normalfont\large\bfseries\color{mycolor}\centering}
  {\thesubsection}{1em}{}

\renewcommand{\printtoctitle}[1]{\HUGE\normalfont\color{black}#1}

\usepackage[backend=bibtex, sorting=none]{biblatex}
\addbibresource{../bibliography.bib}

\usepackage{amsmath}
\usepackage{amsthm}
\usepackage{thmtools}
\usepackage{mathtools}

\newtheorem{principle}{Principle}[chapter]
\newtheorem{lemma}{Lemma}[chapter]
\theoremstyle{definition}
\newtheorem{example}{Example}[chapter]
\renewcommand\qedsymbol{q.e.d.}

\theoremstyle{remark}
\newtheorem{case}{Case}

\newcommand{\dv}[2]{\frac{d#1}{d#2}}
\newcommand{\dvin}[3]{\frac{d#1}{d#2}\Big\vert_{#3}}
\newcommand{\dvd}[2]{\frac{d^2#1}{d#2^2}}
\newcommand{\dvf}[2]{\frac{\delta #1}{\delta #2}}
\newcommand{\pdv}[2]{\frac{\partial#1}{\partial#2}}
\newcommand{\pdvd}[3]{\frac{\partial^2 #1}{\partial#2 \partial#3}}
\newcommand{\pdvdu}[2]{\frac{\partial^2 #1}{\partial#2^2}}
\newcommand{\integ}[3]{\int_{#1}^{#2}d#3~}
\newcommand{\poi}[2]{[#1,~#2]}
\newcommand{\poiexp}[2]{\pdv{#1}{q^i} \pdv{#2}{p_i} - \pdv{#2}{q^i} \pdv{#1}{p_i}}

\newcommand{\comm}[2]{[#1,~#2]}
\newcommand{\set}[2]{\{#1\colon#2\}}
\newcommand{\inner}[2]{\langle#1,~#2\rangle}
\newcommand{\av}[1]{\langle#1\rangle}
\newcommand{\avp}[2]{\langle#1\rangle_{#2}}
\newcommand{\ket}[1]{\vert#1\rangle}
\newcommand{\bra}[1]{\langle#1\vert}
\newcommand{\braket}[2]{\langle#1\vert#2\rangle}

\newtheoremstyle{colored}{}{}{\itshape}{}{\color{mycolor}\normalfont\bfseries\indent}{}{\newline}{}

\declaretheorem[
  style=colored,
  name=Definition,
  numberwithin=chapter,
]{definition}

\declaretheorem[
  style=colored,
  name=Theorem,
  numberwithin=chapter,
]{theorem}

\declaretheorem[
  style=colored,
  name=Corollary,
  numberwithin=chapter,
]{corollary}

\declaretheorem[
  style=colored,
  name=Law,
  numberwithin=chapter,
]{law}

\declaretheorem[
  style=colored,
  name=Principle,
  numberwithin=chapter,
]{princ}

\usepackage{amsfonts}
\usepackage{dsfont}
\usepackage{yfonts}
\usepackage{amssymb}

\let\oldproof\proof
\renewcommand{\proof}{\color{darkgray}\oldproof}

\let\oldexample\example
\renewcommand{\example}{\color{darkgray}\oldexample}

\usepackage{cancel}
\usepackage{indentfirst}

\usepackage{tikz}
\usepackage{amssymb}
\usepackage{pgfplots}
\usepgfplotslibrary{patchplots}
\usetikzlibrary{patterns, positioning, arrows}
\pgfplotsset{compat=1.15}

\DeclareMathOperator{\tr}{tr}
\DeclareMathOperator{\str}{str}
\DeclareMathOperator{\real}{Re}
\DeclareMathOperator{\imm}{Im}
\DeclareMathOperator{\sgn}{sgn}
\DeclareMathOperator{\spann}{span}
\DeclareMathOperator{\vol}{vol}

\usetikzlibrary{positioning, arrows.meta}




\title{statistical mechanics}
\author{Matteo Zandi \\ ~ \\ (matteo.zandi2@studio.unibo.it)}
\date{\today}

\newcommand{\subt}{what happens when there are too many particles?}

\begin{document}

\frontmatter

\pagestyle{empty}
{\raggedleft\vspace*{\baselineskip}
{\LARGE Matteo Zandi}\\[0.35\textheight]
{\HUGE \textcolor{mycolor}{\textbf{On~\thetitle:}}}\\[\baselineskip]
{\LARGE \subt }\\[\baselineskip]
{\large \thedate}\par
\vspace*{2\baselineskip}
\vfill
{\large matteo.zandi2@studio.unibo.it}\par
\vspace*{\baselineskip}}
\clearpage
\pagestyle{headings}

\blankpage

\tableofcontents

\mainmatter

\begin{pycode}
import sympy as sy
def plot1(x, f, rangex, rangey, fig, leg, negx, negy):
    rangexx = rangex
    rangeyy = rangey
    if negx == True:
        rangexx = 0
    if negy == True:
        rangeyy = 0
    x = sy.Symbol('x')
    p = sy.plot((f, (x, -rangexx, rangex)), ylim=[-rangeyy, rangey], legend= leg, show=False, line_color='#00AFB3')
    p.save(f'fig/fig{fig}.pgf')
    print(r'\input{fig/fig'+ rf'{fig}' + r'.pgf}')

def plot4(x, f, g, h, l, rangex, rangey, fig, leg, negx, negy):
    rangexx = rangex
    rangeyy = rangey
    if negx == True:
        rangexx = 0
    if negy == True:
        rangeyy = 0
    x = sy.Symbol('x')
    p = sy.plot((f, (x, -rangexx, rangex)), (g, (x, -rangex, rangex)), (h, (x, -rangex, rangex)), (l, (x, -rangex, rangex)), ylim=[-rangeyy, rangey], legend= leg, show=False, line_color='#00AFB3')
    p[3].line_color='black'
    p.save(f'fig/fig{fig}.pgf')
    print(r'\input{fig/fig'+ rf'{fig}' + r'.pgf}')

def der(y, x):
    x = sy.Symbol(x) 
    derivative = sy.diff(y, x)
    return sy.latex(derivative) 
 
def indint(integrand, x): 
    x = sy.Symbol(x) 
    integral = sy.integrate(integrand,x) 
    return sy.latex(integral) 

def defint(integrand, x, min, max): 
    x = sy.Symbol(x) 
    integral = sy.integrate(integrand, (x, min, max)) 
    return sy.latex(integral) 

def infint(integrand, x): 
    x = sy.Symbol(x) 
    integral = sy.integrate(integrand, (x, float('-inf'), float('inf'))) 
    return sy.latex(integral) 

def ode(ode, y, x): 
    x = sy.Symbol(x) 
    y = sy.Function(y) 
    lhs, rhs = ode.split('=') 
    ode = sy.Eq(sy.S(lhs),sy.S(rhs)) 
    sol = sy.dsolve(ode,y(x)) 
    return sy.latex(sol) 

# \py{ode("Derivative(y(x),x,x) + y(x) = 0", "y", "x")} ~.
 
def odeic(ode, y, x, ic): 
    x  = sy.Symbol(x) 
    y  = sy.Function(y) 
    lhs,rhs = ode.split('=') 
    ode = sy.Eq(sy.S(lhs),sy.S(rhs)) 
    sol = sy.dsolve(ode,y(x), ics= sy.S(ic)) 
    return sy.latex(sol) 

#\py{odeic("Derivative(y(x),x,x) + y(x) = 0", "y", "x", "{y(0):1, y(x).diff(x).subs(x, 0): 0}")} ~.

def matrixmult(A, B):
    C = A*B
    return sy.latex(C)

def Taylor(x, f, point, order):
    x = sy.Symbol('x')
    ts = sy.series(f, x, point, order) 
    return sy.latex(ts)

def limit(x, f, point):
    x = sy.Symbol('x')
    lim = sy.limit(f, x, point) 
    return sy.latex(lim)

\end{pycode}

\chapter*{Introduction}

    In these lectures notes, we will cover the part of physics which studies systems composed by a large amount of constituents, like particles: thermodynamics and statistical mehcanics. In the first part, we will recall some notions of thermodynamics, in the language of differential geometry: the laws of thermodynamics and thermodynamics potentials. In the second part, we will study classical statistical mechanics, starting from the basic notions of classical (hamiltonian) mechanics till the $3$ ensembles: microcanonical, canonical and grancanonical. In the third part, we will study quantum statistical mechanics, starting from the basic notions of quantum mechanics, in the language of canonical and second quantisation, till the properties of bosons and fermions. At the end of each of these last $2$ parts, there will be some applications and exercises. In the last part, we will superficially introduce classical phase transitions and the classical Ising model. 

\part{Thermodynamics}

\chapter{The 2 laws, which are 4}

    In this chapter, we will recall some notions of thermodynamics: states, equilibrium and the laws of thermodynamics.

\section{Equilibrium}

    The topic of which thermodynamics studies is a class of systems composed by a large amount of particles, roughly speaking Avogadro number $N_A \simeq 6 \times 10^{23}$ consituents, once it reaches a macroscopical equilibrium configuration. To understand the notion of equilibrium, consider a system immersed in its surroundings which can interact with the latter by exchanging matter and/or energy (mechanical, electric, magnetic, chemical work) or it can be completely isolated. Once a sufficient amount of time has gone past, it reaches a configuration. Which particular configuration and its stability can be selected by the different boundary conditions the system finds itself in, i.e.~the specification on how the system is in contact and how it interacts with its surroundings. In other words, there is only one and only one final equilibrium configuration towards to the system evolves, once boundary conditions have been given. However, the way the system reaches the equilibrium configuration is irreversible. Equilibrium therefore means that once the system has reached its final configuration, it will stay there forever. 

\section{States}

    A state is a macroscopic configuration. Mathematically speaking, it is a point in the manifold $\mathcal M$ of thermodynamic states. To describe it, we need a chart, given by macroscopical physical quantities, called thermodynamical variables. They can be divided into two groups, one conjugate to the other, according to their behaviour when the physical system is rescaled, i.e.~when volume and number of particles change: extensive variables do scale with it whereas intensive ones do not. Some of them are written in Table~\ref{table:td:1}. However, we have to be careful, since only volume is (by definition) extensive and all the others quantities can be considered extensive only if the surface terms are negligible when we take the thermodynamic limit. 

    Each physical system has an equation of state, i.e.~a functional relation among thermodynamic quantities, which restrict the number of independent variables. Geometrically, it means that the only admissible states are a submanifold of the entire manifold of states, given by the constrain induced by the equation of state.

    \begin{example}[Perfect gas]
        Consider a perfect gas. A chart on its $3$-dimensional manifold can be $(p, V, T)$ and its equation of state is $PV = N k_B T$. This means that the allowed states are in a $2$-dimensional manifold embedded in $\mathbb R^3$.
    \end{example}

    \begin{table}
        \centering
        \begin{tabular}{c | c }
            Extensive & Intensive \\
            \hline
            energy $E$ & - \\ 
            entropy $S$ & temperature $T$ \\ 
            volume $V$ & pression $p$\\ 
            number of particles $N$ & chemical potential $\mu$ \\ 
            polarization $\mathbf P$ & electric field $\mathbf E$ \\ 
            magnetization $\mathbf M$ & magnetic field $\mathbf B$ \\ 
        \end{tabular}
        \caption{Extensive and intensive thermodynamical variables.}
        \label{table:td:1}
    \end{table}

\section{The laws of thermodynamics}

    Thermodynamics is governed by a set of laws which every system must obey. They are particular kind of laws, since they are limitation laws: they tell us only which processes cannot happen. 

    \begin{law}[0th]
        Let $A$ and $B$ be $2$ thermodynamic systems in thermal contact. At equilibrium, only a subset of states $\mathcal A \subset \mathcal M_A \times \mathcal M_B$ is accessible and not the whole manifold. Mathematically, it means that there exists a functional relation of the kind
        \begin{equation}\label{a1}
            F_{AB} (a,b)= 0 ~,
        \end{equation}
        with $a \in \mathcal M_A$ and $b \in \mathcal M_B$. Moreover, thermal equilibrium is an equivalence class, which can be proved that it means 
        \begin{equation}\label{a2}
            F_{AB} (a,b) = f_A(a) - f_B(b) ~.
        \end{equation}
        Putting together~\eqref{a1} and~\eqref{a2}, we define the empirical temperature 
        \begin{equation*}
            t_A = f_A(a) = t_B = f_B(b) ~.
        \end{equation*}
    \end{law}

    It is a limitation law because it limits which possible configuration a system can reach, when in thermal contact with a second one. 

    \begin{law}[1st]
        Let $\delta Q$ be the infinitesimal heat and $\delta L$ the infinitesimal work exchanged in a quasi-static process ($\delta Q > 0$ means absorbed by the system, $\delta L > 0$ means performed by the system). For any cyclic process, i.e.~processes in which the initial and the final state coincide, we have
        \begin{equation*}
            \oint (\delta Q - \delta L) = 0 ~.
        \end{equation*}
        This means that $\delta Q - \delta L$ is a $1$-form, which vanishes when line-integrated along a closed curve in $\mathcal M$. This implies, by the Poincaré lemma, that it is an exact differential 
        \begin{equation*}
            dE = \delta Q - \delta L ~,
        \end{equation*}
        called the internal energy. However, notice that heat and work are not exact differential, since $\oint \delta Q \neq 0$ and $\oint \delta H \neq 0$. 
        
        The generalisation to a system that can exchange matter is given by
        \begin{equation}\label{td:1st}
            \oint (\delta Q - \delta L + \mu dN) = \oint dE = 0 ~, \quad dE = \delta Q - \delta L + \mu dN ~,
        \end{equation} 
        where $\mu$ is the chemical potential (the necessary energy to add or remove a particle). Furthermore, we can expressed both $\delta Q$ and $\delta L$ as a linear combination of infinitesimal change of independent coordinates, e.g. $\delta L = p dV + B dM$. We assume that the internal energy is extensive and, therefore, the chemical potential is intensive.
    \end{law}

    It is a limitation law because it limits the configuration that a system can reach in isolation to those with $E = const$. 

    \begin{law}[2nd]
        For any cyclic process
        \begin{equation*}
            \oint \frac{\delta Q}{T} \begin{cases}
                = 0 & \textnormal{reversible process} \\
                < 0 & \textnormal{irreversible process} \\
            \end{cases} ~.
        \end{equation*}
        For reversible processes, $\frac{\delta Q}{T} = 0$ is an exact differential. This implies that we can define a function
        \begin{equation*}
            S(a) - S(b) = \int_a^b \frac{\delta S}{T} ~,
        \end{equation*}
        called the entropy. The integral is along any reversible path. Therefore, we have  
        \begin{equation}\label{td:2nd}
            dS \begin{cases}
                = 0 & \textnormal{reversible process} \\
                < 0 & \textnormal{irreversible process} \\
            \end{cases} ~.
        \end{equation}
    \end{law}

    It is a limitation law because it limits the configuration that a system can reach in isolation to whose in which entropy cannot decrease. 

    \begin{law}[3rd]
        Isothermal and adiabatic processes coincides whet $T=0$, or, equivalently, it is impossible to reach $T=0$ with a finite number of processes. Mathematically,
        \begin{equation*}
            \Delta S \rightarrow 0 ~\textnormal{as}~ T \rightarrow 0 ~.
        \end{equation*}
        Therefore, $T=0$ is a singular point. Furthermore, if it were possible to reach $T=0$, the second law $\delta Q \leq 0$ implies that it is impossible to raise the temperature. It is a thermodynamic features, since it can be proved that it is impossible to realize an engine with efficency $\eta = 1$.
    \end{law}

    It is a limitation law because it limits the configuration that a system can reach in isolation to whose in which $T \neq 0$.

\chapter{Thermodynamic potentials}

    In this chapter, we will study thermodynamic potentials: energy, entropy, Helmolts free energy, enthalpy, Gibbs free energy and grand potential. We will derive their definition, their differential and their equations of state.

\section{Internal energy}

\subsection{Fundamental equation}

    Thermodynamic potentials are functions defined in the manifold, which are suited for a particolar choice of the $3$ coordinates and therefore they are useful if we find the system with the other coordinates constant. The first thermodynamic potential we are going to study is the internal energy $E$. It is defined by the first law of thermodynamic and its differential is
    \begin{equation}\label{td:d:e}
        dE \leq T dS - pdV + \mu dN ~.
    \end{equation}
    This relation is called the fundamental equation of thermodynamics.
    \begin{proof}
        In fact, we invert~\eqref{td:1st}
        \begin{equation*}
            \delta Q = dE + \delta L - \mu dN ~,
        \end{equation*}
        we use $\delta L = p dV$ and we put it into~\eqref{td:2nd}
        \begin{equation}
            dS \leq \frac{\delta Q}{T} = \frac{dE + p dV - \mu dN}{T} ~.
        \end{equation}
        Finally, we isolate $dE$
        \begin{equation}
            dE \leq TdS - p dV + \mu dN ~.
        \end{equation}
    \end{proof} 

\subsection{Equation of state}

    Notice that the non-differential variables are intensive and the differential variables are extensive. This tells us that $E(S, V, N)$ is a function of the extensive variables $S$, $V$ and $N$. The intensive variables $T$, $p$ and $\mu$ can be derived from $E$ by the following relations 
    \begin{equation}\label{td:es:e}
        T = \pdv{E}{S} \Big \vert_{V,N} ~, \quad p = - \pdv{E}{V} \Big \vert_{S,N} ~, \quad \mu = \pdv{E}{N} \Big \vert_{S,V} ~. 
    \end{equation}
    This are called the equation of state of the system, since we can calculate one variable from it, e.g. $T = T(S,V,N)$, $p = p(S,V,N)$ or $\mu = \mu(S,V,N)$. 
    \begin{proof}
        At constant $V$ and $N$,~\eqref{td:d:e} becomes
        \begin{equation*}
            dE = TdS - p \underbrace{dV}_0 + \mu \underbrace{dN}_0 = TdS ~,
        \end{equation*}
        hence 
        \begin{equation*}
            T = \pdv{E}{S} \Big \vert_{V,N} ~.
        \end{equation*}

        At constant $S$ and $N$,~\eqref{td:d:e} becomes
        \begin{equation*}
            dE = T\underbrace{dS}_0 - p dV + \mu \underbrace{dN}_0 = - p dV ~,
        \end{equation*}
        hence 
        \begin{equation*}
            p = - \pdv{E}{V} \Big \vert_{S,N} ~.
        \end{equation*}

        At constant $S$ and $V$,~\eqref{td:d:e} becomes
        \begin{equation*}
            dE = T\underbrace{dS}_0 - p \underbrace{dV}_0 + \mu dN = \mu dN ~,
        \end{equation*}
        hence 
        \begin{equation*}
            \mu = \pdv{E}{S} \Big \vert_{S,V} ~.
        \end{equation*}
    \end{proof}

\subsection{Extensive and intensive}

    $E$ is an extensive variable, i.e.~an homogeneous function of degree one of the extensive variables 
    \begin{equation*}
        E(\lambda S, \lambda V, \lambda N) = \lambda E(S, V, N) ~, \quad \forall \lambda > 0 ~.
    \end{equation*}
    The physical meaning is that if we rescale the volume, the energy is rescaled bu the same amount.
    Moreover, the intensive variable are homogeneous function of degree zero of the extensive variables 
    \begin{equation}\label{a5}
        T(S, V, N) = T(\frac{S}{N}, \frac{V}{N}) ~, \quad p(S, V, N) = p(\frac{S}{N}, \frac{V}{N}) ~, \quad \mu(S, V, N) = \mu(\frac{S}{N}, \frac{V}{N}) ~.
    \end{equation}
    By homogeneity properties 
    \begin{equation*}
        E = N E (\frac{S}{N}, \frac{V}{N}, 1) = N e ~, \quad  S = N S(\frac{E}{N}, \frac{V}{N}, 1) = N s ~,
    \end{equation*}
    we can define specific energy and entropy 
    \begin{equation*}
        e = \frac{E}{N} = e(s, v) ~, \quad s = \frac{S}{N} = s(e, v) ~,
    \end{equation*}
    where $v = \frac{V}{N}$ is the specific volume.
    The Euler's theorem allows us to state that, if $E$ is smooth, it can be written as 
    \begin{equation*}
        E = S \pdv{E}{S} + V \pdv{E}{V} + N \pdv{E}{N} ~,
    \end{equation*}
    or, using~\eqref{td:d:e} and~\eqref{td:es:e}, 
    \begin{equation}\label{td:e}
        E = TS - pV + \mu N ~.
    \end{equation}

\subsection{Integrability condition} 

    In order to be an exact differential, the exterior derivative of the right handed side of~\eqref{td:d:e} must have a null exterior derivative
    \begin{equation}\label{td:int:e}
        - \pdv{T}{V} \Big \vert_{S,N} = \pdv{p}{S} \Big \vert_{V,N} ~, \quad 
        \pdv{T}{N} \Big \vert_{S,V} = \pdv{\mu}{S} \Big \vert_{N, V} ~, \quad 
        - \pdv{p}{N} \Big \vert_{V,S} = \pdv{\mu}{V} \Big \vert_{N, S} ~. 
    \end{equation}
    \begin{proof}
        By means of the exterior derivative, we have 
        \begin{equation*}
        \begin{aligned}
            d (dE) & = d (T dS) - d (p dV) + d (\mu dN) \\ & = \pdv{T}{S} \underbrace{dS \wedge dS}_0 + \pdv{T}{V} dV \wedge dS + \pdv{T}{N} dN \wedge dS - \pdv{p}{S} dS \wedge dV - \pdv{p}{V} \underbrace{dV \wedge dV}_0 \\ & \quad - \pdv{p}{N} dN \wedge dV + \pdv{\mu}{S} dS \wedge dN + \pdv{\mu}{V} dV \wedge dN + \pdv{\mu}{N} \underbrace{dN \wedge dN}_0 \\ & = \pdv{T}{V} dV \wedge dS + \pdv{T}{N} dN \wedge dS - \pdv{p}{S} dS \wedge dV \\ & \quad - \pdv{p}{N} dN \wedge dV + \pdv{\mu}{S} dS \wedge dN + \pdv{\mu}{V} dV \wedge dN ~.
        \end{aligned}
        \end{equation*}

        At constant $N$ 
        \begin{equation*}
        \begin{aligned}
            0 & = d^2 E = \pdv{T}{V} dV \wedge dS + \pdv{T}{N} \underbrace{dN}_0 \wedge dS - \pdv{p}{S} dS \wedge dV \\ & \quad - \pdv{p}{N} \underbrace{dN}_0 \wedge dV + \pdv{\mu}{S} dS \wedge \underbrace{dN}_0 + \pdv{\mu}{V} dV \wedge \underbrace{dN}_0 \\ & = \pdv{T}{V} dV \wedge dS - \pdv{p}{S} dS \wedge dV = \pdv{T}{V} dV \wedge dS + \pdv{p}{S} dV \wedge dS ~,
        \end{aligned}
        \end{equation*}
        hence, by the linear independence of $V$ and $S$,
        \begin{equation*}
            - \pdv{T}{V} \Big \vert_{S,N} = \pdv{p}{S} \Big \vert_{V,N} ~.
        \end{equation*}

        At constant $V$ 
        \begin{equation*}
        \begin{aligned}
            0 & = d^2 E = \pdv{T}{V} \underbrace{dV}_0 \wedge dS + \pdv{T}{N} dN \wedge dS - \pdv{p}{S} dS \wedge \underbrace{dV}_0 \\ & \qquad - \pdv{p}{N} dN \wedge \underbrace{dV}_0 + \pdv{\mu}{S} dS \wedge dN + \pdv{\mu}{V} \underbrace{dV}_0 \wedge dN \\ & = \pdv{T}{N} dN \wedge dS + \pdv{\mu}{S} dS \wedge dN = \pdv{T}{N} dN \wedge dS - \pdv{\mu}{S} dN \wedge dS~,
        \end{aligned}
        \end{equation*}
        hence, by the linear independence of $N$ and $S$,
        \begin{equation*}
            \pdv{T}{N} \Big \vert_{S,V} = \pdv{\mu}{S} \Big \vert_{N, V} ~.
        \end{equation*}

        At constant $S$ 
        \begin{equation*}
        \begin{aligned}
            0 & = d^2 E = \pdv{T}{V} dV \wedge \underbrace{dS}_0 + \pdv{T}{N} dN \wedge \underbrace{dS}_0 - \pdv{p}{S} \underbrace{dS}_0 \wedge dV \\ & \qquad - \pdv{p}{N} dN \wedge dV + \pdv{\mu}{S} \underbrace{dS}_0 \wedge dN + \pdv{\mu}{V} dV \wedge dN \\ & = - \pdv{p}{N} dN \wedge dV + \pdv{\mu}{V} dV \wedge dN = - \pdv{p}{N} dN \wedge dV - \pdv{\mu}{V} dN \wedge dV ~,
        \end{aligned}
        \end{equation*}
        hence, by the linear independence of $N$ and $V$,
        \begin{equation*}
            - \pdv{p}{N} \Big \vert_{V,S} = \pdv{\mu}{V} \Big \vert_{N, S} ~.
        \end{equation*}
    \end{proof}

\section{Entropy}

\subsection{Gibbs-Duhen relation}

    The Gibbs-Duhem relation expresses the chemical potential $\mu$ in terms of the pressure $p$ and the temperature $T$ 
    \begin{equation}\label{td:gd}
        S dT - Vdp + N d\mu = 0 ~, \quad d \mu = v dp - s dT ~.
    \end{equation}
    \begin{proof}
        Computing the differential of~\eqref{td:e} 
        \begin{equation*}
            dE = T dS + S dT -p dV + \mu dN + N d\mu 
        \end{equation*}
        and comparing it with~\eqref{td:d:e}
        \begin{equation*}
            dE = \cancel{T dS} + S dT - \cancel{p dV} + - V dp + \cancel{\mu dN} + N d\mu = \cancel{T dS} - \cancel{p dV} + \cancel{\mu dN} ~,
        \end{equation*}
        we obtain 
        \begin{equation*}
            S dT - V dp + N d\mu = 0 ~.
        \end{equation*}
        which can be written as 
        \begin{equation*}
            d \mu = \frac{V}{N} dp - \frac{S}{N} dT = v dp - s dT ~.
        \end{equation*}
    \end{proof}

    Inverting~\eqref{td:d:e}, we obtained the entropy differential
    \begin{equation}\label{td:d:s}
        dS = \frac{1}{T} dE + \frac{p}{T} dV - \frac{\mu}{T} dN ~.
    \end{equation}
    Its equations of state are 
    \begin{equation}\label{td:es:s}
        \frac{1}{T} = \pdv{S}{E} \Big \vert_{V, N} ~, \quad \frac{p}{T} = \pdv{S}{V} \Big \vert_{E, N} ~, \quad - \frac{\mu}{T} = \pdv{S}{N} \Big \vert_{E, V} ~.
    \end{equation}

\section{Thermodynamic states as a manifold}

    $S$, $V$ and $N$ as independent (local) coordinates, i.e.~a chart for $\mathcal M$. In our case, it can be thought as an open subset of $\mathbb R^3$. A reversible process is a path. An irreversible process is an oriented path. Different thermodynamic systems are not connected by any process. Therefore, the manifold is path-connected and simply connected. Solving thermodynamics means find the fundamental equation 
    \begin{equation*}
        E = E(S, V, N) ~.
    \end{equation*}
    However, we could have chosen as fundamental equation 
    \begin{equation*}
        S = S(E, V, N) 
    \end{equation*}
    and a chart would have had $E$, $V$ and $N$ as coordinates. At least one of the local coordinates in any chart for $\mathcal M$ must be extensive. 
    \begin{proof}
        By the $0th$ law and~\eqref{a5}, there must exist a functional relation between intensive variables. This means that one of the three is already fixed once the other two are given and they cannot be used all three as independent coordinates.
    \end{proof}

    There are different thermodynamic potentials, which are functions of $3$ independent variables that can be used to define a different chart for $\mathcal M$. Therefore, there are different aproaches to thermodynamics.

    Thermodynamic potentials can be obtained by various kind of Legendre transforms of~\eqref{td:d:e}, which exchange the role of an extensive variable to its conjugate intensive variable as independent variable. We require that the hypothesis of the inverse function theorem are satisfied, e.g. 
    \begin{equation*}
        \pdvdu{E}{S} \Big \vert_{V,N} \neq 0 ~, \quad \pdvdu{E}{V} \Big \vert_{S,N} \neq 0 ~, \quad \pdvdu{E}{N} \Big \vert_{S, V} \neq 0 ~.
    \end{equation*}

    The thermodynamic potentials are the Helmoltz free energy $F$, the entalpy $H$, the Gibbs free energy $G$ and the granpotential $\Omega$.

\section{Helmoltz free energy} 

    The Helmoltz free energy is defined as 
    \begin{equation*}
        F = E - TS ~.
    \end{equation*}
    Its differential is 
    \begin{equation}\label{td:d:f}
        dF \leq - S dT - p dV + \mu dN ~.
    \end{equation}
    Therefore
    \begin{equation*}
        F = F(T, V, N) ~.
    \end{equation*}
    \begin{proof}
        By a Legendre transform, which means to complete a differential
        \begin{equation*}
            dE \leq T dS - p dV + \mu dN = d(TS) - S dT - p dV + \mu dN ~,
        \end{equation*}
        hence 
        \begin{equation*}
            dF = d(E - TS) \leq - S dT - p dV + \mu dN ~.
        \end{equation*}
    \end{proof}

    The equations of state are
    \begin{equation}\label{td:es:f}
        S = - \pdv{F}{T} \Big \vert_{V,N} ~, \quad p = - \pdv{F}{V} \Big \vert_{T,N} ~, \quad \mu = \pdv{F}{N} \Big \vert_{T,V} ~. 
    \end{equation}
    \begin{proof}
        At constant $V$ and $N$
        \begin{equation*}
            dF = - S dT - p \underbrace{dV}_0 + \mu \underbrace{dN}_0 = - S dT~,
        \end{equation*}
        hence 
        \begin{equation*}
            S = - \pdv{F}{T} \Big \vert_{V,N} ~.
        \end{equation*}

        At constant $T$ and $N$
        \begin{equation*}
            dF = - S \underbrace{dT}_0 - p dV + \mu \underbrace{dN}_0 = - pdV ~,
        \end{equation*}
        hence 
        \begin{equation*}
            p = - \pdv{F}{V} \Big \vert_{T,N} ~.
        \end{equation*}

        At constant $T$ and $V$
        \begin{equation*}
            dF = - S \underbrace{dT}_0 - p \underbrace{dV}_0 + \mu dN = \mu dN~,
        \end{equation*}
        hence 
        \begin{equation*}
            \mu = \pdv{F}{N} \Big \vert_{T,V} ~.
        \end{equation*}
    \end{proof}

    The integrability conditions are 
    \begin{equation}\label{td:int:f}
        \pdv{S}{V} \Big \vert_{T,N} = \pdv{p}{T} \Big \vert_{V,N} ~, \quad 
        - \pdv{S}{N} \Big \vert_{T,V} = \pdv{\mu}{T} \Big \vert_{N, V} ~, \quad 
        - \pdv{p}{N} \Big \vert_{V,T} = \pdv{\mu}{V} \Big \vert_{N, T} ~. 
    \end{equation}
    \begin{proof}
        By means of the exterior derivative, we have 
        \begin{equation*}
        \begin{aligned}
            d (dF) & = - d (S dT) - d (p dV) + d (\mu dN) \\ & = - \pdv{S}{T} \underbrace{dT \wedge dT}_0 - \pdv{S}{V} dV \wedge dT - \pdv{S}{N} dN \wedge dT - \pdv{p}{T} dT \wedge dV - \pdv{p}{V} \underbrace{dV \wedge dV}_0 \\ & \quad - \pdv{p}{N} dN \wedge dV + \pdv{\mu}{T} dT \wedge dN + \pdv{\mu}{V} dV \wedge dN + \pdv{\mu}{N} \underbrace{dN \wedge dN}_0 \\ & = - \pdv{S}{V} dV \wedge dT - \pdv{S}{N} dN \wedge dT - \pdv{p}{T} dT \wedge dV \\ & \quad - \pdv{p}{N} dN \wedge dV + \pdv{\mu}{T} dT \wedge dN + \pdv{\mu}{V} dV \wedge dN ~.
        \end{aligned}
        \end{equation*}

        At constant $N$ 
        \begin{equation*}
        \begin{aligned}
            0 & = d^2 F = - \pdv{S}{V} dV \wedge dT - \pdv{S}{N} \underbrace{dN}_0 \wedge dT - \pdv{p}{T} dT \wedge dV \\ & \quad - \pdv{p}{N} \underbrace{dN}_0 \wedge dV + \pdv{\mu}{T} dT \wedge \underbrace{dN}_0 + \pdv{\mu}{V} dV \wedge \underbrace{dN}_0 \\ & = - \pdv{S}{V} dV \wedge dT - \pdv{p}{T} dT \wedge dV = - \pdv{S}{V} dV \wedge dT + \pdv{p}{T} dV \wedge dT  ~,
        \end{aligned}
        \end{equation*}
        hence, by the linear independence of $V$ and $T$,
        \begin{equation*}
            \pdv{S}{V} \Big \vert_{T,N} = \pdv{p}{T} \Big \vert_{V,N} ~.
        \end{equation*}

        At constant $V$ 
        \begin{equation*}
        \begin{aligned}
            0 & = d^2 F = - \pdv{S}{V} \underbrace{dV}_0 \wedge dT - \pdv{S}{N} dN \wedge dT - \pdv{p}{T} dT \wedge \underbrace{dV}_0 \\ & \quad - \pdv{p}{N} dN \wedge \underbrace{dV}_0 + \pdv{\mu}{T} dT \wedge dN + \pdv{\mu}{V} \underbrace{dV}_0 \wedge dN \\ & = - \pdv{S}{N} dN \wedge dT + \pdv{\mu}{T} dT \wedge dN = - \pdv{S}{N} dN \wedge dT - \pdv{\mu}{T} dN \wedge dT~,
        \end{aligned}
        \end{equation*}
        hence, by the linear independence of $N$ and $T$,
        \begin{equation*}
            - \pdv{S}{N} \Big \vert_{T,V} = \pdv{\mu}{T} \Big \vert_{N, V} ~.
        \end{equation*}

        At constant $T$ 
        \begin{equation*}
        \begin{aligned}
            0 & = d^2 F = - \pdv{S}{V} dV \wedge \underbrace{dT}_0 - \pdv{S}{N} dN \wedge \underbrace{dT}_0 - \pdv{p}{T} \underbrace{dT}_0 \wedge dV \\ & \quad - \pdv{p}{N} dN \wedge dV + \pdv{\mu}{T} \underbrace{dT}_0 \wedge dN + \pdv{\mu}{V} dV \wedge dN \\ & = - \pdv{p}{N} dN \wedge dV + \pdv{\mu}{V} dV \wedge dN =- \pdv{p}{N} dN \wedge dV - \pdv{\mu}{V} dN \wedge dV ~,
        \end{aligned}
        \end{equation*}
        hence, by the linear independence of $N$ and $V$,
        \begin{equation*}
            - \pdv{p}{N} \Big \vert_{V,T} = \pdv{\mu}{V} \Big \vert_{N, T} ~.
        \end{equation*}
    \end{proof}

\section{Enthalpy} 

    The enthalpy is defined as 
    \begin{equation*}
        H = E + pV ~.
    \end{equation*}
    Its differential is 
    \begin{equation}\label{td:d:h}
        dH \leq TdS + Vdp + \mu dN ~.
    \end{equation}
    Therefore
    \begin{equation*}
        H = H(p, S, N) ~.
    \end{equation*}
    \begin{proof}
        By a Legendre transform, which means to complete a differential
        \begin{equation*}
            dE \leq T dS - p dV + \mu dN = TdS - d(pV) + V dp + \mu dN ~,
        \end{equation*}
        hence 
        \begin{equation*}
            dH = d(E + pV) \leq TdS + Vdp + \mu dN ~.
        \end{equation*}
    \end{proof}

    The equations of state are
    \begin{equation}\label{td:es:h}
        T = \pdv{H}{S} \Big \vert_{p,N} ~, \quad V = - \pdv{H}{p} \Big \vert_{S,N} ~, \quad \mu = \pdv{H}{N} \Big \vert_{S, p} ~. 
    \end{equation}
    \begin{proof}
        At constant $p$ and $N$
        \begin{equation*}
            dH = TdS + V\underbrace{dp}_0 + \mu \underbrace{dN}_0 ~,
        \end{equation*}
        hence 
        \begin{equation*}
            T = \pdv{H}{S} \Big \vert_{p,N} ~.
        \end{equation*}

        At constant $S$ and $N$
        \begin{equation*}
            dH = T\underbrace{dS}_0 + Vdp + \mu \underbrace{dN}_0~,
        \end{equation*}
        hence 
        \begin{equation*}
            V = - \pdv{H}{p} \Big \vert_{S,N} ~.
        \end{equation*}

        At constant $S$ and $p$
        \begin{equation*}
            dH = T\underbrace{dS}_0 + V\underbrace{dp}_0 + \mu dN ~,
        \end{equation*}
        hence 
        \begin{equation*}
            \mu = \pdv{H}{N} \Big \vert_{S, p} ~.
        \end{equation*}
    \end{proof}

    The integrability conditions are 
    \begin{equation}\label{td:int:h}
        \pdv{V}{S} \Big \vert_{p,N} = \pdv{T}{p} \Big \vert_{S,N} ~, \quad 
        \pdv{V}{N} \Big \vert_{p,S} = \pdv{\mu}{p} \Big \vert_{N, S} ~, \quad 
        \pdv{\mu}{S} \Big \vert_{N,p} = \pdv{T}{N} \Big \vert_{S, p} ~. 
    \end{equation}
    \begin{proof}
        By means of the exterior derivative, we have 
        \begin{equation*}
        \begin{aligned}
            d (dH) & = d (T dS) + d (V dp) + d (\mu dN) \\ & = \pdv{T}{S} \underbrace{dS \wedge dS}_0 + \pdv{T}{p} dp \wedge dS + \pdv{T}{N} dN \wedge dS + \pdv{V}{S} dS \wedge dp + \pdv{V}{p} \underbrace{dp \wedge dp}_0 \\ & \quad + \pdv{V}{N} dN \wedge dp + \pdv{\mu}{S} dS \wedge dN + \pdv{\mu}{p} dp \wedge dN + \pdv{\mu}{N} \underbrace{dN \wedge dN}_0 \\ & = \pdv{T}{p} dp \wedge dS + \pdv{T}{N} dN \wedge dS + \pdv{V}{S} dS \wedge dp \\ & \quad + \pdv{V}{N} dN \wedge dp + \pdv{\mu}{S} dS \wedge dN + \pdv{\mu}{p} dp \wedge dN  ~.
        \end{aligned}
        \end{equation*}

        At constant $N$ 
        \begin{equation*}
        \begin{aligned}
            0 & = d^2 H = \pdv{T}{p} dp \wedge dS + \pdv{T}{N} \underbrace{dN}_0 \wedge dS + \pdv{V}{S} dS \wedge dp \\ & \quad + \pdv{V}{N} \underbrace{dN}_0 \wedge dp + \pdv{\mu}{S} dS \wedge \underbrace{dN}_0 + \pdv{\mu}{p} dp \wedge \underbrace{dN}_0 \\ & = \pdv{T}{p} dp \wedge dS + \pdv{V}{S} dS \wedge dp = \pdv{T}{p} dp \wedge dS - \pdv{V}{S} dS \wedge dp ~,
        \end{aligned}
        \end{equation*}
        hence, by the linear independence of $S$ and $p$,
        \begin{equation*}
            \pdv{V}{S} \Big \vert_{p,N} = \pdv{T}{p} \Big \vert_{S,N} ~.
        \end{equation*}

        At constant $S$ 
        \begin{equation*}
        \begin{aligned}
            0 & = d^2 H = \pdv{T}{p} dp \wedge \underbrace{dS}_0 + \pdv{T}{N} dN \wedge \underbrace{dS}_0 + \pdv{V}{S} \underbrace{dS}_0 \wedge dp \\ & \quad + \pdv{V}{N} dN \wedge dp + \pdv{\mu}{S} \underbrace{dS}_0 \wedge dN + \pdv{\mu}{p} dp \wedge dN \\ & = \pdv{V}{N} dN \wedge dp + \pdv{\mu}{p} dp \wedge dN = \pdv{V}{N} dN \wedge dp - \pdv{\mu}{p} dN \wedge dp ~,
        \end{aligned}
        \end{equation*}
        hence, by the linear independence of $N$ and $p$,
        \begin{equation*}
            \pdv{V}{N} \Big \vert_{p,S} = \pdv{\mu}{p} \Big \vert_{N, S} ~.
        \end{equation*}

        At constant $p$ 
        \begin{equation*}
        \begin{aligned}
            0 & = d^2 H = \pdv{T}{p} \underbrace{dp}_0 \wedge dS + \pdv{T}{N} dN \wedge dS + \pdv{V}{S} dS \wedge \underbrace{dp}_0 \\ & \quad + \pdv{V}{N} dN \wedge \underbrace{dp}_0 + \pdv{\mu}{S} dS \wedge dN + \pdv{\mu}{p} \underbrace{dp}_0 \wedge dN \\ & = \pdv{T}{N} dN \wedge dS + \pdv{\mu}{S} dS \wedge dN = \pdv{T}{N} dN \wedge dS - \pdv{\mu}{S} dS \wedge dN ~,
        \end{aligned}
        \end{equation*}
        hence, by the linear independence of $S$ and $N$,
        \begin{equation*}
            \pdv{\mu}{S} \Big \vert_{N,p} = \pdv{T}{N} \Big \vert_{S, p} ~.
        \end{equation*}
    \end{proof}

\section{Gibbs free energy} 

    The Gibbs free energy is defined as 
    \begin{equation*}
        G = E - TS + pV = F + pV = H - TS ~.
    \end{equation*}
    Its differential is 
    \begin{equation} \label{td:d:g}
        dG \leq - SdT + Vdp + \mu dN ~.
    \end{equation}
    Therefore
    \begin{equation*}
        G = G(p, T, N) ~.
    \end{equation*}
    \begin{proof}
        By a Legendre transform, which means to complete a differential
        \begin{equation*}
            dE \leq T dS - p dV + \mu dN = d(TS) - S dT - d(pV) + V dp + \mu dN ~,
        \end{equation*}
        hence 
        \begin{equation*}
            dG = d(E - TS + pV) \leq - S dT + Vdp + \mu dN ~.
        \end{equation*}
    \end{proof}

    The equations of state are
    \begin{equation}\label{td:es:g}
        S = - \pdv{G}{T} \Big \vert_{p,N} ~, \quad V = \pdv{G}{p} \Big \vert_{T,N} ~, \quad \mu = \pdv{G}{N} \Big \vert_{p,T} ~. 
    \end{equation}
    \begin{proof}
        At constant $p$ and $N$
        \begin{equation*}
            dG = - S dT + V\underbrace{dp }_0 + \mu \underbrace{dN}_0  ~,
        \end{equation*}
        hence 
        \begin{equation*}
            S = - \pdv{G}{T} \Big \vert_{p,N}  ~.
        \end{equation*}

        At constant $T$ and $N$
        \begin{equation*}
            dG = - S \underbrace{dT}_0  + Vdp + \mu \underbrace{dN}_0  ~,
        \end{equation*}
        hence 
        \begin{equation*}
            V = \pdv{G}{p} \Big \vert_{T,N} ~.
        \end{equation*}

        At constant $p$ and $T$
        \begin{equation*}
            dG = - S \underbrace{dT}_0  + V\underbrace{dp}_0  + \mu dN ~,
        \end{equation*}
        hence 
        \begin{equation*}
            \mu = \pdv{G}{N} \Big \vert_{p,T} ~.
        \end{equation*}
    \end{proof}

    The integrability conditions are 
    \begin{equation}\label{td:int:g}
        - \pdv{V}{T} \Big \vert_{p,N} = \pdv{S}{p} \Big \vert_{T,N} ~, \quad 
        \pdv{V}{N} \Big \vert_{p,T} = \pdv{\mu}{p} \Big \vert_{N, T} ~, \quad 
        - \pdv{S}{N} \Big \vert_{T,p} = \pdv{\mu}{T} \Big \vert_{N, p} ~. 
    \end{equation}
    \begin{proof}
        By means of the exterior derivative, we have 
        \begin{equation*}
        \begin{aligned}
            d (dG) & = - d (S dT) + d (V dp) + d (\mu dN) \\ & = - \pdv{S}{T} \underbrace{dT \wedge dT}_0 - \pdv{S}{p} dp \wedge dT - \pdv{S}{N} dN \wedge dT + \pdv{V}{T} dT \wedge dp + \pdv{V}{p} \underbrace{dp \wedge dp}_0 \\ & \quad + \pdv{V}{N} dN \wedge dp + \pdv{\mu}{T} dT \wedge dN + \pdv{\mu}{p} dp \wedge dN + \pdv{\mu}{N} \underbrace{dN \wedge dN}_0 \\ & = - \pdv{S}{p} dp \wedge dT - \pdv{S}{N} dN \wedge dT + \pdv{V}{T} dT \wedge dp \\ & \quad + \pdv{V}{N} dN \wedge dp + \pdv{\mu}{T} dT \wedge dN + \pdv{\mu}{p} dp \wedge dN ~.
        \end{aligned}
        \end{equation*}

        At constant $N$ 
        \begin{equation*}
        \begin{aligned}
            0 & = d^2 G = - \pdv{S}{p} dp \wedge dT - \pdv{S}{N} \underbrace{dN}_0 \wedge dT + \pdv{V}{T} dT \wedge dp \\ & \quad + \pdv{V}{N} \underbrace{dN}_0 \wedge dp + \pdv{\mu}{T} dT \wedge \underbrace{dN}_0 + \pdv{\mu}{p} dp \wedge \underbrace{dN}_0 \\ & = - \pdv{S}{p} dp \wedge dT + \pdv{V}{T} dT \wedge dp = - \pdv{S}{p} dp \wedge dT - \pdv{V}{T} dp \wedge dT ~,
        \end{aligned}
        \end{equation*}
        hence, by the linear independence of $p$ and $T$,
        \begin{equation*}
            - \pdv{V}{T} \Big \vert_{p,N} = \pdv{S}{p} \Big \vert_{T,N} ~.
        \end{equation*}

        At constant $T$ 
        \begin{equation*}
        \begin{aligned}
            0 & = d^2 G = - \pdv{S}{p} dp \wedge \underbrace{dT}_0 - \pdv{S}{N} dN \wedge \underbrace{dT}_0 + \pdv{V}{T} \underbrace{dT}_0 \wedge dp \\ & \quad + \pdv{V}{N} dN \wedge dp + \pdv{\mu}{T} \underbrace{dT}_0 \wedge dN + \pdv{\mu}{p} dp \wedge dN \\ & = \pdv{V}{N} dN \wedge dp + \pdv{\mu}{p} dp \wedge dN = \pdv{V}{N} dN \wedge dp - \pdv{\mu}{p} dp \wedge dN ~,
        \end{aligned}
        \end{equation*}
        hence, by the linear independence of $p$ and $N$,
        \begin{equation*}
            \pdv{V}{N} \Big \vert_{p,T} = \pdv{\mu}{p} \Big \vert_{N, T} ~.
        \end{equation*}

        At constant $p$ 
        \begin{equation*}
        \begin{aligned}
            0 & = d^2 G = - \pdv{S}{p} \underbrace{dp}_0 \wedge dT - \pdv{S}{N} dN \wedge dT + \pdv{V}{T} dT \wedge \underbrace{dp}_0 \\ & \quad + \pdv{V}{N} dN \wedge \underbrace{dp}_0 + \pdv{\mu}{T} dT \wedge dN + \pdv{\mu}{p} \underbrace{dp}_0 \wedge dN \\ & = - \pdv{S}{N} dN \wedge dT + \pdv{\mu}{T} dT \wedge dN = - \pdv{S}{N} dN \wedge dT - \pdv{\mu}{T} dN \wedge dT ~,
        \end{aligned}
        \end{equation*}
        hence, by the linear independence of $N$ and $T$,
        \begin{equation*}
            - \pdv{S}{N} \Big \vert_{T,p} = \pdv{\mu}{T} \Big \vert_{N, p} ~.
        \end{equation*}

    \end{proof}

\section{Grand potential} 

    The grand potential is defined as 
    \begin{equation}\label{td:o}
        \Omega = E - TS - \mu N = F - \mu N ~.
    \end{equation}
    Its differential is 
    \begin{equation}\label{td:d:o}
        d\Omega \leq - SdT - pdV - N d\mu ~.
    \end{equation}
    Therefore
    \begin{equation*}
        \Omega = \Omega(T, V, \mu) ~.
    \end{equation*}
    \begin{proof}
        By a Legendre transform, which means to complete a differential
        \begin{equation*}
            dE \leq T dS - p dV + \mu dN = d(TS) - SdT - p dV + (\mu N) - N d\mu ~,
        \end{equation*}
        hence 
        \begin{equation*}
            d\Omega = d(E - TS - \mu N) \leq - SdT - p dV - N d\mu ~.
        \end{equation*}
    \end{proof}

    The equations of state are
    \begin{equation}\label{td:es:o}
        S = - \pdv{\Omega}{T} \Big \vert_{\mu,V} ~, \quad p = - \pdv{\Omega}{V} \Big \vert_{T,\mu} ~, \quad \mu = - \pdv{\Omega}{N} \Big \vert_{T,V} ~. 
    \end{equation}
    \begin{proof}
        At constant $\mu$ and $V$
        \begin{equation*}
            d\Omega = - SdT - p\underbrace{dV}_0 - N \underbrace{d\mu}_0 = - S dT ~,
        \end{equation*}
        hence 
        \begin{equation*}
            S = - \pdv{\Omega}{T} \Big \vert_{\mu,V} ~.
        \end{equation*}

        At constant $T$ and $\mu$
        \begin{equation*}
            d\Omega = - S \underbrace{dT}_0 - pdV - N \underbrace{d\mu}_ 0 = - p dV ~,
        \end{equation*}
        hence 
        \begin{equation*}
            p = - \pdv{\Omega}{V} \Big \vert_{T,\mu} ~.
        \end{equation*}

        At constant $T$ and $V$
        \begin{equation*}
            d\Omega = - S\underbrace{dT}_0 - p\underbrace{dV}_0 - N d\mu = - N d\mu~,
        \end{equation*}
        hence 
        \begin{equation*}
            \mu = - \pdv{\Omega}{N} \Big \vert_{T,V} ~.
        \end{equation*}
    \end{proof}

    The integrability conditions are 
    \begin{equation}\label{td:int:o}
        \pdv{S}{\mu} \Big \vert_{T,V} = \pdv{N}{T} \Big \vert_{\mu,V} ~, \quad 
        \pdv{S}{V} \Big \vert_{T,\mu} = \pdv{p}{T} \Big \vert_{V, \mu} ~, \quad 
        \pdv{p}{\mu} \Big \vert_{V,T} = \pdv{N}{V} \Big \vert_{\mu, T} ~. 
    \end{equation}
    \begin{proof}
        By means of the exterior derivative, we have 
        \begin{equation*}
        \begin{aligned}
            d (d\Omega) & = - d (S dT) - d (p dV) - d (N d\mu) \\ & = - \pdv{S}{T} \underbrace{dT \wedge dT}_0 - \pdv{S}{V} dV \wedge dT - \pdv{S}{\mu} d\mu \wedge dT - \pdv{p}{T} dT \wedge dV - \pdv{p}{V} \underbrace{dV \wedge dV}_0 \\ & \quad - \pdv{p}{\mu} d\mu \wedge dV - \pdv{N}{T} dT \wedge d\mu - \pdv{N}{V} dV \wedge d\mu - \pdv{N}{\mu} \underbrace{d\mu \wedge d\mu}_0 \\ & = - \pdv{S}{V} dV \wedge dT - \pdv{S}{\mu} d\mu \wedge dT - \pdv{p}{T} dT \wedge dV \\ & \quad - \pdv{p}{\mu} d\mu \wedge dV - \pdv{N}{T} dT \wedge d\mu - \pdv{N}{V} dV \wedge d\mu ~.
        \end{aligned}
        \end{equation*}

        At constant $\mu$ 
        \begin{equation*}
        \begin{aligned}
            0 & = d^2 \Omega = - \pdv{S}{V} dV \wedge dT - \pdv{S}{\mu} \underbrace{d\mu}_0 \wedge dT - \pdv{p}{T} dT \wedge dV \\ & \quad - \pdv{p}{\mu} \underbrace{d\mu}_0 \wedge dV - \pdv{N}{T} dT \wedge \underbrace{d\mu}_0 - \pdv{N}{V} dV \wedge \underbrace{d\mu}_0 \\ & = - \pdv{S}{V} dV \wedge dT - \pdv{p}{T} dT \wedge dV = - \pdv{S}{V} dV \wedge dT + \pdv{p}{T} dV \wedge dT ~,
        \end{aligned}
        \end{equation*}
        hence, by the linear independence of $V$ and $T$,
        \begin{equation*}
            \pdv{S}{V} \Big \vert_{T,\mu} = \pdv{p}{T} \Big \vert_{V, \mu} ~.
        \end{equation*}

        At constant $V$ 
        \begin{equation*}
        \begin{aligned}
            0 & = d^2 \Omega = - \pdv{S}{V} \underbrace{dV}_0 \wedge dT - \pdv{S}{\mu} d\mu \wedge dT - \pdv{p}{T} dT \wedge \underbrace{dV}_0 \\ & \quad - \pdv{p}{\mu} d\mu \wedge \underbrace{dV}_0 - \pdv{N}{T} dT \wedge d\mu - \pdv{N}{V} \underbrace{dV}_0 \wedge d\mu \\ & = - \pdv{S}{\mu} d\mu \wedge dT - \pdv{N}{T} dT \wedge d\mu = - \pdv{S}{\mu} d\mu \wedge dT + \pdv{N}{T} d\mu \wedge dT ~,
        \end{aligned}
        \end{equation*}
        hence, by the linear independence of $\mu$ and $T$,
        \begin{equation*}
            \pdv{S}{\mu} \Big \vert_{T,V} = \pdv{N}{T} \Big \vert_{\mu,V} ~.
        \end{equation*}

        At constant $T$ 
        \begin{equation*}
        \begin{aligned}
            0 & = d^2 \Omega = - \pdv{S}{V} dV \wedge \underbrace{dT}_0 - \pdv{S}{\mu} d\mu \wedge \underbrace{dT}_0 - \pdv{p}{T} \underbrace{dT}_0 \wedge dV \\ & \quad - \pdv{p}{\mu} d\mu \wedge dV - \pdv{N}{T} \underbrace{dT}_0 \wedge d\mu - \pdv{N}{V} dV \wedge d\mu \\ & = - \pdv{p}{\mu} d\mu \wedge dV - \pdv{N}{V} dV \wedge d\mu=  - \pdv{p}{\mu} d\mu \wedge dV + \pdv{N}{V} d\mu \wedge dV ~,
        \end{aligned}
        \end{equation*}
        hence, by the linear independence of $N$ and $V$,
        \begin{equation*}
            \pdv{p}{\mu} \Big \vert_{V,T} = \pdv{N}{V} \Big \vert_{\mu, T} ~.
        \end{equation*}
    \end{proof}

\subsection{Comments}

    The thermodynamic potential are not homogeneous functions since they depend on mixed extensive and intensive variables. However, they are extensive, i.e. 
    \begin{equation}\label{a6}
        F = N f(T, v) ~, \quad H = N h(p, s) ~, \quad G = N g(T, p) ~, \quad \Omega = N f \omega (T, \mu) ~. 
    \end{equation}
    Notice that the chemical potential is also the Gibbs free energy per particle
    \begin{equation}
        g(T, p) = \mu(T, p) ~.
    \end{equation}
    \begin{proof}
        In fact, using~\eqref{td:es:g} and \eqref{a6}
        \begin{equation*}
            \mu = \pdv{G}{N} = \pdv{Ng }{N} = g ~.
        \end{equation*}
    \end{proof}

    Furthermore, notice that 
    \begin{equation*}
        \Omega = - pV ~.
    \end{equation*}
    \begin{proof}
        Using~\eqref{td:e} and~\eqref{td:o}
        \begin{equation*}
            \Omega = E - TS - \mu N = \cancel{TS} - pV + \cancel{\mu N} - \cancel{TS} - \cancel{\mu N} = - pV ~.
        \end{equation*}
    \end{proof}

\section{Summary I}

    A summary of all charts and differentials is given by 
    \begin{equation*}
        E(S, V, N) ~, \quad dE = TdS - p dV + \mu dN ~,
    \end{equation*}
    \begin{equation*}
        S(E, V, N) ~, \quad dS = dE/T + p dV/T - \mu dN/T ~,
    \end{equation*}
    \begin{equation*}
        F(T, V, N) ~, \quad dF = - S dT - p dV + \mu dN ~,
    \end{equation*}
    \begin{equation*}
        H(S, p, N) ~, \quad dH = TdS + V dp + \mu dN ~,
    \end{equation*}
    \begin{equation*}
        G(T, p, N) ~, \quad d G = - SdT + V dp + \mu dN ~,
    \end{equation*}
    \begin{equation*}
        \Omega(T, V, \mu) ~, \quad d\Omega = TdS - p dV + \mu dN ~.
    \end{equation*}

    A summary of all the equations of state is given by 
    \begin{equation*}
        T = \pdv{E}{S} \Big \vert_{V,N} ~, \quad p = - \pdv{E}{V} \Big \vert_{S,N} ~, \quad \mu = \pdv{E}{N} \Big \vert_{S,V} ~,
    \end{equation*}
    \begin{equation*}
        \frac{1}{T} = \pdv{S}{E} \Big \vert_{V, N} ~, \quad \frac{p}{T} = \pdv{S}{V} \Big \vert_{E, N} ~, \quad - \frac{\mu}{T} = \pdv{S}{N} \Big \vert_{E, V} ~,
    \end{equation*}
    \begin{equation*}
        S = - \pdv{F}{T} \Big \vert_{V,N} ~, \quad p = - \pdv{F}{V} \Big \vert_{T,N} ~, \quad \mu = \pdv{F}{N} \Big \vert_{T,V} ~,
    \end{equation*}
    \begin{equation*}
        T = \pdv{H}{S} \Big \vert_{p,N} ~, \quad V = - \pdv{H}{p} \Big \vert_{S,N} ~, \quad \mu = \pdv{H}{N} \Big \vert_{S, p} ~,
    \end{equation*}
    \begin{equation*}
        S = - \pdv{G}{T} \Big \vert_{p,N} ~, \quad V = \pdv{G}{p} \Big \vert_{T,N} ~, \quad \mu = \pdv{G}{N} \Big \vert_{p,T} ~,
    \end{equation*}
    \begin{equation*}
        S = - \pdv{\Omega}{T} \Big \vert_{\mu,V} ~, \quad p = - \pdv{\Omega}{V} \Big \vert_{T,\mu} ~, \quad \mu = - \pdv{\Omega}{N} \Big \vert_{T,V} ~.
    \end{equation*}

    A summary of all integrability conditions is given by 
    \begin{equation*}
        - \pdv{T}{V} \Big \vert_{S,N} = \pdv{p}{S} \Big \vert_{V,N} ~, \quad 
        \pdv{T}{N} \Big \vert_{S,V} = \pdv{\mu}{S} \Big \vert_{N, V} ~, \quad 
        - \pdv{p}{N} \Big \vert_{V,S} = \pdv{\mu}{V} \Big \vert_{N, S} ~,
    \end{equation*}
    \begin{equation*}
        \pdv{S}{V} \Big \vert_{T,N} = \pdv{p}{T} \Big \vert_{V,N} ~, \quad 
        - \pdv{S}{N} \Big \vert_{T,V} = \pdv{\mu}{T} \Big \vert_{N, V} ~, \quad 
        - \pdv{p}{N} \Big \vert_{V,T} = \pdv{\mu}{V} \Big \vert_{N, T} ~,
    \end{equation*}
    \begin{equation*}
        \pdv{V}{S} \Big \vert_{p,N} = \pdv{T}{p} \Big \vert_{S,N} ~, \quad 
        \pdv{V}{N} \Big \vert_{p,S} = \pdv{\mu}{p} \Big \vert_{N, S} ~, \quad 
        \pdv{\mu}{S} \Big \vert_{N,p} = \pdv{T}{N} \Big \vert_{S, p} ~,
    \end{equation*}
    \begin{equation*}
        - \pdv{V}{T} \Big \vert_{p,N} = \pdv{S}{p} \Big \vert_{T,N} ~, \quad 
        \pdv{V}{N} \Big \vert_{p,T} = \pdv{\mu}{p} \Big \vert_{N, T} ~, \quad 
        - \pdv{S}{N} \Big \vert_{T,p} = \pdv{\mu}{T} \Big \vert_{N, p} ~,
    \end{equation*}
    \begin{equation*}
        \pdv{S}{\mu} \Big \vert_{T,V} = \pdv{N}{T} \Big \vert_{\mu,V} ~, \quad 
        \pdv{S}{V} \Big \vert_{T,\mu} = \pdv{p}{T} \Big \vert_{V, \mu} ~, \quad 
        \pdv{p}{\mu} \Big \vert_{V,T} = \pdv{N}{V} \Big \vert_{\mu, T} ~.
    \end{equation*}

\chapter{Maxwell's relations and stability conditions}

    In this 

\section{Maxwell's relations}

    Integrability condition can be written as jacobian determinant. 

    For the energy, they are
    \begin{equation*}
        \pdv{(p, S, V )}{(N, S, V)} = - \pdv{(\mu, S, N)}{(V, S, N)} = \pdv{\mu, S, N}{N, S, V} ~.
    \end{equation*}
    \begin{proof}
        Using the first of~\eqref{td:int:e}
        \begin{equation*}
            - \pdv{T}{V} \Big \vert_{S, N} = - \pdv{p}{S} \Big \vert_{V, N} \rightarrow \pdv{(T, N, S)}{(V, N, S)} = - \pdv{(p, N, V)}{(S, N, V)} = \pdv{(p, N, V)}{(V, N, S)} ~,
        \end{equation*} 
        hence, inverting the right-handed side
        \begin{equation*}
            1 = \pdv{(T, N, S)}{(V, N, S)} \pdv{(p, N, V)}{(V, N, S)}^{-1} = \pdv{(T, N, S)}{(V, N, S)} \pdv{(V, N, S)}{(p, N, V)} = \pdv{(T, N, S)}{(p, N, V)} ~.
        \end{equation*} 

        Using the second of~\eqref{td:int:e}
        \begin{equation*}
            \pdv{T}{N} \Big \vert_{S, V} = - \pdv{\mu}{S} \Big \vert_{N, V} \rightarrow \pdv{(T, V, S)}{(N, V, S)} = - \pdv{(\mu, V, N)}{(S, V, N)} = \pdv{(\mu, V, N)}{(N, V, S)} ~,
        \end{equation*} 
        hence, inverting the right-handed side
        \begin{equation*}
            1 = \pdv{(T, V, S)}{(N, V, S)} \pdv{(\mu, V, N)}{(N, V, S)}^{-1} = \pdv{(T, V, S)}{(N, V, S)} \pdv{(N, V, S)}{(\mu, V, N)} = \pdv{(T, V, S)}{(\mu, V, N)} ~.
        \end{equation*} 

        Using the third of~\eqref{td:int:e}
        \begin{equation*}
            \pdv{p}{N} \Big \vert_{V, S} = - \pdv{\mu}{V} \Big \vert_{N, S} \rightarrow \pdv{(p, S, V )}{(N, S, V)} = - \pdv{(\mu, S, N)}{(V, S, N)} = \pdv{(\mu, S, N)}{(N, S, V)} ~,
        \end{equation*} 
        hence, inverting the right-handed side
        \begin{equation*}
            1 = \pdv{(p, S, V )}{(N, S, V)} \pdv{(\mu, S, N)}{(N, S, V)}^{-1} = \pdv{(p, S, V )}{(N, S, V)} \pdv{(N, S, V)}{(\mu, S, N)} = \pdv{(p, S, V )}{(\mu, S, N)} ~.
        \end{equation*} 
    \end{proof}

    TO BE CONTINUED.


    Not all the Maxwell's relations are independent, but only $6$ of them 
    \begin{equation*}
        \pdv{(p, V, S)}{(\mu, N, S)} = 1 ~, \quad \pdv{(p, V, T)}{(\mu, N, T)} = 1 ~, \quad \pdv{(p, V, N)}{(T, S, N)} = 1 ~, 
    \end{equation*}
    \begin{equation*}
        \pdv{(T, S, \mu)}{(p, V, \mu)} = 1 ~, \quad \pdv{(T, S, p)}{(N, \mu, p)} = 1 ~, \quad \pdv{(T, S, V)}{(N, \mu, V)} = 1 ~.
    \end{equation*}

    The intergability conditions written in term of jacobian determinant give rise to the geometrical interpretation: the coordinate transformations, which mean that we changed into a different chart of independent variables, preserves the volume.

\section{Stability conditions}

    Every thermodynamic potential has a natural chart. In fact, the configuration of stable equilbrium can be obtained by a set of variational principle, which can be derived by fixing to constants the natural independent variables. This variations principle derive from the second law of thermodynamics, since all systems evolve spontaneously to maximise the entropy. Therefore, minima of the thermodynamic potentials correspond to stable equilibrium under boundary condition which keep constant the natural variables
    \begin{equation*}
        (T, V, N) = const \rightarrow \delta F = 0 ~, \delta^2 F > 0 ~, 
    \end{equation*}
    \begin{equation*}
        (S, p, N) = const \rightarrow \delta H = 0 ~, \delta^2 H > 0 ~, 
    \end{equation*}
    \begin{equation*}
        (T, p, N) = const \rightarrow \delta G = 0 ~, \delta^2 G > 0 ~, 
    \end{equation*}
    \begin{equation*}
        (T, V, \mu) = const \rightarrow \delta \Omega = 0 ~, \delta^2 \Omega > 0 ~.
    \end{equation*}

    Equilibrium of two subsystems requires that $T$, $p$ and $\mu$ are equal.
    \begin{proof}
        Consider two subsystems $A$ and $B$ with extensive variables $(E_A, V_A, N_A)$ and $(E_B, V_B, N_B)$. Therefore $E = E_A + E_B$, $V = V_A + V_B$ and $N = N_A + N_B$. The whole system is at fixed boundary conditions $E, V, S = const$. The entropy is additive 
        \begin{equation*}
            S = S_A + S_B = S_A(E_A, V_A, N_A) - S_B(E - E_A, V-V_A, N-N_A) ~.
        \end{equation*}
        Computing its derivative and imposing it to zero, using~\eqref{td:es:s}
        \begin{equation*}
        \begin{aligned}
            0 & = \delta S = \pdv{S_A}{E_A} \delta E_A + \pdv{S_A}{E_A} \delta E_A + \pdv{S_A}{V_A} \delta V_A + \pdv{S_A}{N_A} \delta N_A \\ & \quad + \pdv{S_B}{E_B} \underbrace{\delta (E - E_A)}_{- \delta E_A} + \pdv{S_B}{V_B} \underbrace{\delta (V - V_A)}_{- \delta V_A} + \pdv{S_B}{N_B} \underbrace{\delta (N - N_A)}_{- \delta N_A} \\ & = \pdv{S_A}{E_A} \delta E_A + \pdv{S_A}{V_A} \delta V_A + \pdv{S_A}{N_A} \delta N_A - \pdv{S_B}{E_B} \delta E_A - \pdv{S_B}{V_B}  \delta V_A - \pdv{S_B}{N_B} \delta N_A  \\ & = \delta E_A \Big ( \underbrace{\pdv{S_A}{E_A}}_{\frac{1}{T_A}} - \underbrace{\pdv{S_B}{E_B}}_{\frac{1}{T_B}} \Big) + \delta V_A \Big ( \underbrace{\pdv{S_A}{V_A}}_{\frac{p_A}{T_A}} - \underbrace{\pdv{S_B}{E_B}}_{\frac{p_B}{T_B}} \Big) + \delta N_A \Big (\underbrace{\pdv{S_A}{N_A}}_{ - \frac{\mu_A}{T_A}} - \underbrace{\pdv{S_B}{N_B}}_{- \frac{\mu_B}{T_B}} \Big) \\ & = \delta E_A \Big ( \frac{1}{T_A} - \frac{1}{T_B} \Big) + \delta V_A \Big ( \frac{p_A}{T_A} - \frac{p_B}{T_B} \Big) + \delta N_A \Big (- \frac{\mu_A}{T_A} + \frac{\mu_B}{T_B} \Big) ~,
        \end{aligned}
        \end{equation*}
        hence, by the arbitrarity of $\delta E_A$, $\delta V_A$ and $\delta N_A$,
        \begin{equation*}
            T_A = T_B ~, \quad p_A = p_B ~, \quad \mu_A = \mu_B ~.
        \end{equation*}
    \end{proof}

    At $T, p, N = const$, the stability condition is 
    \begin{equation}\label{stab}
    \begin{aligned}
        & E_{SS} = \pdv{T}{S} \Big \vert_V > 0 ~, \quad E_{VV} = - \pdv{p}{V} \Big \vert_S > 0 ~, \\ & E_{SS}E_{VV} - E^2_{SV} = - \pdv{T}{S} \Big \vert_V \pdv{p}{V} \Big \vert_S - \Big ( \pdv{p}{S} \Big \vert_V \Big )^2 = - \pdv{T}{S} \Big \vert_V \pdv{p}{V} \Big \vert_S - \Big ( \pdv{T}{V} \Big \vert_S \Big )^2 > 0 ~, 
    \end{aligned}
    \end{equation}
    \begin{proof}
        We know that $E = E(S, V, N)$. AT constant $N$, its variation is 
        \begin{equation*}
        \begin{aligned}
            \delta E & = \underbrace{\pdv{E}{S} \Big \vert_V }_T \delta S + \underbrace{\pdv{E}{V} \Big \vert_S}_{-p} \delta V  \\ & \quad + \frac{1}{2} \Big ( \underbrace{\pdvdu{E}{S} \Big \vert_V}_{E_{SS}} \delta S^2 + 2 \underbrace{\pdvd{E}{S}{V}}_{E_{SV}} \delta S \delta V + \underbrace{\pdvdu{E}{V} \Big \vert_S }_{E_{VV}} \delta V^2 \Big) \\ & = T \delta S - p \delta V + \frac{1}{2} \Big ( E_{SS} \delta S^2 + 2 E_{SV} \delta S \delta V + E_{VV} \delta V^2 \Big) ~.
        \end{aligned}
        \end{equation*}

        The first derivative terms vanishes, since 
        \begin{equation*}
        \begin{aligned}
            \delta G &= \delta E - T \delta S + p \delta V \\ & = \cancel{T \delta S} - \cancel{p \delta V} + \frac{1}{2} \Big ( E_{SS} \delta S^2 + 2 E_{SV} \delta S \delta V + E_{VV} \delta V^2 \Big) - \cancel{T \delta S} + \cancel{p \delta V} \\ & = \frac{1}{2} \Big ( E_{SS} \delta S^2 + 2 E_{SV} \delta S \delta V + E_{VV} \delta V^2 \Big) ~.
        \end{aligned}
        \end{equation*}

        The condition to be a minimum is that 
        \begin{equation*}
            E_{SS} > 0 ~, \quad E_{VV} > 0 ~, \quad E_{SS} E_{VV} - E_{SV}^2 > 0 ~.
        \end{equation*}

        Respectively, they become 
        \begin{equation*}
            E_{SS} = \pdv{}{S} \underbrace{\pdv{E}{S}}_{T} = \pdv{T}{S} > 0 ~,
        \end{equation*}
        \begin{equation*}
            E_{VV} = \pdv{}{V} \underbrace{\pdv{E}{V}}_{-p} = - \pdv{p}{V} > 0 ~,
        \end{equation*}
        \begin{equation*}
            E_{SS}E_{VV} - E^2_{SV} = - \pdv{T}{S} \Big \vert_V \pdv{p}{V} \Big \vert_S - \Big ( \pdv{p}{S} \Big \vert_V \Big )^2 = - \pdv{T}{S} \Big \vert_V \pdv{p}{V} \Big \vert_S - \Big ( \pdv{T}{V} \Big \vert_S \Big )^2 > 0 ~.
        \end{equation*}
    \end{proof}

    We define the stability conditions in terms of the specific heat 
    \begin{equation*}
        C_V = T \pdv{S}{T} \Big \vert_{V} > 0 ~,
    \end{equation*}
    the adiabatic compressibility
    \begin{equation*}
        \chi_S = - \frac{1}{V} \pdv{V}{p} \Big \vert_S > 0
    \end{equation*}
    and the isothermal compressibility 
    \begin{equation*}
        \chi_T = - \frac{1}{V} \pdv{V}{p} \Big \vert_T > 0 ~.
    \end{equation*}
    \begin{proof}
        For the first, using~\eqref{stab} and $T > 0$
        \begin{equation*}
            C_V = T \pdv{S}{T} \Big \vert_{V} > 0 ~.
        \end{equation*}

        For the second, using~\eqref{stab} and $V > 0$
        \begin{equation*}
            \chi_S = - \frac{1}{V} \pdv{V}{p} \Big \vert_S > 0 ~.
        \end{equation*}
        
        For the third, using~\eqref{stab} and~\eqref{intcondgib}
        \begin{equation*}
        \begin{aligned}
            0 & < \pdv{T}{V} \Big \vert_S \pdv{T}{V} \Big \vert_S + \pdv{T}{S} \Big \vert_V \pdv{p}{V} \Big \vert_S \\ &  - \pdv{T}{V} \Big \vert_S \pdv{p}{S} \Big \vert_V + \pdv{T}{S} \Big \vert_V \pdv{p}{V} \Big \vert_S \\ & = \pdv{(T,p)}{(S,V)} \\ & = \pdv{(T,p)}{(S,V)} = \pdv{(T,p)}{(T,V)} \pdv{(T,V)}{(S,V)} \\ & = \pdv{p}{V} \Big \vert_T \pdv{T}{S} \Big \vert_V \\ & = \frac{T}{C_V} \pdv{p}{V} \Big \vert_T ~,
        \end{aligned}
        \end{equation*}
        hence, by $T>0$, $C_V>0$ and $V>0$, 
        \begin{equation*}
            \chi_T = - \frac{1}{V} \pdv{V}{p} \Big \vert_T > 0 ~.
        \end{equation*}
    \end{proof}

    Consequently to stability, $F$ is a concave of $T$ and covex of $V$, whereas $G$ is concave of both $T$ and $p$. 
    \begin{proof}
        For the concavity of $F$ of $T$
        \begin{equation*}
            C_V = T \pdv{S}{T} \Big \vert_{V} = - T \pdvdu{F}{T} \Big \vert_V > 0 ~,
        \end{equation*}
        hence 
        \begin{equation*}
            \pdvdu{F}{T} \Big \vert_V < 0 ~.
        \end{equation*}

        For the convexity of $F$ of $V$
        \begin{equation*}
            \chi_T = - \frac{1}{V} \pdv{V}{p} \Big \vert_T = \Big ( V \pdvdu{F}{V} \Big \vert_T \Big)^{-1} > 0 ~,
        \end{equation*}
        hence 
        \begin{equation*}
            \pdvdu{F}{V} \Big \vert_T > 0 ~.
        \end{equation*}

        For the concavity of $G$ of $T$
        \begin{equation*}
            C_P = T \pdv{S}{T} \Big \vert_{P} = - T \pdvdu{G}{T} \Big \vert_p > 0 ~,
        \end{equation*}
        hence 
        \begin{equation*}
            \pdvdu{G}{T} \Big \vert_p < 0 ~.
        \end{equation*}

        For the concavity of $G$ of $p$
        \begin{equation*}
            \chi_T = - \frac{1}{V} \pdv{V}{p} \Big \vert_T = - \frac{1}{V} \pdvdu{G}{p} \Big \vert_T > 0 ~,
        \end{equation*}
        hence 
        \begin{equation*}
            \pdvdu{G}{p} \Big \vert_T < 0 ~.
        \end{equation*}
    \end{proof}

    Furthermore, the second law of thermodynamics can be expressed, in order to maximise the entropy, by imposing that first derivatives vanish and the hessian, i.e. the matrix with its second derivatives, must be negative defined. Therefore, it must be (locally) concave in $E$, $V$ and $N$.

    When cease to work at constant $N$, the stability condition is 
    \begin{equation*}
        \pdv{N}{\mu} \Big \vert_{V, T} = \frac{N^2}{V} \chi_T > 0 ~.
    \end{equation*}
    \begin{proof}
        In fact 
        \begin{equation*}
        \begin{aligned}
            \pdv{N}{\mu} \Big \vert_{V, T} & = \pdv{(N, V, T)}{(\mu, V, T)} \\ & = \pdv{(N, V, T)}{(N, p, T)} \pdv{(N, p, T)}{(p, V, T)} 1 \pdv{(p, V, T)}{(\mu, V, T)} \\ & = \pdv{(N, V, T)}{(N, p, T)} \pdv{(N, p, T)}{(p, V, T)} \pdv{(p, V, T)}{(\mu, N, T)} \pdv{(p, V, T)}{(\mu, V, T)} \\ & = \pdv{(N, V, T)}{(N, p, T)} \pdv{(N, p, T)}{(\mu, N, T)} \pdv{(p, V, T)}{(\mu, V, T)} \\ & = - \pdv{V}{p} \Big \vert_{N, T}  \pdv{p}{\mu} \Big \vert_{V, T}  \pdv{p}{\mu} \Big \vert_{N, T} 
        \end{aligned}
        \end{equation*}

        Now we use~\eqref{td:gd}
        \begin{equation*}
            \pdv{p}{\mu} \Big \vert_{V, T} = \pdv{p}{\mu} \Big \vert_{N, T} = \Big ( \pdv{\mu}{p} \Big \vert_T \Big) = \frac{N}{V} ~,
        \end{equation*}
        hence 
        \begin{equation*}
        \pdv{N}{\mu} \Big \vert_{V, T} = - \frac{N^2}{V^2} \pdv{V}{p} \Big \vert_{N,T} = \frac{N^2}{V} \chi_T > 0 ~.
        \end{equation*}
    \end{proof}

    Moreover, we have the relation 
    \begin{equation*}
        \chi_T (C_P - C_V) = T V \alpha_p^2 ~,
    \end{equation*}
    which implies that 
    \begin{equation*}
        C_P > C_V \iff \chi_T > \chi_S ~.
    \end{equation*}

    \begin{proof}
        We start from
        \begin{equation*}
            C_V = T \pdv{S}{T} \Big \vert_V = \pdv{E}{T} \Big \vert_V ~,
        \end{equation*}
        \begin{equation*}
            C_p = T \pdv{S}{T} \Big \vert_p = \pdv{E}{T} \Big \vert_p + p \pdv{V}{T} \Big \vert_p  ~,
        \end{equation*}
        which imply that 
        \begin{equation*}
            T dS = C_V dT + ( \pdv{E}{V} \Big \vert_{T} + p) dV = C_V dT + T \pdv{p}{T} \Big \vert_V dV ~,
        \end{equation*}
        \begin{equation*}
            T dS = C_p dT + ( \pdv{E}{p} \Big \vert_T + \pdv{V}{p} \Big \vert_T ) dp = C_p dT - T \pdv{V}{T} \Big \vert_p dp ~.
        \end{equation*}

        Comparing them 
        \begin{equation*}
            (C_p - C_V) dT = T (\pdv{V}{T} \Big \vert_p dp + \pdv{p}{T} \Big \vert_V dV) ~,
        \end{equation*}
        \begin{equation*}
            (C_p - C_V) = T \pdv{V}{T} \Big \vert_p \pdv{p}{T} \Big \vert_V ~.
        \end{equation*}

        We use 
        \begin{equation*}
            \pdv{p}{T} \Big \vert_V = \pdv{(p, V)}{T, V} = \pdv{(p,V)}{(p, T)} \pdv{(p, T)}{(T, V)} = - \pdv{V}{T} \Big \vert_p \pdv{p}{V} \Big \vert_T ~,
        \end{equation*}
        hence 
        \begin{equation*}
            C_p - C_V = \frac{T}{V \chi_T} \Big ( \pdv{V}{T} \Big \vert_p \Big)^2 ~,
        \end{equation*}
        or, defining the thermal expansion coefficient
        \begin{equation}
            \alpha_p = \frac{1}{V} \pdv{V}{T} \Big \vert_p~,
        \end{equation}
        we have 
        \begin{equation*}
            \chi_T (C_P - C_V) = T V \alpha_p^2 ~,
        \end{equation*}

        Finally, we obtain 
        \begin{equation*}
            \frac{C_p}{C_V} = \frac{\chi_T}{\chi_S} ~.
        \end{equation*}
    \end{proof}

\section{Statistical mechanics}
    
    It is important to say that this is all thermodynamics can tell us, thus in order to find the explicit expression of $E$, we must go into statistical mechanics.

\part{Classical statistical mechanics}

\chapter{Classical mechanics}

    A state constitued by a system of N particles is described by a point in a $2N$-dimensional manifold $\mathcal M^N$, called the phase space, which is the Cartesian product of N single particle manifolds 
    \begin{equation*}
        \{(q^i, ~p_i)\} \in \mathcal M^N
    \end{equation*}
    where $i = 1, \ldots N$.

    An observable is a smooth real function 
    \begin{equation*}
        f ~\colon~ \mathcal M^N \rightarrow \mathbb R
    \end{equation*}

    and its measurement in a fixed point $(\tilde q^i, ~\tilde p_i)$ is its value in it 
    \begin{equation*}
        f = f(\tilde q^i, ~\tilde p_i)
    \end{equation*}

    The time evolution is governed by a real function, called the hamiltonian $H(q^i, ~p_i, ~t)$, which is the solution of the equations of motion, called the Hamilton's equations
    \begin{equation*}
        \dot q^i = \pdv{H}{p_i} \quad \dot p_i = - \pdv{H}{q^i}
    \end{equation*}
    
    \begin{theorem}[Conservation of energy] \label{consen}
        If the hamiltonian does not depend explicitly on time, it can be intepreted physically as the energy of the system, which is constants
        \begin{equation*}
            H(q^i(t), ~p_i(t)) = H(q^i(0), ~p_i(0)) = E = const
        \end{equation*}
    \end{theorem}

    Since they are deterministic, once the initial conditions are given, the trajectory in phase space is completely determined

\section{Probability density distribution}

    A macrostate is defined by setting the macroscopic thermodynamical quantities. A microstate is the knowledge of the phase space behaviour $(q^i, ~p_i)$. 

    In general, there are more microstates associated to the same macrostates, raising the concept of ensemble: fixing a macrostate, it is created a large number of copies of the same physical system but with different microstates. It can be studied with the introduction of a probability density distribution 
    \begin{equation*}
        \rho(q_i(t), ~p_i(t),~t)
    \end{equation*}
    such that it satisfies the following properties
    \begin{enumerate}
        \item positivity, i.e.
        \begin{equation*}
            \rho(q_i, ~p_i, ~t) \geq 0
        \end{equation*}
        \item normalisation, i.e.
        \begin{equation*}
            \int_{\mathcal M^n} \underbrace{\prod_{i=1}^N d^d q^i d^d p^i}_{d\Gamma} ~ \rho(q_i, ~p_i, ~t) = \int_{\mathcal M^n} d\Gamma ~ \rho(q_i, ~p_i, ~t) = 1
        \end{equation*}
    \end{enumerate}

    To solve the dimensional problem of the volume element $d\Gamma$, which must be adimensional but it has the dimension of an action to the power of $d$, it can be introduced the adimensional volume element 
    \begin{equation*}
        d \Omega = \frac{d\Gamma}{h^d} = \frac{\prod_{i=1}^N d^d q^i d^d p^i}{h^d}
    \end{equation*}
    where the scale factor $h$ has the dimension of an action.

    The probability to find the system in a finite portion of the phase space $\mathcal U \subset \mathcal M^N$ is 
    \begin{equation*}
        \int_{\mathcal U} d\Gamma ~ \rho(q_i, ~p_i, ~t) 
    \end{equation*}

\section{Liouville's theorem}

    The flow of a system of particles keeps trasf of all their motions. See Figure.

    \begin{theorem}[Liouville]
        The volume through the flow generated by the hamilton's equations is constant. See Figure. Mathematically
        \begin{equation*}
            vol \Omega(t=0) = vol \Omega(t) ~\Rightarrow~ \dv{\rho}{t} = \pdv{\rho}{t} + \poi{\rho}{H} = 0
        \end{equation*}
    \end{theorem}

    \begin{proof}
        Maybe in the future.
    \end{proof}

    The physical intepretation of this theorem is that particles do not appear nor disappear due to conservation of charge, mass, etc...

    For stationary systemas, i.e. when $\pdv{\rho}{t} = 0$, the necessary condition for equilibrium is $\poi{\rho}{H} = 0$, which is satisfied only if 
    \begin{equation*}
        \rho = const 
    \end{equation*}
    like in the microcanonical ensemble, and 
    \begin{equation*}
        \rho = \rho(H)
    \end{equation*}
    like in the canonical or the grancanonical ensembles.

    \begin{proof}
        Maybe in the future.
    \end{proof}

    The average value of an observable is weighted by the probability density distribution
    \begin{equation}\label{clav}
        \av{f} = \int_{\mathcal M^N} d\Gamma ~ \rho(q^i, ~p_i) f(q^i, ~p_i)
    \end{equation}
    and the standard deviation is 
    \begin{equation*}
        (\Delta f)^2 = \av{f^2} - \av{f}^2
    \end{equation*}

\section{Time-independent Hamiltonian}

    Consider a time-independent hamiltonian. Since the energy is constant for the theorem~\ref{consen}.

    \begin{equation}\label{norm}
        \int_{\mathcal M^N} \rho = 1
    \end{equation}
    \begin{equation}\label{T}
        \pdv{S}{E} = \frac{1}{T}
    \end{equation}
    \begin{equation}\label{F}
        \pdv{F}{T} = - S
    \end{equation}
    \begin{equation}\label{OM}
        \Omega = - pV = E - TS - \mu N
    \end{equation}

\chapter{Microcanonical ensemble}

    A microcanonical ensemble is a system which is isolated from the environment, i.e. it cannot exchange neither energy nor matter, so $E$, $N$ and $V$ are fixed. Since energy is conserved and the hamiltonian is time-independent, the trajectory of motion is restricted on the surface $S_E$ and not on all the phase space.
    
    Assume an a-priory uniform probability 
    \begin{equation*}
        \rho_{mc}(q^i, ~p_i) = C \delta (\mathcal H(q^i, p_i) - E)
    \end{equation*}
    where $C$ is a normalisation constant, which can be evaluated by~\eqref{norm}
    \begin{equation*}
        1 = \int_{\mathcal M^N} d\Omega \rho_{mc} = \int_{\mathcal M^N} d\Omega C \delta(\mathcal H - E) = C \int_{\mathcal M^N} d\Omega \delta(\mathcal H - E) = C \omega(E)
    \end{equation*}

    Hence
    \begin{equation*}
        \rho_{mc}(q^i, ~p_i) = \frac{1}{\omega(E)} \delta (\mathcal H(q^i, p_i) - E)
    \end{equation*}

    Consider a displacement on an infinitesimal displacement of energy $\Delta E \ll 1$, then 
    \begin{equation*}
        \Gamma (E) = \integ{E}{E+dE}{E'} \omega(E') \simeq \omega(E) \Delta E
    \end{equation*}
    and the distribution is 
    \begin{equation*}
        \rho_{mc}(q^i, p_i) = \begin{cases}
            \frac{1}{\Gamma(E)} & \mathcal H \in [E, E + \Delta E] \\
            0 & otherwise
        \end{cases}
    \end{equation*}

    Let $f(q^i, p_i)$ be an observable, then its microcanonical average is 
    \begin{equation}\label{obs}
        \avp{f(q^i, p_i)}{mc} = \int_{\mathcal M} d\Omega ~ \rho_{mc} f = \int_{\mathcal M} d\Omega ~ \frac{1}{\omega(E)} \delta (\mathcal H - E) f = \frac{1}{\omega(E)} \int_{S_E} dS_E ~ f = \avp{f}{E} 
    \end{equation}

\section{Thermodynamics potentials}

    The microcanonical entropy $S_{mc}$ is defined by 
    \begin{equation}\label{entropymc}
        S_{mc} (E, V, N) = k_B \ln \omega(E)
    \end{equation}

    The logarithm is justified by the fact that the volume of a N-particle phase space is $(W_1)^N$, where $W_1$ is the volume of a single particle phase space. According to the properties of the logarithm, entropy becomes extensive.

    In the thermodynamic limit, the following equations hold 
    \begin{equation*}
        s_{mc} = \lim_{td} \frac{S_{mc}}{N} = k_B \lim_{td} \frac{\log \omega(E)}{N} = \underbrace{k_B \lim_{td} \frac{\log \Sigma(E)}{N}}_{\mathcal H \in [0, E]} = \underbrace{k_B \lim_{td} \frac{\log \Gamma(E)}{N}}_{\mathcal H \in [E, E + \Delta E]}
    \end{equation*}

    Entropy is additive, so given two sistems $1$ and $2$
    \begin{equation*}
        s_{mc}^{tot} = s_{mc}^{(1)} + s_{mc}^{(2)}
    \end{equation*}

    \begin{proof}
        Consider two isolated systems in contact at equilibrium with the same temperature $T = T_1 = T_2$. The total energy is $E = E_1 + E_2 + E_{surface}$ but, in the thermodynamic limit, the energy exchanged by the surface is a subleading term ($E_1$ and $E_2$ go as $L^3$ whereas $E_{surface}$ goes as $L^2$) and can be neglected. The energy density is 
        \begin{equation*}
        \begin{aligned}
            \omega(E) & = \int_{\mathcal M^N} d\Gamma_1 d\Gamma_2 \delta(\mathcal H - E) \\ & = \int dE_1 \int dS_{E_1} \int dE_2 \int dS_{E_2} \delta (E - E_1 - E_2) \\ & = \int dE_1 \int dE_2 \omega_1(E_1) \omega_2(E_2) \delta (E - E_1 - E_2) \\ & = \integ{0}{E}{E_1} \omega_1(E_1) \omega_2(E_2 = E - E_1)
        \end{aligned}
        \end{equation*}
        Since the integrand is a positive function with a maximum in $_1 \in [0, E]$
        \begin{equation}\label{proof1}
        \begin{aligned}
            \integ{0}{E}{E_1} \omega_1(E_1) \omega_2(E_2 = E - E_1) & \leq \omega_1(E^*_1) \omega_2(E^*_2 = E - E^*_1) \integ{0}{E}{E_1} \\ & = \omega_1(E^*_1) \omega_2(E^*_2 = E - E^*_1) E
        \end{aligned}
        \end{equation}

        On the other hand, it is always possible to find a value for $\Delta E$ in order to have 
        \begin{equation} \label{proof2}
            \Delta E \omega_1(E^*_1) \omega_2(E^*_2) \leq \omega(E)
        \end{equation}

        Putting together~\eqref{proof1} and~\eqref{proof2}
        \begin{equation*}
            \Delta E \omega_1(E^*_1) \omega_2(E^*_2) \leq \omega(E) \leq \omega_1(E^*_1) \omega_2(E^*_2) E
        \end{equation*}
        \begin{equation*}
            \omega_1(E^*_1) \Delta E \omega_2(E^*_2) \Delta E \leq \omega(E) \Delta E \leq \frac{E}{\Delta E} \omega_1(E^*_1) \Delta E \omega_2(E^*_2) \Delta E
        \end{equation*}
        \begin{equation*}
            \Gamma_1(E^*_1) \Gamma(E^*_2) \leq \Gamma(E) \leq \frac{E}{\Delta E}\Gamma(E^*_1) \Gamma(E^*_2)
        \end{equation*}
        Since the logarithm is a monotomic function
        \begin{equation*}
            \log \Big ( \Gamma_1(E^*_1) \Gamma(E^*_2) \Big ) \leq \log \Gamma(E) \leq \log \Big ( \frac{E}{\Delta E}\Gamma(E^*_1) \Gamma(E^*_2) \Big )
        \end{equation*}
        \begin{equation*}
            k_B \log \Big ( \Gamma_1(E^*_1) \Gamma(E^*_2) \Big ) \leq k_B \log \Gamma(E) \leq k_B \log \Big ( \frac{E}{\Delta E}\Gamma(E^*_1) \Gamma(E^*_2) \Big )
        \end{equation*}
        \begin{equation*}
            k_B \log \Gamma_1(E^*_1) + k_B \log \Gamma(E^*_2) \leq k_B \log \Gamma(E) \leq k_B \log \frac{E}{\Delta E} + k_B \log \Gamma(E^*_1) + k_B \log \Gamma(E^*_2)
        \end{equation*}
        \begin{equation*}
            \frac{k_B \log \Gamma_1(E^*_1) + k_B \log \Gamma(E^*_2)}{N} \leq \frac{k_B \log \Gamma(E)}{N} \leq \frac{k_B \log \frac{E}{\Delta E} + k_B \log \Gamma(E^*_1) + k_B \log \Gamma(E^*_2)}{N}
        \end{equation*}

        In the thermodynamic limit, the last term vanishes, since $\lim_{td} \frac{1}{N} \log \frac{N}{\Delta N} = 0$. Hence 
        \begin{equation*}
            s_{mc}(E) = s_{mc}^{(1)} + s_{mc}^{(2)}
        \end{equation*}
 
    \end{proof}

    The last result tells also that at equilibrium entropy is maximum.

    In the thermodynamic limit, microcanonical entropy coincides with the thermodynamical one 
    \begin{equation*}
        s_{mc} = s_{td}
    \end{equation*}

    \begin{proof}
        Since entropy is maximum at equilibrium, also $\Gamma_1(E_1) \Gamma_2(E_2)$ is so and
        \begin{equation*}
        \begin{aligned}
            0 & = \delta (\Gamma_1(E^*_1) \Gamma_2(E^*_2 = E - E^*_1)) \\ & = \delta \Gamma_1(E^*_1) \Gamma_2 (E^*_2) + \Gamma_1(E^*_1) \delta \Gamma_2 (E^*_2) \\ & = \pdv{\Gamma_1}{E_1} \Big\vert_{E^*_1} \delta E_1 \Gamma_2 (E^*_2) + \Gamma_1(E^*_1) \pdv{\Gamma_2}{E_2} \Big\vert_{E^*_2} \delta E_2 
        \end{aligned}
        \end{equation*}

        Since $E = const$, $0 = \delta E = \delta E_1 + \delta E_2$, $\delta E_2 = \delta E_1$ and
        \begin{equation*}
            0 = \pdv{\Gamma_1}{E_1} \Big\vert_{E^*_1} \delta E_1 \Gamma_2 (E^*_2) - \Gamma_1(E^*_1) \pdv{\Gamma_2}{E_2} \Big\vert_{E^*_2} \delta E_1 
        \end{equation*}
        \begin{equation*}
            0 = \pdv{\Gamma_1}{E_1} \Big\vert_{E^*_1} \Gamma_2 (E^*_2) - \Gamma_1(E^*_1) \pdv{\Gamma_2}{E_2} \Big\vert_{E^*_2} 
        \end{equation*}
        \begin{equation*}
            \pdv{\Gamma_1}{E_1} \Big\vert_{E^*_1} \Gamma_2 (E^*_2) = \Gamma_1(E^*_1) \pdv{\Gamma_2}{E_2} \Big\vert_{E^*_2} 
        \end{equation*}
        \begin{equation*}
            \frac{1}{\Gamma_1 (E^*_1)} \pdv{\Gamma_1}{E_1} \Big\vert_{E^*_1} = \frac{1}{\Gamma_2 (E^*_2)} \pdv{\Gamma_2}{E_2} \Big\vert_{E^*_2} 
        \end{equation*}
        \begin{equation*}
            \pdv{\log \Gamma_1}{E_1} \Big\vert_{E^*_1} = \pdv{\log \Gamma_2}{E_2} \Big\vert_{E^*_2} 
        \end{equation*}

        Using the thermodynamical relation~\eqref{T}
        \begin{equation*}
            S_{mc} (E) = S_{td} (E) \times const
        \end{equation*}
        where the constant can be chosen in order to have $k_B$ in the same unit.
    \end{proof}

    The universal Boltzmann's formula is 
    \begin{equation}\label{unboltz}
        s_{mc} = s_{td} = k_B \log \omega(E) = - k_B \avp{\log \rho_{mc}}{mc} ~.
    \end{equation}

    \begin{proof}
        
    Using~\eqref{obs}, 
    \begin{equation*}
    \begin{aligned}
        \avp{\log \rho_{mc}}{mc} & = \int d\Gamma \rho_{mc} \log \rho_{mc} \\ & = \int d\Gamma \frac{1}{\omega(E)} \delta (\mathcal H - E) \log \Big ( \frac{1}{\omega(E)} \delta (\mathcal H - E) \Big) \\ & = \int dS_E \frac{1}{\omega(E)} \log \frac{1}{\omega(E)} \\ & = - \frac{1}{\omega(E)} \log \omega(E) \int dS_E \\ & = - \log \omega (E)
    \end{aligned}
    \end{equation*}
    \end{proof}

\chapter{Canonical ensemble}

    A canonical ensemble is a system which is immersed in a bigger environment or reservoir, which can exchange energy but not matter, so $T$, $N$ and $V$ are fixed. Globally, energy is conserved, since the universe composed by the union of the system and the environment can be considered as a microcanonical ensemble. 

    The canonical probability density distribution is 
    \begin{equation*}
        \rho_c (q^i, p_i) = \frac{1}{Z_N} \exp (-\beta \mathcal H(q^i, p_i))
    \end{equation*}
    where $\beta$ is 
    \begin{equation*}
        \beta = \frac{1}{k_B T}
    \end{equation*}
    and $Z_N$ is the partition function 
    \begin{equation}
        Z_N[V, T] = \int_{\mathcal M^N} d\Omega ~\exp (-\beta \mathcal H(q^i, p_i))
    \end{equation}
    which depends on the temperature through $\beta$ and volume and temperature due to the integration domain $\mathcal M^N = V \otimes \mathbb R^d$.

    Notice that the probability is a function of the hamiltonian, like Liouville's theorem said.

    \begin{proof}
        Consider the universe as a microcanonical ensemble. Its probability density distribution is 
        \begin{equation*}
            \rho_{mc} (q_i^{(1)}, p_i^{(1)}, q_i^{(2)}, p_i^{(2)}) = \frac{1}{\omega(E)} \delta (\mathcal H (q_i^{(1)}, p_i^{(1)}, q_i^{(2)}, p_i^{(2)}) - E)
        \end{equation*}
        where the total hamiltonian is 
        \begin{equation*}
            \mathcal H (q_i^{(1)}, p_i^{(1)}, q_i^{(2)}, p_i^{(2)}) = \mathcal H_1 (q_i^{(1)}, p_i^{(1)}) + \mathcal H_2 (q_i^{(2)}, p_i^{(2)})
        \end{equation*}

        Integrating it to all the possible state in the environment
        \begin{equation*}
            \rho^{(1)} = \int d\Omega_2  \rho_{mc} = \int d\Omega_2 \frac{1}{\omega(E)} \delta(\mathcal H - E) = \frac{1}{\omega(E)} \int dS_{E_2} = \frac{1}{\omega(E)} \omega(E_2 = E - E_1)
        \end{equation*}
        and the corresponding entropy is 
        \begin{equation*}
            S_2 (E_2) = k_B \ln \omega_2 (E_2)
        \end{equation*}

        Applying small variation $\delta E_1$ to $E_1$ to preserve equilibrium, the entropy trasforms, using~\eqref{T}
        \begin{equation*}
            k_B \ln \omega_2 (E_2) = S_{mc}(E) - E_1 \pdv{S_{mc}}{E} \Big \vert_{E_2} = S_{mc}(E) - E_1 \frac{1}{T} 
        \end{equation*}
        \begin{equation*}
            \ln \omega_2 (E_2) = \frac{S_{mc}(E)}{k_B} - E_1 \frac{1}{k_B T} 
        \end{equation*}
        \begin{equation*}
            \omega_2 (E_2) = \exp (\frac{S_{mc}(E)}{k_B} - E_1 \frac{1}{k_B T}) = \exp (\frac{S_{mc}(E)}{k_B}) \exp (- \frac{E_1}{k_B T}) 
        \end{equation*}

        Putting together, dropping the indices
        \begin{equation}
            \rho_c = \frac{\omega(2)(E_2)}{\omega(E)} = \frac{1}{\omega(E)} \exp (\frac{S_{mc}(E)}{k_B}) \exp (- \frac{E_1}{k_B T}) = C \exp (- \frac{E_1}{k_B T})
        \end{equation}
        where $C$ is a normalisation constant, which can be evaluated by~\eqref{norm}
        \begin{equation*}
            1 = \int_{\mathcal M^N} d\Omega \rho = \int_{\mathcal M^N} d\Omega C \exp (- \frac{E_1}{k_B T}) = C \int_{\mathcal M^N} d\Omega \exp (- \frac{E_1}{k_B T}) 
        \end{equation*}
    \end{proof}

    The partition function can also be written as 
    \begin{equation*}
        Z_N[T, V] = \integ{0}{\infty}{E} \omega(E) \exp (-\beta E)
    \end{equation*}

    \begin{proof}
        Foliating the phase space in energy hyper-surfaces 
        \begin{equation*}
            Z_N = \int_{\mathcal M^N} d\Omega \exp (- \beta \mathcal H) = \integ{0}{\infty}{E} \int dS_E ~ \exp (-\beta \mathcal H) = \integ{0}{\infty}{E} \omega(E) \exp (-\beta E)
        \end{equation*}
    \end{proof}

    Taking also in consideration indistinguishable particles, the partition function 
    \begin{equation*}
        Z_N = \int \frac{\prod_{i=1}^N d^d q^i d^d p^i}{h^{dN} \zeta_N} \exp (- \beta \mathcal H) 
    \end{equation*}
    where $\zeta_N$ is 
    \begin{equation*}
        \zeta_N = \begin{cases}
            1 & \textnormal{distinguishable} \\
            N! & \textnormal{indistinguishable}
        \end{cases}
    \end{equation*}

    The partition function of two systems is the multiplication of the single system ones
    \begin{equation}
        Z_N = Z_{N_1} Z_{N_2}
    \end{equation}

    \begin{proof}
        Since $\mathcal H = \mathcal H_1 + \mathcal H_2$, 
        \begin{equation*}
        \begin{aligned}
        \end{aligned}
        \end{equation*}
    \end{proof}

    If the hamiltonian is the sum of $N$ identical ones, like $N$ non-interacting particles
    \begin{equation*}
        \mathcal H = \sum_{i = 1}^{N} \mathcal H_i
    \end{equation*} 
    the partition function becomes 
    \begin{equation*}
        Z_N = \frac{(Z_1)^N}{\zeta_N}
    \end{equation*}

    \begin{proof}
        Denominating $Z_1$ the single-particle partition function
        \begin{equation*}
        \begin{aligned}
            Z_N & = \int_{\mathcal M^N = \mathcal M^{(1)} \otimes \ldots \otimes \mathcal M^{(1)}} \prod_{i=1}^N \frac{d^d q^i d^d p^i}{h^{dN} \zeta_N} \exp (-\beta \mathcal H) \\ & = \int_{\mathcal M^N = \mathcal M^{(1)} \otimes \ldots \otimes \mathcal M^{(1)}} \prod_{i=1}^N \frac{d^d q^i d^d p^i}{h^{dN} \zeta_N} \exp (-\beta \sum_{i = 1}^{N} \mathcal H_i) \\ & = \int_{\mathcal M^N = \mathcal M^{(1)} \otimes \ldots \otimes \mathcal M^{(1)}} \prod_{i=1}^N \frac{d^d q^i d^d p^i}{h^{dN} \zeta_N} \prod_{i=1}^{N}\exp (-\beta \mathcal H_i) \\ & = \int_{\mathcal M^N = \mathcal M^{(1)} \otimes \ldots \otimes \mathcal M^{(1)}} \prod_{i=1}^N \frac{d^d q^i d^d p^i}{h^{dN} \zeta_N} \exp (-\beta \mathcal H_i) \\ & = \frac{Z_1 Z_1 \ldots Z_1}{\zeta_N} = \frac{(Z_1)^N}{\zeta_N}
        \end{aligned}
        \end{equation*}
    \end{proof}

    Let $f(q^i, p_i)$ be an observable, then its canonical average is 
    \begin{equation*}\label{obsc}
        \avp{f(q^i, p_i)}{c} = \int_{\mathcal M} d\Omega ~ \rho_{c} f = \int_{\mathcal M} d\Omega ~ \frac{\exp (-\beta \mathcal H)}{Z_N} f
    \end{equation*}

\section{Thermodynamics variable}

    The canonical Helmotz free energy $F$ is defined by 
    \begin{equation}\label{ZF}
        Z_[V, T] = \exp(-\beta F[N, V, T])
    \end{equation}
    or, equivalently,
    \begin{equation}\label{can:f}
        F[V, N ,T] = -\frac{1}{\beta} \ln Z_N
    \end{equation}

    Furthermore, the canonical internal energy is 
    \begin{equation}\label{Ec}
        E = \avp{\mathcal H}{c} = \int d\Omega \frac{\exp(-\beta (\mathcal H))}{Z_N} \mathcal H
    \end{equation}

    \begin{proof}
        By normalisation condition 
        \begin{equation*}
            1 = \int d\Omega \frac{\exp(-\beta \mathcal H)}{Z_N} = \int d\Omega \frac{\exp(-\beta \mathcal H)}{\exp(-\beta F)} = \int d\Omega \exp (- \beta (\mathcal H - F))
        \end{equation*}

        Since $F$ depends on the temperature, it is possible to derive with respect to $\beta$
        \begin{equation*}
        \begin{aligned}
            0 & = \pdv{}{\beta} \Big ( \int d\Omega \exp (- \beta (\mathcal H - F)) \Big) \\ & = \int d\Omega \exp (-\beta (\mathcal H - F)) \Big (-(\mathcal H - F) + \beta \pdv{F}{\beta}) \\ & = - \underbrace{\int d\Omega \frac{\exp(-\beta \mathcal H)}{Z_N} \mathcal H}_{E} + F \underbrace{\int d\Omega \frac{\exp(-\beta \mathcal H)}{Z_N}}_{1} + \beta \pdv{F}{\beta} \underbrace{\int d\Omega \frac{\exp(-\beta \mathcal H)}{Z_N}}_{1} \\ & = - E + F + \beta \pdv{F}{\beta}
        \end{aligned}
        \end{equation*}
        Hence, using~\eqref{F}
        \begin{equation*}
            F = E + \beta \pdv{F}{\beta} = E + T \pdv{F}{T} = E - TS
        \end{equation*}
        showing that is indeed the Helmotz free energy.
    \end{proof}

    Notice that in the last result, the entropy can be also written as 
    \begin{equation} \label{can:s}
        S_c = \frac{E - F}{T}
    \end{equation}
    
    The internal energy can also be written as 
    \begin{equation}\label{can:en}
        E = - \pdv{}{\beta} \ln Z_N
    \end{equation}

    \begin{proof}
        Using~\eqref{Ec},
        \begin{equation*}
            - \pdv{}{\beta} \ln Z_N = - \frac{1}{Z_N} \pdv{Z_N}{\beta} = - \frac{1}{Z_N} \pdv{}{\beta} \int d\Omega \exp (-\beta \mathcal H) = \int d\Omega \frac{\exp(-\beta \mathcal H)}{Z_N}  \mathcal H = \avp{\mathcal H}{c} = E
        \end{equation*}
    \end{proof}

    The universal Boltzmann's formula is still valid
    \begin{equation*}
        S_c = -k_B \avp{\ln \rho_c}{c} 
    \end{equation*}


    \begin{proof}
        Using~\eqref{Ec} and~\eqref{can:f}
        \begin{equation*}
            \begin{aligned}
            -k_B \avp{\ln \rho_c}{c} & = -k_B \int d\Omega \rho_c \ln \rho_c \\ & = -k_B \int d\Omega \rho_c \ln \frac{\exp(-\beta \mathcal H)}{Z_N} \\ & = -k_B \int d\Omega \rho_c \ln \exp(-\beta \mathcal H) - k_B \int d\Omega \rho_c \ln Z_N \\ & = k_B \int d\Omega \beta \mathcal H - k_B \underbrace{\ln Z_N}_{\beta F} \underbrace{\int d\Omega \rho_c}_{1} \\ & =  \frac{E - F}{T} = S_c
        \end{aligned}
        \end{equation*}
    \end{proof}

\section{Equipartition theorem}

    \begin{theorem}[Generalised equipartition theorem]
        Let $\xi \in [a,b]$ and $\xi_j$ with $j \neq 1$ all the other coordinates or momenta. Suppose also 
        \begin{equation}\label{cond}
            \int \prod_{j \neq 1} d \xi_j [\xi_1 \exp(-\beta \mathcal H)]_a^b = 0
        \end{equation}
        Then 
        \begin{equation}\label{equi}
            \avp{\xi_1 \pdv{\mathcal H}{\xi_1}}{c} = k_B T
        \end{equation}
    \end{theorem}

    \begin{proof}
        By normalisation condition 
        \begin{equation*}
            1 = \int d\Omega \frac{\exp(-\beta \mathcal H)}{Z_N} = \frac{1}{Z_N} \int \prod_{j \neq 1} d \xi_j \exp(-\beta \mathcal H)
        \end{equation*}
        Using
        \begin{equation*}
            d\xi_1 (\xi_1 \exp(-\beta \mathcal H)) = d\xi_1 \exp(-\beta \mathcal H) + \xi \exp(-\beta \mathcal H) (-\beta) \pdv{\mathcal H}{\xi_1} d\xi_1
        \end{equation*}
        and integrating per parts
        \begin{equation*}
        \begin{aligned}        
            1 & = \frac{1}{Z_N} \underbrace{\int \prod_{j \neq 1} d \xi_j [\xi_1 \exp(-\beta \mathcal H)]_a^b}_{0} + \frac{\beta}{Z_N} \int \prod_{j \neq 1} d \xi_j d\xi_1 \xi_1 \pdv{\mathcal H}{\xi_1} \exp (-\beta \mathcal H) \\ & = \beta \int d\Omega \xi_1 \pdv{\mathcal H}{\xi_1} \frac{\exp (- \beta \mathcal H)}{Z_N} \\ & = \beta \avp{\xi_1 \pdv{\mathcal H}{\xi_1}}{c}
        \end{aligned}
        \end{equation*}
        Hence
        \begin{equation*}
            \avp{\xi_1 \pdv{\mathcal H}{\xi_1}}{c} = \frac{1}{\beta} = k_B T
        \end{equation*}
    \end{proof}

    Examples of system that satisfies the condition~\eqref{cond} are hamiltonians which depend on the square of momentum or confining potentials which go to infinity on the extremes $a$ and $b$. 

    \begin{corollary}[Equipartition theorem]
        If $\xi_1$ appears quadratically in $\mathcal H$, then its contribution to $E$ is $\frac{1}{2} k_B T$
    \end{corollary}

    \begin{proof}
        Consider $\mathcal H = A \xi_1^2 + B \xi_j^2$ with $j \neq 1$, then by the previous theorem 
        \begin{equation*}
            \avp{\xi_1 \pdv{\mathcal H}{\xi_1}}{c} = \avp{\xi 2 A \xi_1}{c} = k_B T
        \end{equation*}
        and 
        \begin{equation*}
            \avp{A \xi_1^2}{c} = \frac{1}{2} k_B T
        \end{equation*}
    \end{proof}

\chapter{Grancanonical ensemble}

    A grancanonical ensemble is a system which is immersed in a bigger environment or reservoir, which can exchange both energy and matter, so $T$, and $V$ are fixed. Globally, both energy and number of particles are conserved, since the universe composed by the union of the system and the environment can be considered as a microcanonical ensemble. First, with the same method used in the previous chapter, microcanonical can be transformed into canonical. Now, the universe is canonical and, globally, the number of particles is conserved. 

    The grancanonical probability density distribution is 
    \begin{equation*}
        \rho_{gc} (q^i, p_i) = \frac{\exp(-\beta \mathcal H_1)}{N_1! h^{d N_1}} \frac{Z_{N_2} [T, V_2]}{Z_N [T, V]}
    \end{equation*}
    
    \begin{proof}
        Consider the universe as a canonical ensemble. Its probability density distribution is 
        \begin{equation*}
            \rho_c (q_i^{(1)}, p_i^{(1)}, q_i^{(2)}, p_i^{(2)}) = \frac{\exp (-\beta \mathcal H (q_i^{(1)}, p_i^{(1)}, q_i^{(2)}, p_i^{(2)}))}{Z_N[T, V]}
        \end{equation*}
        where the total hamiltonian is 
        \begin{equation*}
            \mathcal H (q_i^{(1)}, p_i^{(1)}, q_i^{(2)}, p_i^{(2)}) = \mathcal H_1 (q_i^{(1)}, p_i^{(1)}) + \mathcal H_2 (q_i^{(2)}, p_i^{(2)})
        \end{equation*}

        Integrating it to all the possible state in the environment
        \begin{equation*}
        \begin{aligned}
            \rho^{(1)} & = \int d\Omega_2 ~ \rho_c \\ & = \int \prod_{i=1}^N \frac{d^d q_i^{(2)} d^d p_i^{(2)}}{N! h^{dN}} \frac{\exp(-\beta (\mathcal H_1 + \mathcal H_2))}{Z_N} \\ & = \frac{\exp(-\beta \mathcal H_1)}{N_1! h^{d N_1}} \frac{1}{Z_N} \int \prod_{i=1}^N \frac{d^d q_i^{(2)} d^d p_i^{(2)}}{N_2! h^{d N_2}} \exp(-\beta \mathcal H_2) \\ & = \frac{\exp(-\beta \mathcal H_1)}{N_1! h^{d N_1}} \frac{Z_{N_2} [T, V_2]}{Z_N [T, V]}
        \end{aligned}
        \end{equation*}
    \end{proof}

    The normalisation condition becomes 
    \begin{equation*}
        \sum_{N_1 = 0}^{N} \int_{\mathcal M^{N_1}} d\Omega_1 \rho_{gc} = 1
    \end{equation*}

    \begin{proof}
        Using the expression to evaluate the power of a sum 
        \begin{equation*}
            (a + b)^n = \sum_{i=1}^{n} \binom{n}{i} a^i b^{n-i} 
        \end{equation*}
        and 
        \begin{equation*}
        \begin{aligned}
            \int_{\mathcal M^{N_1}} d\Omega_1 ~ \rho_{gc} & = \int_{\mathcal M^{N_1}} d\Omega_1 \frac{\exp(-\beta \mathcal H_1)}{N_1! h^{d N_1}} \frac{Z_{N_2} [T, V_2]}{Z_N [T, V]} \\ & = \frac{N!}{N_1! N_2} \frac{\int_{\mathcal M^{N_1}} d\Omega_1 ~ \exp(-\beta \mathcal H_1) \int_{\mathcal M^{N_2}} d\Omega_2 ~ \exp(-\beta \mathcal H_2)}{\int_{\mathcal M^N} d\Omega ~ \exp(-\beta \mathcal H)} \\ & = \frac{N!}{N_1! N_2} \frac{\frac{\int_{\mathcal M^{N_1}} d\Omega_1 ~ \exp(-\beta \mathcal H_1)}{(V_1)^{N_1}} \frac{\int_{\mathcal M^{N_2}} d\Omega_2 ~ \exp(-\beta \mathcal H_2)}{(V_2)^{N_2}}}{\frac{\int_{\mathcal M^N} d\Omega ~ \exp(-\beta \mathcal H)}{V^N}} \frac{(V_1)^{N_1} (V_2)^{N_2}}{V^N} 
        \end{aligned}
        \end{equation*}
        which in the thermodynamical limit 
        \begin{equation*}
            \lim_{td} \frac{\frac{\int_{\mathcal M^{N_1}} d\Omega_1 ~ \exp(-\beta \mathcal H_1)}{(V_1)^{N_1}} \frac{\int_{\mathcal M^{N_2}} d\Omega_2 ~ \exp(-\beta \mathcal H_2)}{(V_2)^{N_2}}}{\frac{\int_{\mathcal M^N} d\Omega ~ \exp(-\beta \mathcal H)}{V^N}} = 1
        \end{equation*}

        Hence 
        \begin{equation*}
            \int_{\mathcal M^{N_1}} d\Omega_1 ~ \rho_{gc} = \frac{N!}{N_1! N_2} \frac{(V_1)^{N_1} (V_2)^{N_2}}{V^N} 
        \end{equation*}
        and the normalisation condition becomes, using $N = N_1 + N_2$, 
        \begin{equation*}
            \sum_{N_1 = 0}^{N} \int_{\mathcal M^{N_1}} d\Omega_1 \rho_{gc} = \sum_{N_1 = 0}^{N} \frac{N!}{N_1! N_2!} \frac{(V_1)^{N_1} (V_2)^{N_2}}{V^N} = \sum_{N_1 = 0}^{N} \binom{N}{N_1} \Big ( \frac{V}{V} \Big)^{N_1}  \Big ( \frac{V_2}{V} \Big)^{N - N_1} = \Big ( \frac{V_1 + V_2}{V} \Big)^N 
        \end{equation*}
        which in the thermodynamical limit is 
        \begin{equation*}
            \lim_{td} \Big ( \frac{V_1 + V_2}{V} \Big)^N  = 1
        \end{equation*}
    \end{proof}

\section{Thermodynamical potentials} 

    The grancanonical probability density distribution can be also written as 
    \begin{equation*}
        \rho_{gc} (q_i, p_i) = \frac{\exp(-\beta (\mathcal H (q_i, p_i) - \mu N))}{\mathcal Z}
    \end{equation*}
    where $\mu$ is the chemical potential and $\mathcal Z$ is the grancanonical partition function
    \begin{equation*}
        \mathcal Z = \sum_{N = 0}^{\infty} z^N Z_N = \exp(-\beta \Omega) 
    \end{equation*}
    where $z = \exp(\beta \mu)$ is the fugacity and $\Omega$ is the granpotential. 

    \begin{proof}
        Using~\eqref{ZF} and Taylor expanding to first order in $N_1 \ll N$ and $V_1 \ll V$, 
        \begin{equation*}
        \begin{aligned}
                \frac{Z_{N_2}[T, V]}{Z_N[T, V]} & = \frac{\exp(-\beta F(T, N_2, V_2))}{\exp(-\beta F(T, N, V))} \\ & = \exp(-\beta (F(T, N-N_1, V-V_1) - F(T, N, V))) \\ & \simeq \exp(-\beta(\underbrace{\pdv{F}{N} \Big \vert_{T, V}}_{\mu} (-N_1) + \underbrace{\pdv{F}{V} \Big \vert_{T, N}}_{-p} (-V_1))) \\ & = \exp(-\beta(-\mu N_1 + p V_1))
        \end{aligned}
        \end{equation*}

        Hence, now all the degrees of freedom of the environment has been removed
        \begin{equation*}
        \begin{aligned}
            \rho_{gc} & = \frac{\exp(\beta \mathcal H)}{N! h^{dN}} \exp (-\beta (-\mu N + p V)) \\ & = \frac{\exp(\beta \mathcal H)}{N! h^{dN}} \underbrace{\exp ( \beta \mu)^N}_{z^N} \exp(-\beta p V) \\ & = \frac{z^N \exp(\beta \mathcal H)}{N! h^{dN}} \exp(-\beta p V)
        \end{aligned}
        \end{equation*}
        where we introduced the fugacity.
        
        Recall~\eqref{OM}, the normalisation condition becomes 
        \begin{equation*}
        \begin{aligned}
            1 & = \sum_{N=0}^{\infty} \int_{\mathcal M^N} d\Omega \rho_{gc} \\ & = \sum_{N=0}^{\infty} \int_{\mathcal M^N} d\Omega \frac{z^N \exp(\beta \mathcal H)}{N! h^{dN}} \exp(-\beta p V) \\ & = \exp(-\beta p V) \sum_{N=0}^{\infty} z^N \frac{\int_{\mathcal M^N} d\Omega}{h^{dN} N!} \\ & = \exp(-\beta p V) \underbrace{\sum_{N=0}^{\infty} z^N Z_N}_{\mathcal Z} \\ & = \exp(-\beta p V) \mathcal Z
        \end{aligned}
        \end{equation*}
        Hence 
        \begin{equation*}
            \mathcal Z = sum_{N=0}^{\infty} z^N Z_N = \exp(\beta p V)
        \end{equation*}
        and 
        \begin{equation*}
            \rho_{gc} (q_i, p_i) = \frac{\exp(-\beta (\mathcal H(q_i, p_i) - \mu N))}{\mathcal Z} = \frac{\exp(-\beta \mathfrak H(q_i, p_i) )}{\mathcal Z}
        \end{equation*}
        where $\mathfrak H = \mathcal H - \mu N$ is the grancanonical hamiltonian.
    \end{proof}

    Let $f(q^i, p_i)$ be an observable, then its grancanonical average is 
    \begin{equation*}\label{obsgc}
    \begin{aligned}
        \avp{f(q^i, p_i)}{gc} & = \sum_{N = 0}^{\infty} \int_{\mathcal M} d\Omega ~ \rho_{gc} f_N \\ & = \sum_{N = 0}^{\infty} \int_{\mathcal M} d\Omega ~ \frac{\exp (-\beta (\mathcal H - \mu N ))}{\mathcal Z} f_N \\ & = \frac{1}{\mathcal Z} \sum_{N=0}^{\infty} z^N Z_N \int_{\mathcal M} d\Omega \frac{\exp(-\beta \mathcal H)}{Z_N} f_N \\ & = \frac{1}{\mathcal Z} \sum_{N=0}^{\infty} z^N Z_N \avp{f_N}{c}
    \end{aligned}
    \end{equation*}

    The grancanonical internal energy is 
    \begin{equation}\label{gran:e}
        E = - \pdv{}{\beta} \ln \mathcal Z \Big \vert_z
    \end{equation}

    \begin{proof}
        \begin{equation*}
        \begin{aligned}
            - \pdv{}{\beta} \ln \mathcal Z \Big \vert_z & = - \frac{1}{\mathcal Z} \pdv{}{\beta} \mathcal Z \Big \vert_z \\ &  = - \frac{1}{\mathcal Z} \pdv{}{\beta} \mathcal \sum_{N=0}^{\infty} z^N Z_N \Big \vert_z \\ & = - \sum_{N=0}^{\infty} \frac{z^N}{\mathcal Z} \pdv{}{\beta} \int d\Omega \exp (-\beta \mathcal H) \\ & = \sum_{N=0}^{\infty} \int d\Omega ~ \frac{\exp(-\beta (\mathcal H + \mu N))}{\mathcal Z} \mathcal H \\ & = \avp{\mathcal H}{gc} = E  
        \end{aligned}
        \end{equation*}
    \end{proof}

    The grancanonical number of particles is 
    \begin{equation}\label{gran:n}
        \avp{N}{gc} = z \pdv{}{z} \ln \mathcal Z \Big \vert_T
    \end{equation}

    \begin{proof}
        \begin{equation*}
        \begin{aligned}
            z \pdv{}{z} \ln \mathcal Z \Big \vert_T & = \frac{z}{\mathcal Z} \pdv{}{z}\mathcal Z \Big \vert_T \\ & = \frac{z}{\mathcal Z} \pdv{}{z} \sum_{N=0}^{\infty} z^N Z_N \\ & = frac{z}{\mathcal Z} \sum_{N=0}^{\infty} N z^{N-1} Z_N \\ & = \sum_{N=0}^{\infty} z^N Z_N N = \avp{N}{gc}
        \end{aligned}
        \end{equation*}
    \end{proof}

    The grancanonical potential is 
    \begin{equation}\label{gran:o}
        \Omega = - \frac{1}{\beta} \ln \mathcal Z
    \end{equation}

    The universal Boltzmann's formula is still valid
    \begin{equation*}
        S_{gc} = -k_B \avp{\ln \rho_{gc}}{gc} 
    \end{equation*}

    \begin{proof}
        Using~\eqref{gran:o},
        \begin{equation*}
        \begin{aligned}
            -k_B \avp{\ln \rho_{gc}}{gc} & = -k_B \int d\Omega ~ \rho_{gc} \ln \rho_{gc} \\ & = -k_B \sum_{N=0}^{\infty} \frac{z^N}{\mathcal Z} \int d\Omega ~ \exp(- \beta \mathcal H) \ln \rho_{gc} \\ & = = -k_B \sum_{N=0}^{\infty} \frac{z^N}{\mathcal Z} \int d\Omega ~ \exp(- \beta \mathcal H) (- \beta \mathcal H + \beta \mu N + \ln \mathcal Z) \\ & = k_B \beta \underbrace{\sum_{N=0}^{\infty} \frac{z^N}{\mathcal Z} \int d\Omega ~ \exp(-\beta \mathcal H) \mathcal H}_{E} - k_B \beta \mu \underbrace{\sum_{N=0}^{\infty} \frac{z^N}{\mathcal Z} \int d\Omega ~ \exp(-\beta \mathcal H) N}_{N} \\ & \quad + k_B \ln \mathcal Z \underbrace{\sum_{N=0}^{\infty} \frac{z^N}{\mathcal Z} \int d\Omega ~ \exp(-\beta \mathcal H)}_{1} \\ & = \frac{E - \mu N - \Omega}{T} = S
        \end{aligned}
        \end{equation*}
    \end{proof}

\chapter{Entropy}

    The Boltzmann's universal law allows us to define entropy in terms of number of states
    \begin{equation*}
        S = - k_B \av{\ln \rho} = k_B \ln \Sigma = \lim_{TD} S_{TD}
    \end{equation*}

    Thermodynamics tells us that equilibrium corresponds to maximum entropy.

    We consider a canonical ensemble with a discrete set of energy values, but it can be generalised for grancanonical and continuous energy levels. Therefore, the probability density distribution is~\eqref{candist}
    \begin{equation*}
        \rho_c (E_r) = \frac{\exp(-\beta E_r)}{Z_N}
    \end{equation*}
    where the canonical partition function~\eqref{zn} becomes 
    \begin{equation*}
        Z_N = \int_{\mathcal M^N} d\Omega~ \exp(-\beta \mathcal H(q^i, p_i)) = int_0^\infty dE \int_{S_E} dS_E ~ \exp(-\beta E) \simeq \sum_{r=1}^{p} g_r \exp(-\beta E_r)
    \end{equation*}
    where we foliated $\mathcal M^N$ in energy surfaces $S_E$ and $g_r$ is the multiplicity or degeneracy, i.e.~how many levels have the same energy.

    So far, we have started from an a-priori probability density distribution and from it derive the entropy. From now on, we will change the picture and do the converse: the probability distibution is the one corresponding to maximum entropy, given the macroscopic constains. To do so, we introduce the Shannon's information entropy
    \begin{equation*}
        H = - \sum_{i = 1}^{N} p_i \ln p_i
    \end{equation*}
    which is the only function with the following properties for a random variable $x$ such that it has $N$ possible outcomes $x_i$ with probability $p_i$ 
    \begin{enumerate}
        \item it is continuous with $p_i$,
        \item is monotonically increasing with $N$,
        \item it is invariant under compositions of subsystems, i.e.~change how we collect in group.
    \end{enumerate}

\subsection{Inference problem}

    Given a certain constraint for a function $\av{f}$, what is the expectation value for another function $g$? The answer can be found with the principle of maximum entropy, subjected to Lagrange multipliers given by the constraints 
    \begin{equation*}
        \sum_{i=1}^{N} p_i = 1 \quad \sum_{i=1}^{N} p_i f(x_i) = \av{f(x)}
    \end{equation*}
    Hence, the problem reduces to maximise the function
    \begin{equation}\label{entr}
        H = - \sum_{i=1}^{N} p_i \ln p_i + \alpha \Big( \sum_{i=1}^{N} p_i - 1 \Big) + \beta \Big( \sum_{i=1}^{N} p_i f(x_i) - \av{f} \Big)
    \end{equation}

    In particular, we need to count the number of ways $W_{\{n_r\}}$ we can find $n_r$ systems with energy $E_r$, given a set of discrete energy levels $E_r$, each of degeneracy $g_r$ on which we distribute $n_r$ particles. Hence, the probability density distribution $n_r^*$ is the one which maximises~\eqref{entr}, with entropy 
    \begin{equation*}
        S = \ln W_{\{n_r\}}
    \end{equation*} 
    and the constrains 
    \begin{equation*}
        N = \sum_{r} n_r \quad E = \sum_r n_r E_r
    \end{equation*}

    In order to count $W_{\{n_r\}}$, we need to take into account distinguishablility or not of particles. Therefore, we decomposed it into 
    \begin{equation*}
        W_{\{n_r\}} = W_{\{n_r\}}^{(1)} W_{\{n_r\}}^{(2)}
    \end{equation*}
    where $W_{\{n_r\}}^{(1)}$ countsin how many we can put $n_r$ particles in the energy level $E_r$ and $W_{\{n_r\}}^{(1)}$ consider the degeneracy of these levels.

\subsection{Boltzmann distribution}
\subsection{Bose-Einstein distribution}
\subsection{Fermi-Dirac distribution}

\part{Applications of classical statistical mechanics}

\chapter{Microcanonical ensemble}

\section{Non-relativistic ideal gas in d-dimensions}

    Consider a non-relativistic ideal (non-interacting) gas of $N$ particles in an $d$-dimensional manifold with a finite volume $V^N$: $\mathcal M_N = V^N \times \mathbb R^{dN}$. Its hamiltonian is 
    \begin{equation*}
        H = \sum_i \frac{p^2_i}{2m} ~.
    \end{equation*}

    The number of states $\Sigma(E)$ is 
    \begin{equation*}
        \Sigma(E) = \frac{2V^{N}}{\xi_N d N \Gamma(dN/2)} \Big ( \frac{2 \pi m E}{h^2}\Big)^{dN/2} ~.
    \end{equation*}
    \begin{proof}
        By definition, 
        \begin{equation*}
        \begin{aligned}
            \Sigma (E) = \int_{H (q_i, p_i) \leq E} d\Omega = \int_{H (q_i, p_i) \leq E} \frac{\prod_i d^d q_i d^d p_i}{h^{dN} \xi_N} = \frac{1}{h^{dN} \xi_N} \int_{H (q_i, p_i) \leq E} \prod_i d^d q_i d^d p_i ~.
        \end{aligned}
        \end{equation*}

        From the energy,
        \begin{equation*}
            H = \sum_i \frac{p^2_i}{2m} \leq E ~,
        \end{equation*}
        \begin{equation*}
            \sum_i p^2_i \leq 2mE ~.
        \end{equation*}

        Hence, by the volume of a $dN$-sphere of radius $\sqrt{2mE}$ (See Appendix 1),
        \begin{equation*}
        \begin{aligned}
            \Sigma (E) & = \frac{1}{h^{dN} \xi_N} \int_{\sum_i p^2_i \leq 2mE} \prod_i d^d q_i d^d p_i \\ & = \frac{1}{h^{dN} \xi_N} \underbrace{\int_{V^N} \prod_i d^d q_i}_{V^N} \underbrace{\int_{\sum_i p^2_i \leq 2mE} \prod_i d^d p_i}_{\frac{\pi^{dN/2} (2mE)^{dN/2}}{\Gamma (dN/2 + 1)}} \\ & = \frac{V^N}{h^{dN} \xi_N} \frac{\pi^{dN/2} (2mE)^{dN/2}}{\Gamma (dN/2 + 1)} \\ & = \Big (\frac{2 \pi m E}{h^2} \Big)^{dN/2} \frac{2 V^N}{\Gamma (dN/2)\xi_N d N} ~.
        \end{aligned}
        \end{equation*}
    \end{proof}

    The density state $\omega(E)$ is
    \begin{equation*}
        \omega (E) = \frac{V^N}{\xi_N \Gamma(dN/2)} \Big ( \frac{2 \pi m}{h^2} \Big )^{\frac{dN}{2}} E^{dN/2-1} ~.
    \end{equation*}
    \begin{proof}
        By definition, 
        \begin{equation*}
        \begin{aligned}
            \omega (E) &= \pdv{\Sigma (E)}{E} \\ & = \pdv{}{E} \Big (\frac{2 \pi m E}{h^2} \Big)^{dN/2} \frac{2 V^N}{\Gamma (dN/2)\xi_N d N} \\ & = \Big (\frac{2 \pi m}{h^2} \Big)^{dN/2} \frac{2 V^N}{\Gamma (dN/2)\xi_N d N} \pdv{E^{dN/2}}{E} \\ & = \Big (\frac{2 \pi m}{h^2} \Big)^{dN/2} \frac{2 V^N}{\Gamma (dN/2)\xi_N d N} \frac{dN}{2} E^{dN/2 - 1} \\ & = \frac{V^N}{\xi_N \Gamma(dN/2)} \Big ( \frac{2 \pi m}{h^2} \Big )^{\frac{dN}{2}} E^{dN/2-1} ~.
        \end{aligned}
        \end{equation*}
    \end{proof}

    Notice that 
    \begin{equation}\label{mc1}
        \omega(E) = \frac{dN}{2E} \Sigma(E) ~, \quad \Gamma(E) = \omega(E) \Delta E = \frac{dN}{2E} \Sigma(E) \Delta E ~.
    \end{equation}
    As a consequence, in the thermodynamic limit, we have the following equivalent relations
    \begin{equation*}
        \lim_{TD} \frac{\ln \Gamma (E)}{N} = \lim_{TD} \frac{\ln \omega (E)}{N} = \lim_{TD} \frac{\ln \Sigma (E)}{N} ~.
    \end{equation*}
    \begin{proof}
        Observing~\eqref{mc1}, we find that the logarithmic expression differs only for factors $\ln \Delta E$ and $\ln \frac{dN}{2E}$, which are neglectible in the thermodynamic limit since they do not scale as $N$.
    \end{proof}

    The entropy is 
    \begin{equation*}
        \frac{S}{k_B} = \ln \Gamma(E) = \ln \omega(E) + \ln \Delta E = \ln \Sigma(E) = \ln \Sigma(E) + \ln \frac{dN}{2E} + \ln \Delta E ~,
    \end{equation*}
    In the thermodynamic limit, it becomes
    \begin{equation*}
        S = k_B \begin{cases}
            \frac{d}{2} N + N \ln \Big ( V (\frac{4 \pi m E}{d N h^2})^{d/2} \Big) & \textnormal{for distinguishable particles} \\
            \frac{d + 2}{2} N + N \ln \Big ( \frac{V}{N} (\frac{4 \pi m E}{d N h^2})^{d/2} \Big) & \textnormal{for indistinguishable particles} \\
        \end{cases} ~.
    \end{equation*}
    \begin{proof}
        By definition, using the Stirling approximation (See Appendix 2),
        \begin{equation*}
        \begin{aligned}
            \frac{S}{k_B} & = \ln \Sigma (E) \\ & = \ln \Big ( \frac{2V^{N}}{\xi_N d N \Gamma(dN/2)} \Big ( \frac{2 \pi m E}{h^2}\Big)^{dN/2} \Big ) \\ & = \cancel{\ln 2} + N \ln V - \ln \xi_N - \cancel{\ln d} - \cancel{\ln N} - \ln \Gamma (dN/2) + N \ln \Big (\frac{2 \pi m E}{h^2} \Big )^{d/2} \\ & =  N \ln V - \ln \xi_N - \underbrace{\ln \Gamma (dN/2)}_{\frac{dN}{2} \ln \frac{dN}{2} - \frac{dN}{2}} + N \ln \Big (\frac{2 \pi m E}{h^2} \Big )^{d/2} \\ & = N \ln V - \ln \xi_N - \frac{dN}{2} \ln \frac{dN}{2} + \frac{dN}{2} + N \ln \Big (\frac{2 \pi m E}{h^2} \Big )^{d/2} \\ & = N \ln V - \ln \xi_N - N \ln \Big(\frac{dN}{2} \Big)^{d/2} + \frac{dN}{2} + N \ln \Big (\frac{2 \pi m E}{h^2} \Big )^{d/2} \\ & = - \ln \xi_N + \frac{dN}{2} + N \ln \Big ( V (\frac{4 \pi m E}{d N h^2})^{d/2} \Big) ~.
        \end{aligned}
        \end{equation*}

        Now, we treat the distinguishable and indistinguishable case separately. For distinguishable particles $\xi_N = 1$, we find
        \begin{equation*}
            \frac{S}{k_B} = - \ln 1 + \frac{dN}{2} + N \ln \Big (V \frac{4 \pi m E}{dNh^2} \Big )^{d/2} = \frac{d}{2} N + N \ln \Big ( V (\frac{4 \pi m E}{d N h^2})^{d/2} \Big) ~.
        \end{equation*}
        For indistinguishable particles $\xi_N = N!$, we find
        \begin{equation*}
        \begin{aligned}
            \frac{S}{k_B} & = - \underbrace{\ln N!}_{N \ln N - N} + \frac{dN}{2} + N \ln \Big ( V (\frac{4 \pi m E}{d N h^2})^{d/2} \Big) \\ & = - N \ln N + N + \frac{dN}{2} + N \ln \Big ( V (\frac{4 \pi m E}{d N h^2})^{d/2} \Big) \\ & = \frac{d + 2}{2} N + N \ln \Big ( \frac{V}{N} (\frac{4 \pi m E}{d N h^2})^{d/2} \Big) ~.
        \end{aligned}
        \end{equation*}
    \end{proof}

    The internal energy is 
    \begin{equation*}
        E = \frac{d N k_B T}{2} ~.
    \end{equation*}
    \begin{proof}
        By~\eqref{ses}
        \begin{equation*}
            \frac{1}{T} = \pdv{S}{E} = k_B \frac{dN}{2} \pdv{}{E} \ln E = k_B \frac{dN}{2E} ~,
        \end{equation*}
        hence 
        \begin{equation*}
            E = \frac{d N k_B T}{2} ~.
        \end{equation*}
    \end{proof}

    The equation of state is  
    \begin{equation*}
        p V = N k_B T ~.
    \end{equation*}
    \begin{proof}
        By~\eqref{ses}
        \begin{equation*}
            \frac{p}{T} = \pdv{S}{V} = k_B N \pdv{}{V} \ln V = k_B \frac{N}{V}  ~,
        \end{equation*}
        hence 
        \begin{equation*}
            pV = N k_B T ~.
        \end{equation*}
    \end{proof}

\section{Non-relativistic ideal gas in 3-dimensions}

    Now, consider the case in which $d = 3$.
    The number of states $\Sigma(E)$ is 
    \begin{equation*}
        \Sigma(E) = \frac{2V^{N}}{\xi_N d N \Gamma(3N/2)} \Big ( \frac{2 \pi m E}{h^2}\Big)^{3N/2} ~.
    \end{equation*}
    The density state $\omega(E)$ is
    \begin{equation*}
        \omega (E) = \frac{V^N}{\xi_N \Gamma(3N/2)} \Big ( \frac{2 \pi m}{h^2} \Big )^{\frac{3N}{2}} E^{3N/2-1} ~.
    \end{equation*}
    Notice that 
    \begin{equation*}
        \omega(E) = \frac{3N}{2E} \Sigma(E) ~, \quad \Gamma(E) = \omega(E) \Delta E = \frac{3N}{2E} \Sigma(E) \Delta E ~.
    \end{equation*}
    The entropy is 
    \begin{equation*}
        \frac{S}{k_B} = \ln \Gamma(E) = \ln \omega(E) + \ln \Delta E = \ln \Sigma(E) = \ln \Sigma(E) + \ln \frac{3N}{2E} + \ln \Delta E ~,
    \end{equation*}
    In the thermodynamic limit, it becomes
    \begin{equation*}
        S = k_B \begin{cases}
            \frac{3}{2} N + N \ln \Big ( V (\frac{4 \pi m E}{3 N h^2})^{3/2} \Big) & \textnormal{for distinguishable particles} \\
            \frac{5}{2} N + N \ln \Big ( \frac{V}{N} (\frac{4 \pi m E}{3 N h^2})^{3/2} \Big) & \textnormal{for indistinguishable particles} \\
        \end{cases} ~.
    \end{equation*}
    The internal energy is 
    \begin{equation*}
        E = \frac{3}{2} N k_B T ~.
    \end{equation*}
    The equation of state is  
    \begin{equation*}
        p V = N k_B T ~.
    \end{equation*}

\section{Gas of harmonic oscillators in 3-dimensions}

    Consider a non-relativistic (non-interacting) gas of $N$ particles in an $d$-dimensional manifold confined by an harmonic potential of frequency $\omega$. Its hamiltonian is 
    \begin{equation*}
        H = \sum_i \Big ( \frac{p^2_i}{2m} + \frac{m \omega^2}{2} q_i^2 \Big ) ~.
    \end{equation*}

    The number of states $\Sigma(E)$ is 
    \begin{equation*}
        \Sigma(E) = \frac{1}{\xi_N \Gamma(dN/2) d N} \Big ( \frac{2 \pi E}{h \omega}\Big)^{dN} ~.
    \end{equation*}
    \begin{proof}
        By definition, 
        \begin{equation*}
        \begin{aligned}
            \Sigma (E) = \int_{H (q_i, p_i) \leq E} d\Omega = \int_{H (q_i, p_i) \leq E} \frac{\prod_i d^d q_i d^d p_i}{h^{dN} \xi_N} = \frac{1}{h^{dN} \xi_N} \int_{H (q_i, p_i) \leq E} \prod_i d^d q_i d^d p_i ~.
        \end{aligned}
        \end{equation*}

        We make a change of variable into $x_j$, with $j = 1, \ldots 2dN$,
        \begin{equation*}
            p_i = \sqrt{2mE} x_j ~, \quad q_i = \sqrt{\frac{2E}{m\omega^2}} x_{dN + j} ~.
        \end{equation*}
        The differentials become 
        \begin{equation*}
            d p_i = \sqrt{2mE} dx_j  ~, \quad d q_i = \sqrt{\frac{2E}{m\omega^2}} d x_{dN + j} ~.
        \end{equation*}

        From the energy,
        \begin{equation*}
            H = \sum_i \Big ( \frac{p^2_i}{2m} + \frac{m \omega^2}{2} q_i^2 \Big )  \leq E ~,
        \end{equation*}
        \begin{equation*}
            \sum_j x^2_j \leq 1 ~.
        \end{equation*}

        Hence, by the volume of a $2dN$-sphere of radius $1$ (See Appendix 1),
        \begin{equation*}
        \begin{aligned}
            \Sigma (E) & = \frac{1}{h^{dN} \xi_N} \int_{\sum_i \Big ( \frac{p^2_i}{2m} + \frac{m \omega^2}{2} q_i^2 \Big ) \leq E} \prod_i d^d q_i d^d p_i \\ & = \frac{1}{h^{dN} \xi_N} (2mE)^{dN/2} \Big (\frac{2E}{m\omega^2} \Big )^{dN/2} \underbrace{\int_{\sum_j x^2_j \leq 1} \prod_j d x_j}_{ \frac{\pi^{dN}}{\Gamma (dN + 1)}} \\ & = \frac{1}{\xi_N \Gamma (dN + 1)} \Big (\frac{2 \pi E}{h \omega} \Big )^{dN} \\ & = \frac{1}{\xi_N \Gamma (dN) dN} \Big (\frac{2 \pi E}{h \omega} \Big )^{dN} ~.
        \end{aligned}
        \end{equation*}
    \end{proof}

    The density state $\omega(E)$ is
    \begin{equation*}
        \omega (E) = \frac{1}{\xi_N \Gamma (dN)} \Big (\frac{2 \pi}{h \omega} \Big )^{dN} E^{dN-1} ~.
    \end{equation*}
    \begin{proof}
        By definition, 
        \begin{equation*}
        \begin{aligned}
            \omega (E) &= \pdv{\Sigma (E)}{E} \\ & = \pdv{}{E} \frac{1}{\xi_N \Gamma (dN) dN} \Big (\frac{2 \pi E}{h \omega} \Big )^{dN} \\ & = \frac{1}{\xi_N \Gamma (dN) dN} \Big (\frac{2 \pi}{h \omega} \Big )^{dN} \pdv{}{E} E^{dN} \\ & = \frac{1}{\xi_N \Gamma (dN) dN} \Big (\frac{2 \pi}{h \omega} \Big )^{dN} dN E^{dN-1} \\ & = \frac{1}{\xi_N \Gamma (dN)} \Big (\frac{2 \pi}{h \omega} \Big )^{dN} E^{dN-1} ~.
        \end{aligned}
        \end{equation*}
    \end{proof}

    In the thermodynamic limit, the entropy becomes
    \begin{equation*}
        S = k_B \begin{cases}
            d N + N \ln \Big (\frac{2 \pi E }{h \omega d N} \Big)^{d} & \textnormal{for distinguishable particles} \\
            (d+1) N + N \ln \Big ( \frac{1}{N} (\frac{2 \pi E }{h \omega d N} )^d \Big ) & \textnormal{for indistinguishable particles} \\
        \end{cases} ~.
    \end{equation*}
    \begin{proof}
        By definition, using the Stirling approximation (See Appendix 2),
        \begin{equation*}
        \begin{aligned}
            \frac{S}{k_B} & = \ln \Sigma (E) \\ & = \ln \frac{1}{\xi_N \Gamma(dN/2) d N} \Big ( \frac{2 \pi E}{h \omega} \Big)^{dN} \\ & = - \ln \xi_N - \ln \Gamma (dN) - \cancel{\ln d} - \cancel{\ln N} + N \ln \Big (\frac{2 \pi E }{h \omega} \Big )^{d} \\ & - \ln \xi_N - \underbrace{\ln \Gamma (dN)}_{dN \ln (dN) - dN} + N \ln \Big (\frac{2 \pi E }{h \omega} \Big )^{d} \\ & = - \ln \xi_N - dN \ln (dN) + d N + N \ln \Big (\frac{2 \pi E }{h \omega} \Big )^{d} \\ & = - \ln \xi_N + d N + N \ln \Big (\frac{2 \pi E }{h \omega d N} \Big )^{d} ~.
        \end{aligned}
        \end{equation*}

        Now, we treat the distinguishable and indistinguishable case separately. For distinguishable particles $\xi_N = 1$, we find
        \begin{equation*}
            \frac{S}{k_B} = - \ln 1 + d N + N \ln \Big (\frac{2 \pi E }{h \omega d N} \Big )^{d} = d N + N \ln \Big (\frac{2 \pi E }{h \omega d N} \Big )^{d} ~.
        \end{equation*}
        For indistinguishable particles $\xi_N = N!$, we find
        \begin{equation*}
        \begin{aligned}
            \frac{S}{k_B} & = - \underbrace{\ln N!}_{N \ln N - N} + d N + N \ln \Big (\frac{2 \pi E }{h \omega d N} \Big )^{d} \\ & =  - N \ln N + N + d N + N \ln \Big (\frac{2 \pi E }{h \omega d N} \Big )^{d} \\ & = (d+1) N + N \ln \Big ( \frac{1}{N} \frac{2 \pi E }{h \omega d N} \Big )^{d} ~.
        \end{aligned}
        \end{equation*}
    \end{proof}

    The internal energy is 
    \begin{equation*}
        E = d N k_B T ~.
    \end{equation*}
    \begin{proof}
        By~\eqref{ses}
        \begin{equation*}
            \frac{1}{T} = \pdv{S}{E} = k_B dN \pdv{}{E} \ln E = k_B \frac{dN}{E} ~,
        \end{equation*}
        hence 
        \begin{equation*}
            E = d N k_B T ~.
        \end{equation*}
    \end{proof}

\chapter{Canonical ensemble}

\section{Non-relativistic ideal gas in d-dimensions}

    Consider an indistinguishable non-relativistic ideal (non-interacting) gas of $N$ particles in an $d$-dimensional manifold with a finite volume $V^N$: $\mathcal M_N = V^N \times \mathbb R^{dN}$. If we did not confined the particles is a finite volume, we would have found undesidered divergences. Its hamiltonian is 
    \begin{equation*}
        H = \sum_i \frac{p^2_i}{2m} ~.
    \end{equation*}

    The canonical partition function $Z$ is 
    \begin{equation*}
        Z = \frac{V^N}{\xi_N \lambda^{dN}_T} = \frac{V^N}{ \xi_N} (\frac{2 m \pi}{\beta h^2})^{dN/2} ~.
    \end{equation*}
    \begin{proof}
        By definition, using the gaussian integral (See appendix 3),
        \begin{equation*}
        \begin{aligned}
            Z & = \int_{\mathcal M^N} d\Omega \exp(- \beta H (q_i, p_i)) \\ & = \int_{\mathcal M^N} \frac{\prod_i d^d q_i d^d p_i}{h^{dN} \xi_N} \exp(- \beta H (q_i, p_i)) \\ & = \frac{1}{h^{dN} \xi_N} \int_{\mathcal M^N} \prod_i d^d q_i d^d p_i \exp(- \beta H (q_i, p_i)) \\ & = \frac{1}{h^{dN} \xi_N} \underbrace{\int_{ V^N} \prod_i d^d q_i}_{V^N} \underbrace{\prod_i \int_{\mathcal M^N} d^d p_i \exp(- \beta \frac{p^2_i}{2m})}_{(\frac{2 m \pi}{\beta})^{dN/2}} \\ & = \frac{V^N}{h^{dN} \xi_N} (\frac{2 m \pi}{\beta})^{dN/2} \\ & = \frac{V^N}{ \xi_N} (\frac{2 m \pi}{\beta h^2})^{dN/2} \\ & = \frac{V^N}{\xi_N \lambda^{dN}_T}
        \end{aligned}
        \end{equation*}
        where we have defined the thermal wavelength 
        \begin{equation*}
            \lambda_T = \sqrt{\frac{\beta h^2}{2 m \pi}} ~.
        \end{equation*}
    \end{proof}

    For indistinguishable particles, the canonical partition function $Z$ is 
    \begin{equation*}
        Z = \frac{V^N}{N! \lambda^{dN}_T} = \frac{V^N}{N!} (\frac{2 m \pi}{\beta h^2})^{dN/2} ~.
    \end{equation*}

    An useful intermediary formula is 
    \begin{equation*}
        \ln Z = N (1 - \ln (\frac{N}{V} \lambda_T^d)) = N (1 - \ln (n \lambda_T^d)) ~.
    \end{equation*}
    \begin{proof}
        In fact, using the Stirling approximation (See Appendix 2),
        \begin{equation*}
        \begin{aligned}
            \ln Z & = \ln \frac{V^N}{N! \lambda^{dN}_T} \\ & = N \ln (V \lambda_T^d) - \underbrace{\ln N!}_{N \ln N - N} \\ & = N - N \frac{V \lambda_T^d}{N} \\ & = N (1 - \ln (\frac{N}{V} \lambda_T^d)) \\ & = N (1 - \ln (n \lambda_T^d))  ~,
        \end{aligned}
        \end{equation*}
        where we have defined the density
        \begin{equation*}
            n = \frac{N}{V} ~.
        \end{equation*}
    \end{proof}
    
    The internal energy $E$ is 
    \begin{equation*}
        E = \frac{d}{2} N k_B T ~.
    \end{equation*}
    \begin{proof}
        By~\eqref{can:e}
        \begin{equation*}
        \begin{aligned}
            E & = - \pdv{\ln Z}\beta \\ & = - \pdv{}{\beta} N (1 - \ln (n \lambda_T^d)) \\ & = - Nd \pdv{}{\beta} \ln (\lambda_T) \\ & = - Nd \pdv{}{\beta} \ln (\beta^{1/2}) \\ & = \frac{Nd}{2} \frac{1}{\beta} \\ & = \frac{d}{2} N k_B T ~.
        \end{aligned}
        \end{equation*}
    \end{proof}
    
    The Helmoltz free energy $F$ is 
    \begin{equation*}
        F = \frac{N}{\beta} (\ln (n \lambda_T^d) - 1) ~.
    \end{equation*}
    \begin{proof}
        By~\eqref{can:f}
        \begin{equation*}
            F = - \frac{\ln Z}{\beta} = \frac{N}{\beta} (\ln (n \lambda_T^d) - 1) ~.
        \end{equation*}
    \end{proof}
    
    The entropy $S$ is 
    \begin{equation*}
        S = N k_B \Big ( \frac{d+2}{2} - \ln (n \lambda_T^d) \Big ) ~.
    \end{equation*}
    \begin{proof}
        By~\eqref{can:s}
        \begin{equation*}
        \begin{aligned}
            S & = \frac{E - F}{T} \\ & = \frac{1}{T} \Big ( \frac{d}{2} N k_B T - \frac{N}{\beta} (\ln (n \lambda_T^d) - 1) \Big ) \\ & = \frac{N}{\beta T} \Big ( \frac{d+2}{2} - \ln (n \lambda_T^d) \Big ) \\ & = N k_B \Big ( \frac{d+2}{2} - \ln (n \lambda_T^d) \Big )
        \end{aligned}
        \end{equation*}
    \end{proof}

    Entropy becomes negative at a certain critical temperature
    \begin{equation*}
        T_c = \frac{2 m \pi k_B}{h^2} e^{(d+2)/2} n^{-2/d} ~.
    \end{equation*}
    \begin{proof}
        In fact, $S < 0$ for 
        \begin{equation*}
            N k_B \Big ( \frac{d+2}{2} - \ln (n \lambda_T^d) \Big ) < 0 ~,
        \end{equation*}
        \begin{equation*}
            \frac{d+2}{2} - \ln (n \lambda_T^d) < 0 ~,
        \end{equation*}
        \begin{equation*}
            \frac{d+2}{2} < \ln (n \lambda_T^d) ~,
        \end{equation*}
        \begin{equation*}
            e^{(d+2)/2} < n \lambda_T^d  ~,
        \end{equation*}
        \begin{equation*}
            e^{(d+2)/2} < n \Big ( \frac{h^2 \beta}{2 m \pi} \Big )^{d/2}  ~,
        \end{equation*}
        \begin{equation*}
            e^{(d+2)/d} n^{2/d} < \frac{h^2 \beta}{2 m \pi} ~,
        \end{equation*}
        \begin{equation*}
            \frac{2 m \pi}{h^2} e^{(d+2)/2} n^{-2/d} < \beta ~,
        \end{equation*}
        hence 
        \begin{equation*}
            T < \frac{2 m \pi k_B}{h^2} e^{(d+2)/2} n^{-2/d} = T_c ~.
        \end{equation*}
    \end{proof}
    
    The equation of state is 
    \begin{equation}\label{ides}
        p V = N k_B T ~.
    \end{equation}
    \begin{proof}
        By\eqref{fes}
        \begin{equation*}
            p = - \pdv{F}{V} = - \pdv{}{V} \frac{N}{\beta} (\ln (n \lambda_T^d) - 1) = \frac{N}{\beta} \pdv{}{V} \ln V = \frac{N}{V \beta} ~,
        \end{equation*}
        hence 
        \begin{equation*}
            p V = N k_B T ~.
        \end{equation*}
    \end{proof}
    
    The chemical potential $\mu$ is 
    \begin{equation*}
        \mu = frac{1}{\beta} \ln (n \lambda_T^d) ~.
    \end{equation*}
    \begin{proof}
        By~\eqref{fes}
        \begin{equation*}
            \mu = \pdv{F}{N} = \pdv{}{N} \frac{N}{\beta} (\ln (n \lambda_T^d) - 1) = \frac{1}{\beta} (\ln (n \lambda_T^d) - \cancel{1}) + \cancel{\frac{1}{\beta} } = frac{1}{\beta} \ln (n \lambda_T^d) ~.
        \end{equation*}
    \end{proof}

    The specific heats $C_V$ and $C_p$ are 
    \begin{equation*}
        C_V = N \frac{d}{2} k_B ~, \quad C_p = N \frac{d+2}{2} k_B ~. 
    \end{equation*}
    \begin{proof}
        At $V$ constant
        \begin{equation*}
            C_V = \pdv{E}{T} = \pdv{}{T} \frac{d}{2} N k_B T = N \frac{d}{2} k_B ~.
        \end{equation*}

        At $p$ constant, using~\eqref{ides}
        \begin{equation*}
            C_p = C_V + p \pdv{V}{T} = = C_V + p \pdv{}{T} \frac{N k_B T}{p} = N \frac{d}{2} k_B + N k_B = \frac{d + 2}{2} k_B ~.
        \end{equation*}
    \end{proof}

\section{Non-relativistic ideal gas in 3-dimensions}

    Now, consider the case in which $d = 3$.
    For indistinguishable particles, the canonical partition function $Z$ is 
    \begin{equation*}
        Z = \frac{V^N}{N! \lambda^{3N}_T} = \frac{V^N}{N!} (\frac{2 m \pi}{\beta h^2})^{3N/2} ~.
    \end{equation*}
    
    The internal energy $E$ is 
    \begin{equation*}
        E = \frac{3}{2} N k_B T ~.
    \end{equation*}
    
    The Helmoltz free energy $F$ is 
    \begin{equation*}
        F = \frac{N}{\beta} (\ln (n \lambda_T^3) - 1) ~.
    \end{equation*}
    
    The entropy $S$ is 
    \begin{equation*}
        S = N k_B \Big ( \frac{5}{2} - \ln (n \lambda_T^3) \Big ) ~.
    \end{equation*}
    
    Entropy becomes negative at a certain critical temperature
    \begin{equation*}
        T_c = \frac{2 m \pi k_B}{h^2} e^{3/2} n^{-2/3} ~.
    \end{equation*}
    A plot of this is in Figure~\ref{can:ent}.
    \begin{figure}
        \centering
        \scalebox{0.7}{\pyc{plot1('x', '5/2 - log(1 / x**(3/2))', 1, 3, 0, True, False, False)}}
        \caption{A plot of the entropy $S$ as a function of $T$. We have used $x = \frac{2 \pi m k_B T n^{2/3}}{h^2}$ and $f(x) = \frac{S}{N k_B}$.}
        \label{can:ent}
    \end{figure}
    
    The equation of state is 
    \begin{equation*}
        p V = N k_B T ~.
    \end{equation*}
    
    The chemical potential $\mu$ is 
    \begin{equation*}
        \mu = \frac{1}{\beta} \ln (n \lambda_T^3) ~.
    \end{equation*}
    A plot of this is in Figure~\ref{can:mu}.
    \begin{figure}
        \centering
        \scalebox{0.7}{\pyc{plot1('x', 'x * log(1 / x**(3/2)) ', 2, 3, 1, True, False, False)}}
        \caption{A plot of the chemical potential $\mu$ as a function of $T$. We have used $x = \frac{2 \pi m k_B T n^{2/3}}{h^2}$ and $f(x) = \frac{2 \pi m \mu}{h^2 n^{3/2}}$.}
        \label{can:mu}
    \end{figure}
    
    The specific heats $C_V$ and $C_p$ are 
    \begin{equation*}
        C_V = N \frac{3}{2} k_B ~, \quad C_p = N \frac{5}{2} k_B ~. 
    \end{equation*}

    Notice that there are two problems: entropy cannot be negative and the specific heat $C_V \rightarrow 0$ for $T \rightarrow 0$, by thermodynamics. This means that this model is not correct and we must go quantum.
    
\section{Gas of harmonic oscillators in d-dimensions}

    Consider a distinguishable non-relativistic (non-interacting) gas of $N$ particles in an $d$-dimensional manifold confined by an harmonic potential of frequency $\omega$. Its hamiltonian is 
    \begin{equation*}
        H = \sum_i \Big ( \frac{p^2_i}{2m} + \frac{m \omega^2}{2} q_i^2 \Big ) ~.
    \end{equation*}

    The canonical partition function $Z$ is 
    \begin{equation*}
        Z = \frac{1}{\xi_N} \Big (\frac{2 \pi k_B T}{h \omega})^{dN} ~.
    \end{equation*}
    \begin{proof}
        By definition, using the gaussian integral (See appendix 3),
        \begin{equation*}
        \begin{aligned}
            Z & = \int_{\mathcal M^N} d\Omega \exp(- \beta H (q_i, p_i)) \\ & = \int_{\mathcal M^N} \frac{\prod_i d^d q_i d^d p_i}{h^{dN} \xi_N} \exp(- \beta H (q_i, p_i)) \\ & = \frac{1}{h^{dN} \xi_N} \int_{\mathcal M^N} \prod_i d^d q_i d^d p_i \exp(- \beta H (q_i, p_i)) \\ & = \frac{1}{h^{dN} \xi_N} \underbrace{\int_{ V^N} \prod_i d^d q_i \exp(- \beta \frac{m \omega^2}{2} q_i^2)}_{(\frac{2 \pi}{m \omega \beta})^{dN/2}} \underbrace{\prod_i \int_{\mathcal M^N} d^d p_i \exp(- \beta \frac{p^2_i}{2m})}_{(\frac{2 m \pi}{\beta})^{dN/2}} \\ & = \frac{1}{h^{dN} \xi_N} (\frac{2 \pi}{m \omega \beta})^{dN/2} (\frac{2 m \pi}{\beta})^{dN/2} \\ & = \frac{1}{\xi_N} \Big (\frac{2\pi}{h \omega \beta} \Big )^{dN} \\ & = \frac{1}{\xi_N} \Big (\frac{2 \pi k_B T}{h \omega} \Big )^{dN} ~.
        \end{aligned}
        \end{equation*}
    \end{proof}

    For distinguishable particles, the canonical partition function $Z$ is 
    \begin{equation*}
        Z = \Big (\frac{2 \pi k_B T}{h \omega} \Big )^{dN} = (Z_1)^N ~.
    \end{equation*}

    An useful intermediary formula is 
    \begin{equation*}
        \ln Z = d N \ln \frac{2 \pi k_B T}{h \omega} ~.
    \end{equation*}
    \begin{proof}
        In fact, using the Stirling approximation (See Appendix 2),
        \begin{equation*}
            \ln Z = \ln \Big (\frac{2 \pi k_B T}{h \omega} \Big )^{dN} = d N \ln \frac{2 \pi k_B T}{h \omega} ~.
        \end{equation*}
    \end{proof}
    
    The internal energy $E$ is 
    \begin{equation*}
        E = d N k_B T ~.
    \end{equation*}
    \begin{proof}
        By~\eqref{can:e}
        \begin{equation*}
            E = - \pdv{\ln Z}\beta = - \pdv{}{\beta} d N \ln \frac{2 \pi}{h \omega \beta} = d N \frac{1}{\beta} = d N k_B T ~.
        \end{equation*}
    \end{proof}
    
    The Helmoltz free energy $F$ is 
    \begin{equation*}
        F = \frac{dN}{\beta} \ln \frac{h \omega}{2 \pi k_B T} ~.
    \end{equation*}
    \begin{proof}
        By~\eqref{can:f}
        \begin{equation*}
            F = - \frac{\ln Z}{\beta} = - \frac{dN}{\beta} \ln \frac{2 \pi k_B T}{h \omega} = \frac{dN}{\beta} \ln \frac{h \omega}{2 \pi k_B T} ~.
        \end{equation*}
    \end{proof}
    
    The entropy $S$ is 
    \begin{equation*}
        S = d N k_B (1 - \ln \frac{h \omega}{2 \pi k_B T}) ~.
    \end{equation*}
    \begin{proof}
        By~\eqref{can:s}
        \begin{equation*}
            S = \frac{E - F}{T} = \frac{1}{T} \Big ( d N k_B T - \frac{dN}{\beta} \ln \frac{h \omega}{2 \pi k_B T} \Big ) = d N k_B (1 - \ln \frac{h \omega}{2 \pi k_B T}) ~.
        \end{equation*}
    \end{proof}

    Entropy becomes negative at a certain critical temperature
    \begin{equation*}
        T_c = \frac{h \omega}{2 \pi k_B e} ~.
    \end{equation*}
    \begin{proof}
        In fact, $S < 0$ for 
        \begin{equation*}
            d N k_B (1 - \ln \frac{h \omega}{2 \pi k_B T}) < 0 ~,
        \end{equation*}
        \begin{equation*}
            1 - \ln \frac{h \omega}{2 \pi k_B T} < 0 ~,
        \end{equation*}
        \begin{equation*}
            1 < \ln \frac{h \omega}{2 \pi k_B T} ~,
        \end{equation*}
        \begin{equation*}
            e < \frac{h \omega}{2 \pi k_B T}  ~,
        \end{equation*}
        hence 
        \begin{equation*}
            T < \frac{h \omega}{2 \pi k_B e} = T_c ~.
        \end{equation*}
    \end{proof}
    
    The equation of state is 
    \begin{equation}\label{idesharm}
        p = 0 ~.
    \end{equation}
    \begin{proof}
        By\eqref{fes}
        \begin{equation*}
            p = - \pdv{F}{V} = 0 ~.
        \end{equation*}
    \end{proof}
    
    The chemical potential $\mu$ is 
    \begin{equation*}
        \mu = \frac{d}{\beta} \ln \frac{h \omega}{2 \pi k_B T} ~.
    \end{equation*}
    \begin{proof}
        By~\eqref{fes}
        \begin{equation*}
            \mu = \pdv{F}{N} = \pdv{}{N} \frac{dN}{\beta} \ln \frac{h \omega}{2 \pi k_B T} = \frac{d}{\beta} \ln \frac{h \omega}{2 \pi k_B T} ~.
        \end{equation*}
    \end{proof}

    The specific heats $C_V$ and $C_p$ are 
    \begin{equation*}
        C_V = d N k_B  ~, \quad C_p = d N k_B ~. 
    \end{equation*}
    \begin{proof}
        At $V$ constant
        \begin{equation*}
            C_V = \pdv{E}{T} = \pdv{}{T} d N k_B T  = d N k_B ~.
        \end{equation*}

        At $p$ constant, using~\eqref{idesharm}
        \begin{equation*}
            C_p = C_V + p \pdv{V}{T} = = C_V + p \pdv{}{T} \frac{N k_B T}{p} = C_V = d N k_B  ~.
        \end{equation*}
    \end{proof}

\section{Gas of harmonic oscillators in 1-dimension}

    Now, consider the case in which $d = 1$.
    For distinguishable particles, the canonical partition function $Z$ is 
    \begin{equation*}
        Z = \Big (\frac{2 \pi k_B T}{h \omega} \Big )^{N} = (Z_1)^N ~.
    \end{equation*}
    
    The internal energy $E$ is 
    \begin{equation*}
        E = N k_B T ~.
    \end{equation*}
   
    The Helmoltz free energy $F$ is 
    \begin{equation*}
        F = \frac{N}{\beta} \ln \frac{h \omega}{2 \pi k_B T} ~.
    \end{equation*}
    
    The entropy $S$ is 
    \begin{equation*}
        S = N k_B (1 - \ln \frac{h \omega}{2 \pi k_B T}) ~.
    \end{equation*}
 
    Entropy becomes negative at a certain critical temperature
    \begin{equation*}
        T_c = \frac{h \omega}{2 \pi k_B e} ~.
    \end{equation*}
    A plot of this is in Figure~\ref{can:ent2}.
    \begin{figure}
        \centering
        \scalebox{0.7}{\pyc{plot1('x', '1 - log(1 / x)', 2, 3, 2, True, False, False)}}
        \caption{A plot of the entropy $S$ as a function of $T$. We have used $x = \frac{2 \pi k_B T}{h \omega}$ and $f(x) = \frac{S}{N k_B}$.}
        \label{can:ent2}
    \end{figure}

    The equation of state is 
    \begin{equation*}
        p = 0 ~.
    \end{equation*}
    
    The chemical potential $\mu$ is 
    \begin{equation*}
        \mu = \frac{1}{\beta} \ln \frac{h \omega}{2 \pi k_B T} ~.
    \end{equation*}
    A plot of this is in Figure~\ref{can:mu2}.
    \begin{figure}
        \centering
        \scalebox{0.7}{\pyc{plot1('x', 'x * log(1 / x)', 3, 3, 3, True, False, False)}}
        \caption{A plot of the chemical potential $\mu$ as a function of $T$. We have used $x = \frac{2 \pi k_B T}{h \omega}$ and $f(x) = \frac{2 \pi \mu}{h \omega}$.}
        \label{can:mu2}
    \end{figure}

    The specific heats $C_V$ and $C_p$ are 
    \begin{equation*}
        C_V = N k_B  ~, \quad C_p = N k_B ~. 
    \end{equation*}

    Notice that also here there are two problems: entropy cannot be negative and the specific heat $C_V \rightarrow 0$ for $T \rightarrow 0$, by thermodynamics. This means that this model is not correct and we must go quantum.

\section{Ultra-relativistic ideal gas}

    Consider an indistinguishable ultra-relativistic ideal (non-interacting) gas of $N$ particles in an $d$-dimensional manifold with a finite volume $V^N$: $\mathcal M_N = V^N \times \mathbb R^{dN}$. Its hamiltonian is 
    \begin{equation*}
        H = \sum_i c |p_i| ~.
    \end{equation*}

    The canonical partition function $Z$ is 
    \begin{equation*}
        Z = \frac{1}{\xi_N} \Big (\frac{8\pi V}{(\beta h c)^3} \Big )^N ~.
    \end{equation*}
    \begin{proof}
        By definition, using the gaussian integral (See appendix 3),
        \begin{equation*}
        \begin{aligned}
            Z & = \int_{\mathcal M^N} d\Omega \exp(- \beta H (q_i, p_i)) \\ & = \int_{\mathcal M^N} \frac{\prod_i d^3 q_i d^3 p_i}{h^{3N} \xi_N} \exp(- \beta H (q_i, p_i)) \\ & = \frac{1}{h^{3N} \xi_N} \int_{\mathcal M^N} \prod_i d^3 q_i d^3 p_i \exp(- \beta H (q_i, p_i)) \\ & = \frac{1}{h^{3N} \xi_N} \underbrace{\int_{ V^N} \prod_i d^d q_i}_{V^N} \prod_i \int_{\mathcal M^N} d^d p_i \exp(- \beta c p_i) \\ & = \frac{V^N}{h^{3N} \xi_N} \prod_i \int_{\mathcal M^N} d^d p_i \exp(- \beta c p_i) ~.
        \end{aligned}
        \end{equation*}

        Now, in order to evaluate the integral, we use the polar coordinates in the momentum space $(p, \theta, \phi)$
        \begin{equation*}
            \prod_i \int_{\mathcal M^N} d^3 p_i \exp(- \beta c p_i) = \prod_i 4 \pi \int_0^\infty dp ~ p^2 \exp(- \beta c p_i)
        \end{equation*}
        We change variable 
        \begin{equation*}
            z = \beta c p ~, \quad dz = - \beta c dp ~,
        \end{equation*}
        and we find 
        \begin{equation*}
            \prod_i \frac{4\pi}{(\beta c)^3} \underbrace{\int_0^\infty dz ~ z^2 \exp(- z)}_{\Gamma (3)} = \prod_i \frac{4\pi}{(\beta c)^3} \underbrace{\Gamma (3)}_2 = \prod_i \frac{8\pi}{(\beta c)^3} = \Big (\frac{8\pi}{(\beta c)^3} \Big )^N ~.
        \end{equation*}

        Therefore 
        \begin{equation*}
            Z = \frac{V^N}{h^{3N} \xi_N} \Big (\frac{8\pi}{(\beta c)^3} \Big )^N = \frac{1}{\xi_N} \Big (\frac{8\pi V}{(\beta h c)^3} \Big )^N ~.
        \end{equation*}
    \end{proof}

    For indistinguishable particles, the canonical partition function $Z$ is 
    \begin{equation*}
        Z = \frac{1}{N!} \Big (\frac{8\pi V}{(\beta h c)^3} \Big )^N ~.
    \end{equation*}

    An useful intermediary formula is 
    \begin{equation*}
        \ln Z = N (1 - \ln \frac{n (\beta h c)^3}{8\pi}) ~.
    \end{equation*}
    \begin{proof}
        In fact, using the Stirling approximation (See Appendix 2),
        \begin{equation*}
        \begin{aligned}
            \ln Z & = \ln \frac{1}{N!} \Big (\frac{8\pi V}{(\beta h c)^3} \Big )^N \\ & = - \underbrace{\ln N!}_{N \ln N - N} + N \ln \frac{8\pi V}{(\beta h c)^3} \\ & = N (1 - \ln \frac{N (\beta h c)^3}{8\pi V}) \\ & = N (1 - \ln \frac{n (\beta h c)^3}{8\pi})  ~,
        \end{aligned}
        \end{equation*}
        where we have defined the density
        \begin{equation*}
            n = \frac{N}{V} ~.
        \end{equation*}
    \end{proof}
    
    The internal energy $E$ is 
    \begin{equation*}
        E = 3 N k_B T ~.
    \end{equation*}
    \begin{proof}
        By~\eqref{can:e}
        \begin{equation*}
            E = - \pdv{\ln Z}{\beta} = - \pdv{}{\beta} N (1 - \ln \frac{n (\beta h c)^3}{8\pi}) = N \pdv{}{\beta} \ln (\beta^3) = 3 N \frac{\beta^2}{\beta^3} = 3 N \frac{1}{\beta} = 3 N k_B T ~.
        \end{equation*}

        As an aside, it can be also derived from the generalised equipartion theorem~\eqref{equi}. In fact
        \begin{equation*}
            k_B T = \av{p_i \pdv{H}{p_i}} = \av{p_i \pdv{}{p_i} c \sqrt{p_1^2 + p_2^2 + p_3^2}} = \av{c \frac{p_i^2}{\sqrt{p_1^2 + p_2^2 + p_3^2}}} ~,
        \end{equation*}
        hence 
        \begin{equation*}
            \av{H} = \av{c \frac{p_1^2 + p_2^2 + p_3^2}{\sqrt{p_1^2 + p_2^2 + p_3^2}}} = \sum_{i=1}^3 \underbrace{\av{c \frac{p_i^2}{\sqrt{p_1^2 + p_2^2 + p_3^2}}}}_{k_B T} = 3 k_B T ~.
        \end{equation*}
    \end{proof}
    
    The Helmoltz free energy $F$ is 
    \begin{equation*}
        F = \frac{N}{\beta} (\ln \frac{n (\beta h c)^3}{8\pi} - 1) ~.
    \end{equation*}
    \begin{proof}
        By~\eqref{can:f}
        \begin{equation*}
            F = - \frac{\ln Z}{\beta} = \frac{N}{\beta} (\ln \frac{n (\beta h c)^3}{8\pi} - 1) ~.
        \end{equation*}
    \end{proof}
    
    The entropy $S$ is 
    \begin{equation*}
        S = N k_B (4 - \ln \frac{n (\beta h c)^3}{8\pi} ) ~.
    \end{equation*}
    \begin{proof}
        By~\eqref{can:s}
        \begin{equation*}
        \begin{aligned}
            S & = \frac{E - F}{T} \\ & = \frac{1}{T} \Big ( 3 N k_B T - \frac{N}{\beta} (\ln \frac{n (\beta h c)^3}{8\pi} - 1)  \Big ) \\ & = 3 N k_B - N k_B (\ln \frac{n (\beta h c)^3}{8\pi} - 1) \\ & = N k_B (4 - \ln \frac{n (\beta h c)^3}{8\pi} ) ~.
        \end{aligned}
        \end{equation*}
    \end{proof}

    Entropy becomes negative at a certain critical temperature
    \begin{equation*}
        T_c = \frac{hc}{k_B} \Big (\frac{n}{8\pi e^4} \Big)^{1/3} ~.
    \end{equation*}
    \begin{proof}
        In fact, $S < 0$ for 
        \begin{equation*}
            N k_B (4 - \ln \frac{n (\beta h c)^3}{8\pi} ) < 0 ~,
        \end{equation*}
        \begin{equation*}
            4 - \ln \frac{n (\beta h c)^3}{8\pi} < 0 ~,
        \end{equation*}
        \begin{equation*}
            4 < \ln \frac{n (\beta h c)^3}{8\pi} ~,
        \end{equation*}
        \begin{equation*}
            e^{4} < \frac{n (\beta h c)^3}{8\pi} ~,
        \end{equation*}
        \begin{equation*}
            e^{4} < \frac{n (h c)^3}{8\pi k_B^3 T^3}  ~,
        \end{equation*}
        \begin{equation*}
            T^3 < \frac{n (h c)^3}{8\pi k_B^3 e^4} ~,
        \end{equation*}
        hence 
        \begin{equation*}
            T < \frac{hc}{k_B} \Big (\frac{n}{8\pi e^4} \Big)^{1/3} = T_c ~.
        \end{equation*}
    \end{proof}
    A plot of this is in Figure~\ref{can:ent3}.
    \begin{figure}
        \centering
        \scalebox{0.7}{\pyc{plot1('x', '4 - log(1 / x**3)', 3, 5, 4, True, False, False)}}
        \caption{A plot of the entropy $S$ as a function of $T$. We have used $x = \frac{(8 \pi)^{1/3} k_B T}{h c n^{1/3}}$ and $f(x) = \frac{S}{N k_B}$.}
        \label{can:ent3}
    \end{figure}
    
    The equation of state is 
    \begin{equation}\label{idesultra}
        p V = N k_B T ~.
    \end{equation}
    \begin{proof}
        By\eqref{fes}
        \begin{equation*}
            p = - \pdv{F}{V} = - \pdv{}{V} \frac{N}{\beta} (\ln \frac{n (\beta h c)^3}{8\pi} - 1) = \frac{N}{\beta} \pdv{}{V} \ln V = \frac{N}{V \beta} ~,
        \end{equation*}
        hence 
        \begin{equation*}
            p V = N k_B T ~.
        \end{equation*}
    \end{proof}
    
    The chemical potential $\mu$ is 
    \begin{equation*}
        \mu = \frac{1}{\beta} \ln \frac{n (\beta h c)^3}{8\pi} ~.
    \end{equation*}
    \begin{proof}
        By~\eqref{fes}
        \begin{equation*}
            \mu = \pdv{F}{N} = \pdv{}{N} \frac{N}{\beta} (\ln \frac{n (\beta h c)^3}{8\pi} - 1) = \frac{1}{\beta} (\ln \frac{n (\beta h c)^3}{8\pi} - 1 + 1) = \frac{1}{\beta} \ln \frac{n (\beta h c)^3}{8\pi} ~.
        \end{equation*}
    \end{proof}
    A plot of this is in Figure~\ref{can:mu3}.
    \begin{figure}
        \centering
        \scalebox{0.7}{\pyc{plot1('x', 'x * log(1 / x**3)', 3, 4, 5, True, False, False)}}
        \caption{A plot of the chemical potential $\mu$ as a function of $T$. We have used $x = \frac{(8 \pi)^{1/3} k_B T}{h c n^{1/3}}$ and $f(x) = \frac{(8 \pi)^{1/3} \mu}{h c n^{1/3}}$.}
        \label{can:mu3}
    \end{figure}

    The specific heats $C_V$ and $C_p$ are 
    \begin{equation*}
        C_V = 3 N k_B ~, \quad C_p = 4 N k_B T ~. 
    \end{equation*}
    \begin{proof}
        At $V$ constant
        \begin{equation*}
            C_V = \pdv{E}{T} = \pdv{}{T} 3 N k_B T = 3 N k_B ~.
        \end{equation*}

        At $p$ constant, using~\eqref{idesultra}
        \begin{equation*}
            C_p = C_V + p \pdv{V}{T} =  C_V + p \pdv{}{T} \frac{N k_B T}{p} = 3 N k_B + N k_B = 4 N k_B T ~.
        \end{equation*}
    \end{proof}

\section{Maxwell-Boltzmann velocity distribution}

    Consider a non-relativistic ideal (non-interacting) gas of $N$ particles in an $3$-dimensional manifold $\mathcal M^N = \mathbb R^6$, confined into a potential $V(q_i)$ . In this discussion we put $h = 1$. Its hamiltonian is 
    \begin{equation*}
        H = \sum_i \Big ( \frac{p^2_i}{2m} + V(q_i) \Big ) ~.
    \end{equation*}

    The probability distribution density $\rho_c$ for each particle is 
    \begin{equation*}
        \rho_c (q_i, p_i) = \frac{\exp (- \beta ( \frac{p^2_i}{2m} + V(q_i) ))}{(\frac{2\pi m}{\beta})^{3/2} \int_{\mathbb R^3} d^3 q \exp(- \beta V(q))} ~.
    \end{equation*}
    \begin{proof}
        By definition, 
        \begin{equation*}
            \rho_c (q_i, p_i) = \mathcal N \exp (- \beta ( \frac{p^2_i}{2m} + V(q_i) )) ~,
        \end{equation*}
        where the normalisation constant is, using the gaussian integral (see appendix 3) 
        \begin{equation*}
        \begin{aligned}
            1 & = \int_{\mathbb R^6} \prod_i d^3 q ~ d^3 p \mathcal N \exp(- \beta ( \frac{p^2}{2m} + V(q))) \\ & = \mathcal N \int_{\mathbb R^3} d^3 q ~ \exp(- \beta V(q))  \underbrace{\int_{\mathbb R^3} d^3 p ~ \exp(- \beta \frac{p^2}{2m})}_{\Big ( \frac{2\pi m}{\beta}\Big)^{3/2}} \\ & = \mathcal N  \Big ( \frac{2\pi m}{\beta}\Big)^{3/2} \int_{\mathbb R^3} d^3 q ~ \exp(- \beta V(q)) ~,
        \end{aligned}
        \end{equation*}
        hence 
        \begin{equation*}
            \mathcal N = \Big ( \Big ( \frac{2\pi m}{\beta}\Big)^{3/2} \int_{\mathbb R^3} d^3 q ~ \exp(- \beta V(q))  \Big )^{-1} ~.
        \end{equation*}
    \end{proof}

    The marginal probability density distribution is 
    \begin{equation*}
        \rho(q_i) = \frac{\exp (- \beta V(q_i) )}{\int_{\mathbb R^3} d^3 q \exp(- \beta V(q))} ~.
    \end{equation*}
    \begin{proof}
        By definition, 
        \begin{equation*}
        \begin{aligned}
            \rho(q_i) & = \int_{\mathbb R^3} d^3 p ~ \rho_c (q_i, p) \\ & = \int_{\mathbb R^3} d^3 p ~ \frac{\exp (- \beta ( \frac{p^2}{2m} + V(q_i) ))}{(\frac{2\pi m}{\beta})^{3/2} \int_{\mathbb R^3} d^3 q \exp(- \beta V(q))} \\ & = \frac{\exp (- \beta V(q_i) )}{\int_{\mathbb R^3} d^3 q \exp(- \beta V(q))} \frac{\cancel{\int_{\mathbb R^3} d^3 p ~ \exp(- \beta \frac{p^2}{2m})}}{\cancel{(\frac{2\pi m}{\beta})^{3/2}}} \\ & = \frac{\exp (- \beta V(q_i) )}{\int_{\mathbb R^3} d^3 q \exp(- \beta V(q))} ~.
        \end{aligned}
        \end{equation*}
    \end{proof}

    If we have a potential defined as 
    \begin{equation*}
        V (q) = \begin{cases}
            0 & \textnormal{inside a region } \mathcal A\\
            \infty & \textnormal{outside a region } \mathcal A\\
        \end{cases} ~,
    \end{equation*}
    the probability is null outside this region and uniform inside it. 
    \begin{proof}
        In fact 
        \begin{equation*}
            \rho(q_i) = \frac{1}{\int_{\mathcal A} d^3 q} = \frac{1}{\mathcal A} ~. 
        \end{equation*}
    \end{proof}

    The momentum probability density distribution is 
    \begin{equation*}
        \rho(p) = (2\pi m k_B T)^{-3/2} \exp(- \beta \frac{p^2}{2m}) = \prod_i (2\pi m k_B T)^{-1/2} \exp(- \beta \frac{p^2_i}{2m}) ~.
    \end{equation*}
    \begin{proof}
        By definition, 
        \begin{equation*}
        \begin{aligned}
            \rho(p) & = \int_{\mathbb R^3} d^3 q ~ \rho_c (q, p) \\ & = \int_{\mathbb R^3} d^3 q ~ \frac{\exp (- \beta ( \frac{p^2}{2m} + V(q) ))}{(\frac{2\pi m}{\beta})^{3/2} \int_{\mathbb R^3} d^3 q' \exp(- \beta V(q'))} \\ & = \frac{\exp(- \beta \frac{p^2}{2m})}{(\frac{2\pi m}{\beta})^{3/2}} \frac{\cancel{\int_{\mathbb R^3} d^3 q ~ \exp (- \beta V(q) )}}{\cancel{\int_{\mathbb R^3} d^3 q' \exp(- \beta V(q'))}} \\ & = \frac{\exp(- \beta \frac{p^2}{2m})}{(\frac{2\pi m}{\beta})^{3/2}} \\ & = (2\pi m k_B T)^{-3/2} \exp(- \beta \frac{p^2}{2m}) \\ & = \prod_i (2\pi m k_B T)^{-1/2} \exp(- \beta \frac{p^2_i}{2m}) ~.
        \end{aligned}
        \end{equation*}
    \end{proof}
    A plot of this is in Figure~\ref{max:mom}.
    \begin{figure}
        \centering
        \scalebox{0.7}{\pyc{plot1('x', 'exp(- x**2)', 3, 2, 6, True, False, False)}}
        \caption{A plot of the momentum probability density distribution. We have used $x = \sqrt{\frac{\beta}{2m}} p$ and $f(x) = (2\pi m k_B T)^{3/2} \rho$.}
        \label{max:mom}
    \end{figure}
    
    The velocity probability density distribution is
    \begin{equation*}
        \rho(p) = (\frac{m}{2\pi k_B T})^{1/2} \exp(- \beta \frac{m v^2_i}{2}) ~.
    \end{equation*}
    \begin{proof}
        With a change of variable 
        \begin{equation*}
            p_i = m v_i ~, \quad \rho(v_i) dv_i = \rho(p_i) dp_i = \rho(p_i) m dv_i ~,
        \end{equation*}
        we find
        \begin{equation*}
            \rho(v_i) = m \rho (p_i) = (\frac{2\pi k_B T}{m})^{-1/2} \exp(- \beta \frac{m^2 v^2_i}{2m}) = (\frac{m}{2\pi k_B T})^{1/2} \exp(- \beta \frac{m v^2_i}{2}) ~.
        \end{equation*}
    \end{proof}

    The velocity modulus probability density distribution is 
    \begin{equation*}
        \rho (v) = (\frac{m}{2\pi k_B T})^{3/2} 4 \pi v^2 \exp(- \beta \frac{m v^2}{2}) ~.
    \end{equation*}
    \begin{proof}
        With a change of variable into the polar coordinates $(v, \theta, \phi)$
        \begin{equation*}
            \rho(v_1, v_2, v_3) dv_1 dv_2 dv_3 = \rho(v_1, v_2, v_3) v^2 \sin \theta d\theta d\phi dv = \rho(\theta, \phi, v) d\theta d\phi dv ~,
        \end{equation*}
        we find
        \begin{equation*}
            \rho(v) = 4 \pi v^2 \prod_i \rho (v_i) = (\frac{m}{2\pi k_B T})^{3/2} 4 \pi v^2 \exp(- \beta \frac{m v^2}{2})  ~.
        \end{equation*}
    \end{proof}
    A plot of this is in Figure~\ref{max:vel}.
    \begin{figure}
        \centering
        \scalebox{0.7}{\pyc{plot1('x', 'x**2 * exp(- x**2)', 3, 0.5, 7, True, True, True)}}
        \caption{A plot of the velocity modulus probability density distribution. We have used $x = \sqrt{\frac{\beta m}{2}} v$ and $f(x) = \rho$.}
        \label{max:vel}
    \end{figure}

    The most probable velocity value is 
    \begin{equation*}
        v_p = \sqrt{\frac{2 k_B T}{m}} ~.
    \end{equation*}
    \begin{proof}
        By definition, 
        \begin{equation*}
            0 = \dv{\rho(v)}{v} = 2 v \exp(- \beta \frac{m v^2}{2}) - \beta m v^3 \exp(- \beta \frac{m v^2}{2}) ~,
        \end{equation*}
        hence 
        \begin{equation*}
            v_p = \sqrt{\frac{2k_B T}{m}} ~.
        \end{equation*}
    \end{proof}

    The mean velocity value is 
    \begin{equation*}
        \av{v} = \sqrt{\frac{8 k_B T}{\pi m}} ~.
    \end{equation*}
    \begin{proof}
        By definition, 
        \begin{equation*}
            \av{v} = \int_{\mathbb R^3} dv ~ \rho(v) v = (\frac{m}{2\pi k_B T})^{3/2} 4 \pi \int_{\mathbb R^3} dv ~ v^3 \exp(- \beta \frac{m v^2}{2}) ~.
        \end{equation*}

        We make a change of variables 
        \begin{equation*}
            t = \frac{m \beta v^2}{2} ~, \quad dt = m \beta v dv ~,
        \end{equation*}
        hence 
        \begin{equation*}
        \begin{aligned}
            \av{v} & =  (\frac{m}{2\pi k_B T})^{3/2} 4 \pi \Big (\frac{2}{m \beta} \Big) \frac{1}{m \beta} \underbrace{\int_0^\infty dt ~ t \exp(- t)}_{\Gamma (2)} \\ & = \sqrt{\frac{8}{m \pi \beta}} \underbrace{\Gamma (2)}_1 = \sqrt{\frac{8}{m \pi \beta}} \\ & = \sqrt{\frac{8 k_B T}{\pi m}} ~.
        \end{aligned}
        \end{equation*}
    \end{proof}

    The mean square velocity value is 
    \begin{equation*}
        \av{v^2} = \frac{3 k_B T}{m} ~.
    \end{equation*}
    \begin{proof}
        By definition, 
        \begin{equation*}
            \av{v^2} = \int_{\mathbb R^3} dv ~ \rho(v) v^2 = (\frac{m}{2\pi k_B T})^{3/2} 4 \pi \int_{\mathbb R^3} dv ~ v^4 \exp(- \beta \frac{m v^2}{2}) ~.
        \end{equation*}

        We make a change of variables 
        \begin{equation*}
            t = \frac{m \beta v^2}{2} ~, \quad dt = m \beta v dv ~,
        \end{equation*}
        hence 
        \begin{equation*}
        \begin{aligned}
            \av{v^2} & =  (\frac{m}{2\pi k_B T})^{3/2} 4 \pi \Big (\frac{2}{m \beta} \Big) \frac{1}{m \beta} \Big ( \frac{2}{m \beta} \Big)^{1/2} \underbrace{\int_0^\infty dt ~ t^{3/2} \exp(- t)}_{\Gamma (5/2)} \\ & = \frac{4}{\sqrt{\pi} m \beta} \underbrace{\Gamma (5/2)}_{\frac{3 \sqrt{\pi}}{4}} \\ & = \frac{3}{\beta m} \\ & = \frac{3 k_B T}{m} ~.
        \end{aligned}
        \end{equation*}
    \end{proof}

\section{Magnetic solid}

    A solid, composed by $N$ atoms/molecules with an intrinsic magnetic moment $\boldsymbol \mu$ in an external magnetic field $\mathbf B$, can be modelled by an hamiltonian 
    \begin{equation*}
        H = - \sum_i \boldsymbol \mu \cdot \mathbf B = - \mu B \sum_i \cos \theta_i ~.
    \end{equation*}
    where the phase space coordinates are $\phi_i$ and $\theta_i$.

    The canonical partition function $Z$ is 
    \begin{equation*}
        Z = \Big (\frac{4 \pi \sinh (\beta \mu B)}{\beta \mu B} \Big )^N ~.
    \end{equation*}
    \begin{proof}
        By definition
        \begin{equation*}
        \begin{aligned}
            Z & = \int_{\mathcal M} d \Omega \exp(- \beta H (\theta_i)) \\ & = \underbrace{\prod_i \int_0^{2\pi} d\phi_i}_{(2\pi)^N} \prod_i \int_0^\pi d\theta_i ~ \sin \theta_i \exp(- \beta \mu B \cos \theta_i) \\ & = (2\pi)^N \prod_i \int_0^\pi d\theta_i ~ \sin \theta_i \exp(- \beta \mu B \cos \theta_i)  ~.
        \end{aligned}
        \end{equation*}
        

        We make a change of variable 
        \begin{equation*}
            x_i  = \cos \theta_i ~, \quad d x_i = - \sin \theta_i d\theta_i ~,
        \end{equation*}
        with extremis 
        \begin{equation*}
            \theta_i = 0 \rightarrow x_i = 1 ~, \quad \theta_i = \pi \rightarrow x_i = 0 ~,
        \end{equation*}
        hence
        \begin{equation*}
        \begin{aligned}
            Z & = (2\pi)^N \prod_i \int_{-1}^1 dx_i ~ \exp(- \beta \mu B x) \\ & = (2\pi)^N \Big ( \frac{\exp(- \beta \mu B x)}{- \beta \mu B} \Big \vert_{-1}^{1} \Big )^N \\ & = (2\pi)^N \Big ( \frac{1}{- \beta \mu B} \underbrace{(\exp(- \beta \mu B) - \exp(\beta \mu B))}_{- 2 \sinh \beta \mu B}\Big )^N \\ & = (2\pi)^N \Big ( \frac{1}{\beta \mu B} (2 \sinh (\beta \mu B)) \Big )^N \\ & = \Big (\frac{4 \pi \sinh (\beta \mu B)}{\beta \mu B} \Big )^N ~.
        \end{aligned}
        \end{equation*}
    \end{proof}

    An useful intermediary formula is 
    \begin{equation*}
        \ln Z = N \ln \frac{4 \pi \sinh (\beta \mu B)}{\beta \mu B}  ~.
    \end{equation*}
    \begin{proof}
        In fact,
        \begin{equation*}
            \ln Z = \ln \Big (\frac{4 \pi \sinh (\beta \mu B)}{\beta \mu B} \Big )^N = N \ln \frac{4 \pi \sinh (\beta \mu B)}{\beta \mu B}  ~.
        \end{equation*}
    \end{proof}
    
    The internal energy $E$ is 
    \begin{equation*}
        E = - N \mu B (\coth (\beta \mu B) + \frac{1}{\beta \mu B} ) ~.
    \end{equation*}
    \begin{proof}
        By~\eqref{can:e}
        \begin{equation*}
        \begin{aligned}
            E & = - \pdv{\ln Z}{\beta} \\ & = - \pdv{}{\beta} N \ln \frac{4 \pi \sinh (\beta \mu B)}{\beta \mu B} \\ & = - N \pdv{}{\beta} \ln \sinh (\beta \mu B) + N \pdv{}{\beta} \ln \beta \\ & = - N \mu B \coth (\beta \mu B) + \frac{N}{\beta} \\ & = - N \mu B (\coth (\beta \mu B) + \frac{1}{\beta \mu B} ) ~.
        \end{aligned}
        \end{equation*}

        To study the limit for $\beta \rightarrow 0$ or $T \rightarrow \infty$, we Taylor expand for the variable $x = \beta \mu B$
        \begin{equation*}
            \lim_{x \rightarrow 0} \frac{E}{N \mu B} (x) \simeq \py{limit('x', '- 1 * ((exp(x) + exp(-x))/((exp(x) - exp(-x)))) + 1 / x', 0)} ~,
        \end{equation*}
        hence 
        \begin{equation*}
            E \xrightarrow{T \rightarrow \infty} 0 ~.
        \end{equation*}
        To study the limit for $\beta \rightarrow \infty$ or $T \rightarrow 0$, we Taylor expand for the variable $x = \beta \mu B$
        \begin{equation*}
            \lim_{x \rightarrow \infty} \frac{E}{N \mu B} (x) \simeq \py{limit('x', '- coth(1 / x) + x', 0)} ~,
        \end{equation*}
        hence 
        \begin{equation*}
            E \xrightarrow{T \rightarrow 0} - N \mu B ~.
        \end{equation*}
    \end{proof}
    A plot of this is in Figure~\ref{can:magen}.
    \begin{figure}
        \centering
        \scalebox{0.7}{\pyc{plot1('x', '- coth(1 / x) + x', 5, 1.2, 10, True, True, False)}}
        \caption{A plot of the internal energy $E$ as a function of $T$. We have used $x = \frac{1}{\beta \mu B}$ and $f(x) = \frac{E}{N \mu B}$.}
        \label{can:magen}
    \end{figure}
    
    The Helmoltz free energy $F$ is 
    \begin{equation*}
        F = - \frac{N}{\beta} \ln \frac{4 \pi \sinh (\beta \mu B)}{\beta \mu B} ~.
    \end{equation*}
    \begin{proof}
        By~\eqref{can:f}
        \begin{equation*}
            F = - \frac{\ln Z}{\beta} = - \frac{N}{\beta} \ln \frac{4 \pi \sinh (\beta \mu B)}{\beta \mu B} ~.
        \end{equation*}
    \end{proof}
    
    The entropy $S$ is 
    \begin{equation*}
        S = N k_B \Big ( \ln \frac{4 \pi \sinh (\beta \mu B)}{\beta \mu B}  - \beta \mu B (\coth (\beta \mu B) - \frac{1}{\beta \mu B} ) \Big ) ~.
    \end{equation*}
    \begin{proof}
        By~\eqref{can:s}
        \begin{equation*}
        \begin{aligned}
            S & = \frac{E - F}{T} = \frac{1}{T} \Big (- N \mu B (\coth (\beta \mu B) + \frac{1}{\beta \mu B} ) + \frac{N}{\beta} \ln \frac{4 \pi \sinh (\beta \mu B)}{\beta \mu B}  \Big ) \\ & = N k_B \Big ( \ln \frac{4 \pi \sinh (\beta \mu B)}{\beta \mu B}  - \beta \mu B (\coth (\beta \mu B) - \frac{1}{\beta \mu B} ) \Big )~. 
        \end{aligned}
        \end{equation*}
    \end{proof}

    The intrinsic magnetic moment $\mathbf M$ is 
    \begin{equation*}
        \mathbf M = (0, 0, N\mu (\coth(\beta \mu B) - \frac{1}{\beta \mu B})) ~.
    \end{equation*}
    \begin{proof}
        By definition, since we have oriented $\mathbf B = (0, 0, B)$,
        \begin{equation*}
            M_x = - \pdv{F}{B_x} = M_y = - \pdv{F}{B_y} = 0 ~,
        \end{equation*}
        but 
        \begin{equation*}
        \begin{aligned}
            M_z & = - \pdv{F}{B} \\ & = \pdv{}{B} \frac{N}{\beta} \ln \frac{4 \pi \sinh (\beta \mu B)}{\beta \mu B} \\ & = \frac{N}{\beta} \pdv{}{\beta} \ln \sinh (\beta \mu B) - \frac{N}{\beta} \pdv{}{B} \ln B \\ & = N \mu \coth(\beta \mu B) - \frac{N}{\beta B} \\ & = N \mu \Big (\coth (\beta \mu B) - \frac{1}{\beta \mu B} \Big ) ~.
        \end{aligned}
        \end{equation*}
    \end{proof}
    A plot of this is in Figure~\ref{can:mag}.
    \begin{figure}
        \centering
        \scalebox{0.7}{\pyc{plot1('x', 'coth(x) - 1 / x', 7, 1, 8, True, True, True)}}
        \caption{A plot of the intrinsic magnetic moment $\mathbf M$ as a function of $\beta$. We have used $x = \beta \mu B$ and $f(x) = \frac{M_z}{N \mu}$.}
        \label{can:mag}
    \end{figure}

    The isothermal susceptibility $\chi_\beta$ is 
    \begin{equation*}
        \chi_\beta = N \mu^2 \beta ( \frac{1}{(\beta \mu B)^2} - \frac{1}{\sinh^2 (\beta \mu H)}) ~.
    \end{equation*}
    \begin{proof}
        By definition
        \begin{equation*}
        \begin{aligned}
            \chi_\beta & = \pdv{M}{B} \\ & = \pdv{}{B} N\mu (\coth(\beta \mu B) - \frac{1}{\beta \mu B}) \\ & = N \mu ( - \frac{\beta \mu}{\sinh^2 (\beta \mu B)} + \frac{\beta \mu}{(\beta \mu B)^2} ) \\ & = N \mu^2 \beta ( \frac{1}{(\beta \mu B)^2} - \frac{1}{\sinh^2 (\beta \mu H)}) ~.
        \end{aligned}
        \end{equation*}
    \end{proof}
    A plot of this is in Figure~\ref{can:sus}.
    \begin{figure}
        \centering
        \scalebox{0.7}{\pyc{plot1('x', 'x * (1 / x**2 - 1 / sinh(x)**2)', 5, 0.5, 9, True, True, True)}}
        \caption{A plot of the intrinsic magnetic moment $\mathbf M$ as a function of $\beta$. We have used $x = \beta \mu B$ and $f(x) = \frac{B \chi_\beta}{N \mu}$.}
        \label{can:sus}
    \end{figure}

    For $T \rightarrow \infty$, the Curie law is 
    \begin{equation*}
        \chi_\beta = \frac{C}{T} ~,
    \end{equation*}
    where the Curie constant is 
    \begin{equation*}
        C = \frac{N \mu^2}{3 k_B} ~.
    \end{equation*}
    \begin{proof}
        To study the limit for $\beta \rightarrow 0$ or $T \rightarrow \infty$, we Taylor expand for the variable $x = \beta \mu B$
        \begin{equation*}
            \frac{B \chi_\beta}{N \mu} (x) \simeq \py{Taylor('x', 'x * ( x**(-2) - sinh(x)**(-2))', 0, 2)} ~,
        \end{equation*}
        hence 
        \begin{equation*}
            \frac{B \chi_\beta}{N \mu} = \frac{\beta \mu B}{3} ~,
        \end{equation*}
        which means 
        \begin{equation*}
            \chi_\beta = \frac{N \mu^2}{3 k_B} \frac{1}{T} = \frac{C}{T} ~.
        \end{equation*}
    \end{proof}

\chapter{Grancanonical ensemble}

\section{Non-relativistic ideal gas in d-dimensions}

    Consider an indistinguishable non-relativistic ideal (non-interacting) gas of $N$ particles in an $d$-dimensional manifold with a finite volume $V^N$: $\mathcal M_N = V^N \times \mathbb R^{dN}$. 

    Recall that the canonical partition function is
    \begin{equation*}
        Z = \frac{V^N}{N! \lambda^{dN}_T} = \frac{V^N}{N!} (\frac{2 m \pi}{\beta h^2})^{dN/2} ~.
    \end{equation*}

    The grancanonical partition function is 
    \begin{equation*}
        \mathcal Z = \exp(\frac{z V}{\lambda_T^d}) ~.
     \end{equation*}
    \begin{proof}
        By definition, using the Taylor expansion of the exponential,
        \begin{equation*}
            \mathcal Z = \sum_{N=0}^\infty z^N Z_N = \sum_{N=0}^\infty \frac{1}{N!} \Big ( \frac{z V}{\lambda_T^d} \Big)^N = \exp(\frac{z V}{\lambda_T^d}) ~.
        \end{equation*}
    \end{proof}
    
    The internal energy $E$ is 
    \begin{equation*}
        E = \frac{z V}{\lambda^d_T} \frac{d}{2 \beta} ~.
    \end{equation*}
    \begin{proof}
        By~\eqref{gran:e}
        \begin{equation*}
        \begin{aligned}
            E & = - \pdv{\ln \mathcal Z}{\beta} \Big \vert_z \\ & = - \pdv{}{\beta} \ln \exp(\frac{z V}{\lambda_T^d}) \\ & = - \pdv{}{\beta} \frac{z V}{\lambda^d_T} \\ & = - \frac{1}{zV} \pdv{}{\beta} \Big ( \frac{2 m \pi} {\beta h^2} \Big )^{d/2} \\ & = - \frac{1}{z V} \Big ( \frac{2 m \pi}{h^2} \Big )^{d/2} \pdv{}{\beta} \beta^{-d/2} \\ & = \frac{1}{zV} \Big ( \frac{h^2}{2 m \pi} \Big )^{d/2} \frac{d}{2} \beta^{-d/2 - 1} \\ & = \frac{z V}{\lambda^d_T} \frac{d}{2 \beta} ~.
        \end{aligned}
        \end{equation*}
    \end{proof}
    
    The number of particle $N$ is 
    \begin{equation*}
        N = \frac{V}{\lambda_T^d} ~.
    \end{equation*}
    \begin{proof}
        By~\eqref{gran:n}
        \begin{equation*}
            N = z \pdv{}{z} \ln \mathcal Z = z \pdv{}{z} \frac{z V}{\lambda_T^d} = \frac{V}{\lambda_T^d} ~.
        \end{equation*}
    \end{proof}
    
    The equation of state is 
    \begin{equation*}
        p = \frac{z}{\beta \lambda_T^d} ~.
    \end{equation*}
    \begin{proof}
        By definition
        \begin{equation*}
            p = \frac{1}{\beta V} \ln \mathcal Z = \frac{1}{\beta V} \frac{z V}{\lambda_T^d} = \frac{z}{\beta \lambda_T^d} ~.
        \end{equation*}
    \end{proof}

\section{Van der Waals potential}

\chapter{Entropy}

\section{Maxwell-Boltzmann distribution}

    We can distribute $N$ particle in $p$ boxes in ways 
    \begin{equation*}
        W^{(1)}_{n_r} = \frac{N!}{n_1! \ldots n_p!} ~,
    \end{equation*}
    whereas there is no restriction for the states 
    \begin{equation*}
        W^{(2)}_{n_r} = \prod_r g_r^{n_r} ~,
    \end{equation*}
    hence 
    \begin{equation*}
        W_{n_r} = \frac{N!}{n_1! \ldots n_p!} \prod_{r=1}^p g_r^{n_r} = N! \prod_{r=1}^p \frac{g_r^{n_r}}{n_r!} ~.
    \end{equation*}

    Maximising the constrained entropy, we find the Boltzmann canonical distribution 
    \begin{equation*}
        p_r^* = \frac{n_r^*}{N} = \frac{g_r \exp(- \beta E_r)}{\sum_r g_r \exp(- \beta E_r)} ~.
    \end{equation*}
    \begin{proof}
        The entropy is, using the Stirling approximation (see Appendix 2),
        \begin{equation*}
        \begin{aligned}
            S & = \ln W_{n_r} \\ & = \ln \Big (N! \prod_{r=1}^p \frac{g_r^{n_r}}{n_r!} \Big) \\ & = \ln N! + \sum_{r=1}^p \ln \frac{g_r^{n_r}}{n_r!} \\ & = \underbrace{\ln N!}_{N \ln N - N} + \sum_{r=1}^p (\ln g_r^{n_r} - \underbrace{\ln n_r!}_{n_r \ln n_r - n_r} ) \\ & = N \ln N - \cancel{N} + \sum_{r=1}^p n_r \ln g_r - \sum_{r=1}^p n_r \ln n_r - \cancel{\sum_{r=1}^p n_r} \\ & = N \ln N + \sum_{r=1}^p n_r \ln g_r + \sum_{r=1}^p n_r \ln n_r ~.
        \end{aligned}
        \end{equation*}

        The constrained entropy is
        \begin{equation*}
            S =  N \ln N + \sum_{r=1}^p n_r \ln g_r - \sum_{r=1}^p n_r \ln n_r + \alpha \Big (N - \sum_{r=1}^p n_r \Big) + \beta \Big (E - \sum_{r=1}^p n_r E_r \Big ) ~.
        \end{equation*}

        The maximum is 
        \begin{equation*}
            0 = \pdv{S}{n_r} = \ln g_r - \ln n_r - 1 - \alpha - \beta E_r ~,
        \end{equation*}
        hence 
        \begin{equation*}
            n_r^* = \frac{g_r \exp(- \beta E_r)}{\exp(1 + \alpha)} ~.
        \end{equation*}

        We find $\alpha$ by the normalisation condition 
        \begin{equation*}
            N = \sum_r n_r^* = \sum_r \frac{g_r \exp(- \beta E_r)}{\exp(1 + \alpha)} ~,
        \end{equation*}
        hence 
        \begin{equation*}
            \exp(1 + \alpha) = \frac{\sum_r g_r \exp(- \beta E_r)}{N} ~.
        \end{equation*}

        Finally, if we identify $\beta = 1/k_B T$ the probability distribution density is
        \begin{equation*}
            p_r^* = \frac{n_r^*}{N} = \frac{g_r \exp(- \beta E_r)}{\sum_r g_r \exp(- \beta E_r)} ~.
        \end{equation*}
    \end{proof}
    A plot of this is in Figure~\ref{en:bol}.
    \begin{figure}
        \centering
        \scalebox{0.7}{\pyc{plot1('x', 'exp(-x)', 5, 10, 11, True, False, True)}}
        \caption{A plot of the probability density distribution $p_r^*$ as a function of $\beta E_r$. We have used $x = \beta E_r$ and $f(x) = p_r^* \frac{\sum_r g_r \exp(- \beta E_r)}{g_r}$.}
        \label{en:bol}
    \end{figure}

\section{Fermi-Dirac distribution}

    For fermions, there is no restriction on how we can distribute $N$ particle in $p$ boxes in ways, since they are indistinguishable
    \begin{equation*}
        W^{(1)}_{n_r} = 1 ~,
    \end{equation*}
    whereas we can distribute $n_r$ objects in $g_r$ boxes
    \begin{equation*}
        W^{(2)}_{n_r} = \prod_r \binom{g_r}{n_r} = \prod_r \frac{g_r!}{n_r! (g_r - n_r)!} ~,
    \end{equation*}
    hence 
    \begin{equation*}
        W_{n_r} = \prod_r \binom{g_r}{n_r} = \prod_r \frac{g_r!}{n_r! (g_r - n_r)!} ~.
    \end{equation*}
    
    Maximising the constrained entropy, we find the Bose-Einstein distribution 
    \begin{equation*}
        n_r^* = \frac{g_r}{\exp(\alpha + \beta E_r) + 1} ~.
    \end{equation*}
    \begin{proof}
        The entropy is, using the Stirling approximation (see Appendix 2),
        \begin{equation*}
        \begin{aligned}
            S & = \ln W_{n_r} \\ & = \ln \Big (\prod_r \frac{g_r!}{n_r! (g_r - n_r)!} \Big) \\ & = \sum_r \Big ( \underbrace{\ln g_r!}_{g_r \ln g_r - g_r} - \underbrace{\ln n_r!}_{n_r \ln n_r - n_r} - \underbrace{\ln (g_r - n_r)!}_{(g_r - n_r) \ln (g_r - n_r) - g_r + n_r} \Big) \\ & = \sum_r \Big ( g_r \ln g_r - \cancel{g_r} - n_r \ln n_r + \cancel{n_r} - (g_r - n_r) \ln (g_r - n_r) + \cancel{g_r} - \cancel{n_r} \Big) \\ & = \sum_r \Big ( g_r \ln g_r - n_r \ln n_r - (g_r - n_r) \ln (g_r - n_r) \Big) 
             ~.
        \end{aligned}
        \end{equation*}
    
        The constrained entropy is
        \begin{equation*}
            S =  \sum_r \Big ( g_r \ln g_r - n_r \ln n_r - (g_r - n_r) \ln (g_r - n_r) \Big) + \alpha \Big (N - \sum_{r=1}^p n_r \Big) + \beta \Big (E - \sum_{r=1}^p n_r E_r \Big ) ~.
        \end{equation*}
    
        The maximum is 
        \begin{equation*}
        \begin{aligned}
            0 & = \pdv{S}{n_r} \\ & = - \ln n_r - \cancel{1} + \ln (g_r - n_r) + \cancel{1} - \alpha - \beta E_r \\ & = - \ln n_r + \ln (g_r - n_r) - \alpha - \beta E_r \\ & = \ln (\frac{g_r}{n_r} - 1) - \alpha - \beta E_r~ ,
        \end{aligned}
        \end{equation*}
        hence 
        \begin{equation*}
            \frac{g_r}{n_r} - 1 = \exp(\alpha + \beta E_r) ~,
        \end{equation*}
        \begin{equation*}
            n_r^* = \frac{g_r}{\exp(\alpha + \beta E_r) + 1} ~.
        \end{equation*}
    \end{proof}
    A plot of this is in Figure~\ref{en:fd}.
    \begin{figure}
        \centering
        \scalebox{0.7}{\pyc{plot1('x', '1 / ( exp(x) + 1)', 5, 2, 12, True, False, True)}}
        \caption{A plot of the Fermi-Dirac distribution $n_r^*$ as a function of $\alpha + \beta E_r$. We have used $x = \alpha + \beta E_r $ and $f(x) = \frac{n_r^*}{g_r}$.}
        \label{en:fd}
    \end{figure}

\section{Bose-Einstein distribution}

    For bosons, there is no restriction on how we can distribute $N$ particle in $p$ boxes in ways, since they are indistinguishable
    \begin{equation*}
        W^{(1)}_{n_r} = 1 ~,
    \end{equation*}
    whereas we can distribute $n_r$ objects in $g_r$ boxes
    \begin{equation*}
        W^{(2)}_{n_r} = \prod_r \binom{n_r + g_r - 1}{n_r} = \prod_r \frac{(n_r + g_r - 1)!}{n_r! (g_r - 1)!} ~,
    \end{equation*}
    hence 
    \begin{equation*}
        W_{n_r} = \prod_r \frac{(n_r + g_r - 1)!}{n_r! (g_r - 1)!} ~.
    \end{equation*}
    
    Maximising the constrained entropy, we find the Bose-Einstein distribution 
    \begin{equation*}
        n_r^* = \frac{g_r}{\exp(\alpha + \beta E_r) - 1} ~.
    \end{equation*}
    \begin{proof}
        The entropy is, using the Stirling approximation (see Appendix 2),
        \begin{equation*}
        \begin{aligned}
            S & = \ln W_{n_r} \\ & = \ln \prod_r \frac{(n_r + g_r - 1)!}{n_r! (g_r - 1)!} \\ & = \sum_r \Big (\underbrace{\ln (n_r + g_r - 1)!}_{(n_r + g_r - 1) \ln (n_r + g_r - 1) - n_r - g_r + 1} - \underbrace{\ln n_r!}_{n_r \ln n_r - n_r} - \underbrace{\ln (g_r - 1)!}_{(g_r - 1) \ln (g_r - 1) - g_r + 1} \Big ) \\ & = \sum_r \Big ( (n_r + g_r - 1) \ln (n_r + g_r - 1) - \cancel{n_r} - \cancel{g_r} + \cancel{1} \\ & \quad - n_r \ln n_r + \cancel{n_r }- (g_r - 1) \ln (g_r - 1) + \cancel{g_r} - \cancel{1} \Big ) \\ & = \sum_r \Big ( (n_r + g_r - 1) \ln (n_r + g_r - 1) \ln n_r - n_r \ln n_r - (g_r - 1) \ln (g_r - 1) \Big ) 
        \end{aligned}
        \end{equation*}
    
        The constrained entropy is
        \begin{equation*}
        \begin{aligned}
            S & = \sum_r \Big ( (n_r + g_r - 1) \ln (n_r + g_r - 1) \ln n_r - n_r \ln n_r - (g_r - 1) \ln (g_r - 1) \Big ) \\ & \quad + \alpha \Big (N - \sum_{r=1}^p n_r \Big) + \beta \Big (E - \sum_{r=1}^p n_r E_r \Big ) ~.
        \end{aligned}
        \end{equation*}
    
        The maximum is 
        \begin{equation*}
        \begin{aligned}
            0 & = \pdv{S}{n_r} \\ & = \ln (n_r + g_r - 1) + \cancel{1} - \ln n_r - \cancel{1} - \alpha - \beta E_r \\ & = \ln (n_r + g_r - 1) - \ln n_r - \alpha - \beta E_r \\ & = \ln (\frac{g_r - 1}{n_r} + 1) - \alpha - \beta E_r~ ,
        \end{aligned}
        \end{equation*}
        hence, for $g_r \gg 1$,
        \begin{equation*}
            \frac{g_r - 1}{n_r} + 1 = \exp(\alpha + \beta E_r) ~,
        \end{equation*}
        \begin{equation*}
            n_r^* = \frac{g_r - 1}{\exp(\alpha + \beta E_r) - 1} \simeq \frac{g_r}{\exp(\alpha + \beta E_r) - 1} ~.
        \end{equation*}
    \end{proof}
    A plot of this is in Figure~\ref{en:be}.
    \begin{figure}
        \centering
        \scalebox{0.7}{\pyc{plot1('x', '1 / ( exp(x) - 1)', 5, 5, 13, True, False, True)}}
        \caption{A plot of the Bose-Einstein distribution $n_r^*$ as a function of $\alpha + \beta E_r$. We have used $x = \alpha + \beta E_r $ and $f(x) = \frac{n_r^*}{g_r}$.}
        \label{en:be}
    \end{figure}

\section{Two-levels system}

    Consider a system composed by $2$ levels of energies $\epsilon_+ = + \epsilon$ and $\epsilon_- = - \epsilon$. The constrains are 
    \begin{equation*}
        E = \epsilon (n_+ - n_-) ~, \quad N = n_+ + n_- ~.
    \end{equation*}
    They can be inverted as
    \begin{equation}
        n_+ = \frac{N}{2} + \frac{E}{2\epsilon} ~, \quad n_+ = \frac{N}{2} - \frac{E}{2\epsilon} ~.
    \end{equation}

    We can distribute $N$ objects in $n_+$ boxes
    \begin{equation*}
        \Omega(E) = \binom{N}{n_+} = \frac{N!}{n_+! (N - n_+)!}  = \frac{N!}{n_+! n_-!} ~.
    \end{equation*}
    
    The entropy is 
    \begin{equation*}
        S = - N k_B \Big ( (\frac{1}{2} + \frac{E}{2\epsilon N}) \ln (\frac{1}{2} + \frac{E}{2\epsilon N} ) + (\frac{1}{2} - \frac{E}{2\epsilon N}) \ln (\frac{1}{2} - \frac{E}{2\epsilon N} ) \Big ) ~.
    \end{equation*}
    \begin{proof}
        By definition, using the Stirling approximation (see Appendix 2),
        \begin{equation*}
        \begin{aligned}
            \frac{S}{k_B} & = \ln \Omega (E) \\ & = \ln \frac{N!}{n_+! n_-!} \\ & = \underbrace{\ln N!}_{N \ln N - N} - \underbrace{\ln n_+!}_{n_+ \ln n_+ - n_+} - \underbrace{\ln n_-!}_{n_- \ln n_- - n_-} \\ & = N \ln N - \cancel{N} - n_+ \ln n_+ + \cancel{n_+} - n_- \ln n_- + \cancel{n_-} \\ & = N \ln N - n_+ \ln n_+ - n_- \ln n_- \\ & = (n_+ + n_-) \ln N - n_+ \ln n_+ - n_- \ln n_- \\ & = n_+ \ln \frac{N}{n_+} + n_- \ln \frac{N}{n_-} \\ & = (\frac{N}{2} + \frac{E}{2\epsilon}) \ln \frac{N}{\frac{N}{2} + \frac{E}{2\epsilon}} + (\frac{N}{2} - \frac{E}{2\epsilon}) \ln \frac{N}{\frac{N}{2} - \frac{E}{2\epsilon}} \\ & = N \Big ( (\frac{1}{2} + \frac{E}{2\epsilon N}) \ln \frac{1}{\frac{1}{2} + \frac{E}{2\epsilon N}} + (\frac{1}{2} - \frac{E}{2\epsilon N}) \ln \frac{1}{\frac{1}{2} - \frac{E}{2\epsilon N}} \Big ) \\ & = - N \Big ( (\frac{1}{2} + \frac{E}{2\epsilon N}) \ln (\frac{1}{2} + \frac{E}{2\epsilon N} ) + (\frac{1}{2} - \frac{E}{2\epsilon N}) \ln (\frac{1}{2} - \frac{E}{2\epsilon N} )\Big ) ~.
        \end{aligned}
        \end{equation*}
    \end{proof}
    A plot of this is in Figure~\ref{en:s}.
    \begin{figure}
        \centering
        \scalebox{0.7}{\pyc{plot1('x', '-( (1 / 2 + x) * ln (1/2 + x) +  (1 / 2 - x) * ln (1/2 - x) )', 1, 1, 14, True, False, True)}}
        \caption{A plot of the entropy $S$ as a function of $E$. We have used $x = \frac{E}{2 \epsilon N} $ and $f(x) = \frac{S}{N k_B}$.}
        \label{en:s}
    \end{figure}

    The temperature is 
    \begin{equation*}
        T = \frac{2 \epsilon}{k_B} \frac{1}{\ln \frac{\frac{1}{2} - \frac{E}{2 \epsilon N}}{\frac{1}{2} + \frac{E}{2 \epsilon N}}} ~.
    \end{equation*}
    \begin{proof}
        Using~\eqref{T}
        \begin{equation*}
        \begin{aligned}
            T & = (\pdv{S}{E})^{-1} \\ & = - (\frac{k_B}{2\epsilon} \ln (\frac{1}{2} + \frac{E}{2 \epsilon N}) + \cancel{\frac{k_B}{2\epsilon}} - \frac{k_B}{2\epsilon} \ln (\frac{1}{2} - \frac{E}{2 \epsilon N}) - \cancel{\frac{k_B}{2\epsilon}} )^{-1} \\ & = - (\frac{k_B}{2\epsilon} \ln \frac{\frac{1}{2} + \frac{E}{2 \epsilon N}}{\frac{1}{2} - \frac{E}{2 \epsilon N}})^{-1} \\ & = - \frac{2 \epsilon}{k_B} \frac{1}{\ln \frac{\frac{1}{2} + \frac{E}{2 \epsilon N}}{\frac{1}{2} - \frac{E}{2 \epsilon N}}} \\ & = \frac{2 \epsilon}{k_B} \frac{1}{\ln \frac{\frac{1}{2} - \frac{E}{2 \epsilon N}}{\frac{1}{2} + \frac{E}{2 \epsilon N}}} ~.
        \end{aligned}
        \end{equation*}
    \end{proof}
    A plot of this is in Figure~\ref{en:t}.
    \begin{figure}
        \centering
        \scalebox{0.7}{\pyc{plot1('x', '1 / (ln ( (0.5 - x) / (0.5 + x) ) )', 10, 10, 15, True, False, False)}}
        \caption{A plot of the temperature $T$ as a function of $E$. We have used $x = \frac{E}{2 \epsilon N} $ and $f(x) = \frac{k_B T}{2 \epsilon}$.}
        \label{en:t}
    \end{figure}


\part{Quantum statistical mechanics}

\chapter{Quantum mechanics}

    In quantum mechanics, a pure state is described by a normalised vector in a Hilbert space $\ket{\psi} \in \mathcal H$, which is a vector space on $\mathbb C$, i.e.~in which a linear superposition is still in the space $\lambda_1 \ket{\psi_1} + \lambda_2 \ket{\psi_2}$, endowed with a scalar product $\braket{\psi}{\phi}$ through which it is possible to associate a probability and the normalisation condition $||\psi||^2 = \braket{\psi}{\psi} = 1$. Furthermore, the normalisatin condition ensures that a state is not only a vector, but a ray in a Hilbert space, since two states are physically equivalent if $\ket{\psi} \sim \exp(i \phi) \ket{\psi}$. It can be seen as an equivalence class of states.

\section{Projectors}

    It is possible to uniquely determine the state via a projection operator or projector 
    \begin{equation*}
        P_\psi = \frac{\ket{\psi} \bra{\psi}}{\braket{\psi}{\psi}} ~,
    \end{equation*}
    which for normalisation states becomes 
    \begin{equation*}
        P_\psi = \ket{\psi} \bra{\psi} ~.
    \end{equation*}
    \begin{proof}
        If $\ket{\psi} \mapsto \exp(i \phi) \ket{\psi}$ and $\bra{\psi} \mapsto \exp(- i \phi) \bra{\psi}$, we have 
        \begin{equation*}
            P_\psi \mapsto \cancel{\exp(i \phi)} \ket{\psi} \cancel{\exp(- i \phi)} \bra{\psi} = \ket{\psi} \bra{\psi} = P_\psi ~.
        \end{equation*}
    \end{proof}
    It projects onto the $1$-dimensional subspace generated by the state $\ket{\psi}$
    \begin{equation*}
        P_\psi \colon \mathcal H \rightarrow \mathcal H_\psi ~,
    \end{equation*}
    where $\mathcal H_\psi = \{\lambda \ket{\psi} \colon \lambda \in \mathbb C\}$.
    \begin{proof}
        In fact, $\forall \ket{\psi} \in \mathcal H$, we decomposed it into 
        \begin{equation*}
            \ket{\psi} = \alpha \ket{\psi} + \beta \ket{\psi^\perp}
        \end{equation*}
        and the action of the projector
        \begin{equation*}
            P_\psi \ket{\psi} = \alpha \underbrace{P_\psi}_{\ket{\psi} \bra{\psi}} \ket{\psi} + \beta \underbrace{P_\psi \ket{\psi^\perp}}_0 = \alpha \ket{\psi} \underbrace{\braket{\psi}{\psi}}_1 = \alpha \ket{\psi} ~.
        \end{equation*}
    \end{proof}

    The projector is also called density matrix $\rho_\psi$. It satisfies the following properties 
    \begin{enumerate}
        \item boundness, i.e. 
            \begin{equation*}
                ||\rho_\psi|| < C~,
            \end{equation*}
        \item hermiticity, i.e. 
            \begin{equation*}
                \rho_\psi^\dagger = \rho_\psi ~,
            \end{equation*}
        \item idempotence, i.e. 
            \begin{equation*}
                \rho_\psi^2 = \rho_\psi ~,
            \end{equation*}
        \item positive defined, i.e. $\forall \ket{\phi} \in \mathcal H$
            \begin{equation*}
                \bra{\phi} \rho_\psi \ket{\phi} \geq 0 ~,
            \end{equation*}
        \item unit trace, i.e. 
            \begin{equation*}
                \tr \rho_\psi = 1 ~.
            \end{equation*}
    \end{enumerate}
    Actually, there is a theorem which ensures that an operators such that it satifies these 5 conditions is indeed the projector.
    \begin{proof}
        For the hermiticity
        \begin{equation*}
            \rho_\psi^\dagger = (\ket{\psi} \bra{\psi})^\dagger = \bra{\psi}^\dagger \ket{\psi}^\dagger = \ket{\psi} \bra{\psi} = \rho_\psi ~.
        \end{equation*}

        For the idempotence
        \begin{equation*}
            \rho_\psi^2 = (\ket{\psi} \bra{\psi})^2 = \ket{\psi} \underbrace{\braket{\psi}{\psi}}_1 \bra{\psi} = \ket{\psi} \bra{\psi} = \rho_\psi ~.
        \end{equation*}
    \end{proof}

    Given an orthonormal basis $\{\ket{e_n}\}_{n=1}^\infty$ of a separable Hilbert space, the trace is defined as 
    \begin{equation*}
        \tr A = \sum_{n=1}^\infty A_{nn} = \sum_{n=1}^\infty \bra{e_n} A \ket{e_n} ~.
    \end{equation*}
    If it convergent, it is called a trace-class operator. Furthermore, if it is absolute convergent, the trace is independent on the choice of the basis.

\section{Observable}

    An observable is a linear hermitian operator acting on the Hilbert space. We require the self-adjointness because its eigenvalues are real and it always admit an eigenbasis, such that every state can be expanded in this basis 
    \begin{equation*}
        A \ket{\psi_n} = \lambda_n \ket{\psi_n} ~,
    \end{equation*}
    such that 
    \begin{equation*}
        \lambda_n \in \mathbb R
    \end{equation*}
    and $\forall \ket{\phi} \in \mathcal H$
    \begin{equation*}
        \ket{\phi} = \sum_{n=1}^{\infty} c_n \ket{\psi_n} ~.
    \end{equation*}

    The projectors on the eigenstates are orthogonal 
    \begin{equation*}
        P_n P_m = P_n P_m = 0 ~.
    \end{equation*}

    If we prepare the system in a state $\ket{\psi}$, a measurement of an observable $A$ can have outcomes $\lambda_n$ with probability $P_n = |c_n|^2$ where we have defined 
    \begin{equation*}
        \ket{\phi} = \sum_n c_n \ket{\psi_n}~, \quad A \ket{\psi_n} = \lambda_n \ket{\psi_n} ~.
    \end{equation*}
    Its average value is 
    \begin{equation*}
        \av{A} = \sum_{n} \lambda_n P_n = \sum_{n} \lambda_n |c_n|^2 = \tr(A \rho_\psi) ~. 
    \end{equation*}

\section{Composite system}

    For $2$ particles, the total Hilbert space is the tensor product between the single particle Hilbert spaces 
    \begin{equation*}
        \mathcal H_{tot} = \mathcal H_1 \otimes \mathcal H_2 ~. 
    \end{equation*}
    Given an orthonormal basis for each Hilbert space $\{\ket{\psi_n}\} \in \mathcal H_1$ and $\{\ket{\phi_m}\} \in \mathcal H_2$, the total orthonormal basis is 
    \begin{equation*}
        \{\ket{\psi_n}_1 \ket{\phi_m}_2 = \ket{\psi_n \phi_m}\} 
    \end{equation*}
    such that a generic state can be expanded into this basis, $\forall \ket{\phi} \in \mathcal H_{tot}$
    \begin{equation*}
        \ket{\phi} = \sum_n \sum_m a_{nm} \ket{\psi_n \phi_m} ~,
    \end{equation*}
    where the normalisation condition is $\sum_{nm} |a_{nm}|^2 = 1$.

    If the $2$ particle are identical, we have $\mathcal H_1 = \mathcal H_2 = \mathcal H$. Therefore $\mathcal H_{tot} = \mathcal H^{\otimes 2}$.

    The scalar product is 
    \begin{equation*}
        \braket{\psi_n \phi_m}{\psi_{n'} \phi_{m'}} = \braket{\psi_n}{\psi_{n'}} \braket{\phi_m}{\phi_{m'}} ~.
    \end{equation*}

\section{N distinguishable particles}

    A single particle lives in $\mathbb R^3$ and its Hilbert space is $\mathcal H = L^2 (\mathbb R^3) \ni \psi(x)$. The scalar product is 
    \begin{equation*}
        \braket{\psi}{\phi} = \int d^3 x ~ \psi^*(x) \phi(x) ~,
    \end{equation*}
    where the normalisation condition is 
    \begin{equation*}
        ||\psi||^2 = \braket{\psi}{\psi} = \int_{\mathbb R^3} d^3 x ~ |\psi(x)|^2 < infty ~.
    \end{equation*}

    For $N$ distinguishable particles, the total Hilbert space is $\mathcal H_{tot} = \mathcal H \otimes \ldots \otimes \mathcal H$ and an orthonormal basis is $\ket{\psi_{n_1} \ldots \psi_{n_N}}$ where $\ket{\psi_{n_j}}$ is a single particle orthonormal basis. Hence, $N$ distinguishable particle live in $\mathbb R^{3N}$ and their Hilbert space is $\mathcal H_N = L^2(\mathbb R^3) \otimes \ldots \otimes L^2(\mathbb R^3) = L^2 (\mathbb R^{3N}) \ni \psi(x_1, \ldots x_N)$. Therefore, an orthonormal basis is $\{u_{\alpha_1 (x_1)} \ldots u_{\alpha_N (x_N)} = u_{\alpha_1 \ldots \alpha_N} (x_1, \ldots x_N)\}$ where $\{u_\alpha (x)\}$ is the single particle orthonormal basis.

    A generic state can be expanded in this basis as 
    \begin{equation*}
        \psi(x_1, \ldots x_N) = \sum_{\alpha_1 \ldots \alpha_N} c_{\alpha_1 \ldots \alpha_N} u_{\alpha_1 \ldots \alpha_N} (x_1, \ldots x_N) ~.
    \end{equation*}
    For instance, choosing $\alpha_1 = a$ and $\alpha_2 = b$ or viceversa
    \begin{equation*}
        u_{\alpha_1 = a} (x_1) u_{\alpha_2 = b} (x_2) \neq u_{\alpha_1 = b} (x_1) u_{\alpha_2 = a} (x_2) ~.
    \end{equation*}

    If the particle are indistinguishable, they are invariant under permutations
    \begin{equation*}\label{perm}
        \psi(x_1, \ldots x_N) \mapsto \psi(P(x_1, \ldots x_N)) = \exp(i \alpha_P) \psi (x_1, \ldots x_N) ~,
    \end{equation*}
    where $P$ belongs to the permutation group.

\chapter{Permutation group}

    The premutation of $N$ elements form a group $P_N$. This group is generated by transposition $\{\sigma_i\}_{i=1}^N$. In fact, any permutation can be defined by consecutive transposition, where a transposition is defined as 
    \begin{equation*}
        \sigma_i \colon (1,2,\ldots, i, i+1, \ldots N) \mapsto (1,2,\ldots, i+1, i, \ldots N) ~.
    \end{equation*}
    However, this decomposition is not unique but the number of transposition in its decomposition is always even or odd. Therefore, we can define the sign of permutation $\forall P \in P_N$
    \begin{equation*}
        sign(P) = \begin{cases}
            + 1 & \textnormal{even number of transposition in its decomposition } \\
            - 1 & \textnormal{odd number of transposition in its decomposition } \\
        \end{cases} ~.
    \end{equation*}

    Transpositions follow the properties 
    \begin{enumerate}
        \item if $|i - j| > 2$ \begin{equation}\label{prop1}
            \sigma_i \sigma_j = \sigma_j \sigma_i ~,
        \end{equation} 
        \item \begin{equation}\label{prop2}
            \sigma_i \sigma_{i+1} \sigma_i = \sigma_{i+1} \sigma_i \sigma_{i+1} ~,
        \end{equation}
        \item \begin{equation}\label{prop3}
            (\sigma_i)^2 = \mathbb I ~.
        \end{equation}
    \end{enumerate}

    Hence, we can calculate explicitly~\eqref{perm}, which is 
    \begin{equation}
        \alpha_P = \alpha_1 + \ldots \alpha_N~.
    \end{equation}
    \begin{proof}
        In fact
        \begin{equation*}
        \begin{aligned}
        \psi(P(x_1,\ldots x_N)) & = \psi((\sigma_{\alpha_1} \ldots \sigma_{\alpha_N}) (x_1,\ldots x_N)) \\ & = \exp (i \alpha_1) \psi((\sigma_{\alpha_2} \ldots \sigma_{\alpha_N}) ) \\ & ~~ \vdots \\ & = \exp (i \alpha_1) \ldots \exp (i \alpha_N) \psi(x_1,\ldots x_N) \\ & = \exp (i (\alpha_1 + \ldots \alpha_N)) \psi(x_1,\ldots x_N) \\ &  = \exp (i \alpha_P) \psi(x_1,\ldots x_N)~,
        \end{aligned}
        \end{equation*}
        where $P = \sigma_{\alpha_1} \ldots \sigma_{\alpha_N}$.
    \end{proof}

    Furthermore, $\alpha_P = 0, m\pi$, which correspond respectively to a bosonic totally symmetric wavefunction, i.e. under $P$
    \begin{equation*}
        \psi(x_1, \ldots x_N) \xmapsto{P} + 1 \psi(x_1, \ldots x_N) ~,
    \end{equation*}
    or to a fermionic totally antisymmetric wavefunction, i.e. under $P$ 
    \begin{equation*}
        \psi(x_1, \ldots x_N) \xmapsto{P} (-1)^m \psi(x_1, \ldots x_N) = \begin{cases}
            + & \textnormal{sign(P) = +1} \\
            - & \textnormal{sign(P) = -1} \\
        \end{cases}~.
    \end{equation*}

    \begin{proof}
        For~\eqref{prop1}
        \begin{equation*}
            \psi(x_1, \ldots x_N) \xmapsto{\sigma_i} \exp(i \alpha_i) \psi(x_1, \ldots x_N)\xmapsto{\sigma_i \sigma_j} \exp(i \alpha_i) \exp(i \alpha_j) \psi(x_1, \ldots x_N) ~,
        \end{equation*}
        \begin{equation*}
            \psi(x_1, \ldots x_N) \xmapsto{\sigma_j} \exp(i \alpha_j) \psi(x_1, \ldots x_N)\xmapsto{\sigma_j \sigma_i} \exp(i \alpha_j) \exp(i \alpha_i) \psi(x_1, \ldots x_N) ~,
        \end{equation*}
        which means that 
        \begin{equation*}
            \exp(i \alpha_i) \exp(i \alpha_j) = \exp(i \alpha_j) \exp(i \alpha_i) ~.
        \end{equation*}

        For~\eqref{prop2}
        \begin{equation*}
            \psi(x_1, \ldots x_N) \xmapsto{\sigma_i} \exp(i \alpha_i) \psi(x_1, \ldots x_N)\xmapsto{\sigma_i \sigma_{i+1}} \exp(i \alpha_i) \exp(i \alpha_{i+1}) \psi(x_1, \ldots x_N) \xmapsto{\sigma_i \sigma_{i+1} \sigma_i} \exp(i \alpha_i) \exp(i \alpha_{i+1}) \exp(i \alpha_i) \psi(x_1, \ldots x_N) ~,
        \end{equation*}
        \begin{equation*}
            \psi(x_1, \ldots x_N) \xmapsto{\sigma_{i+1}} \exp(i \alpha_{i+1}) \psi(x_1, \ldots x_N)\xmapsto{\sigma_{i+1} \sigma_i} \exp(i \alpha_{i+1}) \exp(i \alpha_i) \psi(x_1, \ldots x_N) \xmapsto{\sigma_{i+1} \sigma_i \sigma_{i+1} } \exp(i \alpha_{i+1}) \exp(i \alpha_i) \exp(i \alpha_{i+1}) \psi(x_1, \ldots x_N) ~,
        \end{equation*}
        which means that 
        \begin{equation*}
            \exp(i \alpha_i) \exp(i \alpha_{i+1}) \exp(i \alpha_i) = \exp(i \alpha_{i+1}) \exp(i \alpha_i) \exp(i \alpha_{i+1}) ~,
        \end{equation*}
        where we have used the fact that $\exp(i \alpha_i) \exp(i \alpha_j)$ commutes. Therefore, $\forall i= 1, \ldots N-1$ and $\alpha_i \in [0, 2\pi[$ we have $\alpha_i = \alpha_{i+1} = \alpha$.

        For~\eqref{prop3}
        \begin{equation*}
            \exp(i \alpha)^2 = \exp (2 i \alpha) = \mathbb I = \exp(0) ~,
        \end{equation*}
        which means that 
        \begin{equation*}
            \alpha = 0, 2\pi ~.
        \end{equation*}

        Finally, there are only two possibilities 
        \begin{equation*}
            \psi(x_1, \ldots x_N) \xmapsto{\sigma_i} \underbrace{\exp(i 0)}_{+1} \psi(x_1, \ldots x_N)
        \end{equation*}
        and 
        \begin{equation*}
            \psi(x_1, \ldots x_N) \xmapsto{\sigma_i} \underbrace{\exp(i \pi)}_{-1} \psi(x_1, \ldots x_N) ~.
        \end{equation*}
    \end{proof}

\section{Symmetric/antisymmetric Hilbert space} 

    Consider $2$ distinguishable particles. In general, the Hilbert space is $\mathcal H_{tot} = \mathcal H \otimes \mathcal H$. If the particle are indistinguishable, we can decomposed the Hilbert space into $\mathcal H_{tot} = \mathcal H_S \otimes_\perp  \mathcal H_A$. In fact, given two states $\ket{a}_1 \in \mathcal H_1$ and $\ket{b}_2 \in \mathcal H_2$, we have 
    \begin{equation*}
    \begin{aligned}
        \ket{a}_1\ket{b}_2 & = \frac{2}{2} \ket{a}_1\ket{b}_2 + \frac{1}{2} \ket{b}_1\ket{a}_2 - \frac{1}{2} \ket{b}_1\ket{a}_2 \\ & = \underbrace{\frac{\ket{a}_1\ket{b}_2 + \ket{b}_1\ket{a}_2}{2}}_{\ket{\psi_S}} + \underbrace{\frac{\ket{a}_1\ket{b}_2 - \ket{b}_1\ket{a}_2}{2}}_{\ket{\psi_A}} \\ & = \ket{\psi_S} + \ket{\psi_A} ~.
    \end{aligned}
    \end{equation*}
    Notice that Pauli's exclusion principle is encoded into the antysymmetric part, because if $a = b$ we have $\ket{\psi_A} = 0$. It is also an orthogonal decomposition. In fact 
    \begin{equation*}
    \begin{aligned}
        \braket{\psi_S}{\psi_A} & = \frac{\bra{a}_1\bra{b}_2 + \bra{b}_1\bra{a}_2}{2} \frac{\ket{a}_1 \ket{b}_2 - \ket{b}_1 \ket{a}_2}{2} \\ & = \frac{1}{4} (\underbrace{\braket{a}{a}_1}_1 \underbrace{\braket{b}{b}_2}_1 - \cancel{\braket{a}{b}_1 \braket{b}{a}_2} + \cancel{\braket{b}{a}_1 \braket{a}{b}_2} - \underbrace{\braket{b}{b}_1}_1 \underbrace{\braket{a}{a}_2}_1 ) = 0 ~. 
    \end{aligned}
    \end{equation*}

    The decomposition is equivalent to define two orthogonal projectors: the symmetriser 
    \begin{equation*}
        \hat S \colon \mathcal H \rightarrow mathcal H_S
    \end{equation*}
    and the antisymmetriser 
    \begin{equation*}
        \hat A \colon \mathcal H \rightarrow mathcal H_A ~,
    \end{equation*}
    such that they satisfy the properties 
    \begin{equation*}
        \hat S^\dagger = \hat S~, \quad \hat A^\dagger = \hat A~, \quad \hat S^2 = \hat S~, \quad \hat A^2 = \hat A~, \quad \hat S \hat A = \hat A \hat S = 0 ~.
    \end{equation*}

    Generalising for $N$ particles, we have $\mathcal H_{tot} = \mathcal H \otimes \ldots \mathcal H$ and a state is $\ket{a_1}_1 \ldots \ket{a_N}_N$ where $\ket{a_j} \in \mathcal H$. The symmetriser is 
    \begin{equation*}
        \hat S \colon \ket{\psi} \mapsto \frac{1}{N!} \sum_{P \in P_N} \ket{a_{P(1)}}_1 \ldots \ket{a_P(N)}_N
    \end{equation*}
    and the antisymmetriser is 
    \begin{equation*}
        \hat A \colon \ket{\psi} \mapsto \frac{1}{N!} \sum_{P \in P_N} sgn(P) \ket{a_{P(1)}}_1 \ldots \ket{a_P(N)}_N
    \end{equation*}
    where $P(1, \ldots N) \mapsto (P(1), \ldots P(N))$. They satisfy the orthogonal projector properties. Notice that for $N > 2$ particles, the total Hilbert space is $\mathcal H_{tot} = \mathcal H_S \otimes \mathcal H_A \otimes \mathcal H'$, where bosons work only in $\mathcal H_S$, fermions work only in $\mathcal H_A$ and $\mathcal H'$ is not physical.

    For distinguishable particles, we have $\mathcal H_{tot} = \mathcal H^{\otimes N}$ with orthonormal basis $\{u_{\alpha_1}(x_1) \ldots u_{\alpha_N}(x_N)\}_{\alpha_1, \ldots \alpha_N=0}^\infty$ labelled by the ordered set $(\alpha_1, \ldots \alpha_N)$. In this case, we are specifying which particle is in which states. However, for indistinguishable particles, we lose information because we know only how many particle are in each state. We label the states with $n_1, \ldots n_j$ with $j=1, \ldots \infty$, which are the occupation number. For bosons, we have $n_k = 0, 1, \ldots, \infty$, whereas for fermions, we have $n_k = 0, 1$. For both cases, there is the constrain $N = \sum_k n_k$, which is an infinite sum but mostly are zero occupied.

\chapter{Second quantisation} 

\section{Bosonic case}

    We define creation and annihilation operators such that they satisfies the properties 
    \begin{equation*}
        [\hat a, \hat a^\dagger]_- = \hat a \hat a^\dagger - \hat a^\dagger \hat a = \mathbb I~.
    \end{equation*}
    Furthermore, the number operator $\hat N = \hat a^\dagger \hat a$ such that 
    \begin{equation*}
        [\hat N, \hat a] = - \hat a~, \quad [\hat N, \hat a^\dagger] = \hat a^\dagger ~.
    \end{equation*} 

    By analogy with the harmonic oscillator, the ground state is the vacuum 
    \begin{equation*}
        \hat a \ket{0} = 0 ~,
    \end{equation*}
    and a generic state is defined by the ladder operators
    \begin{equation*}
        \ket{\psi} = \frac{1}{\sqrt{n!}} (\hat a^\dagger)^N \ket{0} ~.
    \end{equation*}

\section{Fermionic case}

    We define creation and annihilation operators such that they satisfies the properties 
    \begin{equation*}
        [\hat a, \hat a^\dagger]_+ = \hat a \hat a^\dagger + \hat a^\dagger \hat a = \mathbb I~.
    \end{equation*}
    Furthermore, the number operator $\hat N = \hat a^\dagger \hat a$ such that 
    \begin{equation*}
        [\hat N, \hat a] = - \hat a~, \quad [\hat N, \hat a^\dagger] = \hat a^\dagger ~.
    \end{equation*} 

    The properties can be obtained from the Pauli matrices 
    \begin{equation*}
        \sigma_\pm = \sigma_1 \pm i \sigma_2 ~,
    \end{equation*}
    such that 
    \begin{equation*}
        (\sigma_+)^\dagger = \sigma_- ~, \quad (\sigma_-)^\dagger = \sigma_+ ~, \quad (\sigma_+)^2 = (\sigma_-)^2 = 0 ~, \quad [\sigma_-, \sigma_+]_+ = \mathbb I ~.
    \end{equation*}

    By analogy with the harmonic oscillator, the ground state is the vacuum 
    \begin{equation*}
        \hat a \ket{0} = 0 ~,
    \end{equation*}
    and a generic state is defined by the ladder operators
    \begin{equation*}
        \ket{\psi} = \frac{1}{\sqrt{n!}} (\hat a^\dagger)^N \ket{0} ~.
    \end{equation*}

    However, the anticommutator relation ensures the validity of the Pauli's exclusion principle. In fact, we have 
    \begin{equation*}
        a^2 = (\hat a^\dagger)^2 = 0 ~.
    \end{equation*}

\section{Fock space}

    Consider a single particle Hilbert space $\mathcal H$ with an orthonormal basis $\{\ket{e_n}\}_{n=1}^\infty$. To each $\ket{e_n}$, we associate an annihilation and a creation operators 
    \begin{equation*}
        \ket{e_n} \mapsto \{\hat a_n, \hat a_n^\dagger \}_{n=1}^\infty ~,
    \end{equation*}
    such that they satisfy 
    \begin{equation*}
        [\hat a_n, \hat a_m]_\pm = [\hat a_n^\dagger, \hat a_m^\dagger]_\pm = 0 ~, \quad [\hat a_n, \hat a_m^\dagger]_\pm = \delta_{nm} ~,
    \end{equation*}
    where the minus sign correponds to the commutator (bosons) and the plus sign to the anticommutator (fermions). 

    The vacuum state is defined as 
    \begin{equation*}
        \hat a_n \ket{0} = 0 \quad \forall n~.
    \end{equation*}

    For each $\ket{e_n}$, we associate a number operator $\hat n_k = \hat a_k^\dagger \hat a_k$ such that 
    \begin{equation*}
        \hat n_k \hat a_k^\dagger \ket{0} = 1 \hat a_k^\dagger \ket{0} ~, \quad \hat n_{k'} \hat a_k^\dagger \ket{0} = 0 \quad k' \neq k ~.
    \end{equation*}
    For a $n$ particle state, we have 
    \begin{equation*}
        \hat a_k^\dagger \ket{0} = \ket{n_1=0, \ldots n_k=1, \ldots n_N=0} = \ket{e_k} ~.
    \end{equation*}

    However, for 
    \begin{equation*}
        \hat a_{k_1}^\dagger \hat a_{k_2}^\dagger \ket{0} = \ket{e_{k_1}} \ket{e_{k_2}} 
    \end{equation*}
    we have for fermions, if $k_1 = k_2 = k$
    \begin{equation*}
    (\hat a^\dagger_k)^2 \ket{0} = 0 ~,
    \end{equation*}
    whereas for bosons 
    \begin{equation*}
        (\hat a^\dagger_k)^2 \ket{0} \neq 0 ~.
    \end{equation*}
    Furthermore, if $k_1 \neq  k_2$, we have for fermions
    \begin{equation*}
        \hat a^\dagger_{k_1} \hat a^\dagger_{k_2} \ket{0} = - \hat a^\dagger_{k_2} \hat a^\dagger_{k_1} \ket{0} ~,
    \end{equation*}
    whereas for bosons 
    \begin{equation*}
        \hat a^\dagger_{k_1} \hat a^\dagger_{k_2} \ket{0} = \hat a^\dagger_{k_2} \hat a^\dagger_{k_1} \ket{0} ~.
    \end{equation*}

\section{Alternative way}

    There is a $1-1$ correspondence between the orthonormal basis  $\{\ket{e_n}\}_{n=1}^\infty$ of $\mathcal H$ and the orthonormal basis $\{\hat a_k \ket{0}\}_{k=1}^\infty$ of $\mathcal H_{S/A}$. Hence for $N$ particles, we have 
    \begin{equation*}
        \mathcal H_{S/A}^{(N)} = \{\ket{n_1, \ldots n_k, \ldots} = \frac{1}{\sqrt{ \prod_j n_j}} (\hat a_1^\dagger)^{n_1} \ldots (\hat a_k^\dagger)^{n_k} \ldots \ket{0} \} ~.
    \end{equation*} 

    If $N$ is not fixed, like the passage from canonical to grancanonicl ensemble, the total Fock space is 
    \begin{equation*}
        \mathcal F = \bigoplus_{N=0}^\infty \mathcal H^{(N)}_{S/A} ~.
    \end{equation*} 

    It satisfies the following properties 
    \begin{enumerate}
        \item orthonormality, i.e. 
            \begin{equation*}
                \braket{{n'}_1, \ldots {n'}_k, \ldots}{n_1, \ldots n_k, \ldots} = \delta_{{n'}_1, n_1} \ldots \delta_{{n'}_k, n_k} \ldots  ~,
            \end{equation*}
        \item annihilation $\hat a_k \colon \mathcal H^{(N)}_{S/A} \rightarrow \mathcal H^{(N-1)}_{S/A}$, i.e.
            \begin{equation*}
                \hat a_k \ket{n_1, \ldots n_k, \ldots} = \eta_k \sqrt{n_k} \ket{n_1, \ldots (n_k - 1), \ldots} ~,
            \end{equation*}
            where for bosons $\eta_k = 1$ and for fermions $\eta_k = (-1)^{\sum_{j < k} n_j}$,
        \item creation $\hat a_k^\dagger \colon \mathcal H^{(N)}_{S/A} \rightarrow \mathcal H^{(N+1)}_{S/A}$, i.e. for bosons
            \begin{equation*}
                \hat a^\dagger_k \ket{n_1, \ldots n_k, \ldots} = \sqrt{n_k + 1} \ket{n_1, \ldots (n_k + 1), \ldots} ~,
            \end{equation*}
            and for fermions
            \begin{equation*}
                \hat a^\dagger_k \ket{n_1, \ldots n_k, \ldots} = \eta_k \sqrt{1 - n_k} \ket{n_1, \ldots (n_k + 1), \ldots} ~,
            \end{equation*}
        \item number operator $\hat n_k = \hat a_k^\dagger \hat a_k$ such that 
            \begin{equation*}
                \hat n_k \ket{n_1, \ldots n_k, \ldots} = n_k \ket{n_1, \ldots n_k, \ldots}
            \end{equation*}
        and the total number operator $\hat N = \sum_k \hat n_k \sum_k \hat a^\dagger_k \hat a_k$ such that 
        \begin{equation*}
            \hat N \ket{n_1, \ldots n_k, \ldots} = \Big (\sum_k n_k \Big ) \ket{n_1, \ldots n_k, \ldots} ~.
        \end{equation*}
    \end{enumerate}

\section{Field operators} 

    In the first quantisation, we quantise observables to operators, while, in the second quantisation, we quantise fields to operators. Now, a generic particle state is represented by $\ket{f} = \sum_k f_k \ket{e_k} \in \mathcal H$, which is equivalent to $sum_k f_k \hat a_k^\dagger \ket{0}$. Hence, we define the field operators
    \begin{equation*}
        \hat \psi^\dagger (f) = \sum_k f_k \hat a^\dagger_k ~, \quad \hat \psi (f) = \sum_k f_k^* \hat a_k ~,
    \end{equation*}
    in order to get a state $\hat \psi (f) \ket{0}$. The related commutator relations become
    \begin{equation*}
        [\hat \psi (f), \hat \psi^\dagger (g)]_\pm = \braket{f}{g}\mathbb I ~.
    \end{equation*}
    \begin{proof}
        In fact,
        \begin{equation*}
            [\hat \psi (f), \hat \psi^\dagger (g)]_\pm = [\sum_k f^*_k \hat a_k, \sum_m g_m \hat a^\dagger]_\pm = \sum_k \sum_m f^*_k g_m \underbrace{[\hat a_k, \hat a^\dagger_m]}_{\delta_{km} \mathbb I} = \sum_k \sum_m f^*_k g_m \underbrace{\delta_{km}}_{k= m} \mathbb I = \sum_k f^*_k g_k \mathbb I = \braket{f}{g} \mathbb I ~.
        \end{equation*}
        where we have used $\ket{f} \sum_k f_k \ket{e_k}$, $\ket{g} = \sum_m g_m \ket{e_m}$ and $\braket{f}{g} = \sum_k \sum_m f^*_k g_m \underbrace{\braket{e_k}{e_m}}_{\delta_{km}} = \sum_k f^*_k g_k$.
    \end{proof}

    Consider a single particle state in $\mathcal H = L^2(\mathbb R^d) \ni \psi(x)$ with an orthonormal basis $u_k(x)$ such that to each ket there are ladder operators $\hat a_k$ and $\hat a_k^\dagger$. Hence $L^2(\mathbb R^d) \ni f(x) = \sum_k f_k u_k(x)$ and we define field operators
    \begin{equation*}
        \hat \psi(x) = \sum_k u_k^* (x) \hat a_k ~, \quad \hat \psi^\dagger (x) = \sum_k u_k (x) \hat a_k^\dagger ~,
    \end{equation*}
    which is alinear superposition of annihilation and creation operators. Actually, it is called an operator-valued function because its output is an operator. In fact 
    \begin{equation*}
        \int_{\mathbb R^d} d^d x ~ \psi^\dagger (x) \sum_k u_k^* (x) \hat a_k^\dagger = \sum_k \hat a_k^\dagger \int_{\mathbb R^d} d^d x ~ u^*_k(x) f(x) = \sum_k \hat a_k^\dagger f_k~,
    \end{equation*}
    where we have exchanged sum and integral because they are convergent. 

    The commutation relations are 
    \begin{equation*}
        [\psi(x), \psi^\dagger (y)] = \mathbb I \delta (x - y) ~.
    \end{equation*}
    \begin{proof}
        In fact,
        \begin{equation*}
            [\hat \psi (f), \hat \psi^\dagger (g)]_\pm = [\int d^d x ~ f^* (x) \hat \psi(x), \int d^d y ~ g(y) \hat \psi^\dagger (y)]_\pm = \int d^d x \int d^d y ~ f^*(x) g(y) [\psi(x), \psi^\dagger (y)]  ~,
        \end{equation*}
        which must be equal to 
        \begin{equation*}
            \braket{f}{g} = \int d^d x ~ f^* (x) g(x) ~.
        \end{equation*}
        Hence 
        \begin{equation*}
            [\psi(x), \psi^\dagger (y)] = \mathbb I \delta (x - y) ~.
        \end{equation*}
    \end{proof}

    For instance, a plane wave $u(x) = \exp (i \mathbf k \cdot \mathbf x) $ and $\hat \psi(x) = \sum_k \hat a_k^\dagger \exp(i \mathbf k \cdot \mathbf x)$.

    Notice that field operators are basis independent

\section{Operators}

    Consider a Fock space $\mathcal F = \bigoplus_{N=0}^\infty \mathcal H^{(N)}_{B/F}$ with orthonormal basis $\ket{n_1, \ldots n_k, \ldots} = \frac{1}{\sqrt{\prod_j n_j !}} (\hat a^\dagger_1)^{n_1} \ldots (\hat a_k^\dagger)^{n_k} \ldots \ket{0}$, which is in $1-1$ correspondence to the orthonormal basis $\psi_{n_1 \ldots n_k \ldots} (x_1, \ldots x_k, \ldots) = c_N \begin{bmatrix} \hat S \\ \hat A \\ \end{bmatrix} u_{\alpha_1} (x_1) \ldots u_{\alpha_k} (x_k) \ldots$, where $hat S$ is the symmetriser and $\hat A$ is the antisymmetriser.

    We define a one-body operator, associated to a system in which all the particles are the same, as 
    \begin{equation*}
        \hat O^{(1)} = \sum_{j=1}^{N} \hat O(\hat p_j, \hat x_j) ~.
    \end{equation*}
    Since it is self-adjoint, it exists an orthonormal basis of eigenvalues $\{u_\alpha (x)\}$, such that 
    \begin{equation*}
        \hat O(\hat p, \hat x) u_\alpha (x) = \epsilon_\alpha u_\alpha (x) ~.
    \end{equation*}

    Since 
    \begin{equation*}
    \begin{aligned}
        \hat O^{(1)} \psi_{n_1 \ldots n_k \ldots} (x_1, \ldots x_k, \ldots) & = \Big ( \sum_{j=1}^{\infty} \hat O(\hat p_j, \hat x_j) \Big) \psi_{n_1 \ldots n_k \ldots} (x_1, \ldots x_k, \ldots) \\ & = \Big ( \sum_{j=1}^{\infty} \hat O(\hat p_j, \hat x_j) \Big) c_N \begin{bmatrix} \hat S \\ \hat A \\ \end{bmatrix} u_{\alpha_1} (x_1) \ldots u_{\alpha_k} (x_k) \ldots \\ & = c_N \begin{bmatrix} \hat S \\ \hat A \\ \end{bmatrix} \Big ( \sum_{j=1}^{\infty} \hat O(\hat p_j, \hat x_j) u_{\alpha_1} (x_1) \ldots u_{\alpha_k} (x_k) \ldots \Big) \\ & = c_N \begin{bmatrix} \hat S \\ \hat A \\ \end{bmatrix} \Big ( \sum_{j=1}^{\infty}  u_{\alpha_1} (x_1) \ldots \underbrace{\hat O(\hat p_j, \hat x_j) u_{\alpha_j} (x_j)}_{\epsilon_{\alpha_j} u_{\alpha_j} (x_j) } \ldots \Big) \\ & = \Big (\sum_{j=1}^{\infty} \epsilon_j n_j \Big ) \psi_{n_1 \ldots n_k \ldots} (x_1, \ldots x_k, \ldots) ~.
    \end{aligned}
    \end{equation*}

    For the Fock space, we have 
    \begin{equation*}
        \hat O^{(1)}_F = \sum_{j=1}^{\infty} \epsilon_j \hat n_j = \sum_{j=1}^{\infty} \epsilon_j \hat a_j^\dagger \hat a_j ~,
    \end{equation*}
    where 
    \begin{equation*}
        \epsilon_j = \bra{u_j (x)} \hat O (\hat p_j, \hat x_j) \ket{u_j(x)} ~.
    \end{equation*}
    Hence 
    \begin{equation*}
        \hat O^{(1)}_F = \sum_{j=1}^{\infty} \bra{u_j (x)} \hat O (\hat p_j, \hat x_j) \ket{u_j(x)} \hat a_j^\dagger \hat a_j ~.
    \end{equation*}

    Since it is dependent of the basis, because we choose the eigenbasis, we choose a different arbitrary basis 
    \begin{equation*}
        \psi^\dagger (x) = \sum_k u_k (x) \hat a^\dagger_k = \sum_m v_m (x) b^\dagger_m ~,
    \end{equation*}
    and we define the one-body operator
    \begin{equation*}
        \hat O^{(1)}_F = \int d^d x ~ \hat \varphi^\dagger (x) \hat O (\hat p, \hat x) \hat \varphi (x) ~,
    \end{equation*}
    which this time is basis independent.
    \begin{proof}
        In fact 
        \begin{equation*}
        \begin{aligned}
            \int d^d x ~ \hat \varphi^\dagger (x) \hat O (\hat p, \hat x) \hat \varphi (x) & = \int d^d x ~ \Big ( \sum_k u_k(x) \hat a^\dagger (x) \Big ) \hat O (\hat p, \hat x) \Big ( \sum_m u_m^* (x) \hat a_m (x) \Big ) \\ & = \sum_k \sum_m \hat a_k^\dagger \hat a_m \int d^d x ~ u_k (x) \underbrace{\hat O(\hat p, \hat x) u^*_m (x)}_{\epsilon_m u^*_m (x)} \\ & = \sum_k \sum_m \hat a_k^\dagger \hat a_m \epsilon_m \underbrace{\int d^d x ~ u_k (x) u^*_m (x)}_{\delta_{km}} \\ & = \sum_k \sum_m \hat a_k^\dagger \hat a_m \epsilon_m \underbrace{\delta_{km}}_{k = m} \\ & = \sum_k\hat a_k^\dagger \hat a_k \epsilon_k = \hat O^{(1)}_F ~.
        \end{aligned}
        \end{equation*}
    \end{proof}
    It can be written as 
    \begin{equation*}
        \hat O^{(1)}_F = \sum_k \sum_m t_{km} \hat b_k^\dagger \hat h_m ~,
    \end{equation*}
    where the transition amplitude is
    \begin{equation*}
        t_{km} = \bra{v_k} \hat O (\hat p, \hat x) \ket{v_m} ~.
    \end{equation*}




\part{Quantum statistical mechanics}

\chapter{Microcanonical ensemble}

    The microcanonical ensemble is characterised by constant volume, energy and number of particle. Since $N$ is fixed, we can work in the Hilbert space $\mathcal H_{tot}$. Given a time-independent hamiltonian operator $\hat H$, we find the energy eigenbasis $\ket{\psi_j} \in \mathcal H_{tot}$ 
    \begin{equation*}
        \hat H \ket{\psi_j} = E_j \ket{\psi_j} ~.
    \end{equation*}
    However, there could be some degeneracy we want to consider, i.e. $E_{j,\alpha} = E_{j, \beta}$ for $\ket{\psi_{j, \alpha}} \neq \ket{\psi_{j, \beta}}$. Therefore, we have 
    \begin{equation}\label{eneigen}
        \hat H \ket{\psi_{j,\alpha}} = E_j \ket{\psi_{j,\alpha}} ~,
    \end{equation}
    where $\alpha = 1, \ldots n_j$.

    The density operator for mixed states is~\eqref{mix}
    \begin{equation*}
        \rho_{mc} = \sum_{\alpha=1}^{n_j} p_\alpha \ket{\psi_{j, \beta}} \bra{\psi_{j, \beta}} ~,
    \end{equation*}
    where $p_\alpha$ is the probability for the eigenstate $\ket{\psi_{j, \beta}}$. Since $E = E_j$ is fixed, all the eigenstates have the same probability to occur. Therefore $p_\alpha = \frac{1}{n_j}$ and 
    \begin{equation*}
        \rho_{mc} = \frac{1}{n_j} \sum_{\alpha=1}^{n_j} \ket{\psi_{j, \alpha}} \bra{\psi_{j, \alpha}} = \frac{1}{n_j} \hat P_j ~,
    \end{equation*}
    where 
    \begin{equation*}
        P_j = \sum_{\alpha=1}^{n_j} \ket{\psi_{j, \alpha}} \bra{\psi_{j, \alpha}}
    \end{equation*} 
    is the projector onto the energy eigenspace. Notice that we can expand the hamiltonian using~\eqref{spec}
    \begin{equation}\label{endec}
        \hat H = \sum_j E_j \hat P_j ~.
    \end{equation}

    The average of an observable $\hat A$ in the microcanonical ensemble is 
    \begin{equation*}
        \av{A}_{mc} = \frac{1}{n_j} \sum_{\alpha=1}^{n_j} \bra{\psi_{n,\alpha}} \hat A \ket{\psi_{n,\alpha}} ~.
    \end{equation*}
    \begin{proof}
        In fact, choosing an orthonormal basis $\ket{e_j}$, the trace is 
        \begin{equation*}
            \tr_{\mathcal H_{tot}} \hat A = \sum_j \bra{e_j} \hat A \ket{e_j} ~.
        \end{equation*}
        Therefore, using~\eqref{avobs}
        \begin{equation*}
        \begin{aligned}
            \av{A}_{mc} & = \tr_{\mathcal H_{tot}} (\hat A \rho_{mc}) \\ & = \tr_{\mathcal H_{tot}} \Big ( \hat A \frac{1}{n_j} \sum_{\alpha=1}^{n_j} \ket{\psi_{j, \alpha}} \bra{\psi_{j, \alpha}} \Big) \\ & = \frac{1}{n_j} \sum_{\alpha=1}^{n_j} \tr_{\mathcal H_{tot}} \Big ( \hat A \ket{\psi_{j, \alpha}} \bra{\psi_{j, \alpha}} \Big) \\ & = \frac{1}{n_j} \sum_{\alpha=1}^{n_j} \bra{\psi_{j, \alpha}} \hat A \ket{\psi_{j, \alpha}} ~.
        \end{aligned}
        \end{equation*}
    \end{proof}

    The entropy in the microcanonical ensemble is 
    \begin{equation*}
        S_{mc} = k_B \log n_j ~,
    \end{equation*}
    where $n_j$ is the number of states with $E = E_j$. Notice that it is similar to the classical case~\eqref{entropymc}.
    \begin{proof}
        In fact, using~\eqref{unboltz}
        \begin{equation*}
        \begin{aligned}
            S_{mc} = - k_B \av{\log \rho_{mc}}_{mc} = - k_B \tr_{\mathcal H_{tot}} ( \rho_{mc} \log \rho_{mc}) ~.
        \end{aligned}
        \end{equation*}

        In matrix notation, the density operator is 
        \begin{equation*}
        \begin{aligned}
            \rho_{mc} & = \begin{bmatrix}
                \begin{bmatrix}
                    \frac{1}{n_1} & 0 & \ldots & 0 \\
                    0 & \frac{1}{n_1} & \ldots & 0 \\
                    \ldots & \ldots & \ldots & \ldots \\
                    0 & 0 & \ldots & \frac{1}{n_1} \\
                \end{bmatrix} & 0 & \ldots & 0 & \ldots & \ldots \\ 0 & 
                \begin{bmatrix}
                    \frac{1}{n_2} & 0 & \ldots & 0 \\
                    0 & \frac{1}{n_2} & \ldots & 0 \\
                    \ldots & \ldots & \ldots & \ldots \\
                    0 & 0 & \ldots & \frac{1}{n_2} \\
                \end{bmatrix} & \ldots & 0 & \ldots & \ldots \\ 
                \ldots & \ldots & \ldots & \ldots & \ldots & \ldots \\
                0 & 0 & \ldots & \begin{bmatrix}
                    \frac{1}{n_j} & 0 & \ldots & 0 \\
                    0 & \frac{1}{n_j} & \ldots & 0 \\
                    \ldots & \ldots & \ldots & \ldots \\
                    0 & 0 & \ldots & \frac{1}{n_j} \\
                \end{bmatrix} & \ldots & \ldots \\
                \ldots & \ldots & \ldots & \ldots & \ldots & \ldots \\
                \ldots & \ldots & \ldots & \ldots & \ldots & \ldots \\
            \end{bmatrix} \\ & = \sum_j \begin{bmatrix}
                0 & 0 & \ldots & 0 & \ldots & \ldots \\ 
                0 & 0 & \ldots & 0 & \ldots & \ldots \\ 
                \ldots & \ldots & \ldots & \ldots & \ldots & \ldots \\
                0 & 0 & \ldots & \begin{bmatrix}
                    \frac{1}{n_j} & 0 & \ldots & 0 \\
                    0 & \frac{1}{n_j} & \ldots & 0 \\
                    \ldots & \ldots & \ldots & \ldots \\
                    0 & 0 & \ldots & \frac{1}{n_j} \\
                \end{bmatrix} & \ldots & \ldots \\
                \ldots & \ldots & \ldots & \ldots & \ldots & \ldots \\
            \end{bmatrix}
        \end{aligned} ~.
        \end{equation*}

        In order to compute the logarithm of $0$, we use a trick: we define a small parameter $\epsilon$ and we make it go to zero. In this way, the limit becomes $\epsilon \log \epsilon \xrightarrow{\epsilon \rightarrow 0} = 0$. Finally, we compute the trace 
        \begin{equation*}
        \begin{aligned}
            \tr_{\mathcal H_{tot}} ( \rho_{mc} \log \rho_{mc}) & = \tr \begin{bmatrix}
                0 & 0 & \ldots & 0 & \ldots & \ldots \\ 
                0 & 0 & \ldots & 0 & \ldots & \ldots \\ 
                \ldots & \ldots & \ldots & \ldots & \ldots & \ldots \\
                0 & 0 & \ldots & \begin{bmatrix}
                    \frac{1}{n_j} \log \frac{1}{n_j} & 0 & \ldots & 0 \\
                    0 & \frac{1}{n_j} \log \frac{1}{n_j} & \ldots & 0 \\
                    \ldots & \ldots & \ldots & \ldots \\
                    0 & 0 & \ldots & \frac{1}{n_j} \log \frac{1}{n_j} \\
                \end{bmatrix} & \ldots & \ldots \\
                \ldots & \ldots & \ldots & \ldots & \ldots & \ldots \\
            \end{bmatrix} \\ & = \sum_j \frac{1}{n_j} \log \frac{1}{n_j} = n_j \frac{1}{n_j} \log \frac{1}{n_j} = - \log n_j ~.
        \end{aligned}
        \end{equation*}
        Hence, 
        \begin{equation*}
            S_{mc} = - k_B \tr_{\mathcal H_{tot}} ( \rho_{mc} \log \rho_{mc}) = k_B \log n_j ~.
        \end{equation*}
    \end{proof}

    Notice that entropy is always a positive function, since there is at least one state occupied $n_j \geq 1$, which implies $S \geq 0$.

\chapter{Canonical ensemble}

    The canonical ensemble is characterised by constant volume, temperature and number of particle. Energy, which can be exchange in an external reservoir, can be in one of the eigenstates~\eqref{eneigen} with probability 
    \begin{equation}\label{prob}
        p_j \propto \exp(- \beta E_j) ~.
    \end{equation}

    Consider a family of projectors $\{\hat P_j\}$, the density matrix of a mixed states is 
    \begin{equation*}
        \rho_c = \frac{1}{Z_N } \sum_j \exp(- \beta E_j) \hat P_j = \frac{\exp(- \beta \hat H)}{Z_N} ~,
    \end{equation*}
    where the quantum canonical partition function is 
    \begin{equation*}
        Z_N(T,V) = \tr_{\mathcal H_{tot}} \Big ( \frac{\exp(- \beta \hat H)}{Z_N} \Big) ~.
    \end{equation*}
    \begin{proof}
        For a mixed state, the density matrix is~\eqref{mix}
        \begin{equation*}
            \rho_c = \sum_j p_j \hat P_j = C \sum_j \exp(- \beta E_j) \hat P_J ~,
        \end{equation*}
        where the probability is given by~\eqref{prob} and $C$ is a normalisation function.

        Moreover, using~\eqref{endec}
        \begin{equation*}
        \begin{aligned}
            \rho_c & = C \sum_j \exp(- \beta E_j) \hat P_J \\ & = C \sum_j \sum_k \frac{1}{k!} (-\beta E_j)^k \underbrace{\hat P_j}_{(P_j)^k} \\ & = C \sum_j \sum_k \frac{1}{k!} (-\beta E_j \hat P_j)^k \\ & = C \sum_k \frac{1}{k!} (-\beta \sum_j E_j \hat P_j)^k \\ & = C \exp(- \beta \underbrace{\sum_j E_j \hat P_j}_{\hat H}) \\ & = C \exp(- \beta \hat H) ~,
        \end{aligned}
        \end{equation*}
        where we have used the Taylor expansion of the exponential, one of the properties of the projectors~\eqref{idem} and we have exchanged the two series.

        Finally, We set $C = \frac{1}{Z_N}$, where $Z_N$ is the quantum canonical partition function, and by the normalisation condition 
        \begin{equation*}
            1 = \tr_{\mathcal H_{tot}} \rho_c = \frac{1}{Z_N} \tr_{\mathcal H_{tot}} \exp(- \beta \hat H) ~,
        \end{equation*}
        hence 
        \begin{equation*}
            Z_N = \tr_{\mathcal H_{tot}} \exp(- \beta \hat H) ~.
        \end{equation*}
    \end{proof}

    We define the Helmoltz free energy
    \begin{equation*}
        Z_N = \exp(- \beta F) ~,
    \end{equation*}
    or equivalently 
    \begin{equation*}
        F = - \frac{1}{\beta} \log Z_N ~.
    \end{equation*}
    The average energy is 
    \begin{equation*}
        E = \av{\hat H}_c = - \pdv{}{\beta} \log Z_N ~.
    \end{equation*}
    \begin{proof}
        In fact, 
        \begin{equation*}
        \begin{aligned}
            E & = \av{\hat H}_c \\ & = \tr_{\mathcal H_{tot}} (\hat H \rho_c) \\ & = \tr_{\mathcal H_{tot}} \Big ( \hat H \frac{\exp(- \beta \hat H)}{Z_N} \Big ) \\ & = \frac{1}{Z_N} \tr_{\mathcal H_{tot}} \Big (- \pdv{}{\beta} \exp(- \beta \hat H) \Big) \\ & = - \frac{1}{Z_N} \pdv{}{\beta} \underbrace{\tr_{\mathcal H_{tot}} \exp(- \beta \hat H)}_{Z_N} \\ & = - \frac{1}{Z_N} \pdv{}{\beta} Z_N \\ & = - \pdv{}{\beta} \log Z_N ~.
        \end{aligned}
        \end{equation*}
    \end{proof}

    The entropy is 
    \begin{equation*}
        S = \frac{E - F}{T} = \pdv{F}{T} ~.
    \end{equation*}
    \begin{proof}
        In fact, using~\eqref{unboltz}
        \begin{equation*}
        \begin{aligned}
            S_c & = - k_B \av{\log \rho_c}_c \\ & = - k_B \tr_{\mathcal H_{tot}} (\rho_c \log \rho_c) \\ & = - k_B \tr_{\mathcal H_{tot}} (\frac{\exp(- \beta \hat H)}{Z_N} \log \frac{\exp(- \beta \hat H)}{Z_N}) \\ & = - k_B \tr_{\mathcal H_{tot}} \Big (\frac{\exp(- \beta \hat H)}{Z_N} (\log \exp(- \beta \hat H) - \log Z_N) \Big ) \\ & = - k_B \tr_{\mathcal H_{tot}} (\frac{\exp(- \beta \hat H)}{Z_N} (- \beta \hat H - \log Z_N)) \\ & = k_B \beta ~ \underbrace{\tr_{\mathcal H_{tot}} (\frac{\exp(- \beta \hat H)}{Z_N} \hat H )}_E + k_B \tr_{\mathcal H_{tot}} (\frac{\exp(- \beta \hat H)}{Z_N} \underbrace{\log Z_N}_{- \beta F} ) \\ & = \frac{E}{T} - k_B \beta F ~ \frac{1}{Z_N} \underbrace{\tr_{\mathcal H_{tot}} (\exp(- \beta \hat H))}_{Z_N} \\ & = \frac{E-F}{T} ~.
        \end{aligned}
        \end{equation*}
    \end{proof}
    Notice that the entropy is well defined because the trace of the exponential of the energy eigenvalues diverges only if they are negative. Thus, we assume that $E_j \geq \min E_j = 0$.

\chapter{Grancanonical ensemble}

    The grancanonical ensemble is characterised by constant volume, temperature and chemical potential. Since $N$ is not fixed, we work in the full Fock space $\mathcal F_N$. However, we restrict the hamiltonian operator in the Fock space to the condition that it conserves the number of particles, i.e. $[\hat H, \hat N] = 0$ 
    \begin{equation*}
        \hat H \Big \vert_{\mathcal F_N} = \hat H_N ~.
    \end{equation*}
    An example of physical system which does not satisfy this condition is a photons absorbed by an electron. Energy can be in one of the eigenstates, each for a fixed $N$
    \begin{equation*}
        \hat H^{(N)} \ket{\psi_{j, \alpha}^{(N)}} = E_j^{(N)} \ket{\psi_{j, \alpha}^{(N)}} ~,
    \end{equation*}
    with probability 
    \begin{equation}\label{prob2}
        p_j^{(N)} \propto \exp(- \beta (E_j - \mu N)) ~.
    \end{equation}

    Consider a family of projectors $\{\hat P_j^{(N)}\}$
    \begin{equation*}
        \hat P_j^{N} = \sum_\alpha \ket{\psi_{j, \alpha}^{(N)}} \bra{\psi_{j, \alpha}^{(N)}} ~,
    \end{equation*}  
    the density matrix of a mixed states is 
    \begin{equation*}
        \rho_{gc} = \frac{1}{\mathcal Z} \sum_N \sum_j \exp(- \beta (E_j - \mu N)) \hat P_j^{(N)} = \frac{\exp(- \beta (\hat H - \mu \hat N))}{\mathcal Z} ~,
    \end{equation*}
    where $z = \exp(\beta \mu)$ is the fugacity and the quantum grancanonical partition function is 
    \begin{equation*}
        \mathcal Z = \sum_{N=0}^{\infty} \tr_{\mathcal H_{tot}} \Big ( \exp(- \beta (\hat H - \mu \hat N)) \Big) = \sum_{N=0}^\infty z^N Z_N ~.
    \end{equation*}
    \begin{proof}
        For a mixed state, the density matrix is~\eqref{mix}
        \begin{equation*}
            \rho_{gc} = \sum_N \sum_j p_j \hat P_j^{(N)} = C \sum_N \sum_j \exp(- \beta (E_j^{(N)} - \mu N)) \hat P_j^{(N)} ~,
        \end{equation*}
        where the probability is given by~\eqref{prob2} and $C$ is a normalisation function.

        Moreover, using~\eqref{endec} and~\eqref{numb}
        \begin{equation*}
        \begin{aligned}
            \rho_{gc} & = C \sum_N \sum_j \exp(- \beta (E_j - \mu N)) \hat P_j^{(N)} \\ & = C \sum_N \sum_j \sum_k \frac{1}{k!} (-\beta (E_j^{(N)} - \mu N))^k \underbrace{\hat P_j^{(N)}}_{(P_j^{(N)})^k} \\ & = C \sum_j \sum_k \frac{1}{k!} (-\beta (E_j^{(N)} \hat P_j^{(N)} - \nu N P_j^{(N)}))^k \\ & = C \sum_k \frac{1}{k!} (-\beta \sum_N \sum_j (E_j^{(N)} \hat P_j^{(N)} - \mu N P_j^{(N)}))^k \\ & = C \exp(- \beta (\underbrace{\sum_j \sum_N E_j^{(N)} \hat P_j^{(N)}}_{\hat H}) - \mu \underbrace{\sum_j \sum_N N \hat P_j^{(N)}}_{\hat N}) \\ & = C \exp(- \beta (\hat H - \mu \hat N)) ~,
        \end{aligned}
        \end{equation*}
        where we have used the Taylor expansion of the exponential, one of the properties of the projectors~\eqref{idem} and we have exchanged the two series.

        Finally, We set $C = \frac{1}{\mathcal Z}$, where $\mathcal Z$ is the quantum canonical partition function, and by the normalisation condition 
        \begin{equation*}
            1 = \tr_{\mathcal F} \rho_{gc} = \sum_N \frac{1}{\mathcal H_{tot}} \tr_{\mathcal F} \exp(- \beta (\hat H - \mu \hat N)) ~,
        \end{equation*}
        hence 
        \begin{equation*}
            \mathcal Z = \tr_{\mathcal F} \exp(- \beta (\hat H - \mu \hat N)) = \sum_{N=0}^{\infty} \tr_{\mathcal H_{tot}} \exp(- \beta (\hat H - \mu \hat N)) = \sum_{N=0}^{\infty} z^N \underbrace{\tr_{\mathcal H_{tot}} \exp(- \beta \hat H)}_{Z_N} = \sum_N z^N Z_N ~.
        \end{equation*}
    \end{proof}

    Consider an observable $\hat A$ such that it conserves the number of particles, i.e. $[\hat A, \hat N]$, the average value is 
    \begin{equation*}
        \av{\hat A}_{gc} = \tr_{\mathcal F} (\hat A \rho_{gc}) = \frac{1}{\mathcal Z} \sum_{N=0}^{\infty} z^N Z_N \av{\hat A}_c ~.
    \end{equation*}
    \begin{proof}
        In fact, 
        \begin{equation*}
        \begin{aligned}
            \av{\hat A}_{gc} & = \tr_{\mathcal F} (\hat A \rho_{gc}) \\ & = \sum_{N=0}^{\infty} \tr_{\mathcal H_{tot}} \Big (\hat A \frac{z^N \exp(- \beta \hat H)}{\mathcal Z}) = \frac{1}{\mathcal Z} \sum_{N=0}^{\infty} z^N \tr_{\mathcal H_{tot}} (\hat A \exp(- \beta \hat H)) \\ & = \frac{1}{\mathcal Z} \sum_{N=0}^{\infty} z^N Z_N \underbrace{\frac{\tr_{\mathcal H_{tot}} (\hat A \exp(- \beta \hat H))}{Z_N}}_{\av{\hat A}_c} \\ & = \frac{1}{\mathcal Z} \sum_{N=0}^{\infty} z^N Z_N \av{\hat A}_c ~.
        \end{aligned}
        \end{equation*}
    \end{proof}
    
    We define the granpotential 
    \begin{equation*}
        \Omega = - \frac{1}{\beta} \log \mathcal Z ~,
    \end{equation*}
    the energy in the grancanonical is 
    \begin{equation*}
        E - \mu N = \av{\hat H - \mu \hat N} = - \pdv{}{\beta} \log \mathcal Z ~.
    \end{equation*}
    \begin{proof}
        In fact 
        \begin{equation*}
        \begin{aligned}
            E - \mu N & = \av{\hat H - \mu \hat N} \\ & = \tr_{\mathcal F} \Big ( (\hat H - \mu \hat N) \frac{\exp( - \beta (\hat H - \mu \hat N))}{\mathcal Z} \Big) \\ & = - \frac{1}{\mathcal Z} \pdv{}{\beta} \underbrace{\tr_{\mathcal F} (\exp(- \beta (\hat H - \mu \hat N)))}_{\mathcal Z} \\ & = - \frac{1}{\mathcal Z} \pdv{}{\beta} \mathcal Z \\ & = - \pdv{}{\beta} \log \mathcal Z ~.
        \end{aligned}
        \end{equation*}
    \end{proof}

    The entropy in the grancanonical ensemble is 
    \begin{equation*}
        S = \frac{E - \mu N - \Omega}{T} ~.
    \end{equation*}
    \begin{proof}
        In fact 
        \begin{equation*}
        \begin{aligned}
            S & = - k_B \av{\log \rho_{gc}}_{gc} \\ & = - k_B \tr_{\mathcal F} ( \rho_{gc} \log \rho_{gc}) \\ & = - k_B \tr_{\mathcal F} \Big ( \frac{\exp(- \beta (\hat H - \mu \hat N))}{\mathcal Z} \log \frac{\exp(- \beta (\hat H - \mu \hat N))}{\mathcal Z} \Big) \\ & = - k_B \tr_{\mathcal F} \Big ( \frac{\exp(- \beta (\hat H - \mu \hat N))}{\mathcal Z} (\log \exp(- \beta (\hat H - \mu \hat N)) - \log \mathcal Z) \Big) \\ & = k_B \beta \underbrace{\tr_{\mathcal F} \frac{\exp(- \beta (\hat H - \mu \hat N))}{\mathcal Z} (\hat H - \mu \hat N)}_{E - \mu N} + k_B \underbrace{\tr_{\mathcal F} \log \mathcal Z }_{- \beta \Omega} \\ & = \frac{E - \mu N - \Omega}{T} ~.
        \end{aligned}
        \end{equation*}
    \end{proof}
\part{Application of quantum statistical mechanics}

\chapter{Quantum ensemble}

\section{Magnetic 1/2-spin}

    Consider a system composed by $N$ distinguishable magnetic dipoles in an external magnetic field along the $z$-axis, with spin $S = 1/2$. Its hamiltonian is 
    \begin{equation*}
        \hat H = \sum_i S^{(z)}_i B ~,
    \end{equation*}
    where
    \begin{equation*}
        S^{(z)}_i = \frac{1}{2} \begin{bmatrix}
            1 & 0 \\
            0 & -1 \\
        \end{bmatrix} ~.
    \end{equation*}

    The canonical partition function is 
    \begin{equation*}
        Z = \Big ( 2 \cosh \frac{\beta B}{2} \Big)^N ~.
    \end{equation*}
    \begin{proof}
        By definition, for distinguishable particles,
        \begin{equation*}
        \begin{aligned}
            Z & = (Z_1)^N \\ & = \Big (\tr_{\mathcal H} \exp(- \beta \hat H_1) \Big)^N \\ & = \Big (\tr_{\mathcal H} \exp(- \beta B \begin{bmatrix}
                \frac{1}{2} & 0 \\ 0 & - \frac{1}{2} \\ 
            \end{bmatrix}) \Big)^N \\ & = \Big(\tr_{\mathcal H} \begin{bmatrix}
                \exp(- \frac{\beta B}{2}) & 0 \\ 0 & \exp(\frac{\beta B}{2}) \\ 
            \end{bmatrix} \Big )^N \\ & = \Big( \exp(- \frac{\beta B}{2}) + \exp(\frac{\beta B}{2}) \Big)^N \\ & = \Big ( 2 \cosh \frac{\beta B}{2} \Big)^N ~.
        \end{aligned}
        \end{equation*}
    \end{proof}

    The Helmoltz free energy $F$ is 
    \begin{equation*}
        F = - N k_B T \ln \Big ( 2 \cosh \frac{\beta B}{2} \Big) ~.
    \end{equation*}
    \begin{proof}
        By definition, 
        \begin{equation*}
            F = - \frac{\ln Z}{\beta} = - N k_B T \ln \Big ( 2 \cosh \frac{\beta B}{2} \Big) ~.
        \end{equation*}
    \end{proof}

    The internal energy $E$ is 
    \begin{equation*}
        E = - N \frac{B}{2} \tanh \frac{\beta B}{2} ~.
    \end{equation*}
    \begin{proof}
        By definition, 
        \begin{equation*}
            E = - \pdv{\ln Z}{\beta} = - N \pdv{\beta} \ln \Big (2 \cosh \frac{\beta B}{2} \Big) = - N \frac{B}{2} \tanh \frac{\beta B}{2} ~.
        \end{equation*}
    \end{proof}
    A plot of this is in Figure~\ref{qm:e}.
    \begin{figure}
        \centering
        \scalebox{0.7}{\pyc{plot1('x', '- ( tanh (1 / x))', 4, 1, 16, True, True, False)}}
        \caption{A plot of the energy $E$ as a function of $T$. We have used $x = \frac{2 k_B T}{B} $ and $f(x) = \frac{2E}{BN}$.}
        \label{qm:e}
    \end{figure}

    The magnetisation $M$ is 
    \begin{equation*}
        M = - \frac{N}{2} \tanh \frac{\beta B}{2} ~. 
    \end{equation*}
    \begin{proof}
        By definition, 
        \begin{equation*}
            M = \pdv{F}{B} = - N k_B T \pdv{}{B} \ln \Big ( 2 \cosh \frac{\beta B}{2} \Big) = - \frac{N}{2} \tanh \frac{\beta B}{2} ~.
        \end{equation*}
    \end{proof}
    A plot of this is in Figure~\ref{qm:m}.
    \begin{figure}
        \centering
        \scalebox{0.7}{\pyc{plot1('x', '- ( tanh (1 / x))', 4, 1, 17, True, True, False)}}
        \caption{A plot of the magnetisation $M$ as a function of $T$. We have used $x = \frac{2 k_B T}{B} $ and $f(x) = \frac{2M}{N}$.}
        \label{qm:m}
    \end{figure}

\section{Magnetic 1-spin}

    Consider a system composed by $N$ distinguishable magnetic dipoles in an external magnetic field along the $z$-axis, with spin $S = 1$. Its hamiltonian is 
    \begin{equation*}
        \hat H = \sum_i S^{(z)}_i B ~,
    \end{equation*}
    where
    \begin{equation*}
        S^{(z)}_i = \begin{bmatrix}
            1 & 0 & 0 \\
            0 & 0 & 0 \\
            0 & 0 & - 1 \\
        \end{bmatrix} ~.
    \end{equation*}

    The canonical partition function is 
    \begin{equation*}
        Z = \Big ( 2 \cosh (\beta B) + 1 \Big)^N ~.
    \end{equation*}
    \begin{proof}
        By definition, for distinguishable particles,
        \begin{equation*}
        \begin{aligned}
            Z & = (Z_1)^N \\ & = \Big (\tr_{\mathcal H} \exp(- \beta \hat H_1) \Big)^N \\ & = \Big (\tr_{\mathcal H} \exp(- \beta B \begin{bmatrix}
                1 & 0 & 0 \\ 0 & 0 & 0 \\ 0 & 0 & - 1 \\ 
            \end{bmatrix}) \Big)^N \\ & = \Big(\tr_{\mathcal H} \begin{bmatrix}
                \exp(- \beta B) & 0 & 0 \\ 0 & 1 & 0 \\ 0 & 0 & \exp(\beta B) \\ 
            \end{bmatrix} \Big )^N \\ & = \Big( \exp(- \beta B) + 1 + \exp(\beta B) \Big)^N \\ & = \Big ( 2 \cosh (\beta B) + 1 \Big)^N ~.
        \end{aligned}
        \end{equation*}
    \end{proof}

    The Helmoltz free energy $F$ is 
    \begin{equation*}
        F = - N k_B T \ln \Big ( 2 \cosh (\beta B) + 1 \Big) ~.
    \end{equation*}
    \begin{proof}
        By definition, 
        \begin{equation*}
            F = - \frac{\ln Z}{\beta} = - N k_B T \ln \Big ( 2 \cosh (\beta B) + 1 \Big) ~.
        \end{equation*}
    \end{proof}

    The internal energy $E$ is 
    \begin{equation*}
        E = - 2 N B \frac{\sinh (\beta B)}{2 \cosh (\beta B) + 1} ~.
    \end{equation*}
    \begin{proof}
        By definition, 
        \begin{equation*}
            E = - \pdv{\ln Z}{\beta} = - N \pdv{}{\beta} \ln \Big ( 2 \cosh (\beta B) + 1 \Big) = - 2 N B \frac{\sinh (\beta B)}{2 \cosh (\beta B) + 1} ~.
        \end{equation*}
    \end{proof}
    A plot of this is in Figure~\ref{qm:e1}.
    \begin{figure}
        \centering
        \scalebox{0.7}{\pyc{plot1('x', '- (( sinh(1 / x) ) / (2 * cosh (1 /x) + 1))', 4, 1, 18, True, True, False)}}
        \caption{A plot of the energy $E$ as a function of $T$. We have used $x = \frac{k_B T}{B} $ and $f(x) = \frac{2E}{BN}$.}
        \label{qm:e1}
    \end{figure}

    The magnetisation $M$ is 
    \begin{equation*}
        M = - \frac{N}{2} \frac{\sinh (\beta B)}{2 \cosh (\beta B) + 1} ~. 
    \end{equation*}
    \begin{proof}
        By definition, 
        \begin{equation*}
            M = \pdv{F}{B} = - N k_B T \pdv{}{B} \ln \Big ( 2 \cosh (\beta B) + 1 \Big) = - 2 N \frac{\sinh (\beta B)}{2 \cosh (\beta B) + 1}  ~.
        \end{equation*}
    \end{proof}
    A plot of this is in Figure~\ref{qm:m2}.
    \begin{figure}
        \centering
        \scalebox{0.7}{\pyc{plot1('x', '- (( sinh (1 / x) ) / (2 * cosh (1 / x) + 1))', 4, 1, 19, True, True, False)}}
        \caption{A plot of the magnetisation $M$ as a function of $T$. We have used $x = \frac{2 k_B T}{B} $ and $f(x) = \frac{M}{2N}$.}
        \label{qm:m2}
    \end{figure}

\section{Quantum harmonic oscillators}

    Consider a system composed by $N$ distinguishable quantum harmonic oscillators. Its hamiltonian is 
    \begin{equation*}
        \hat H = \sum_i \hbar \omega (\hat a_i \hat a_i^\dagger + \frac{1}{2}) ~.
    \end{equation*}

    The canonical partition function is 
    \begin{equation*}
        Z = \Big ( \frac{\exp(- \frac{\beta \hbar \omega}{2})}{1 - \exp(- \beta \hbar \omega)} \Big)^N ~.
    \end{equation*}
    \begin{proof}
        By definition, for distinguishable particles,
        \begin{equation*}
        \begin{aligned}
            Z & = (Z_1)^N \\ & = \Big (\tr_{\mathcal H} \exp(- \beta \hat H_1) \Big)^N \\ & = \Big (\tr_{\mathcal H_i} \exp(- \beta \hbar \omega (\hat a_i \hat a_i^\dagger + \frac{1}{2})) \Big)^N \\ & = \Big ( \sum_i \bra{n_i}\exp(- \beta \hbar \omega (\hat a_i \hat a_i^\dagger + \frac{1}{2})) \ket{n_i }\Big)^N \\ & = \Big ( \sum_i \exp(- \beta \hbar \omega (n_i + \frac{1}{2})) \Big)^N \\ & = \Big ( \exp(- \frac{\beta \hbar \omega}{2}) \sum_i \exp(- \beta \hbar \omega)^{n_i} \Big)^N \\ & = \Big ( \exp(- \frac{\beta \hbar \omega}{2}) \frac{1}{1 - \exp(- \beta \hbar \omega)} \Big)^N
            \\ & = \Big ( \frac{\exp(- \frac{\beta \hbar \omega}{2})}{1 - \exp(- \beta \hbar \omega)} \Big)^N ~.
        \end{aligned}
        \end{equation*}
    \end{proof}

    The Helmoltz free energy $F$ is 
    \begin{equation*}
        F = N k_B T ( \frac{\beta \hbar \omega}{2} + \ln (1 - \exp(- \beta \hbar \omega))) ~.
    \end{equation*}
    \begin{proof}
        By definition, 
        \begin{equation*}
        \begin{aligned}
            F & = - \frac{\ln Z}{\beta} \\ & = - N k_B T \ln \Big ( \frac{\exp(- \frac{\beta \hbar \omega}{2})}{1 - \exp(- \beta \hbar \omega)} \Big) \\ & = - N k_B T ( \ln \exp(- \frac{\beta \hbar \omega}{2}) - \ln (1 - \exp(- \beta \hbar \omega) )) \\ & = - N k_B T ( - \frac{\beta \hbar \omega}{2} - \ln (1 - \exp(- \beta \hbar \omega))) \\ & = N k_B T ( \frac{\beta \hbar \omega}{2} + \ln (1 - \exp(- \beta \hbar \omega))) ~.
        \end{aligned}
        \end{equation*}
    \end{proof}

    The internal energy $E$ is 
    \begin{equation*}
        E = N ( \frac{\hbar \omega}{2} + \frac{\hbar \omega}{\exp(- \beta \hbar \omega) - 1} ) ~.
    \end{equation*}
    \begin{proof}
        By definition, 
        \begin{equation*}
        \begin{aligned}
            F & = - \pdv{\ln Z}{\beta} \\ & = - N \pdv{}{\beta} \ln \Big ( \frac{\exp(- \frac{\beta \hbar \omega}{2})}{1 - \exp(- \beta \hbar \omega)} \Big) \\ & = - N \pdv{}{\beta} ( \ln \exp(- \frac{\beta \hbar \omega}{2}) - \ln (1 - \exp(- \beta \hbar \omega) )) \\ & = - N \pdv{}{\beta} ( - \frac{\beta \hbar \omega}{2} - \ln (1 - \exp(- \beta \hbar \omega))) \\ & = N \pdv{}{\beta} ( \frac{\beta \hbar \omega}{2} + \ln (1 - \exp(- \beta \hbar \omega))) \\ & = N ( \frac{\hbar \omega}{2} - \frac{\hbar \omega}{1 - \exp(- \beta \hbar \omega)} ) \\ & = N ( \frac{\hbar \omega}{2} + \frac{\hbar \omega}{\exp(- \beta \hbar \omega) - 1} ) ~.
        \end{aligned}
        \end{equation*}
    \end{proof}

    The specific heat is 
    \begin{equation*}
        C_V = N \frac{\hbar^2 \omega^2}{k_B T^2} \frac{\exp(\beta \hbar \omega)}{(\exp(\beta \hbar \omega) - 1)^2} ~.
    \end{equation*}
    \begin{proof}
        In fact 
        \begin{equation*}
            C_V = \pdv{E}{T} = N \pdv{}{T} ( \frac{\hbar \omega}{2} + \frac{\hbar \omega}{\exp(- \beta \hbar \omega) - 1} ) = N \frac{\hbar^2 \omega^2}{k_B T^2} \frac{\exp(\beta \hbar \omega)}{(\exp(\beta \hbar \omega) - 1)^2} ~.
        \end{equation*}
    \end{proof}

\chapter{Fermions}

\section{White dwarf}

    A white dwarf is an helium star woth mass $M \sim 10^{30} kg$ and a density of $\rho = 10^{10} kg/m^3$ at a temperature of $10^{7} K$. Our approxiated model is composed by $N$ electrons and $N/2$ helium nuclei.

    Assuming $M = N(m_e + 2 m_p) \sim 2 N m_p$, the electronic dentity is 
    \begin{equation*}
        n = 3 \times 10^{36} m^{-3} ~.
    \end{equation*}
    \begin{proof}
        In fact 
        \begin{equation*}
            n = \frac{N}{V} = \frac{N \rho}{M} = \frac{N \rho}{2 N m_p} = \frac{\rho}{2 m_p} = \frac{10^{10}}{2 \times 1.6 \times 10^{-27}} = 3 \times 10^{36} m^{-3} ~.
        \end{equation*}
    \end{proof}

    The Fermi momentum $p_F$ is 
    \begin{equation*}
        p_F = h \Big ( \frac{3 n}{4 \pi g} \Big)^{1/3} = 6.63 \times 10^{-34}  \times \Big ( \frac{3 \times 10^{10}}{4 \times 3.14 \times 2} \Big)^{1/3} = 0.88 Mev/c ~.
    \end{equation*}
    \begin{proof}
        In fact, using $p = \hbar k$
        \begin{equation*}
            N = g \sum_{n} \rightarrow g \frac{V}{(2\pi)^3} \int d^3 k = g \frac{V}{(2\pi \hbar)^3} \int d^3 p = g \frac{4 \pi V}{(2\pi \hbar)^3} \int_0^{p_F} dp p^2 = g \frac{4 \pi V}{(2\pi \hbar)^3} \frac{p_F^3}{3} ~,
        \end{equation*}
        hence 
        \begin{equation*}
            p_F = h \Big ( \frac{3 n}{4 \pi g} \Big)^{1/3} = 6.63 \times 10^{-34}  \times \Big ( \frac{3 \times 10^{10}}{4 \times 3.14 \times 2} \Big)^{1/3} = 0.88 Mev/c ~.
        \end{equation*}
    \end{proof}

    The Fermi energy $\epsilon_F$ is 
    \begin{equation*}
        \epsilon_F = \sqrt{(p_F c)^2 + (mc^2)^2} - mc^2 = 0.5 Mev ~.
    \end{equation*}

    The Fermi temperature $T_F$ is 
    \begin{equation*}
        T_F = \frac{\epsilon_F}{k_B} = 10^{10} K ~,
    \end{equation*}
    which means that we are in the regime $T \ll T_F$ and we can use $T=0$.

    The internal energy $E$ is 
    \begin{equation*}
        E = \frac{\pi V m^4 c^5}{\pi^2 \hbar^3} f(x_F) ~.
    \end{equation*}
    \begin{proof}
        In fact,
        \begin{equation*}
        \begin{aligned}
            E & = g \sum_{n} \epsilon \rightarrow g \frac{V}{(2\pi)^3} \int d^3 k \epsilon \\ & = g \frac{V}{(2\pi \hbar)^3} \int d^3 p \epsilon \\ & = g \frac{4 \pi V}{(2\pi \hbar)^3} \int_0^{p_F} dp p^2 \epsilon \\ & = g \frac{4 \pi V}{(2\pi \hbar)^3} \int_0^{p_F} dp p^2 c \sqrt{p^2 + (mc)^2} ~.
        \end{aligned}
        \end{equation*}

        Now we make a change of variable 
        \begin{equation*}
            x = \frac{p}{mc} ~, \quad dp = mc dx ~,
        \end{equation*}
        hence 
        \begin{equation*}
        \begin{aligned}
            E & = g \frac{4 \pi V}{(2\pi \hbar)^3} c (mc)^3 \int_0^{x_F} dx ~ x^2 (mc)\sqrt{x^2 + 1} \\ & =  \frac{4 g \pi V m^4 c^5}{h^3} \int_0^{x_F} dx ~ x^2\sqrt{x^2 + 1} \\ & = \frac{4 g \pi V m^4 c^5}{h^3} f(x_F) \\ & = \frac{V m^4 c^5}{\pi^2 \hbar^3} f(x_F) ~,
        \end{aligned}
        \end{equation*}
        where 
        \begin{equation*}
            f(x_F) = \int_0^{x_F} dx ~ x^2 \sqrt{x^2 + 1} ~.
        \end{equation*}
    \end{proof}

    The pressure $P$ is 
    \begin{equation*}
        P = \frac{m^4 c^5}{\pi^2 \hbar^3} \Big (\frac{x_F^3}{3} \sqrt{1 + x_F^2} - f(x_F)) ~.
    \end{equation*}
    \begin{proof}
        In fact,
        \begin{equation*}
        \begin{aligned}
            P & = - \pdv{E}{V} \\ & = - \pdv{}{V} \frac{ V m^4 c^5}{\pi^2 \hbar^3} f(x_F) \\ & = - \frac{\pi m^4 c^5}{\pi^2 \hbar^3} f(x_F) - \frac{ V m^4 c^5}{\pi^2 \hbar^3} \pdv{x_F}{V} \pdv{f(x_F)}{x_F} \\ & = - \frac{ m^4 c^5}{\pi^2 \hbar^3} f(x_F) - \frac{ V m^4 c^5}{\pi^2 \hbar^3} \pdv{}{V} \Big (\frac{h}{mc} \Big ( \frac{3 N}{4 \pi g V} \Big)^{1/3}) \pdv{f(x_F)}{x_F} \\ & = - \frac{m^4 c^5}{\pi^2 \hbar^3} f(x_F) - \frac{V m^4 c^5}{\pi^2 \hbar^3} \Big (\frac{h}{mc} \Big ( \frac{3 N}{4 \pi g V} \Big)^{1/3}) \pdv{}{V} V^{-1/3} \pdv{f(x_F)}{x_F} \\ & = - \frac{m^4 c^5}{\pi^2 \hbar^3} f(x_F) + \frac{1}{3} \frac{V m^4 c^5}{\pi^2 \hbar^3} \Big (\frac{h}{mc} \Big ( \frac{3 N}{4 \pi g V} \Big)^{1/3}) V^{-4/3} \pdv{f(x_F)}{x_F} \\ & = \frac{m^4 c^5}{\pi^2 \hbar^3} \Big (\frac{x_F^3}{3} \sqrt{1 + x_F^2} - f(x_F)) ~.
        \end{aligned}
        \end{equation*}
    \end{proof}

    Now, we solve the integral 
    \begin{equation*}
        f(x) = \py{indint('x**2 * sqrt(x**2 + 1)', 'x')} ~.
    \end{equation*}
    In the non-relativistic limit, $x_F \ll 1$, we can make the approximations 
    \begin{equation*}
        g(x) = \frac{x^3}{3} \sqrt{1 + x^2} = \py{Taylor('x', 'x**3 / 3 * sqrt(1 + x**2)', 0, 6)}
    \end{equation*}
    and 
    \begin{equation*}
        f(x_F) = \py{Taylor('x', 'x**5 / (4 * sqrt(x**2 + 1)) + ( 3 * x**3) / (8 * sqrt(x**2 + 1)) + x / (8 * sqrt(x**2 + 1)) - asinh(x) / 8', 0, 6)} ~.
    \end{equation*}

    In the ultra-relativistic limit, $x_F \gg 1$ or equivalemty $y_F = 1 / x_F \ll 1$, we can make the approximations 
    \begin{equation*}
        g(1/x) = \py{Taylor('x', '(1 / x)**3 / 3 * sqrt(1 + (1 / x)**2)', 0, -1)}
    \end{equation*}
    and 
    \begin{equation*}
        f(1 / x) = \py{Taylor('x', '(1 / x)**5 / (4 * sqrt((1 / x)**2 + 1)) + ( 3 * (1 / x)**3) / (8 * sqrt((1 / x)**2 + 1)) + (1 / x) / (8 * sqrt((1 / x)**2 + 1)) - (ln ( 1 / x + sqrt((1/x)**2 + 1))) / 8', 0, -1)} ~.
    \end{equation*}
    
    Imposing the equilibrium condition $dE = 0$, between the gravitational and the pressure forces, and the structure of a sphere, the pressure must be 
    \begin{equation*}
        P = \frac{\alpha G M^2}{4 \pi R^4}
    \end{equation*}
    and the Fermi momentum is 
    \begin{equation*}
        p_F = \frac{\hbar}{R} \Big ( \frac{9 \pi M}{8 m_p} \Big)^{1/3} ~.
    \end{equation*}
    \begin{proof}
        For the gravitational force 
        \begin{equation*}
            E_g = - \alpha \frac{G M^2}{R} ~, \quad dE_g = \alpha \frac{GM^2}{R^2} dR ~.
        \end{equation*}
        For the pressure force 
        \begin{equation*}
            E_p = - p V = - p \frac{4}{3} \pi R^3 ~, \quad dE_p = - 4 \pi p R^2 dR ~.
        \end{equation*}
        Imposing the equilibrium condition, 
        \begin{equation*}
            0 = dE = dE_g + dE_p = alpha \frac{GM^2}{R^2} dR - 4 \pi p R^2 dR ~,
        \end{equation*}
        hence 
        \begin{equation*}
            p = \frac{\alpha G M^2}{4 \pi R^4} ~.
        \end{equation*}

        The Fermi momentum is 
        \begin{equation*}
            p_F = h \Big ( \frac{3 n}{4 \pi g} \Big)^{1/3} = h \Big ( \frac{3}{8 \pi} \frac{M}{2 m_p \frac{4}{3} \pi R^3} \Big)^{1/3} = \frac{\hbar}{R} \Big ( \frac{9 \pi M}{8 m_p} \Big)^{1/3}  ~.
        \end{equation*}
    \end{proof}

    In the ultra-relativistic limit
    \begin{equation*}
        P = \frac{m^4 c^5}{12 \pi \hbar^3} (x_F^4 - x_F^2) = \frac{\alpha G M^2}{4 \pi R^4} ~.
    \end{equation*}

\chapter{Bosons}


\part{Phase transition}

\chapter{Classical phase transitions}

    Consider the phase diagram of the water. Microscopically, they all have the same hamiltonian, however, the macroscopical variables changes. There are three phases 
    \begin{enumerate}
        \item solid, i.e. it has its own shape and volume, 
        \item liquid, i.e. it has its own volume but it has the shape of the container,
        \item gas, i.e. it has the shape and volume of the container. 
    \end{enumerate}
    There are lines, called coexistence lines, along which $2$ phases are in equilibrium. They are lines because, other than $T_1 = T_2$ and $p_1 = p_2$, we have a costrain 
    \begin{equation*}
        \mu_1(p, T) = \mu_2 (p, T) ~.
    \end{equation*}
    This reduce to a line. 

    Furthermore, there are points, called coexistence points or triple point, in which $3$ phases are in equailibrium. They are points because, other than $T_1 = T_2 = T_3$ and $p_1 = p_2 = p_3$, we have the costrains
    \begin{equation*}
        \mu_1(p, T) = \mu_2 (p, T) = \mu_3 (p, T) ~.
    \end{equation*}
    This reduce to a point.
    
\section{Symmetries} 

    We can use symmetries of the system to study it. We can distinguish solid from fluid by the translation or rotations invariance. In fact, solid has only discrete invariance, whereas fluid has continuous invariance. However, we cannot distinguish with symmetries between gas and liquid.

\section{Clausius-Clapeyron equation}

    On the coexistence line, the Clausius-Clapeyron equation is 
    \begin{equation*}
        \dv{p}{T} = \frac{s_2 - s_1}{v_1 - v_2} = \frac{\Delta q}{T \Delta v}  ~.
    \end{equation*}
    \begin{proof}
        In order to remain on the coexistence line, the costrain relation holds
        \begin{equation*}
            \mu_1(p, T) = \mu_2 (p, T) ~.
        \end{equation*}
        We differentiate it, keeping in mind that $p = p(T)$
        \begin{equation*}
            \pdv{\mu_1}{T} \Big \vert_p + \dv{\mu_1}{p} \Big \vert_T \dv{p}{T} = \pdv{\mu_2}{T} \Big \vert_p + \dv{\mu_2}{p} \Big \vert_T \dv{p}{T} ~,
        \end{equation*}
        hence 
        \begin{equation*}
            \dv{p}{T} = \frac{\pdv{\mu_1}{T} \vert_p - \pdv{\mu_2}{T} \vert_p}{\pdv{\mu_2}{p} \vert_T - \pdv{\mu_2}{p} \vert_T} ~.
        \end{equation*}

        At fixed number of particle, we can use the Gibbs free energy $G(p, T, N) = \mu (p, T) N$ or the Gibbs free energy per particle 
        \begin{equation*}
            g = \frac{G}{N} = \mu(p,T) ~.
        \end{equation*}
        Using the relations~\eqref{ges}
        \begin{equation*}
            \pdv{\mu}{p} \Big \vert_T = \pdv{g}{p} \Big \vert_T = \frac{1}{N} \pdv{G}{p} \Big \vert_T = \frac{V}{N} = v ~,
        \end{equation*}
        \begin{equation*}
            \pdv{\mu}{T} \Big \vert_p = \pdv{g}{T} \Big \vert_p = \frac{1}{N} \pdv{G}{T} \Big \vert_p = - \frac{S}{N} = - s ~,
        \end{equation*}
        we obtain
        \begin{equation*}
            \dv{p}{T} = \frac{\pdv{\mu_1}{T} \vert_p - \pdv{\mu_2}{T} \vert_p}{\pdv{\mu_2}{p} \vert_T - \pdv{\mu_2}{p} \vert_T} = \frac{s_2 - s_1}{v_2 - v_1} ~.
        \end{equation*}

        Furthermore, when there is a phase change, temperature remains constant whereas the thermal energy put in the system is transformed into latent heat
        \begin{equation*}
            \Delta s = \frac{\Delta q}{T} ~.
        \end{equation*}
        Therefore
        \begin{equation*}
            \dv{p}{T} = \frac{\Delta q}{T \Delta v} ~.
        \end{equation*}
    \end{proof}


    There could be $2$ different kind of phase transitions 
    \begin{enumerate}
        \item $1$st order phase transitions, i.e. those in which the $1$st derivatives of thermodynamic potentials are discontinuous;
        \item continuous phase transitions, i.e. those in which the higher derivatives of thermodynamic potentials are discontinuous.
    \end{enumerate}

    In our case, the former are those in which there is a jump $v_2 \neq v_1$ and $s_2 \neq s_1$ and the latter are those in which $v_2 = v_1$ and $s_2 = s_1$. 

\chapter{Theorems of Lee and Young}

    Consider a classical fluid in a volume $V \subset \mathbb R^3$. We treat it in the grancanonical ensemble. The grancanonical partition function is 
    \begin{equation*}
        \mathcal Z [V, T, z] = \sum_{N=0}^\infty z^n Z_N[T, V] ~.
    \end{equation*}

    The canonical partition function is 
    \begin{equation*}
    \begin{aligned}
        Z_N & = \frac{1}{N! h^{3N}} \int_{V^N} \prod_i d^3 q^i \underbrace{\int_{\mathbb R^{3N}} \prod_i d^3 p^i  \exp (- \beta \sum_j \frac{p_j}{2m}}_{\frac{1}{\lambda_T^{3N}}} + U_N(q^i)) \\ & = \frac{1}{N! \lambda_T^{3N}} \underbrace{\int_{V^N}\prod_i d^3 q^i \exp (- \beta U_N(q^i))}_{Q_N (T, V)} = \frac{Q_N(T, V)}{N! \lambda_T^{3N}} ~.
    \end{aligned}
    \end{equation*}
    Notice that $Q_N(T,V) > 0$.

    Therefore 
    \begin{equation*}
        \mathcal Z = \sum_{N=0}^\infty z^n \frac{Q_N}{N! \lambda_T^{3N}} ~.
    \end{equation*}
    which is a power series in $z$. Now we promote $z$ into a complex variables, keeping in mind that the physical states are only the ones for $z \in \mathbb R^+$

    We make the assumption that $U_N \geq - BN$ with $B > 0$, which means that it grows no more than $N$ order. This implies that 
    \begin{equation*}
        \exp(- \beta U_N) \leq \exp(\beta B N) ~,
    \end{equation*}
    hence 
    \begin{equation*}
        Q_N = \int_{V^N} \prod_i d^3 q^i \exp (- \beta U_N(q^i)) \leq \exp(\beta B N) \int_{V^N}\prod_i d^3 q^i = \exp(\beta B N) V^N 
    \end{equation*}
    and 
    \begin{equation*}
        Z_N = \frac{Q_N}{N! \lambda_T^{3N}} \leq \frac{V^N}{N! \lambda^{3N}_T} \exp(\beta B N) ~.
    \end{equation*}
    Therefore 
    \begin{equation*}
        |\mathcal Z| \leq \sum_{N=0}^\infty \frac{|z|^N}{N! \lambda_T^{3N}} V^N \exp(\beta B N) = \exp(\frac{V \exp(\beta B) |z|}{\lambda_T^3}) ~,
    \end{equation*}
    which, given the fact that it is an exponential, has an infinite convergence radius. We have proved that it is analytical $\forall z \in \mathbb C$, in particular for $z \in \mathbb R^+$. Furthermore, $\mathcal Z$ cannot become vanishing since it is convergent and it is a sum of positive terms. We introduce the granpotential 
    \begin{equation*}
        \Omega = - \frac{1}{\beta} \ln \mathcal Z ~,
    \end{equation*}
    which is well defined, since $\mathcal Z \neq 0$ and analytical $\forall z \in \mathbb R^+$. 

    Finally, all thermodynamic functions are analytical for $z \in \mathbb R^+$ and there are no phase transitions. How it it possible? We have not yet computed the thermodynamic limit.

    Consider a system composed of hard sphere occupying a finite volume $v$. The maximum number of particles is $M = \frac{V}{v}$. $\mathcal Z$ is a polynomial function in $z$ of degree $M$ and, by the fundamental theorem of algebra, it has $M$ zeroes but none in $\mathbb R^+$. If we go into the thermodynamic limit, $V \rightarrow \infty$, $M \rightarrow \infty$ and the number of zeroes increases. However, it holds that 
    \begin{enumerate}
        \item $\forall V, M$, there exists an open subset of $\mathbb R^+$ which does not contain zeroes, i.e. $\mathcal Z (V \rightarrow \infty, T, \mu)$ has no zeroes on $\mathbf R^+$,
        \item if zeroes accumulate towards a certain $z = z_c$, then  $\mathcal Z (V \rightarrow \infty, T, \mu)$ has a zero in $z = z_c$.
    \end{enumerate}
    This means that $\Omega$ is no longer analytical at $z = z_c$. Now, there is no more equilbrium and phase transitions come up from the dark. 

    This statements can be written down in terms of 
    \begin{equation*}
        \psi = \lim_{td} \frac{\ln \mathcal Z}{V} ~.
    \end{equation*}
    Hence 
    \begin{equation*}
        p \beta = \psi ~, \quad n = z \pdv{}{z} \psi ~.
    \end{equation*}
    \begin{proof}
        For the first 
        \begin{equation*}
            \Omega = - p V = - \frac{1}{\beta} \ln \mathcal Z ~,
        \end{equation*}
        hence
        \begin{equation*}
            p \beta = \frac{\ln \mathcal Z}{V} = \psi ~.
        \end{equation*}

        For the second,
        \begin{equation*}
            N = z \pdv{}{z} \ln \Omega = - \frac{z}{\beta} \pdv{}{z} \ln \Omega ~,
        \end{equation*}
        hence
        \begin{equation*}
            n = \frac{N}{V} = - \frac{z}{\beta} \pdv{}{z} \frac{\ln \Omega}{V} =  - \frac{z}{\beta} \pdv{}{z} \psi ~.
        \end{equation*}
    \end{proof}

    \begin{theorem}
        If $U_N \geq - BN$ with $B > 0$, if the boundary of the volume does not increases fastes than $V^{2/3}$, in order to neglect surface terms, then $\psi$ exists, it is continuous and monotonically increasing.
    \end{theorem}
    \begin{theorem}
        Given an open subset of an interval of $\mathbb R^+$ such that it doesn not contain zeroes, then $\psi$ exists and it is analytic.
    \end{theorem}
    \begin{corollary}
        A phase transitions may appear at $z = z_c$ if it is an accumulation point of zeroes. This point divides $\mathbb R^+$ into $2$ regions corresponding to $2$ different phases. Furthermore, $\psi$ is continuous but it is not analytic: $1st$ order phase transitions or continuous phase transitions.
    \end{corollary}

\chapter{Ising model}

    Consider a system composed by a discrete lattice, for example an hypercubic lattice of dimension $d$. For each vertex, there is a degree of freedom, characterised by the approximation of a spin that could have only two values $\sigma = \pm 1$. Two adjacent verteces are called nearest neighborhood. Each site has therefore $z$ nearest neighborhood, called the coordination number. For a dimension $d$ hybercube, $z = 2 d$. A possible configuration stae is defined as $\{\sigma_i\}_{i \in \mathcal L}$. The phase space is a discrete space composed by $2^N$ states $\{\{\sigma_i\}_{i \in \mathcal L}, \sigma_i = \pm 1\}$. 

    The hamiltonian of the system is 
    \begin{equation*}
        H(\sigma_i) = H_{int} + H_{field} ~,
    \end{equation*}
    where
    \begin{equation*}
        H_{int} = - J \sum_{i \text{near} j} \sigma_i \sigma_j 
    \end{equation*}
    and 
    \begin{equation*}
        H_{field} = - B \sum_{i=1}^{N} \sigma_i ~.
    \end{equation*}
    $B$ is an external magnetic field and $J$ is the interaction constant. Notice that in order to have a phase transitions, we have to allow interactions. For $J > 0$, the minimum energy configuration is the one in which all the spins are aligned $\sigma_i = \sigma_j ~, \quad \forall i, j$. This model is called ferromagnetic model. For $J < 0$, the minimum energy configuration is the one in which all the spins are antialigned $\sigma_i = - \sigma_j ~, \quad \forall i, j$. This model is called antiferromagnetic model. For $B > 0$, the minimum energy configuration is the one in which all the spins are aligned upwards $\sigma_i = + 1$. For $B < 0$, the minimum energy configuration is the one in which all the spins are aligned upwards $\sigma_i = - 1$. 

    In the canonical ensemble, the partition function is 
    \begin{equation*}
        Z_N = \sum_{\sigma_i = \pm 1} \exp(- \beta H(\sigma_i)) ~,
    \end{equation*}
    where the sum is made over all the $2^N$ states. The Helmoltz free energy is 
    \begin{equation*}
        F = E - TS = - \frac{1}{\beta} \ln Z_N ~.
    \end{equation*}
    where $E = \av{H}_c$. The thermodynamic equilibrium correspond to the configuration of minimum free energy.

    Suppose the external magnetic field is shut down. The ground state is the one with minimum energy and the entropy is small, because there are only $2$ states possible with all aligned spins. The excited state is the one with minimum energy and the entropy is big, because all spins point in all direction. Recall that entropy is $S = k_B \ln \Gamma(E)$. The minimal configuration of free energy is therefore at low $T$ with minimum $E$, i.e. all aligned, and at high $T$ with large $S$, i.e. random alignment.

    To stimate the alignment, we introduce the magnetisation 
    \begin{equation*}
        M = \av{\sum_{i=1}^N \sigma_i}_c = \sum_{i=1}^N \av{\sigma_i}_c ~,
    \end{equation*}
    where we have used the translation invariance. 

    Computing the phase diagram, we find that along the $T$-axis at $B=0$, $m \neq 0$ for $T < T_c$ and $m = 0$ for $T > T_c$, where $T_c$ is the critical temperature. The former is called the ferromagnetic phase and the latter is called the paramagnetic phase. We can use the magnetisation as an order parameter, since when it is zero there is disorder and when it is different from zero, there is order. In the neighborhood of $T_c$, we have the behaviour for $T < T_c$
    \begin{equation*}
        M \sim (T - T_c)^\beta ~,
    \end{equation*}
    where $\beta$ is a parameter. It characterise the phase transition, since it tells which speed $M \rightarrow 0$ when approacing $T \rightarrow T_c$.

\section{Correlation}

\section{Symmetry breaking}

    Consider the Ising model hamiltonian. The first term is invariant under 
    \begin{equation*}
        \sigma_i \rightarrow - \sigma_i ~, \quad \sigma_i \rightarrow \sigma_i ~.
    \end{equation*}
    Therefore, it is invariant under the global symmetry group $\mathbb Z_2$. If it were the only term, i.e. with $B=0$, the whole hamiltonian would be invariant under this group. Howver, the second term breaks explicitly the symmetry. Moreover, noticing that under this symmetry $m \rightarrow - m$, the only possible value of $m$ would be zero. Hence, for $T > T_c$ there is indeed $m=0$. But for $T < T_c$, the equilibrium state is no longer invariant under this symmetry. The hamiltonian remains the same but states are not invariant. There is a spontaneous symmetry breaking, spontaneous because $H$ is stil invariant under $\mathbb Z_2$.

\chapter{Mean-field treatment} 

    Consider the Ising model. In general, it is diffucult to compute the canonical partition function,
    \begin{equation*}
        Z_N = \sum_{\{\sigma_i = \pm 1\}} \exp(- \beta H) \neq (Z_1)^N ~,
    \end{equation*}
    since it is interacting. However, we can make an useful approximation 
    \begin{equation*}
        \sigma_i \sigma_j = ((\sigma_i - m) + m)((\sigma_j - m) + m) = m^2 + m(\sigma_i - m) + m (\sigma_j - m) + (\sigma_i - m)(\sigma_j - m) ~,
    \end{equation*}
    in which we keep only the first constant term and the second linear, but we neglect the last quadratic fluctuation term. Therefore 
    \begin{equation*}
        \sigma_i \sigma_j =  m^2 + m(\sigma_i - m) + m (\sigma_j - m) = m^2 + m\sigma_i - m^2 + m \sigma_j - m^2 = - m^2 + m (\sigma_i + \sigma_j) ~.
    \end{equation*}
    The hamiltonian in the mean-field approximation becomes 
    \begin{equation*}
        H_{mf} = - J \sum_{i \text{nn} j} (- m^2 + m(\sigma_i + \sigma_j)) - B \sum_i \sigma_i = m^2 J \sum_{i \text{nn} j} 1 - J m \sum_{i \text{nn} j} (\sigma_i + \sigma_j) - B \sum_i \sigma_i ~.
    \end{equation*}
    The number of links, given the coordination number $z$ which tells how many neighboring sites, is $NZ/2$. Hence 
    \begin{equation*}
        H_{mf} = \frac{m^2 z N J}{2} - Jmz \sum_i \sigma_i - B \sum_i \sigma_i = frac{m^2 z N J}{2} - (J m z + B) \sum_i \sigma_i ~.
    \end{equation*}
    The physical intepretation of the mean-field treatment is that we do not have to compute every links with respect to each others but only with respect to the mean field $m$. It is valid only if fluctuations are smaller than the mean-field. The partition function is 
    \begin{equation*}
    \begin{aligned}
        Z_N^{mf} & \sum_{\{\sigma_i = \pm 1\}} \exp(- \beta H_{mf}) \\ & = \exp(- \beta \frac{J z n m^2}{2}) \sum_{\{\sigma_i = \pm 1\}} \exp(\beta (B + Jmz) \sum_i \sigma_i) \\ & = \exp(- \beta \frac{J z n m^2}{2}) (\sum_{\{\sigma_i = \pm 1\}} \exp(\beta (B + Jmx) \sigma_i))^N \\ & = \exp(- \beta \frac{J z n m^2}{2}) (\exp(\beta(B + Jmz)) + \exp(- \beta (B + Jmz)))^N \\ & = \exp(- \beta \frac{J z n m^2}{2}) (2 \cosh (\beta (B + Jmz))) ~.
    \end{aligned}
    \end{equation*}
    The Helmoltz free energy is 
    \begin{equation*}
    \begin{aligned}
        F & = - \frac{1}{\beta} \ln Z_N^{mf} \\ & = - \frac{1}{\beta} (- \beta \frac{J z N m^2}{2}) N \ln (2 \cosh (\beta (B + Jmz))) \\ & = \frac{J z N m^2}{2} N \ln (2 \cosh (\beta (B + Jmz))) ~.
    \end{aligned}
    \end{equation*}
    The magnetisation is 
    \begin{equation*}
    \begin{aligned}
         m & = \frac{1}{N} \av{\sum_i \sigma_i}_c \\ & = \frac{1}{N} \sum_{\{\sigma_i = \pm 1\}} \sum_i \sigma_i \exp(- \beta H) \\ & = - \frac{1}{\beta N} \sum_{\{\sigma_i = \pm 1\}} \frac{1}{Z_N} \pdv{}{\beta} \exp(- \beta H) \\ & = - \frac{1}{\beta N} \pdv{\ln Z_N}{\beta} ~.
    \end{aligned}
    \end{equation*}
    Hence 
    \begin{equation*}
        m = \tanh (\beta (B + J m z)) ~.
    \end{equation*}
    We have a self-contistet equation for m to solve. The condition for solution is $B >0$ then $m > 0$ and $B < 0$ then $m < 0$. Particular attention we can study for $B = 0$ then 
    \begin{equation*}
        m = \tanh \frac{J m z}{k_B T} = \tanh \frac{T_c m}{T} ~,
    \end{equation*}
    where $T_c = J z / k_B$ is the critical temperature. Notice that it depends on $z$. Calling $\tilde m = T_c m / T$, we have 
    \begin{equation*}
        \frac{T \tilde m}{T_c} = \tanh \tilde m ~.
    \end{equation*}
    The solutions are points that intersect a straigh line and an hyperbolic tangent. If $T > T_c$, there is only one solution $m = 0$. If $T > T_c$, there are two solutions $\pm m_0$, one positive and one negative. See Figure~\ref{mf:m}.
    \begin{figure}
        \centering
        \scalebox{0.7}{\pyc{plot4('x', '2* x', 'x / 2', 'x', 'tanh(x)', 5, 5, 20, True, False, False)}}
        \caption{A plot of the graphical solution of $T \tilde m / T_c = \tanh \tilde m$.}
        \label{mf:m}
    \end{figure}
    We have proven that $m$ is indeed an order parameter. 
    
\appendix

\part{Appendix}

\chapter{Volume of an N-dimensional sphere}

    In this appendix chapter, we will prove that the volume of an $N$-dimensional sphere of radius $R$ is 
    \begin{equation}\label{app:volumen}
        V_n (R) = \frac{\pi^{n/2} R^n}{\Gamma(n/2 + 1)} ~.
    \end{equation}
    \begin{proof}
        Consider the rotationally invariant function $f$ 
        \begin{equation*}
            f(x_1, \ldots x_n) = \exp(- \frac{1}{2} \sum_{i=1}^{n} x_i^2 ) = \prod_{i=1}^{n} \exp(- \frac{1}{2} x_i^2 ) =~.
        \end{equation*}
        Using the Gaussian integral, this function can be integrated over all $\mathbb R^n$, with volume element $dV = dx_1 \ldots dx_n$, and it gives
        \begin{equation*}
        \begin{aligned}
            \int_{\mathbb R^n} dV ~ f & = \int_{\mathbb R^n} \prod_{i=1}^n dx_i ~ f = \int_{\mathbb R^n} \prod_{i=1}^n dx_i ~ \exp(- \frac{1}{2} \sum_{i=1}^{n} x_i^2 ) \\ & = \prod_{i=1}^{n} \underbrace{( \int_{\mathbb R} dx_i ~ \exp(- \frac{1}{2} x_i^2 ))}_{(2 \pi)^{1/2}} = \prod_{i=1}^{n} (2 \pi)^{1/2} = (2 \pi)^{n/2} ~.
        \end{aligned}
        \end{equation*}
        Exploiting the rotational invariant property, we can decomposed the volume element into a surface element $dA$, which integrated gives an $(n-1)$-dimensional sphere $S^{n-1} (r)$ of radius $r$, multiplied by a length element $dr$, i.e.
        \begin{equation*}
            \int_{\mathbb R^n} dV ~ f = \int_0^\infty dr \int_{S^{n-1} (r)} dA ~ f ~.
        \end{equation*}
        Since the area is proportial to the radius, e.g.~for $n=3$ the area is $A \propto r^2$, the radius-dependence of the area is given by $A_{n-1}(r) = r^{n-1} A_{n-1} (1)$. Therefore, putting it inside the integral, we obtain 
        \begin{equation*}
            A_{n-1} (1) \int_0^\infty dr r^{n-1} \exp(- \frac{1}{2} r^2) ~.
        \end{equation*}
        Now, we make a change of variables into 
        \begin{equation*}
            t = \frac{r^2}{2} ~, \quad r = (2t)^{1/2} ~, \quad dr = 2^{-1/2} t^{-1/2} dt
        \end{equation*}
        to have the integral of the gamma function
        \begin{equation*}
        \begin{aligned}
            \int_0^\infty dr ~ r^{n-1} \exp(- \frac{1}{2} r^2) & = 2^{(n-1)/2} 2^{-1/2}\int_0^\infty dt ~ t^{(n-1)/2} t^{-1/2} \exp(-t) \\ & = 2^{n/2 - 1} \underbrace{\int_0^\infty dt ~ t^{n/2 - 1} \exp(-t)}_{\Gamma(n/2)} = 2^{n/2 - 1} \Gamma(n/2) ~.
        \end{aligned}
        \end{equation*}
        Now, we combine the two results together to obtain the surface
        \begin{equation*}
            (2 \pi)^{n/2} = A_{n-1} (1) 2^{n/2 - 1} \Gamma(n/2) ~,
        \end{equation*}
        hence 
        \begin{equation*}
            A_{n-1} (1) = \frac{2 \pi^{n/2}}{\Gamma(n/2)} ~.
        \end{equation*}
        Finally, in order to find the volume we need to integrate from $0$ to $R$ 
        \begin{equation*}
        \begin{aligned}
            V_n(R) & = \int_0^R dr A_{n-1} (r) = \int_0^R dr ~ A_{n-1} (1) r^{n-1} = \frac{2 \pi^{n/2}}{\Gamma(n/2)} \int_0^R dr ~ r^{n-1} \\ & = \frac{2 \pi^{n/2}}{\Gamma(n/2)} \frac{r^n}{n} \Big \vert_0^R = \frac{2 \pi^{n/2}}{n\Gamma(n/2)} R^n = \frac{\pi^{n/2} R^n}{\Gamma(n/2 + 1)} ~.
        \end{aligned}
        \end{equation*}
    \end{proof}

\chapter{Stirling approximation}

    In this appendix chapter, we will prove the Stirling approximation 
    \begin{equation}\label{app:stirl}
        \ln n! \simeq n \ln n - n ~.
    \end{equation}
    \begin{proof}
        The factorial can be expressed in integral form via the gamma function 
        \begin{equation*}
            \Gamma (n + 1) = n! = \int_0^\infty dt ~ t^n \exp(-t) ~.
        \end{equation*}
        Now, we make a change of variables into 
        \begin{equation*}
            t = n x ~, \quad x = \frac{t}{n} ~, \quad dx = \frac{dt}{n} ~,
        \end{equation*}
        to have 
        \begin{equation*}
        \begin{aligned}
            \int_0^\infty dt ~ t^n \exp(-t) & = int_0^\infty dt ~ \exp(\ln t^n) \exp(-t) \\ & = \int_0^\infty dt ~ \exp(n\ln t - t) \\ & = n \int_0^\infty dx ~ \exp(n \ln (nx) - nx) \\ & = n \int_0^\infty dx ~ \exp(n \ln x + n \ln n - nx) \\ & = n \exp(n \ln n) \int_0^\infty dx ~ \exp(n (\ln x - x)) ~.
        \end{aligned}
        \end{equation*}
        In the limit for which $n$ is large, we can use the Laplace approximation method 
        \begin{equation*}
            \int_a^b dx ~ \exp(n f(x)) \simeq \exp(n f(x_0)) \sqrt{\frac{2\pi}{n |f'' (x_0)|}} ~.
        \end{equation*}
        where $x_0 \in [a, b]$ is a stationary point of $f(x)$. A simple sketch of the proof is given by means of the Taylor expansion around $x_0$
        \begin{equation*}
            f(x) \simeq f(x_0) - \frac{1}{2} |f''(x_0)| (x - x_0)^2 ~,
        \end{equation*}
        hence, integrating the Gaussian integral,
        \begin{equation*}
        \begin{aligned}
            \int_a^b dx ~ \exp(n f(x)) & \simeq \exp(n f(x_0)) \int_a^b dx ~ \exp(- \frac{n}{2} |f''(x_0)| (x - x_0)^2) \\ & = \sqrt{\frac{2\pi}{n |f'' (x_0)|}} ~.
        \end{aligned}
        \end{equation*}
        In our case, $a=0$, $b=\infty$ and $f(x) = \ln x - x$, which has a maximum in $x_0 = 1$ and second derivatives equals to $|f''(x)| = 1 / x^2$. Therefore
        \begin{equation*}
            \int_0^\infty dx ~ \exp(n (\ln x - x)) \simeq \exp(n (\ln x_0 - x_0)) \sqrt{\frac{2\pi x_0^2}{n}} \Big \vert_{x_0 = 1} = \exp(- n) \sqrt{\frac{2\pi}{n}} ~.
        \end{equation*}
        Now, we combine the two results together
        \begin{equation*}
            n! \simeq n \exp(n \ln n) \exp(- n) \sqrt{\frac{2\pi}{n}} = \exp(n \ln n - n) \sqrt{2 \pi n} = n^n \exp(-n) \sqrt{\frac{2\pi}{n}} ~,
        \end{equation*}
        which can be rewritten in terms of logarithms rather than exponentials 
        \begin{equation*}
            \ln n! \simeq \ln (n^n \exp(-n) \sqrt{\frac{2\pi}{n}} ) = n \ln n - n + O(\ln n) ~.
        \end{equation*}
    \end{proof}

\chapter{Gaussian integral}

    In this appendix chapter, we will prove that the Gaussian integral is
    \begin{equation}\label{app:gauss}
        \int_{-\infty}^\infty dx ~ \exp(- x^2) = \sqrt{\pi} ~.
    \end{equation}
    \begin{proof}
        We start from the square Gaussian integral, which it is the square same integral for the mute properties of the integration variables 
        \begin{equation*}
        \begin{aligned}
            \Big (\int_{-\infty}^\infty dx ~ \exp(- x^2) \Big)^2 & = \int_{-\infty}^\infty dx ~ \exp(- x^2) \int_{-\infty}^\infty dy ~ \exp(- y^2) \\ & = \int_{-\infty}^\infty dx \int_{-\infty}^\infty dy ~ \exp(- (x^2 + y^2)) ~.
        \end{aligned}
        \end{equation*}
        Now, we make a change of variables and we use polar coordinates $(r, \theta)$
        \begin{equation*}
            r^2 = x^2 + y^2 ~, \quad \theta = \arctan \frac{y}{x} ~, \quad dx ~ dy = r ~ dr ~ d\theta ~, \quad (r, \theta) \in [0, \infty) \times [0, 2\pi] ~,
        \end{equation*} 
        to obtain
        \begin{equation*}
        \begin{aligned}
            \int_{-\infty}^\infty dx \int_{-\infty}^\infty dy ~ \exp(- (x^2 + y^2)) & = \underbrace{\int_0^{2\pi} d\theta}_{2\pi} \int_0^\infty dr ~ r \exp(- r^2) \\ & = 2 \pi \int_0^\infty dr ~ r \exp(- r^2) \\ & = \pi \int_0^\infty dr ~ 2 r \exp(- r^2) \\ & = \pi \exp(- r^2) \Big \vert_0^{\cancel \infty} = \pi ~.
        \end{aligned}
        \end{equation*}
        Now, we combine the two results together
        \begin{equation*}
            \Big (\int_{-\infty}^\infty dx ~ \exp(- x^2) \Big)^2 = \pi ~,
        \end{equation*}
        hence
        \begin{equation*}
            \int_{-\infty}^\infty dx ~ \exp(- x^2) = \sqrt{\pi} ~.
        \end{equation*}
    \end{proof}
    

\backmatter

\nocite{smlecture}
\nocite{ercolessi}

\clearpage
\phantomsection
\printbibliography

\end{document}
