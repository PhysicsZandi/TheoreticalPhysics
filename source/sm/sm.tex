\documentclass[a4paper, 12pt]{memoir}

\usepackage[a4paper, top = 4cm, bottom = 4cm, left = 3cm, right = 3cm]{geometry}

\usepackage[T1]{fontenc}
\usepackage[utf8]{inputenc}
\usepackage{pythontex} 
\usepackage{nopageno} 
\usepackage{pgf}

\usepackage{tocloft}
\newcommand{\listequationsname}{List of Equations}
\newlistof{listofequations}{equ}{\listequationsname}
\newcommand{\myequation}[1]{%
	\addcontentsline{equ}{equation}{\protect\numberline{\theequation}#1}\par
}
\makeatletter
\let\l@equation\l@figure
\makeatother

\usepackage{xcolor}
\xdefinecolor{mycolor}{RGB}{0,175,179} 
\usepackage{hyperref}
\hypersetup{colorlinks, linkcolor={mycolor}, citecolor={mycolor}, urlcolor={mycolor}}

\usepackage{lipsum}

\renewcommand{\aftertoctitle}{\afterchaptertitle\par\nobreak\hfill{\normalfont{Page}}\par\nobreak}

\usepackage{titlesec}
\titleformat{\part}[display]
  {\normalfont\HUGE\bfseries\color{mycolor}\centering}
  {Part \thepart}{20pt}{\HUGE\normalfont\color{black}}
\titleformat{\chapter}[display]
  {\normalfont\HUGE\bfseries\color{mycolor}\centering}
  {Chapter \thechapter}{20pt}{\HUGE\normalfont\color{black}}
\titleformat{\section}
  {\normalfont\Large\bfseries\color{mycolor}\centering}
  {\thesection}{1em}{}
\titleformat{\subsection}
  {\normalfont\large\bfseries\color{mycolor}\centering}
  {\thesubsection}{1em}{}

\renewcommand{\printtoctitle}[1]{\HUGE\normalfont\color{black}#1}

\usepackage[backend=bibtex, sorting=none]{biblatex}
\addbibresource{../bibliography.bib}

\usepackage{amsmath}
\usepackage{amsthm}
\usepackage{thmtools}
\usepackage{mathtools}

\newtheorem{principle}{Principle}[chapter]
\newtheorem{lemma}{Lemma}[chapter]
\theoremstyle{definition}
\newtheorem{example}{Example}[chapter]
\renewcommand\qedsymbol{q.e.d.}

\theoremstyle{remark}
\newtheorem{case}{Case}

\newcommand{\dv}[2]{\frac{d#1}{d#2}}
\newcommand{\dvin}[3]{\frac{d#1}{d#2}\Big\vert_{#3}}
\newcommand{\dvd}[2]{\frac{d^2#1}{d#2^2}}
\newcommand{\dvf}[2]{\frac{\delta #1}{\delta #2}}
\newcommand{\pdv}[2]{\frac{\partial#1}{\partial#2}}
\newcommand{\pdvd}[3]{\frac{\partial^2 #1}{\partial#2 \partial#3}}
\newcommand{\pdvdu}[2]{\frac{\partial^2 #1}{\partial#2^2}}
\newcommand{\integ}[3]{\int_{#1}^{#2}d#3~}
\newcommand{\poi}[2]{[#1,~#2]}
\newcommand{\poiexp}[2]{\pdv{#1}{q^i} \pdv{#2}{p_i} - \pdv{#2}{q^i} \pdv{#1}{p_i}}

\newcommand{\comm}[2]{[#1,~#2]}
\newcommand{\set}[2]{\{#1\colon#2\}}
\newcommand{\inner}[2]{\langle#1,~#2\rangle}
\newcommand{\av}[1]{\langle#1\rangle}
\newcommand{\avp}[2]{\langle#1\rangle_{#2}}
\newcommand{\ket}[1]{\vert#1\rangle}
\newcommand{\bra}[1]{\langle#1\vert}
\newcommand{\braket}[2]{\langle#1\vert#2\rangle}

\newtheoremstyle{colored}{}{}{\itshape}{}{\color{mycolor}\normalfont\bfseries\indent}{}{\newline}{}

\declaretheorem[
  style=colored,
  name=Definition,
  numberwithin=chapter,
]{definition}

\declaretheorem[
  style=colored,
  name=Theorem,
  numberwithin=chapter,
]{theorem}

\declaretheorem[
  style=colored,
  name=Corollary,
  numberwithin=chapter,
]{corollary}

\declaretheorem[
  style=colored,
  name=Law,
  numberwithin=chapter,
]{law}

\declaretheorem[
  style=colored,
  name=Principle,
  numberwithin=chapter,
]{princ}

\usepackage{amsfonts}
\usepackage{dsfont}
\usepackage{yfonts}
\usepackage{amssymb}

\let\oldproof\proof
\renewcommand{\proof}{\color{darkgray}\oldproof}

\let\oldexample\example
\renewcommand{\example}{\color{darkgray}\oldexample}

\usepackage{cancel}
\usepackage{indentfirst}

\usepackage{tikz}
\usepackage{amssymb}
\usepackage{pgfplots}
\usepgfplotslibrary{patchplots}
\usetikzlibrary{patterns, positioning, arrows}
\pgfplotsset{compat=1.15}

\DeclareMathOperator{\tr}{tr}
\DeclareMathOperator{\str}{str}
\DeclareMathOperator{\real}{Re}
\DeclareMathOperator{\imm}{Im}
\DeclareMathOperator{\sgn}{sgn}
\DeclareMathOperator{\spann}{span}
\DeclareMathOperator{\vol}{vol}

\usetikzlibrary{positioning, arrows.meta}




\title{statistical mechanics}
\date{\today}

\newcommand{\subt}{what happens when there are too many particles?}

\begin{document}

\frontmatter

\pagestyle{empty}
{\raggedleft\vspace*{\baselineskip}
{\LARGE Matteo Zandi}\\[0.35\textheight]
{\HUGE \textcolor{mycolor}{\textbf{On~\thetitle:}}}\\[\baselineskip]
{\LARGE \subt }\\[\baselineskip]
{\large \thedate}\par
\vspace*{2\baselineskip}
\vfill
{\large matteo.zandi2@studio.unibo.it}\par
\vspace*{\baselineskip}}
\clearpage
\pagestyle{headings}

\blankpage

\tableofcontents

\mainmatter

\begin{pycode}
import sympy as sy
def plot1(x, f, rangex, rangey, fig, leg, negx, negy):
    rangexx = rangex
    rangeyy = rangey
    if negx == True:
        rangexx = 0
    if negy == True:
        rangeyy = 0
    x = sy.Symbol('x')
    p = sy.plot((f, (x, -rangexx, rangex)), ylim=[-rangeyy, rangey], legend= leg, show=False, line_color='#00AFB3')
    p.save(f'fig/fig{fig}.pgf')
    print(r'\input{fig/fig'+ rf'{fig}' + r'.pgf}')

def plot4(x, f, g, h, l, rangex, rangey, fig, leg, negx, negy):
    rangexx = rangex
    rangeyy = rangey
    if negx == True:
        rangexx = 0
    if negy == True:
        rangeyy = 0
    x = sy.Symbol('x')
    p = sy.plot((f, (x, -rangexx, rangex)), (g, (x, -rangex, rangex)), (h, (x, -rangex, rangex)), (l, (x, -rangex, rangex)), ylim=[-rangeyy, rangey], legend= leg, show=False, line_color='#00AFB3')
    p[3].line_color='black'
    p.save(f'fig/fig{fig}.pgf')
    print(r'\input{fig/fig'+ rf'{fig}' + r'.pgf}')

def der(y, x):
    x = sy.Symbol(x) 
    derivative = sy.diff(y, x)
    return sy.latex(derivative) 
 
def indint(integrand, x): 
    x = sy.Symbol(x) 
    integral = sy.integrate(integrand,x) 
    return sy.latex(integral) 

def defint(integrand, x, min, max): 
    x = sy.Symbol(x) 
    integral = sy.integrate(integrand, (x, min, max)) 
    return sy.latex(integral) 

def infint(integrand, x): 
    x = sy.Symbol(x) 
    integral = sy.integrate(integrand, (x, float('-inf'), float('inf'))) 
    return sy.latex(integral) 

def ode(ode, y, x): 
    x = sy.Symbol(x) 
    y = sy.Function(y) 
    lhs, rhs = ode.split('=') 
    ode = sy.Eq(sy.S(lhs),sy.S(rhs)) 
    sol = sy.dsolve(ode,y(x)) 
    return sy.latex(sol) 

# \py{ode("Derivative(y(x),x,x) + y(x) = 0", "y", "x")} ~.
 
def odeic(ode, y, x, ic): 
    x  = sy.Symbol(x) 
    y  = sy.Function(y) 
    lhs,rhs = ode.split('=') 
    ode = sy.Eq(sy.S(lhs),sy.S(rhs)) 
    sol = sy.dsolve(ode,y(x), ics= sy.S(ic)) 
    return sy.latex(sol) 

#\py{odeic("Derivative(y(x),x,x) + y(x) = 0", "y", "x", "{y(0):1, y(x).diff(x).subs(x, 0): 0}")} ~.

def matrixmult(A, B):
    C = A*B
    return sy.latex(C)

def Taylor(x, f, point, order):
    x = sy.Symbol('x')
    ts = sy.series(f, x, point, order) 
    return sy.latex(ts)

def limit(x, f, point):
    x = sy.Symbol('x')
    lim = sy.limit(f, x, point) 
    return sy.latex(lim)

\end{pycode}

\chapter*{Introduction}

    In these notes, we will study the mathematical framework of statistical mechanics: the part of physics which focuses on systems composed by a large amount of microscopic constituents, like atoms or molecules. Statistical mechanics arises out from thermodynamics that deals with macroscopic physical quantities and it is able to explain them by means of the notions of ensemble and probability density distribution. In the first part, we will focus on classical systems, starting from some notions of thermodynamics in the language of differential geometry and classical Hamiltonian mechanics, passing through microcanonical, canonical and grand canonical ensembles, finishing with a superficial introduction of classical phase transitions. In the second part, we will focus on quantum systems: ensembles of identical particles in the language of second quantisation to describe Fermi-Dirac and Bose-Einstein gases.
    
\part{Thermodynamics}

\chapter{The 2 laws, which are 4}

    In this chapter, we will recall some notions of thermodynamics: states, equilibrium and the laws of thermodynamics.

\section{Equilibrium}

    The topic of which thermodynamics studies is a class of systems composed by a large amount of particles, roughly speaking Avogadro number $N_A \simeq 6 \times 10^{23}$ constituents, once it reaches a macroscopic equilibrium configuration. To understand the notion of equilibrium, consider a system immersed in its surroundings. It can either interact with it by exchanging matter and/or energy (mechanical, electric, magnetic, chemical work) or be completely isolated. Once a sufficient amount of time has passed By, it reaches a stable configuration. Which particular configuration and its stability can be selected by different boundary conditions the system finds itself in, i.e.~the specification on how the system is in contact and how it interacts with its surroundings. In other words, there is only one and only one final equilibrium configuration towards to the system evolves, once boundary conditions have been given. However, the way the system reaches the equilibrium configuration is irreversible. Equilibrium therefore means that once the system has reached its final configuration, it will stay there forever. 

\section{States}

    A state is a macroscopic configuration. Mathematically speaking, it is a point in the manifold $\mathcal M$ of thermodynamic states. To describe it, we need a chart given by macroscopic physical quantities, called thermodynamic variables. They can be divided into two groups, one conjugate to the other, according to their behaviour when the physical system is rescaled, i.e.~when volume and number of particles change: extensive variables do scale with it whereas intensive ones do not. Some of them are written in Table~\ref{table:td:1}. However, we have to be careful, since only volume is (by definition) extensive and all the others quantities can be considered extensive only if the surface terms are negligible when we take the thermodynamic limit, i.e.~when we first describe the system with finite volume $V$ and number of particles $N$ and then we go to the limit in which $V \rightarrow \infty$ and $N \rightarrow \infty$ but keeping the density fixed $n = N / V$.

    Each physical system has an equation of state, i.e.~a functional relation among thermodynamic quantities which restrict the number of independent variables. Geometrically, it means that the only admissible states are a submanifold $\mathcal A \subset \mathcal M$ of the entire manifold of states, given by the constraint induced by the equation of state.

    \begin{example}[Perfect gas]
        Consider a perfect gas. A chart on its $3$-dimensional manifold can be $(p, V, T)$ and its equation of state is $PV = N k_B T$. This means that the allowed states are in a $2$-dimensional manifold embedded in $\mathbb R^3$.
    \end{example}

    \begin{table}[h!]
        \centering
        \begin{tabular}{c | c }
            Extensive & Intensive \\
            \hline
            energy $E$ & - \\ 
            entropy $S$ & temperature $T$ \\ 
            volume $V$ & pressure $p$\\ 
            number of particles $N$ & chemical potential $\mu$ \\ 
            polarization $\mathbf P$ & electric field $\mathbf E$ \\ 
            magnetization $\mathbf M$ & magnetic field $\mathbf B$ \\ 
        \end{tabular}
        \caption{Extensive and intensive thermodynamic variables.}
        \label{table:td:1}
    \end{table}

\section{The laws of thermodynamics}

    Thermodynamics is governed by a set of laws that every system must obey. They are a particular kind of laws, since they are limitation laws: they tell us only which processes cannot happen. Usually they are referred as the two laws of thermodynamics, but actually they are $4$.

    \begin{law}[0th]
        Let $A$ and $B$ be two thermodynamic systems in thermal contact. At equilibrium, only a subset of states $\mathcal A \subset \mathcal M_A \times \mathcal M_B$ is accessible and not the whole manifold. Mathematically, it means that there exists a functional relation of the kind
        \begin{equation}\label{td:proof1}
            F_{AB} (a,b)= 0 ~,
        \end{equation}
        with $a \in \mathcal M_A$ and $b \in \mathcal M_B$. Moreover, thermal equilibrium is an equivalence class, which can be proved that it means 
        \begin{equation}\label{td:proof2}
            F_{AB} (a,b) = f_A(a) - f_B(b) ~.
        \end{equation}
        The combination of both~\eqref{td:proof1} and~\eqref{td:proof2} allows us to define the empirical temperature 
        \begin{equation}\label{td:0th}
            t_A = f_A(a) = t_B = f_B(b) ~.
        \end{equation}\myequation{0th law of thermodynamic}
    \end{law}
    \noindent It is a limitation law because it limits the configuration that a system can reach in isolation when it is in thermal contact with a second one. 

    \begin{law}[1st]
        Let $\delta Q$ be an infinitesimal heat and $\delta L$ an infinitesimal work exchanged in a quasi-static process ($\delta Q > 0$ means absorbed by the system, $\delta L > 0$ means performed by the system). For any cyclic process, i.e.~processes in which the initial and the final states coincide, we have
        \begin{equation*}
            \oint (\delta Q - \delta L) = 0 ~.
        \end{equation*}
        This means that $\delta Q - \delta L$ is a $1$-form that vanishes when line-integrated along a closed curve in $\mathcal M$ and, by the Poincaré lemma, it is also an exact differential, called the internal energy
        \begin{equation*}
            dE = \delta Q - \delta L ~,
        \end{equation*}
        However, heat and work are not exact differential, since $\oint \delta Q \neq 0$ and $\oint \delta H \neq 0$. 
        
        The generalisation for a system that can exchange matter is given by
        \begin{equation}\label{td:1st}
            \oint (\delta Q - \delta L + \mu dN) = \oint dE = 0 ~, \quad dE = \delta Q - \delta L + \mu dN ~,
        \end{equation}\myequation{1st law of thermodynamic}
        where $\mu$ is the chemical potential, i.e.~the necessary energy to add or remove a particle. Furthermore, we can express both $\delta Q$ and $\delta L$ as a linear combination of infinitesimal change of independent coordinates, e.g. $\delta L = p dV + B dM$. In the following, the only work considered will be the mechanical one $\delta L = p dV$. We assume that the internal energy is extensive and, therefore, the chemical potential is intensive.
    \end{law}

    It is a limitation law because it limits the configuration that a system can reach in isolation to those with $E = const$. 

    \begin{law}[2nd]
        For any cyclic process, we have
        \begin{equation*}
            \oint \frac{\delta Q}{T} \begin{cases}
                = 0 & \textnormal{reversible process} \\
                < 0 & \textnormal{irreversible process} \\
            \end{cases} ~.
        \end{equation*}
        For reversible processes, $\frac{\delta Q}{T} = 0$ is an exact differential. This implies that we can define a function, called entropy, which is always integrated along any reversible path
        \begin{equation}\label{td:2nde}
            S(a) - S(b) = \int_a^b \frac{\delta Q}{T} ~,
        \end{equation}
        Therefore, we have  
        \begin{equation}\label{td:2nd}
            dS \begin{cases}
                = 0 & \textnormal{reversible process} \\
                < 0 & \textnormal{irreversible process} \\
            \end{cases} ~.
        \end{equation}\myequation{2nd law of thermodynamic}
    \end{law}

    It is a limitation law because it limits the configuration that a system can reach in isolation to whose in which entropy cannot increase. 

    \begin{law}[3rd]
        Isothermal and adiabatic processes coincides when $T=0$, or, equivalently, it is impossible to reach $T=0$ with a finite number of processes. Mathematically,
        \begin{equation}\label{td:3rd}
            \Delta S \rightarrow 0 ~\textnormal{as}~ T \rightarrow 0 ~.
        \end{equation}\myequation{3rd law of thermodynamic}
        Therefore, $T=0$ is a singular point. Furthermore, if it were possible to reach $T=0$, the second law $\delta Q \leq 0$ implies that it is impossible to raise the temperature. It is a thermodynamic feature, since it can be proved that it is impossible to realize an engine with efficiency $\eta = 1$.
    \end{law}

    It is a limitation law because it limits the configuration that a system can reach in isolation to whose in which $T \neq 0$.

\chapter{Thermodynamic potentials}

    In this chapter, we will study thermodynamic potentials: energy $E$, entropy $S$, Helmholtz free energy $F$, enthalpy, Gibbs free energy and grand potential. We will derive their definition, their differential and their equations of state. 
    
    Thermodynamic potentials are functions defined in the manifold, which are suited for a particular choice of the $3$ coordinates (boundary conditions) and, therefore, they are useful if we find the system with all the other coordinates constant. 
    
\section{Internal energy}
    
    The first thermodynamic potential we are going to study is the internal energy $E$, which is defined by the first law of thermodynamic~\eqref{td:1st}. Its differential is
    \begin{equation}\label{td:d:e}
        dE \leq T dS - pdV + \mu dN ~.
    \end{equation}
    This relation is called the fundamental equation of thermodynamics.
    \begin{proof}
        In fact, we invert~\eqref{td:1st}
        \begin{equation*}
            \delta Q = dE + \delta L - \mu dN ~,
        \end{equation*}
        we use $\delta L = p dV$ and we put it into~\eqref{td:2nd}
        \begin{equation}
            dS \leq \frac{\delta Q}{T} = \frac{dE + p dV - \mu dN}{T} ~.
        \end{equation}
        Finally, we isolate $dE$
        \begin{equation}
            dE \leq TdS - p dV + \mu dN ~.
        \end{equation}
    \end{proof} 

    Notice that non-differential variables are intensive and differential one are extensive. This tells us that $E(S, V, N)$ is a function of the extensive variables $S$, $V$ and $N$. The intensive variables $T$, $p$ and $\mu$ can be derived from $E$ by the following relations 
    \begin{equation}\label{td:es:e}
        T = \pdv{E}{S} \Big \vert_{V,N} ~, \quad p = - \pdv{E}{V} \Big \vert_{S,N} ~, \quad \mu = \pdv{E}{N} \Big \vert_{S,V} ~. 
    \end{equation}
    These functional relations are called the equation of state of the system, since we can calculate one variable from it, e.g. $T = T(S,V,N)$, $p = p(S,V,N)$ or $\mu = \mu(S,V,N)$. 
    \begin{proof}
        At constant $V$ and $N$,~\eqref{td:d:e} becomes
        \begin{equation*}
            dE = TdS - p \underbrace{dV}_0 + \mu \underbrace{dN}_0 = TdS ~,
        \end{equation*}
        hence 
        \begin{equation*}
            T = \pdv{E}{S} \Big \vert_{V,N} ~.
        \end{equation*}
        At constant $S$ and $N$,~\eqref{td:d:e} becomes
        \begin{equation*}
            dE = T\underbrace{dS}_0 - p dV + \mu \underbrace{dN}_0 = - p dV ~,
        \end{equation*}
        hence 
        \begin{equation*}
            p = - \pdv{E}{V} \Big \vert_{S,N} ~.
        \end{equation*}
        At constant $S$ and $V$,~\eqref{td:d:e} becomes
        \begin{equation*}
            dE = T\underbrace{dS}_0 - p \underbrace{dV}_0 + \mu dN = \mu dN ~,
        \end{equation*}
        hence 
        \begin{equation*}
            \mu = \pdv{E}{S} \Big \vert_{S,V} ~.
        \end{equation*}
    \end{proof}

    $E$ is an extensive variable, i.e. it is an homogeneous function of degree one of the extensive variables 
    \begin{equation}\label{td:omoe}
        E(\lambda S, \lambda V, \lambda N) = \lambda E(S, V, N) ~, \quad \forall \lambda > 0 ~.
    \end{equation}
    The physical meaning is that if we rescale the volume, the energy is rescaled by the same amount. Moreover, since energy is an homogeneous function of degree one of extensive variables and intensive variables are derivative of the energy with respect to extensive variables, we can conclude that intensive variable are homogeneous function of degree zero of the extensive variables 
    \begin{equation}\label{a5}
    \begin{gathered}
        T(S, V, N) = T \Big(\frac{S}{N}, \frac{V}{N} \Big) ~, \quad p(S, V, N) = p \Big (\frac{S}{N}, \frac{V}{N} \Big) ~, \\ \mu(S, V, N) = \mu \Big(\frac{S}{N}, \frac{V}{N} \Big) ~.
    \end{gathered}
    \end{equation}
    By homogeneity properties~\eqref{td:omoe}, using $\lambda = N$, we can therefore write
    \begin{equation*}
    \begin{gathered}
        E = E(S, V, N) = E \Big(N \frac{S}{N}, N \frac{V}{N}, N \Big) = N E \Big(\frac{S}{N}, \frac{V}{N}, 1 \Big) = N e ~, \\  S = S(E, V, N) = S \Big(N \frac{E}{N}, N \frac{V}{N}, N \Big) = N S \Big(\frac{E}{N}, \frac{V}{N}, 1 \Big) = N s ~,
    \end{gathered}
    \end{equation*}
    where we have defined specific energy $e$, specific entropy $s$ and specific volume $v$ as
    \begin{equation*}
        e = \frac{E}{N} = e(s, v) ~, \quad s = \frac{S}{N} = s(e, v) ~, \quad v = \frac{V}{N} ~.
    \end{equation*}
    The Euler's theorem allows us to state that, if $E$ is smooth, it can be written as 
    \begin{equation*}
        E = S \pdv{E}{S} + V \pdv{E}{V} + N \pdv{E}{N} ~,
    \end{equation*}
    or, equivalently, 
    \begin{equation}\label{td:e}
        E = TS - pV + \mu N ~.
    \end{equation}
    \begin{proof}
        In fact, using~\eqref{td:es:e}, we obtain
        \begin{equation*}
            E = S \underbrace{\pdv{E}{S}}_T + V \underbrace{\pdv{E}{V}}_{-p} + N \underbrace{\pdv{E}{N}}_\mu = TS - pV + \mu N ~.
        \end{equation*}
    \end{proof}

    In order to be an exact differential, the exterior derivative of the right-handed side of~\eqref{td:d:e} must have a null exterior derivative, which leads to the integrability conditions
    \begin{equation}\label{td:int:e}
        - \pdv{T}{V} \Big \vert_{S,N} = \pdv{p}{S} \Big \vert_{V,N} ~, \quad 
        \pdv{T}{N} \Big \vert_{S,V} = \pdv{\mu}{S} \Big \vert_{N, V} ~, \quad 
        - \pdv{p}{N} \Big \vert_{V,S} = \pdv{\mu}{V} \Big \vert_{N, S} ~. 
    \end{equation}
    \begin{proof}
        By means of the exterior derivative, we have 
        \begin{equation*}
        \begin{aligned}
            d (dE) & = d (T dS) - d (p dV) + d (\mu dN) \\ & = \pdv{T}{S} \underbrace{dS \wedge dS}_0 + \pdv{T}{V} dV \wedge dS + \pdv{T}{N} dN \wedge dS - \pdv{p}{S} dS \wedge dV - \pdv{p}{V} \underbrace{dV \wedge dV}_0 \\ & \quad - \pdv{p}{N} dN \wedge dV + \pdv{\mu}{S} dS \wedge dN + \pdv{\mu}{V} dV \wedge dN + \pdv{\mu}{N} \underbrace{dN \wedge dN}_0 \\ & = \pdv{T}{V} dV \wedge dS + \pdv{T}{N} dN \wedge dS - \pdv{p}{S} dS \wedge dV \\ & \quad - \pdv{p}{N} dN \wedge dV + \pdv{\mu}{S} dS \wedge dN + \pdv{\mu}{V} dV \wedge dN ~.
        \end{aligned}
        \end{equation*}
        At constant $N$, we obtain
        \begin{equation*}
        \begin{aligned}
            0 & = d^2 E = \pdv{T}{V} dV \wedge dS + \pdv{T}{N} \underbrace{dN}_0 \wedge dS - \pdv{p}{S} dS \wedge dV \\ & \quad - \pdv{p}{N} \underbrace{dN}_0 \wedge dV + \pdv{\mu}{S} dS \wedge \underbrace{dN}_0 + \pdv{\mu}{V} dV \wedge \underbrace{dN}_0 \\ & = \pdv{T}{V} dV \wedge dS - \pdv{p}{S} dS \wedge dV = \pdv{T}{V} dV \wedge dS + \pdv{p}{S} dV \wedge dS ~,
        \end{aligned}
        \end{equation*}
        hence, by the linear independence of $V$ and $S$, we find
        \begin{equation*}
            - \pdv{T}{V} \Big \vert_{S,N} = \pdv{p}{S} \Big \vert_{V,N} ~.
        \end{equation*}
        At constant $V$, we obtain
        \begin{equation*}
        \begin{aligned}
            0 & = d^2 E = \pdv{T}{V} \underbrace{dV}_0 \wedge dS + \pdv{T}{N} dN \wedge dS - \pdv{p}{S} dS \wedge \underbrace{dV}_0 \\ & \qquad - \pdv{p}{N} dN \wedge \underbrace{dV}_0 + \pdv{\mu}{S} dS \wedge dN + \pdv{\mu}{V} \underbrace{dV}_0 \wedge dN \\ & = \pdv{T}{N} dN \wedge dS + \pdv{\mu}{S} dS \wedge dN = \pdv{T}{N} dN \wedge dS - \pdv{\mu}{S} dN \wedge dS~,
        \end{aligned}
        \end{equation*}
        hence, by the linear independence of $N$ and $S$, we find
        \begin{equation*}
            \pdv{T}{N} \Big \vert_{S,V} = \pdv{\mu}{S} \Big \vert_{N, V} ~.
        \end{equation*}
        At constant $S$, we obtain
        \begin{equation*}
        \begin{aligned}
            0 & = d^2 E = \pdv{T}{V} dV \wedge \underbrace{dS}_0 + \pdv{T}{N} dN \wedge \underbrace{dS}_0 - \pdv{p}{S} \underbrace{dS}_0 \wedge dV \\ & \qquad - \pdv{p}{N} dN \wedge dV + \pdv{\mu}{S} \underbrace{dS}_0 \wedge dN + \pdv{\mu}{V} dV \wedge dN \\ & = - \pdv{p}{N} dN \wedge dV + \pdv{\mu}{V} dV \wedge dN = - \pdv{p}{N} dN \wedge dV - \pdv{\mu}{V} dN \wedge dV ~,
        \end{aligned}
        \end{equation*}
        hence, by the linear independence of $N$ and $V$, we find
        \begin{equation*}
            - \pdv{p}{N} \Big \vert_{V,S} = \pdv{\mu}{V} \Big \vert_{N, S} ~.
        \end{equation*}
    \end{proof}

\section{Entropy}

    The second thermodynamic potential we are going to study is the entropy $S$. Inverting~\eqref{td:d:e}, we obtained its differential
    \begin{equation}\label{td:d:s}
        dS = \frac{1}{T} dE + \frac{p}{T} dV - \frac{\mu}{T} dN ~.
    \end{equation}
    Therefore, its equations of state are 
    \begin{equation}\label{td:es:s}
        \frac{1}{T} = \pdv{S}{E} \Big \vert_{V, N} ~, \quad \frac{p}{T} = \pdv{S}{V} \Big \vert_{E, N} ~, \quad - \frac{\mu}{T} = \pdv{S}{N} \Big \vert_{E, V} ~.
    \end{equation}

    The Gibbs-Duhem relation expresses the chemical potential $\mu$ in terms of the pressure $p$ and the temperature $T$ 
    \begin{equation}\label{td:gd}
        S dT - Vdp + N d\mu = 0 ~, \quad d \mu = v dp - s dT ~.
    \end{equation}
    \begin{proof}
        Computing the differential of~\eqref{td:e} 
        \begin{equation*}
            dE = T dS + S dT -p dV + \mu dN + N d\mu 
        \end{equation*}
        and comparing it with~\eqref{td:d:e}
        \begin{equation*}
            dE = \cancel{T dS} + S dT - \cancel{p dV} + - V dp + \cancel{\mu dN} + N d\mu = \cancel{T dS} - \cancel{p dV} + \cancel{\mu dN} ~,
        \end{equation*}
        we obtain 
        \begin{equation*}
            S dT - V dp + N d\mu = 0 ~,
        \end{equation*}
        which can be written as 
        \begin{equation*}
            d \mu = \frac{V}{N} dp - \frac{S}{N} dT = v dp - s dT ~.
        \end{equation*}
    \end{proof}

\section{Thermodynamic states as a manifold}

    Since an equilibrium state is a point in the manifold $\mathcal M$, we need a chart to describe it, which in our case can be thought as an open subset of $\mathbb R^3$. An example of independent local coordinates are $S$, $V$ and $N$ and they can be used to solve thermodynamic, i.e.~to find explicitly the fundamental equation 
    \begin{equation}\label{td:coord:e}
        E = E(S, V, N) ~.
    \end{equation}
    However, we could have chosen another thermodynamic potential, like the entropy
    \begin{equation}\label{td:coord:s}
        S = S(E, V, N) 
    \end{equation}
    and a chart would have had $E$, $V$ and $N$ as coordinates. Notice that at least one of the local coordinates in any chart for $\mathcal M$ must always be extensive. 
    \begin{proof}
        By the $0th$ law and~\eqref{a5}, there must exist a functional relation between intensive variables. This means that one of the three is already fixed once the other two are given and they cannot be used all three as independent coordinates.
    \end{proof}
    Therefore, there are different thermodynamic potentials that we can use: all functions of $3$ independent variables (of which one at least must be extensive) that can be used to define a different chart for $\mathcal M$. This implies that there are different approaches to thermodynamics. The standard method to find other potentials is to apply various kind of Legendre transform of~\eqref{td:d:e}, which exchanges the role of an extensive variable to its conjugate intensive variable as independent variable. The only requirement we need is that the hypothesis of the inverse function theorem are satisfied, e.g. 
    \begin{equation*}
        \pdvdu{E}{S} \Big \vert_{V,N} \neq 0 ~, \quad \pdvdu{E}{V} \Big \vert_{S,N} \neq 0 ~, \quad \pdvdu{E}{N} \Big \vert_{S, V} \neq 0 ~.
    \end{equation*}
    In the next sections, we will study the most important in thermodynamics: Helmholtz free energy $F$, enthalpy $H$, Gibbs free energy $G$ and grand potential $\Omega$.

\section{Helmholtz free energy} 

    The Helmholtz free energy is defined as 
    \begin{equation}\label{td:def:f}
        F = E - TS ~.
    \end{equation}
    Its differential is 
    \begin{equation}\label{td:d:f}
        dF \leq - S dT - p dV + \mu dN ~.
    \end{equation}
    Its associated chart is
    \begin{equation}\label{td:coord:f}
        F = F(T, V, N) ~.
    \end{equation}
    \begin{proof}
        By a Legendre transform, which means to complete a differential, we obtain
        \begin{equation*}
            dE \leq T dS - p dV + \mu dN = d(TS) - S dT - p dV + \mu dN ~,
        \end{equation*}
        hence,
        \begin{equation*}
            dF = d(E - TS) \leq - S dT - p dV + \mu dN ~.
        \end{equation*}
    \end{proof}
    The equations of state are
    \begin{equation}\label{td:es:f}
        S = - \pdv{F}{T} \Big \vert_{V,N} ~, \quad p = - \pdv{F}{V} \Big \vert_{T,N} ~, \quad \mu = \pdv{F}{N} \Big \vert_{T,V} ~. 
    \end{equation}
    \begin{proof}
        At constant $V$ and $N$, we have
        \begin{equation*}
            dF = - S dT - p \underbrace{dV}_0 + \mu \underbrace{dN}_0 = - S dT~,
        \end{equation*}
        hence,
        \begin{equation*}
            S = - \pdv{F}{T} \Big \vert_{V,N} ~.
        \end{equation*}
        At constant $T$ and $N$, we have
        \begin{equation*}
            dF = - S \underbrace{dT}_0 - p dV + \mu \underbrace{dN}_0 = - pdV ~,
        \end{equation*}
        hence,
        \begin{equation*}
            p = - \pdv{F}{V} \Big \vert_{T,N} ~.
        \end{equation*}
        At constant $T$ and $V$, we have
        \begin{equation*}
            dF = - S \underbrace{dT}_0 - p \underbrace{dV}_0 + \mu dN = \mu dN~,
        \end{equation*}
        hence,
        \begin{equation*}
            \mu = \pdv{F}{N} \Big \vert_{T,V} ~.
        \end{equation*}
    \end{proof}
    The integrability conditions are 
    \begin{equation}\label{td:int:f}
        \pdv{S}{V} \Big \vert_{T,N} = \pdv{p}{T} \Big \vert_{V,N} ~, \quad 
        - \pdv{S}{N} \Big \vert_{T,V} = \pdv{\mu}{T} \Big \vert_{N, V} ~, \quad 
        - \pdv{p}{N} \Big \vert_{V,T} = \pdv{\mu}{V} \Big \vert_{N, T} ~. 
    \end{equation}
    \begin{proof}
        By means of the exterior derivative, we have 
        \begin{equation*}
        \begin{aligned}
            d (dF) & = - d (S dT) - d (p dV) + d (\mu dN) \\ & = - \pdv{S}{T} \underbrace{dT \wedge dT}_0 - \pdv{S}{V} dV \wedge dT - \pdv{S}{N} dN \wedge dT - \pdv{p}{T} dT \wedge dV - \pdv{p}{V} \underbrace{dV \wedge dV}_0 \\ & \quad - \pdv{p}{N} dN \wedge dV + \pdv{\mu}{T} dT \wedge dN + \pdv{\mu}{V} dV \wedge dN + \pdv{\mu}{N} \underbrace{dN \wedge dN}_0 \\ & = - \pdv{S}{V} dV \wedge dT - \pdv{S}{N} dN \wedge dT - \pdv{p}{T} dT \wedge dV \\ & \quad - \pdv{p}{N} dN \wedge dV + \pdv{\mu}{T} dT \wedge dN + \pdv{\mu}{V} dV \wedge dN ~.
        \end{aligned}
        \end{equation*}
        At constant $N$, we obtain 
        \begin{equation*}
        \begin{aligned}
            0 & = d^2 F = - \pdv{S}{V} dV \wedge dT - \pdv{S}{N} \underbrace{dN}_0 \wedge dT - \pdv{p}{T} dT \wedge dV \\ & \quad - \pdv{p}{N} \underbrace{dN}_0 \wedge dV + \pdv{\mu}{T} dT \wedge \underbrace{dN}_0 + \pdv{\mu}{V} dV \wedge \underbrace{dN}_0 \\ & = - \pdv{S}{V} dV \wedge dT - \pdv{p}{T} dT \wedge dV = - \pdv{S}{V} dV \wedge dT + \pdv{p}{T} dV \wedge dT  ~,
        \end{aligned}
        \end{equation*}
        hence, by the linear independence of $V$ and $T$, we find
        \begin{equation*}
            \pdv{S}{V} \Big \vert_{T,N} = \pdv{p}{T} \Big \vert_{V,N} ~.
        \end{equation*}
        At constant $V$, we obtain
        \begin{equation*}
        \begin{aligned}
            0 & = d^2 F = - \pdv{S}{V} \underbrace{dV}_0 \wedge dT - \pdv{S}{N} dN \wedge dT - \pdv{p}{T} dT \wedge \underbrace{dV}_0 \\ & \quad - \pdv{p}{N} dN \wedge \underbrace{dV}_0 + \pdv{\mu}{T} dT \wedge dN + \pdv{\mu}{V} \underbrace{dV}_0 \wedge dN \\ & = - \pdv{S}{N} dN \wedge dT + \pdv{\mu}{T} dT \wedge dN = - \pdv{S}{N} dN \wedge dT - \pdv{\mu}{T} dN \wedge dT~,
        \end{aligned}
        \end{equation*}
        hence, by the linear independence of $N$ and $T$, we find
        \begin{equation*}
            - \pdv{S}{N} \Big \vert_{T,V} = \pdv{\mu}{T} \Big \vert_{N, V} ~.
        \end{equation*}
        At constant $T$, we obtain
        \begin{equation*}
        \begin{aligned}
            0 & = d^2 F = - \pdv{S}{V} dV \wedge \underbrace{dT}_0 - \pdv{S}{N} dN \wedge \underbrace{dT}_0 - \pdv{p}{T} \underbrace{dT}_0 \wedge dV \\ & \quad - \pdv{p}{N} dN \wedge dV + \pdv{\mu}{T} \underbrace{dT}_0 \wedge dN + \pdv{\mu}{V} dV \wedge dN \\ & = - \pdv{p}{N} dN \wedge dV + \pdv{\mu}{V} dV \wedge dN =- \pdv{p}{N} dN \wedge dV - \pdv{\mu}{V} dN \wedge dV ~,
        \end{aligned}
        \end{equation*}
        hence, by the linear independence of $N$ and $V$, we find
        \begin{equation*}
            - \pdv{p}{N} \Big \vert_{V,T} = \pdv{\mu}{V} \Big \vert_{N, T} ~.
        \end{equation*}
    \end{proof}

\section{Enthalpy} 

    The enthalpy is defined as 
    \begin{equation*}
        H = E + pV ~.
    \end{equation*}
    Its differential is 
    \begin{equation}\label{td:d:h}
        dH \leq TdS + Vdp + \mu dN ~.
    \end{equation}
    Its associated chart is
    \begin{equation*}
        H = H(p, S, N) ~.
    \end{equation*}
    \begin{proof}
        By a Legendre transform, which means to complete a differential, we obtain
        \begin{equation*}
            dE \leq T dS - p dV + \mu dN = TdS - d(pV) + V dp + \mu dN ~,
        \end{equation*}
        hence,
        \begin{equation*}
            dH = d(E + pV) \leq TdS + Vdp + \mu dN ~.
        \end{equation*}
    \end{proof}
    The equations of state are
    \begin{equation}\label{td:es:h}
        T = \pdv{H}{S} \Big \vert_{p,N} ~, \quad V = - \pdv{H}{p} \Big \vert_{S,N} ~, \quad \mu = \pdv{H}{N} \Big \vert_{S, p} ~. 
    \end{equation}
    \begin{proof}
        At constant $p$ and $N$, we have
        \begin{equation*}
            dH = TdS + V\underbrace{dp}_0 + \mu \underbrace{dN}_0 ~,
        \end{equation*}
        hence,
        \begin{equation*}
            T = \pdv{H}{S} \Big \vert_{p,N} ~.
        \end{equation*}
        At constant $S$ and $N$, we have
        \begin{equation*}
            dH = T\underbrace{dS}_0 + Vdp + \mu \underbrace{dN}_0~,
        \end{equation*}
        hence,
        \begin{equation*}
            V = - \pdv{H}{p} \Big \vert_{S,N} ~.
        \end{equation*}
        At constant $S$ and $p$, we have
        \begin{equation*}
            dH = T\underbrace{dS}_0 + V\underbrace{dp}_0 + \mu dN ~,
        \end{equation*}
        hence,
        \begin{equation*}
            \mu = \pdv{H}{N} \Big \vert_{S, p} ~.
        \end{equation*}
    \end{proof}
    The integrability conditions are 
    \begin{equation}\label{td:int:h}
        \pdv{V}{S} \Big \vert_{p,N} = \pdv{T}{p} \Big \vert_{S,N} ~, \quad 
        \pdv{V}{N} \Big \vert_{p,S} = \pdv{\mu}{p} \Big \vert_{N, S} ~, \quad 
        \pdv{\mu}{S} \Big \vert_{N,p} = \pdv{T}{N} \Big \vert_{S, p} ~. 
    \end{equation}
    \begin{proof}
        By means of the exterior derivative, we have 
        \begin{equation*}
        \begin{aligned}
            d (dH) & = d (T dS) + d (V dp) + d (\mu dN) \\ & = \pdv{T}{S} \underbrace{dS \wedge dS}_0 + \pdv{T}{p} dp \wedge dS + \pdv{T}{N} dN \wedge dS + \pdv{V}{S} dS \wedge dp + \pdv{V}{p} \underbrace{dp \wedge dp}_0 \\ & \quad + \pdv{V}{N} dN \wedge dp + \pdv{\mu}{S} dS \wedge dN + \pdv{\mu}{p} dp \wedge dN + \pdv{\mu}{N} \underbrace{dN \wedge dN}_0 \\ & = \pdv{T}{p} dp \wedge dS + \pdv{T}{N} dN \wedge dS + \pdv{V}{S} dS \wedge dp \\ & \quad + \pdv{V}{N} dN \wedge dp + \pdv{\mu}{S} dS \wedge dN + \pdv{\mu}{p} dp \wedge dN  ~.
        \end{aligned}
        \end{equation*}
        At constant $N$, we obtain
        \begin{equation*}
        \begin{aligned}
            0 & = d^2 H = \pdv{T}{p} dp \wedge dS + \pdv{T}{N} \underbrace{dN}_0 \wedge dS + \pdv{V}{S} dS \wedge dp \\ & \quad + \pdv{V}{N} \underbrace{dN}_0 \wedge dp + \pdv{\mu}{S} dS \wedge \underbrace{dN}_0 + \pdv{\mu}{p} dp \wedge \underbrace{dN}_0 \\ & = \pdv{T}{p} dp \wedge dS + \pdv{V}{S} dS \wedge dp = \pdv{T}{p} dp \wedge dS - \pdv{V}{S} dS \wedge dp ~,
        \end{aligned}
        \end{equation*}
        hence, by the linear independence of $S$ and $p$, we find
        \begin{equation*}
            \pdv{V}{S} \Big \vert_{p,N} = \pdv{T}{p} \Big \vert_{S,N} ~.
        \end{equation*}
        At constant $S$, we obtain
        \begin{equation*}
        \begin{aligned}
            0 & = d^2 H = \pdv{T}{p} dp \wedge \underbrace{dS}_0 + \pdv{T}{N} dN \wedge \underbrace{dS}_0 + \pdv{V}{S} \underbrace{dS}_0 \wedge dp \\ & \quad + \pdv{V}{N} dN \wedge dp + \pdv{\mu}{S} \underbrace{dS}_0 \wedge dN + \pdv{\mu}{p} dp \wedge dN \\ & = \pdv{V}{N} dN \wedge dp + \pdv{\mu}{p} dp \wedge dN = \pdv{V}{N} dN \wedge dp - \pdv{\mu}{p} dN \wedge dp ~,
        \end{aligned}
        \end{equation*}
        hence, by the linear independence of $N$ and $p$, we find
        \begin{equation*}
            \pdv{V}{N} \Big \vert_{p,S} = \pdv{\mu}{p} \Big \vert_{N, S} ~.
        \end{equation*}
        At constant $p$, we obtain
        \begin{equation*}
        \begin{aligned}
            0 & = d^2 H = \pdv{T}{p} \underbrace{dp}_0 \wedge dS + \pdv{T}{N} dN \wedge dS + \pdv{V}{S} dS \wedge \underbrace{dp}_0 \\ & \quad + \pdv{V}{N} dN \wedge \underbrace{dp}_0 + \pdv{\mu}{S} dS \wedge dN + \pdv{\mu}{p} \underbrace{dp}_0 \wedge dN \\ & = \pdv{T}{N} dN \wedge dS + \pdv{\mu}{S} dS \wedge dN = \pdv{T}{N} dN \wedge dS - \pdv{\mu}{S} dS \wedge dN ~,
        \end{aligned}
        \end{equation*}
        hence, by the linear independence of $S$ and $N$, we find
        \begin{equation*}
            \pdv{\mu}{S} \Big \vert_{N,p} = \pdv{T}{N} \Big \vert_{S, p} ~.
        \end{equation*}
    \end{proof}

\section{Gibbs free energy} 

    The Gibbs free energy is defined as 
    \begin{equation*}
        G = E - TS + pV = F + pV = H - TS ~.
    \end{equation*}
    Its differential is 
    \begin{equation} \label{td:d:g}
        dG \leq - SdT + Vdp + \mu dN ~.
    \end{equation}
    Its associated chart is
    \begin{equation*}
        G = G(p, T, N) ~.
    \end{equation*}
    \begin{proof}
        By a Legendre transform, which means to complete a differential, we obtain
        \begin{equation*}
            dE \leq T dS - p dV + \mu dN = d(TS) - S dT - d(pV) + V dp + \mu dN ~,
        \end{equation*}
        hence 
        \begin{equation*}
            dG = d(E - TS + pV) \leq - S dT + Vdp + \mu dN ~.
        \end{equation*}
    \end{proof}
    The equations of state are
    \begin{equation}\label{td:es:g}
        S = - \pdv{G}{T} \Big \vert_{p,N} ~, \quad V = \pdv{G}{p} \Big \vert_{T,N} ~, \quad \mu = \pdv{G}{N} \Big \vert_{p,T} ~. 
    \end{equation}
    \begin{proof}
        At constant $p$ and $N$, we have
        \begin{equation*}
            dG = - S dT + V\underbrace{dp }_0 + \mu \underbrace{dN}_0  ~,
        \end{equation*}
        hence,
        \begin{equation*}
            S = - \pdv{G}{T} \Big \vert_{p,N}  ~.
        \end{equation*}
        At constant $T$ and $N$, we have
        \begin{equation*}
            dG = - S \underbrace{dT}_0  + Vdp + \mu \underbrace{dN}_0  ~,
        \end{equation*}
        hence,
        \begin{equation*}
            V = \pdv{G}{p} \Big \vert_{T,N} ~.
        \end{equation*}
        At constant $p$ and $T$, we have
        \begin{equation*}
            dG = - S \underbrace{dT}_0  + V\underbrace{dp}_0  + \mu dN ~,
        \end{equation*}
        hence,
        \begin{equation*}
            \mu = \pdv{G}{N} \Big \vert_{p,T} ~.
        \end{equation*}
    \end{proof}
    The integrability conditions are 
    \begin{equation}\label{td:int:g}
        - \pdv{V}{T} \Big \vert_{p,N} = \pdv{S}{p} \Big \vert_{T,N} ~, \quad 
        \pdv{V}{N} \Big \vert_{p,T} = \pdv{\mu}{p} \Big \vert_{N, T} ~, \quad 
        - \pdv{S}{N} \Big \vert_{T,p} = \pdv{\mu}{T} \Big \vert_{N, p} ~. 
    \end{equation}
    \begin{proof}
        By means of the exterior derivative, we have 
        \begin{equation*}
        \begin{aligned}
            d (dG) & = - d (S dT) + d (V dp) + d (\mu dN) \\ & = - \pdv{S}{T} \underbrace{dT \wedge dT}_0 - \pdv{S}{p} dp \wedge dT - \pdv{S}{N} dN \wedge dT + \pdv{V}{T} dT \wedge dp + \pdv{V}{p} \underbrace{dp \wedge dp}_0 \\ & \quad + \pdv{V}{N} dN \wedge dp + \pdv{\mu}{T} dT \wedge dN + \pdv{\mu}{p} dp \wedge dN + \pdv{\mu}{N} \underbrace{dN \wedge dN}_0 \\ & = - \pdv{S}{p} dp \wedge dT - \pdv{S}{N} dN \wedge dT + \pdv{V}{T} dT \wedge dp \\ & \quad + \pdv{V}{N} dN \wedge dp + \pdv{\mu}{T} dT \wedge dN + \pdv{\mu}{p} dp \wedge dN ~.
        \end{aligned}
        \end{equation*}
        At constant $N$, we obtain
        \begin{equation*}
        \begin{aligned}
            0 & = d^2 G = - \pdv{S}{p} dp \wedge dT - \pdv{S}{N} \underbrace{dN}_0 \wedge dT + \pdv{V}{T} dT \wedge dp \\ & \quad + \pdv{V}{N} \underbrace{dN}_0 \wedge dp + \pdv{\mu}{T} dT \wedge \underbrace{dN}_0 + \pdv{\mu}{p} dp \wedge \underbrace{dN}_0 \\ & = - \pdv{S}{p} dp \wedge dT + \pdv{V}{T} dT \wedge dp = - \pdv{S}{p} dp \wedge dT - \pdv{V}{T} dp \wedge dT ~,
        \end{aligned}
        \end{equation*}
        hence, by the linear independence of $p$ and $T$, we find
        \begin{equation*}
            - \pdv{V}{T} \Big \vert_{p,N} = \pdv{S}{p} \Big \vert_{T,N} ~.
        \end{equation*}
        At constant $T$, we obtain
        \begin{equation*}
        \begin{aligned}
            0 & = d^2 G = - \pdv{S}{p} dp \wedge \underbrace{dT}_0 - \pdv{S}{N} dN \wedge \underbrace{dT}_0 + \pdv{V}{T} \underbrace{dT}_0 \wedge dp \\ & \quad + \pdv{V}{N} dN \wedge dp + \pdv{\mu}{T} \underbrace{dT}_0 \wedge dN + \pdv{\mu}{p} dp \wedge dN \\ & = \pdv{V}{N} dN \wedge dp + \pdv{\mu}{p} dp \wedge dN = \pdv{V}{N} dN \wedge dp - \pdv{\mu}{p} dp \wedge dN ~,
        \end{aligned}
        \end{equation*}
        hence, by the linear independence of $p$ and $N$, we find
        \begin{equation*}
            \pdv{V}{N} \Big \vert_{p,T} = \pdv{\mu}{p} \Big \vert_{N, T} ~.
        \end{equation*}
        At constant $p$, we obtain
        \begin{equation*}
        \begin{aligned}
            0 & = d^2 G = - \pdv{S}{p} \underbrace{dp}_0 \wedge dT - \pdv{S}{N} dN \wedge dT + \pdv{V}{T} dT \wedge \underbrace{dp}_0 \\ & \quad + \pdv{V}{N} dN \wedge \underbrace{dp}_0 + \pdv{\mu}{T} dT \wedge dN + \pdv{\mu}{p} \underbrace{dp}_0 \wedge dN \\ & = - \pdv{S}{N} dN \wedge dT + \pdv{\mu}{T} dT \wedge dN = - \pdv{S}{N} dN \wedge dT - \pdv{\mu}{T} dN \wedge dT ~,
        \end{aligned}
        \end{equation*}
        hence, by the linear independence of $N$ and $T$, we find
        \begin{equation*}
            - \pdv{S}{N} \Big \vert_{T,p} = \pdv{\mu}{T} \Big \vert_{N, p} ~.
        \end{equation*}
    \end{proof}

\section{Grand potential} 

    The grand potential is defined as 
    \begin{equation}\label{td:o}
        \Omega = E - TS - \mu N = F - \mu N ~.
    \end{equation}
    Its differential is 
    \begin{equation}\label{td:d:o}
        d\Omega \leq - SdT - pdV - N d\mu ~.
    \end{equation}
    Its associated chart is
    \begin{equation}\label{td:coord:o}
        \Omega = \Omega(T, V, \mu) ~.
    \end{equation}
    \begin{proof}
        By a Legendre transform, which means to complete a differential, we obtain
        \begin{equation*}
            dE \leq T dS - p dV + \mu dN = d(TS) - SdT - p dV + (\mu N) - N d\mu ~,
        \end{equation*}
        hence,
        \begin{equation*}
            d\Omega = d(E - TS - \mu N) \leq - SdT - p dV - N d\mu ~.
        \end{equation*}
    \end{proof}
    The equations of state are
    \begin{equation}\label{td:es:o}
        S = - \pdv{\Omega}{T} \Big \vert_{\mu,V} ~, \quad p = - \pdv{\Omega}{V} \Big \vert_{T,\mu} ~, \quad \mu = - \pdv{\Omega}{N} \Big \vert_{T,V} ~. 
    \end{equation}
    \begin{proof}
        At constant $\mu$ and $V$, we have
        \begin{equation*}
            d\Omega = - SdT - p\underbrace{dV}_0 - N \underbrace{d\mu}_0 = - S dT ~,
        \end{equation*}
        hence,
        \begin{equation*}
            S = - \pdv{\Omega}{T} \Big \vert_{\mu,V} ~.
        \end{equation*}
        At constant $T$ and $\mu$, we have
        \begin{equation*}
            d\Omega = - S \underbrace{dT}_0 - pdV - N \underbrace{d\mu}_ 0 = - p dV ~,
        \end{equation*}
        hence,
        \begin{equation*}
            p = - \pdv{\Omega}{V} \Big \vert_{T,\mu} ~.
        \end{equation*}
        At constant $T$ and $V$, we have
        \begin{equation*}
            d\Omega = - S\underbrace{dT}_0 - p\underbrace{dV}_0 - N d\mu = - N d\mu~,
        \end{equation*}
        hence 
        \begin{equation*}
            \mu = - \pdv{\Omega}{N} \Big \vert_{T,V} ~.
        \end{equation*}
    \end{proof}
    The integrability conditions are 
    \begin{equation}\label{td:int:o}
        \pdv{S}{\mu} \Big \vert_{T,V} = \pdv{N}{T} \Big \vert_{\mu,V} ~, \quad 
        \pdv{S}{V} \Big \vert_{T,\mu} = \pdv{p}{T} \Big \vert_{V, \mu} ~, \quad 
        \pdv{p}{\mu} \Big \vert_{V,T} = \pdv{N}{V} \Big \vert_{\mu, T} ~. 
    \end{equation}
    \begin{proof}
        By means of the exterior derivative, we have 
        \begin{equation*}
        \begin{aligned}
            d (d\Omega) & = - d (S dT) - d (p dV) - d (N d\mu) \\ & = - \pdv{S}{T} \underbrace{dT \wedge dT}_0 - \pdv{S}{V} dV \wedge dT - \pdv{S}{\mu} d\mu \wedge dT - \pdv{p}{T} dT \wedge dV - \pdv{p}{V} \underbrace{dV \wedge dV}_0 \\ & \quad - \pdv{p}{\mu} d\mu \wedge dV - \pdv{N}{T} dT \wedge d\mu - \pdv{N}{V} dV \wedge d\mu - \pdv{N}{\mu} \underbrace{d\mu \wedge d\mu}_0 \\ & = - \pdv{S}{V} dV \wedge dT - \pdv{S}{\mu} d\mu \wedge dT - \pdv{p}{T} dT \wedge dV \\ & \quad - \pdv{p}{\mu} d\mu \wedge dV - \pdv{N}{T} dT \wedge d\mu - \pdv{N}{V} dV \wedge d\mu ~.
        \end{aligned}
        \end{equation*}
        At constant $\mu$, we obtain
        \begin{equation*}
        \begin{aligned}
            0 & = d^2 \Omega = - \pdv{S}{V} dV \wedge dT - \pdv{S}{\mu} \underbrace{d\mu}_0 \wedge dT - \pdv{p}{T} dT \wedge dV \\ & \quad - \pdv{p}{\mu} \underbrace{d\mu}_0 \wedge dV - \pdv{N}{T} dT \wedge \underbrace{d\mu}_0 - \pdv{N}{V} dV \wedge \underbrace{d\mu}_0 \\ & = - \pdv{S}{V} dV \wedge dT - \pdv{p}{T} dT \wedge dV = - \pdv{S}{V} dV \wedge dT + \pdv{p}{T} dV \wedge dT ~,
        \end{aligned}
        \end{equation*}
        hence, by the linear independence of $V$ and $T$, we find
        \begin{equation*}
            \pdv{S}{V} \Big \vert_{T,\mu} = \pdv{p}{T} \Big \vert_{V, \mu} ~.
        \end{equation*}
        At constant $V$, we obtain
        \begin{equation*}
        \begin{aligned}
            0 & = d^2 \Omega = - \pdv{S}{V} \underbrace{dV}_0 \wedge dT - \pdv{S}{\mu} d\mu \wedge dT - \pdv{p}{T} dT \wedge \underbrace{dV}_0 \\ & \quad - \pdv{p}{\mu} d\mu \wedge \underbrace{dV}_0 - \pdv{N}{T} dT \wedge d\mu - \pdv{N}{V} \underbrace{dV}_0 \wedge d\mu \\ & = - \pdv{S}{\mu} d\mu \wedge dT - \pdv{N}{T} dT \wedge d\mu = - \pdv{S}{\mu} d\mu \wedge dT + \pdv{N}{T} d\mu \wedge dT ~,
        \end{aligned}
        \end{equation*}
        hence, by the linear independence of $\mu$ and $T$, we find
        \begin{equation*}
            \pdv{S}{\mu} \Big \vert_{T,V} = \pdv{N}{T} \Big \vert_{\mu,V} ~.
        \end{equation*}
        At constant $T$, we obtain
        \begin{equation*}
        \begin{aligned}
            0 & = d^2 \Omega = - \pdv{S}{V} dV \wedge \underbrace{dT}_0 - \pdv{S}{\mu} d\mu \wedge \underbrace{dT}_0 - \pdv{p}{T} \underbrace{dT}_0 \wedge dV \\ & \quad - \pdv{p}{\mu} d\mu \wedge dV - \pdv{N}{T} \underbrace{dT}_0 \wedge d\mu - \pdv{N}{V} dV \wedge d\mu \\ & = - \pdv{p}{\mu} d\mu \wedge dV - \pdv{N}{V} dV \wedge d\mu=  - \pdv{p}{\mu} d\mu \wedge dV + \pdv{N}{V} d\mu \wedge dV ~,
        \end{aligned}
        \end{equation*}
        hence, by the linear independence of $N$ and $V$, we find
        \begin{equation*}
            \pdv{p}{\mu} \Big \vert_{V,T} = \pdv{N}{V} \Big \vert_{\mu, T} ~.
        \end{equation*}
    \end{proof}

    A few comments can be made about these potentials. Notice that  they are not homogeneous functions since they depend on mixed extensive and intensive variables. However, they are extensive, i.e. 
    \begin{equation}\label{a6}
        F = N f(T, v) ~, \quad H = N h(p, s) ~, \quad G = N g(T, p) ~, \quad \Omega = N f \omega (T, \mu) ~,
    \end{equation}
    where $f$ is the specific Helmholtz free energy, $h$ is the specific enthalpy, $g$ is the specific Gibbs free energy and $\omega$ is the specific grand potential. Furthermore, observe that the chemical potential is also the Gibbs free energy per particle
    \begin{equation}
        g(T, p) = \mu(T, p) ~.
    \end{equation}
    \begin{proof}
        In fact, using~\eqref{td:es:g} and~\eqref{a6}
        \begin{equation*}
            \mu = \pdv{G}{N} = \pdv{}{N}(Ng) = g ~.
        \end{equation*}
    \end{proof}
    Finally, notice that 
    \begin{equation}\label{td:o2}
        \Omega = - pV ~.
    \end{equation}
    \begin{proof}
        Using~\eqref{td:e} and~\eqref{td:o}
        \begin{equation*}
            \Omega = E - TS - \mu N = \cancel{TS} - pV + \cancel{\mu N} - \cancel{TS} - \cancel{\mu N} = - pV ~.
        \end{equation*}
    \end{proof}

\section{Summary}

    A summary of all charts and differentials is given by 
    \begin{equation*}
        E(S, V, N) ~, \quad dE = TdS - p dV + \mu dN ~,
    \end{equation*}
    \begin{equation*}
        S(E, V, N) ~, \quad dS = dE/T + p dV/T - \mu dN/T ~,
    \end{equation*}
    \begin{equation*}
        F(T, V, N) ~, \quad dF = - S dT - p dV + \mu dN ~,
    \end{equation*}
    \begin{equation*}
        H(S, p, N) ~, \quad dH = TdS + V dp + \mu dN ~,
    \end{equation*}
    \begin{equation*}
        G(T, p, N) ~, \quad d G = - SdT + V dp + \mu dN ~,
    \end{equation*}
    \begin{equation*}
        \Omega(T, V, \mu) ~, \quad d\Omega = TdS - p dV + \mu dN ~.
    \end{equation*}

    A summary of all the equations of state is given by 
    \begin{equation*}
        T = \pdv{E}{S} \Big \vert_{V,N} ~, \quad p = - \pdv{E}{V} \Big \vert_{S,N} ~, \quad \mu = \pdv{E}{N} \Big \vert_{S,V} ~,
    \end{equation*}
    \begin{equation*}
        \frac{1}{T} = \pdv{S}{E} \Big \vert_{V, N} ~, \quad \frac{p}{T} = \pdv{S}{V} \Big \vert_{E, N} ~, \quad - \frac{\mu}{T} = \pdv{S}{N} \Big \vert_{E, V} ~,
    \end{equation*}
    \begin{equation*}
        S = - \pdv{F}{T} \Big \vert_{V,N} ~, \quad p = - \pdv{F}{V} \Big \vert_{T,N} ~, \quad \mu = \pdv{F}{N} \Big \vert_{T,V} ~,
    \end{equation*}
    \begin{equation*}
        T = \pdv{H}{S} \Big \vert_{p,N} ~, \quad V = - \pdv{H}{p} \Big \vert_{S,N} ~, \quad \mu = \pdv{H}{N} \Big \vert_{S, p} ~,
    \end{equation*}
    \begin{equation*}
        S = - \pdv{G}{T} \Big \vert_{p,N} ~, \quad V = \pdv{G}{p} \Big \vert_{T,N} ~, \quad \mu = \pdv{G}{N} \Big \vert_{p,T} ~,
    \end{equation*}
    \begin{equation*}
        S = - \pdv{\Omega}{T} \Big \vert_{\mu,V} ~, \quad p = - \pdv{\Omega}{V} \Big \vert_{T,\mu} ~, \quad \mu = - \pdv{\Omega}{N} \Big \vert_{T,V} ~.
    \end{equation*}

    A summary of all integrability conditions is given by 
    \begin{equation*}
        - \pdv{T}{V} \Big \vert_{S,N} = \pdv{p}{S} \Big \vert_{V,N} ~, \quad 
        \pdv{T}{N} \Big \vert_{S,V} = \pdv{\mu}{S} \Big \vert_{N, V} ~, \quad 
        - \pdv{p}{N} \Big \vert_{V,S} = \pdv{\mu}{V} \Big \vert_{N, S} ~,
    \end{equation*}
    \begin{equation*}
        \pdv{S}{V} \Big \vert_{T,N} = \pdv{p}{T} \Big \vert_{V,N} ~, \quad 
        - \pdv{S}{N} \Big \vert_{T,V} = \pdv{\mu}{T} \Big \vert_{N, V} ~, \quad 
        - \pdv{p}{N} \Big \vert_{V,T} = \pdv{\mu}{V} \Big \vert_{N, T} ~,
    \end{equation*}
    \begin{equation*}
        \pdv{V}{S} \Big \vert_{p,N} = \pdv{T}{p} \Big \vert_{S,N} ~, \quad 
        \pdv{V}{N} \Big \vert_{p,S} = \pdv{\mu}{p} \Big \vert_{N, S} ~, \quad 
        \pdv{\mu}{S} \Big \vert_{N,p} = \pdv{T}{N} \Big \vert_{S, p} ~,
    \end{equation*}
    \begin{equation*}
        - \pdv{V}{T} \Big \vert_{p,N} = \pdv{S}{p} \Big \vert_{T,N} ~, \quad 
        \pdv{V}{N} \Big \vert_{p,T} = \pdv{\mu}{p} \Big \vert_{N, T} ~, \quad 
        - \pdv{S}{N} \Big \vert_{T,p} = \pdv{\mu}{T} \Big \vert_{N, p} ~,
    \end{equation*}
    \begin{equation*}
        \pdv{S}{\mu} \Big \vert_{T,V} = \pdv{N}{T} \Big \vert_{\mu,V} ~, \quad 
        \pdv{S}{V} \Big \vert_{T,\mu} = \pdv{p}{T} \Big \vert_{V, \mu} ~, \quad 
        \pdv{p}{\mu} \Big \vert_{V,T} = \pdv{N}{V} \Big \vert_{\mu, T} ~.
    \end{equation*}

\chapter{Stability conditions}

    In this chapter, we will rewrite integrability condition in terms of Jacobian determinant and we will study what are the stability conditions that a system must fulfill in order to be in equilibrium.

\section{Maxwell's relations}

    Integrability condition, called also Maxwell's relations, can be written as Jacobian determinant in the following way 
    \begin{equation*}
        \pdv{a}{b} \Big \vert_{c, d} = \pdv{(a, c, d)}{(b, c, d)} ~,
    \end{equation*}
    such that it satisfies the property 
    \begin{equation*}
        \pdv{(a, c, d)}{(b, c, d)} = - \pdv{(c, a, d)}{(b, c, d)} = -\pdv{(a, c, d)}{(c, b, d)} = \pdv{(a, d, c)}{(b, c, d)} = \pdv{(a, c, d)}{(b, d, c)} ~.
    \end{equation*}
    
    For the energy, they are 
    \begin{equation*}
        \pdv{(T, S, N)}{(p, V, N)} = 1 ~, \quad \pdv{(T, S, V)}{(N, \mu, V)} = 1 ~, \quad \pdv{(p, V, S)}{(\mu, N, S)} = 1 ~.
    \end{equation*}
    \begin{proof}
        Using the first of~\eqref{td:int:e}, we obtain
        \begin{equation*}
            - \pdv{T}{V} \Big \vert_{S, N} = \pdv{p}{S} \Big \vert_{V, N} \rightarrow \pdv{(T, S, N)}{(V, S, N)} = - \pdv{(p, V, N)}{(S, V, N)} = \pdv{(p, V, N)}{(V, S, N)} ~,
        \end{equation*} 
        hence, inverting the right-handed side, we find
        \begin{equation*}
            1 = \pdv{(T, S, N)}{(V, S, N)} \pdv{(p, V, N)}{(V, S, N)}^{-1} = \pdv{(T, S, N)}{(V, S, N)} \pdv{(V, S, N)}{(p, V, N)} = \pdv{(T, S, N)}{(p, V, N)} ~.
        \end{equation*} 
        Using the second of~\eqref{td:int:e}, we obtain
        \begin{equation*}
            \pdv{T}{N} \Big \vert_{S, V} = \pdv{\mu}{S} \Big \vert_{N, V} \rightarrow \pdv{(T, S, V)}{(N, S, V)} = \pdv{(\mu, N, V)}{(S, N, V)} = - \pdv{(\mu, N, V)}{(N, S, V)} = \pdv{(N, \mu, V)}{(N, S, V)} ~,
        \end{equation*} 
        hence, inverting the right-handed side, we find
        \begin{equation*}
            1 = \pdv{(T, S, V)}{(N, S, V)} \pdv{(N, \mu, V)}{(N, S, V)}^{-1} = \pdv{(T, S, V)}{(N, S, V)} \pdv{(N, S, V)}{(N, \mu, V)} = \pdv{(T, S, V)}{(N, \mu, V)} ~.
        \end{equation*} 
        Using the third of~\eqref{td:int:e}, we obtain
        \begin{equation*}
            - \pdv{p}{N} \Big \vert_{V, S} = \pdv{\mu}{V} \Big \vert_{N, S} \rightarrow \pdv{(p, V, S)}{(N, V, S)} = - \pdv{(\mu, N, S)}{(V, N ,S)} = \pdv{(\mu, N, S)}{(N, V ,S)} ~,
        \end{equation*} 
        hence, inverting the right-handed side, we find
        \begin{equation*}
            1 = \pdv{(p, V, S)}{(N, V, S)} \pdv{(\mu, N, S)}{(N, V ,S)}^{-1} = \pdv{(p, V, S)}{(N, V, S)} \pdv{(N, V ,S)}{(\mu, N, S)} = \pdv{(p, V, S)}{(\mu, N, S)} ~.
        \end{equation*} 
    \end{proof}
    For the Helmholtz free energy, they are 
    \begin{equation*}
        \pdv{(S, T, N)}{(V, p, N)} = 1 ~, \quad \pdv{(S, T, V)}{(\mu, N, V)} = 1 ~, \quad \pdv{(p, V, T)}{(\mu, N, T)} = 1 ~.
    \end{equation*}
    \begin{proof}
        Using the first of~\eqref{td:int:f}, we obtain
        \begin{equation*}
            \pdv{S}{V} \Big \vert_{T, N} = \pdv{p}{T} \Big \vert_{V, N} \rightarrow \pdv{(S, T, N)}{(V, T, N)} = \pdv{(p, V, N)}{(T, V, N)} = - \pdv{(p, V, N)}{(V, T, N)} = \pdv{(V, p, N)}{(V, T, N)} ~,
        \end{equation*} 
        hence, inverting the right-handed side, we find
        \begin{equation*}
            1 = \pdv{(S, T, N)}{(V, T, N)} \pdv{(V, p, N)}{(V, T, N)}^{-1} = \pdv{(S, T, N)}{(V, T, N)} \pdv{(V, T, N)}{(V, p, N)} = \pdv{(S, T, N)}{(V, p, N)} ~.
        \end{equation*} 
        Using the second of~\eqref{td:int:f}, we obtain
        \begin{equation*}
            - \pdv{S}{N} \Big \vert_{T, V} = \pdv{\mu}{T} \Big \vert_{N, V} \rightarrow \pdv{(S, T, V)}{(N, T, V)} = - \pdv{(\mu, N, V)}{(T, N, V)} = \pdv{(\mu, N, V)}{(N, T, V)} ~,
        \end{equation*} 
        hence, inverting the right-handed side, we find
        \begin{equation*}
            1 = \pdv{(S, T, V)}{(N, T, V)}  \pdv{(\mu, N, V)}{(N, T, V)}^{-1} = \pdv{(S, T, V)}{(N, T, V)} \pdv{(N, T, V)}{(\mu, N, V)} = \pdv{(S, T, V)}{(\mu, N, V)} ~.
        \end{equation*} 
        Using the third of~\eqref{td:int:f}, we obtain
        \begin{equation*}
            - \pdv{p}{N} \Big \vert_{V, T} = \pdv{\mu}{V} \Big \vert_{N, T} \rightarrow \pdv{(p, V, T)}{(N, V, T)} = - \pdv{(\mu, N, T)}{(V, N ,T)} = \pdv{(\mu, N, T)}{(N, V ,T)} ~,
        \end{equation*} 
        hence, inverting the right-handed side, we find
        \begin{equation*}
            1 = \pdv{(p, V, T)}{(N, V, T)} \pdv{(\mu, N, T)}{(N, V ,T)}^{-1} = \pdv{(p, V, T)}{(N, V, T)} \pdv{(N, V ,T)}{(\mu, N, T)} = \pdv{(p, V, T)}{(\mu, N, T)} ~.
        \end{equation*} 
    \end{proof}
    For the enthalpy, they are 
    \begin{equation*}
        \pdv{(V, p, N)}{(S, T, N)} = 1 ~, \quad \pdv{(V, p, S)}{(N, \mu, S)} = 1 ~, \quad \pdv{(\mu, N, p)}{(S, T, p)} = 1 ~.
    \end{equation*}
    \begin{proof}
        Using the first of~\eqref{td:int:h}, we obtain
        \begin{equation*}
            \pdv{V}{S} \Big \vert_{p, N} = \pdv{T}{p} \Big \vert_{S, N} \rightarrow \pdv{(V, p, N)}{(S, p, N)} = \pdv{(T, S, N)}{(p, S, N)} = - \pdv{(T, S, N)}{(S, p, N)} = \pdv{(S, T, N)}{(S, p, N)} ~,
        \end{equation*} 
        hence, inverting the right-handed side, we find
        \begin{equation*}
            1 = \pdv{(V, p, N)}{(S, p, N)} \pdv{(S, T, N)}{(S, p, N)}^{-1} = \pdv{(V, p, N)}{(S, p, N)} \pdv{(S, p, N)}{(S, T, N)} = \pdv{(V, p, N)}{(S, T, N)} ~.
        \end{equation*} 
        Using the second of~\eqref{td:int:h}, we obtain
        \begin{equation*}
            \pdv{V}{N} \Big \vert_{p, S} = \pdv{\mu}{p} \Big \vert_{N, S} \rightarrow \pdv{(V, p, S)}{(N, p, S)} = \pdv{(\mu, N, S)}{(p, N, S)} = - \pdv{(\mu, N, S)}{(N, p, S)} = \pdv{(N, \mu, S)}{(N, p, S)} ~,
        \end{equation*} 
        hence, inverting the right-handed side, we find
        \begin{equation*}
            1 = \pdv{(V, p, S)}{(N, p, S)} \pdv{(N, \mu, S)}{(N, p, S)}^{-1} = \pdv{(V, p, S)}{(N, p, S)} \pdv{(N, p, S)}{(N, \mu, S)} = \pdv{(V, p, S)}{(N, \mu, S)} ~.
        \end{equation*} 
        Using the third of~\eqref{td:int:h}, we obtain
        \begin{equation*}
            \pdv{\mu}{S} \Big \vert_{N, p} = \pdv{T}{N} \Big \vert_{S, p} \rightarrow \pdv{(\mu, N, p)}{(S, N, p)} = \pdv{(T, S, p)}{(N, S, p)} = - \pdv{(T, S, p)}{(S, N,p)} = \pdv{(S, T, p)}{(S, N, p)}~,
        \end{equation*} 
        hence, inverting the right-handed side, we find
        \begin{equation*}
            1 = \pdv{(\mu, N, p)}{(S, N, p)} \pdv{(S, T, p)}{(S, N, p)}^{-1} = \pdv{(\mu, N, p)}{(S, N, p)} \pdv{(S, N, p)}{(S, T, p)} = \pdv{(\mu, N, p)}{(S, T, p)} ~.
        \end{equation*} 
    \end{proof}
    For the Gibbs free energy, they are 
    \begin{equation*}
        \pdv{(V, p, N)}{(S, T, N)} = 1 ~, \quad \pdv{(V, p, T)}{(N, \mu, T)} = 1 ~, \quad \pdv{(S, T, p)}{(\mu, N, p)} = 1 ~.
    \end{equation*}
    \begin{proof}
        Using the first of~\eqref{td:int:g}, we obtain
        \begin{equation*}
            - \pdv{V}{T} \Big \vert_{p, N} = \pdv{S}{p} \Big \vert_{T, N} \rightarrow \pdv{(V, p, N)}{(T, p, N)} = - \pdv{(S, T, N)}{(p, T, N)} = \pdv{(S, T, N)}{(T, p, N)} ~,
        \end{equation*} 
        hence, inverting the right-handed side, we find
        \begin{equation*}
            1 = \pdv{(V, p, N)}{(T, p, N)} \pdv{(S, T, N)}{(T, p, N)}^{-1} = \pdv{(V, p, N)}{(T, p, N)} \pdv{(T, p, N)}{(S, T, N)} = \pdv{(V, p, N)}{(S, T, N)} ~.
        \end{equation*} 
        Using the second of~\eqref{td:int:g}, we obtain
        \begin{equation*}
            \pdv{V}{N} \Big \vert_{p, T} = \pdv{\mu}{p} \Big \vert_{N, T} \rightarrow \pdv{(V, p, T)}{(N, p, T)} = \pdv{(\mu, N, T)}{(p, N, T)} = - \pdv{(\mu, N, T)}{(N, p, T)} = \pdv{(N, \mu, T)}{(N, p, T)} ~,
        \end{equation*} 
        hence, inverting the right-handed side, we find
        \begin{equation*}
            1 = \pdv{(V, p, T)}{(N, p, T)} \pdv{(N, \mu, T)}{(N, p, T)}^{-1} = \pdv{(V, p, T)}{(N, p, T)} \pdv{(N, p, T)}{(N, \mu, T)} = \pdv{(V, p, T)}{(N, \mu, T)} ~.
        \end{equation*} 
        Using the third of~\eqref{td:int:g}, we obtain
        \begin{equation*}
            - \pdv{S}{N} \Big \vert_{T, p} = \pdv{\mu}{T} \Big \vert_{N, p} \rightarrow \pdv{(S, T, p)}{(N, T, p)} = - \pdv{(\mu, N, p)}{(T, N, p)} = \pdv{(\mu, N, p)}{(N, T, p)} ~,
        \end{equation*} 
        hence, inverting the right-handed side, we find
        \begin{equation*}
            1 = \pdv{(S, T, p)}{(N, T, p)} \pdv{(\mu, N, p)}{(N, T, p)}^{-1} = \pdv{(S, T, p)}{(N, T, p)} \pdv{(N, T, p)}{(\mu, N, p)} = \pdv{(S, T, p)}{(\mu, N, p)} ~.
        \end{equation*} 
    \end{proof}
    For the grand potential, they are 
    \begin{equation*}
        \pdv{(S, T, V)}{(\mu, N, V)} = 1 ~, \quad \pdv{(S, T, \mu)}{(V, p, \mu)} = 1 ~, \quad \pdv{(p, V, T)}{(\mu, N, T)} = 1 ~.
    \end{equation*}
    \begin{proof}
        Using the first of~\eqref{td:int:o}, we obtain
        \begin{equation*}
            \pdv{S}{\mu} \Big \vert_{T, V} = \pdv{N}{T} \Big \vert_{\mu, V} \rightarrow \pdv{(S, T, V)}{(\mu, T, V)} = \pdv{(N, \mu, V)}{(T, \mu, V)} = - \pdv{(N, \mu, V)}{(\mu, T, V)} = \pdv{(\mu, N, V)}{(\mu, T, V)} ~,
        \end{equation*} 
        hence, inverting the right-handed side, we find
        \begin{equation*}
            1 = \pdv{(S, T, V)}{(\mu, T, V)} \pdv{(\mu, N, V)}{(\mu, T, V)}^{-1} = \pdv{(S, T, V)}{(\mu, T, V)} \pdv{(\mu, T, V)}{(\mu, N, V)} = \pdv{(S, T, V)}{(\mu, N, V)} ~.
        \end{equation*} 
        Using the second of~\eqref{td:int:o}, we obtain
        \begin{equation*}
            \pdv{S}{V} \Big \vert_{T, \mu} = \pdv{p}{T} \Big \vert_{V, \mu} \rightarrow \pdv{(S, T, \mu)}{(V, T, \mu)} = \pdv{(p, V, \mu)}{(T, V, \mu)} = - \pdv{(p, V, \mu)}{(V, T, \mu)} = \pdv{(V, p, \mu)}{(V, T, \mu)} ~,
        \end{equation*} 
        hence, inverting the right-handed side, we find
        \begin{equation*}
            1 = \pdv{(S, T, \mu)}{(V, T, \mu)} \pdv{(V, p, \mu)}{(V, T, \mu)}^{-1} = \pdv{(S, T, \mu)}{(V, T, \mu)} \pdv{(V, T, \mu)}{(V, p, \mu)} = \pdv{(S, T, \mu)}{(V, p, \mu)} ~.
        \end{equation*} 
        Using the third of~\eqref{td:int:o}, we obtain
        \begin{equation*}
            \pdv{p}{\mu} \Big \vert_{V, T} = \pdv{N}{V} \Big \vert_{\mu, T} \rightarrow \pdv{(p, V, T)}{(\mu, V, T)} = \pdv{(N, \mu, T)}{(V, \mu, T)} = - \pdv{(N, \mu, T)}{(\mu, V, T)} = \pdv{(\mu, N, T)}{(\mu, V, T)} ~,
        \end{equation*} 
        hence, inverting the right-handed side, we find
        \begin{equation*}
            1 = \pdv{(p, V, T)}{(\mu, V, T)} \pdv{(\mu, N, T)}{(\mu, V, T)}^{-1} = \pdv{(p, V, T)}{(\mu, V, T)} \pdv{(\mu, V, T)}{(\mu, N, T)} = \pdv{(p, V, T)}{(\mu, N, T)} ~.
        \end{equation*} 
    \end{proof}

    A summary of all integrability conditions is given by 
    \begin{equation*}
        \pdv{(T, S, N)}{(p, V, N)} = 1 ~, \quad \pdv{(T, S, V)}{(N, \mu, V)} = 1 ~, \quad \pdv{(p, V, S)}{(\mu, N, S)} = 1 ~,
    \end{equation*}
    \begin{equation*}
        \pdv{(S, T, N)}{(V, p, N)} = 1 ~, \quad \pdv{(S, T, V)}{(\mu, N, V)} = 1 ~, \quad \pdv{(p, V, T)}{(\mu, N, T)} = 1 ~,
    \end{equation*}
    \begin{equation*}
        \pdv{(V, p, N)}{(S, T, N)} = 1 ~, \quad \pdv{(V, p, S)}{(N, \mu, S)} = 1 ~, \quad \pdv{(\mu, N, p)}{(S, T, p)} = 1 ~,
    \end{equation*}
    \begin{equation*}
        \pdv{(V, p, N)}{(S, T, N)} = 1 ~, \quad \pdv{(V, p, T)}{(N, \mu, T)} = 1 ~, \quad \pdv{(S, T, p)}{(\mu, N, p)} = 1 ~,
    \end{equation*}
    \begin{equation*}
        \pdv{(S, T, V)}{(\mu, N, V)} = 1 ~, \quad \pdv{(S, T, \mu)}{(V, p, \mu)} = 1 ~, \quad \pdv{(p, V, T)}{(\mu, N, T)} = 1 ~.
    \end{equation*}

    Notice that not all the Maxwell's relations are independent, but only $6$ of them are
    \begin{equation}\label{td:max}
    \begin{gathered}
        \pdv{(p, V, S)}{(\mu, N, S)} = 1 ~, \quad \pdv{(p, V, T)}{(\mu, N, T)} = 1 ~, \quad \pdv{(p, V, N)}{(T, S, N)} = 1 ~, \\ \pdv{(T, S, \mu)}{(p, V, \mu)} = 1 ~, \quad \pdv{(T, S, p)}{(N, \mu, p)} = 1 ~, \quad \pdv{(T, S, V)}{(N, \mu, V)} = 1 ~.
    \end{gathered}
    \end{equation}
    The geometrical interpretation is that coordinate transformations, which mean that we changed into a different chart of independent thermodynamic variables, preserve volumes. 

\section{Stability conditions}

    Every thermodynamic potential has a natural chart. In fact, the configuration of stable equilbrium can be obtained by a set of variational principle, which can be derived by fixing to constants the natural independent variables. This variations principle derive from the second law of thermodynamics, since all systems evolve spontaneously to maximise entropy. Therefore, minima of the thermodynamic potentials correspond to stable equilibrium under boundary condition which keep constant the natural variables
    \begin{equation*}
        (T, V, N) = \text{const} \rightarrow \delta F = 0 ~, \delta^2 F > 0 ~, 
    \end{equation*}
    \begin{equation*}
        (S, p, N) = \text{const} \rightarrow \delta H = 0 ~, \delta^2 H > 0 ~, 
    \end{equation*}
    \begin{equation*}
        (T, p, N) = \text{const} \rightarrow \delta G = 0 ~, \delta^2 G > 0 ~, 
    \end{equation*}
    \begin{equation*}
        (T, V, \mu) = \text{const} \rightarrow \delta \Omega = 0 ~, \delta^2 \Omega > 0 ~.
    \end{equation*}

    Equilibrium of two subsystems requires that $T$, $p$ and $\mu$ are equal.
    \begin{proof}
        Consider two subsystems $A$ and $B$ with extensive variables $(E_A, V_A, N_A)$ and $(E_B, V_B, N_B)$. Therefore $E = E_A + E_B$, $V = V_A + V_B$ and $N = N_A + N_B$. The whole system is at fixed boundary conditions $E, V, S = const$. The entropy is additive 
        \begin{equation*}
            S = S_A + S_B = S_A(E_A, V_A, N_A) - S_B(E - E_A, V-V_A, N-N_A) ~.
        \end{equation*}
        Computing its derivative and imposing it to zero, using~\eqref{td:es:s}
        \begin{equation*}
        \begin{aligned}
            0 = \delta S & = \pdv{S_A}{E_A} \delta E_A + \pdv{S_A}{E_A} \delta E_A + \pdv{S_A}{V_A} \delta V_A + \pdv{S_A}{N_A} \delta N_A \\ & \quad + \pdv{S_B}{E_B} \underbrace{\delta (E - E_A)}_{- \delta E_A} + \pdv{S_B}{V_B} \underbrace{\delta (V - V_A)}_{- \delta V_A} + \pdv{S_B}{N_B} \underbrace{\delta (N - N_A)}_{- \delta N_A} \\ & = \pdv{S_A}{E_A} \delta E_A + \pdv{S_A}{V_A} \delta V_A + \pdv{S_A}{N_A} \delta N_A - \pdv{S_B}{E_B} \delta E_A - \pdv{S_B}{V_B}  \delta V_A - \pdv{S_B}{N_B} \delta N_A  \\ & = \delta E_A \Big ( \underbrace{\pdv{S_A}{E_A}}_{\frac{1}{T_A}} - \underbrace{\pdv{S_B}{E_B}}_{\frac{1}{T_B}} \Big) + \delta V_A \Big ( \underbrace{\pdv{S_A}{V_A}}_{\frac{p_A}{T_A}} - \underbrace{\pdv{S_B}{E_B}}_{\frac{p_B}{T_B}} \Big) + \delta N_A \Big (\underbrace{\pdv{S_A}{N_A}}_{ - \frac{\mu_A}{T_A}} - \underbrace{\pdv{S_B}{N_B}}_{- \frac{\mu_B}{T_B}} \Big) \\ & = \delta E_A \Big ( \frac{1}{T_A} - \frac{1}{T_B} \Big) + \delta V_A \Big ( \frac{p_A}{T_A} - \frac{p_B}{T_B} \Big) + \delta N_A \Big (- \frac{\mu_A}{T_A} + \frac{\mu_B}{T_B} \Big) ~,
        \end{aligned}
        \end{equation*}
        hence, by arbitrarity of $\delta E_A$, $\delta V_A$ and $\delta N_A$,
        \begin{equation*}
            T_A = T_B ~, \quad p_A = p_B ~, \quad \mu_A = \mu_B ~.
        \end{equation*}
    \end{proof}

    At $T, p, N = const$, the stability condition is 
    \begin{equation}\label{td:stab}
    \begin{aligned}
        & E_{SS} = \pdv{T}{S} \Big \vert_V > 0 ~, \quad E_{VV} = - \pdv{p}{V} \Big \vert_S > 0 ~, \\ & E_{SS}E_{VV} - E^2_{SV} = - \pdv{T}{S} \Big \vert_V \pdv{p}{V} \Big \vert_S - \Big ( \pdv{p}{S} \Big \vert_V \Big )^2 = - \pdv{T}{S} \Big \vert_V \pdv{p}{V} \Big \vert_S - \Big ( \pdv{T}{V} \Big \vert_S \Big )^2 > 0 ~.
    \end{aligned}
    \end{equation}
    \begin{proof}
        We know that $E = E(S, V, N)$ by~\eqref{td:coord:e}. At constant $N$, its variation is 
        \begin{equation*}
        \begin{aligned}
            \delta E & = \underbrace{\pdv{E}{S} \Big \vert_V }_T \delta S + \underbrace{\pdv{E}{V} \Big \vert_S}_{-p} \delta V  \\ & \quad + \frac{1}{2} \Big ( \underbrace{\pdvdu{E}{S} \Big \vert_V}_{E_{SS}} \delta S^2 + 2 \underbrace{\pdvd{E}{S}{V}}_{E_{SV}} \delta S \delta V + \underbrace{\pdvdu{E}{V} \Big \vert_S }_{E_{VV}} \delta V^2 \Big) \\ & = T \delta S - p \delta V + \frac{1}{2} \Big ( E_{SS} \delta S^2 + 2 E_{SV} \delta S \delta V + E_{VV} \delta V^2 \Big) ~.
        \end{aligned}
        \end{equation*}
        Via the variation of the Gibbs free energy, the first derivative terms vanishes. In fact
        \begin{equation*}
        \begin{aligned}
            0 = \delta G &= \delta E - T \delta S + p \delta V \\ & = \cancel{T \delta S} - \cancel{p \delta V} + \frac{1}{2} \Big ( E_{SS} \delta S^2 + 2 E_{SV} \delta S \delta V + E_{VV} \delta V^2 \Big) - \cancel{T \delta S} + \cancel{p \delta V} \\ & = \frac{1}{2} \Big ( E_{SS} \delta S^2 + 2 E_{SV} \delta S \delta V + E_{VV} \delta V^2 \Big) ~.
        \end{aligned}
        \end{equation*}
        By imposing that $\delta^2 E > 0$, the condition to be a minimum is that 
        \begin{equation*}
            E_{SS} > 0 ~, \quad E_{VV} > 0 ~, \quad E_{SS} E_{VV} - E_{SV}^2 > 0 ~,
        \end{equation*}
        which, respectively, become, using~\eqref{td:es:e} 
        \begin{equation*}
            E_{SS} = \pdv{}{S} \underbrace{\pdv{E}{S} \Big \vert_V}_{T}  = \pdv{T}{S} \Big \vert_V > 0 ~,
        \end{equation*}
        \begin{equation*}
            E_{VV} = \pdv{}{V} \underbrace{\pdv{E}{V} \Big \vert_S}_{-p} = - \pdv{p}{V} \Big \vert_S > 0 ~,
        \end{equation*}
        \begin{equation*}
            E_{SS}E_{VV} - E^2_{SV} = - \pdv{T}{S} \Big \vert_V \pdv{p}{V} \Big \vert_S - \Big ( \pdv{p}{S} \Big \vert_V \Big )^2 = - \pdv{T}{S} \Big \vert_V \pdv{p}{V} \Big \vert_S - \Big ( \pdv{T}{V} \Big \vert_S \Big )^2 > 0 ~,
        \end{equation*}
        where we have computed, using the fact that partial derivatives commute,
        \begin{equation*}
            E_{SV} = \pdv{}{S} \Big \vert_V \underbrace{\pdv{E}{V} \Big \vert_S}_{-p} = - \pdv{p}{S} \Big \vert_V = \pdv{}{V} \Big \vert_S \underbrace{\pdv{E}{S} \Big \vert_V}_{T} = \pdv{T}{V} \Big \vert_S ~.
        \end{equation*}
    \end{proof}

\section{Specific heats and compressibilities}

    Stability conditions can be written in terms of two thermodynamic quantities that are defined as second derivatives of thermodynamic potentials: specific heat at constant volume or pressure and adiabatic or isothermal compressibility.

    Specific heat at constant volume is defined as 
    \begin{equation}\label{td:cv}
        C_V = \frac{\delta Q}{d T} \Big \vert_V = T \pdv{S}{T} \Big \vert_V ~,
    \end{equation}
    specific heat at constant pressure is defined as 
    \begin{equation}\label{td:cp}
        C_p = \frac{\delta Q}{d T} \Big \vert_p = T \pdv{S}{T} \Big \vert_p ~,
    \end{equation}
    adiabatic compressibility is defined as 
    \begin{equation*}
        \chi_S = - \frac{1}{V} \pdv{V}{p} \Big \vert_S 
    \end{equation*}
    and isothermal compressibility is defined as 
    \begin{equation}
        \chi_T = - \frac{1}{V} \pdv{V}{p} \Big \vert_T ~.
    \end{equation}
    Therefore, stability conditions can be writtes as 
    \begin{equation*}
        C_V > 0 ~, \quad C_p > 0 ~, \quad  \chi_S > 0 ~, \quad \chi_T > 0 ~.
    \end{equation*}
    \begin{proof}
        For the first, using the first of~\eqref{td:stab} and $T > 0$,
        \begin{equation*}
            C_V = T \pdv{S}{T} \Big \vert_{V} > 0 ~.
        \end{equation*}
        For the second, using the fact that $C_p > C_V > 0$.
        For the third, using the second of~\eqref{td:stab} and $V > 0$
        \begin{equation*}
            \chi_S = - \frac{1}{V} \pdv{V}{p} \Big \vert_S > 0 ~.
        \end{equation*}
        For the fourth, using the third of~\eqref{td:stab} and~\eqref{td:int:g}
        \begin{equation*}
        \begin{aligned}
            0 & < \pdv{T}{V} \Big \vert_S \pdv{T}{V} \Big \vert_S + \pdv{T}{S} \Big \vert_V \pdv{p}{V} \Big \vert_S = - \pdv{T}{V} \Big \vert_S \pdv{p}{S} \Big \vert_V + \pdv{T}{S} \Big \vert_V \pdv{p}{V} \Big \vert_S \\ & = \pdv{(T,p)}{(S,V)} = \pdv{(T,p)}{(T,V)} \pdv{(T,V)}{(S,V)} = \pdv{p}{V} \Big \vert_T \pdv{T}{S} \Big \vert_V = \frac{T}{C_V} \pdv{p}{V} \Big \vert_T ~,
        \end{aligned}
        \end{equation*}
        hence, by $T>0$, $C_V>0$ and $V>0$, 
        \begin{equation*}
            \chi_T = - \frac{1}{V} \pdv{V}{p} \Big \vert_T > 0 ~.
        \end{equation*}
    \end{proof}
    Moreover, we have the relations between the specific heats and compressibilities are
    \begin{equation*}
        \chi_T (C_P - C_V) = T V \alpha_p^2 ~, \quad C_p (\chi_T - \chi_S) = T V \alpha_p^2 ~,
    \end{equation*}
    which implies that 
    \begin{equation*}
        C_P > C_V \iff \chi_T > \chi_S ~.
    \end{equation*}
    \begin{proof}
        Using~\eqref{td:d:e}, at constant $V$, we have
        \begin{equation*}
            T dS = dE ~,
        \end{equation*}
        and at constant $p$, we have
        \begin{equation*}
            T dS = dE + p dV ~,
        \end{equation*}
        hence, we find
        \begin{equation*}
            C_V = T \pdv{S}{T} \Big \vert_V = \pdv{E}{T} \Big \vert_V ~,
        \end{equation*}
        which imply that, by isolating $TdS$,
        \begin{equation*}
            T dS = C_V dT + ( \pdv{E}{V} \Big \vert_{T} + p) dV = C_V dT + T \pdv{p}{T} \Big \vert_V dV ~,
        \end{equation*}
        and 
        \begin{equation*}
            C_p = T \pdv{S}{T} \Big \vert_p = \pdv{E}{T} \Big \vert_p + p \pdv{V}{T} \Big \vert_p  ~,
        \end{equation*}
        which implies that, by isolating $TdS$,
        \begin{equation*}
            T dS = C_p dT + ( \pdv{E}{p} \Big \vert_T + \pdv{V}{p} \Big \vert_T ) dp = C_p dT - T \pdv{V}{T} \Big \vert_p dp ~.
        \end{equation*}
        Comparing them, we find 
        \begin{equation*}
            (C_p - C_V) dT = T (\pdv{V}{T} \Big \vert_p dp + \pdv{p}{T} \Big \vert_V dV) ~,
        \end{equation*}
        which, assuming that $T(p, V)$, becomes
        \begin{equation*}
            (C_p - C_V) = T \pdv{V}{T} \Big \vert_p \pdv{p}{T} \Big \vert_V ~.
        \end{equation*}
        Now, we compute
        \begin{equation*}
            \pdv{p}{T} \Big \vert_V = \pdv{(p, V)}{(T, V)} = \pdv{(p,V)}{(p, T)} \pdv{(p, T)}{(T, V)} = - \pdv{(p,V)}{(p, T)} \pdv{(p, T)}{(V, T)} = - \pdv{V}{T} \Big \vert_p \pdv{p}{V} \Big \vert_T ~,
        \end{equation*}
        hence, we find
        \begin{equation*}
            C_p - C_V = - T \pdv{V}{T} \Big \vert_p \pdv{V}{T} \Big \vert_p \underbrace{\pdv{p}{V} \Big \vert_T}_{- \frac{1}{V \chi_T}} = \frac{T}{V \chi_T} \Big ( \pdv{V}{T} \Big \vert_p \Big)^2 ~,
        \end{equation*}
        or, equivalently,
        \begin{equation*}
            \chi_T (C_P - C_V) = T V \alpha_p^2 ~.
        \end{equation*}
        where we have defined the thermal expansion coefficient
         \begin{equation}
            \alpha_p = \frac{1}{V} \pdv{V}{T} \Big \vert_p ~.
        \end{equation}
        Moreover, using
        \begin{equation*}
            \pdv{a}{b} \Big \vert_c = \pdv{a}{b} \Big \vert_d + \pdv{a}{d} \Big \vert_b \pdv{d}{b} \Big \vert_c ~,
        \end{equation*}
        we can express $C_V$ in terms of $C_p$, using~\eqref{td:int:g} 
        \begin{equation*}
            \chi_T = - \frac{1}{V} \pdv{V}{p} \Big \vert_T = - \frac{1}{V} \pdv{V}{p} \Big \vert_S - \frac{1}{V} \pdv{V}{S} \Big \vert_p \pdv{S}{p} \Big \vert_T = \chi_S - \frac{1}{V} \pdv{V}{S} \Big \vert_p \pdv{S}{p} \Big \vert_T ~,
        \end{equation*}
        hence, we find, using the first of~\eqref{td:int:g}
        \begin{equation*}
        \begin{aligned}
            \chi_T - \chi_S & = - \frac{1}{V} \pdv{V}{S} \Big \vert_p \pdv{S}{p} \Big \vert_T = \frac{1}{V} \pdv{V}{S} \Big \vert_p \pdv{V}{T} \Big \vert_p  = \frac{1}{V} \pdv{(V, p)}{(S, p)} \pdv{(V, p)}{(T, p)}\\ &  = \frac{1}{V} \pdv{(V, p)}{(T, p)} \pdv{(T, p)}{(S, p)} \pdv{(V, p)}{(T, p)} = \frac{1}{V} \pdv{(T, p)}{(S, p)} (\pdv{(V, p)}{(T, p)})^2 \\ & = \frac{1}{V} \underbrace{\pdv{T}{S} \Big \vert_p}_{\frac{C}{C_p}} (\underbrace{\pdv{V}{T} \Big \vert_p}_{V \alpha_p})^2 = \frac{T V \alpha_p^2}{C_p} 
        \end{aligned}
        \end{equation*}
        or, equivalently,
        \begin{equation*}
            C_p (\chi_T - \chi_S) = T V \alpha_p^2 ~.
        \end{equation*}
        Finally, comparing the two expressions, we have 
        \begin{equation*}
            T V \alpha_p^2 = C_p (\chi_T - \chi_S) = \chi_T (C_p - C_T) ~,
        \end{equation*}
        \begin{equation*}
            \frac{C_p - C_V}{C_p} = 1 - \frac{C_V}{C_p} = \frac{\chi_T - \chi_S}{\chi_T} = 1 - \frac{\chi_S}{\chi_T} ~,
        \end{equation*}
        hence, we find 
        \begin{equation*}
            \frac{C_p}{C_V} = \frac{\chi_S}{\chi_T} ~.
        \end{equation*}
    \end{proof}

    In order to explicitly compute these quantities, it is useful to see that $C_V$ is 
    \begin{equation}\label{td:cv2}
        C_V = \pdv{E}{T} \Big \vert_{V, N}
    \end{equation}
    and that $C_V$ and $C_p$ are related by 
    \begin{equation}\label{td:cp2}
        C_p = C_V + p \pdv{V}{T} \Big \vert_{p, N}~.
    \end{equation}
    \begin{proof}
        For the first, at constant $V$ and $N$,~\eqref{td:d:e} becomes 
        \begin{equation*}
            dE = T dS ~,
        \end{equation*}
        hence, using~\eqref{td:cv}, we find 
        \begin{equation*}
            C_V = T \pdv{S}{T} \Big \vert_{V, N} = \pdv{E}{T} \Big \vert_{V, N} ~.
        \end{equation*}
        For the second, at constant $p$ and $N$,~\eqref{td:d:e} becomes 
        \begin{equation*}
            dE = T dS - pdV ~,
        \end{equation*}
        hence, using~\eqref{td:cp}, we find 
        \begin{equation*}
            C_p = T \pdv{S}{T} \Big \vert_{p, N} = \pdv{E}{T} \Big \vert_{V, N} + p \pdv{V}{T} \Big \vert_{p, N} = C_V + p \pdv{V}{T} \Big \vert_{p, N} ~.
        \end{equation*}
    \end{proof}

    Consequently to stability, $F$ is a concave of $T$ and convex of $V$, whereas $G$ is concave of both $T$ and $p$
    \begin{equation*}
        \pdvdu{F}{T} \Big \vert_V < 0 ~, \quad \pdvdu{F}{V} \Big \vert_T > 0 ~, \quad \pdvdu{G}{T} \Big \vert_p < 0 ~, \quad \pdvdu{G}{p} \Big \vert_T < 0 ~.
    \end{equation*}
    \begin{proof}
        For the concavity of $F$ in $T$, using the first of~\eqref{td:es:f}, we have
        \begin{equation*}
            C_V = T \pdv{S}{T} \Big \vert_{V} = - T \pdvdu{F}{T} \Big \vert_V > 0 ~,
        \end{equation*}
        hence, we find
        \begin{equation*}
            \pdvdu{F}{T} \Big \vert_V < 0 ~.
        \end{equation*}
        For the convexity of $F$ in $V$, using the second of~\eqref{td:es:f} and inverting it, we have
        \begin{equation*}
            \chi_T = - \frac{1}{V} \pdv{V}{p} \Big \vert_T = \Big ( V \pdvdu{F}{V} \Big \vert_T \Big)^{-1} > 0 ~,
        \end{equation*}
        hence, we find
        \begin{equation*}
            \pdvdu{F}{V} \Big \vert_T > 0 ~.
        \end{equation*}
        For the concavity of $G$ in $T$, using the first of~\eqref{td:es:g}, we have
        \begin{equation*}
            C_P = T \pdv{S}{T} \Big \vert_{P} = - T \pdvdu{G}{T} \Big \vert_p > 0 ~,
        \end{equation*}
        hence, we find
        \begin{equation*}
            \pdvdu{G}{T} \Big \vert_p < 0 ~.
        \end{equation*}
        For the concavity of $G$ in $p$, using the second of~\eqref{td:es:g}, we have
        \begin{equation*}
            \chi_T = - \frac{1}{V} \pdv{V}{p} \Big \vert_T = - \frac{1}{V} \pdvdu{G}{p} \Big \vert_T > 0 ~,
        \end{equation*}
        hence, we find
        \begin{equation*}
            \pdvdu{G}{p} \Big \vert_T < 0 ~.
        \end{equation*}
    \end{proof}
    Furthermore, the second law of thermodynamics~\eqref{td:2nd} can be expressed, in order to maximise entropy, by imposing that its first derivatives vanish and the hessian, i.e.~the matrix with its second derivatives, must be negative defined. Therefore, it must be (locally) concave in $E$, $V$ and $N$
    \begin{equation*}
        \pdvdu{S}{E} \Big \vert_{V, N} < 0 ~, \quad \pdvdu{S}{V} \Big \vert_{E, N} < 0 ~, \quad \pdvdu{S}{N} \Big \vert_{E, V} < 0 ~.
    \end{equation*}

    When we cease to work at constant $N$, the stability condition becomes
    \begin{equation*}
        \pdv{N}{\mu} \Big \vert_{V, T} = \frac{N^2}{V} \chi_T > 0 ~.
    \end{equation*}
    \begin{proof}
        In fact, using the second of~\eqref{td:max} and the properties of the Jacobian determinant, we obtain
        \begin{equation*}
        \begin{aligned}
            \pdv{N}{\mu} \Big \vert_{V, T} & = \pdv{(N, V, T)}{(\mu, V, T)} = \pdv{(N, V, T)}{(\mu, V, T)} \pdv{(p, V, T)}{(\mu, N, T)} = \pdv{(N, V, T)}{(N, p, T)}\pdv{(N, p, T)}{(\mu, V, T)} \pdv{(p, V, T)}{(\mu, N, T)} \\ & = \pdv{(N, V, T)}{(N, p, T)} \pdv{(N, p, T)}{(p, V, T)} \pdv{(p, V, T)}{(\mu, V, T)} \pdv{(p, V, T)}{(\mu, N, T)} = \pdv{(N, V, T)}{(N, p, T)} \pdv{(N, p, T)}{(\mu, N, T)} \pdv{(p, V, T)}{(\mu, V, T)} \\ & = - \pdv{(V, N, T)}{(p, N, T)} \pdv{(p, N, T)}{(\mu, N, T)} \pdv{(p, V, T)}{(\mu, V, T)} = - \pdv{V}{p} \Big \vert_{N, T}  \pdv{p}{\mu} \Big \vert_{V, T} \pdv{p}{\mu} \Big \vert_{N, T} ~.
        \end{aligned}
        \end{equation*}
        Now, using~\eqref{td:gd} and the fact that $\mu$ is intensive, we have
        \begin{equation*}
            \pdv{p}{\mu} \Big \vert_{V, T} = \pdv{p}{\mu} \Big \vert_{N, T} = \Big ( \pdv{\mu}{p} \Big \vert_T \Big)^{-1} = \frac{1}{v} = \frac{N}{V} ~,
        \end{equation*}
        hence, we find
        \begin{equation*}
        \pdv{N}{\mu} \Big \vert_{V, T} = - \frac{N^2}{V^2} \pdv{V}{p} \Big \vert_{N,T} = \frac{N^2}{V} \chi_T > 0 ~.
        \end{equation*}
    \end{proof}

    Finally, we find an important stability condition
    \begin{equation}\label{td:muT}
        \pdv{\mu}{T} \Big \vert_N \leq 0 ~.
    \end{equation}
    \begin{proof}
        In fact, using~\eqref{td:gd} and the second law of thermodynamic~\eqref{td:2nd}, we find 
        \begin{equation*}
            \pdv{\mu}{T} \Big \vert_{N, p} = - s \leq 0 ~.
        \end{equation*}
    \end{proof}

\part{Classical statistical mechanics}

\chapter{Classical mechanics}

    In this chapter, we will recall some basic notion of classical (Hamiltonian) mechanics: starting from general definitions of states, equations of motion, observables, time evolution and investigate the nature of phase space with its foliation using constant energy hypersurfaces, Liouville's theorem and ergodicity.

\section{States, observables, time evolution}
    
    Consider a dynamical system composed by $N$ degrees of freedom in a $d$-dimensional space. A physical state is defined as a point $P$ in a $2dN$-dimensional manifold $\mathcal M^N$, called the phase space, which is the Cartesian product of $N$ single-particle $2d$-dimensional manifolds $\mathcal M$. More formally, the phase space is the cotangent bundle $T^*\mathcal C$ of the configuration space $\mathcal C$. In this (smooth) manifold, we locally introduce a chart, which looks like $\mathbb R^{2dN}$, labelled by generalised coordinates $q^i$ and generalised momenta $p_i$, where $i = 1, \ldots dN$. Therefore, a state can be individuated by
    \begin{equation*}
        \{(q^i, ~p_i)\}_{i=1}^{dN} \in \mathcal M^{dN} ~.
    \end{equation*}
    However, we can combine this $2$ different set of coordinates $q^i$ and $p_i$ into a more convenient single one $\xi_i$ in the following way: 
    \begin{equation}\label{cm:symplcoord}
    \xi^j = \begin{cases}
        q^i & j = 1, \ldots dN \\
        p_i & j = dN+1, \ldots 2dN \\
    \end{cases} ~.
    \end{equation}
    In order to compute integrals, we can also introduce the standard (Lebesgue) measure
    \begin{equation}\label{cm:measure}
        d\Gamma = \prod_{i=1}^N d^d q^i ~ d^d p_i = \prod_{j=1}^{2N} d^d \xi_j ~.
    \end{equation}
    In the phase space, an observable is a smooth real function 
    \begin{equation*}
        f \colon \mathcal M^{dN} \rightarrow \mathbb R
    \end{equation*}
    and a measurement of this observable at a given time $t_0$ in a fixed point $(q^i(t_0), ~\tilde p_i(t_0))$ is the evaluation of the corresponding function in that point 
    \begin{equation*}
        f = f(q^i(t_0), p_i(t_0)) ~.
    \end{equation*}
    In order to describe time evolution of the system, we need to introduce a special real function $H(q^i, ~p_i, ~t)$, called the Hamiltonian of the system. Through it, we can find how the systems evolve in time by solving the equations of motion, called the Hamilton's equations,
    \begin{equation}\label{cm:hameq}
        \dot q^i = \pdv{H}{p_i} ~, \quad \dot p_i = - \pdv{H}{q^i} ~, \quad \text{or} \quad \dot \xi^j = J^{jk} \pdv{H}{\xi^k} ~,
    \end{equation}
    where $J^{jk}$ is the symplectic matrix, a $2N \times 2N$ constant-valued matrix, given by
    \begin{equation*}
        J^{jk} = \begin{bmatrix}
            0 & \mathbb I_N \\
            -\mathbb I_N & 0 \\
        \end{bmatrix} ~.
    \end{equation*}
    \begin{example}
        Consider a $1$-dimensional harmonic oscillator of mass $m$ and frequency $\omega$, vibrating around an equilibrium position $q_0$. Its Hamiltonian is given by
        \begin{equation*}
            H (q, p) = \frac{p^2}{2m} + \frac{m \omega^2}{2} (q - q_0)^2 ~,
        \end{equation*}
        whereas the Hamilton's equations are
        \begin{equation*}
            \dot q = \pdv{H}{p} = \frac{p}{m} ~, \quad \dot p = - \pdv{H}{q} = - m \omega^2 (q - q_0) ~.
        \end{equation*}
    \end{example}
    It is important to highlight that the Hamilton's equations are deterministic: once initial conditions $(q^i(t_0), p_i(t_0)) \in \mathcal M^{dN}$ are given, the trajectory in phase space of the system is uniquely and completely determined. In other words, there is one and only one trajectory passing through each point of the phase space and two trajectories can never intersect. It is a consequence of the existence and uniqueness theorem of differential equations. Furthermore, an important theorem allows us to understand what the Hamiltonian is.

    \begin{theorem}[Conservation of energy]\label{cm:consen}
        If the Hamiltonian does not depend explicitly on time, it can be interpreted physically as the energy of the system, which is constants
        \begin{equation*}
            H(q^i(t), p_i(t)) = H(q^i(t_0), p_i(t_0)) = E = \text{const} ~.
        \end{equation*}
    \end{theorem}

\section{Probability density distribution}

    A macrostate is defined by the knowledge of macroscopic thermodynamic quantities (boundary conditions like $p, V, N, T, \ldots$), whereas a microstate is defined by the knowledge of the microscopic behaviour of the system in the phase space $(q^i, p_i)$, which is fixed by the initial conditions. However, in general, there are more microstates associated to the same macrostates, since the information carried by a microstate is much more than the one carried by a macrostate. Intuitively, we can say that a system in a defined macrostate can be represented not by only one but by several microstates. This gives rise to the concept of ensemble. We fix a macrostate set-up, we create a large amount of copies of the same physical system and we look at the different microstate that can represent this macrostate. Mathematically, it can be studied with the introduction of a probability density distribution 
    \begin{equation*}
        \rho(q^i(t), p_i(t),t) \colon \mathcal M^{dN} \rightarrow \mathbb R^+ ~,
    \end{equation*}
    which satisfies the following properties
    \begin{enumerate}
        \item positivity, i.e.
        \begin{equation*}
            \rho(q^i, p_i, t) \geq 0 ~,
        \end{equation*}
        \item normalisation, i.e.
        \begin{equation}\label{cm:norm}
            \int_{\mathcal M^n} \prod_{i=1}^N d^d q^i d^d p_i ~ \rho(q^i, p_i, t) = \int_{\mathcal M^n} d\Gamma ~ \rho(q^i, p_i, t) = 1 ~.
        \end{equation}
    \end{enumerate}
    The probability to find the system in a finite portion of the phase space $\mathcal U \subset \mathcal M^{dN}$ is 
    \begin{equation*}
        \int_{\mathcal U} d\Gamma ~ \rho(q^i, p_i, t) ~.
    \end{equation*}

    Notice that there is a dimensional problem: a measure must be dimensionless but $d\Gamma$ in~\eqref{cm:measure} has the dimension of an action at the power of $dN$, i.e. $[d\Gamma] = [E]^{dN} [t]^{dN}$. To solve this problem, we introduce an ad hoc constant $h$, called the scale factor, which leads is to a dimensionless volume element
    \begin{equation}\label{cm:measure2}
        d \Omega = \frac{d\Gamma}{h^{dN}} = \frac{\prod_{i=1}^N d^d q^i d^d p^i}{h^{dN}} ~.
    \end{equation}
    It is only in quantum mechanics that $h$ will be interpretated as the Planck's constant.

\section{Liouville's theorem}

    Given $2$ functions of the phase space $f(q^i, p_i)$ and $g(q^i, p_i)$, we can define a bilinear mapping, called the Poisson's brackets, as
    \begin{equation}\label{cm:poi}
        \{f, g\} = \pdv{f}{q^i} \pdv{g}{p_i} - \pdv{f}{p_i} \pdv{g}{q^i} = \pdv{f}{\xi^j} J^{jk} \pdv{g}{\xi^k} ~,
    \end{equation}
    such that it satisfies the following properties, $\forall h(q^i, p_i)$
    \begin{enumerate}
        \item antisymmetry, i.e.
        \begin{equation*}
            \{f,g\} = - \{g, f\} ~,
        \end{equation*}
        \item Leibniz rule, i.e.
        \begin{equation*}
            \{f, gh\} = g \{f, h\} + \{f, g\} h ~,
        \end{equation*}
        \item Jacobi identity, i.e.
        \begin{equation*}
            \{f , \{g, h\}\} + \{g , \{h, f\}\} + \{h , \{f, g\}\} = 0  ~.
        \end{equation*} 
    \end{enumerate}
    Notice that the symplectic matrix can be defined by the Poisson brackets
    \begin{equation*}
        \{\xi^j, \xi^k \} = J^{jk} ~.
    \end{equation*}
    A canonical transformation $\xi \rightarrow \eta$ is a transformation of the phase space coordinates that preserve the structure of the Poisson's brackets 
    \begin{equation*}
        \{\xi^j, \xi^k\} = \{\eta^j, \eta^k\} = J^{jk} ~.
    \end{equation*}

    To a system of first-order differential equations, like the Hamilton's ones, we can associate an Hamiltonian flow generated by the Hamiltonian vector field
    \begin{equation*}
        \mathbf H = J^{jk} \pdv{H}{\xi^k} \pdv{}{\xi^j} ~.
    \end{equation*}
    The physical intepretation is that, with a fluid analogy, it keeps track of the motion of all particles.
    \begin{theorem}
        Time evolution governed by the Hamilton's equations is a canonical transformation.
    \end{theorem}
    \begin{theorem}
        Canonical transformations preserve volumes in phase space.
    \end{theorem}
    \begin{proof}
        A canonical transformation can be written as 
        \begin{equation*}
            J^{ab} = \{\eta^a, \eta^b\} = \pdv{\eta^a}{\xi^j} J^{jk} \pdv{\eta^b}{\xi^k} ~.
        \end{equation*}
        Now, we compute the determinant of this expression, observing that $M = \partial \eta / \partial \xi$ is the Jacobian matrix of this transformation, and we obtain, using the properties of the determinant
        \begin{equation*}
            \det J = \det (M J M^T) = \det^2 M \det J ~,
        \end{equation*}
        hence 
        \begin{equation*}
            |\det J| = 1 ~.
        \end{equation*}
    \end{proof}
    Combining the last $2$ theorems, we can state an important theorem, named after Liouville. 
    \begin{theorem}[Liouville]
        The volume through the (Hamiltonian) flow generated by the Hamilton's equations is constant
        \begin{equation*}
            \vol \Omega(t_0) = \vol \Omega(t) ~.
        \end{equation*}
        See Figure~\ref{fig:liou}.
    \end{theorem}
    \begin{figure}[h!]
        \centering
        \begin{tikzpicture}
            \path[->] (0, -0.2) edge [bend right] node[left, xshift=-2mm, yshift=0.3cm] {$\mathbf H$} (4, 0.8);

            \draw[smooth cycle, tension=0.4] plot coordinates{(2,2) (-2.5,0) (3,-2) (6,1)} node at (3,2.3) {$\mathcal M$};
        
            \draw[smooth cycle] 
                plot coordinates {(0.5, -1) (0.7, -0.2) (0.0, 0.6) (-0.5, 0.6) (-0.8, -0.2) (-0.7, -0.5) (-0.4, -0.7)} 
                node [label={[label distance=-0.3cm, xshift=-1cm, yshift=0.4cm]:$\Omega (t_0)$}] {};
            \draw[smooth cycle] 
                plot coordinates {(5, 0) (4.7, 0.8) (4.0, 1.1) (3.5, 1.4) (3.2, 0.8) (3.3, 0.5) (3.6, 0.3) (4.5, 0.0)} 
                node [label={[label distance=-0.8cm, xshift=0.75cm, yshift=1.25cm]:$\Omega(t)$}] {};
        \end{tikzpicture}
        \label{fig:liou}
        \caption{The (Hamiltonian) flow of the system in which $\vol \Omega(t_0) = \vol \Omega(t)$.}
    \end{figure}
    An important corollary of the Liouville's theorem about the property of the probability density distribution can be stated.
    \begin{corollary}
        The probability density distribution is constant in time. Mathematically
        \begin{equation}\label{cm:liou}
            \dv{\rho}{t} = \pdv{\rho}{t} + \{\rho, H\} = 0 ~.
        \end{equation}
    \end{corollary}
    The physical interpretation of this corollary is that particles do not appear nor disappear due to conservation of charge, mass, etc.
    \begin{proof}
        Consider the flow of a portion of phase space as it was a fluid with associated density $\rho$ and current $\mathbf J = \rho \mathbf v$. By the Liouville's theorem, it must satisfy a continuity equation 
        \begin{equation*}
            \pdv{\rho}{t} + \boldsymbol \nabla \cdot \mathbf J = 0 ~.
        \end{equation*}
        We introduce a local chart $(q^i, p_i)$ for the manifold and for the tangent space, in order to have $\mathbf v = (\dot q^i, \dot p_i)$ and $\boldsymbol \nabla = (\partial/\partial q^i, \partial/\partial p_i)$. Therefore, the continuity equation becomes
        \begin{equation*}
        \begin{aligned}
            0 & = \pdv{\rho}{t} + \boldsymbol \nabla \cdot \mathbf J = \pdv{\rho}{t} + \sum_i (\partial/\partial q^i, \partial/\partial p_i) \cdot (\rho \dot q^i, \rho \dot p_i) \\ & = \pdv{\rho}{t} + \sum_i ( \pdv{}{q^i} (\rho \dot q^i) + \pdv{}{p_i} (\rho \dot p_i) ) \\ & = \pdv{\rho}{t} + \sum_i ( \pdv{\rho}{q^i} \underbrace{\dot q^i}_{\pdv{H}{p_i}} + \rho \pdv{}{q^i} \underbrace{\dot q^i}_{\pdv{H}{p_i}} + \pdv{\rho}{p_i} \underbrace{\dot p_i}_{-\pdv{H}{q^i}} + \rho \pdv{}{p_i} \underbrace{\dot p_i}_{-\pdv{H}{q^i}} ) \\ & = \pdv{\rho}{t} + \sum_i ( \pdv{\rho}{q^i} \pdv{H}{p_i} + \cancel{\rho \pdvd{H}{p_i}{q^i}} - \pdv{\rho}{p_i} \pdv{H}{q^i} - \cancel{\rho \pdvd{H}{q^i}{p_i}} ) \\ & = \pdv{\rho}{t} + \sum_i ( \pdv{\rho}{q^i} \pdv{H}{p_i} - \pdv{\rho}{p_i} \pdv{H}{q^i} ) = \pdv{\rho}{t} + \{\rho, H\} ~,
        \end{aligned}
        \end{equation*}
        where we have used the Hamilton's equations~\eqref{cm:hameq}, the fact that partial derivatives commute and the definition of Poisson's brackets~\eqref{cm:poi}.
    \end{proof}
    For stationary systems, i.e~when $\pdv{\rho}{t} = 0$, the necessary condition for equilibrium is $\{\rho, H\} = 0$, which is satisfied only if 
    \begin{equation}\label{cm:rc}
        \rho(q^i, p_i) = const ~,
    \end{equation}
    like in the microcanonical ensemble, or 
    \begin{equation}\label{cm:rh}
        \rho(q^i, p_i) = \rho(H(q^i,p_i)) ~,
    \end{equation}
    like in the canonical or in the grand canonical ensemble.
    \begin{proof}
        For the first, we have 
        \begin{equation*}
            \{\rho, H\} = \underbrace{\pdv{\rho}{q^i}}_0 \pdv{H}{p_i} - \underbrace{\pdv{\rho}{p_i}}_0 \pdv{H}{q^i} = 0 ~.
        \end{equation*}
        For the second, we have 
        \begin{equation*}
        \begin{aligned}
            \{\rho, H\} = \pdv{\rho}{q^i} \pdv{H}{p_i} - \pdv{\rho}{p_i} \pdv{H}{q^i} = \cancel{\pdv{\rho}{H} \pdv{H}{q^i} \pdv{H}{p_i}} - \cancel{\pdv{\rho}{H} \pdv{H}{p_i} \pdv{H}{q^i}} = 0 ~
        \end{aligned}
        \end{equation*}
        where we have developed the dependence of $\rho$ on $H$ by the chain rule and the fact that partial derivatives commute.
    \end{proof}

    The average value of an observable $f$ is given by the volume of the function in phase space, weighted by the probability density distribution
    \begin{equation}\label{cm:av}
        \av{f} = \int_{\mathcal M^{dN}} d\Gamma ~ \rho(q^i, ~p_i) f(q^i, ~p_i) ~,
    \end{equation}
    while the standard deviation is defined by
    \begin{equation*}
        (\Delta f)^2 = \av{f^2} - \av{f}^2 ~.
    \end{equation*}

\section{Energy foliation}

    Consider a time-independent Hamiltonian $H = H(q^i, p_i)$, which implies by the theorem~\eqref{cm:consen} that energy is conserved $E = \text{const}$. In this case, there exists $2dN - 1$ independent (local) constants of motion. However, we are interested in global integrals, which are isolating, i.e.~they admit hypersurfaces in phase space, and foliating, i.e.~they admit a foliation of phase space via surfaces of constant value, called level surfaces. The most important global foliating isolating integral is the energy, sometimes it is even the only one. Therefore, we foliate the whole phase space into $2dN - 1$-dimensional hypersurfaces $S_E$ of constant energy. See Figure~\ref{fig:fol}. Notice that the structure of the manifold does not depend on the dynamics, but hypersurfaces do, because they depend on the different Hamiltonian $H$ chosen by the dynamics. In fact, different Hamiltonian will have different foliations. Furthermore, a different choice of the initial conditions means a different hypersurface.
    \begin{figure}[h!]
        \centering
        \begin{tikzpicture}
            \path[->] (0, -0.2) edge (5, 1) node[left, xshift=4.8cm, yshift=1.4cm] {$E$};

            \draw[smooth cycle, tension=0.4] plot coordinates{(2,2) (-2.5,0) (3,-2) (6,1)} node at (3,2.3) {$\mathcal M$};

            \draw[smooth cycle, tension=0] 
                plot coordinates { (0.75, 0.5) (0.75, 1.5) (4.5, 0.5) (4.5, -0.5)} 
                node [label={[label distance=-0.3cm, xshift=0.5cm, yshift=0.4cm]:$S_{E'}$}] {};
            \draw[smooth cycle, tension=0] 
                plot coordinates {(-1, -0.5) (-1, 0.5) (3.5, -0.5) (3.5, -1.5)} 
                node [label={[label distance=-0.8cm, xshift=0.4cm, yshift=0.8cm]:$S_{E}$}] {};
        \end{tikzpicture}
        \label{fig:fol}
        \caption{Foliation of the phase space $\mathcal M$ by hypersurfaces $S_E$ if constant energy. In this case $E' > E$.}
    \end{figure}
    Taking advantage of this foliation structure, it is easier to compute integrals in phase space. In fact, if we define the gradient of the Hamiltonian,
    \begin{equation*}
        \boldsymbol \nabla H = \pdv{H}{\boldsymbol \xi} ~,
    \end{equation*}
    which is by definition orthogonal to the energy hypersurfaces and has norm equals to 
    \begin{equation*}
        ||\boldsymbol \nabla H|| = \sqrt{\sum_i (\pdv{H}{\xi_i})^2} ~,
    \end{equation*}
    then we can decompose the phase space measure~\eqref{cm:measure} into 
    \begin{equation}\label{cm:fol}
        d \Gamma = dA dl = \underbrace{\frac{dA}{||\boldsymbol \nabla H||}}_{dS_E} \underbrace{||\boldsymbol \nabla H|| dl}_{dE} = dS_E dE ~,
    \end{equation}
    where $dA$ is the area element of the energy hypersurfaces and $dl$ is the line element orthogonal to this surface. Intuitively, we have passed from an integration of a volume of phase space $d\Gamma$ into an integration over hypersurfaces depending on the energy $dS_E$ along with an integration over energy $dE$. $dE$ is invariant, therefore $dS_E$ is an invariant area element. The volume of phase space enclosed by the energy hypersurface $S_E$ 
    \begin{equation*}
        \Sigma(E) = \{ \boldsymbol \xi \in \mathcal M \colon 0 \leq H(\xi) \leq E \} 
    \end{equation*}
    is given by 
    \begin{equation}\label{cm:vol}
        \Sigma(E) = \int_{0 \leq H \leq E} d\Gamma = \int_0^E dH \int_{S_E} dS_{H} = \int_0^E dH ~ \omega(H)
    \end{equation}
    where we have defined the density of states $\omega$ as 
    \begin{equation}\label{cm:denst}
        \omega (E) = \int_{\mathcal M^{dN}} d\Gamma ~ \delta (H- E) = \int_{S_E} dS_H = \pdv{\Sigma}{E} ~,
    \end{equation}
    exploiting a Dirac delta $\delta (H - E)$ to constrain ourselves into the hypersurface of energy $H = E$.

\section{Ergodicity}

    An important and fundamental concept in statistical mechanics is ergodicity. A physical system is said to be ergodic on an hypersurface of constant energy $S_E$ if and only if, in time evolution, almost\footnote{Up to a null measure set} every point $\xi^j \in S_E$ passes through every neighbourhood $U \subset S_E$. In other word, in a finite time interval $t \in (- \infty, \infty)$, the system samples every small neighbourhood of the surface and the set of trajectories is dense. Ergodicity implies that the only isolating integral is energy and there are no other conservation laws. In these notes, we will consider only ergodic systems.

    Given an observable $f$, we can associate $2$ different in principle average values~\eqref{cm:av}
    \begin{enumerate}
        \item phase-space average over the energy hypersurface $S_E$, since the motion is confined inside $S_E$
            \begin{equation*}
                \av{f}_E = \frac{1}{\omega(E)} \int_{S_E} dS_H ~ f = \frac{1}{\omega(E)} \int_{\mathcal M} d \Gamma ~ \delta (H - E) f = \frac{1}{\omega(E)} \pdv{}{E} \int_{\Sigma(E)} d \Gamma ~ f ~, 
            \end{equation*}
        \item (infinite) time average, that can be experimentally obtained by observing the system over a long amount of time 
            \begin{equation*}
                \av{f}_\infty = \lim_{t \rightarrow \infty} \int_{t_0}^{t_0 + \infty} dt ~ f(q^i(t), p_i(t)) ~, 
            \end{equation*}
            which is valid for almost every initial condition and it is independent of the initial time $t_0$.
    \end{enumerate}
    Ergodicity and average values are connected by a theorem.
    \begin{theorem}
        A system is ergodic if and only if 
        \begin{equation}\label{cm:erg}
            \av{f}_E = \av{f}_\infty ~.
        \end{equation}
    \end{theorem}
    Physically, this means that we can compare experimentally averages measured in laboratories with theoretical averages computed with ensemble theory. 

    Anyway, we shall stress that thermodynamics provides the completely macroscopic description of a physical system (equations of state, relations between thermodynamic quantities) once a single thermodynamic potential has been given. It will be a job for differents kind of ensemble to compute which thermodynamic potential. They are listed in Table~\ref{table:cm:1}.
    \begin{table}[h!]
        \centering
        \begin{tabular}{c | c }
            Ensemble & Thermodynamic potential \\
            \hline
            microcanonical & entropy $S$ \\ 
            canonical & Helmholtz free energy $F$ \\ 
            grand canonical & grand potential $\Omega$ \\ 
        \end{tabular}
        \caption{Ensembles with associated thermodynamic potential.}
        \label{table:cm:1}
    \end{table}
    
\chapter{Microcanonical ensemble}

    In this chapter, we will study the microcanonical ensemble: probability density distribution and we will recover thermodynamics by means of the entropy.

\section{Microcanonical probability density distribution}

    Consider a physical system that is isolated from the environment, i.e.~it cannot exchange neither energy nor matter so that boundary conditions $E$, $N$ and $V$ are fixed. Of course, isolated systems are a bit nonphysical. Since energy is conserved and the Hamiltonian is time-independent, the trajectory of motion is restricted on the surface $S_E$ and not on all phase space. This kind of set-up is called microcanonical ensemble and it has associated a probability density distribution, which a priori is uniform
    \begin{equation*}
        \rho_{mc}(q^i, ~p_i) = C \delta (H(q^i, p_i) - E) ~,
    \end{equation*}
    where $C$ is a normalisation constant, which can be evaluated by~\eqref{cm:norm}
    \begin{equation*}
        1 = \int_{\mathcal M^N} d\Omega ~ \rho_{mc} = \int_{\mathcal M^N} d\Omega ~ C \delta(H - E) = C \underbrace{\int_{\mathcal M^N} d\Omega ~ \delta(H - E)}_{\omega(E)} = C \omega(E) ~.
    \end{equation*}
    Hence, the microcanonical probability density distribution is
    \begin{equation}\label{mc:pdd}
        \rho_{mc}(q^i, ~p_i) = \frac{1}{\omega(E)} \delta (H(q^i, p_i) - E) ~.
    \end{equation}
    Notice that the probability is constant, like Liouville's theorem states~\eqref{cm:rc}. This distribution can be deduced by the following argument: suppose that the system has not exactly energy equals to $E$, but it is in a range $H \in [E, E + \Delta E]$, where $\Delta E$ is an infinitesimal displacement of energy, i.e. $\Delta E \ll 1$. The volume in phase space~\eqref{cm:vol} becomes 
    \begin{equation*}
        \Gamma (E) = \integ{E}{E+\Delta E}{E'} \omega(E') \simeq \omega(E) \Delta E = \pdv{\Sigma(E)}{E} \Delta E ~.
    \end{equation*}
    Consequently, the distribution becomes 
    \begin{equation*}
        \rho_{mc}(q^i, p_i) = \begin{cases}
            \frac{1}{\Gamma(E)} & H \in [E, E + \Delta E] \\
            0 & otherwise
        \end{cases}
    \end{equation*}
    Therefore, in the limit for which $\Delta E \rightarrow 0$, we can find exactly~\eqref{mc:pdd}, up to the normalisation constant.

    Let $f(q^i, p_i)$ be an observable, then its microcanonical average (or equivalenty its energy hypersurface average) is 
    \begin{equation}\label{mc:av}
        \avp{f(q^i, p_i)}{mc} = \int_{\mathcal M} d\Omega ~ \rho_{mc} f = \int_{\mathcal M} d\Omega ~ \frac{1}{\omega(E)} \delta (H - E) f = \frac{1}{\omega(E)} \int_{S_E} dS_E ~ f = \avp{f}{E} ~.
    \end{equation}

\section{Entropy as microcanonical potential}

    A local chart with coordinates $(E, V, N)$ is suitable for entropy~\eqref{td:coord:s}. Hence, we need to find an expression for this thermodynamic potential. The first guess is to define the microcanonical entropy $S_{mc}$ as 
    \begin{equation}\label{mc:s}
        S_{mc} (E, V, N) = k_B \ln \Gamma(E) ~,
    \end{equation}
    which, in the thermodynamic limit, it is equivalent to
    \begin{equation}\label{mc:tdlim}
        s_{mc} = \lim_{td} \frac{S_{mc}}{N} = k_B \lim_{td} \frac{\log \omega(E)}{N} = \underbrace{k_B \lim_{td} \frac{\log \Sigma(E)}{N}}_{H \in [0, E]} = \underbrace{k_B \lim_{td} \frac{\log \Gamma(E)}{N}}_{H \in [E, E + \Delta E]} ~.
    \end{equation}
    The logarithm is justified by the fact that the volume of a $N$-particle phase space is $(W_1)^N$, where $W_1$ is the volume of a single particle phase space. According to the properties of the logarithm, in this way, entropy becomes extensive.

    Now, we need to prove that~\eqref{mc:s} is indeed the thermodynamic entropy. The first property that it must fulfill is additivity: given $2$ subsystems $A$ and $B$, the total entropy is the sum of the $2$ separately subsystems entropy
    \begin{equation*}
        s_{mc}^{tot} = s_{mc}^{(1)} + s_{mc}^{(2)} ~.
    \end{equation*}
    \begin{proof}
        Consider two isolated subsystems $A$ and $B$ in contact at thermal equilibrium with the same temperature $T = T_1 = T_2$ but whatsoever volume $V_1$ and $V_2$ and energies $E_1$ and $E_2$. The total system is isolated and it can be treated as a microcanonical ensemble. The entropy of the subsystems will be respectively 
        \begin{equation*}
            S_1 = k_B \ln \Gamma_1(E_1) ~, \quad S_2 = k_B \ln \Gamma_2 (E_2) ~.
        \end{equation*}
        where $\Gamma_1 (E_1) \simeq \omega_1 (E_1) \Delta E_1$ and $\Gamma_2 (E_2) \simeq \omega_2 (E_2) \Delta E_2$. The total energy is $E = E_1 + E_2 + E_{surface}$ but, in the thermodynamic limit, the energy exchanged by the surface is a subleading term ($E_1$ and $E_2$ go as $L^3$ whereas $E_{surface}$ goes as $L^2$) and then it can be neglected. Therefore, the total energy becomes $E = E_1 + E_2$. The joint density of states is 
        \begin{equation*}
        \begin{aligned}
            \omega(E) & = \int_{\mathcal M^N} \underbrace{d\Gamma_1}_{dE_1 dS_{E_1}} ~ \underbrace{d\Gamma_2}_{dE_2 dS_{E_2}} ~ \delta(H - E) \\ & = \int dE_1 \int dS_{E_1} \int dE_2 \int dS_{E_2} ~ \delta (E - E_1 - E_2) \\ & = \int dE_1 \int dE_2 ~ \omega_1(E_1) \omega_2(E_2) \delta (E - E_1 - E_2) \\ & = \int_0^E dE_1 ~ \omega_1(E_1) \omega_2(E_2 = E - E_1) ~.
        \end{aligned}
        \end{equation*}
        Notice that the integrand is a positive function and it has a maximum in $E^*_1 \in [0, E]$. Hence, we can find an upper bound for the integral, which is
        \begin{equation}\label{proof1}
        \begin{aligned}
            \integ{0}{E}{E_1} \omega_1(E_1) \omega_2(E_2 = E - E_1) & \leq \omega_1(E^*_1) \omega_2(E^*_2 = E - E^*_1) \underbrace{\int_0^E dE_1}_E \\ & = \omega_1(E^*_1) \omega_2(E^*_2 = E - E^*_1) E ~.
        \end{aligned}
        \end{equation}
        On the other hand, it is always possible to find a value for a small enough $\Delta E$ in order to have 
        \begin{equation} \label{proof2}
            \omega_1(E^*_1) \omega_2(E^*_2 = E - E^*_1) \Delta E \leq \omega(E) ~.
        \end{equation}
        Putting together~\eqref{proof1} and~\eqref{proof2}, we obtain, after a series of manipulations
        \begin{equation*}
            \Delta E \omega_1(E^*_1) \omega_2(E^*_2) \leq \omega(E) \leq \omega_1(E^*_1) \omega_2(E^*_2) E ~,
        \end{equation*}
        \begin{equation*}
            \underbrace{\omega_1(E^*_1) \Delta E}_{\Gamma_1 (E^*_1)} \underbrace{\omega_2(E^*_2) \Delta E}_{\Gamma_2 (E^*_2)} \leq \underbrace{\omega(E) \Delta E}_{\Gamma(E)} \leq \frac{E}{\Delta E} \underbrace{\omega_1(E^*_1) \Delta E}_{\Gamma_1 (E^*_1)} \underbrace{\omega_2(E^*_2) \Delta E}_{\Gamma_2 (E^*_2)} ~,
        \end{equation*}
        \begin{equation*}
            \Gamma_1(E^*_1) \Gamma(E^*_2) \leq \Gamma(E) \leq \frac{E}{\Delta E}\Gamma(E^*_1) \Gamma(E^*_2) ~.
        \end{equation*}
        Now, we can take the logarithm of this inequality, since it is a monotonic function. After another series of manipulations, we obtain
        \begin{equation*}
            \log ( \Gamma_1(E^*_1) \Gamma(E^*_2) ) \leq \log \Gamma(E) \leq \log ( \frac{E}{\Delta E}\Gamma(E^*_1) \Gamma(E^*_2) ) ~,
        \end{equation*}
        \begin{equation*}
            k_B \log \Big ( \Gamma_1(E^*_1) \Gamma(E^*_2) \Big ) \leq k_B \log \Gamma(E) \leq k_B \log \Big ( \frac{E}{\Delta E}\Gamma(E^*_1) \Gamma(E^*_2) \Big ) ~,
        \end{equation*}
        \begin{equation*}
        \begin{aligned}
            k_B \log \Gamma_1(E^*_1) + k_B \log \Gamma(E^*_2) & \leq k_B \log \Gamma(E) \\ & \leq k_B \log \frac{E}{\Delta E} + k_B \log \Gamma(E^*_1) + k_B \log \Gamma(E^*_2) ~,
        \end{aligned}
        \end{equation*}
        \begin{equation*}
        \begin{aligned}
            \frac{k_B \log \Gamma_1(E^*_1) + k_B \log \Gamma(E^*_2)}{N} & \leq \frac{k_B \log \Gamma(E)}{N} \\ & \leq \frac{k_B \log \Gamma(E^*_1) + k_B \log \Gamma(E^*_2)}{N} + \frac{k_B \log \frac{E}{\Delta E}}{N} ~.
        \end{aligned}
        \end{equation*}
        Finally, we take the thermodynamic limit, noticing that the last term vanishes, since $E$ goes like $N$ and $\lim_{N \rightarrow \infty} \frac{1}{N} \log N = 0$
        \begin{equation*}
        \begin{aligned}
            \underbrace{\lim_{TD} \frac{k_B \log \Gamma_1(E^*_1)}{N}}_{s_{mc}^{(1)}} + \underbrace{\lim_{TD} \frac{k_B \log \Gamma(E^*_2)}{N}}_{s_{mc}^{(2)}} & \leq \underbrace{\lim_{TD} \frac{k_B \log \Gamma(E)}{N}}_{s_{mc}} \\ & \leq \underbrace{\lim_{TD} \frac{k_B \log \Gamma_1(E^*_1)}{N}}_{s_{mc}^{(1)}} + \underbrace{\lim_{TD} \frac{k_B \log \Gamma(E^*_2)}{N}}_{s_{mc}^{(2)}} ~,
        \end{aligned}
        \end{equation*}
        hence, we find
        \begin{equation}\label{proof3}
            s_{mc}(E) = s_{mc}^{(1)} (E_1^*) + s_{mc}^{(2)} (E_2^*) ~.
        \end{equation}
    \end{proof}

    From~\eqref{proof3}, we can also deduce $2$ properties that entropy fulfills: for isolated system, spontaneous processes leads to an increase in entropy, which is verified because spontaneous processes leads also to an increase in phase space volume; at equilibrium, entropy is maximum, which is verified by the asterisks. Furthermore, in the thermodynamic limit, microcanonical entropy coincides with thermodynamic entropy
    \begin{equation*}
        s_{mc} = s_{td} ~.
    \end{equation*}
    \begin{proof}
        Since entropy is maximum at equilibrium, also $\Gamma_1(E_1) \Gamma_2(E_2)$ is so and it has null variation
        \begin{equation*}
        \begin{aligned}
            0 & = \delta (\Gamma_1(E^*_1) \Gamma_2(E^*_2 = E - E^*_1)) \\ & = \delta \Gamma_1(E^*_1) \Gamma_2 (E^*_2) + \Gamma_1(E^*_1) \delta \Gamma_2 (E^*_2) \\ & = \pdv{\Gamma_1}{E_1} \Big\vert_{E^*_1} \delta E_1 \Gamma_2 (E^*_2) + \Gamma_1(E^*_1) \pdv{\Gamma_2}{E_2} \Big\vert_{E^*_2} \delta E_2 ~,
        \end{aligned}
        \end{equation*}
        where we have used Leibniz rule and we have exploited the dependence on energy.

        Since $E = \text{const}$, it has a null variation $0 = \delta E = \delta E_1 + \delta E_2$ and 
        \begin{equation}\label{proof4}
            \delta E_2 = - \delta E_1 ~.
        \end{equation} 
        Hence, with a series of manipulations, we obtain
        \begin{equation*}
            0 = \pdv{\Gamma_1}{E_1} \Big\vert_{E^*_1} \delta E_1 \Gamma_2 (E^*_2) - \Gamma_1(E^*_1) \pdv{\Gamma_2}{E_2} \Big\vert_{E^*_2} \delta E_1  ~,
        \end{equation*}
        \begin{equation*}
            0 = \pdv{\Gamma_1}{E_1} \Big\vert_{E^*_1} \Gamma_2 (E^*_2) - \Gamma_1(E^*_1) \pdv{\Gamma_2}{E_2} \Big\vert_{E^*_2}  ~,
        \end{equation*}
        \begin{equation*}
            \pdv{\Gamma_1}{E_1} \Big\vert_{E^*_1} \Gamma_2 (E^*_2) = \Gamma_1(E^*_1) \pdv{\Gamma_2}{E_2} \Big\vert_{E^*_2} ~,
        \end{equation*}
        \begin{equation*}
            \frac{1}{\Gamma_1 (E^*_1)} \pdv{\Gamma_1}{E_1} \Big\vert_{E^*_1} = \frac{1}{\Gamma_2 (E^*_2)} \pdv{\Gamma_2}{E_2} \Big\vert_{E^*_2} ~,
        \end{equation*}
        \begin{equation*}
            \pdv{\log \Gamma_1}{E_1} \Big\vert_{E^*_1} = \pdv{\log \Gamma_2}{E_2} \Big\vert_{E^*_2} ~.
        \end{equation*}
        This is a relation valid for all systems at equilibrium, which is the $0$th law of thermodynamic. Finally, we use the first thermodynamic relation in~\eqref{td:es:s} to have
        \begin{equation*}
            S_{mc} (E) = S_{td} (E) \times const
        \end{equation*}
        where the constant can be chosen the Boltzmann constant, in order to have $k_B$ in the same unit of energy over temperature.
    \end{proof}

    The universal Boltzmann's formula is 
    \begin{equation}\label{mc:unibol}
        s_{mc} = s_{td} = k_B \log \omega(E) = - k_B \avp{\log \rho_{mc}}{mc} ~.
    \end{equation}
    \begin{proof}
        In fact, using~\eqref{mc:av}, the properties of logarithms and the fact that $\omega(E)$ is constant, we obtain
        \begin{equation*}
        \begin{aligned}
            \avp{\log \rho_{mc}}{mc} & = \int d\Gamma ~ \rho_{mc} \log \rho_{mc} = \int d\Gamma ~ \frac{1}{\omega(E)} \delta (H - E) \log \Big ( \frac{1}{\omega(E)} \delta (H - E) \Big) \\ & = \int dS_E ~ \frac{1}{\omega(E)} \log \frac{1}{\omega(E)} = - \frac{1}{\omega(E)} \log \omega(E) \underbrace{\int dS_E }_{\omega(E)} = - \log \omega (E) ~.
        \end{aligned}
        \end{equation*}
    \end{proof}

\chapter{Canonical ensemble}

    In this chapter, we will study the canonical ensemble: probability density distribution canonical partition function and we will recover thermodynamics by means of the Helmholtz free energy.

\section{Canonical probability density distribution}

    Consider a physical system that can exchange energy with the environment but matter, so that the boundary conditions $T$, $N$ and $V$ are fixed. Notice that temperature has substituted energy. Physically, it can be thought as the system is immersed in a bigger reservoir with $N_1 \ll N_2$ and $V_1 \ll V_2$ but at equilibrium with the same temperature $T_1 = T_2 = T$. See Figure~\ref{fig:c}. 

    \begin{figure}[h!]
        \centering
        \begin{tikzpicture}
            \draw[smooth cycle, tension=0.4] plot coordinates{(2,2) (-2.5,0) (3,-2) (6,1)} node at (3,2.3) {Universe};

            \draw[smooth cycle, tension=0.4] 
                plot coordinates { (0.75, 0) (1.25, 1.5) (3.5, 1.5) (4, 0)} 
                node [label={[label distance=-0.3cm, xshift=-1cm, yshift=0.4cm]:System}] {}
                node [label={[label distance=-0.3cm, xshift=-2cm, yshift=-1.25cm]:Environment}] {};

            \node[] at (0.9, 1) {$E_2 \leftrightarrow E_1$};
        \end{tikzpicture}
        \label{fig:c}
        \caption{Pictorial representation of canonical ensemble.}
    \end{figure}

    Globally, energy is conserved and the universe, composed by the union of the system and the environment, can be considered isolated and, therefore, a microcanonical ensemble. This kind of set-up is called canonical ensemble and it has associated a probability density distribution 
    \begin{equation}\label{c:pdd}
        \rho_c (q^i, p_i) = \frac{1}{Z_N} \exp (-\beta H(q^i, p_i)) ~,
    \end{equation}
    where $\beta$ is 
    \begin{equation*}
        \beta = \frac{1}{k_B T}
    \end{equation*}
    and $Z_N$ is the partition function 
    \begin{equation}\label{c:z}
        Z_N[V, T] = \int_{\mathcal M^N} d\Omega ~\exp (-\beta H(q^i, p_i)) ~,
    \end{equation}
    which depends on the temperature through $\beta$ and volume and number of particles due to the integration domain $\mathcal M^N = V \otimes \mathbb R^d$. Notice that the probability is a function of the Hamiltonian, like Liouville's theorem states~\eqref{cm:rh}.
    \begin{proof}
        Consider the universe as a microcanonical ensemble, with associated probability density distribution 
        \begin{equation*}
            \rho_{mc} (q_i^{(1)}, p_i^{(1)}, q_i^{(2)}, p_i^{(2)}) = \frac{1}{\omega(E)} \delta (H (q_i^{(1)}, p_i^{(1)}, q_i^{(2)}, p_i^{(2)}) - E) ~,
        \end{equation*}
        where $1$ is the system, $2$ is the environment and the total Hamiltonian is 
        \begin{equation*}
            H (q_i^{(1)}, p_i^{(1)}, q_i^{(2)}, p_i^{(2)}) = H_1 (q_i^{(1)}, p_i^{(1)}) + H_2 (q_i^{(2)}, p_i^{(2)}) ~.
        \end{equation*}
        To find the probability density distribution for only the degrees of freedom associated to the system, we have to integrate over the degrees of freedom of the environment to wash them out
        \begin{equation*}
            \rho^{(1)} = \int d\Omega_2 ~ \rho_{mc} = \int d\Omega_2 ~ \frac{1}{\omega(E)} \delta(H - E) = \frac{1}{\omega(E)} \underbrace{\int dS_{E_2}}_{\omega(E_2)} = \frac{1}{\omega(E)} \omega(E_2 = E - E_1) ~.
        \end{equation*}
        Notice that this distribution is normalised, since 
        \begin{equation*}
        \begin{aligned}
            \int d\Omega_1 ~ \rho^{(1)} & = \frac{1}{\omega(E)} \int d\Omega_1 \omega(E_2 = E - E_1) \\ & = \frac{1}{\omega(E)} \int_0^E dE_1 ~ \underbrace{\int_{S_{E_1}} dS_{H_1}}_{\omega(E-1)} \omega(E_2 = E - E_1) \\ & = \frac{1}{\omega(E)} \underbrace{\int_0^E dE_1 \omega(E_1) \omega(E_2 = E - E_1) }_{\omega(E)} = 1 ~,
        \end{aligned}
        \end{equation*}
        where the last expression follows from $E = E_1 + E_2$, $d\Omega = d\Omega_1 d\Omega_2$ and 
        \begin{equation*}
        \begin{aligned}
            \omega(E) & = \int d\Omega ~ \delta(H - E) = \int d\Omega_1 d\Omega_2 ~ \delta(E - E_1 - E_2) \\ & = \int dE_1 \int dE_2 \underbrace{\int_{S_{E_1}} dS_{H_1}}_{\omega(E_1)} \underbrace{\int_{S_{E_2}} dS_{H_2}}_{\omega(E_2)} ~ \delta(E - E_1 - E_2) \\ & = \int dE_1 \int dE_2 ~\omega(E_1) \omega(E_2) \delta(E - E_1 - E_2)\\ &  = \int dE_1 ~ \omega(E_1) \omega(E_2 = E - E_1) ~.
        \end{aligned}
        \end{equation*}
        In order to compute $\omega(E_2)$, we introduce the entropy 
        \begin{equation*}
            S_2 (E_2) = k_B \ln \omega_2 (E_2) ~.
        \end{equation*}
        For the considerations made in the microcanonical, at equilibrium, entropy is at maximum but if we make small variation $\delta E_1$ to $E_1$, in order to preserve equilibrium, the entropy get Taylor expanded around $E$ as
        \begin{equation*}
            k_B \ln \omega_2 (E_2) = S_2 (E_2) \simeq S_2(E) + \underbrace{\delta E_2}_{- \delta E_1 \simeq - E_1} \underbrace{\pdv{S_2}{E_2} \Big \vert_{E = E_2}}_{\frac{1}{T}} = S_2 (E) - E_1 \frac{1}{T} ~,
        \end{equation*} 
        where we have used~\eqref{proof4} and the first thermodynamic relation in~\eqref{td:es:s}. Hence, the density of states of the system is
        \begin{equation*}
            \omega_2 (E_2) = \exp (\frac{S_2 (E)}{k_B} - E_1 \frac{1}{k_B T}) = \exp (\frac{S_2 (E)}{k_B}) \exp (- \frac{E_1}{k_B T}) ~.
        \end{equation*}
        Finally, putting together and dropping indices, we obtain
        \begin{equation}
            \rho_c = \frac{\omega_2 (E_2)}{\omega(E)} = \underbrace{\frac{1}{\omega(E)} \exp (\frac{S_{mc}(E)}{k_B})}_C \exp (- \frac{E_1}{k_B T}) = C \exp (- \frac{E_1}{k_B T}) ~,
        \end{equation}
        where $C$ is a normalisation constant, which can be evaluated by~\eqref{cm:norm}
        \begin{equation*}
            1 = \int_{\mathcal M^N} d\Omega \rho = \int_{\mathcal M^N} d\Omega ~ C \exp (- \frac{E_1}{k_B T}) = C \int_{\mathcal M^N} d\Omega ~ \exp (- \frac{E_1}{k_B T}) = C Z_N ~. 
        \end{equation*}
    \end{proof}

    The partition function can also be written as 
    \begin{equation*}
        Z_N[T, V] = \int_{0}^{\infty} dE~ \omega(E) \exp (-\beta E) ~.
    \end{equation*}
    \begin{proof}
        In fact, using~\eqref{cm:fol}, we have
        \begin{equation*}
            Z_N = \int_{\mathcal M^N} d\Omega \exp (- \beta H) = \integ{0}{\infty}{E} \underbrace{\int dS_E}_{\omega(E)} ~ \exp (-\beta H) = \integ{0}{\infty}{E} \omega(E) \exp (-\beta E) ~.
        \end{equation*}
    \end{proof}

    Now, it is important to distinguish $2$ different type of particles: indistinguishable and distinguishable particles. Particles are called indistinguishable if they cannot be distinguished When compared with others and they are called distinguishable otherwise. In principle, in classical physics, there are no indistinguishable particles, because even if they all have the same mass, charge, spin, etc, we could always follow their trajectory. Therefore, indistinguishable particles arise only in quantum mechanics, when the uncertainty principle is introduce and the trajectory distinction cannot be made anymore. To take into account this property, we introduce a new term $\zeta_N$, defined as 
    \begin{equation}\label{cm:dist}
        \zeta_N = \begin{cases}
            1 & \textnormal{distinguishable} \\
            N! & \textnormal{indistinguishable}
        \end{cases} ~.
    \end{equation}
    Hence, the partition function becomes
    \begin{equation*}
        Z_N = \int \frac{\prod_{i=1}^N d^d q^i d^d p^i}{h^{dN} \zeta_N} ~ \exp (- \beta H) = \int \frac{d\Omega_N}{\zeta_N} ~ \exp (- \beta H) ~,
    \end{equation*}
    where we have redefined the phase space measure~\eqref{cm:measure2}. 
    For distinguishable particles, e.g.~particles of $n$ different species with energy $H = \sum_{i = 1}^{n} H_i$ and number of particles $N = \sum_{i=1}^n N_i$, the total partition function is the multiplication of single species partition functions
    \begin{equation*}
        Z_N = \prod_{i = 1}^{n} Z_{N_i} ~.
    \end{equation*}
    \begin{proof}
        In fact, 
        \begin{equation*}
        \begin{aligned}
            Z_N & = \int_{\mathcal M = \mathcal M^{(1)}\times \ldots \times \mathcal M^{(n)}} d\Omega ~\exp (-\beta H) \\ & = \int_{\mathcal M = \mathcal M^{(1)}\times \ldots \times \mathcal M^{(n)}} \prod_{i = 1}^{n} d\Omega_i \ldots d\Omega_n ~ \exp(-\beta \sum_{i = 1}^{N} H_i) \\ & = \prod_{i = 1}^{n} \int_{\mathcal M^{(i)}} d\Omega_i ~ \exp(- \beta H_i) = \prod_{i = 1}^{n} Z_{N_i} ~.
        \end{aligned}
        \end{equation*}
    \end{proof}
    \begin{example}
        If we have only $2$ types of particles, the total partition function is 
        \begin{equation*}
            Z_N = Z_{N_1} Z_{N_2} ~.
        \end{equation*}
    \end{example}
    Consequently, if all particles are of the same species, and therefore identical, the total partition function is 
    \begin{equation}\label{c:zdist}
        Z_N = (Z_1)^N ~,
    \end{equation}
    where $Z_1$ is the single-particle partition function. 
    If we take into consideration also indistinguishability, for different species, the canonical partition function becomes 
    \begin{equation*}
        Z_N = \frac{1}{\zeta_N} \prod_{i=1}^{n} Z_{N_i} ~,
    \end{equation*}
    whereas, for same species particles, it becomes
    \begin{equation*}
        Z_N = \frac{(Z_1)^N}{\zeta_N} ~.
    \end{equation*}
    
    Let $f(q^i, p_i)$ be an observable, then its canonical average is 
    \begin{equation}\label{c:av}
        \av{f(q^i, p_i)}_{c} = \int_{\mathcal M} d\Omega ~ \rho_{c} f = \int_{\mathcal M} d\Omega ~ \frac{\exp (-\beta H)}{Z_N} f ~.
    \end{equation}

\section{Helmholtz free energy as canonical potential}

    A local chart with coordinates $(T, V, N)$ is suitable for Helmholtz free energy~\eqref{td:coord:f}. Hence, we need to find an expression for this thermodynamic potential. The first guess is to define the canonical Helmholtz free energy as 
    \begin{equation}\label{c:zf}
        Z_N [V, T] = \exp(-\beta F[T, V, N]) ~,
    \end{equation}
    or, equivalently,
    \begin{equation}\label{c:f}
        F[T, V, N] = -\frac{1}{\beta} \ln Z_N ~.
    \end{equation}
    We define the canonical internal energy as
    \begin{equation}\label{c:e}
        E = \avp{H}{c} = \int d\Omega \frac{\exp(-\beta (H))}{Z_N} H ~.
    \end{equation}
    \begin{proof}
        By normalisation condition, we have
        \begin{equation*}
            1 = \int d\Omega \frac{\exp(-\beta H)}{Z_N} = \int d\Omega \frac{\exp(-\beta H)}{\exp(-\beta F)} = \int d\Omega \exp (- \beta (H - F)) ~.
        \end{equation*}
        Now, since $F$ depends on the temperature $F(T)$ or $F(\beta)$, it is possible to derive it with respect to $\beta$, where the left handed side is null because it is the derivative of a constant $1$, and we obtain
        \begin{equation*}
        \begin{aligned}
            0 & = \pdv{}{\beta} \Big ( \int d\Omega \exp (- \beta (H - F)) \Big) \\ & = \int d\Omega \exp (-\beta (H - F)) \Big (-(H - F) + \beta \pdv{F}{\beta}) \\ & = - \underbrace{\int d\Omega \frac{\exp(-\beta H)}{Z_N} H}_{E} + F \underbrace{\int d\Omega \frac{\exp(-\beta H)}{Z_N}}_{1} + \beta \pdv{F}{\beta} \underbrace{\int d\Omega \frac{\exp(-\beta H)}{Z_N}}_{1} \\ & = - E + F + \beta \pdv{F}{\beta} ~.
        \end{aligned}
        \end{equation*}
        Hence, using the first of~\eqref{td:es:f}, we find
        \begin{equation*}
            F = E - \beta \pdv{F}{\beta} = E + T \pdv{F}{T} = E - TS ~,
        \end{equation*}
        where in the first step we have used 
        \begin{equation*}
            \beta \pdv{}{\beta} = \frac{1}{k_B} T \pdv{T}{\beta} \pdv{}{T} = \frac{1}{k_B T} (\pdv{}{T} \frac{1}{k_B T})^{-1} \pdv{}{T} = - \frac{1}{T} T^2 \pdv{}{T} = - T \pdv{}{T} ~.
        \end{equation*}
        This result shows explicitly that $F$ is indeed the Helmotz free energy~\eqref{td:def:f}.
    \end{proof}
    Notice that canonical entropy can be written as 
    \begin{equation} \label{c:s}
        S_c = \frac{E - F}{T} ~.
    \end{equation}
    An useful expression for the internal energy is
    \begin{equation}\label{c:e2}
        E = - \pdv{}{\beta} \ln Z_N ~,
    \end{equation}
    \begin{proof}
        Using~\eqref{c:e},
        \begin{equation*}
        \begin{aligned}
            E & = \int d\Omega \frac{\exp(-\beta H)}{Z_N} H = - \frac{1}{Z_N} \pdv{}{\beta} \int d\Omega \exp (-\beta H) \\ & = - \frac{1}{Z_N} \pdv{Z_N}{\beta} = - \pdv{}{\beta} \ln Z_N ~,
        \end{aligned}
        \end{equation*}
        where we have used the trick to extract the derivative with respect to $\beta$.
    \end{proof}
    The universal Boltzmann's formula~\eqref{mc:unibol} is valid also in this ensemble
    \begin{equation*}
        S_c = -k_B \avp{\ln \rho_c}{c}  ~.
    \end{equation*}
    \begin{proof}
        In fact, using~\eqref{c:av},~\eqref{c:e},~\eqref{c:s} and~\eqref{c:f}
        \begin{equation*}
        \begin{aligned}
            -k_B \avp{\ln \rho_c}{c} & = -k_B \int d\Omega \rho_c \ln \rho_c  = -k_B \int d\Omega \rho_c \ln \frac{\exp(-\beta H)}{Z_N} \\ & = -k_B \int d\Omega \rho_c \ln \exp(-\beta H) - k_B \int d\Omega \rho_c \ln Z_N \\ & = k_B  \beta \underbrace{\int d\Omega \rho_c H }_E - k_B \underbrace{\ln Z_N}_{\beta F} \underbrace{\int d\Omega \rho_c}_{1} = \frac{E - F}{T} = S_c ~.
        \end{aligned}
        \end{equation*}
    \end{proof}

\section{Equipartition theorem}

    An important theorem, that can be proved in the canonical ensemble, is the famous equipartition theorem. To be honest, it was also possible to derive it in the microcanonical ensemble, but in the canonical is much way easier. In this section, we will prove a generalised version of it.

    \begin{theorem}[Generalised equipartition theorem]
        Let $\xi \in [a,b]$ be a sympletic coordinate. Let $\xi_j$ with $j \neq 1$ be all the other ones. Suppose also 
        \begin{equation}\label{equi:cond}
            \int \prod_{j \neq 1} d \xi_j [\xi_1 \exp(-\beta H)]_a^b = 0 ~.
        \end{equation}
        Then 
        \begin{equation}\label{equi:thm}
            \avp{\xi_1 \pdv{H}{\xi_1}}{c} = k_B T ~.
        \end{equation}
    \end{theorem}

    \begin{proof}
        By normalisation condition, we have
        \begin{equation*}
            1 = \int d\Omega ~ \frac{\exp(-\beta H)}{Z_N} = \frac{1}{Z_N} \int d\xi_1 \prod_{j \neq 1} d \xi_j ~ \exp(-\beta H) ~,
        \end{equation*}
        where we have omitted dimensional and indistinguishability factors for convenience.
        Now, we use a differential relation, which is equivalent to an intergation by parts,
        \begin{equation*}
            d(\xi_1 \exp(-\beta H)) = d\xi_1 \exp(-\beta H) + \xi \exp(-\beta H) (-\beta) \pdv{H}{\xi_1} d\xi_1  ~,
        \end{equation*}
        we invert it 
        \begin{equation*}
            d\xi_1 \exp(-\beta H) = d(\xi_1 \exp(-\beta H)) + \beta \xi \exp(-\beta H) \pdv{H}{\xi_1} d\xi_1
        \end{equation*}
        and we insert it to find
        \begin{equation*}
        \begin{aligned}        
            1 & = \frac{1}{Z_N} \int (\prod_{j \neq 1} d \xi_j) (d(\xi_1 \exp(-\beta H)) + \beta \xi \exp(-\beta H) \pdv{H}{\xi_1} d\xi_1) \\ & = \frac{1}{Z_N} \underbrace{\prod_{j \neq 1} d \xi_j [\xi_1 \exp(-\beta H)]_a^b}_{0~\text{by hp}} + \frac{\beta}{Z_N} \int \underbrace{\prod_{j \neq 1} d \xi_j d\xi_1}_{d\Omega} ~ \xi_1 \pdv{H}{\xi_1} \exp (-\beta H) \\ & = \beta \int d\Omega ~ \xi_1 \pdv{H}{\xi_1} \frac{\exp (- \beta H)}{Z_N} = \beta \avp{\xi_1 \pdv{H}{\xi_1}}{c}
        \end{aligned}
        \end{equation*}
        where we have used the hypothesis~\eqref{equi:cond}.
        Hence, we obtain
        \begin{equation*}
            \avp{\xi_1 \pdv{H}{\xi_1}}{c} = \frac{1}{\beta} = k_B T ~.
        \end{equation*}
    \end{proof}

    One may wonder which are the physical systems that satisfy the strange condition~\eqref{equi:cond}. Examples are systems with Hamiltonian composed by a quadratic momentum (with $a = -\infty$ and $b = \infty$) or a confining potential which go to infinity on the extremes $a$ and $b$. For the first case, we have $p \in (-\infty, \infty)$, $H = H(p^2)$ and
    \begin{equation*}
        p \exp(-\beta p^2) \Big \vert_{-\infty}^\infty = 0 ~,
    \end{equation*}
    whereas for the second case, we have $q \in [a, b]$, $V = V(q)$ such that $V(a) = V(b) = \infty$ and 
    \begin{equation*}
        q \exp(-\beta V(q)) \Big \vert_a^b = 0 ~.
    \end{equation*}

    \begin{corollary}[Equipartition theorem]
        If $\xi_1$ appears quadratically in $H$, then its contribution to $E$ is $\frac{1}{2} k_B T$.
    \end{corollary}

    \begin{proof}
        Consider $H = A \xi_1^2 + B \xi_j^2$ with $j \neq 1$, then by the means of the previous theorem, we obtain 
        \begin{equation*}
            \avp{\xi_1 \pdv{H}{\xi_1}}{c} = \avp{\xi 2 A \xi_1}{c} = k_B T ~,
        \end{equation*}
        hence
        \begin{equation*}
            \avp{A \xi_1^2}{c} = \frac{1}{2} k_B T ~.
        \end{equation*}
    \end{proof}

    \begin{example}
        Consider a perfect gas, composed by $N$ particles in $3$ dimension, with Hamiltonian $H = \sum_{i=1}^{3N} \frac{p_i^2}{2m}$. Applying the equipartition theorem, since there are $3N$ quadratic momenta terms, the energy is $E = \frac{3}{2} N k_B T$.
    \end{example}
    \begin{example}
        Consider a system composed by $N$ uncoupled harmonic oscillators of masses $m_i$ and frequencies $\omega_i$ in $3$ dimension with Hamiltonian $H = \sum_{i=1}^{3N} \frac{p_i^2}{2m_i} + \frac{m_i \omega_i q_i^2}{2}$. Applying the equipartition theorem, since there are $3N$ quadratic momenta terms and $3N$ quadratic coordinates terms, the energy is $E = 3 N k_B T$, known as the Dulong-Petit laws for the specific heat at constant volume $C_V = 3 N k_B$.
    \end{example}

\chapter{Grand canonical ensemble}

    In this chapter, we will study the grand canonical ensemble: probability density distribution, grand canonical partition function and we will recover thermodynamics by means of the grand potential.

\section{Grand canonical probability density distribution}

    Consider a physical system that can exchange both energy and matter with the environment, so that boundary conditions $T$, $V$ and $\mu$ are fixed. Notice that chemical potential has substituted number of particles. Physically, it can be thought as the system is immersed in a bigger reservoir with $N_1 \ll N_2$ and $V_1 \ll V_2$ but at equilibrium with the same temperature $T_1 = T_2 = T$. See Figure~\ref{fig:gc}. 

    \begin{figure}[h!]
        \centering
        \begin{tikzpicture}
            \draw[smooth cycle, tension=0.4] plot coordinates{(2,2) (-2.5,0) (3,-2) (6,1)} node at (3,2.3) {Universe};

            \draw[smooth cycle, tension=0.4] 
                plot coordinates { (0.75, 0) (1.25, 1.5) (3.5, 1.5) (4, 0)} 
                node [label={[label distance=-0.3cm, xshift=-1cm, yshift=0.4cm]:System}] {}
                node [label={[label distance=-0.3cm, xshift=-2cm, yshift=-1.25cm]:Environment}] {};

            \node[] at (0.9, 1) {$E_2 \leftrightarrow E_1$};
            \node[] at (0.75, 0.5) {$N_2 \leftrightarrow N_1$};
        \end{tikzpicture}
        \label{fig:gc}
        \caption{Pictorial representation of grand canonical ensemble.}
    \end{figure}
    
    Globally, the number of particles is conserved and the universe, composed by the union of the system and the environment, can be considered a canonical ensemble. In fact we suppose that we have had already gone from microcanonical to canonical. This kind of set-up is called grand canonical ensemble and it has associated a probability density distribution 
    \begin{equation}\label{gc:pdd}
        \rho_c (q^i, p_i) = \frac{z^N}{\mathcal Z} \exp (- \beta H(q^i, p_i, N)) ~,
    \end{equation}
    where $z$ is the fugacity
    \begin{equation*}
        z = \exp(\beta \mu)
    \end{equation*}
    and $\mathcal Z$ is the grand canonical partition function 
    \begin{equation}\label{gc:z}
        \mathcal Z (z, V, T) = \sum_{N = 0}^{\infty} z^N Z_N = \sum_{N = 0}^{\infty} z^N \int_{\mathcal M^N} d\Omega \exp(-\beta H) ~,
    \end{equation}
    which depends on fugacity $z$ explicitly, the temperature through $\beta$ and volume due to the integration domain $\mathcal M^N = V \otimes \mathbb R^d$. It does not depend on number of particles because of the sum. Notice that the probability is a function of the Hamiltonian, like Liouville's theorem states~\eqref{cm:rh}.
    \begin{proof}
        Consider the universe as a canonical ensemble, with associated probability density distribution is 
        \begin{equation*}
            \rho_c (q_i^{(1)}, p_i^{(1)}, q_i^{(2)}, p_i^{(2)}) = \frac{\exp (-\beta H (q_i^{(1)}, p_i^{(1)}, q_i^{(2)}, p_i^{(2)}))}{Z_N[T, V]} ~,
        \end{equation*}
        where $1$ is the system, $2$ is the environment and the total Hamiltonian is 
        \begin{equation*}
            H (q_i^{(1)}, p_i^{(1)}, q_i^{(2)}, p_i^{(2)}) = H_1 (q_i^{(1)}, p_i^{(1)}) + H_2 (q_i^{(2)}, p_i^{(2)}) ~.
        \end{equation*}
        Following the same procedure used in the canonical ensemble, we integrate over the degrees of freedom of the environment
        \begin{equation*}
        \begin{aligned}
            \rho^{(1)} & = \int d\Omega_2 ~ \rho_c = \int d\Omega_2 \frac{\exp(-\beta (H_1 + H_2))}{Z_N} \\ & = \exp(-\beta H_1) \frac{1}{Z_N} \underbrace{\int d\Omega_2 \exp(-\beta H_2)}_{Z_{N_2}} = \exp(-\beta H_1) \frac{Z_{N_2} [T, V_2]}{Z_N [T, V]} ~.
        \end{aligned}
        \end{equation*}
        Now, we have to find the normalisation factor for the distribution. We start from the normalisation condition
        \begin{equation*}
        \begin{aligned}
            1 = \sum_{N_1 = 0}^{N} \int_{\mathcal M^{N_1}} d\Omega_1 ~ \rho_{gc} = \sum_{N_1 = 0}^{N} \int_{\mathcal M^{N_1}} d\Omega_1 ~\exp(-\beta H_1) \frac{Z_{N_2} [T, V_2]}{Z_N [T, V]}  ~.
        \end{aligned}
        \end{equation*}
        Now, since the normalisation changes with the number of particles, we explicitate this dependence in the phase space measure by writing all the factor $1 / N!$ that derives from\eqref{cm:dist}. Hence, recalling that the canonical partition function is an integral, we obtain
        \begin{equation*}
        \begin{aligned}
            & \sum_{N_1 = 0}^{N} \frac{N!}{N_1! N_2!} \int_{\mathcal M^{N_1}} d\Omega_1 ~\exp(-\beta H_1) \frac{Z_{N_2} [T, V_2]}{Z_N [T, V]} \\ & = \sum_{N_1 = 0}^{N} \frac{N!}{N_1! N_2} \frac{\int_{\mathcal M^{N_1}} d\Omega_1 ~ \exp(-\beta H_1) \int_{\mathcal M^{N_2}} d\Omega_2 ~ \exp(-\beta H_2)}{\int_{\mathcal M^N} d\Omega ~ \exp(-\beta H)} \\ & = \sum_{N_1 = 0}^{N} \frac{N!}{N_1! N_2} \frac{(V_1)^{N_1} (V_2)^{N_2}}{V^N} \frac{\frac{\int_{\mathcal M^{N_1}} d\Omega_1 ~ \exp(-\beta H_1)}{(V_1)^{N_1}} \frac{\int_{\mathcal M^{N_2}} d\Omega_2 ~ \exp(-\beta H_2)}{(V_2)^{N_2}}}{\frac{\int_{\mathcal M^N} d\Omega ~ \exp(-\beta H)}{V^N}}  ~.
        \end{aligned}
        \end{equation*}
        Going into the thermodynamic limit, we at once recognise that the last term is equal to $1$ 
        \begin{equation*}
            \lim_{td} \frac{\frac{\int_{\mathcal M^{N_1}} d\Omega_1 ~ \exp(-\beta H_1)}{(V_1)^{N_1}} \frac{\int_{\mathcal M^{N_2}} d\Omega_2 ~ \exp(-\beta H_2)}{(V_2)^{N_2}}}{\frac{\int_{\mathcal M^N} d\Omega ~ \exp(-\beta H)}{V^N}} = 1 ~,
        \end{equation*}
        On the other hand, using $N = N_1 + N_2$ and 
        \begin{equation*}
            (a + b)^n = \sum_{i=1}^{n} \binom{n}{i} a^i b^{n-i}  ~,
        \end{equation*}
        the remaining term can be rewritten as 
        \begin{equation*}
            \sum_{N_1 = 0}^{N} \frac{N!}{N_1! N_2!} \frac{(V_1)^{N_1} (V_2)^{N_2}}{V^N} = \sum_{N_1 = 0}^{N} \binom{N}{N_1} \Big ( \frac{V}{V} \Big)^{N_1}  \Big ( \frac{V_2}{V} \Big)^{N - N_1} = \Big ( \frac{V_1 + V_2}{V} \Big)^N  ~,
        \end{equation*}
        which in the thermodynamic limit goes as well to $1$ 
        \begin{equation*}
            \lim_{td} \Big ( \frac{V_1 + V_2}{V} \Big)^N  = 1 ~.
        \end{equation*} 
        Hence, we find
        \begin{equation*}
            \lim_{td} \sum_{N_1 = 0}^{N} \frac{N!}{N_1! N_2!} \int_{\mathcal M^{N_1}} d\Omega_1 ~\exp(-\beta H_1) \frac{Z_{N_2} [T, V_2]}{Z_N [T, V]} = 1 ~.
        \end{equation*}
        Now, using~\eqref{c:zf} and~\eqref{td:es:f}, we Taylor expand at the first order in $N_1 \ll N = N_2$ and $V_1 \ll V_2 = V$ and we obtain 
        \begin{equation*}
        \begin{aligned}
            \frac{Z_{N_2}[T, V]}{Z_N[T, V]} & = \frac{\exp(-\beta F(T, N_2, V_2))}{\exp(-\beta F(T, N, V))} \\ & = \exp(-\beta (F(T, N-N_1, V-V_1) - F(T, N, V))) \\ & \simeq \exp(-\beta(\underbrace{\pdv{F}{N} \Big \vert_{T, V}}_{\mu} (-N_1) + \underbrace{\pdv{F}{V} \Big \vert_{T, N}}_{-p} (-V_1))) \\ & = \exp(-\beta(-\mu N_1 + p V_1)) ~.
        \end{aligned}
        \end{equation*}
        Hence, we drop indices and we find 
        \begin{equation*}
        \begin{aligned}
            \rho_{gc} & = \frac{\exp(-\beta H)}{N!} \exp (-\beta (-\mu N + p V)) = \frac{\exp(-\beta H)}{N!} \underbrace{\exp ( \beta \mu)^N}_{z^N} \exp(-\beta p V) \\ & = \frac{z^N \exp(-\beta H)}{N!} \exp(-\beta p V) ~,
        \end{aligned}
        \end{equation*}
        where we have introduced the fugacity $z = \exp(\beta \mu)$.
        Finally, using the normalisation condition, we obtain
        \begin{equation*}
        \begin{aligned}
            1 & = \sum_{N=0}^{\infty} \int_{\mathcal M^N} d\Omega \rho_{gc} = \sum_{N=0}^{\infty} \int_{\mathcal M^N} d\Omega \frac{z^N \exp(-\beta H)}{N!} \exp(-\beta p V) \\ & = \exp(-\beta p V) \sum_{N=0}^{\infty} z^N \underbrace{\int_{\mathcal M^N} {d\Omega}{N!} \exp(- \beta H)}_{Z_N} = \exp(-\beta p V) \underbrace{\sum_{N=0}^{\infty} z^N Z_N}_{\mathcal Z}= \exp(-\beta p V) \mathcal Z ~.
        \end{aligned}
        \end{equation*}
        Therefore
        \begin{equation}\label{proof5}
            \mathcal Z = \sum_{N=0}^{\infty} z^N Z_N = \exp(\beta p V)
        \end{equation}
        and 
        \begin{equation*}
            \rho_{gc} (q_i, p_i) = \frac{\exp(-\beta (H(q_i, p_i) - \mu N))}{\mathcal Z} = \frac{\exp(-\beta \mathfrak H(q_i, p_i) )}{\mathcal Z} ~,
        \end{equation*}
        where $\mathfrak H = H - \mu N$ is the grand canonical hamiltonian.
    \end{proof}

    Let $f(q^i, p_i)$ be an observable, then its grancanonical average is 
    \begin{equation*}\label{gc:av}
    \begin{aligned}
        \avp{f(q^i, p_i)}{gc} & = \sum_{N = 0}^{\infty} \int_{\mathcal M} d\Omega ~ \rho_{gc} f_N = \sum_{N = 0}^{\infty} \int_{\mathcal M} d\Omega ~ \frac{\exp (-\beta (H - \mu N ))}{\mathcal Z} f_N \\ & = \frac{1}{\mathcal Z} \sum_{N=0}^{\infty} z^N Z_N \int_{\mathcal M} d\Omega \frac{\exp(-\beta H)}{Z_N} f_N = \frac{1}{\mathcal Z} \sum_{N=0}^{\infty} z^N Z_N \avp{f_N}{c} ~,
    \end{aligned}
    \end{equation*}
    which shows that we can compute it from the canonical average.

\section{Grand potential as grand canonical potential} 

    A local chart with coordinates $(T, V, \mu)$ is suitable for grand potential~\eqref{td:coord:o}. Hence, we need to find an expression for this thermodynamic potential. The first guess is to define the grand potential as 
    \begin{equation}\label{gc:zo}
        \mathcal Z = \exp(- \beta \Omega[T, V, \mu]) ~,
    \end{equation}
    or, equivalently,
    \begin{equation}\label{gc:o}
        \Omega = - \frac{1}{\beta} \ln \mathcal Z ~.
    \end{equation}
    \begin{proof}
        It is indeed the grand potential, using~\eqref{td:o2} and~\eqref{proof5}
        \begin{equation*}
            \mathcal Z = \exp(- \beta \Omega) = \exp(\beta p V) ~.
        \end{equation*}
    \end{proof}

    The grand canonical internal energy is defined as
    \begin{equation}\label{gc:e}
        E = \av{H}_{gc} = \sum_{N = 0}^{\infty} \int_{\mathcal M} d\Omega ~ \frac{\exp (-\beta (H - \mu N ))}{\mathcal Z} H ~,
    \end{equation}
    but we can be also compute it with
    \begin{equation}\label{gc:e2}
        E = - \pdv{}{\beta} \ln \mathcal Z \Big \vert_z ~.
    \end{equation}
    \begin{proof}
        In fact, using~\eqref{gc:av}
        \begin{equation*}
        \begin{aligned}
            E & = \sum_{N=0}^{\infty} \int d\Omega ~ \frac{\exp(-\beta (H + \mu N))}{\mathcal Z} H = - \sum_{N=0}^{\infty} \frac{z^N}{\mathcal Z} \pdv{}{\beta} \underbrace{\int d\Omega \exp (-\beta H)}_{Z_N} \\ &  = - \frac{1}{\mathcal Z} \pdv{}{\beta} \underbrace{\sum_{N=0}^{\infty} z^N Z_N}_{\mathcal Z} \Big \vert_z = - \frac{1}{\mathcal Z} \pdv{}{\beta} \mathcal Z \Big \vert_z ~,
        \end{aligned}
        \end{equation*}
        where we have used the trick to extract the derivative with respect to $\beta$, keeping $z$ constant.
    \end{proof}

    The grand canonical number of particles is defined as
    \begin{equation}\label{gc:n}
        N = \av{N}_{gc} = \sum_{N = 0}^{\infty} \int_{\mathcal M} d\Omega ~ \frac{\exp (-\beta (H - \mu N ))}{\mathcal Z} N ~,
    \end{equation}
    but we can be also compute it with
    \begin{equation}\label{gc:n2}
        N = z \pdv{}{z} \ln \mathcal Z \Big \vert_T ~.
    \end{equation}
    \begin{proof}
        In fact, using~\eqref{gc:av}
        \begin{equation*}
        \begin{aligned}
            N & = \sum_{N=0}^{\infty} z^N Z_N N = \frac{z}{\mathcal Z} \sum_{N=0}^{\infty} N z^{N-1} Z_N \\ & = \frac{z}{\mathcal Z} \pdv{}{z} \sum_{N=0}^{\infty} z^N Z_N \Big \vert_T = \frac{z}{\mathcal Z} \pdv{}{z}\mathcal Z \Big \vert_T = z \pdv{}{z} \ln \mathcal Z \Big \vert_T ~,
        \end{aligned}
        \end{equation*}
        where we have used the trick to extract the derivative with respect to $z$, keeping $T$ constant.
    \end{proof}

    The universal Boltzmann's formula~\eqref{mc:unibol} is still valid in this ensemble
    \begin{equation*}
        S_{gc} = - k_B \avp{\ln \rho_{gc}}{gc} ~.
    \end{equation*}
    \begin{proof}
        Using~\eqref{gc:av},~\eqref{gc:o},~\eqref{gc:e},~\eqref{gc:n} and the inverse of~\eqref{td:o}
        \begin{equation*}
        \begin{aligned}
            - k_B \avp{\ln \rho_{gc}}{gc} & = -k_B \int d\Omega ~ \rho_{gc} \ln \rho_{gc} = -k_B \sum_{N=0}^{\infty} \frac{z^N}{\mathcal Z} \int d\Omega ~ \exp(- \beta H) \ln \rho_{gc} \\ & = -k_B \sum_{N=0}^{\infty} \frac{z^N}{\mathcal Z} \int d\Omega ~ \exp(- \beta H) (- \beta H + \beta \mu N + \ln \mathcal Z) \\ & = k_B \beta \underbrace{\sum_{N=0}^{\infty} \frac{z^N}{\mathcal Z} \int d\Omega ~ \exp(-\beta H) H}_{E} - k_B \beta \mu \underbrace{\sum_{N=0}^{\infty} \frac{z^N}{\mathcal Z} \int d\Omega ~ \exp(-\beta H) N}_{N} \\ & \quad + k_B \ln \mathcal Z \underbrace{\sum_{N=0}^{\infty} \frac{z^N}{\mathcal Z} \int d\Omega ~ \exp(-\beta H)}_{1} = \frac{E - \mu N - \Omega}{T} = S ~.
        \end{aligned}
        \end{equation*}
    \end{proof}

\chapter{Entropy and counting of states}

    In this chapter, we will introduce entropy using a different approach. 
    
\section{Shannon's entropy} 
    
    Standardly, entropy is defined by the $2$nd law of thermodynamic~\eqref{td:2nd}, which tells us also that an equilibrium system is characterised to be the configuration with maximum entropy. However, in the microcanonical ensemble, entropy is defined in terms of the number of states~\eqref{mc:s} or by the Boltzmann's universal law~\eqref{mc:unibol}
    \begin{equation*}
        S = - k_B \av{\ln \rho} = k_B \ln \Sigma = \lim_{td} S_{td} ~.
    \end{equation*}
    Now, consider a system in the canonical ensemble with a discrete set of energy values (but it can be generalise for the grancanonical one and for continuous energy levels), with associated probability density distribution~\eqref{c:pdd}
    \begin{equation*}
        \rho_c (E_r) = \frac{\exp(-\beta E_r)}{Z_N} ~,
    \end{equation*}
    where the canonical partition function~\eqref{c:z} is 
    \begin{equation*}
        Z_N = \int_{\mathcal M^N} d\Omega~ \exp(-\beta H(q^i, p_i)) = \int_0^\infty dE \int_{S_E} dS_E ~ \exp(-\beta E) \simeq \sum_{r=1}^{p} g_r \exp(-\beta E_r)
    \end{equation*}
    and $g_r$ is the multiplicity or degeneracy, i.e.~how many levels have the same energy. Notice that we have knowledge only of the energy levels, we do not have the completely description of the microscopic degrees of freedom.

    So far, we have started from an a-priori probability density distribution, based on the knowledge of phase space (microstates, equations of motion, ergodicity, etc), and at the end we have derived the entropy. However, we can change the picture and do the converse: the probability distibution is the one corresponding to maximum entropy, given the macroscopic constains. Entropy becomes the starting concept and the distribution the inference. Quantitavitely, we introduce the Shannon's information entropy
    \begin{equation}\label{e:shannon}
        H = - \sum_{i = 1}^{N} p_i \ln p_i ~.
    \end{equation}
    It is the only function, up to constants, that, given a random variable $x$ such that it has $N$ possible outcomes $x_i$ with probability $p_i$, has the following properties
    \begin{enumerate}
        \item it is continuous with $p_i$,
        \item is monotonically increasing with $N$,
        \item it is invariant under compositions of subsystems, i.e.~independent on the choice of how we collect in group, e.g.~a dice can be collected in even and odd numbers or in greater and less than a fixed number.
    \end{enumerate}

\section{Inference problem}

    To study this problem, we need to investigate the concept of probability. In ensemble theory, the probability is interpreted to be objective, since it can be obtained by studying the infinite-limit of frequency and occurencies. On the other hand, probability can be associated to the human ignorance and expectation values are given by available information. Now, we have to solve the inference problem: given certain constraints for a function $\av{f}$, what is the expectation value for another function $g$? The answer can be found with the principle of maximum entropy, subjected to Lagrange multipliers given by constraints 
    \begin{equation*}
        \sum_{i=1}^{N} p_i = 1 ~, \quad \sum_{i=1}^{N} p_i f(x_i) = \av{f(x)} ~.
    \end{equation*}
    Hence, the problem reduces to maximise the constrained entropy
    \begin{equation}\label{e:constrain}
        H = - \sum_{i=1}^{N} p_i \ln p_i + \alpha \Big( \sum_{i=1}^{N} p_i - 1 \Big) + \beta \Big( \sum_{i=1}^{N} p_i f(x_i) - \av{f} \Big) + \text{other constraints} \ldots~.
    \end{equation}
    In particular, we can manipulate the first term and express the result in terms of how many states are occupied. In fact, if we introduce the number of ways $W_{\{n_r\}}$ we can find $n_r$ systems with energy $E_r$, given a set of discrete energy levels $E_r$, each of degeneracy $g_r$ on which we distribute $n_r$ particles, the inference problem transforms into finding the density distribution $n_r^*$, i.e.~the one which maximises~\eqref{e:constrain}, with entropy
    \begin{equation}\label{e:shannon2}
        S = \ln W_{\{n_r\}}
    \end{equation} 
    and constraint on energy and number of particles
    \begin{equation*}
        N = \sum_{r} n_r ~, \quad E = \sum_r n_r E_r ~.
    \end{equation*}
    Finally, in order to count $W_{\{n_r\}}$, we need to take into account distinguishablility of particles. Therefore, we decomposed it into two terms
    \begin{equation}\label{e:count}
        W_{\{n_r\}} = W_{\{n_r\}}^{(1)} W_{\{n_r\}}^{(2)} ~,
    \end{equation}
    where $W_{\{n_r\}}^{(1)}$ counts in how many we can put $n_r$ particles in the energy level $E_r$ and $W_{\{n_r\}}^{(2)}$ takes into account the degeneracy of these levels. In this way, under the assumption of validity of Stirling's approximation (large number of particles) and of smoothness of $n_r$, Boltzmann's classical, Fermi-Dirac's and Bose-Einstein's quantum distributions can be all derived.


\chapter{Applications}

    In this chapter, we will study different physical systems. In the microcanonical ensemble, we will analyse a non-relativistic ideal gas and a gas of harmonic oscillators. In the canonical ensemble, we will analyse a non-relativistic ideal gas, a gas of harmonic oscillators, an ultra-relativistic gas, a magnetic solid and the Maxwell-Boltzmann distribution. In the grand canonical ensemble, we will analyse a non-relativistic ideal gas and a gas of harmonic oscillators. Furthermore, we will investigate the counting of state approach with the Maxwell-Boltzmann, Fermi-Dirac and Bose-Einstein distributions, and the two-level system.

\section{Microcanonical non-relativistic ideal gas}

    \begin{exercise}
        Consider a non-relativistic ideal, i.e.~non-interacting, gas of $N$ particles with mass $m$, living in an $2dN$-dimensional manifold with a finite volume $V^N$: $\mathcal M^N = V^N \times \mathbb R^{dN}$, with Hamiltonian 
        \begin{equation*}
            H = \sum_{i=1}^{dN} \frac{p^2_i}{2m} ~.
        \end{equation*}
        Find the number of states $\Sigma(E)$, the density of states $\omega(E)$, the entropy $S$, the internal energy $E$ and the equation of state $p = p(V, T, N)$. Finally, express the same physical quantities in the case $d = 3$.
    \end{exercise}

    The number of states $\Sigma(E)$ is 
    \begin{equation*}
        \Sigma(E) = \frac{V^{N}}{\xi_N \Gamma(dN/2 + 1)} \Big ( \frac{2 \pi m E}{h^2}\Big)^{dN/2} ~.
    \end{equation*}
    In the case $d = 3$, we obtain 
    \begin{equation*}
        \Sigma(E) = \frac{V^{N}}{\xi_N \Gamma(3N/2 + 1)} \Big ( \frac{2 \pi m E}{h^2}\Big)^{3N/2} ~.
    \end{equation*}
    \begin{proof}
        By definition~\eqref{cm:vol}, we have
        \begin{equation*}
        \begin{aligned}
            \Sigma (E) = \int_{H (q_i, p_i) \leq E} d\Omega = \int_{H (q_i, p_i) \leq E} \frac{\prod_i d^d q_i d^d p_i}{h^{dN} \xi_N} = \frac{1}{h^{dN} \xi_N} \int_{H (q_i, p_i) \leq E} \prod_i d^d q_i d^d p_i ~.
        \end{aligned}
        \end{equation*}
        In order to find the domain of integration, we compute the following inequality
        \begin{equation*}
            H = \sum_i \frac{p^2_i}{2m} \leq E ~, \quad \sum_i p^2_i \leq 2mE ~.
        \end{equation*}
        Hence, by the volume of a $dN$-sphere of radius $\sqrt{2mE}$~\eqref{app:volumen}, we find
        \begin{equation*}
        \begin{aligned}
            \Sigma (E) & = \frac{1}{h^{dN} \xi_N} \int_{\sum_i p^2_i \leq 2mE} \prod_i d^d q_i d^d p_i = \frac{1}{h^{dN} \xi_N} \underbrace{\int_{V^N} \prod_i d^d q_i}_{V^N} \underbrace{\int_{\sum_i p^2_i \leq 2mE} \prod_i d^d p_i}_{\frac{\pi^{dN/2} (2mE)^{dN/2}}{\Gamma (dN/2 + 1)}} \\ & = \frac{V^N}{h^{dN} \xi_N} \frac{\pi^{dN/2} (2mE)^{dN/2}}{\Gamma (dN/2 + 1)} = \frac{V^N}{\Gamma (dN/2 + 1) \xi_N} \Big (\frac{2 \pi m E}{h^2} \Big)^{dN/2} ~.
        \end{aligned}
        \end{equation*}
    \end{proof}

    The density of states $\omega(E)$ is
    \begin{equation}\label{freegas}
        \omega (E) = \frac{V^N}{\xi_N \Gamma(dN/2)} \Big ( \frac{2 \pi m}{h^2} \Big )^{dN/2} E^{dN/2-1} ~.
    \end{equation}
    In the case $d = 3$, we obtain  
    \begin{equation*}
        \omega (E) = \frac{V^N}{\xi_N \Gamma(3N/2)} \Big ( \frac{2 \pi m}{h^2} \Big )^{3N/2} E^{3N/2-1} ~.
    \end{equation*}
    \begin{proof}
        By definition~\eqref{cm:denst}, we have
        \begin{equation*}
        \begin{aligned}
            \omega (E) &= \pdv{\Sigma (E)}{E} = \pdv{}{E} \Big (\frac{2 \pi m E}{h^2} \Big)^{dN/2} \frac{2 V^N}{\Gamma (dN/2)\xi_N d N} = \Big (\frac{2 \pi m}{h^2} \Big)^{dN/2} \frac{2 V^N}{\Gamma (dN/2)\xi_N d N} \pdv{E^{dN/2}}{E} \\ & = \Big (\frac{2 \pi m}{h^2} \Big)^{dN/2} \frac{2 V^N}{\Gamma (dN/2)\xi_N d N} \frac{dN}{2} E^{dN/2 - 1} = \frac{V^N}{\xi_N \Gamma(dN/2)} \Big ( \frac{2 \pi m}{h^2} \Big )^{dN/2} E^{dN/2-1} ~.
        \end{aligned}
        \end{equation*}
    \end{proof}
    Notice that 
    \begin{equation*}
        \omega(E) = \frac{dN}{2E} \Sigma(E) ~, \quad \Gamma(E) = \omega(E) \Delta E = \frac{dN}{2E} \Sigma(E) \Delta E ~,
    \end{equation*}
    which proves, in the thermodynamic limit, that the relations~\eqref{mc:tdlim} are equivalents
    \begin{equation*}
        \lim_{TD} \frac{\ln \Gamma (E)}{N} = \lim_{TD} \frac{\ln \omega (E)}{N} = \lim_{TD} \frac{\ln \Sigma (E)}{N} ~.
    \end{equation*}
    \begin{proof}
        $\omega(E)$, $\Gamma(E)$ and $\Sigma(E)$ differ only, in the logarithmic expression, by factors $\ln \Delta E$ and $\ln \frac{dN}{2E}$, which are negligible in the thermodynamic limit because they do not scale at least as $N$.
    \end{proof}
    The entropy $S$ is
    \begin{equation*}
        \frac{S}{k_B} = \ln \Gamma(E) = \ln \omega(E) + \ln \Delta E = \ln \Sigma(E) + \ln \frac{dN}{2E} + \ln \Delta E ~,
    \end{equation*}
    and, in the thermodynamic limit, it becomes
    \begin{equation*}
        S = k_B \begin{cases}
            \frac{d}{2} N + N \ln \Big ( V (\frac{4 \pi m E}{d N h^2})^{d/2} \Big) & \textnormal{for distinguishable particles} \\
            \frac{d + 2}{2} N + N \ln \Big ( \frac{V}{N} (\frac{4 \pi m E}{d N h^2})^{d/2} \Big) & \textnormal{for indistinguishable particles} \\
        \end{cases} ~,
    \end{equation*}
    In the case $d = 3$, we obtain 
    \begin{equation*}
        S = k_B \begin{cases}
            \frac{3}{2} N + N \ln \Big ( V (\frac{4 \pi m E}{3 N h^2})^{3/2} \Big) & \textnormal{for distinguishable particles} \\
            \frac{5}{2} N + N \ln \Big ( \frac{V}{N} (\frac{4 \pi m E}{3 N h^2})^{3/2} \Big) & \textnormal{for indistinguishable particles} \\
        \end{cases} ~.
    \end{equation*}
    \begin{proof}
        By definition~\eqref{mc:s}, using the Stirling approximation~\eqref{app:stirl}, we have
        \begin{equation*}
        \begin{aligned}
            \frac{S}{k_B} & = \ln \Sigma (E) = \ln \Big ( \frac{V^{N}}{\xi_N \Gamma(dN/2 + 1)} \Big ( \frac{2 \pi m E}{h^2}\Big)^{dN/2} \Big ) \\ & =  N \ln V - \ln \xi_N - \underbrace{\ln \Gamma (dN/2 + 1)}_{\frac{dN}{2} \ln \frac{dN}{2} - \frac{dN}{2}} + N \ln \Big (\frac{2 \pi m E}{h^2} \Big )^{d/2} \\ & = N \ln V - \ln \xi_N - \frac{dN}{2} \ln \frac{dN}{2} + \frac{dN}{2} + N \ln \Big (\frac{2 \pi m E}{h^2} \Big )^{d/2} \\ & = N \ln V - \ln \xi_N - N \ln \Big(\frac{dN}{2} \Big)^{d/2} + \frac{dN}{2} + N \ln \Big (\frac{2 \pi m E}{h^2} \Big )^{d/2} \\ & = - \ln \xi_N + \frac{dN}{2} + N \ln \Big ( V (\frac{4 \pi m E}{d N h^2})^{d/2} \Big) ~.
        \end{aligned}
        \end{equation*} 
        Now, we treat the distinguishable and indistinguishable case separately. For distinguishable particles $\xi_N = 1$, we find
        \begin{equation*}
            \frac{S}{k_B} = - \underbrace{\ln 1}_0 + \frac{dN}{2} + N \ln \Big (V \frac{4 \pi m E}{dNh^2} \Big )^{d/2} = \frac{d}{2} N + N \ln \Big ( V (\frac{4 \pi m E}{d N h^2})^{d/2} \Big) ~.
        \end{equation*}
        For indistinguishable particles $\xi_N = N!$, we find
        \begin{equation*}
        \begin{aligned}
            \frac{S}{k_B} & = - \underbrace{\ln N!}_{N \ln N - N} + \frac{dN}{2} + N \ln \Big ( V (\frac{4 \pi m E}{d N h^2})^{d/2} \Big) \\ & = - N \ln N + N + \frac{dN}{2} + N \ln \Big ( V (\frac{4 \pi m E}{d N h^2})^{d/2} \Big) \\ & = \frac{d + 2}{2} N + N \ln \Big ( \frac{V}{N} (\frac{4 \pi m E}{d N h^2})^{d/2} \Big) ~.
        \end{aligned}
        \end{equation*}
    \end{proof}

    The internal energy is 
    \begin{equation*}
        E = \frac{d N k_B T}{2} ~.
    \end{equation*}
    In the case $d = 3$, we obtain 
    \begin{equation*}
        E = \frac{3 N k_B T}{2}
    \end{equation*}
    \begin{proof}
        Using the first of~\eqref{td:es:e}, we have
        \begin{equation*}
            \frac{1}{T} = \pdv{S}{E} = k_B \frac{dN}{2} \pdv{}{E} \ln E = k_B \frac{dN}{2E} ~,
        \end{equation*}
        hence, we find
        \begin{equation*}
            E = \frac{d N k_B T}{2} ~.
        \end{equation*}
    \end{proof}
    The equation of state is  
    \begin{equation*}
        p V = N k_B T ~,
    \end{equation*}
    which is the same for all dimensions.
    \begin{proof}
        By the second of~\eqref{td:es:s}, we have
        \begin{equation*}
            \frac{p}{T} = \pdv{S}{V} = k_B N \pdv{}{V} \ln V = k_B \frac{N}{V}  ~,
        \end{equation*}
        hence, we find
        \begin{equation*}
            pV = N k_B T ~.
        \end{equation*}
    \end{proof}

\section{Microcanonical gas of harmonic oscillators}

    \begin{exercise}
        Consider a non-relativistic, i.e.~non-interacting, gas of $N$ particles with mass $m$, living in an $2dN$-dimensional manifold confined by an harmonic potential of frequency $\omega$, with Hamiltonian 
        \begin{equation*}
            H = \sum_{i =1}^{dN} \Big ( \frac{p^2_i}{2m} + \frac{m \omega^2}{2} q_i^2 \Big ) ~.
        \end{equation*}
        Find the number of states $\Sigma(E)$, the density of states $\omega(E)$, the entropy $S$, the internal energy $E$ and the equation of state $p = p(V, T, N)$. Finally, express the same physical quantities in the case $d = 1$.
    \end{exercise}

    The number of states $\Sigma(E)$ is 
    \begin{equation*}
        \Sigma(E) = \frac{1}{\xi_N\Gamma(dN + 1)} \Big ( \frac{2 \pi E}{h \omega}\Big)^{dN} ~.
    \end{equation*}
    In the case $d = 1$, we obtain 
    \begin{equation*}
        \Sigma(E) = \frac{1}{\xi_N \Gamma(N+1)} \Big ( \frac{2 \pi E}{h \omega}\Big)^{N} ~.
    \end{equation*}
    \begin{proof}
        By definition~\eqref{cm:vol}, we have
        \begin{equation*}
        \begin{aligned}
            \Sigma (E) = \int_{H (q_i, p_i) \leq E} d\Omega = \int_{H (q_i, p_i) \leq E} \frac{\prod_i d^d q_i d^d p_i}{h^{dN} \xi_N} = \frac{1}{h^{dN} \xi_N} \int_{H (q_i, p_i) \leq E} \prod_i d^d q_i d^d p_i ~.
        \end{aligned}
        \end{equation*}
        We make a change of variable, similar to~\eqref{cm:symplcoord}, into $x_j$, with $j = 1, \ldots 2dN$, given by
        \begin{equation*}
            p_i = \sqrt{2mE} x_j ~, \quad q_i = \sqrt{\frac{2E}{m\omega^2}} x_{dN + j} ~,
        \end{equation*}
        \begin{equation*}
            d p_i = \sqrt{2mE} dx_j  ~, \quad d q_i = \sqrt{\frac{2E}{m\omega^2}} d x_{dN + j} ~.
        \end{equation*}
        In order to find the domain of integration, we express energy in terms of the new variable and we compute the following inequality
        \begin{equation*}
            H = \sum_i \Big ( \frac{p^2_i}{2m} + \frac{m \omega^2}{2} q_i^2 \Big ) = \sum_j E x_j^2 \leq E ~, \quad \sum_j x^2_j \leq 1 ~.
        \end{equation*}
        Hence, by the volume of a $2dN$-sphere of radius $1$~\eqref{app:volumen}, we find
        \begin{equation*}
        \begin{aligned}
            \Sigma (E) & = \frac{1}{h^{dN} \xi_N} \int_{\sum_i \Big ( \frac{p^2_i}{2m} + \frac{m \omega^2}{2} q_i^2 \Big ) \leq E} \prod_i d^d q_i d^d p_i \\ & = \frac{1}{h^{dN} \xi_N} (2mE)^{dN/2} \Big (\frac{2E}{m\omega^2} \Big )^{dN/2} \underbrace{\int_{\sum_j x^2_j \leq 1} \prod_j d x_j}_{ \frac{\pi^{dN}}{\Gamma (dN + 1)}} \\ & = \frac{1}{\xi_N \Gamma (dN + 1)} \Big (\frac{2 \pi E}{h \omega} \Big )^{dN} ~.
        \end{aligned}
        \end{equation*}
    \end{proof}
    The density state $\omega(E)$ is
    \begin{equation}\label{harmosc}
        \omega (E) = \frac{1}{\xi_N \Gamma (dN)} \Big (\frac{2 \pi}{h \omega} \Big )^{dN} E^{dN-1} ~.
    \end{equation}
    In the case $d = 1$, we obtain 
    \begin{equation*}
        \omega (E) = \frac{1}{\xi_N \Gamma (N)} \Big (\frac{2 \pi}{h \omega} \Big )^{N} E^{N-1} ~.
    \end{equation*}
    \begin{proof}
        By definition~\eqref{cm:denst}, we have
        \begin{equation*}
        \begin{aligned}
            \omega (E) & = \pdv{\Sigma (E)}{E} = \pdv{}{E} \frac{1}{\xi_N dN \Gamma (dN)} \Big (\frac{2 \pi E}{h \omega} \Big )^{dN} = \frac{1}{\xi_N dN \Gamma (dN)} \Big (\frac{2 \pi}{h \omega} \Big )^{dN} \pdv{}{E} E^{dN} \\ & = \frac{1}{\xi_N dN \Gamma (dN)} \Big (\frac{2 \pi}{h \omega} \Big )^{dN} dN E^{dN-1} = \frac{1}{\xi_N \Gamma (dN)} \Big (\frac{2 \pi}{h \omega} \Big )^{dN} E^{dN-1} ~.
        \end{aligned}
        \end{equation*}
    \end{proof}
    The entropy $S$ is
    \begin{equation*}
        \frac{S}{k_B} = \ln \Gamma(E) = \ln \omega(E) + \ln \Delta E = \ln \Sigma(E) + \ln \frac{dN}{E} + \ln \Delta E ~,
    \end{equation*}
    and, in the thermodynamic limit, it becomes
    \begin{equation*}
        S = k_B \begin{cases}
            d N + N d \ln \Big (\frac{2 \pi E }{h \omega d N} \Big) & \textnormal{for distinguishable particles} \\
            (d+1) N + N d \ln \Big ( \frac{1}{N} (\frac{2 \pi E }{h \omega d N} ) \Big ) & \textnormal{for indistinguishable particles} \\
        \end{cases} ~.
    \end{equation*}
    In the case $d = 1$, we obtain 
    \begin{equation*}
        S = k_B \begin{cases}
            N + N \ln \Big (\frac{2 \pi E }{h \omega N} \Big)  & \textnormal{for distinguishable particles} \\
            2N + N \ln \Big ( \frac{1}{N} (\frac{2 \pi E }{h \omega N} ) \Big ) & \textnormal{for indistinguishable particles} \\
        \end{cases} ~.
    \end{equation*}
    \begin{proof}
        By definition~\eqref{mc:s}, using the Stirling approximation~\eqref{app:stirl}, the entropy is
        \begin{equation*}
        \begin{aligned}
            \frac{S}{k_B} & = \ln \Sigma (E) = \ln \frac{1}{\xi_N \Gamma(dN + 1)} \Big ( \frac{2 \pi E}{h \omega} \Big)^{dN} \\ & = - \ln \xi_N - \underbrace{\ln \Gamma (dN + 1)}_{dN \ln (dN) - dN} + d N \ln \Big (\frac{2 \pi E }{h \omega} \Big ) \\ & = - \ln \xi_N - dN \ln (dN) + d N + dN \ln \Big (\frac{2 \pi E }{h \omega} \Big ) \\ & = - \ln \xi_N + d N + d N \ln \Big (\frac{2 \pi E }{h \omega d N} \Big ) ~.
        \end{aligned}
        \end{equation*}
        Now, we treat the distinguishable and indistinguishable case separately. For distinguishable particles $\xi_N = 1$, we find
        \begin{equation*}
            \frac{S}{k_B} = - \underbrace{\ln 1}_0 + d N + d N \ln \Big (\frac{2 \pi E }{h \omega d N} \Big ) = d N + d N \ln \Big (\frac{2 \pi E }{h \omega d N} \Big ) ~.
        \end{equation*}
        For indistinguishable particles $\xi_N = N!$, we find
        \begin{equation*}
        \begin{aligned}
            \frac{S}{k_B} & = - \underbrace{\ln N!}_{N \ln N - N} + d N + d N \ln \Big (\frac{2 \pi E }{h \omega d N} \Big ) \\ & =  - N \ln N + N + d N + d N \ln \Big (\frac{2 \pi E }{h \omega d N} \Big ) \\ & = (d+1) N + d N \ln \Big ( \frac{1}{N} \frac{2 \pi E }{h \omega d N} \Big ) ~.
        \end{aligned}
        \end{equation*}
    \end{proof}

    The internal energy is 
    \begin{equation*}
        E = d N k_B T ~.
    \end{equation*}
    In the case $d = 3$, we obtain 
    \begin{equation*}
        E = 3 N k_B T ~.
    \end{equation*}
    \begin{proof}
        Using the first of~\eqref{td:es:s}, we have
        \begin{equation*}
            \frac{1}{T} = \pdv{S}{E} = k_B dN \pdv{}{E} \ln E = k_B \frac{dN}{E} ~,
        \end{equation*}
        hence, we find
        \begin{equation*}
            E = d N k_B T ~.
        \end{equation*}
    \end{proof}
    The equation of state is  
    \begin{equation*}
        p = 0 ~,
    \end{equation*}
    which is the same for all dimensions.
    \begin{proof}
        By the second of~\eqref{td:es:s}, we have
        \begin{equation*}
            \frac{p}{T} = \pdv{S}{V} = 0 ~,
        \end{equation*}
        hence, we find
        \begin{equation*}
            p = 0 ~.
        \end{equation*}
    \end{proof}

\section{Canonical non-relativistic ideal gas}

    \begin{exercise}
        Consider a non-relativistic ideal, i.e.~non-interacting, gas of $N$ indistinguishable particles with mass $m$, living in an $2dN$-dimensional manifold with a finite volume $V^N$: $\mathcal M^N = V^N \times \mathbb R^{dN}$, with Hamiltonian 
        \begin{equation*}
            H = \sum_{i=1}^{dN} \frac{p^2_i}{2m} ~.
        \end{equation*}
        Find the canonical partition function $Z$, the internal energy $E$, the Helmholtz partition function $F$, the entropy $S$ (and the temperature $T_c$ under which it becomes negative), the equation of state $p = p(V, T, N)$, the chemical potential $\mu$ and the specific heats $C_V$ and $C_p$. Finally, express the same physical quantities in the case $d = 3$.
    \end{exercise}

    In the following, we will use the thermal wavelength 
    \begin{equation*}
        \lambda_T = \sqrt{\frac{\beta h^2}{2 m \pi}} ~.
    \end{equation*}

    The canonical partition function $Z$ is 
    \begin{equation}\label{idcan}
        Z = \frac{V^N}{N! \lambda^{dN}_T} ~.
    \end{equation}
    In the case $d = 3$, we obtain 
    \begin{equation*}
        Z = \frac{V^N}{N! \lambda^{3N}_T} ~.
    \end{equation*}
    \begin{proof}
        By definition~\eqref{c:z}, using the gaussian integral~\eqref{app:gauss}, we have
        \begin{equation*}
        \begin{aligned}
            Z & = \int_{\mathcal M^N} d\Omega ~ \exp(- \beta H (q_i, p_i)) = \int_{\mathcal M^N} \frac{\prod_i d^d q_i d^d p_i}{h^{dN} N!} \exp(- \beta H (q_i, p_i)) \\ & = \frac{1}{h^{dN} N!} \int_{\mathcal M^N} \prod_i d^d q_i d^d p_i ~ \exp(- \beta H (q_i, p_i)) \\ & = \frac{1}{h^{dN} N!} \underbrace{\int_{ V^N} \prod_i d^d q_i}_{V^N} \underbrace{\prod_i \int_{\mathcal M^N} d^d p_i ~ \exp(- \beta \frac{p^2_i}{2m})}_{(\frac{2 m \pi}{\beta})^{dN/2}} \\ & = \frac{V^N}{h^{dN} N!} (\frac{2 m \pi}{\beta})^{dN/2} = \frac{V^N}{ N!} (\frac{2 m \pi}{\beta h^2})^{dN/2} = \frac{V^N}{N! \lambda^{dN}_T} ~.
        \end{aligned}
        \end{equation*}
    \end{proof}
    An useful intermediary formula is 
    \begin{equation*}
        \ln Z = N (1 - \ln (n \lambda_T^d)) ~.
    \end{equation*}
    \begin{proof}
        In fact, using the Stirling approximation~\eqref{app:stirl}, we have
        \begin{equation*}
        \begin{aligned}
            \ln Z & = \ln \frac{V^N}{N! \lambda^{dN}_T}= N \ln (V \lambda_T^d) - \underbrace{\ln N!}_{N \ln N - N} = N - N \frac{V \lambda_T^d}{N} \\ & = N (1 - \ln (\frac{N}{V} \lambda_T^d)) = N (1 - \ln (n \lambda_T^d))  ~.
        \end{aligned}
        \end{equation*}
    \end{proof}
    The internal energy $E$ is 
    \begin{equation*}
        E = \frac{d}{2} N k_B T ~.
    \end{equation*}
    In the case $d=3$, we obtain 
    \begin{equation*}
        E = \frac{3}{2} N k_B T ~.
    \end{equation*}
    \begin{proof}
        Using~\eqref{c:e2}, we have
        \begin{equation*}
        \begin{aligned}
            E & = - \pdv{\ln Z}\beta  = - \pdv{}{\beta} N (1 - \ln (n \lambda_T^d)) = - Nd \pdv{}{\beta} \ln (\lambda_T) \\ & = - Nd \pdv{}{\beta} \ln (\beta^{1/2}) = \frac{Nd}{2} \frac{1}{\beta} = \frac{d}{2} N k_B T ~.
        \end{aligned}
        \end{equation*}
    \end{proof}
    The Helmholtz free energy $F$ is 
    \begin{equation*}
        F = \frac{N}{\beta} (\ln (n \lambda_T^d) - 1) ~.
    \end{equation*}
    In the case $d=3$, we obtain 
    \begin{equation*}
        F = \frac{N}{\beta} (\ln (n \lambda_T^3) - 1) ~.
    \end{equation*}
    \begin{proof}
        Using~\eqref{c:f}, we have
        \begin{equation*}
            F = - \frac{\ln Z}{\beta} = \frac{N}{\beta} (\ln (n \lambda_T^d) - 1) ~.
        \end{equation*}
    \end{proof}
    The entropy $S$ is 
    \begin{equation*}
        S = N k_B \Big ( \frac{d+2}{2} - \ln (n \lambda_T^d) \Big ) ~.
    \end{equation*}
    In the case $d=3$, we obtain 
    \begin{equation*}
        S = N k_B \Big ( \frac{5}{2} - \ln (n \lambda_T^3) \Big ) ~.
    \end{equation*}
    \begin{proof}
        Using~\eqref{c:s}, we have
        \begin{equation*}
        \begin{aligned}
            S & = \frac{E - F}{T} = \frac{1}{T} \Big ( \frac{d}{2} N k_B T - \frac{N}{\beta} (\ln (n \lambda_T^d) - 1) \Big ) \\ & = \frac{N}{\beta T} \Big ( \frac{d+2}{2} - \ln (n \lambda_T^d) \Big )  = N k_B \Big ( \frac{d+2}{2} - \ln (n \lambda_T^d) \Big )
        \end{aligned}
        \end{equation*}
    \end{proof}
    Entropy becomes negative at a certain critical temperature
    \begin{equation}\label{ex:negs1}
        T_c = \frac{2 m \pi k_B}{h^2} e^{(d+2)/2} n^{-2/d} ~.
    \end{equation}
    In the case $d=3$, we obtain 
    \begin{equation*}
        T_c = \frac{2 m \pi k_B}{h^2} e^{5/2} n^{-2/3} ~.
    \end{equation*}
    A plot of the entropy as a function of $T$ is in Figure~\ref{fig:c:ent}.
    \begin{figure}
        \centering
        \scalebox{0.7}{\pyc{plot1('x', '5/2 - log(1 / x**(3/2))', 1, 3, 0, True, False, False)}}
        \caption{A plot of the entropy $S$ as a function of $T$. We have used $x = \frac{2 \pi m k_B T n^{2/3}}{h^2}$ and $f(x) = \frac{S}{N k_B}$.}
        \label{fig:c:ent}
    \end{figure}
    \begin{proof}
        In fact, after a series of manipulations, $S < 0$ for 
        \begin{equation*}
            N k_B \Big ( \frac{d+2}{2} - \ln (n \lambda_T^d) \Big ) < 0 ~, \quad \frac{d+2}{2} - \ln (n \lambda_T^d) < 0 ~, \quad \frac{d+2}{2} < \ln (n \lambda_T^d) ~,
        \end{equation*}
        \begin{equation*}
            e^{(d+2)/2} < n \lambda_T^d  ~, \quad e^{(d+2)/2} < n \Big ( \frac{h^2 \beta}{2 m \pi} \Big )^{d/2}  ~, \quad e^{(d+2)/d} n^{2/d} < \frac{h^2 \beta}{2 m \pi} ~,
        \end{equation*}
        \begin{equation*}
            \frac{2 m \pi}{h^2} e^{(d+2)/2} n^{-2/d} < \beta ~,
        \end{equation*}
        hence, we find
        \begin{equation*}
            T < \frac{2 m \pi k_B}{h^2} e^{(d+2)/2} n^{-2/d} = T_c ~.
        \end{equation*}
    \end{proof}
    The equation of state is 
    \begin{equation}\label{ides}
        p V = N k_B T ~.
    \end{equation}
    which is the same for all dimensions.
    \begin{proof}
        By the second of~\eqref{td:es:f}, we have
        \begin{equation*}
            p = - \pdv{F}{V} = - \pdv{}{V} \frac{N}{\beta} (\ln (n \lambda_T^d) - 1) = \frac{N}{\beta} \pdv{}{V} \ln V = \frac{N}{V \beta} ~,
        \end{equation*}
        hence, we find
        \begin{equation*}
            p V = N k_B T ~.
        \end{equation*}
    \end{proof}
    The chemical potential $\mu$ is 
    \begin{equation*}
        \mu = \frac{1}{\beta} \ln (n \lambda_T^d) ~.
    \end{equation*}
    In the case $d=3$, we obtain 
    \begin{equation*}
        \mu = \frac{1}{\beta} \ln (n \lambda_T^3) ~.
    \end{equation*}
    A plot of the chemical potential as a function of $T$ is in Figure~\ref{fig:c:chem}.
    \begin{figure}
        \centering
        \scalebox{0.7}{\pyc{plot1('x', 'x * log(1 / x**(3/2)) ', 2, 2, 1, True, False, False)}}
        \caption{A plot of the chemical potential $\mu$ as a function of $T$. We have used $x = \frac{2 \pi m k_B T n^{2/3}}{h^2}$ and $f(x) = \frac{2 \pi m \mu}{h^2 n^{3/2}}$.}
        \label{fig:c:chem}
    \end{figure}
    \begin{proof}
        By the third of~\eqref{td:es:f}, we have
        \begin{equation*}
            \mu = \pdv{F}{N} = \pdv{}{N} \frac{N}{\beta} (\ln (n \lambda_T^d) - 1) = \frac{1}{\beta} (\ln (n \lambda_T^d) - 1) + \frac{1}{\beta} = \frac{1}{\beta} \ln (n \lambda_T^d) ~.
        \end{equation*}
    \end{proof}
    The specific heats $C_V$ and $C_p$ are 
    \begin{equation*}
        C_V = N \frac{d}{2} k_B ~, \quad C_p = N \frac{d+2}{2} k_B ~. 
    \end{equation*}
    In the case $d=3$, we obtain 
    \begin{equation*}
        C_V = N \frac{3}{2} k_B ~, \quad C_p = N \frac{5}{2} k_B ~. 
    \end{equation*}
    \begin{proof}
        At $V$ constant, by definition~\eqref{td:cv2}, we find
        \begin{equation*}
            C_V = \pdv{E}{T} = \pdv{}{T} \frac{d}{2} N k_B T = N \frac{d}{2} k_B ~.
        \end{equation*}
        At $p$ constant, using~\eqref{td:cp2} and~\eqref{ides}, we find
        \begin{equation*}
            C_p = C_V + p \pdv{V}{T} = C_V + p \pdv{}{T} \frac{N k_B T}{p} = N \frac{d}{2} k_B + N k_B = \frac{d + 2}{2} k_B ~.
        \end{equation*}
    \end{proof}
    Notice that there are two problems: entropy cannot be negative and the specific heat $C_V \rightarrow 0$ for $T \rightarrow 0$, by thermodynamics. This means that this model is not correct and we must go quantum.

\section{Canonical gas of harmonic oscillators}

    \begin{exercise}
        Consider a non-relativistic (non-interacting) gas of $N$ distinguishable particles in an $d$-dimensional manifold confined by an harmonic potential of frequency $\omega$, with Hamiltonian 
        \begin{equation*}
            H = \sum_i \Big ( \frac{p^2_i}{2m} + \frac{m \omega^2}{2} q_i^2 \Big ) ~.
        \end{equation*}
        Find the canonical partition function $Z$, the internal energy $E$, the Helmholtz partition function $F$, the entropy $S$ (and the temperature $T_c$ under which it becomes negative), the equation of state, the chemical potential $\mu$ and the specific heats $C_V$ and $C_p$. Finally, express the same physical quantities in the case $d = 1$.
    \end{exercise}

    The canonical partition function $Z$ is 
    \begin{equation}\label{harmz}
        Z = \Big (\frac{1}{\hbar \omega \beta} \Big )^{dN} ~.
    \end{equation}
    In the case $d = 1$, we obtain
    \begin{equation*}
        Z = \Big (\frac{1}{\hbar \omega \beta} \Big )^{N} ~.
    \end{equation*}
    \begin{proof}
        By definition~\eqref{c:z}, using the gaussian integral~\eqref{app:gauss}, we have
        \begin{equation*}
        \begin{aligned}
            Z & = \int_{\mathcal M^N} d\Omega \exp(- \beta H (q_i, p_i)) = \int_{\mathcal M^N} \frac{\prod_i d^d q_i d^d p_i}{h^{dN} } \exp(- \beta H (q_i, p_i)) \\ & = \frac{1}{h^{dN} } \int_{\mathcal M^N} \prod_i d^d q_i d^d p_i \exp(- \beta H (q_i, p_i)) \\ & = \frac{1}{h^{dN} } \underbrace{\int_{ V^N} \prod_i d^d q_i \exp(- \beta \frac{m \omega^2}{2} q_i^2)}_{(\frac{2 \pi}{m \omega \beta})^{dN/2}} \underbrace{\prod_i \int_{\mathcal M^N} d^d p_i \exp(- \beta \frac{p^2_i}{2m})}_{(\frac{2 m \pi}{\beta})^{dN/2}} \\ & = \frac{1}{h^{dN} } (\frac{2 \pi}{m \omega \beta})^{dN/2} (\frac{2 m \pi}{\beta})^{dN/2} = \Big (\frac{2\pi}{h \omega \beta} \Big )^{dN} = \Big (\frac{1}{\hbar \omega \beta} \Big )^{dN}~.
        \end{aligned}
        \end{equation*}
    \end{proof}
    An useful intermediary formula is 
    \begin{equation*}
        \ln Z = - d N \ln (\hbar \omega \beta) ~.
    \end{equation*}
    \begin{proof}
        In fact, using the Stirling approximation~\eqref{app:stirl}, we have
        \begin{equation*}
            \ln Z = - \ln (\frac{}{\hbar \omega \beta})^{dN} = - d N \ln (\hbar \omega \beta)  ~.
        \end{equation*}
    \end{proof}
    The internal energy $E$ is 
    \begin{equation*}
        E = d N k_B T ~.
    \end{equation*}
    In the case $d = 1$, we obtain 
    \begin{equation*}
        E = N k_B T~.
    \end{equation*}
    \begin{proof}
        Using~\eqref{c:e2}, we have
        \begin{equation*}
            E = - \pdv{\ln Z}{\beta} = \pdv{}{\beta} d N \ln (\hbar \omega \beta) = d N \frac{1}{\beta} = d N k_B T ~.
        \end{equation*}
    \end{proof}
    The Helmholtz free energy $F$ is 
    \begin{equation*}
        F = \frac{dN}{\beta} \ln (\hbar \omega \beta) ~.
    \end{equation*}
    In the case $d = 1$, we obtain 
    \begin{equation*}
        F = \frac{N}{\beta} \ln (\hbar \omega \beta) ~.
    \end{equation*}
    \begin{proof}
        Using~\eqref{c:f}, we have
        \begin{equation*}
            F = - \frac{\ln Z}{\beta} = \frac{dN}{\beta} \ln (\hbar \omega \beta) ~.
        \end{equation*}
    \end{proof}
    The entropy $S$ is 
    \begin{equation*}
        S = d N k_B (1 - \ln (\hbar \omega \beta)) ~.
    \end{equation*}
    In the case $d=1$, we obtain 
    \begin{equation*}
        S = N k_B (1 - \ln (\hbar \omega \beta)) ~.
    \end{equation*}
    \begin{proof}
        Using~\eqref{c:s}, we have
        \begin{equation*}
            S = \frac{E - F}{T} = \frac{1}{T} \Big ( d N k_B T - \frac{dN}{\beta} \ln (\hbar \omega \beta) \Big ) = d N k_B (1 - \ln (\hbar \omega \beta)) ~.
        \end{equation*}
    \end{proof}
    Entropy becomes negative at a certain critical temperature
    \begin{equation}\label{ex:negs2}
        T_c = \frac{\hbar \omega}{k_B e} ~.
    \end{equation}
    which is the same for all dimensions.
    A plot of the entropy $S$ as a function of $T$ is in Figure~\ref{fig:c:ent2}.
    \begin{figure}
        \centering
        \scalebox{0.7}{\pyc{plot1('x', '1 - log(1 / x)', 2, 3, 2, True, False, False)}}
        \caption{A plot of the entropy $S$ as a function of $T$. We have used $x = \frac{2 \pi k_B T}{h \omega}$ and $f(x) = \frac{S}{N k_B}$.}
        \label{fig:c:ent2}
    \end{figure}
    \begin{proof}
        In fact, after a series of manipulations, $S < 0$ for 
        \begin{equation*}
            d N k_B (1 - \ln (\hbar \omega \beta)) < 0 ~, \quad 1 - \ln (\hbar \omega \beta) < 0 ~,
        \end{equation*}
        \begin{equation*}
            1 < \ln (\hbar \omega \beta) ~, \quad e < \hbar \omega \beta = \frac{\hbar \omega}{k_B T}  ~,
        \end{equation*}
        hence, we find
        \begin{equation*}
            T < \frac{\hbar \omega}{k_B e} = T_c ~.
        \end{equation*}
    \end{proof}
    The equation of state is 
    \begin{equation}\label{idesharm}
        p = 0 ~.
    \end{equation}
    which is the same for all dimensions.
    \begin{proof}
        By the second of~\eqref{td:es:f}, we have
        \begin{equation*}
            p = - \pdv{F}{V} = 0 ~.
        \end{equation*}
    \end{proof}
    The chemical potential $\mu$ is 
    \begin{equation*}
        \mu = \frac{d}{\beta} \ln (\hbar \omega \beta) ~.
    \end{equation*}
    In the case $d=1$, we obtain
    \begin{equation*}
        \mu = \frac{1}{\beta} \ln (\hbar \omega \beta) ~.
    \end{equation*}
    A plot of this the chemical potential $\mu$ as a function of $T$ is in Figure~\ref{fig:c:mu2}.
    \begin{figure}
        \centering
        \scalebox{0.7}{\pyc{plot1('x', 'x * log(1 / x)', 3, 3, 3, True, False, False)}}
        \caption{A plot of the chemical potential $\mu$ as a function of $T$. We have used $x = \frac{2 \pi k_B T}{h \omega}$ and $f(x) = \frac{2 \pi \mu}{h \omega}$.}
        \label{fig:c:mu2}
    \end{figure}
    \begin{proof}
        By the third of~\eqref{td:es:f}, we have
        \begin{equation*}
            \mu = \pdv{F}{N} = \pdv{}{N} \frac{dN}{\beta} \ln (\hbar \omega \beta) = \frac{d}{\beta} \ln (\hbar \omega \beta) ~.
        \end{equation*}
    \end{proof}
    The specific heats $C_V$ and $C_p$ are 
    \begin{equation*}
        C_V = d N k_B  ~, \quad C_p = d N k_B ~. 
    \end{equation*}
    \begin{proof}
        At $V$ constant, by definition~\eqref{td:cv2}, we find
        \begin{equation*}
            C_V = \pdv{E}{T} = \pdv{}{T} d N k_B T  = d N k_B ~.
        \end{equation*}
        At $p$ constant, using~\eqref{td:cp2} and~\eqref{idesharm}, we find
        \begin{equation*}
            C_p = C_V + p \pdv{V}{T} = C_V = d N k_B ~.
        \end{equation*}
    \end{proof}

    Notice that also here there are two problems: entropy cannot be negative and the specific heat $C_V \rightarrow 0$ for $T \rightarrow 0$, by thermodynamics. This means that this model is not correct and we must go quantum.

\section{Grand canonical non-relativistic ideal gas}

    \begin{exercise}
        Consider a non-relativistic ideal (non-interacting) gas of $N$ indistinguishable particles with mass $m$, living in an $d$-dimensional manifold with a finite volume $V^N$: $\mathcal M^N = V^N \times \mathbb R^{dN}$, with Hamiltonian 
        \begin{equation*}
            H = \sum_i \frac{p^2_i}{2m} ~.
        \end{equation*}
        Find the grand canonical partition function $Z$, the internal energy $E$, the number of particles $N$ and the equation of state.
    \end{exercise}

    The grancanonical partition function $\mathcal Z$ is 
    \begin{equation*}
        \mathcal Z = \exp(\frac{z V}{\lambda_T^d}) ~.
    \end{equation*}
    \begin{proof}
        By definition~\eqref{gc:z}, using~\eqref{idcan} and the Taylor expansion of the exponential, we have
        \begin{equation*}
            \mathcal Z = \sum_{N=0}^\infty z^N Z_N = \sum_{N=0}^\infty \frac{1}{N!} \Big ( \frac{z V}{\lambda_T^d} \Big)^N = \exp(\frac{z V}{\lambda_T^d}) ~.
        \end{equation*}
    \end{proof}
    The internal energy $E$ is 
    \begin{equation*}
        E = \frac{z V}{\lambda^d_T} \frac{d}{2 \beta} ~.
    \end{equation*}
    \begin{proof}
        Using~\eqref{gc:e2}, we have
        \begin{equation*}
        \begin{aligned}
            E & = - \pdv{\ln \mathcal Z}{\beta} \Big \vert_z = - \pdv{}{\beta} \ln \exp(\frac{z V}{\lambda_T^d}) = - \pdv{}{\beta} \frac{z V}{\lambda^d_T} \\ & = - \frac{1}{zV} \pdv{}{\beta} \Big ( \frac{2 m \pi} {\beta h^2} \Big )^{d/2} = - \frac{1}{z V} \Big ( \frac{2 m \pi}{h^2} \Big )^{d/2} \pdv{}{\beta} \beta^{-d/2} \\ & = \frac{1}{zV} \Big ( \frac{h^2}{2 m \pi} \Big )^{d/2} \frac{d}{2} \beta^{-d/2 - 1} = \frac{z V}{\lambda^d_T} \frac{d}{2 \beta} ~.
        \end{aligned}
        \end{equation*}
    \end{proof}
    The number of particle $N$ is 
    \begin{equation*}
        N = \frac{zV}{\lambda_T^d} ~.
    \end{equation*}
    \begin{proof}
        Using~\eqref{gc:n2}, we have
        \begin{equation*}
            N = z \pdv{}{z} \ln \mathcal Z = z \pdv{}{z} \frac{z V}{\lambda_T^d} = \frac{zV}{\lambda_T^d} ~.
        \end{equation*}
    \end{proof}
    The equation of state is 
    \begin{equation*}
        p = \frac{z}{\beta \lambda_T^d} ~.
    \end{equation*}
    \begin{proof}
        By the second of~\eqref{td:es:o} and~\eqref{gc:o}, we have
        \begin{equation*}
            p = - \pdv{\Omega}{V} = - \frac{1}{\beta V} \ln \mathcal Z = \frac{1}{\beta V} \frac{z V}{\lambda_T^d} = \frac{z}{\beta \lambda_T^d} ~.
        \end{equation*}
    \end{proof}

\section{Grand canonical gas of harmonic oscillators}

    \begin{exercise}
        Consider a non-relativistic (non-interacting) gas of $N$ distinguishable particles in an $d$-dimensional manifold confined by an harmonic potential of frequency $\omega$, with Hamiltonian 
        \begin{equation*}
            H = \sum_i \Big ( \frac{p^2_i}{2m} + \frac{m \omega^2}{2} q_i^2 \Big ) ~.
        \end{equation*}
        Find the grand canonical partition function $Z$, the internal energy $E$, the number of particles $N$ and the equation of state.
    \end{exercise}

    The grancanonical partition function $\mathcal Z$ is 
    \begin{equation*}
        \mathcal Z = \frac{1}{1 - \frac{z}{\hbar \beta \omega}} ~.
    \end{equation*}
    Notice that this series converges only if $\hbar \beta \omega \ll 1$, which is the high temperature limit.
    \begin{proof}
        By definition~\eqref{gc:z}, using~\eqref{harmz} and the geometrical series, we have
        \begin{equation*}
            \mathcal Z = \sum_{N=0}^\infty z^N Z_N = \sum_{N=0}^\infty \Big ( \frac{z}{\hbar \beta \omega} \Big) = \frac{1}{1 - \frac{z}{\hbar \beta \omega}}  ~.
        \end{equation*}
    \end{proof}
    The internal energy $E$ is 
    \begin{equation*}
        E = \frac{z}{\beta (\hbar \omega \beta - z)} ~.
    \end{equation*}
    \begin{proof}
        Using~\eqref{gc:e2}, we have
        \begin{equation*}
        \begin{aligned}
            E & = - \pdv{\ln \mathcal Z}{\beta} \Big \vert_z = - \pdv{}{\beta} \ln \frac{1}{1 - \frac{z}{\hbar \beta \omega}} = \pdv{}{\beta} \ln (1 - \frac{z}{\hbar \beta \omega} ) \\ & = \frac{1}{1 - \frac{z}{\hbar \beta \omega}} \frac{z}{\hbar \omega \beta^2} = \frac{z}{\beta (\hbar \omega \beta - z)} ~.
        \end{aligned}
        \end{equation*}
    \end{proof}
    The number of particle $N$ is 
    \begin{equation*}
        N = \frac{z}{\hbar \omega \beta - z} ~.
    \end{equation*}
    \begin{proof}
        Using~\eqref{gc:n2}, we have
        \begin{equation*}
            N = z \pdv{}{z} \ln \mathcal Z = z \pdv{}{z} \ln \frac{1}{1 - \frac{z}{\hbar \beta \omega}} = - z \pdv{}{z} \ln (1 - \frac{z}{\hbar \beta \omega}) = \frac{1}{1 - \frac{z}{\hbar \beta \omega}}\frac{z}{\hbar \omega \beta} = \frac{z}{\hbar \omega \beta - z}~.
        \end{equation*}
    \end{proof}
    The equation of state is 
    \begin{equation*}
        p = - \frac{1}{\beta V}\ln (1 - \frac{z}{\hbar \beta \omega}) ~.
    \end{equation*}
    \begin{proof}
        By the second of~\eqref{td:es:o} and~\eqref{gc:o}, we have
        \begin{equation*}
            p = - \pdv{\Omega}{V} = - \frac{1}{\beta V} \ln \mathcal Z = \frac{1}{\beta V} \ln (1 - \frac{z}{\hbar \beta \omega}) ~.
        \end{equation*}
    \end{proof}

\section{Canonical ultra-relativistic ideal gas}

    \begin{exercise}
        Consider an ultra-relativistic ($p_i c \gg m c^2$) ideal (non-interacting) gas of $N$ indistinguishable particles with mass $m$, living in an $3$-dimensional manifold with a finite volume $V^N$: $\mathcal M^N = V^N \times \mathbb R^{3N}$, with Hamiltonian
        \begin{equation*}
            H = \sum_i \sqrt{m^2 c^4 + p_i^2 c^2} \xrightarrow{p_i c \gg m c^2} \sum_i c |p_i| ~.
        \end{equation*}
        Find the canonical partition function $Z$, the internal energy $E$, the Helmholtz partition function $F$, the entropy $S$ (and the temperature $T_c$ under which it becomes negative), the equation of state, the chemical potential $\mu$ and the specific heats $C_V$ and $C_p$. 
    \end{exercise}

    The canonical partition function $Z$ is 
    \begin{equation*}
        Z = \frac{1}{N!} \Big (\frac{8\pi V}{(\beta h c)^3} \Big )^N ~.
    \end{equation*}
    \begin{proof}
        By definition~\eqref{c:z}, using the gaussian integral~\eqref{app:gauss}, we have
        \begin{equation*}
        \begin{aligned}
            Z & = \int_{\mathcal M^N} d\Omega \exp(- \beta H (q_i, p_i)) = \int_{\mathcal M^N} \frac{\prod_i d^3 q_i d^3 p_i}{h^{3N} N!} \exp(- \beta H (q_i, p_i)) \\ & = \frac{1}{h^{3N} N!} \int_{\mathcal M^N} \prod_i d^3 q_i d^3 p_i \exp(- \beta H (q_i, p_i)) \\ & = \frac{1}{h^{3N} N!} \underbrace{\int_{ V^N} \prod_i d^d q_i}_{V^N} \prod_i \int_{\mathcal M^N} d^d p_i \exp(- \beta c p_i)  = \frac{V^N}{h^{3N} N!} \prod_i \int_{\mathcal M^N} d^d p_i \exp(- \beta c p_i) ~.
        \end{aligned}
        \end{equation*}
        Now, in order to evaluate the integral, we use the polar coordinates in the momentum space $(p, \theta, \phi) = (p, \Omega)$ 
        \begin{equation*}
            \prod_i \int_{\mathcal M^N} d^3 p_i \exp(- \beta c p_i) = \prod_i 4 \pi \int_0^\infty dp ~ p^2 \exp(- \beta c p_i)
        \end{equation*}
        where $\int d\Omega = 4\pi$. We make a change of variables into
        \begin{equation*}
            z = \beta c p ~, \quad dz = - \beta c dp ~,
        \end{equation*}
        and we obtain 
        \begin{equation*}
            \prod_i \frac{4\pi}{(\beta c)^3} \underbrace{\int_0^\infty dz ~ z^2 \exp(- z)}_{\Gamma (3)} = \prod_i \frac{4\pi}{(\beta c)^3} \underbrace{\Gamma (3)}_2 = \prod_i \frac{8\pi}{(\beta c)^3} = \Big (\frac{8\pi}{(\beta c)^3} \Big )^N ~.
        \end{equation*}
        Hence, we find 
        \begin{equation*}
            Z = \frac{V^N}{h^{3N} N!} \Big (\frac{8\pi}{(\beta c)^3} \Big )^N = \frac{1}{N!} \Big (\frac{8\pi V}{(\beta h c)^3} \Big )^N ~.
        \end{equation*}
    \end{proof}

    An useful intermediary formula is 
    \begin{equation*}
        \ln Z = N (1 - \ln \frac{n (\beta h c)^3}{8\pi}) ~.
    \end{equation*}
    \begin{proof}
        In fact, using the Stirling approximation~\eqref{app:stirl}, we have
        \begin{equation*}
        \begin{aligned}
            \ln Z & = \ln \frac{1}{N!} \Big (\frac{8\pi V}{(\beta h c)^3} \Big )^N = - \underbrace{\ln N!}_{N \ln N - N} + N \ln \frac{8\pi V}{(\beta h c)^3} \\ & = N (1 - \ln \frac{N (\beta h c)^3}{8\pi V}) = N (1 - \ln \frac{n (\beta h c)^3}{8\pi}) ~.
        \end{aligned}
        \end{equation*}
    \end{proof}
    
    The internal energy $E$ is 
    \begin{equation*}
        E = 3 N k_B T ~.
    \end{equation*}
    \begin{proof}
        Using~\eqref{c:e2}
        \begin{equation*}
            E = - \pdv{\ln Z}{\beta} = - \pdv{}{\beta} N (1 - \ln \frac{n (\beta h c)^3}{8\pi}) = N \pdv{}{\beta} \ln (\beta^3) = 3 N \frac{\beta^2}{\beta^3} = 3 N \frac{1}{\beta} = 3 N k_B T ~.
        \end{equation*}

        As an aside, it can be also derived from the generalised equipartion theorem~\eqref{equi:thm}. In fact, we have
        \begin{equation*}
            k_B T = \av{p_i \pdv{H}{p_i}} = \av{p_i \pdv{}{p_i} c \sqrt{p_1^2 + p_2^2 + p_3^2}} = \av{c \frac{p_i^2}{\sqrt{p_1^2 + p_2^2 + p_3^2}}} ~,
        \end{equation*}
        hence, we find
        \begin{equation*}
            \av{H} = \av{c \frac{p_1^2 + p_2^2 + p_3^2}{\sqrt{p_1^2 + p_2^2 + p_3^2}}} = \sum_{i=1}^3 \underbrace{\av{c \frac{p_i^2}{\sqrt{p_1^2 + p_2^2 + p_3^2}}}}_{k_B T} = 3 k_B T ~.
        \end{equation*}
    \end{proof}
    
    The Helmholtz free energy $F$ is 
    \begin{equation*}
        F = \frac{N}{\beta} (\ln \frac{n (\beta h c)^3}{8\pi} - 1) ~.
    \end{equation*}
    \begin{proof}
        Using~\eqref{c:f}, we have
        \begin{equation*}
            F = - \frac{\ln Z}{\beta} = \frac{N}{\beta} (\ln \frac{n (\beta h c)^3}{8\pi} - 1) ~.
        \end{equation*}
    \end{proof}
    
    The entropy $S$ is 
    \begin{equation*}
        S = N k_B (4 - \ln \frac{n (\beta h c)^3}{8\pi} ) ~.
    \end{equation*}
    \begin{proof}
        Using~\eqref{c:s}
        \begin{equation*}
        \begin{aligned}
            S & = \frac{E - F}{T} = \frac{1}{T} \Big ( 3 N k_B T - \frac{N}{\beta} (\ln \frac{n (\beta h c)^3}{8\pi} - 1)  \Big ) \\ & = 3 N k_B - N k_B (\ln \frac{n (\beta h c)^3}{8\pi} - 1) = N k_B (4 - \ln \frac{n (\beta h c)^3}{8\pi} ) ~.
        \end{aligned}
        \end{equation*}
    \end{proof}
    Entropy becomes negative at a certain critical temperature
    \begin{equation*}
        T_c = \frac{hc}{k_B} \Big (\frac{n}{8\pi e^4} \Big)^{1/3} ~.
    \end{equation*}
    A plot of the entropy $S$ as a function of $T$ is in Figure~\ref{fig:c:ent3}.
    \begin{figure}
        \centering
        \scalebox{0.7}{\pyc{plot1('x', '4 - log(1 / x**3)', 3, 5, 4, True, False, False)}}
        \caption{A plot of the entropy $S$ as a function of $T$. We have used $x = \frac{(8 \pi)^{1/3} k_B T}{h c n^{1/3}}$ and $f(x) = \frac{S}{N k_B}$.}
        \label{fig:c:ent3}
    \end{figure}
    \begin{proof}
        In fact, after a series of manipulations, $S < 0$ for 
        \begin{equation*}
            N k_B (4 - \ln \frac{n (\beta h c)^3}{8\pi} ) < 0 ~, \quad  4 - \ln \frac{n (\beta h c)^3}{8\pi} < 0 ~, \quad 4 < \ln \frac{n (\beta h c)^3}{8\pi} ~,
        \end{equation*}
        \begin{equation*}
             \quad e^{4} < \frac{n (\beta h c)^3}{8\pi} ~, \quad e^{4} < \frac{n (h c)^3}{8\pi k_B^3 T^3}  ~, \quad T^3 < \frac{n (h c)^3}{8\pi k_B^3 e^4} ~,
        \end{equation*}
        hence, we find
        \begin{equation*}
            T < \frac{hc}{k_B} \Big (\frac{n}{8\pi e^4} \Big)^{1/3} = T_c ~.
        \end{equation*}
    \end{proof}
    
    The equation of state is 
    \begin{equation}\label{idesultra}
        p V = N k_B T ~.
    \end{equation}
    \begin{proof}
        By the second of~\eqref{td:es:f}, we have
        \begin{equation*}
            p = - \pdv{F}{V} = - \pdv{}{V} \frac{N}{\beta} (\ln \frac{n (\beta h c)^3}{8\pi} - 1) = \frac{N}{\beta} \pdv{}{V} \ln V = \frac{N}{V \beta} ~,
        \end{equation*}
        hence, we find
        \begin{equation*}
            p V = N k_B T ~.
        \end{equation*}
    \end{proof}
    
    The chemical potential $\mu$ is 
    \begin{equation*}
        \mu = \frac{1}{\beta} \ln \frac{n (\beta h c)^3}{8\pi} ~.
    \end{equation*}
    A plot of the chemical potential $\mu$ as a function of $T$ is in Figure~\ref{fig:c:mu3}.
    \begin{figure}
        \centering
        \scalebox{0.7}{\pyc{plot1('x', 'x * log(1 / x**3)', 3, 4, 5, True, False, False)}}
        \caption{A plot of the chemical potential $\mu$ as a function of $T$. We have used $x = \frac{(8 \pi)^{1/3} k_B T}{h c n^{1/3}}$ and $f(x) = \frac{(8 \pi)^{1/3} \mu}{h c n^{1/3}}$.}
        \label{fig:c:mu3}
    \end{figure}
    \begin{proof}
        By the third of~\eqref{td:es:f}, we have
        \begin{equation*}
            \mu = \pdv{F}{N} = \pdv{}{N} \frac{N}{\beta} (\ln \frac{n (\beta h c)^3}{8\pi} - 1) = \frac{1}{\beta} (\ln \frac{n (\beta h c)^3}{8\pi} - 1 + 1) = \frac{1}{\beta} \ln \frac{n (\beta h c)^3}{8\pi} ~.
        \end{equation*}
    \end{proof}

    The specific heats $C_V$ and $C_p$ are 
    \begin{equation*}
        C_V = 3 N k_B ~, \quad C_p = 4 N k_B T ~. 
    \end{equation*}
    \begin{proof}
        At $V$ constant, by definition~\eqref{td:cv2}, we find
        \begin{equation*}
            C_V = \pdv{E}{T} = \pdv{}{T} 3 N k_B T = 3 N k_B ~.
        \end{equation*}
        At $p$ constant, using~\eqref{td:cp2} and~\eqref{ides}, we find
        \begin{equation*}
            C_p = C_V + p \pdv{V}{T} = C_V + p \pdv{}{T} \frac{N k_B T}{p} = 3 N k_B + N k_B = 4 N k_B T ~.
        \end{equation*}
    \end{proof}

\section{Canonical magnetic solid}

    \begin{exercise}
        Consider a solid composed by $N$ atoms or molecules with an intrinsic magnetic moment $\boldsymbol \mu$ in an external magnetic field $\mathbf B$. The phase phase is a $2$-dimensional sphere that can be parametrised by $(\theta, \phi)$. The conjugate coordinate and momentum are $q = \phi$ and $p = \cos \theta$. The Hamiltonian is 
        \begin{equation*}
            H = - \sum_i \boldsymbol \mu \cdot \mathbf B = - \mu B \sum_i \cos \theta_i ~,
        \end{equation*}
        where we have oriented $\mathbf B = B \mathbf k$. Find the canonical partition function $Z$, the internal energy $E$, the Helmholtz partition function $F$, the entropy $S$ (and the temperature $T_c$ under which it becomes negative), the magnetisation $\mathbf M$ and the isothermal susceptibility $\chi_\beta$. 
    \end{exercise}

    The canonical partition function $Z$ is 
    \begin{equation*}
        Z = \Big (\frac{4 \pi \sinh (\beta \mu B)}{\beta \mu B} \Big )^N ~.
    \end{equation*}
    \begin{proof}
        By definition~\eqref{c:z}, we have
        \begin{equation*}
        \begin{aligned}
            Z & = \int_{\mathcal M} d \Omega \exp(- \beta H (\theta_i)) = \underbrace{\prod_i \int_0^{2\pi} d\phi_i}_{(2\pi)^N} \prod_i \int_0^\pi d\theta_i ~ \sin \theta_i \exp(- \beta \mu B \cos \theta_i) \\ & = (2\pi)^N \prod_i \int_0^\pi d\theta_i ~ \sin \theta_i \exp(- \beta \mu B \cos \theta_i)  ~.
        \end{aligned}
        \end{equation*}
        Now, in order to evaluate the integral, we make a change of variables into
        \begin{equation*}
            x_i  = \cos \theta_i ~, \quad d x_i = - \sin \theta_i d\theta_i ~,
        \end{equation*}
        with extrema 
        \begin{equation*}
            \theta_i = 0 \rightarrow x_i = 1 ~, \quad \theta_i = \pi \rightarrow x_i = 0 ~,
        \end{equation*}
        hence,
        \begin{equation*}
        \begin{aligned}
            Z & = (2\pi)^N \prod_i \int_{-1}^1 dx_i ~ \exp(- \beta \mu B x) = (2\pi)^N \Big ( \frac{\exp(- \beta \mu B x)}{- \beta \mu B} \Big \vert_{-1}^{1} \Big )^N \\ & = (2\pi)^N \Big ( \frac{1}{- \beta \mu B} \underbrace{(\exp(- \beta \mu B) - \exp(\beta \mu B))}_{- 2 \sinh \beta \mu B}\Big )^N \\ & = (2\pi)^N \Big ( \frac{1}{\beta \mu B} (2 \sinh (\beta \mu B)) \Big )^N = \Big (\frac{4 \pi \sinh (\beta \mu B)}{\beta \mu B} \Big )^N ~.
        \end{aligned}
        \end{equation*}
    \end{proof}

    An useful intermediary formula is 
    \begin{equation*}
        \ln Z = N \ln \frac{4 \pi \sinh (\beta \mu B)}{\beta \mu B}  ~.
    \end{equation*}
    \begin{proof}
        In fact, we have
        \begin{equation*}
            \ln Z = \ln \Big (\frac{4 \pi \sinh (\beta \mu B)}{\beta \mu B} \Big )^N = N \ln \frac{4 \pi \sinh (\beta \mu B)}{\beta \mu B}  ~.
        \end{equation*}
    \end{proof}
    
    The internal energy $E$ is 
    \begin{equation*}
        E = - N \mu B (\coth (\beta \mu B) - \frac{1}{\beta \mu B} ) ~.
    \end{equation*}
    A plot of the internal energy $E$ as a function of $T$ is in Figure~\ref{fig:c:magen}.
    \begin{figure}
        \centering
        \scalebox{0.7}{\pyc{plot1('x', '- coth(1 / x) + x', 5, 1.2, 8, True, True, False)}}
        \caption{A plot of the internal energy $E$ as a function of $T$. We have used $x = \frac{1}{\beta \mu B}$ and $f(x) = \frac{E}{N \mu B}$.}
        \label{fig:c:magen}
    \end{figure}
    \begin{proof}
        Using~\eqref{c:e2}, we have
        \begin{equation*}
        \begin{aligned}
            E & = - \pdv{\ln Z}{\beta} = - \pdv{}{\beta} N \ln \frac{4 \pi \sinh (\beta \mu B)}{\beta \mu B} = - N \pdv{}{\beta} \ln \sinh (\beta \mu B) + N \pdv{}{\beta} \ln \beta \\ & = - N \mu B \coth (\beta \mu B) + \frac{N}{\beta} = - N \mu B (\coth (\beta \mu B) - \frac{1}{\beta \mu B} ) ~.
        \end{aligned}
        \end{equation*}

        We compute the limit for $\beta \rightarrow 0$ or $T \rightarrow \infty$
        \begin{equation*}
            \lim_{x \rightarrow 0} \frac{E}{N \mu B} (x) \simeq \py{limit('x', '- 1 * ((exp(x) + exp(-x))/((exp(x) - exp(-x)))) + 1 / x', 0)} ~,
        \end{equation*}
        hence, we obtain
        \begin{equation*}
            E \xrightarrow{T \rightarrow \infty} 0 ~.
        \end{equation*}
        We compute the limit for $\beta \rightarrow \infty$ or $T \rightarrow 0$
        \begin{equation*}
            \lim_{x \rightarrow \infty} \frac{E}{N \mu B} (x) \simeq \py{limit('x', '- coth(1 / x) + x', 0)} ~,
        \end{equation*}
        hence, we obtain
        \begin{equation*}
            E \xrightarrow{T \rightarrow 0} - N \mu B ~.
        \end{equation*}
    \end{proof}
    
    The Helmholtz free energy $F$ is 
    \begin{equation*}
        F = - \frac{N}{\beta} \ln \frac{4 \pi \sinh (\beta \mu B)}{\beta \mu B} ~.
    \end{equation*}
    \begin{proof}
        Using~\eqref{c:f}, we have
        \begin{equation*}
            F = - \frac{\ln Z}{\beta} = - \frac{N}{\beta} \ln \frac{4 \pi \sinh (\beta \mu B)}{\beta \mu B} ~.
        \end{equation*}
    \end{proof}
    
    The entropy $S$ is 
    \begin{equation*}
        S = N k_B \Big ( \ln \frac{4 \pi \sinh (\beta \mu B)}{\beta \mu B}  - \beta \mu B (\coth (\beta \mu B) - \frac{1}{\beta \mu B} ) \Big ) ~.
    \end{equation*}
    \begin{proof}
        Using~\eqref{c:s}, we have
        \begin{equation*}
        \begin{aligned}
            S & = \frac{E - F}{T} = \frac{1}{T} \Big (- N \mu B (\coth (\beta \mu B) + \frac{1}{\beta \mu B} ) + \frac{N}{\beta} \ln \frac{4 \pi \sinh (\beta \mu B)}{\beta \mu B}  \Big ) \\ & = N k_B \Big ( \ln \frac{4 \pi \sinh (\beta \mu B)}{\beta \mu B}  - \beta \mu B (\coth (\beta \mu B) - \frac{1}{\beta \mu B} ) \Big )~. 
        \end{aligned}
        \end{equation*}
    \end{proof}

    An useful intermediary formula is 
    \begin{equation}\label{c:mag}
        \mathbf M = - \pdv{F}{\mathbf B} \Big \vert_{\beta} ~.
    \end{equation}
    \begin{proof}
        In fact, using~\eqref{c:av} and~\eqref{c:f}, we obtain 
        \begin{equation*}
        \begin{aligned}
            \mathbf M & = \sum_i \av{\boldsymbol \mu_i}_c = \int d\Omega ~ \rho_c \sum_i \boldsymbol \mu_i  = \int d\Omega ~ \frac{\exp(\beta \sum_i \boldsymbol \mu_i \cdot \mathbf B)}{Z} \sum_i \boldsymbol \mu_i \\ & = \frac{1}{\beta Z} \pdv{}{\mathbf B} \underbrace{\int d\Omega ~ \exp(\beta \sum_i \boldsymbol \mu_i \cdot \mathbf B)}_Z = \frac{1}{\beta Z} \pdv{Z}{\mathbf B} = \frac{1}{\beta} \pdv{\ln Z}{\mathbf B} = - \pdv{F}{\mathbf B} \Big \vert_{\beta} ~.
        \end{aligned}
        \end{equation*}
    \end{proof}

    The magnetisation $\mathbf M$ is 
    \begin{equation*}
        \mathbf M = (0, 0, N\mu (\coth(\beta \mu B) - \frac{1}{\beta \mu B})) ~.
    \end{equation*}
    A plot of the magnetisation $\mathbf M$ as a function of $\beta$ is in Figure~\ref{fig:c:mag}.
    \begin{figure}
        \centering
        \scalebox{0.7}{\pyc{plot1('x', 'coth(x) - 1 / x', 7, 1, 9, True, True, True)}}
        \caption{A plot of the intrinsic magnetic moment $\mathbf M$ as a function of $\beta$. We have used $x = \beta \mu B$ and $f(x) = \frac{M_z}{N \mu}$.}
        \label{fig:c:mag}
    \end{figure}
    \begin{proof}
        Since we have oriented $\mathbf B = (0, 0, B)$, in the transversal directions we have
        \begin{equation*}
            M_x = - \pdv{F}{B_x} = M_y = - \pdv{F}{B_y} = 0 ~,
        \end{equation*}
        but in the longitudinal direction we find
        \begin{equation*}
        \begin{aligned}
            M_z & = - \pdv{F}{B} = \pdv{}{B} \frac{N}{\beta} \ln \frac{4 \pi \sinh (\beta \mu B)}{\beta \mu B} = \frac{N}{\beta} \pdv{}{\beta} \ln \sinh (\beta \mu B) - \frac{N}{\beta} \pdv{}{B} \ln B \\ & = N \mu \coth(\beta \mu B) - \frac{N}{\beta B} = N \mu \Big (\coth (\beta \mu B) - \frac{1}{\beta \mu B} \Big ) ~.
        \end{aligned}
        \end{equation*}
    \end{proof}
    
    The isothermal susceptibility $\chi_\beta$, defined as 
    \begin{equation*}
        \chi_\beta = \pdv{M}{B} \Big \vert_\beta ~,
    \end{equation*}  
    is 
    \begin{equation*}
        \chi_\beta = N \mu^2 \beta ( \frac{1}{(\beta \mu B)^2} - \frac{1}{\sinh^2 (\beta \mu H)}) ~.
    \end{equation*}
    A plot of the isothermal susceptibility $\chi_\beta$, as a function of $\beta$ is in Figure~\ref{fig:c:sus}.
    \begin{figure}
        \centering
        \scalebox{0.7}{\pyc{plot1('x', 'x * (1 / x**2 - 1 / sinh(x)**2)', 5, 0.5, 10, True, True, True)}}
        \caption{A plot of the isothermal susceptibility $\chi_\beta$, as a function of $\beta$. We have used $x = \beta \mu B$ and $f(x) = \frac{B \chi_\beta}{N \mu}$.}
        \label{fig:c:sus}
    \end{figure}
    \begin{proof}
        By definition, we have
        \begin{equation*}
        \begin{aligned}
            \chi_\beta & = \pdv{M}{B} = \pdv{}{B} N\mu (\coth(\beta \mu B) - \frac{1}{\beta \mu B}) = N \mu ( - \frac{\beta \mu}{\sinh^2 (\beta \mu B)} + \frac{\beta \mu}{(\beta \mu B)^2} ) \\ & = N \mu^2 \beta ( \frac{1}{(\beta \mu B)^2} - \frac{1}{\sinh^2 (\beta \mu H)}) ~.
        \end{aligned}
        \end{equation*}
    \end{proof}
    
    For $T \rightarrow \infty$, we can recover the Curie law 
    \begin{equation*}
        \chi_\beta = \frac{C}{T} ~,
    \end{equation*}
    where the Curie constant is 
    \begin{equation*}
        C = \frac{N \mu^2}{3 k_B} ~.
    \end{equation*}
    \begin{proof}
        To study the limit for $\beta \rightarrow 0$ or $T \rightarrow \infty$, we Taylor expand for the variable $x = \beta \mu B$
        \begin{equation*}
            \frac{B \chi_\beta}{N \mu} (x) \simeq \py{Taylor('x', 'x * ( x**(-2) - sinh(x)**(-2))', 0, 2)} ~,
        \end{equation*}
        hence, we find
        \begin{equation*}
            \frac{B \chi_\beta}{N \mu} = \frac{\beta \mu B}{3} ~,
        \end{equation*}
        which means that
        \begin{equation*}
            \chi_\beta = \frac{N \mu^2}{3 k_B} \frac{1}{T} = \frac{C}{T} ~.
        \end{equation*}
    \end{proof}

\section{Canonical Maxwell-Boltzmann distribution}

    \begin{exercise}
        Consider a non-relativistic ideal (non-interacting) gas of $N$ particles in an $3$-dimensional manifold $\mathcal M^N = \mathbb R^6$ confined into a potential $V(q_i)$, with Hamiltonian
        \begin{equation*}
            H = \sum_i \Big ( \frac{p^2_i}{2m} + V(q_i) \Big ) ~.
        \end{equation*}
        Find the probability distribution density $\rho_c$, the marginal $\rho(q^i)$, the momentum $\rho(p_i)$, the velocity $\rho(v_i$) and the square velocity $\rho(v)$ ones. Furthermore, compute the most probable velocity $v_p$, the mean velocity $\av{v}$ and the e mean square velocity $\av{v^2}$. Put $h = 1$. 
    \end{exercise}

    The probability distribution density $\rho_c$ for each particle is 
    \begin{equation*}
        \rho_c (q_i, p_i) = \frac{\exp (- \beta ( \frac{p^2_i}{2m} + V(q_i) ))}{(\frac{2\pi m}{\beta})^{3/2} \int_{\mathbb R^3} d^3 q \exp(- \beta V(q))} ~.
    \end{equation*}
    \begin{proof}
        By definition~\eqref{c:pdd}, we have
        \begin{equation*}
            \rho_c (q_i, p_i) = \mathcal N \exp (- \beta ( \frac{p^2_i}{2m} + V(q_i) )) ~,
        \end{equation*}
        where, using the gaussian integral~\eqref{app:gauss}, the normalisation constant is  
        \begin{equation*}
        \begin{aligned}
            1 & = \int_{\mathbb R^6} \prod_i d^3 q ~ d^3 p \mathcal N \exp(- \beta ( \frac{p^2}{2m} + V(q))) \\ & = \mathcal N \int_{\mathbb R^3} d^3 q ~ \exp(- \beta V(q))  \underbrace{\int_{\mathbb R^3} d^3 p ~ \exp(- \beta \frac{p^2}{2m})}_{\Big ( \frac{2\pi m}{\beta}\Big)^{3/2}} \\ & = \mathcal N  \Big ( \frac{2\pi m}{\beta}\Big)^{3/2} \int_{\mathbb R^3} d^3 q ~ \exp(- \beta V(q)) ~,
        \end{aligned}
        \end{equation*}
        hence, we find
        \begin{equation*}
            \mathcal N = \Big ( \Big ( \frac{2\pi m}{\beta}\Big)^{3/2} \int_{\mathbb R^3} d^3 q ~ \exp(- \beta V(q))  \Big )^{-1} ~.
        \end{equation*}
    \end{proof}

    The marginal probability density distribution is 
    \begin{equation*}
        \rho(q_i) = \frac{\exp (- \beta V(q_i) )}{\int_{\mathbb R^3} d^3 q \exp(- \beta V(q))} ~.
    \end{equation*}
    \begin{proof}
        By definition, we have
        \begin{equation*}
        \begin{aligned}
            \rho(q_i) & = \int_{\mathbb R^3} d^3 p ~ \rho_c (q_i, p) = \int_{\mathbb R^3} d^3 p ~ \frac{\exp (- \beta ( \frac{p^2}{2m} + V(q_i) ))}{(\frac{2\pi m}{\beta})^{3/2} \int_{\mathbb R^3} d^3 q \exp(- \beta V(q))} \\ & = \frac{\exp (- \beta V(q_i) )}{\int_{\mathbb R^3} d^3 q \exp(- \beta V(q))} \frac{\cancel{\int_{\mathbb R^3} d^3 p ~ \exp(- \beta \frac{p^2}{2m})}}{\cancel{(\frac{2\pi m}{\beta})^{3/2}}} = \frac{\exp (- \beta V(q_i) )}{\int_{\mathbb R^3} d^3 q \exp(- \beta V(q))} ~.
        \end{aligned}
        \end{equation*}
    \end{proof}

    If we have a potential defined as 
    \begin{equation*}
        V (q) = \begin{cases}
            0 & \textnormal{inside a region } \mathcal A\\
            \infty & \textnormal{outside a region } \mathcal A\\
        \end{cases} ~,
    \end{equation*}
    the probability is null outside this region and uniform inside it. 
    \begin{proof}
        In fact, we find
        \begin{equation*}
            \rho(q_i) = \frac{1}{\int_{\mathcal A} d^3 q} = \frac{1}{\mathcal A} ~. 
        \end{equation*}
    \end{proof}

    The momentum probability density distribution is 
    \begin{equation*}
        \rho(p_i) = (2\pi m k_B T)^{-3/2} \exp(- \beta \frac{p^2}{2m}) = \prod_i (2\pi m k_B T)^{-1/2} \exp(- \beta \frac{p^2_i}{2m}) ~.
    \end{equation*}
    A plot of the momentum probability density distribution is in Figure~\ref{fig:mb:mom}.
    \begin{figure}
        \centering
        \scalebox{0.7}{\pyc{plot1('x', 'exp(- x**2)', 3, 2, 6, True, False, True)}}
        \caption{A plot of the momentum probability density distribution. We have used $x = \sqrt{\frac{\beta}{2m}} p$ and $f(x) = (2\pi m k_B T)^{3/2} \rho$.}
        \label{fig:mb:mom}
    \end{figure}
    \begin{proof}
        By definition, we have
        \begin{equation*}
        \begin{aligned}
            \rho(p_i) & = \int_{\mathbb R^3} d^3 q ~ \rho_c (q, p) = \int_{\mathbb R^3} d^3 q ~ \frac{\exp (- \beta ( \frac{p^2}{2m} + V(q) ))}{(\frac{2\pi m}{\beta})^{3/2} \int_{\mathbb R^3} d^3 q' \exp(- \beta V(q'))} \\ & = \frac{\exp(- \beta \frac{p^2}{2m})}{(\frac{2\pi m}{\beta})^{3/2}} \frac{\cancel{\int_{\mathbb R^3} d^3 q ~ \exp (- \beta V(q) )}}{\cancel{\int_{\mathbb R^3} d^3 q' \exp(- \beta V(q'))}} = \frac{\exp(- \beta \frac{p^2}{2m})}{(\frac{2\pi m}{\beta})^{3/2}} \\ & = (2\pi m k_B T)^{-3/2} \exp(- \beta \frac{p^2}{2m}) = \prod_i (2\pi m k_B T)^{-1/2} \exp(- \beta \frac{p^2_i}{2m}) ~.
        \end{aligned}
        \end{equation*}
    \end{proof}
    
    The velocity probability density distribution is
    \begin{equation*}
        \rho(v_i) = (\frac{m}{2\pi k_B T})^{1/2} \exp(- \beta \frac{m v^2_i}{2}) ~.
    \end{equation*}
    \begin{proof}
        In order to have velocities instead of momenta, we make a change of variables into
        \begin{equation*}
            p_i = m v_i ~, \quad \rho(v_i) dv_i = \rho(p_i) dp_i = \rho(p_i) m dv_i ~,
        \end{equation*}
        hence, we find
        \begin{equation*}
            \rho(v_i) = m \rho (p_i) = (\frac{2\pi k_B T}{m})^{-1/2} \exp(- \beta \frac{m^2 v^2_i}{2m}) = (\frac{m}{2\pi k_B T})^{1/2} \exp(- \beta \frac{m v^2_i}{2}) ~.
        \end{equation*}
    \end{proof}

    The velocity modulus probability density distribution is 
    \begin{equation*}
        \rho (v) = (\frac{m}{2\pi k_B T})^{3/2} 4 \pi v^2 \exp(- \beta \frac{m v^2}{2}) ~.
    \end{equation*}
    A plot of the velocity modulus probability density distribution is in Figure~\ref{fig:mb:vel}.
    \begin{figure}
        \centering
        \scalebox{0.7}{\pyc{plot1('x', 'x**2 * exp(- x**2)', 3, 0.5, 7, True, True, True)}}
        \caption{A plot of the velocity modulus probability density distribution. We have used $x = \sqrt{\frac{\beta m}{2}} v$ and $f(x) = \rho$.}
        \label{fig:mb:vel}
    \end{figure}
    \begin{proof}
        In order to have square modulus velocities, we make a change of variable into the polar coordinates $(v, \theta, \phi)$
        \begin{equation*}
            \rho(v_1, v_2, v_3) dv_1 dv_2 dv_3 = \rho(v_1, v_2, v_3) v^2 \sin \theta d\theta d\phi dv = \rho(\theta, \phi, v) d\theta d\phi dv ~,
        \end{equation*}
        hence, we find
        \begin{equation*}
            \rho(v) = 4 \pi v^2 \prod_i \rho (v_i) = (\frac{m}{2\pi k_B T})^{3/2} 4 \pi v^2 \exp(- \beta \frac{m v^2}{2})  ~.
        \end{equation*}
    \end{proof}
    
    The most probable velocity value is 
    \begin{equation*}
        v_p = \sqrt{\frac{2 k_B T}{m}} ~.
    \end{equation*}
    \begin{proof}
        By definition, we have
        \begin{equation*}
            0 = \dv{\rho(v)}{v} = 2 v \exp(- \beta \frac{m v^2}{2}) - \beta m v^3 \exp(- \beta \frac{m v^2}{2}) ~,
        \end{equation*}
        hence, we find
        \begin{equation*}
            v_p = \sqrt{\frac{2k_B T}{m}} ~.
        \end{equation*}
    \end{proof}

    The mean velocity value is 
    \begin{equation*}
        \av{v} = \sqrt{\frac{8 k_B T}{\pi m}} ~.
    \end{equation*}
    \begin{proof}
        By definition, we have
        \begin{equation*}
            \av{v} = \int_{\mathbb R^3} dv ~ \rho(v) v = (\frac{m}{2\pi k_B T})^{3/2} 4 \pi \int_{\mathbb R^3} dv ~ v^3 \exp(- \beta \frac{m v^2}{2}) ~.
        \end{equation*}
        Now, in order to evaluate the integral, we make a change of variables into
        \begin{equation*}
            t = \frac{m \beta v^2}{2} ~, \quad dt = m \beta v dv ~,
        \end{equation*}
        hence, we find
        \begin{equation*}
        \begin{aligned}
            \av{v} & =  (\frac{m}{2\pi k_B T})^{3/2} 4 \pi \Big (\frac{2}{m \beta} \Big) \frac{1}{m \beta} \underbrace{\int_0^\infty dt ~ t \exp(- t)}_{\Gamma (2)} \\ & = \sqrt{\frac{8}{m \pi \beta}} \underbrace{\Gamma (2)}_1 = \sqrt{\frac{8}{m \pi \beta}} = \sqrt{\frac{8 k_B T}{\pi m}} ~.
        \end{aligned}
        \end{equation*}
    \end{proof}

    The mean square velocity value is 
    \begin{equation*}
        \av{v^2} = \frac{3 k_B T}{m} ~.
    \end{equation*}
    \begin{proof}
        By definition, we have
        \begin{equation*}
            \av{v^2} = \int_{\mathbb R^3} dv ~ \rho(v) v^2 = (\frac{m}{2\pi k_B T})^{3/2} 4 \pi \int_{\mathbb R^3} dv ~ v^4 \exp(- \beta \frac{m v^2}{2}) ~.
        \end{equation*}
        Now, in order to evaluate the integral, we make a change of variables into
        \begin{equation*}
            t = \frac{m \beta v^2}{2} ~, \quad dt = m \beta v dv ~,
        \end{equation*}
        hence, we find
        \begin{equation*}
        \begin{aligned}
            \av{v^2} & =  (\frac{m}{2\pi k_B T})^{3/2} 4 \pi \Big (\frac{2}{m \beta} \Big) \frac{1}{m \beta} \Big ( \frac{2}{m \beta} \Big)^{1/2} \underbrace{\int_0^\infty dt ~ t^{3/2} \exp(- t)}_{\Gamma (5/2)} \\ & = \frac{4}{\sqrt{\pi} m \beta} \underbrace{\Gamma (5/2)}_{\frac{3 \sqrt{\pi}}{4}} = \frac{3}{\beta m} = \frac{3 k_B T}{m} ~.
        \end{aligned}
        \end{equation*}
    \end{proof}

\section{Entropic Maxwell-Boltzmann distribution}

    \begin{exercise}
        Consider $N$ distinguishable particles. Find its probability distribution, using the method of counting of states.
    \end{exercise}

    We have to evaluate the two terms of~\eqref{e:count}. For the first, we can distribute $N$ particle in $p$ boxes in the following way 
    \begin{equation*}
        W^{(1)}_{n_r} = \frac{N!}{n_1! \ldots n_p!} ~,
    \end{equation*}
    whereas, for the second, there is no restriction for the states 
    \begin{equation*}
        W^{(2)}_{n_r} = \prod_r g_r^{n_r} ~,
    \end{equation*}
    hence, we obtain
    \begin{equation*}
        W_{n_r} = \frac{N!}{n_1! \ldots n_p!} \prod_{r=1}^p g_r^{n_r} = N! \prod_{r=1}^p \frac{g_r^{n_r}}{n_r!} ~.
    \end{equation*}

    Maximising the constrained entropy, we find the Boltzmann distribution 
    \begin{equation*}
        p_r^* = \frac{n_r^*}{N} = \frac{g_r \exp(- \beta E_r)}{\sum_r g_r \exp(- \beta E_r)} ~.
    \end{equation*}
    A plot of the Boltzmann distribution $p_r^*$ as a function of $\beta E_r$ is in Figure~\ref{en:bol}.
    \begin{figure}
        \centering
        \scalebox{0.7}{\pyc{plot1('x', 'exp(-x)', 5, 10, 12, True, False, True)}}
        \caption{A plot of the probability density distribution $p_r^*$ as a function of $\beta E_r$. We have used $x = \beta E_r$ and $f(x) = p_r^* \frac{\sum_r g_r \exp(- \beta E_r)}{g_r}$.}
        \label{en:bol}
    \end{figure}
    \begin{proof}
        By definition~\eqref{e:shannon2}, using the Stirling approximation~\eqref{app:stirl}, we have
        \begin{equation*}
        \begin{aligned}
            S & = \ln W_{n_r} = \ln \Big (N! \prod_{r=1}^p \frac{g_r^{n_r}}{n_r!} \Big) = \ln N! + \sum_{r=1}^p \ln \frac{g_r^{n_r}}{n_r!} = \underbrace{\ln N!}_{N \ln N - N} + \sum_{r=1}^p (\ln g_r^{n_r} - \underbrace{\ln n_r!}_{n_r \ln n_r - n_r} ) \\ & = N \ln N - \cancel{N} + \sum_{r=1}^p n_r \ln g_r - \sum_{r=1}^p n_r \ln n_r - \cancel{\sum_{r=1}^p n_r} = N \ln N + \sum_{r=1}^p n_r \ln g_r + \sum_{r=1}^p n_r \ln n_r ~.
        \end{aligned}
        \end{equation*}
        Adding the constraints~\eqref{e:constrain}, we have
        \begin{equation*}
            S =  N \ln N + \sum_{r=1}^p n_r \ln g_r - \sum_{r=1}^p n_r \ln n_r + \alpha \Big (N - \sum_{r=1}^p n_r \Big) + \beta \Big (E - \sum_{r=1}^p n_r E_r \Big ) ~.
        \end{equation*}
        Computing the maximum by putting the derivative to zero, we obtain
        \begin{equation*}
            0 = \pdv{S}{n_r} = \ln g_r - \ln n_r - 1 - \alpha - \beta E_r ~,
        \end{equation*}
        hence, we find
        \begin{equation*}
            n_r^* = \frac{g_r \exp(- \beta E_r)}{\exp(1 + \alpha)} ~.
        \end{equation*}
        We find $\alpha$ by the normalisation condition 
        \begin{equation*}
            N = \sum_r n_r^* = \sum_r \frac{g_r \exp(- \beta E_r)}{\exp(1 + \alpha)} ~,
        \end{equation*}
        hence, we find
        \begin{equation*}
            \exp(1 + \alpha) = \frac{\sum_r g_r \exp(- \beta E_r)}{N} ~.
        \end{equation*}
        Finally, if we identify $\beta = 1/k_B T$, we obtain
        \begin{equation*}
            p_r^* = \frac{n_r^*}{N} = \frac{g_r \exp(- \beta E_r)}{\sum_r g_r \exp(- \beta E_r)} ~.
        \end{equation*}
    \end{proof}
    
\section{Entropic Fermi-Dirac distribution}

    \begin{exercise}
        Consider $N$ fermionic indistinguishable particles. Find its probability distribution, using the method of counting of states.
    \end{exercise}

    We have to evaluate the two terms of~\eqref{e:count}. For the first, there is no restriction on how we can distribute $N$ particle in $p$ boxes in ways, since they are indistinguishable,
    \begin{equation*}
        W^{(1)}_{n_r} = 1 ~,
    \end{equation*}
    whereas, for the second, we can distribute $n_r$ objects in $g_r$ boxes in the following way
    \begin{equation*}
        W^{(2)}_{n_r} = \prod_r \binom{g_r}{n_r} = \prod_r \frac{g_r!}{n_r! (g_r - n_r)!} ~,
    \end{equation*}
    hence, we obtain
    \begin{equation*}
        W_{n_r} = \prod_r \binom{g_r}{n_r} = \prod_r \frac{g_r!}{n_r! (g_r - n_r)!} ~.
    \end{equation*}
    
    Maximising the constrained entropy, we find the Bose-Einstein distribution 
    \begin{equation*}
        n_r^* = \frac{g_r}{\exp(\alpha + \beta E_r) + 1} ~.
    \end{equation*}
    A plot of the Fermi-Dirac distribution $n_r^*$ as a function of $\alpha + \beta E_r$ is in Figure~\ref{en:fd}.
    \begin{figure}
        \centering
        \scalebox{0.7}{\pyc{plot1('x', '1 / ( exp(x) + 1)', 5, 2, 13, True, False, True)}}
        \caption{A plot of the Fermi-Dirac distribution $n_r^*$ as a function of $\alpha + \beta E_r$. We have used $x = \alpha + \beta E_r $ and $f(x) = \frac{n_r^*}{g_r}$.}
        \label{en:fd}
    \end{figure}
    \begin{proof}
        By definition~\eqref{e:shannon2}, using the Stirling approximation~\eqref{app:stirl}, we have
        \begin{equation*}
        \begin{aligned}
            S & = \ln W_{n_r} = \ln \Big (\prod_r \frac{g_r!}{n_r! (g_r - n_r)!} \Big) = \sum_r \Big ( \underbrace{\ln g_r!}_{g_r \ln g_r - g_r} - \underbrace{\ln n_r!}_{n_r \ln n_r - n_r} - \underbrace{\ln (g_r - n_r)!}_{(g_r - n_r) \ln (g_r - n_r) - g_r + n_r} \Big) \\ & = \sum_r \Big ( g_r \ln g_r - \cancel{g_r} - n_r \ln n_r + \cancel{n_r} - (g_r - n_r) \ln (g_r - n_r) + \cancel{g_r} - \cancel{n_r} \Big) \\ & = \sum_r \Big ( g_r \ln g_r - n_r \ln n_r - (g_r - n_r) \ln (g_r - n_r) \Big) 
             ~.
        \end{aligned}
        \end{equation*}
        Adding the constraints~\eqref{e:constrain}, we have
        \begin{equation*}
            S =  \sum_r \Big ( g_r \ln g_r - n_r \ln n_r - (g_r - n_r) \ln (g_r - n_r) \Big) + \alpha \Big (N - \sum_{r=1}^p n_r \Big) + \beta \Big (E - \sum_{r=1}^p n_r E_r \Big ) ~.
        \end{equation*}
        Computing the maximum by putting the derivative to zero, we obtain
        \begin{equation*}
        \begin{aligned}
            0 & = \pdv{S}{n_r} = - \ln n_r - \cancel{1} + \ln (g_r - n_r) + \cancel{1} - \alpha - \beta E_r \\ & = - \ln n_r + \ln (g_r - n_r) - \alpha - \beta E_r = \ln (\frac{g_r}{n_r} - 1) - \alpha - \beta E_r~ ,
        \end{aligned}
        \end{equation*}
        hence, we find
        \begin{equation*}
            \frac{g_r}{n_r} - 1 = \exp(\alpha + \beta E_r) ~,
        \end{equation*}
        \begin{equation*}
            n_r^* = \frac{g_r}{\exp(\alpha + \beta E_r) + 1} ~.
        \end{equation*}
    \end{proof}
    
\section{Entropic Bose-Einstein distribution}

    \begin{exercise}
        Consider $N$ bosonic indistinguishable particles. Find its probability distribution, using the method of counting of states.
    \end{exercise}
    
    We have to evaluate the two terms of~\eqref{e:count}. For the first, there is no restriction on how we can distribute $N$ particle in $p$ boxes in ways, since they are indistinguishable,
    \begin{equation*}
        W^{(1)}_{n_r} = 1 ~,
    \end{equation*}
    whereas, for the second, we can distribute $n_r$ objects in $g_r$ boxes in the following way
    \begin{equation*}
        W^{(2)}_{n_r} = \prod_r \binom{n_r + g_r - 1}{n_r} = \prod_r \frac{(n_r + g_r - 1)!}{n_r! (g_r - 1)!} ~,
    \end{equation*}
    hence, we obtain
    \begin{equation*}
        W_{n_r} = \prod_r \frac{(n_r + g_r - 1)!}{n_r! (g_r - 1)!} ~.
    \end{equation*}
    
    Maximising the constrained entropy, we find the Bose-Einstein distribution 
    \begin{equation*}
        n_r^* = \frac{g_r}{\exp(\alpha + \beta E_r) - 1} ~.
    \end{equation*}
    A plot of the Bose-Einstein distribution $n_r^*$ as a function of $\alpha + \beta E_r$  is in Figure~\ref{en:be}.
    \begin{figure}
        \centering
        \scalebox{0.7}{\pyc{plot1('x', '1 / ( exp(x) - 1)', 5, 5, 14, True, False, True)}}
        \caption{A plot of the Bose-Einstein distribution $n_r^*$ as a function of $\alpha + \beta E_r$. We have used $x = \alpha + \beta E_r $ and $f(x) = \frac{n_r^*}{g_r}$.}
        \label{en:be}
    \end{figure}
    \begin{proof}
        By definition~\eqref{e:shannon2}, using the Stirling approximation~\eqref{app:stirl}, we have
        \begin{equation*}
        \begin{aligned}
            S & = \ln W_{n_r} = \ln \prod_r \frac{(n_r + g_r - 1)!}{n_r! (g_r - 1)!} \\ & = \sum_r \Big (\underbrace{\ln (n_r + g_r - 1)!}_{(n_r + g_r - 1) \ln (n_r + g_r - 1) - n_r - g_r + 1} - \underbrace{\ln n_r!}_{n_r \ln n_r - n_r} - \underbrace{\ln (g_r - 1)!}_{(g_r - 1) \ln (g_r - 1) - g_r + 1} \Big ) \\ & = \sum_r \Big ( (n_r + g_r - 1) \ln (n_r + g_r - 1) - \cancel{n_r} - \cancel{g_r} + \cancel{1} \\ & \quad - n_r \ln n_r + \cancel{n_r }- (g_r - 1) \ln (g_r - 1) + \cancel{g_r} - \cancel{1} \Big ) \\ & = \sum_r \Big ( (n_r + g_r - 1) \ln (n_r + g_r - 1) \ln n_r - n_r \ln n_r - (g_r - 1) \ln (g_r - 1) \Big ) 
        \end{aligned}
        \end{equation*}
        Adding the constraints~\eqref{e:constrain}, we have
        \begin{equation*}
        \begin{aligned}
            S & = \sum_r \Big ( (n_r + g_r - 1) \ln (n_r + g_r - 1) \ln n_r - n_r \ln n_r - (g_r - 1) \ln (g_r - 1) \Big ) \\ & \quad + \alpha \Big (N - \sum_{r=1}^p n_r \Big) + \beta \Big (E - \sum_{r=1}^p n_r E_r \Big ) ~.
        \end{aligned}
        \end{equation*}
        Computing the maximum by putting the derivative to zero, we obtain
        \begin{equation*}
        \begin{aligned}
            0 & = \pdv{S}{n_r} = \ln (n_r + g_r - 1) + \cancel{1} - \ln n_r - \cancel{1} - \alpha - \beta E_r \\ & = \ln (n_r + g_r - 1) - \ln n_r - \alpha - \beta E_r = \ln (\frac{g_r - 1}{n_r} + 1) - \alpha - \beta E_r~ ,
        \end{aligned}
        \end{equation*}
        hence, for $g_r \gg 1$, we find
        \begin{equation*}
            \frac{g_r - 1}{n_r} + 1 = \exp(\alpha + \beta E_r) ~,
        \end{equation*}
        \begin{equation*}
            n_r^* = \frac{g_r - 1}{\exp(\alpha + \beta E_r) - 1} \simeq \frac{g_r}{\exp(\alpha + \beta E_r) - 1} ~.
        \end{equation*}
    \end{proof}

\section{Entropic Two-levels system}

    \begin{exercise}
        Consider a system composed by two levels of energies $\epsilon_+ = + \epsilon$ and $\epsilon_- = - \epsilon$, subjected to constrains
        \begin{equation*}
            E = \epsilon (n_+ - n_-) ~, \quad N = n_+ + n_- ~.
        \end{equation*}. Find, using the method of counting of states, the entropy $S$ and the tempretaure $T$.
    \end{exercise}

    Constraints can be inverted to find
    \begin{equation}
        n_+ = \frac{N}{2} + \frac{E}{2\epsilon} ~, \quad n_+ = \frac{N}{2} - \frac{E}{2\epsilon} ~.
    \end{equation}
    We can distribute $N$ objects in $n_+$ boxes in the following way
    \begin{equation*}
        \Omega(E) = \binom{N}{n_+} = \frac{N!}{n_+! (N - n_+)!}  = \frac{N!}{n_+! n_-!} ~.
    \end{equation*}
    
    The entropy $S$ is 
    \begin{equation*}
        S = - N k_B \Big ( (\frac{1}{2} + \frac{E}{2\epsilon N}) \ln (\frac{1}{2} + \frac{E}{2\epsilon N} ) + (\frac{1}{2} - \frac{E}{2\epsilon N}) \ln (\frac{1}{2} - \frac{E}{2\epsilon N} ) \Big ) ~.
    \end{equation*}
    A plot of the entropy $S$ as a function of $E$ is in Figure~\ref{en:s}.
    \begin{figure}
        \centering
        \scalebox{0.7}{\pyc{plot1('x', '-( (1 / 2 + x) * ln (1/2 + x) +  (1 / 2 - x) * ln (1/2 - x) )', 1, 1, 15, True, False, True)}}
        \caption{A plot of the entropy $S$ as a function of $E$. We have used $x = \frac{E}{2 \epsilon N} $ and $f(x) = \frac{S}{N k_B}$.}
        \label{en:s}
    \end{figure}
    \begin{proof}
        By definition~\eqref{mc:s}, using the Stirling approximation~\eqref{app:stirl}, we have
        \begin{equation*}
        \begin{aligned}
            \frac{S}{k_B} & = \ln \Omega (E) = \ln \frac{N!}{n_+! n_-!} = \underbrace{\ln N!}_{N \ln N - N} - \underbrace{\ln n_+!}_{n_+ \ln n_+ - n_+} - \underbrace{\ln n_-!}_{n_- \ln n_- - n_-} \\ & = N \ln N - \cancel{N} - n_+ \ln n_+ + \cancel{n_+} - n_- \ln n_- + \cancel{n_-} \\ & = N \ln N - n_+ \ln n_+ - n_- \ln n_- = (n_+ + n_-) \ln N - n_+ \ln n_+ - n_- \ln n_- \\ & = n_+ \ln \frac{N}{n_+} + n_- \ln \frac{N}{n_-} = (\frac{N}{2} + \frac{E}{2\epsilon}) \ln \frac{N}{\frac{N}{2} + \frac{E}{2\epsilon}} + (\frac{N}{2} - \frac{E}{2\epsilon}) \ln \frac{N}{\frac{N}{2} - \frac{E}{2\epsilon}} \\ & = N \Big ( (\frac{1}{2} + \frac{E}{2\epsilon N}) \ln \frac{1}{\frac{1}{2} + \frac{E}{2\epsilon N}} + (\frac{1}{2} - \frac{E}{2\epsilon N}) \ln \frac{1}{\frac{1}{2} - \frac{E}{2\epsilon N}} \Big ) \\ & = - N \Big ( (\frac{1}{2} + \frac{E}{2\epsilon N}) \ln (\frac{1}{2} + \frac{E}{2\epsilon N} ) + (\frac{1}{2} - \frac{E}{2\epsilon N}) \ln (\frac{1}{2} - \frac{E}{2\epsilon N} )\Big ) ~.
        \end{aligned}
        \end{equation*}
    \end{proof}
    
    The temperature $T$ is 
    \begin{equation*}
        T = \frac{2 \epsilon}{k_B} \frac{1}{\ln \frac{\frac{1}{2} - \frac{E}{2 \epsilon N}}{\frac{1}{2} + \frac{E}{2 \epsilon N}}} ~.
    \end{equation*}
    A plot of the temperature $T$ as a function of $E$ is in Figure~\ref{en:t}.
    \begin{figure}
        \centering
        \scalebox{0.7}{\pyc{plot1('x', '1 / (ln ( (0.5 - x) / (0.5 + x) ) )', 10, 10, 16, True, False, False)}}
        \caption{A plot of the temperature $T$ as a function of $E$. We have used $x = \frac{E}{2 \epsilon N} $ and $f(x) = \frac{k_B T}{2 \epsilon}$.}
        \label{en:t}
    \end{figure}
    \begin{proof}
        Using the first of~\eqref{td:es:s}, we have
        \begin{equation*}
        \begin{aligned}
            T & = (\pdv{S}{E})^{-1} = - (\frac{k_B}{2\epsilon} \ln (\frac{1}{2} + \frac{E}{2 \epsilon N}) + \cancel{\frac{k_B}{2\epsilon}} - \frac{k_B}{2\epsilon} \ln (\frac{1}{2} - \frac{E}{2 \epsilon N}) - \cancel{\frac{k_B}{2\epsilon}} )^{-1} \\ & = - (\frac{k_B}{2\epsilon} \ln \frac{\frac{1}{2} + \frac{E}{2 \epsilon N}}{\frac{1}{2} - \frac{E}{2 \epsilon N}})^{-1} = - \frac{2 \epsilon}{k_B} \frac{1}{\ln \frac{\frac{1}{2} + \frac{E}{2 \epsilon N}}{\frac{1}{2} - \frac{E}{2 \epsilon N}}} = \frac{2 \epsilon}{k_B} \frac{1}{\ln \frac{\frac{1}{2} - \frac{E}{2 \epsilon N}}{\frac{1}{2} + \frac{E}{2 \epsilon N}}} ~.
        \end{aligned}
        \end{equation*}
    \end{proof}
    

\part{Phase transition}

\chapter{Classical phase transitions}

    In the last part of these notes, we will study phase transitions and critical phenomena of classical physical systems. In particular, we will deal with fluids and magnetic (spin) systems.

\section{Classical fluids}

    Consider a classical fluid, e.g.~water, composed by atoms or molecules interacting via a $2$-body potential, i.e.~a potential that depends only on the reciprocate distance between $2$ constituents. It can present itself in $3$ different phases, which microscopically have the same Hamiltonian, but the macroscopic variables change:
    \begin{enumerate}
        \item solid, i.e.~it has its own shape and volume;
        \item liquid, i.e.~it has its own volume, but it has the shape of the container;
        \item gas, i.e.~it has the shape and volume of the container. 
    \end{enumerate}
    They can be represented in a phase diagram $(T,p)$. See Figure~\eqref{fig:phwater} for the water and Figure~\eqref{fig:phhelium} for the Helium. Notice that Helium has $2$ liquid different phases, normal liquid and superfluid, i.e.~zero viscosity and dissipationless flow. The phase diagram can be divided into region containing a single phase. At the boundary of these regions, we can have $2$ different kind of coexistence phases in equilibrium: coexistence lines and triple points.  

    \begin{figure}[h!]
        \centering
        \begin{tikzpicture}
        \draw[->] (0,0) -- (5,0) node[right] {$T$};
        \draw[->] (0,0) -- (0,5) node[right] {$P$};
                
        \draw[] (0,0) to[bend right=10] (2,2) node[xshift=-1cm, yshift=1cm] {solid};
        \draw[] (2,2) to[bend right=10] (5,2.5) node[yshift=-1.35cm, xshift=-1cm] {liquid};
        \draw[] (2,2) to[bend right=10] (4,4.5) node[yshift=-1.5cm] {vapor};

        \filldraw[black] (2,2) circle (0.05) node[below right] {$t$};
        \filldraw[black] (5,2.5) circle (0.05) node[above right] {$c$};
        
        \end{tikzpicture}
        \caption{Qualitative phase diagram of the water. $t$ is a triple point and $c$ is a critical point.}
        \label{fig:phwater}
    \end{figure}

    \begin{figure}[h!]
        \centering
        \begin{tikzpicture}
        \draw[->] (0,0) -- (5,0) node[right] {$T$};
        \draw[->] (0,0) -- (0,5) node[right] {$P$};
                
        \draw[] (0,3) to[bend right=10] (3,5) node[xshift=-1.75cm, yshift=-0.5cm] {solid};
        \draw[] (1,3.475) to[bend right=10] (3.5,0) node[yshift=1cm, xshift=-2.25cm] {superfluid};
        \draw[] (3.5,0) to[bend right=10] (5,3.5) node[yshift=-0.5cm, xshift=-1.75cm] {liquid} node[yshift=-1.75cm, xshift=0.75cm] {vapor};

        \filldraw[black] (1,3.475) circle (0.05) node[above] {$t_1$};
        \filldraw[black] (3.5,0) circle (0.05) node[below right] {$t_2$};
        \filldraw[black] (5,3.5) circle (0.05) node[above right] {$c$};
        
        \end{tikzpicture}
        \caption{Qualitative phase diagram of the Helium. $t_1$ and $t_2$ are triple points and $c$ is a critical point.}
        \label{fig:phhelium}
    \end{figure} 

    A coexistence line is a line along which $2$ phases are in equilibrium. A coexistence point or triple point is a point in which $3$ phases are in equilibrium. Examples in the water phase diagram of coexistence lines are $S-L$, $L-V$ and $S-V$ separation lines, whereas there is only a single triple point. To study when coexistence phases system are in equilibrium, we exploit the grand canonical ensemble. For coexistence lines, in order to have equilibrium, we have
    \begin{equation*}
        T_1 = T_2 ~, \quad T_1 = T_2 ~, \quad p_1 = p_2 ~, \quad \mu_1 = \mu_2 ~,
    \end{equation*}
    which provide respectively thermal, mechanical and chemical equilibrium. Writing $\mu$ in terms of the other thermodynamic functions, we obtain a constraint 
    \begin{equation*}
        \mu_1(T,p) = \mu_2(T,p) ~,
    \end{equation*}
    which individuates a line in the $(T, p)$ plane. Similarly, for triple points, in order to have equilibrium, we have
    \begin{equation*}
        T_1 = T_2 = T_3 ~, \quad p_1 = p_2 = p_3 ~, \quad \mu_1 = \mu_2 = \mu_3 ~,
    \end{equation*}
    where we have denoted with $1,2,3$ different phases. Writing $\mu$ in terms of the other thermodynamic functions, we obtain $2$ constraint s
    \begin{equation*}
        \mu_1(T,p) = \mu_2(T,p) = \mu_3(T,p) ~,
    \end{equation*}
    which individuates a point in the $(T, p)$ plane.

    We can generalise this result with the Gibbs' phase rule, which states that, in a system with $l$ distinct species, the number of coexistence phases in equilibrium $r$ is bounded above by 
    \begin{equation*}
        r \leq l + 2 ~.
    \end{equation*} 
    In the water case, $l = 1$ and $r = 3$, so that at most we have indeed $3$ coexisting phases.

    Notice that the coexistence curve $L-V$ terminates at a critical point, in which there is no more distinction between liquid and vapor, because we can circumnavigate from right to go from one phase to the other.

\section{Classification of phase transitions}

    Away from a critical point, a phase transition involves latent heat $\Delta q$, since $T$ is constant, but thermal energy is used or released to change the phase. It can be estimated via the Clausius-Clapeyron equation
    \begin{equation}\label{ph:cceq}
        \dv{p}{T} = \frac{s_2 - s_1}{v_1 - v_2} = \frac{\Delta q}{T \Delta v}  ~.
    \end{equation}
    \begin{proof}
        In order to remain along the coexistence line, the constraint holds
        \begin{equation*}
            \mu_1(p, T) = \mu_2 (p, T) ~.
        \end{equation*}
        We differentiate it using the chain rule, keeping in mind that $p = p(T)$,
        \begin{equation*}
            \pdv{\mu_1}{T} \Big \vert_p + \dv{\mu_1}{p} \Big \vert_T \dv{p}{T} = \pdv{\mu_2}{T} \Big \vert_p + \dv{\mu_2}{p} \Big \vert_T \dv{p}{T} ~.
        \end{equation*}
        Hence, we isolate $dp/dT$ and we find
        \begin{equation*}
            \dv{p}{T} = \frac{\pdv{\mu_1}{T} \vert_p - \pdv{\mu_2}{T} \vert_p}{\pdv{\mu_2}{p} \vert_T - \pdv{\mu_2}{p} \vert_T} ~.
        \end{equation*}
        Since a change in phase does not mean a change in total number of particles, we can work with a fixed amount of them. The thermodynamic potential to use is therefore the Gibbs free energy $G(p, T, N) = \mu (p, T) N$ or the Gibbs free energy per particle 
        \begin{equation*}
            g = \frac{G}{N} = \mu(p,T) ~.
        \end{equation*}
        Using the last relation of~\eqref{td:es:g}
        \begin{equation*}
            \pdv{\mu}{p} \Big \vert_T = \pdv{g}{p} \Big \vert_T = \frac{1}{N} \pdv{G}{p} \Big \vert_T = \frac{V}{N} = v ~,
        \end{equation*}
        whereas, using the first relation of~\eqref{td:es:g}
        \begin{equation*}
            \pdv{\mu}{T} \Big \vert_p = \pdv{g}{T} \Big \vert_p = \frac{1}{N} \pdv{G}{T} \Big \vert_p = - \frac{S}{N} = - s ~.
        \end{equation*}
        Combining the $two$, we obtain
        \begin{equation*}
            \dv{p}{T} = \frac{\pdv{\mu_1}{T} \vert_p - \pdv{\mu_2}{T} \vert_p}{\pdv{\mu_2}{p} \vert_T - \pdv{\mu_2}{p} \vert_T} = - \frac{s_1 - s_2}{v_2 - v_1} = \frac{s_2 - s_1}{v_2 - v_1} ~.
        \end{equation*}
        Finally, using the second law of thermodynamics~\eqref{td:2nde}, we find
        \begin{equation*}
            \Delta s = \frac{\Delta q}{T} ~,
        \end{equation*}
        which implies that
        \begin{equation*}
            \dv{p}{T} = \frac{\Delta q}{T \Delta v} ~.
        \end{equation*}
    \end{proof}

    Suppose that in a system there is latent heat. Observing~\eqref{ph:cceq}, we can state that the latent heat $\Delta q$ is different from zero, only when $s_1 \neq s_2$, which corresponds to a change in order of the system. Recalling the first of~\eqref{td:es:g}, we can say that the phase $1$ must be more stable at low temperatures while phase $1$ must be more stable at high temperatures. This implies that there is a cusp-like behaviour of $G$, or equivalently on $\mu$,
    \begin{equation*}
        \pdv{\mu_1}{T} > \pdv{\mu_2}{T} ~.
    \end{equation*}
    Therefore, at the phase transition temperature, the Gibbs free energy is continuous ($\mu_1 = \mu_2$) but its first derivative in $T$ is not, resulting in a cusp-like behaviour. Moreover, if it does change volume as well $v_2 \neq v_1$, its first derivative in $p$ has a similar behaviour. However, at $T = T_c$, discontinuity of first derivatives disappears, since we cannot distinguish anymore the $2$ phases. However, there could be other discontinuities in higher derivatives, e.g.~specific heat or compressibility are defined as second derivatives of thermodynamic potentials. Hence, we classify phase transitions in $2$ different kind 
    \begin{enumerate}
        \item $1$st order phase transitions, i.e.~those in which the $1$st derivatives of thermodynamic potentials are discontinuous;
        \item continuous phase transitions, i.e.~those in which the higher derivatives of thermodynamic potentials are discontinuous.
    \end{enumerate}
    In our case, the former are those in which there is a jump $v_2 \neq v_1$ and $s_2 \neq s_1$ and the latter are those in which $v_2 = v_1$ and $s_2 = s_1$. To summarise, a phase transition happens when there is a singular point for a thermodynamic potential.  
    In the next section, we will develop the mathematical framework of a phase transition in terms of this quantity in the thermodynamic limit.

\section{Theorems of Lee and Young}

    Consider a classical fluid in a volume $V \subset \mathbb R^3$. As mentioned before, we will analyse it in the grand canonical ensemble. The Hamiltonian of the system is 
    \begin{equation*}
        H = \sum_{i=1}^{N} (\frac{p_i^2}{2m} + U_N (q_i)) 
    \end{equation*}
    and the grand canonical partition function is 
    \begin{equation*}
        \mathcal Z (z, T, V) = \sum_{N=0}^\infty z^N \frac{Q_N (T, V)}{N! \lambda_T^3} ~,
    \end{equation*}
    where we have defined a positive quantity
    \begin{equation*}
        Q_N (T, V) = \int_{V^N} \prod_{i=1}^N d^3 q^i \exp (- \beta U_N(q^i)) ~.
    \end{equation*}
    \begin{proof}
        The canonical partition function~\eqref{c:z} is
        \begin{equation*}
        \begin{aligned}
            Z_N & = \int_{\mathcal M^N} \prod_{i=1}^N \frac{d^3 q^i ~ d^3 p^i}{N! h^{3N}}\exp (- \beta \sum_j \frac{p_j}{2m} + U_N(q^i) ) \\ & = \frac{1}{N!} \underbrace{\int_{\mathbb R^{3N}} \prod_{i=1}^N \frac{d^3 p^i}{h^{3N}} ~ \exp(- \beta \frac{p_i}{2m})}_{\frac{1}{\lambda_T^{3N}}} \underbrace{\int_{V^N} \prod_{i=1}^N d^3 q^i ~\exp(-\beta U_N(q_i))}_{Q_N} = \frac{Q_N(T, V)}{N! \lambda_T^{3N}} ~.
        \end{aligned}
        \end{equation*}
        Finally, using ~\eqref{gc:z}, we find 
        \begin{equation*}
            \mathcal Z = \sum_{N=0}^\infty z^N Z_N = \sum_{N=0}^\infty z^n \frac{Q_N}{N! \lambda_T^{3N}} ~.
        \end{equation*}
    \end{proof}
    Now, we need to study when this power series in $z$ converges. The first step is to promote $z$ into a complex variable, but always keeping in mind that the physical states are only the ones for which $z \in \mathbb R^+$. A reasonable assumption for the behaviour of the potential is that $U_N$ is bounded from below by a constant that does not grow faster than $N$, i.e. $U_N \geq - BN$ with $B > 0$. This implies that 
    \begin{equation*}
        |\mathcal Z| \leq \exp(\frac{V \exp(\beta B) |z|}{\lambda_T^3}) ~.
    \end{equation*}
    \begin{proof}
        In fact, using the assumption, we have
        \begin{equation*}
            \exp(- \beta U_N) \leq \exp(\beta B N) ~,
        \end{equation*}
        hence, $Q_N$ becomes
        \begin{equation*}
            Q_N = \int_{V^N} \prod_{i=1}^N d^3 q^i \exp (- \beta U_N(q^i)) \leq \exp(\beta B N) \underbrace{\int_{V^N}\prod_{i=1}^N d^3 q^i}_{V^N} = \exp(\beta B N) V^N 
        \end{equation*}
        and the canonical partition function becomes
        \begin{equation*}
            Z_N = \frac{Q_N}{N! \lambda_T^{3N}} \leq \frac{V^N}{N! \lambda^{3N}_T} \exp(\beta B N) ~.
        \end{equation*}
        Finally, we find that the grand canonical partition function becomes
        \begin{equation*}
            |\mathcal Z| \leq \sum_{N=0}^\infty \frac{|z|^N}{N! \lambda_T^{3N}} V^N \exp(\beta B N) = \exp(\frac{V \exp(\beta B) |z|}{\lambda_T^3}) ~.
        \end{equation*}
    \end{proof}
    Notice that there is a problem. An exponential has an infinite convergence radius and, therefore, $\mathcal Z$ is analytical $\forall z \in \mathbb C$, in particular for $z \in \mathbb R^+$. Furthermore, $\mathcal Z$ cannot vanish $\forall N$ since it is convergent and it is a sum of positive terms. 

    Since we are working in the grand canonical ensemble, we can introduce a redefined grand potential
    \begin{equation*}
        \psi = \frac{\beta \Omega}{V} = \lim_{td} \frac{\ln \mathcal Z}{V} ~,
    \end{equation*}
    for which it is valid 
    \begin{equation*}
        p \beta = \psi ~, \quad n = z \pdv{}{z} \psi ~.
    \end{equation*}
    \begin{proof}
        For the first, using the first of~\eqref{gc:es}
        \begin{equation*}
            \Omega = - p V = - \frac{1}{\beta} \ln \mathcal Z ~,
        \end{equation*}
        hence
        \begin{equation*}
            p \beta = \frac{\ln \mathcal Z}{V} = \psi ~.
        \end{equation*}
        For the second, using the second of~\eqref{gc:es}
        \begin{equation*}
            N = z \pdv{}{z} \ln \mathcal Z = - \frac{z}{\beta} \pdv{}{z} \ln \Omega ~,
        \end{equation*}
        hence
        \begin{equation*}
            n = \frac{N}{V} = - \frac{z}{\beta} \pdv{}{z} \frac{\ln \Omega}{V} =  - \frac{z}{\beta} \pdv{}{z} \psi ~.
        \end{equation*}
    \end{proof}
    These results can be formally stated by $2$ theorems, proved by Lee and Young, in terms of $\psi$.
    \begin{theorem}[Lee, Young I]
        Let $U_N$ be the potential such that $U_N \geq - BN$ with $B > 0$. Let also that boundaries of the volume do not increase faster than $V^{2/3}$, in order to neglect surface terms. Then $\psi$ exists, it is a continuous and monotonically increasing function of $z \in \mathbb R^+$.
    \end{theorem}
    \begin{theorem}[Lee, Young II]
        Given an open subset of the complex plane containing an interval of $\mathbb R^+$ such that it does not contain zeroes of $\mathcal Z$, then $V^{-1} \ln \mathcal Z$ converges uniformly for $V \rightarrow \infty$ in any closed set of this region, $\psi$ exists and it is analytic.
    \end{theorem}
    This means that there are no phase transitions and there is a single stable phase, since there cannot happen singularities for zero-free regions. How is it possible? We have not yet computed the thermodynamic limit. In fact, consider a system composed by hard spheres occupying a finite volume $v$. Since particles cannot overlap, the maximum number of particles is $M = V / v$. Therefore, $\mathcal Z$ is a polynomial function in $z$ of degree $M$ and, by the fundamental theorem of algebra, it has exactly $M$ zeroes but, by the theorems of Lee and Young, no zeros are in $\mathbb R^+$. However, if we go into the thermodynamic limit ($V \rightarrow \infty$ implies that $M \rightarrow \infty$), the number of zeroes increases. It may happen that, at a certain temperature, zeroes accumulate towards an isolated point $z = z_c$, which divides $\mathbb R^+$ into $2$ regions corresponding $2$ different phases. $\mathcal Z (V \rightarrow \infty, T, \mu)$ has a zero in $z = z_c$. Furthermore, $\psi$ is continuous but it is not analytic anymore: $1st$ order phase transitions or continuous phase transitions may occur. See Figure~\ref{fig:accu}.
    
    \begin{figure}[h!]
        \centering
        \begin{tikzpicture}
        \draw[->] (-0.5,0) -- (4,0) node[right] {$\real z$};
        \draw[->] (0,-2) -- (0,2) node[right] {$\imm z$};
                
        \filldraw[black] (1.5, 1) circle (0.05) ;
        \filldraw[black] (1.75, 0.75) circle (0.05) ;
        \filldraw[black] (1.85, 0.5) circle (0.05) ;
        \filldraw[black] (1.95, 0.25) circle (0.05) ;
        \filldraw[black] (1.975, 0.1) circle (0.05) ;
        \filldraw[black] (2, 0) circle (0.05) node[above right] {$z_c$} node[yshift=-0.4cm, xshift=1.1cm] {phase 2} node[yshift=-0.4cm, xshift=-1.1cm] {phase 1};
        \filldraw[black] (1.975, -0.1) circle (0.05) ;
        \filldraw[black] (1.95, -0.25) circle (0.05) ;
        \filldraw[black] (1.85, -0.5) circle (0.05) ;
        \filldraw[black] (1.75, -0.75) circle (0.05) ;
        \filldraw[black] (1.5, -1) circle (0.05) ;
        
        \end{tikzpicture}
        \caption{Accumulation of zeros in $z = z_c$ in the complex plane of $z$ that divides the real axis into two different phases $1$ and $2$.}
        \label{fig:accu}
    \end{figure}

    In the next chapter, we will study the paradigmatic example for phase transitions: the Ising model.

\chapter{Ising model}

    The Ising model deals with spin. However, spin is a quantum physical quantity that does not have a classical counterpart, so when we talk about classical spin, we mean localised magnetic moments that couple with an external magnetic field. 

\section{Simple Ising model}
    
    Consider a system composed by a discrete lattice, e.g.~an hypercubic lattice in Figure~\eqref{fig:latt}, of dimension $d$. Lattice sites are labelled by $i \in \mathbb Z_d$. For each vertex,  there is a degree of freedom (classical spin) attached to a vector $\mathbf S_i \in \mathbf R^n$ with fixed magnitude $|\mathbf S_i| = $ const. Therefore, $\mathbf S_i \in \mathbb S^{n-1}$, where $\mathbb S^{n-1}$ is the $(n-1)$-dimensional sphere of radius $|\mathbf S_i|$. Two adjacent vertices are called neighborhoods. Each site has therefore $z$ neighborhood, called the coordination number. For a $d$-dimensional hybercubic lattice, $z = 2 d$. 

    \begin{figure}[h!]
        \centering
        \begin{tikzpicture}
        \draw[-] (0,0) -- (4,4);
        \draw[-] (2,0) -- (6,4);
        \draw[-] (4,0) -- (8,4);
        \draw[-] (-0.5,1) -- (6.5,1);
        \draw[-] (0.5,2) -- (7.5,2);
        \draw[-] (1.5,3) -- (8.5,3);
                
        \filldraw[black] (1,1) circle (0.05) node[below right] {$i$};
        \filldraw[black] (3,1) circle (0.05) node[below right] {$j$};
        \filldraw[black] (5,1) circle (0.05) ;
        \filldraw[black] (2,2) circle (0.05) ;
        \filldraw[black] (4,2) circle (0.05) ;
        \filldraw[black] (6,2) circle (0.05) ;
        \filldraw[black] (3,3) circle (0.05) ;
        \filldraw[black] (5,3) circle (0.05) ;
        \filldraw[black] (7,3) circle (0.05) ;

        \end{tikzpicture}
        \caption{An hybercubic lattice of dimension $2$. $i$ and $j$ are neighbourhood sites}
        \label{fig:latt}
    \end{figure}

    We will not study the more general model, but we will restrict ourselves to the simple model in which $n = 1$ and the spin $\mathbf S_i$ can have only two values $\mathbf S_i = \sigma_i = \pm 1$. We will denote a possible configuration state as $\{\sigma_i\}_{i \in \mathcal L}$ and the phase space will be a discrete space composed by $2^N$ states $\{\{\sigma_i\}_{i \in \mathcal L} \colon \sigma_i = \pm 1\}$. The Hamiltonian of the system can be decomposed into two parts: a term that describes the interaction between neighborhood sites $H_{int}$ and a term that describes the coupling with an external field $H_{field}$ 
    \begin{equation*}
        H(\sigma_i) = H_{int} + H_{field} ~, \quad H_{int} = - J \sum_{i~\text{near}~j} \sigma_i \sigma_j ~, \quad H_{field} = - B \sum_{i=1}^{N} \sigma_i ~,
    \end{equation*}
    where $B$ is an external magnetic field and $J$ is the interaction constant, which is invariant under translations and rotations. We allow  sites to interact with each other because otherwise there would not have a phase transition.
    We can study what is the minimum energy configuration state according to the sign of $B$ and $J$:
    \begin{enumerate}
        \item for $J > 0$, all the spins are aligned $\sigma_i = \sigma_j$ for $i~\text{near}~j$, called ferromagnetic model;
        \item for $J < 0$, all the spins are antialigned $\sigma_i = - \sigma_j$ for $i~\text{near}~j$, called antiferromagnetic model;
        \item for $B > 0$, all the spins are aligned upwards $\sigma_i = + 1$;
        \item for $B < 0$, all the spins are aligned upwards $\sigma_i = - 1$. 
    \end{enumerate}
    Now, we will analyse the system in the canonical ensemble. Since we are in a discrete space, the canonical partition function is made over a sum of all the $2^N$ states, instead of an integral,
    \begin{equation*}
        Z_N = \sum_{\sigma_i = \pm 1} \exp(- \beta H(\sigma_i)) ~.
    \end{equation*}
    The thermodynamic equilibrium corresponds to the configuration of minimum (Helmoltz) free energy.

    Suppose the external magnetic field is shut down, i.e.~$B = 0$. What is the equilibrium configuration? At low temperature (and low energies), Helmoltz free energy is at minimum when entropy is small and all spins are aligned, because there are only $2$ possible states (all upwards or all downwards). At high temperature (and high energies), Helmoltz free energy is at minimum when entropy is large and all spins are random-aligned, because all spins point in all directions. 
    
    By means of the magnetisation
    \begin{equation}\label{ph:m}
        M = \av{\sum_{i=1}^N \sigma_i}_c = \sum_{i=1}^N \av{\sigma_i}_c ~,
    \end{equation}
    where the second expression follows from translation invariance, we can quantitatively study the phase transition.

\section{Mean-field approximation} 

    In general, it is difficult to compute the total canonical partition function because it cannot be reduced to the computation of the $1$-particle canonical partition function $Z_1$ since there is an interacting term
    \begin{equation*}
        Z_N = \sum_{\{\sigma_i = \pm 1\}} \exp(- \beta H) \neq (Z_1)^N ~.
    \end{equation*}
    However, we can make an useful approximation by neglecting the quadratic fluctuation term in the expansion of the interacting term in the Hamiltonian
    \begin{equation*}
    \begin{aligned}
        \sigma_i \sigma_j & = ((\sigma_i - m) + m)((\sigma_j - m) + m) \\ & = m^2 + m(\sigma_i - m) + m (\sigma_j - m) + (\sigma_i - m)(\sigma_j - m) \\ & \simeq m^2 + m(\sigma_i - m) + m (\sigma_j - m) \\ & = m^2 + m\sigma_i - m^2 + m \sigma_j - m^2 = - m^2 + m (\sigma_i + \sigma_j) ~,
    \end{aligned}
    \end{equation*}
    where $m = M / N$.
    The physical interpretation of the mean-field approximation is the following: when fluctuations with respect to the mean field $m$ are negligible, we do not have to compute every link with respect to each others but only with respect to the mean field $m$. In the mean-field approximation, the magnetisation is given by the equation 
    \begin{equation*}
        m = \tanh(\beta(Jzm + B)) ~.
    \end{equation*}
    \begin{proof}
        In fact, the Hamiltonian is 
        \begin{equation*}
        \begin{aligned}
            H_{mf} & = - J \sum_{i~\text{near}~j}  (- m^2 + m(\sigma_i + \sigma_j)) - B \sum_i \sigma_i \\ & = m^2 J \underbrace{\sum_{i~\text{near}~j} 1}_{\frac{Nz}{2}} - J m \sum_{i~\text{near}~j}  (\sigma_i + \sigma_j) - B \sum_i \sigma_i \\ & = \frac{m^2 z N J}{2} - Jmz \sum_i \sigma_i - B \sum_i \sigma_i = \frac{m^2 z N J}{2} - (J m z + B) \sum_i \sigma_i ~,
        \end{aligned}
        \end{equation*}
        where we have estimates that the number of links, given the coordination number $z$ which tells how many neighboring sites, is $Nz/2$.
        The canonical partition function becomes
        \begin{equation*}
        \begin{aligned}
            Z_N^{mf} & = \sum_{\{\sigma_i = \pm 1\}} \exp(- \beta H_{mf}) = \exp(- \beta \frac{J z n m^2}{2}) \sum_{\{\sigma_i = \pm 1\}} \exp(\beta (B + Jmz) \sum_i \sigma_i) \\ & = \exp(- \beta \frac{J z n m^2}{2}) (\sum_{\{\sigma_i = \pm 1\}} \exp(\beta (B + Jmx) \sigma_i))^N \\ & = \exp(- \beta \frac{J z n m^2}{2}) (\exp(\beta(B + Jmz)) + \exp(- \beta (B + Jmz)))^N \\ & = \exp(- \beta \frac{J z n m^2}{2}) (2 \cosh (\beta (B + Jmz)))^N ~.
        \end{aligned}
        \end{equation*}
        The Helmoltz free energy is 
        \begin{equation*}
        \begin{aligned}
            F & = - \frac{1}{\beta} \ln Z_N^{mf} = - \frac{1}{\beta} (- \beta \frac{J z N m^2}{2}) N \ln (2 \cosh (\beta (B + Jmz))) \\ & = \frac{J z N m^2}{2} N \ln (2 \cosh (\beta (B + Jmz))) ~.
        \end{aligned}
        \end{equation*}
        The magnetisation is 
        \begin{equation*}
        \begin{aligned}
             m & = \frac{1}{N} \av{\sum_i \sigma_i}_c = \frac{1}{N} \sum_{\{\sigma_i = \pm 1\}} \sum_i \sigma_i \exp(- \beta H) \\ & = - \frac{1}{\beta N} \sum_{\{\sigma_i = \pm 1\}} \frac{1}{Z_N} \pdv{}{\beta} \exp(- \beta H) = - \frac{1}{\beta N} \pdv{\ln Z_N}{\beta} ~.
        \end{aligned}
        \end{equation*}
        Hence 
        \begin{equation*}
            m = \tanh (\beta (B + J m z)) ~.
        \end{equation*}
    \end{proof}
    
    Now, we have a self-consistent equation for $m$ to solve. The condition for having a solution is 
    \begin{equation}\label{ph:m1}
        m \begin{cases}
            > 0 & B > 0 \\
            < 0 & B < 0 \\
        \end{cases} ~.
    \end{equation}
    Particular attention is the study of the solution for $B = 0$. The magnetisation becomes
    \begin{equation*}
        m = \tanh \frac{J m z}{k_B T} = \tanh (\frac{T_c}{T} m) ~,
    \end{equation*}
    where $T_c = J z / k_B$ is the critical temperature, that depends on $z$. Calling $\tilde m = T_c m / T$, we have 
    \begin{equation*}
        \frac{T \tilde m}{T_c} = \tanh \tilde m ~,
    \end{equation*}
    which can be solved graphically by finding the intersection between the plots of the right-handed side (a straight line) and of the left-handed side (an hyperbolic tangent).See Figure~\ref{mf:m}.
    \begin{figure}[h!]
        \centering
        \scalebox{0.7}{\pyc{plot4('x', '2* x', 'x / 2', 'x', 'tanh(x)', 5, 5, 20, True, False, False)}}
        \caption{A plot of the graphical solution of $T \tilde m / T_c = \tanh \tilde m$ for different value of $T/T_c = 1/2, 1, 2$.}
        \label{mf:m}
    \end{figure}
    Therefore, by looking at the plot, we can conclude that 
    \begin{enumerate}
        \item for $T \geq T_c$, there is only one solution $m = 0$;
        \item for $T < T_c$, there are two non-trivial solutions $m(T) = \pm m_0 (T)$, one positive and one negative.
    \end{enumerate}
    To summarise 
    \begin{equation}\label{ph:mtc}
        m = \begin{cases}
            0 & T > T_c \\ 
            \pm m_0(T) & T < T_c \\ 
        \end{cases} ~.
    \end{equation}

    Now, we are able to compute the phase diagram $(T, B)$. In fact, for $B \neq 0$, we recover~\eqref{ph:m1}, whereas for $B=0$, $m \neq 0$ for $T < T_c$ (ferromagnetic phase) and $m = 0$ for $T \geq T_c$ (paramagnetic phase). See Figure~\eqref{fig:mtrans}. By looking at it, we can observe that $m$ is an order parameter, since when it is zero there is disorder and when it is different from zero, there is order. It signals as well when there is a phase transition, since by its value we can say if we are in a ferromagnetic or in a paramagnetic phase. See Figure~\eqref{fig:mt}.
    
    \begin{figure}[h!]
        \centering
        \begin{tikzpicture}
        \draw[->] (-0.5,0) -- (5,0) node[right] {$T$};
        \draw[->] (0,-2.5) -- (0,2.5) node[right] {$B$};
                
        \draw[thick] (0,0) -- (2.5,0) node[xshift=-1cm, yshift=0.25cm] {$m \neq 0$} node[xshift=1cm, yshift=0.25cm] {$m = 0$} node[xshift=-1cm, yshift=1.5cm] {$m > 0$} node[xshift=-1cm, yshift=-1.5cm] {$m < 0$};

        \filldraw[black] (2.5,0) circle (0.05) node[below right] {$T_c$};
        
        \end{tikzpicture}
        \caption{Phase diagram of the Ising model.}
        \label{fig:mtrans}
    \end{figure}
    
    \begin{figure}[h!]
        \centering
        \begin{tikzpicture}
        \draw[->] (-0.5,0) -- (5,0) node[right] {$T$};
        \draw[->] (0,-2.5) -- (0,2.5) node[right] {$M$};
                
        \draw[thick] (0,1.5) to[bend left=50] (2.5,0);
        \draw[thick] (0,-1.5) to[bend right=50] (2.5,0);
        \draw[thick] (2.5,0) -- (5,0);

        \filldraw[black] (2.5,0) circle (0.05) node[below right] {$T_c$};
        
        \end{tikzpicture}
        \caption{Qualitative plot of $M$ in function of $T$.}
        \label{fig:mt}
    \end{figure}

    In particular, in a neighborhood of $T_c$, we can estimate that the behaviour is
    \begin{equation}\label{ph:beta}
        M \sim (T - T_c)^\beta ~,
    \end{equation}
    where $\beta \in \mathbb R$ is a parameter and $T < T_C$. $beta$ characterises the phase transition, since it tells at which speed $M \rightarrow 0$ when approaching $T \rightarrow T_c$. It is one of the $6$ so-called critical exponents.

    Other information can be found in the $2$-point correlation function between $2$ different sites
    \begin{equation*}
        G_{ij} = \av{\sigma_i \sigma_j} - \av{\sigma_i} \av{\sigma_j} =  \av{\sigma_i \sigma_j} - m^2 ~,
    \end{equation*}
    which in the limit for which $r = |i - j|$ is large, it can be estimated to be 
    \begin{equation}\label{ph:eta}
        G(r) \propto \begin{cases}
            \exp(- \frac{r}{\xi}) & T \neq T_c \\
            r^{-d+2-\eta} & T = T_c \\
        \end{cases} ~,
    \end{equation}
    where $\xi$ is the correlation length 
    \begin{equation}\label{ph:nu}
        \xi(T) = |1 - \frac{T}{T_c}|^{-\nu} \xrightarrow{T \rightarrow T_c} \infty ~.
    \end{equation}
    where $\eta$ and $\nu$ are critical exponents. Physically, $\xi$ tells us what is the radius inside which all the spins are strongly correlated. For $ T = T_c$, all spins are correlated since they are all aligned. See Figure~\eqref{fig:phg}.

    \begin{figure}[h!]
        \centering
        \scalebox{0.7}{\pyc{plot1('x', 'exp(-x)', 3, 2, 21, True, True, True)}}
        \caption{A plot of the correlation function for $\xi = 1$ at $T \neq T_c$.}
        \label{fig:phg}
    \end{figure}

\section{Spontaneous symmetry breaking}

    In the classical fluid system, we can find phase transition by studying symmetries of the system. In fact, we can distinguish solid from fluid by the translation or rotations invariance, since solid has only discrete invariance, whereas fluid has continuous invariance. However, we cannot distinguish with symmetries between gas and liquid. This means that the phase transition $V-L$ is not breaking any symmetry, but the phase transition $L-S$ does. 
    Also in the Ising model, we can similarly notice that a phase transition can arise from a spontaneous symmetry breaking. In fact, the interacting term in the Hamiltonian $H_{int}$ is invariant under the global symmetry group $\mathbb Z_2$
    \begin{equation*}
        \sigma_i \rightarrow - \sigma_i ~, \quad \sigma_i \rightarrow \sigma_i ~.
    \end{equation*}
    However, the second term breaks explicitly the symmetry, since under this transformation it transforms as $H_{field} \rightarrow - H_{field}$. Moreover, notice that, by definition~\eqref{ph:m}, under this symmetry we have 
    \begin{equation}\label{ph:msymm}
        m = \frac{M}{N} = \frac{1}{N} \sum_{i=1}^N \av{\sigma_i}_c \rightarrow - \frac{1}{N}\sum_{i=1}^N \av{\sigma_i}_c = - \frac{M}{N} = - m ~,
    \end{equation} 
    which implies that the only possible value of $m$ is zero. In fact, for $T > T_c$ there is indeed $m=0$, but for $T < T_c$, the equilibrium state is no longer invariant under this symmetry. The Hamiltonian remains the same, but equilibrium states are not invariant anymore. This is the definition of a spontaneous symmetry breaking. 

    Formally, we can state that, at high temperature, we have a disordered phase, which is highly symmetric that correspond to a symmetry group $G$, whereas, at low temperature, we have an ordered phase, which is lowly symmetric that correspond to a symmetry subgroup $G_0 \subset G$. Let $O$ be an observable (not-invariant under $G$) such that 
    \begin{equation*}
        \phi = \av{O} = \begin{cases}
            0 & T > T_c \\
            \phi_0 (T) \neq 0 & T < T_c \\
        \end{cases} ~.
    \end{equation*}
    Then we say that a symmetry is spontaneously broken and $\phi$ is an ordered parameter. In the Ising model, we can identify $\phi = m$, since it is not invariant under $\mathbb Z_2$ by~\eqref{ph:msymm} and it is indeed a step function in $T = T_c$ by~\eqref{ph:mtc}.
    As phase transitions, also symmetry breaking needs the thermodynamic limit. In fact, we can arrive to a spontaneous symmetry breaking only in a way that are not equivalent. The first one is to shut down the external field and then compute the thermodynamic limit 
    \begin{equation*}
        \av{O}_{N, V, B \neq 0} \xrightarrow{B \rightarrow 0} 0 \xrightarrow{td} 0 ~.
    \end{equation*}
    The second one is to first compute the thermodynamic limit and then to shut down the external field 
    \begin{equation*}
        \av{O}_{N, V, B \neq 0} \xrightarrow{td} \av{O}_{n, B \neq 0} \xrightarrow{B \rightarrow 0} \begin{cases}
            0 & T > T_c \\
            O_0 \neq 0 & T < T_c \\
        \end{cases} ~.
    \end{equation*}
    Therefore, we can individuate $\phi$ as 
    \begin{equation*}
        \phi = \lim_{B \rightarrow 0} \lim_{td} \av{O}_{N, V, B} ~,
    \end{equation*}
    where the two limits do not commute.

\section{Critical exponents and universality classes}

    During the study of phase transitions, we some parameters like~\eqref{ph:beta},~\eqref{ph:eta} and~\eqref{ph:nu}. These are the critical exponents and they describe the behavior of physical quantities near the critical temperature of a phase transitions. We define the reduced temperature 
    \begin{equation*}
        \epsilon = \frac{T_c - T}{T_c} ~,
    \end{equation*}
    which tells us how much we are far away from the phase transition in terms of temperature. The critical exponent associated to an observable $f$ is 
    \begin{equation*}
        \lambda_f = \lim_{\epsilon \rightarrow 0} \frac{\ln f(\epsilon)}{\ln \epsilon} ~,
    \end{equation*}
    so that 
    \begin{equation*}
        f (\epsilon) \simeq g(\epsilon) |\epsilon|^{\lambda_f} ~.
    \end{equation*}
    \begin{proof}
        In fact, for $\epsilon \ll 1$,
        \begin{equation*}
            \frac{\ln f(\epsilon)}{\ln \epsilon} = \frac{\ln g(\epsilon) |\epsilon|^{\lambda_f}}{\ln \epsilon} = \underbrace{\frac{\ln g(\epsilon)}{\ln \epsilon}}_0 + \frac{\lambda_f \ln \epsilon}{\ln epsilon} \simeq \lambda_f ~.
        \end{equation*}
    \end{proof}
    
    The $6$ critical exponents are
    \begin{enumerate}
        \item $\alpha$ in the specific heat $C_{B = 0} \simeq |\epsilon|^{-\alpha}$,
        \item $\beta$ in the specific heat $\phi \simeq |\epsilon|^{\beta}$,
        \item $\gamma$ in the specific heat $\chi_{B = 0} \simeq |\epsilon|^{-\gamma}$,
        \item $\delta$ in the specific heat $B \simeq \sgn(\phi) |\phi|^{\delta}$,
        \item $\nu$ in the specific heat $\xi \simeq |\epsilon|^{-\nu}$,
        \item $\eta$ in the specific heat $G(r) \simeq r^{-d+2-\eta}$.
    \end{enumerate}
    $alpha$, $\gamma$ and $\nu$ have a minus sign to prevent divergences. They are calculated with the scaling hypothesis which states that Helmoltz free energy is an homogeneous function of $\epsilon$ and $B$ 
    \begin{equation*}
        f ( \lambda \epsilon, \lambda B) = \lambda f(\epsilon, B) ~.
    \end{equation*}

    Therefore, phase transitions can be classified into classes that are independent of the microscopic Hamiltonian. Due to the scale invariance, different systems may have the same behaviour of phase transition. The parameter of classification are 
    \begin{enumerate}
        \item dimension of the space $d$,
        \item symmetry group of the Hamiltonian $H$,
        \item residual symmetry subgroup $G_0$.
    \end{enumerate}
    Recall that the mean field approximation is exact for $d \geq 4$ while it works poorly for decreasing $d$. However, for $d = 2$, we have the exact solution.


\part{Quantum statistical mechanics}

\chapter{Quantum Mechanics}

    In this chapter, we will recall some basic notion of quantum mechanics: starting from general definitions of states, equations of motion, observables, time evolution and investigating the formalism of density matrices and the difference between pure quantum and classical mixed states.

\section{States and projectors}

    In quantum mechanics, a pure state of a quantum particle is represented by a normalised vector in a (separable) Hilbert space $\ket{\psi} \in \mathcal H$. This is the best knowledge we can have. An Hilbert space $\mathcal H$ is a vector space on $\mathbb C$, i.e.~in which a linear superposition of vectors is still in the space 
    \begin{equation*}
        \lambda \ket{\psi} + \mu \ket{\phi} \in \mathcal H ~, \quad \forall \ket{\psi}, \ket{\phi} \in \mathcal H ~, \forall \lambda, \mu \in \mathbb C ~, 
    \end{equation*}
    endowed with a scalar product $\braket{\psi}{\phi}$. In particular, via the scalar product, it is possible to associate a norm to the state, which is set to $1$ by the probability interpretation $||\psi||^2 = \braket{\psi}{\psi} = 1$. In the Schroedinger representation, this means that the wave function is a square-integrable function $\psi(t,\mathbf x) \in L^2(\mathbb R^d)$. Furthermore, the probability interpretation tells us that $|\psi(t, \mathbf x)|$ is the probability density to find the particle in a volume element $d^d x$ at time $t$ and the normalisation condition that the total probability to find the particle in the whole $\mathbb R^d$ is $1$ 
    \begin{equation*}
        \int_{\mathbb R^d} d^d x ~ |\psi(t,x)|^2 = 1 ~.
    \end{equation*}
    However, because of that, a state is not associated to a single vector, but to a class of equivalence of them, called a ray in the Hilbert space. This is true since two states are physically equivalent if $\ket{\psi'} = \exp(i \varphi) \ket{\psi}$, because their norms are the same. To remove this ambiguity, we introduce the notion of projection operators or projectors, that uniquely determine a state, 
    \begin{equation*}
        P_\psi = \frac{\ket{\psi} \bra{\psi}}{\braket{\psi}{\psi}} ~,
    \end{equation*}
    which for normalisation states becomes 
    \begin{equation}\label{proj}
        P_\psi = \ket{\psi} \bra{\psi} ~.
    \end{equation}
    \begin{proof}
        If $\ket{\psi'} = \exp(i \varphi) \ket{\psi}$ and $\bra{\psi'} = \exp(- i \varphi) \bra{\psi}$, we have 
        \begin{equation*}
            P_{\psi'} = \ket{\psi'} \bra{\psi'} = \cancel{\exp(i \varphi)} \ket{\psi} \cancel{\exp(- i \varphi)} \bra{\psi} = \ket{\psi} \bra{\psi} = P_\psi ~.
        \end{equation*}
    \end{proof}

    It projects onto the $1$-dimensional subspace $\mathcal H_\psi = \{\lambda \ket{\psi} \colon \lambda \in \mathbb C\}$ generated by the state $\ket{\psi}$
    \begin{equation*}
        P_\psi \colon \mathcal H \rightarrow \mathcal H_\psi ~.
    \end{equation*}
    \begin{proof}
        In fact, $\forall \ket{\phi} \in \mathcal H$, we decomposed the Hilbert space into the direct orthogonal sum of the subspace spanned by $\mathcal H_{\psi}$ and its orthogonal complement $\mathcal H^\perp$:
        \begin{equation*}
            \ket{\phi} = \alpha \ket{\psi} + \beta \ket{\psi^\perp} ~,
        \end{equation*}
        where $\ket{\psi} \in \mathcal H_\psi$, $\ket{\psi^\perp} \in \mathcal H^\perp$ and $\braket{\psi}{\psi^\perp} = 0$. Therefore, the action of the projector is
        \begin{equation*}
            P_\psi \ket{\phi} = \alpha P_\psi \ket{\psi} + \beta P_\psi \ket{\psi^\perp} = \alpha \ket{\psi} \underbrace{\braket{\psi}{\psi}}_1 + \beta \ket{\psi} \underbrace{\braket{\psi}{\psi^\perp}}_0 = \alpha \ket{\psi} \in \mathcal H_\psi ~.
        \end{equation*}
    \end{proof}
    Moreover, since projectors are orthogonal, we can define the projector onto the orthogonal subspace as $P_\psi^\perp = \mathbb I - P_\psi$ such that it satisfies $P_\psi P_\psi^\perp = P_\psi^\perp P_\psi = 0$. This can be generalised for a generic set of orthogonal subspaces. In fact, given an orthonormal basis $\{\ket{e_n}\}$, a projector onto an element of this basis is $P_n = \ket{e_n} \bra{e_n}$ and the orthonormality condition reads as $P_n P_m = P_m P_n = 0$ for $n \neq m$. Projectors satisfy the following properties 
    \begin{enumerate}
        \item boundness, i.e. 
            \begin{equation*}
                ||P_\psi|| < \infty~,
            \end{equation*}
        \item hermiticity, i.e. 
            \begin{equation*}
                P_\psi^\dagger = P_\psi ~,
            \end{equation*}
        \item idempotence, i.e. 
            \begin{equation}\label{idem}
                P_\psi^2 = P_\psi ~,
            \end{equation}
        \item positive defined, i.e. $\forall \ket{\phi} \in \mathcal H$
            \begin{equation*}
                \bra{\phi} P_\psi \ket{\phi} \geq 0 ~,
            \end{equation*}
        \item trace equals to $1$, i.e. 
            \begin{equation*}
                \tr P_\psi = 1 ~.
            \end{equation*}
    \end{enumerate}
    Actually, there is a theorem that ensures that an operator, such that it satifies these 5 conditions, is indeed a projectors.
    \begin{proof}
        For the boundness, $\forall \ket{\phi} \in \mathcal H$
        \begin{equation*}
            ||P_\psi \ket{\phi}||^2 = \bra{\phi} P_\psi^\dagger P_\psi \ket{\phi} = \braket{\phi}{\psi} \underbrace{\braket{\psi}{\psi}}_1 \braket{\psi}{\phi} = |\braket{\psi}{\phi}|^2 \leq ||\phi||^1 ~,
        \end{equation*}
        hence,
        \begin{equation*}
            ||P_\psi|| = \frac{||P_\psi \ket{\phi}||}{||\phi||} \leq 1 ~.
        \end{equation*}
        For the hermiticity
        \begin{equation*}
            P_\psi^\dagger = (\ket{\psi} \bra{\psi})^\dagger = \bra{\psi}^\dagger \ket{\psi}^\dagger = \ket{\psi} \bra{\psi} = P_\psi ~.
        \end{equation*}
        For the idempotence
        \begin{equation*}
            P_\psi^2 = (\ket{\psi} \bra{\psi})^2 = \ket{\psi} \underbrace{\braket{\psi}{\psi}}_1 \bra{\psi} = \ket{\psi} \bra{\psi} = P_\psi ~.
        \end{equation*}
        For the positive definedness 
        \begin{equation*}
            \bra{\phi} P_\psi \ket{\phi} = \braket{\phi}{\psi} \braket{\psi}{\phi} = |\braket{\psi}{\phi}|^2 \geq 0 ~.
        \end{equation*}
        For the trace, since it is independent from the choice of the basis, we choose $\ket{\psi} = \ket{\psi_1}$ such that $\braket{\psi}{\psi_n} = \delta_{n,1}$ and 
        \begin{equation*}
            \tr P_\psi = \sum_{n=0}^{\infty} \bra{\psi_n} P_\psi \ket{\psi_n} = \sum_{n=0}^{\infty} \underbrace{\braket{\psi_n}{\psi}}_{\delta_{n,1}} \braket{\psi}{\psi_n} = \sum_{n=0}^{\infty} \underbrace{\delta_{n,1}}_{n=1} \braket{\psi}{\psi_n} = \braket{\psi}{\psi_1} = \braket{\psi_1}{\psi_1} = 1 ~.
        \end{equation*}
    \end{proof}

    Given an orthonormal basis $\{\ket{e_n}\}_{n=1}^\infty$ of a separable Hilbert space, the trace is defined as 
    \begin{equation*}
        \tr A = \sum_{n=1}^\infty A_{nn} = \sum_{n=1}^\infty \bra{e_n} A \ket{e_n} ~.
    \end{equation*}
    It may happen that this series is not convergent. If it is convergent, the operator $A$ is called a trace-class operator. Furthermore, if it is absolute convergent, the trace is independent on the choice of the basis. Recall that in the finite-dimensional case, the trace of a matrix is always convergent and independent on the choice of the basis.

\section{Observables and time evolution}

    An observable is a linear hermitian operator $\hat A$ acting on the Hilbert space. We require the self-adjointness because, by the spectral theorem, they are always diagonalisable with a positive spectrum. This means that its eigenvalues are real and it always admits an orthonormal eigenbasis $\{\ket{\psi_n}\}$
    \begin{equation}\label{eigen}
        A \ket{\psi_n} = \lambda_n \ket{\psi_n} ~,
    \end{equation}
    where $\lambda_n \in \mathbb R$. In this way, $\forall \ket{\phi} \in \mathcal H$, we can expand it into the eigenbasis 
    \begin{equation}\label{exp}
        \ket{\phi} = \sum_{n=1}^{\infty} c_n \ket{\psi_n} ~,
    \end{equation}
    where $c_n \in \mathbb C$.
    Eigenprojectors, defined as 
    \begin{equation*}
        P_n = \ket{\psi_n} \bra{\psi_n} ~,
    \end{equation*}
    satisfy the following properties 
    \begin{enumerate}
        \item self-adjointness, i.e.
            \begin{equation*}
                P_n^\dagger = P_n ~,
            \end{equation*}
        \item orthonormality, i.e.
            \begin{equation*}
                P_n P_m = \delta_{nm} P_n ~,
            \end{equation*}
        \item completeness relation, i.e.
            \begin{equation}\label{compl}
                \sum_{n = 0}^{\infty} P_n = \mathbb I~,
            \end{equation}
        \item spectral decomposition, i.e.
            \begin{equation}\label{spec}
                \hat A = \sum_{n=0}^{\infty} \lambda_n P_n ~.
            \end{equation}
    \end{enumerate}
    Consider a quantum system in a state $\ket{\psi}$. A measurement of an observable $\hat A$ has outcomes corresponding to its eigenvalues $\lambda_n$ with probability $p_n = |c_n|^2$. Recall that $\lambda_n$ are the coefficients in~\eqref{eigen} and $c_n$ in~\eqref{exp}. Its average value is 
    \begin{equation}\label{avval}
        \av{A} = \bra{\psi} \hat A \ket{\psi} = \sum_{n} \lambda_n |c_n|^2 = \sum_{n} \lambda_n p_n ~,
    \end{equation}
    whereas its standard deviation is 
    \begin{equation*}
        (\Delta A)^2 = \av{A^2} - \av{A}^2 ~.
    \end{equation*}
    \begin{proof}
        In fact, using~\eqref{eigen} and~\eqref{exp}
        \begin{equation*}
        \begin{aligned}
            \av{A} & = \bra{\psi} \hat A \ket{\psi} = \sum_{m=0}^{\infty} c_m^* \bra{\psi_m} \sum_{n=0}^{\infty} c_n \underbrace{\hat A \ket{\psi_n}}_{\lambda_n \ket{\psi_n}} = \sum_{n=0}^{\infty} \sum_{m=0}^{\infty} \lambda_n c^*_m c^n \underbrace{\braket{\psi_m}{\psi_n}}_{\delta_{nm}} \\ & = \sum_{n=0}^{\infty} \sum_{m=0}^{\infty} \lambda_n c^*_m c^n \underbrace{\delta_{nm}}_{n=m} = \sum_{n=0}^{\infty} \lambda_n \underbrace{c^*_n c_n}_{|c_n|^2} = \sum_{n} \lambda_n |c_n|^2 ~.
        \end{aligned}
        \end{equation*}
    \end{proof}
    Notice that measurement in quantum mechanics is a destructive process, since the wave function collapses into one of the eigenstates. 

    Time evolution of a quantum system is governed by a special observable, the Hamiltonian $\hat H$, through the Schroedinger equation
    \begin{equation*}
        i \hbar \pdv{}{t} \ket{\psi(t)} = \hat H \ket{\psi(t)} ~.
    \end{equation*}
    Notice that this equation is linear, consistent with the superposition principle. It is also at first-order in time, meaning that once the initial condition $\ket{\psi(t_0)}$ is fixed, $\ket{\psi(t)}$ is completely determined.
    Moreover, for a time-independent Hamiltonian, time evolution can be equivalently expressed by a unitary operator $\hat U(t)$  
    \begin{equation}\label{tev}
        \ket{\psi(t)} = \hat U(t) \ket{\psi(0)} ~,
    \end{equation}
    where $\hat U(t) = \exp (\frac{i}{\hbar} \hat H t)$. Since it is unitary 
    \begin{equation*}
        \hat U^\dagger (t) = \exp(- \frac{i}{\hbar} \hat H t) = \hat U(-t) = U^{-1} (t) ~,
    \end{equation*}
    it preserves the probability.

\section{Density matrices and mixed states}

    The projector~\eqref{proj} is also called a density matrix $\rho_\psi$. In terms of the density matrix, the average value~\eqref{avval} of an operator $\hat A$ is 
    \begin{equation*}
        \av{A} = \bra{\psi} \hat A \ket{\psi} = \tr (\hat A \rho_\psi) ~.
    \end{equation*}
    \begin{proof}
        In fact, using~\eqref{compl}
        \begin{equation*}
        \begin{aligned}
            \av{A} & = \bra{\psi} \hat A \ket{\psi} = \bra{\psi} \mathbb I \hat A \ket{\psi} = \sum_{n=0}^{\infty} \bra{\psi} P_n \hat A \ket{\psi} = \sum_{n=0}^{\infty} \braket{\psi}{\psi_n} \bra{\psi_n} \hat A \ket{\psi} \\ & = \sum_{n=0}^{\infty} \bra{\psi_n} \hat A \underbrace{\ket{\psi} \bra{\psi}}_{\rho_\psi} \ket{\psi_n} = \sum_{n=0}^{\infty} \bra{\psi_n} \hat A \rho_\psi \ket{\psi_n} = \tr (\hat A \rho_\psi) ~,
        \end{aligned}
        \end{equation*}
        where we have exchanged brakets because they are only numbers.
    \end{proof}
    The time evolution of the density matrix is 
    \begin{equation*}
        \rho_\psi (t) = \exp(- \frac{i}{\hbar} \hat H t) \rho_\psi (0) \exp(\frac{i}{\hbar} \hat H t) ~.
    \end{equation*}
    \begin{proof}
        In fact, using~\eqref{tev}
        \begin{equation*}
            \rho_\psi (t) = \ket{\psi(t)} \bra{\psi(t)} = \exp(- \frac{i}{\hbar} \hat H t) \underbrace{\ket{\psi(0)} \bra{\psi(0)}}_{\rho_\psi (0)} \exp(\frac{i}{\hbar} \hat H t) = \exp(- \frac{i}{\hbar} \hat H t) \rho_\psi (0) \exp(\frac{i}{\hbar} \hat H t) ~.
        \end{equation*}
    \end{proof}

    A mixed state belonging to a classical mixture is a system which can be found in a state $\ket{\psi_n}$ with a probability $p_n$
    \begin{equation*}
        \{\ket{\psi_n}, p_n\} ~,
    \end{equation*}
    where $p_n \geq 0$ and $\sum_{n=0}^{\infty} p_n = 1$. The difference from a pure state is that, in a mixed state, the system is in a classical fixed state before the measurement whereas, in a pure state, the state is in a quantum superposition. The density matrix of a mixed state is 
    \begin{equation}\label{mix}
        \rho = \sum_n p_k \ket{\psi_n} \bra{\psi_n} = \sum_n p_n \rho_n ~,
    \end{equation}
    It defines a statistical ensemble. Similarly to the pure state case, it satisfies the following properties
    \begin{enumerate}
        \item boundness, i.e. 
            \begin{equation*}
                ||\rho|| < \infty~,
            \end{equation*}
        \item hermiticity, i.e. 
            \begin{equation*}
                \rho^\dagger = \rho ~,
            \end{equation*}
        \item positive defined, i.e. $\forall \ket{\phi} \in \mathcal H$
            \begin{equation*}
                \bra{\phi} \rho \ket{\phi} \geq 0 ~,
            \end{equation*}
        \item trace equals to $1$, i.e. 
            \begin{equation*}
                \tr \rho = 1 ~.
            \end{equation*}
    \end{enumerate}
    However, the idempotence property~\eqref{idem} is a particular property of only pure states. There is a theorem that states that a state is pure if and only if $\rho^2 = \rho$.
    \begin{proof}
        In the simple case of orthogonal states $\ket{\psi_n}$, i.e. $\braket{\psi_n}{\psi_m} = \delta_{nm}$, we have 
        \begin{equation*}
        \begin{aligned}
            \rho^2 & = \sum_n p_n \ket{\psi_n} \bra{\psi_n} \sum_m p_m \ket{\psi_m} \bra{\psi_m} = \sum_n \sum_m p_n p_m \ket{\psi_n} \underbrace{\braket{\psi_n}{\psi_m}}_{\delta_{nm}} \bra{\psi_m} \\ & = \sum_n \sum_m p_n p_m \ket{\psi_n} \underbrace{\delta_{nm}}_{n=m} \bra{\psi_m} = \sum_n p^2_n \ket{\psi_n} \bra{\psi_n} = \sum_n p^2_n \rho_n ~.
        \end{aligned}
        \end{equation*}
        This means that if $\rho^2 = \rho$, we obtain 
        \begin{equation*}
            p_n^2 = p_n ~,
        \end{equation*}
        which means that $p_{\overline n} = 1$ for a single $\overline n$ and for all the others $p_n = 0$ for $n \neq \overline n$, but this is indeed a pure state $\rho = \ket{\psi_{\overline n}} \bra{\psi_{\overline n}}$.
    \end{proof}
    The average value of an observable is the same as the pure states 
    \begin{equation}\label{avobs}
        \av{\hat A} = \bra{\psi} \hat A \ket{\psi} = \tr (\rho \hat A) ~.
    \end{equation}
    \begin{proof}
        In fact, 
        \begin{equation*}
            \av{\hat A} = \sum_n p_n \av{\hat A}_n = \sum_n p_n \tr(\hat A \rho_n) = \tr (\hat A \underbrace{\sum_n p_n \rho_n}_\rho) = \tr (\hat \rho) ~,
        \end{equation*}
        where we have used the linearity of the trace.
    \end{proof}

    Notice that in the classical case, the average value of an observable is~\eqref{cm:av}
    \begin{equation*}
        \av{f} = \int_{\mathcal M} d^d x ~ f(x) \rho(x) ~,
    \end{equation*}
    which shows that, in the quantum case, we have substituted the integral with the trace, the function with the observable operator and the density distribution with the density matrix.

\section{Composite systems}

    Consider a quantum system composed by $2$ particles. The total Hilbert space is the tensor product between the $2$ single particle Hilbert spaces 
    \begin{equation*}
        \mathcal H_{tot} = \mathcal H_1 \otimes \mathcal H_2 ~. 
    \end{equation*}
    Given an orthonormal basis for each Hilbert space $\{\ket{\psi_n}\} \in \mathcal H_1$ and $\{\ket{\phi_m}\} \in \mathcal H_2$, the orthonormal basis for the total Hilbert space is 
    \begin{equation*}
        \{\ket{\psi_n}_1 \ket{\phi_m}_2 = \ket{\psi_n \phi_m}\} ~,
    \end{equation*}
    such that a generic state can be expanded into this basis, $\forall \ket{\phi} \in \mathcal H_{tot}$
    \begin{equation*}
        \ket{\phi} = \sum_n \sum_m \alpha_{nm} \ket{\psi_n \phi_m} ~,
    \end{equation*}
    where $\alpha_{nm} \in \mathbb C$ and the normalisation condition reads $\sum_{nm} |\alpha_{nm}|^2 = 1$. 
    If the $2$ particle are identical, we have $\mathcal H_1 = \mathcal H_2 = \mathcal H$. Therefore, $\mathcal H_{tot} = \mathcal H^{\otimes 2}$.
    The scalar product between two states is 
    \begin{equation*}
        \braket{\psi_n \phi_m}{\psi_{n'} \phi_{m'}} = \braket{\psi_n}{\psi_{n'}}_1 \braket{\phi_m}{\phi_{m'}}_2 ~,
    \end{equation*}
    such that if the two states are orthonormal we have
    \begin{equation*}
        \braket{\psi_n \phi_m}{\psi_{n'} \phi_{m'}} = \braket{\psi_n}{\psi_{n'}}_1 \braket{\phi_m}{\phi_{m'}}_2 = \delta_{nn'} \delta_{mm'} ~. 
    \end{equation*}
    
    By linearity, we can generalise this construction for $N$ particles. However, we have to be careful for infinite dimensional Hilbert spaces, since we need the convergence of $\sum_{nm} |\alpha_{nm}|^2$ in order to remain in an Hilbert space. Under this assumption, the total Hilbert space is 
    \begin{equation}\label{comphil}
        \mathcal H_{tot} = \mathcal H_1 \otimes \ldots \otimes \mathcal H_N ~,
    \end{equation}
    its orthonormal basis is 
    \begin{equation*}
        \ket{e_{n_1}} \ldots \ket{e_{n_N}} 
    \end{equation*}
    and its scalar product is 
    \begin{equation*}
        \braket{\cdot}{\cdot} = \prod_k \braket{\cdot}{\cdot}_k ~.
    \end{equation*}
    A generic state can be expanded into the orthonormal basis, $\forall \ket{\phi} \in \mathcal H_{tot}$ 
    \begin{equation*}
        \ket{\phi} = \sum_{n_1, \ldots n_N} \alpha_{n_1, \ldots n_N} \ket{e_{n_1}} \ldots \ket{e_{n_N}} ~.
    \end{equation*}
    If all the particles are identical, we have $\mathcal H_1 = \ldots = \mathcal H_N = \mathcal H$. Therefore $\mathcal H_{tot} = \mathcal H^{\otimes N}$.

\chapter{Ensembles}

    In this chapter, we will study the same three ensembles we have treated in classical mechanics, but this time with the framework of quantum mechanics. For the microcanonical ensemble, we will analyse the density matrix and we will recover thermodynamics by means of the entropy. For the canonical ensemble, we will analyse the density matrix and we will recover thermodynamics by means of the Helmholtz free energy. For the grand canonical ensemble, we will analyse the density matrix and we will recover thermodynamics by means of the grand potential. 

    We will consider only finite volume system, since all the spectra and the degeneracies will be discrete. Of course, we will then go to the thermodynamic limit in order to reestablish contact with thermodynamics.

\section{Microcanonical ensemble}

    The microcanonical ensemble is characterised by constant $V$, $E$ and $N$. Since $N$ is fixed, we can work in the Hilbert space $\mathcal H_{tot}$ (and not in the Fock space). 
    
    Consider a time-independent Hamiltonian operator $\hat H$ and its associated energy eigenbasis $\ket{k} \in \mathcal H_{tot}$, such that
    \begin{equation*}
        \hat H \ket{k} = E_k \ket{k} ~.
    \end{equation*}
    However, there could be some degeneracy we want to consider, i.e. $E_{k,\alpha} = E_{k, \beta}$ for $\ket{k, \alpha} \neq \ket{k, \beta}$. Therefore, the latter expression can be generalised into
    \begin{equation}\label{eneigen}
        \hat H \ket{k,\alpha} = E_k \ket{k,\alpha} ~,
    \end{equation}
    where $\alpha = 1, \ldots n_k$ running over a discrete and finite set. Being orthonormal and complete, it satisfies the properties 
    \begin{equation*}
        \braket{k, \alpha}{k', \beta} = \delta_{kk'} \delta_{\alpha\beta} ~, \quad \sum_{k, \alpha} \ket{k,\alpha} \bra{k,\alpha} = \mathbb I ~.
    \end{equation*}

    The density operator for mixed states is~\eqref{mix}
    \begin{equation*}
        \rho_{mc} = \sum_{k, \alpha} p_{\alpha, k} \ket{k, \alpha} \bra{k, \alpha} ~,
    \end{equation*}
    where $p_{\alpha, k}$ is the probability associated to the eigenstate $\ket{k, \alpha}$. 
    Similarly to the classical case, since $E = E_j$ is fixed, there is an equal probability to occur for each eigenstate with the same energy. Therefore, an uniform probability means $p_{\alpha, k} = \delta_{k j} \frac{1}{n_j}$ and the microcanonical density matrix is
    \begin{equation*}
        \rho_{mc} = \sum_{k, \alpha} \delta_{k j} \frac{1}{n_j} \ket{k, \alpha} \bra{k, \alpha} = \sum_{\alpha = 1}^{n_j} \frac{1}{n_j} \ket{j, \alpha} \bra{j, \alpha} =  \frac{1}{n_j} \underbrace{ \sum_{\alpha = 1}^{n_j} \ket{j, \alpha} \bra{j, \alpha}}_{P_j} = \frac{1}{n_j} P_j ~,
    \end{equation*}
    where 
    \begin{equation*}
        P_j = \sum_{\alpha=1}^{n_j} \ket{j, \alpha} \bra{j, \alpha}
    \end{equation*} 
    is the projector onto the energy eigenspace $\mathcal H_k$. 
    Notice that in the classical case, to derive~\eqref{mc:pdd}, we were free to choose a sharped interval $[E, E + \Delta E]$ and then go to the limit $\Delta E \rightarrow 0$. In quantum mechanics, we have to be careful since the uncertainty principle holds. In fact, in order do to this procedure, we have to respect $\delta E \delta t \sim \hbar$ and we have to wait a suffiecient long time.
    Using~\eqref{spec}, we can expand observables into the eigenbasis, like the Hamiltonian
    \begin{equation}\label{endec}
        \hat H = \sum_j E_j \hat P_j 
    \end{equation}
    or the total number operator 
    \begin{equation}\label{numb}
        \hat N = \sum_j n_j \hat P_j ~.
    \end{equation}

    The average of an observable $\hat A$ in the microcanonical ensemble is 
    \begin{equation*}
        \av{A}_{mc} = \frac{1}{n_j} \sum_{\alpha=1}^{n_j} \bra{j,\alpha} \hat A \ket{j,\alpha} ~.
    \end{equation*}
    \begin{proof}
        In fact, choosing as orthonormal basis $\ket{j, \alpha}$, the trace is 
        \begin{equation*}
            \tr_{\mathcal H_{tot}} \hat A = \sum_k \bra{k, \alpha} \hat A \ket{k, \alpha} ~.
        \end{equation*}
        Therefore, using~\eqref{avobs}
        \begin{equation*}
        \begin{aligned}
            \av{A}_{mc} & = \tr_{\mathcal H_{tot}} (\hat A \rho_{mc}) = \tr_{\mathcal H_{tot}} \Big ( \hat A \frac{1}{n_j} \sum_{\alpha=1}^{n_j} \ket{j, \alpha} \bra{j, \alpha} \Big) = \frac{1}{n_j} \sum_{\alpha=1}^{n_j} \tr_{\mathcal H_{tot}} \Big ( \hat A \ket{j, \alpha} \bra{j, \alpha} \Big) \\ & = \frac{1}{n_j} \sum_{\alpha=1}^{n_j} \sum_k \bra{k, \alpha} \hat A \ket{j, \alpha} \underbrace{\braket{j, \alpha}{k, \alpha}}_{\delta_{jk}} = \frac{1}{n_j} \sum_{\alpha=1}^{n_j} \bra{j, \alpha} \hat A \ket{j, \alpha} ~.
        \end{aligned}
        \end{equation*}
    \end{proof}

    The entropy in the microcanonical ensemble is 
    \begin{equation*}
        S_{mc} = k_B \ln n_j ~,
    \end{equation*}
    where $n_j$ is as before the number of states with $E = E_j$. Notice that it is similar to the classical case~\eqref{mc:s}.
    \begin{proof}
        In fact, using~\eqref{mc:unibol} and~\eqref{avobs}
        \begin{equation*}
        \begin{aligned}
            S_{mc} = - k_B \av{\ln \rho_{mc}}_{mc} = - k_B \tr_{\mathcal H_{tot}} ( \rho_{mc} \ln \rho_{mc}) ~.
        \end{aligned}
        \end{equation*}
        In matrix notation, the density operator is 
        \begin{equation*}
        \begin{aligned}
            \rho_{mc} & = \begin{bmatrix}
                \begin{bmatrix}
                    \frac{1}{n_1} & 0 & \ldots & 0 \\
                    0 & \frac{1}{n_1} & \ldots & 0 \\
                    \ldots & \ldots & \ldots & \ldots \\
                    0 & 0 & \ldots & \frac{1}{n_1} \\
                \end{bmatrix} & 0 & \ldots & 0 & \ldots \\ 0 & 
                \ldots & \ldots & 0 & \ldots \\ 
                \ldots & \ldots & \ldots & \ldots & \ldots \\
                0 & 0 & \ldots & \begin{bmatrix}
                    \frac{1}{n_j} & 0 & \ldots & 0 \\
                    0 & \frac{1}{n_j} & \ldots & 0 \\
                    \ldots & \ldots & \ldots \\
                    0 & 0 & \ldots & \frac{1}{n_j} \\
                \end{bmatrix} & \ldots \\
                \ldots & \ldots & \ldots & \ldots & \ldots  \\
                \ldots & \ldots & \ldots & \ldots & \ldots \\
            \end{bmatrix} \\ & = \sum_j \begin{bmatrix}
                0 & 0 & \ldots & 0 & \ldots \\ 
                0 & 0 & \ldots & 0 & \ldots \\ 
                \ldots & \ldots & \ldots & \ldots & \ldots & \\
                0 & 0 & \ldots & \begin{bmatrix}
                    \frac{1}{n_j} & 0 & \ldots & 0 \\
                    0 & \frac{1}{n_j} & \ldots & 0 \\
                    \ldots & \ldots & \ldots & \ldots \\
                    0 & 0 & \ldots & \frac{1}{n_j} \\
                \end{bmatrix} & \ldots & \\
                \ldots & \ldots & \ldots & \ldots & \ldots  \\
            \end{bmatrix}
        \end{aligned} ~.
        \end{equation*}
        Now, in order to compute the logarithm of $0$, we use a trick: we define a small parameter $\epsilon$ and we make it go to zero. In this way, the limit becomes $\epsilon \ln \epsilon \xrightarrow{\epsilon \rightarrow 0} = 0$. Finally, we compute the trace 
        \begin{equation*}
        \begin{aligned}
            \tr_{\mathcal H_{tot}} ( \rho_{mc} \ln \rho_{mc}) & = \tr \begin{bmatrix}
                0 & 0 & \ldots & 0 & \ldots  \\ 
                0 & 0 & \ldots & 0 & \ldots  \\ 
                \ldots & \ldots & \ldots & \ldots & \ldots  \\
                0 & 0 & \ldots & \begin{bmatrix}
                    \frac{1}{n_j} \ln \frac{1}{n_j} & 0 & \ldots & 0 \\
                    0 & \frac{1}{n_j} \ln \frac{1}{n_j} & \ldots & 0 \\
                    \ldots & \ldots & \ldots & \ldots \\
                    0 & 0 & \ldots & \frac{1}{n_j} \ln \frac{1}{n_j} \\
                \end{bmatrix} & \ldots \\
                \ldots & \ldots & \ldots & \ldots & \ldots  \\
            \end{bmatrix} \\ & = \sum_{n_j} \frac{1}{n_j} \ln \frac{1}{n_j} = n_j \frac{1}{n_j} \ln \frac{1}{n_j} = - \ln n_j ~.
        \end{aligned}
        \end{equation*}
        Hence, we find
        \begin{equation*}
            S_{mc} = - k_B \tr_{\mathcal H_{tot}} ( \rho_{mc} \ln \rho_{mc}) = k_B \ln n_j ~.
        \end{equation*}
    \end{proof}

    Notice that entropy is always a positive function, since there is at least one state occupied $n_j \geq 1$, which implies $S \geq 0$. This solve one of the problems in exercises~\eqref{ex:negs1} and~\eqref{ex:negs2}, in which under a certain temperature, entropy becomes negative. Furthermore, in the thermodynamic limit, the important quantity is specific entropy 
    \begin{equation*}
        s = \lim_{td} \frac{S}{V} ~,
    \end{equation*}
    which is valid only if $n_j$ grows slower than exponentially with $V$.
    Finally, when $T \rightarrow 0$, we have that the most stable equilibrium state is the ground state and 
    \begin{equation*}
        S = k_B \ln {(n_0)}_j ~,
    \end{equation*}
    which agrees with the $3rd$ law of thermodynamics.

\section{Canonical ensemble}

    The canonical ensemble is characterised by constant $V$, $T$ and $N$. Since N
    is fixed, we can work in the Hilbert space $H_{tot}$ (and not in the Fock space).
    By the same arguments of the classical case, energy, which can be exchange in an external reservoir, can be in one of the eigenstates~\eqref{eneigen} with probability 
    \begin{equation}\label{prob}
        p_j \propto \exp(- \beta E_j) ~.
    \end{equation}
    Now, consider a family of projectors $\{\hat P_j\}$ of the eigenstates, the density matrix of mixed states is 
    \begin{equation*}
        \rho_c = \frac{1}{Z_N } \sum_j \exp(- \beta E_j) \hat P_j = \frac{\exp(- \beta \hat H)}{Z_N} ~,
    \end{equation*}
    where the quantum canonical partition function is 
    \begin{equation}\label{qgc:cz}
        Z_N(T,V) = \tr_{\mathcal H_{tot}} \exp(- \beta \hat H) ~.
    \end{equation}
    Notice that they are similar to the classical case~\eqref{c:pdd} and~\eqref{c:z}.
    \begin{proof}
        For a mixed state, the density matrix is~\eqref{mix}
        \begin{equation*}
            \rho_c = \sum_j p_j \hat P_j = C \sum_j \exp(- \beta E_j) \hat P_J ~,
        \end{equation*}
        where the probability is given by~\eqref{prob} and $C$ is a normalisation function.
        Using~\eqref{endec}, we have
        \begin{equation*}
        \begin{aligned}
            \rho_c & = C \sum_j \exp(- \beta E_j) \hat P_j = C \sum_j \sum_k \frac{1}{k!} (-\beta E_j)^k \underbrace{\hat P_j}_{(P_j)^k} = C \sum_j \sum_k \frac{1}{k!} (-\beta E_j \hat P_j)^k \\ & = C \sum_k \frac{1}{k!} (-\beta \sum_j E_j \hat P_j)^k = C \exp(- \beta \underbrace{\sum_j E_j \hat P_j}_{\hat H}) = C \exp(- \beta \hat H) ~,
        \end{aligned}
        \end{equation*}
        where we have used the Taylor expansion of the exponential, one of the properties of the projectors~\eqref{idem} and we have exchanged the two series.
        Finally, we set $C = \frac{1}{Z_N}$, where $Z_N$ is the quantum canonical partition function, and by the normalisation condition
        \begin{equation*}
            1 = \tr_{\mathcal H_{tot}} \rho_c = \frac{1}{Z_N} \tr_{\mathcal H_{tot}} \exp(- \beta \hat H) ~,
        \end{equation*}
        hence, we find 
        \begin{equation*}
            Z_N = \tr_{\mathcal H_{tot}} \exp(- \beta \hat H) ~.
        \end{equation*}
    \end{proof}

    Similarly to the classical case~\eqref{c:zf} and~\eqref{c:f}, we define the Helmholtz free energy
    \begin{equation}\label{qgc:cf}
        Z_N = \exp(- \beta F) ~, \quad F = - \frac{1}{\beta} \ln Z_N ~.
    \end{equation}
    The average energy is 
    \begin{equation}\label{qgc:ce}
        E = \av{\hat H}_c = - \pdv{}{\beta} \ln Z_N ~.
    \end{equation}
    Notice that it is similar to the classical case~\eqref{c:e2}.
    \begin{proof}
        Using~\eqref{avobs}, we have
        \begin{equation*}
        \begin{aligned}
            E & = \av{\hat H}_c  = \tr_{\mathcal H_{tot}} (\hat H \rho_c) = \tr_{\mathcal H_{tot}} \Big ( \hat H \frac{\exp(- \beta \hat H)}{Z_N} \Big ) = \frac{1}{Z_N} \tr_{\mathcal H_{tot}} \Big (- \pdv{}{\beta} \exp(- \beta \hat H) \Big) \\ & = - \frac{1}{Z_N} \pdv{}{\beta} \underbrace{\tr_{\mathcal H_{tot}} \exp(- \beta \hat H)}_{Z_N} = - \frac{1}{Z_N} \pdv{}{\beta} Z_N = - \pdv{}{\beta} \ln Z_N ~,
        \end{aligned}
        \end{equation*}
        where we have used the trick to extract the derivative with respect to $\beta$.
    \end{proof}

    Similarly to the classical case, the entropy is 
    \begin{equation*}
        S = \frac{E - F}{T} = \pdv{F}{T} ~.
    \end{equation*}
    \begin{proof}
        In fact, using~\eqref{mc:unibol} and~\eqref{avobs}, we have
        \begin{equation*}
        \begin{aligned}
            S_c & = - k_B \av{\ln \rho_c}_c = - k_B \tr_{\mathcal H_{tot}} (\rho_c \ln \rho_c) = - k_B \tr_{\mathcal H_{tot}} (\frac{\exp(- \beta \hat H)}{Z_N} \ln \frac{\exp(- \beta \hat H)}{Z_N}) \\ & = - k_B \tr_{\mathcal H_{tot}} \Big (\frac{\exp(- \beta \hat H)}{Z_N} (\ln \exp(- \beta \hat H) - \ln Z_N) \Big ) \\ & = - k_B \tr_{\mathcal H_{tot}} (\frac{\exp(- \beta \hat H)}{Z_N} (- \beta \hat H - \ln Z_N)) \\ & = k_B \beta ~ \underbrace{\tr_{\mathcal H_{tot}} (\frac{\exp(- \beta \hat H)}{Z_N} \hat H )}_E + k_B \tr_{\mathcal H_{tot}} (\frac{\exp(- \beta \hat H)}{Z_N} \underbrace{\ln Z_N}_{- \beta F} ) \\ & = \frac{E}{T} - k_B \beta F ~ \frac{1}{Z_N} \underbrace{\tr_{\mathcal H_{tot}} (\exp(- \beta \hat H))}_{Z_N} = \frac{E-F}{T} ~.
        \end{aligned}
        \end{equation*}
    \end{proof}
    Notice that the entropy is well-defined because the trace of the exponential of the energy eigenvalues diverges only if they are negative. Thus, we assume that $E_j \geq \min E_j = 0$.

\section{Grand canonical ensemble}

    The grand canonical ensemble is characterised by constant $V$, $T$ and $\mu$. Since $N$ is not fixed, we work in the full Fock space $\mathcal F$, which is the direct sum of fixed number of particles Hilbert space $\mathcal H_N$. However, we restrict the Hamiltonian operator in the Fock space to the condition that it conserves the number of particles, i.e. $[\hat H, \hat N] = 0$ 
    \begin{equation*}
        \hat H \Big \vert_{\mathcal F} = \hat H_N ~.
    \end{equation*}
    In this way we will have different representation of the Hamiltonian for each $N$. An example of physical system which does not satisfy this condition is a photons absorbed by an electron. 

    By the same arguments of the classical case, energy can be in one of the energy eigenstates, each for a fixed $N$,
    \begin{equation*}
        \hat H^{(N)} \ket{j, \alpha, (N)} = E_j^{(N)} \ket{j, \alpha, (N)} ~,
    \end{equation*}
    with probability 
    \begin{equation}\label{prob2}
        p_j^{(N)} \propto \exp(- \beta (E_j - \mu N)) ~.
    \end{equation}
    Now, consider a family of projectors $\{\hat P_j^{(N)}\}$ of the eigenstates
    \begin{equation*}
        \hat P_j^{N} = \sum_\alpha \ket{j, \alpha, (N)} \bra{}{j, \alpha, (N)} ~,
    \end{equation*}  
    the density matrix of mixed states is 
    \begin{equation*}
        \rho_{gc} = \frac{1}{\mathcal Z} \sum_N \sum_j \exp(- \beta (E_j - \mu N)) \hat P_j^{(N)} = \frac{\exp(- \beta (\hat H - \mu \hat N))}{\mathcal Z} ~,
    \end{equation*}
    where $z = \exp(\beta \mu)$ is the fugacity and the quantum grand canonical partition function is 
    \begin{equation}\label{qgc:z}
        \mathcal Z = \tr_{\mathcal F} \Big ( \exp(- \beta (\hat H - \mu \hat N)) \Big) = \sum_{N=0}^\infty z^N Z_N ~.
    \end{equation}
    Notice that they are similar to the classical case~\eqref{gc:pdd} and~\eqref{gc:z}.
    \begin{proof}
        For a mixed state, the density matrix is~\eqref{mix}
        \begin{equation*}
            \rho_{gc} = \sum_N \sum_j p_j \hat P_j^{(N)} = C \sum_N \sum_j \exp(- \beta (E_j^{(N)} - \mu N)) \hat P_j^{(N)} ~,
        \end{equation*}
        where the probability is given by~\eqref{prob2} and $C$ is a normalisation function.
        Using~\eqref{endec} and~\eqref{numb}, we have
        \begin{equation*}
        \begin{aligned}
            \rho_{gc} & = C \sum_N \sum_j \exp(- \beta (E_j - \mu N)) \hat P_j^{(N)} \\ & = C \sum_N \sum_j \sum_k \frac{1}{k!} (-\beta (E_j^{(N)} - \mu N))^k \underbrace{\hat P_j^{(N)}}_{(\hat P_j^{(N)})^k} \\ & = C \sum_j \sum_k \frac{1}{k!} (-\beta (E_j^{(N)} \hat P_j^{(N)} - \mu N \hat P_j^{(N)}))^k \\ & = C \sum_k \frac{1}{k!} (-\beta \sum_N \sum_j (E_j^{(N)} \hat P_j^{(N)} - \mu N \hat P_j^{(N)}))^k \\ & = C \exp(- \beta (\underbrace{\sum_j \sum_N E_j^{(N)} \hat P_j^{(N)}}_{\hat H}) - \mu \underbrace{\sum_j \sum_N N \hat P_j^{(N)}}_{\hat N}) \\ & = C \exp(- \beta (\hat H - \mu \hat N)) ~,
        \end{aligned}
        \end{equation*}
        where we have used the Taylor expansion of the exponential, one of the properties of the projectors~\eqref{idem} and we have exchanged the two series. Finally, We set $C = \frac{1}{\mathcal Z}$, where $\mathcal Z$ is the quantum canonical partition function, and by the normalisation condition 
        \begin{equation*}
            1 = \tr_{\mathcal F} \rho_{gc} = \sum_N \frac{1}{\mathcal H_{tot}} \tr_{\mathcal F} \exp(- \beta (\hat H - \mu \hat N)) ~,
        \end{equation*}
        hence, we find
        \begin{equation*}
        \begin{aligned}
            \mathcal Z & = \tr_{\mathcal F} \exp(- \beta (\hat H - \mu \hat N)) = \sum_{N=0}^{\infty} \tr_{\mathcal H_{tot}} \exp(- \beta (\hat H - \mu \hat N)) \\ & = \sum_{N=0}^{\infty} z^N \underbrace{\tr_{\mathcal H_{tot}} \exp(- \beta \hat H)}_{Z_N} = \sum_N z^N Z_N ~.
        \end{aligned}
        \end{equation*}
    \end{proof}

    The average value of an observable $\hat A$, such that it conserves the number of particles, i.e. $[\hat A, \hat N] = 0$, is
    \begin{equation}\label{qgc:av}
        \av{\hat A}_{gc} = \tr_{\mathcal F} (\hat A \rho_{gc}) = \frac{1}{\mathcal Z} \sum_{N=0}^{\infty} z^N Z_N \av{\hat A_N}_c ~,
    \end{equation}
    where $\hat A_N$ is the restriction of $hat A$ on $\mathcal H_N$.
    \begin{proof}
        In fact, using~\eqref{avobs}, we have
        \begin{equation*}
        \begin{aligned}
            \av{\hat A}_{gc} & = \tr_{\mathcal F} (\hat A \rho_{gc}) = \sum_{N=0}^{\infty} \tr_{\mathcal H_{tot}} \Big (\hat A \frac{z^N \exp(- \beta \hat H)}{\mathcal Z}) = \frac{1}{\mathcal Z} \sum_{N=0}^{\infty} z^N \tr_{\mathcal H_{tot}} (\hat A \exp(- \beta \hat H)) \\ & = \frac{1}{\mathcal Z} \sum_{N=0}^{\infty} z^N Z_N \underbrace{\frac{\tr_{\mathcal H_{tot}} (\hat A \exp(- \beta \hat H))}{Z_N}}_{\av{\hat A}_c} = \frac{1}{\mathcal Z} \sum_{N=0}^{\infty} z^N Z_N \av{\hat A}_c ~.
        \end{aligned}
        \end{equation*}
    \end{proof}
    
    Similarly to the classical case~\eqref{gc:zo} and~\eqref{gc:o}, we define the grand potential 
    \begin{equation}\label{qgc:o}
        \mathcal Z = \exp(- \beta \Omega) ~, \quad \Omega = - \frac{1}{\beta} \ln \mathcal Z ~.
    \end{equation}
    The average energy and number of particles is
    \begin{equation*}
        E - \mu N = \av{\hat H - \mu \hat N} = - \pdv{}{\beta} \ln \mathcal Z ~.
    \end{equation*}
    Notice that it is different from the classical case~\eqref{gc:e2}
    \begin{proof}
        Using~\eqref{avobs}, we have
        \begin{equation*}
        \begin{aligned}
            E - \mu N & = \av{\hat H - \mu \hat N}  = \tr_{\mathcal F} \Big ( (\hat H - \mu \hat N) \frac{\exp( - \beta (\hat H - \mu \hat N))}{\mathcal Z} \Big) \\ & = - \frac{1}{\mathcal Z} \pdv{}{\beta} \underbrace{\tr_{\mathcal F} (\exp(- \beta (\hat H - \mu \hat N)))}_{\mathcal Z} = - \frac{1}{\mathcal Z} \pdv{}{\beta} \mathcal Z  = - \pdv{}{\beta} \ln \mathcal Z ~.
        \end{aligned}
        \end{equation*}
    \end{proof}

    The entropy in the grand canonical ensemble is 
    \begin{equation*}
        S = \frac{E - \mu N - \Omega}{T} ~.
    \end{equation*}
    \begin{proof}
        In fact 
        \begin{equation*}
        \begin{aligned}
            S & = - k_B \av{\ln \rho_{gc}}_{gc} = - k_B \tr_{\mathcal F} ( \rho_{gc} \ln \rho_{gc}) \\ & = - k_B \tr_{\mathcal F} \Big ( \frac{\exp(- \beta (\hat H - \mu \hat N))}{\mathcal Z} \ln \frac{\exp(- \beta (\hat H - \mu \hat N))}{\mathcal Z} \Big) \\ & = - k_B \tr_{\mathcal F} \Big ( \frac{\exp(- \beta (\hat H - \mu \hat N))}{\mathcal Z} (\ln \exp(- \beta (\hat H - \mu \hat N)) - \ln \mathcal Z) \Big) \\ & = k_B \beta \underbrace{\tr_{\mathcal F} \frac{\exp(- \beta (\hat H - \mu \hat N))}{\mathcal Z} (\hat H - \mu \hat N)}_{E - \mu N} + k_B \underbrace{\tr_{\mathcal F} \ln \mathcal Z }_{- \beta \Omega} = \frac{E - \mu N - \Omega}{T} ~.
        \end{aligned}
        \end{equation*}
    \end{proof}

\chapter{Identical particles}

    In this chapter, we will develop the formalism to study a system composed by $N$ identical quantum particle: permutation group, which gives rise to bosons and fermions.

\section{Indistinguishability}

    In quantum mechanics, particles are said to be identical if they have all the same quantum numbers, like charge, mass, spin, etc. It follows from the uncertainty principle that identical particles are also indistinguishable, because it prevents the only way to completely distinguish each others: tracking their trajectories. This leads to properties that arise only in the quantum world, like Fermi-Dirac or Bose-Einstein statistics.

    For now, we treat distinguishable particles. A single particle living in $\mathbb R^3$ with Hilbert space $\mathcal H = L^2 (\mathbb R^3) \ni \psi(x)$ and scalar product
    \begin{equation*}
        \braket{\psi}{\phi} = \int d^3 x ~ \psi^*(x) \phi(x) ~,
    \end{equation*}
    where the normalisation condition is 
    \begin{equation*}
        ||\psi||^2 = \braket{\psi}{\psi} = \int_{\mathbb R^3} d^3 x ~ |\psi(x)|^2 < \infty ~.
    \end{equation*}
    If the system is composed by $N$ distinguishable particles living in $\mathbb R^{3N}$, using~\eqref{comphil}, the total Hilbert space is 
    \begin{equation*}
        \mathcal H_N = L^2(\mathbb R^3) \otimes \ldots \otimes L^2(\mathbb R^3) = L^2 (\mathbb R^{3N}) \ni \psi(x_1, \ldots x_N)
    \end{equation*}. 
    and an orthonormal basis is 
    \begin{equation*}
        \{u_{\alpha_1 (x_1)} \ldots u_{\alpha_N (x_N)} = u_{\alpha_1 \ldots \alpha_N} (x_1, \ldots x_N)\} ~,
    \end{equation*} 
    where $\{u_\alpha (x)\}$ is the single particle orthonormal basis. A generic state can be expanded in this basis as 
    \begin{equation*}
        \psi(x_1, \ldots x_N) = \sum_{\alpha_1 \ldots \alpha_N} c_{\alpha_1 \ldots \alpha_N} u_{\alpha_1 \ldots \alpha_N} (x_1, \ldots x_N) ~.
    \end{equation*}
    Now, we can distinguish between distinguishable and indistinguishable particles. In fact, for distinguishable particles, choosing $\alpha_1 = a$ and $\alpha_2 = b$ or viceversa, we obtain
    \begin{equation*}
        u_{\alpha_1 = a} (x_1) u_{\alpha_2 = b} (x_2) \neq u_{\alpha_1 = b} (x_1) u_{\alpha_2 = a} (x_2) ~,
    \end{equation*}
    but if the particle are indistinguishable, we have 
    \begin{equation*}
        u_{\alpha_1 = a} (x_1) u_{\alpha_2 = b} (x_2) \propto u_{\alpha_1 = b} (x_1) u_{\alpha_2 = a} (x_2) ~,
    \end{equation*}
    where the proportionality factor is due to the fact that states are the same up to a global phase factor. Therefore, indistinguishability implies that all observables must be invariant under permutation of particles and $2$ states that differ only for a permutation must have the same probability. Combining all requests means that the physical quantity to be invariant is $|\psi|^2$ and not $\psi$, so that 
    \begin{equation*}
        |\psi(P(x_1, \ldots, x_N))|^2 = |\psi(x_1, \ldots, x_N)|^2 ~,
    \end{equation*}
    where $P$ is a permutation such that $P(\mathbf x_1, \ldots, \mathbf x_N) = (\mathbf x_{i_1}, \ldots, \mathbf x_{i_N})$. Therefore, the two wave function are the same up to a globally $U(1)$ phase factor $\alpha_P$, depending on the permutation
    \begin{equation}\label{perm:phase}
        \psi(P(x_1, \ldots x_N)) = \exp(i \alpha_P) \psi (x_1, \ldots x_N) ~.
    \end{equation}

\section{Permutation group}

    The permutation of $N$ elements form a group $P_N$. In fact, the composition of $2$ permutations $PQ$ is defined as the permutation obtained by applying first $P$ and then $Q$, the identity permutation $\mathbb I$ does not change anything and the inverse is the permutation such that $P P^{-1} = \mathbb I$. This group is generated by transposition, i.e.~a swap of two consecutive elements, since any permutation $P \in P_N$ can be decomposed, into a sequence of 
    \begin{equation*}
        \sigma_i \colon (1,2,\ldots, i, i+1, \ldots N) \mapsto (1,2,\ldots, i+1, i, \ldots N) ~,
    \end{equation*}
    in the following way 
    \begin{equation}\label{perm:dec}
        P = \sigma_{\alpha_1} \sigma_{\alpha_2} \ldots ~.
    \end{equation}
    However, this decomposition is not unique, but there is a quantity that conserve in each possible decomposition: the number of transpositions is always even or odd. Therefore, we can define the sign of a permutation $\forall P \in P_N$
    \begin{equation*}
        \sgn(P) = \begin{cases}
            + 1 & \textnormal{even number of transposition in its decomposition } \\
            - 1 & \textnormal{odd number of transposition in its decomposition } \\
        \end{cases} ~.
    \end{equation*}

    \begin{example}
        Given $4$ numbers $(1,2,3,4)$, 
        \begin{enumerate}
            \item the identity is 
                \begin{equation*}
                    (1,2,3,4) \xmapsto{\mathbb I} (1,2,3,4) ~,
                \end{equation*}
            \item the inverse of  
                \begin{equation*}
                    (1,2,3,4) \xmapsto{P} (2,3,1,4) 
                \end{equation*}
                is 
                \begin{equation*}
                    (1,2,3,4) \xmapsto{P^{-1}} (3,1,2,4) ~,
                \end{equation*}
                so that 
                \begin{equation*}
                    (1,2,3,4) \xmapsto{P} (2,3,1,4) \xmapsto{P^{-1}} (1,2,3,4) ~.
                \end{equation*}
        \end{enumerate}
        Furthermore, $P$ can be decomposed into  
        \begin{equation*}
            (1,2,3,4) \xmapsto{\sigma_1} (2,1,3,4) \xmapsto{\sigma_2} (2,3,1,4) ~,
        \end{equation*}
        so that $P = \sigma_1 \sigma_2$ and its sign is $\sgn(P) = +1$. To see that it is not unique, we can find another more complicated decomposition 
        \begin{equation*}
        \begin{aligned}
            (1,2,3,4) & \xmapsto{\sigma_3} (1,2,4,3) \xmapsto{\sigma_1} (2,1,4,3) \xmapsto{\sigma_2} (2,4,1,3) \\ & \xmapsto{\sigma_3} (2,4,3,1) \xmapsto{\sigma_2} (2,3,4,1) \xmapsto{\sigma_3} (2,3,1,4)  ~,
        \end{aligned}
        \end{equation*}
        so that $P = \sigma_3 \sigma_1 \sigma_2 \sigma_3 \sigma_2 \sigma_3$.
    \end{example}

    Useful properties of transpositions are
    \begin{enumerate}
        \item if $|i - j| > 2$, which means that they are not next to each other,
        \begin{equation}\label{perm:1}
            \sigma_i \sigma_j = \sigma_j \sigma_i ~,
        \end{equation} 
        \item \begin{equation}\label{perm:2}
            \sigma_i \sigma_{i+1} \sigma_i = \sigma_{i+1} \sigma_i \sigma_{i+1} ~,
        \end{equation}
        \item \begin{equation}\label{perm:3}
            (\sigma_i)^2 = \mathbb I ~.
        \end{equation}
    \end{enumerate}
    \begin{proof}
        A transposition can be pictorially seen in Figure~\ref{fig:trasp}. Proofs can be seen in Figure~\ref{fig:traspij}, Figure~\ref{fig:traspi1} and Figure~\ref{fig:trasp2}.
    \end{proof}

    \begin{figure}[h]
        \centering
        \begin{tikzpicture}
        % Draw vertical lines
        \foreach \x/\label in {1/~~~~1~\ldots, 2.1/i, 2.6/i+1, 3.7/\ldots~N~~~~~} {
            \draw (\x,0) -- (\x,1.5) node[above] {\label};
        }
    
        % Exchange between the i and j lines
        \draw[thick, white] (2.1,0.5) -- (2.1,1);
        \draw[thick, white] (2.6,0.5) -- (2.6,1);
        \draw[thick, black] (2.1,0.5) -- (2.6,1) node[right] {$\sigma_i$} ;
        \draw[thick, black] (2.6,0.5) -- (2.1,1);
    
        % Add numbers below the lines
        \foreach \x/\label in {1/~~~~1~\ldots, 2.1/i+1, 2.6/i, 3.7/\ldots~N~~~~~} {
            \node at (\x,-0.3) {\label};
        }
    
        \end{tikzpicture}
        \caption{A pictorial diagram of a transposition $\sigma_i$.}
        \label{fig:trasp}
    \end{figure}
    
    \begin{figure}[h]
        \centering
        \begin{tikzpicture}
        % Draw vertical lines
        \foreach \x/\label in {1/~~~~1~\ldots, 2.1/i, 2.6/i+1, 3.4/j, 3.9/j+1, 5/\ldots~N~~~~~} {
            \draw (\x,0) -- (\x,2.5) node[above] {\label};
        }
    
        \foreach \x/\label in {7/~~~~1~\ldots, 8.1/i, 8.6/i+1, 9.4/j, 9.9/j+1, 11/\ldots~N~~~~~} {
            \draw (\x,0) -- (\x,2.5) node[above] {\label};
        }
    
        
        % Exchange between the i and j lines
        \draw[thick, white] (2.1,1.5) -- (2.1,2);
        \draw[thick, white] (2.6,1.5) -- (2.6,2);
        \draw[thick, black] (2.1,1.5) -- (2.6,2) node[right] {$\sigma_i$} ;
        \draw[thick, black] (2.6,1.5) -- (2.1,2);
        \draw[thick, white] (3.4,0.5) -- (3.4,1);
        \draw[thick, white] (3.9,0.5) -- (3.9,1);
        \draw[thick, black] (3.4,0.5) -- (3.9,1) node[right] {$\sigma_j$};
        \draw[thick, black] (3.9,0.5) -- (3.4,1);
    
        \draw[thick, white] (8.1,0.5) -- (8.1,1);
        \draw[thick, white] (8.6,0.5) -- (8.6,1);
        \draw[thick, black] (8.1,0.5) -- (8.6,1) node[right] {$\sigma_i$} ;
        \draw[thick, black] (8.6,0.5) -- (8.1,1);
        \draw[thick, white] (9.4,1.5) -- (9.4,2);
        \draw[thick, white] (9.9,1.5) -- (9.9,2);
        \draw[thick, black] (9.4,1.5) -- (9.9,2) node[right] {$\sigma_j$} ;
        \draw[thick, black] (9.9,1.5) -- (9.4,2);
        
        % Add numbers below the lines
        \foreach \x/\label in {1/~~~~1~\ldots, 2.1/i+1, 2.6/i, 3.4/j+1, 3.9/j, 5/\ldots~N~~~~~} {
            \node at (\x,-0.3) {\label};
        }
    
        \foreach \x/\label in {7/~~~~1~\ldots, 8.1/i+1, 8.6/i, 9.4/j+1, 9.9/j, 11/\ldots~N~~~~~} {
            \node at (\x,-0.3) {\label};
        }
        \end{tikzpicture}
        \caption{A pictorial diagram of a transposition $\sigma_i \sigma_j$ on the left and $\sigma_j \sigma_i$ on the right, where $|i - j| > 2$.}
        \label{fig:traspij}
    \end{figure}
    
    \begin{figure}[h]
        \centering
        \begin{tikzpicture}
        % Draw vertical lines
        \foreach \x/\label in {1/~~~~1~\ldots, 2.2/i, 3/i+1, 3.7/i+2, 5/\ldots~N~~~~~} {
            \draw (\x,0) -- (\x,3.5) node[above] {\label};
        }
    
        \foreach \x/\label in {7/~~~~1~\ldots, 8.2/i, 9/i+1, 9.7/i+2, 11/\ldots~N~~~~~} {
            \draw (\x,0) -- (\x,3.5) node[above] {\label};
        }
    
        
        % Exchange between the i and j lines
        \draw[thick, white] (2.2,2.5) -- (2.2,3);
        \draw[thick, white] (3,2.5) -- (3,3);
        \draw[thick, black] (2.2,2.5) -- (3,3) node[right] {$\sigma_i$} ;
        \draw[thick, black] (3,2.5) -- (2.2,3);
        \draw[thick, white] (3,1.5) -- (3,2);
        \draw[thick, white] (3.7,1.5) -- (3.7,2);
        \draw[thick, black] (3,1.5) -- (3.7,2) node[right] {$\sigma_{i+1}$};
        \draw[thick, black] (3.7,1.5) -- (3,2);
        \draw[thick, white] (2.2,0.5) -- (2.2,1);
        \draw[thick, white] (3,0.5) -- (3,1);
        \draw[thick, black] (2.2,0.5) -- (3,1) node[right] {$\sigma_i$} ;
        \draw[thick, black] (3,0.5) -- (2.2,1);
    
        \draw[thick, white] (8.2,1.5) -- (8.2,2);
        \draw[thick, white] (9,1.5) -- (9,2);
        \draw[thick, black] (8.2,1.5) -- (9,2) node[right] {$\sigma_i$} ;
        \draw[thick, black] (9,1.5) -- (8.2,2);
        \draw[thick, white] (9,0.5) -- (9,1);
        \draw[thick, white] (9.7,0.5) -- (9.7,1);
        \draw[thick, black] (9,0.5) -- (9.7,1) node[right] {$\sigma_{i+1}$};
        \draw[thick, black] (9.7,0.5) -- (9,1);
        \draw[thick, white] (9,2.5) -- (9,3);
        \draw[thick, white] (9.7,2.5) -- (9.7,3);
        \draw[thick, black] (9,2.5) -- (9.7,3) node[right] {$\sigma_{i+1}$};
        \draw[thick, black] (9.7,2.5) -- (9,3);
        
        
        % Add numbers below the lines
        \foreach \x/\label in {1/~~~~1~\ldots, 2.2/i+2, 3/i+1, 3.7/i, 5/\ldots~N~~~~~} {
            \node at (\x,-0.3) {\label};
        }
    
        \foreach \x/\label in {7/~~~~1~\ldots, 8.2/i+2, 9/i+1, 9.7/i, 11/\ldots~N~~~~~} {
            \node at (\x,-0.3) {\label};
        }
        \end{tikzpicture}
        \caption{A pictorial diagram of a transposition $\sigma_i \sigma_{i+1} \sigma_i$ on the left and $\sigma_{i+1} \sigma_i \sigma_{i+1}$ on the right.}
        \label{fig:traspi1}
    \end{figure}

    \begin{figure}[h]
        \centering
        \begin{tikzpicture}
        % Draw vertical lines
        \foreach \x/\label in {1/~~~~1~\ldots, 2.1/i, 2.6/i+1, 3.7/\ldots~N~~~~~} {
            \draw (\x,0) -- (\x,2.5) node[above] {\label};
        }
    
        
        % Exchange between the i and j lines
        \draw[thick, white] (2.1,0.5) -- (2.1,1);
        \draw[thick, white] (2.6,0.5) -- (2.6,1);
        \draw[thick, black] (2.1,0.5) -- (2.6,1) node[right] {$\sigma_i$} ;
        \draw[thick, black] (2.6,0.5) -- (2.1,1);
    
        \draw[thick, white] (2.1,1.5) -- (2.1,2);
        \draw[thick, white] (2.6,1.5) -- (2.6,2);
        \draw[thick, black] (2.1,1.5) -- (2.6,2) node[right] {$\sigma_i$} ;
        \draw[thick, black] (2.6,1.5) -- (2.1,2);
    
        
        % Add numbers below the lines
        \foreach \x/\label in {1/~~~~1~\ldots, 2.1/i, 2.6/i+1, 3.7/\ldots~N~~~~~} {
            \node at (\x,-0.3) {\label};
        }
    
        \end{tikzpicture}
        \caption{A pictorial diagram of a transposition $(\sigma_i)^2$.}
        \label{fig:trasp2}
    \end{figure}

\section{Bosons and fermions}

    Now, we are able to calculate explicitly~\eqref{perm:phase}, which is 
    \begin{equation}
        \alpha_P = \alpha_1 + \ldots \alpha_N~,
    \end{equation}
    where $\alpha_i$ is the phase factor that label the transposition $\sigma_{\alpha_i}$, i.e.
    \begin{equation*}
        \psi(\sigma_{\alpha_i}(x_1,\ldots x_N)) = \exp(i \alpha_i) \psi (x_1,\ldots x_N) ~.
    \end{equation*}
    \begin{proof}
        In fact, using~\eqref{perm:dec} and properties of the exponential, we obtain
        \begin{equation*}
        \begin{aligned}
            \psi(P(x_1,\ldots, x_N)) & = \psi((\sigma_{\alpha_1} \ldots \sigma_{\alpha_N}) (x_1,\ldots, x_N)) \\ & = \exp (i \alpha_1) \psi((\sigma_{\alpha_2} \ldots \sigma_{\alpha_N}) (x_1,\ldots, x_N)) \\ & ~~ \vdots \\ & = \exp (i \alpha_1) \ldots \exp (i \alpha_N) \psi(x_1,\ldots, x_N) \\ & = \exp (i \underbrace{(\alpha_1 + \ldots \alpha_N)}_{\alpha_P}) \psi(x_1,\ldots x_N) \\ &  = \exp (i \alpha_P) \psi(x_1,\ldots, x_N) ~.
        \end{aligned}
        \end{equation*}
    \end{proof}
    Furthermore, we can also find which are the possible values of $\alpha_P$:
    \begin{enumerate}
        \item $\alpha_P = 0$ and $\exp(i \alpha_P) = 1$, which correspond respectively to a bosonic totally symmetric wavefunction, i.e.~under $P$
        \begin{equation*}
            \psi(x_1, \ldots x_N) \xmapsto{P} (+ 1) \psi(x_1, \ldots x_N) ~,
        \end{equation*}
        \item $\alpha_P = \pi$ and $\exp(i \alpha_P) = \sgn(P)$, which correspond respectively to a fermionic totally antysymmetric wavefunction, i.e.~under $P$
        \begin{equation*}
            \psi(x_1, \ldots x_N) \xmapsto{P} \sgn(P) \psi(x_1, \ldots x_N) = \begin{cases}
                + \psi(x_1, \ldots x_N) & \sgn(P) = +1 \\
                - \psi(x_1, \ldots x_N) & \sgn(P) = - 1 \\
            \end{cases} ~.
        \end{equation*}
    \end{enumerate}
    Coming to hand a theorem that can be proved only in the realm of quantum field theory,  the spin-statistic theorem, we can associate a particular value of the spin to this two statistics: bosons, which have symmetric wavefunctions, are associated to integer spin particles, whereas fermions, which have antisymmetric wavefunctions, are associated to half-integer spin particles.
    \begin{proof}
        Using~\eqref{perm:1}
        \begin{equation*}
            \psi(x_1, \ldots x_N) \xmapsto{\sigma_i} \exp(i \alpha_i) \psi(x_1, \ldots x_N)\xmapsto{\sigma_i \sigma_j} \exp(i \alpha_i) \exp(i \alpha_j) \psi(x_1, \ldots x_N) ~,
        \end{equation*}
        \begin{equation*}
            \psi(x_1, \ldots x_N) \xmapsto{\sigma_j} \exp(i \alpha_j) \psi(x_1, \ldots x_N)\xmapsto{\sigma_j \sigma_i} \exp(i \alpha_j) \exp(i \alpha_i) \psi(x_1, \ldots x_N) ~,
        \end{equation*}
        hence 
        \begin{equation*}
            \exp(i \alpha_i) \exp(i \alpha_j) = \exp(i \alpha_j) \exp(i \alpha_i) ~,
        \end{equation*}
        which means that they commute
        \begin{equation}\label{pr1}
            \alpha_i + \alpha_j = \alpha_j +\alpha_i ~.
        \end{equation}
        Using~\eqref{perm:2}
        \begin{equation*}
        \begin{aligned}
            \psi(x_1, \ldots x_N) & \xmapsto{\sigma_i} \exp(i \alpha_i) \psi(x_1, \ldots x_N) \\ & \xmapsto{\sigma_i \sigma_{i+1}} \exp(i \alpha_i) \exp(i \alpha_{i+1}) \psi(x_1, \ldots x_N) \\ & \xmapsto{\sigma_i \sigma_{i+1} \sigma_i} \exp(i \alpha_i) \exp(i \alpha_{i+1}) \exp(i \alpha_i) \psi(x_1, \ldots x_N) ~,
        \end{aligned}
        \end{equation*}
        \begin{equation*}
        \begin{aligned}
            \psi(x_1, \ldots x_N) & \xmapsto{\sigma_{i+1}} \exp(i \alpha_{i+1}) \psi(x_1, \ldots x_N) \\ & \xmapsto{\sigma_{i+1} \sigma_i} \exp(i \alpha_{i+1}) \exp(i \alpha_i) \psi(x_1, \ldots x_N) \\ & \xmapsto{\sigma_{i+1} \sigma_i \sigma_{i+1} } \exp(i \alpha_{i+1}) \exp(i \alpha_i) \exp(i \alpha_{i+1}) \psi(x_1, \ldots x_N) ~,
        \end{aligned}
        \end{equation*}
        hence 
        \begin{equation*}
            \exp(i \alpha_i) \exp(i \alpha_{i+1}) \exp(i \alpha_i) = \exp(i \alpha_{i+1}) \exp(i \alpha_i) \exp(i \alpha_{i+1}) ~,
        \end{equation*}
        which means that 
        \begin{equation}\label{pr2}
           \alpha_i + \alpha_{i+1} + \alpha_i = \alpha_{i+1} + \alpha_i + \alpha_{i+1} ~. 
        \end{equation}
        Putting together this two properties~\eqref{pr1} and ~\eqref{pr2}, we have
        \begin{equation*}
            \alpha_i + \alpha_{i+1} + \alpha_i = \alpha_{i+1} + \alpha_i + \alpha_{i+1} ~,
        \end{equation*}
        \begin{equation*}
            \cancel{\alpha_{i+1}} + \cancel{\alpha_i }+ \alpha_i = \cancel{\alpha_{i+1}} + \cancel{\alpha_i} + \alpha_{i+1} ~,
        \end{equation*}
        \begin{equation*}
            \alpha_i = \alpha_{i+1} ~.
        \end{equation*}
        Therefore, $\forall i= 1, \ldots N-1$ and $\alpha_i \in [0, 2\pi[$ we have $\alpha_i = \alpha_{i+1} = \alpha$.
        Using~\eqref{perm:3}
        \begin{equation*}
            \exp(i \alpha)^2 = \exp (2 i \alpha) = \mathbb I = \exp(0) ~,
        \end{equation*}
        which means that 
        \begin{equation*}
            \alpha = 0, \pi ~.
        \end{equation*}
        Finally, there are only two possibilities 
        \begin{equation*}
            \psi(x_1, \ldots x_N) \xmapsto{\sigma_i} \underbrace{\exp(i 0)}_{+1} \psi(x_1, \ldots x_N) = \psi(x_1, \ldots x_N)
        \end{equation*}
        and 
        \begin{equation*}
            \psi(x_1, \ldots x_N) \xmapsto{\sigma_i} \underbrace{\exp(i \pi)}_{-1} \psi(x_1, \ldots x_N) = - \psi(x_1, \ldots x_N)~.
        \end{equation*}
    \end{proof}

\chapter{Second quantisation}

    The Hilbert space of indistinguishable particle is smaller than the distinguishable one, because we have seen that the phase factor can only have two possible values. In this chapter, we will see how we can describe such spaces, in terms of the symmetrised or antisymmetrised Hilbert space $\mathcal H_{S/A}$ in the language of first quantisation and in terms of the Fock space $\mathcal F_{B/F}$ in the language of second quantisation.

\section{Symmetric/antisymmetric Hilbert space} 

    Consider $2$ particles. If they are distinguishable, the total Hilbert space is 
    \begin{equation*}
        \mathcal H_{tot} = \mathcal H \otimes \mathcal H ~,
    \end{equation*}
    whereas if the particles are indistinguishable, we can decompose the Hilbert space into
    \begin{equation*}
        \mathcal H_{tot} = \mathcal H_S \oplus_\perp  \mathcal H_A ~.
    \end{equation*} 
    \begin{proof}
        In fact, given two states $\ket{a}_1 \in \mathcal H_1$ and $\ket{b}_2 \in \mathcal H_2$, we have 
    \begin{equation*}
    \begin{aligned}
        \ket{a}_1\ket{b}_2 & = \frac{2}{2} \ket{a}_1\ket{b}_2 + \frac{1}{2} \ket{b}_1\ket{a}_2 - \frac{1}{2} \ket{b}_1\ket{a}_2 \\ & = \underbrace{\frac{\ket{a}_1\ket{b}_2 + \ket{b}_1\ket{a}_2}{2}}_{\ket{\psi_S}} + \underbrace{\frac{\ket{a}_1\ket{b}_2 - \ket{b}_1\ket{a}_2}{2}}_{\ket{\psi_A}} \\ & = \ket{\psi_S} + \ket{\psi_A} ~.
    \end{aligned}
    \end{equation*}
    Furthermore, the permutation group for $2$ particles is $P_2 = \{\mathbb I, \sigma\}$. The symmetric part $\ket{\psi_S} \in \mathcal H_S$, since
    \begin{equation*}
        \sigma \ket{\psi_S} = \sigma \frac{\ket{a}_1\ket{b}_2 + \ket{b}_1\ket{a}_2}{2} = \frac{\ket{b}_1\ket{a}_2 + \ket{a}_1\ket{b}_2}{2} = \frac{\ket{a}_1\ket{b}_2 + \ket{b}_1\ket{a}_2}{2} = \ket{\psi_S} ~,
    \end{equation*}
    where we used the commutativity property. The antisymmetric part is $\ket{\psi_A} \in \mathcal H_A$, since
    \begin{equation*}
        \sigma \ket{\psi_A} = \sigma \frac{\ket{a}_1\ket{b}_2 - \ket{b}_1\ket{a}_2}{2} = \frac{\ket{b}_1\ket{a}_2 - \ket{a}_1\ket{b}_2}{2} = - \frac{\ket{a}_1\ket{b}_2 - \ket{b}_1\ket{a}_2}{2} = - \ket{\psi_A} ~,
    \end{equation*}
    where we used the commutativity property. Finally, the decomposition is orthogonal, since 
    \begin{equation*}
    \begin{aligned}
        \braket{\psi_S}{\psi_A} & = \frac{\bra{a}_1\bra{b}_2 + \bra{b}_1\bra{a}_2}{2} \frac{\ket{a}_1 \ket{b}_2 - \ket{b}_1 \ket{a}_2}{2} \\ & = \frac{1}{4} (\underbrace{\braket{a}{a}_1}_1 \underbrace{\braket{b}{b}_2}_1 - \braket{a}{b}_1 \braket{b}{a}_2 + \braket{b}{a}_1 \braket{a}{b}_2 - \underbrace{\braket{b}{b}_1}_1 \underbrace{\braket{a}{a}_2}_1) \\ & = \frac{1}{4} (- \braket{a}{b}_1 \braket{b}{a}_2 + \braket{b}{a}_1 \braket{a}{b}_2)  
    \end{aligned}
    \end{equation*}
    and 
    \begin{equation*}
    \begin{aligned}
        - \braket{\psi_S}{\psi_A} & = - \frac{\bra{a}_1\bra{b}_2 + \bra{b}_1\bra{a}_2}{2} \frac{\ket{a}_1 \ket{b}_2 - \ket{b}_1 \ket{a}_2}{2} \\ & = - \frac{1}{4} (\underbrace{\braket{a}{a}_1}_1 \underbrace{\braket{b}{b}_2}_1 - \braket{a}{b}_1 \braket{b}{a}_2 + \braket{b}{a}_1 \braket{a}{b}_2 - \underbrace{\braket{b}{b}_1}_1 \underbrace{\braket{a}{a}_2}_1) \\ & = - \frac{1}{4} (- \braket{a}{b}_1 \braket{b}{a}_2 + \braket{b}{a}_1 \braket{a}{b}_2) \\ & = \frac{1}{4} (- \braket{a}{b}_2 \braket{b}{a}_1 + \braket{b}{a}_2 \braket{a}{b}_1 ) ~,
    \end{aligned}
    \end{equation*}
    which means that $\braket{\psi_S}{\psi_A} = - \braket{\psi_S}{\psi_A}$. Therefore, the only solution is $\braket{\psi_S}{\psi_A} = 0$.
    \end{proof}
    
    Notice that Pauli's exclusion principle is encoded into the antisymmetric part, because if $a = b$ we have $\ket{\psi_A} = 0$.

    This decomposition is equivalent to define two orthogonal projectors: the symmetriser 
    \begin{equation*}
        \hat S \colon \mathcal H \rightarrow \mathcal H_S
    \end{equation*}
    and the antisymmetriser 
    \begin{equation*}
        \hat A \colon \mathcal H \rightarrow \mathcal H_A ~,
    \end{equation*}
    such that they satisfy the properties 
    \begin{equation}\label{projprop}
        \hat S^\dagger = \hat S~, \quad \hat A^\dagger = \hat A~, \quad \hat S^2 = \hat S~, \quad \hat A^2 = \hat A~, \quad \hat S \hat A = \hat A \hat S = 0 ~.
    \end{equation}

    Generalising for $N$ particles, if they are distinguishable, the total Hilbert space is 
    \begin{equation*}
        \mathcal H_{tot} = \mathcal H \otimes \ldots \mathcal H
    \end{equation*}
    and a state is $\ket{\psi} = \ket{a_1}_1 \ldots \ket{a_N}_N = \ket{1, \ldots N}$ where $\ket{a_j} \in \mathcal H$. 
    However, if the particle are indistinguishable, similarly to before, we can define the symmetriser
    \begin{equation*}
        \hat S \colon \ket{\psi} \mapsto \frac{1}{N!} \sum_{P \in P_N} \hat P \ket{1, \ldots, N} = \frac{1}{N!} \sum_{P \in P_N} \ket{P(1), \ldots, P(N)}
    \end{equation*}
    and the antisymmetriser
    \begin{equation*}
        \hat A \colon \ket{\psi} \mapsto \frac{1}{N!} \sum_{P \in P_N} sgn(P) \hat P \ket{1, \ldots, N} = \frac{1}{N!} \sum_{P \in P_N} sgn(P) \ket{P(1), \ldots, P(N)} ~,
    \end{equation*}
    where $\hat P: (1, \ldots N) \mapsto (P(1), \ldots P(N))$. Intuitively, this means that we have to permute all the possible indices $1, \ldots N$, keeping in mind the parity if we use the antisymmetriser. They satisfy the orthogonal projector properties~\eqref{projprop}. Notice that for $N > 2$ particles, the total Hilbert space is $\mathcal H_{tot} = \mathcal H_S \otimes \mathcal H_A \otimes \mathcal H'$, where bosons work only in $\mathcal H_S$, fermions work only in $\mathcal H_A$ and $\mathcal H'$ is not physical.

    \begin{example}
        For $N=3$, we can have $\psi_S \in \mathcal H_S$
        \begin{equation*}
            \psi_S = \hat S \psi = \psi(1,2,3) + \psi(1,3,2) + \psi(2,1,3) + \psi(2,3,1) + \psi(3,1,2) + \psi(3,2,1)
        \end{equation*}
        and $\psi_A \in \mathcal H_A$
        \begin{equation*}
            \psi_A = \hat A \psi = \psi(1,2,3) - \psi(1,3,2) - \psi(2,1,3) + \psi(2,3,1) + \psi(3,1,2) - \psi(3,2,1) ~.
        \end{equation*}
        However, we can also have $\psi' \in \mathcal H'$ such that
        \begin{equation*}
            \psi' = \psi(1,2,3) + \psi(1,3,2) -  \psi(2,1,3) - \psi(2,3,1) + \psi(3,1,2) + \psi(3,2,1) ~.
        \end{equation*}
    \end{example}
    
    For $N$ distinguishable particle, consider an orthonormal basis for the total Hilbert space
    \begin{equation*}
        \{u_{\alpha_1}(x_1) \ldots u_{\alpha_N}(x_N)\}_{\alpha_1, \ldots \alpha_N=0}^\infty ~,
    \end{equation*}
    where $\{u_{\alpha_K} (x_k)\}_{\alpha_k = 1}^\infty$ is an orthonormal basis for a single Hilbert space $\mathcal H_1$. Notice that they are labelled by the ordered set $(\alpha_1, \ldots \alpha_N)$ and we are specifying which particle is in which states. 
    On the orher hand, in order to construct an orthonormal basis for $\mathcal H_A$ and $\mathcal H_S$ for $N$ indistinguishable particles, we project the distinguishable orthonormal basis respectively with the antisymmetriser and the symmetriser
    \begin{equation}\label{onbas}
        \ket{n_1, \ldots n_j, \ldots} = C \begin{cases} \hat S \\ \hat A \end{cases} ~ u_{\alpha_1}(x_1) \ldots u_{\alpha_N}(x_N) ~,
    \end{equation}
    where $C$ is a normalisation constant. In fact, bb the properties of the projectors, they are orthonormal but they are not normalised. Therefore, we need to choose a normalisation constant
    \begin{equation*}
        C = \begin{cases}
            \sqrt{\frac{N!}{n_1! \ldots n_k! \ldots}} & \text{for}~ \mathcal H_S \\
            \sqrt{N!} & \text{for}~ \mathcal H_A \\
        \end{cases} ~.
    \end{equation*}
    On the contrary with respect to the distinguishable case, now we lose information, since we know only how many particles are in each state and not anymore which is in which state. In fact, we label the states with $n_1, \ldots n_k, \dots$ with $j=1, \ldots \infty$, which are the occupation number. For bosons, we have $n_k = 0, 1, \ldots, \infty$, whereas for fermions, we have $n_k = 0, 1$. For both cases, there is the constraint $N = \sum_k n_k$, which is an infinite sum but mostly are zero occupied. Moreover, given the set $\alpha_k$, we uniquely determine the occupation number $n_k$. On the other hand, given the occupation number $n_k$, we use the symmetric or antisymmetric properties~\eqref{onbas} to uniquely determine the state, because it is in $1-1$ correspondence to the set $n_k$. This means that we can describe the system using a different approch, based on the occupation number rathen than what we have done so far, called second quantisation, because we make a further quantisation by promoting fields to operators.

\section{Bosonic and fermionic ladder operators}

    In order to construct the Fock space, we need operators to move from an Hilbert space to another with different number of particles. By analogy with the harmonic oscillator, we introduce the ladder operators. 

    The bosonic creation and annihilation operators are the one such that they satisfy the following properties 
    \begin{equation}\label{bos}
        [\hat a, \hat a^\dagger]_- = \hat a \hat a^\dagger - \hat a^\dagger \hat a = \mathbb I~.
    \end{equation}
    Furthermore, the number operator $\hat N = \hat a^\dagger \hat a$ such that 
    \begin{equation*}
        [\hat N, \hat a] = - \hat a~, \quad [\hat N, \hat a^\dagger] = \hat a^\dagger ~.
    \end{equation*} 
    The ground state (vacuum) is 
    \begin{equation*}
        \hat a \ket{0} = 0
    \end{equation*}
    and a generic state is defined by
    \begin{equation*}
        \ket{\psi} = \frac{1}{\sqrt{n!}} (\hat a^\dagger)^n \ket{0} ~.
    \end{equation*}

    The fermionic creation and annihilation operators are the one such that they satisfy the following properties 
    \begin{equation}\label{ferm}
        [\hat a, \hat a^\dagger]_+ = \hat a \hat a^\dagger + \hat a^\dagger \hat a = \mathbb I ~.
    \end{equation}
    Furthermore, the number operator $\hat N = \hat a^\dagger \hat a$ such that 
    \begin{equation*}
        [\hat N, \hat a] = - \hat a~, \quad [\hat N, \hat a^\dagger] = \hat a^\dagger ~.
    \end{equation*} 
    The ground state (vacuum) is
    \begin{equation*}
        \hat a \ket{0} = 0 ~,
    \end{equation*}
    and a generic state is defined by
    \begin{equation*}
        \ket{\psi} = \frac{1}{\sqrt{n!}} (\hat a^\dagger)^N \ket{0} ~.
    \end{equation*}
    However, the anticommutator relation ensures the validity of the Pauli's exclusion principle. In fact, we have 
    \begin{equation*}
        \hat a^2 = (\hat a^\dagger)^2 = 0 ~.
    \end{equation*}

\section{Fock space}

    Now, we need to build the Fock space. 
    
    First, consider a single particle Hilbert space $\mathcal H$ with an orthonormal basis $\{\ket{e_k}\}_{n=1}^\infty$. To each $\ket{e_k}$, we associate annihilation and creation operators
    \begin{equation*}
        \ket{e_k} \mapsto \{\hat a_k, \hat a_k^\dagger\} ~,
    \end{equation*}
    such that they satisfy
    \begin{equation}\label{comm}
        [\hat a_k, \hat a_j]_\pm = [\hat a_k^\dagger, \hat a_j^\dagger]_\pm = 0 ~, \quad [\hat a_k, \hat a_j^\dagger]_\pm = \delta_{kj} ~,
    \end{equation}
    where the minus sign corresponds to the commutator (bosons)~\eqref{bos} and the plus sign to the anticommutator (fermions)~\eqref{ferm}. The normalised vacuum state is defined by the annihilation of every annihilation operator
    \begin{equation*}
        \hat a_k \ket{0} = 0 \quad \forall n~.
    \end{equation*}
    It generates a subspace of dimension $1$ 
    \begin{equation}\label{vac}
        \mathcal H^{(0)}_{S/A} = \{\lambda \ket{0} \colon \lambda \in \mathbb C \} ~.
    \end{equation}
    For each $\ket{e_k}$, we can also associate a number operator $\hat n_k = \hat a_k^\dagger \hat a_k$ such that 
    \begin{equation*}
        \hat n_k \hat a_k^\dagger \ket{0} = 1 \hat a_k^\dagger \ket{0} ~, \quad \hat n_{k'} \hat a_k \ket{0} = 0 \quad k' \neq k ~.
    \end{equation*}

    By analogy, we can define the $1$-particle state
    \begin{equation*}
        \ket{0, \ldots, 1_k, \ldots, 0} = \hat a_k^\dagger \ket{0, \ldots, 0} ~,
    \end{equation*}
    where explicitly means that $\ket{n_1=0, \ldots, n_k=1, \ldots}$. Continuing with this process, we can construct the whole Fock space. Notice, however, that even when we have only $2$ states, bosons and fermions behave differently. In fact, for
    \begin{equation*}
        \hat a_{k_1}^\dagger \hat a_{k_2}^\dagger \ket{0} = \ket{e_{k_1}} \ket{e_{k_2}} 
    \end{equation*}
    we have for fermions, if $k_1 = k_2 = k$
    \begin{equation*}
    (\hat a^\dagger_k)^2 \ket{0} = 0 ~,
    \end{equation*}
    whereas for bosons 
    \begin{equation*}
        (\hat a^\dagger_k)^2 \ket{0} \neq 0 ~.
    \end{equation*}
    Furthermore, if $k_1 \neq  k_2$, we have for fermions
    \begin{equation*}
        \hat a^\dagger_{k_1} \hat a^\dagger_{k_2} \ket{0} = - \hat a^\dagger_{k_2} \hat a^\dagger_{k_1} \ket{0} ~,
    \end{equation*}
    whereas for bosons 
    \begin{equation*}
        \hat a^\dagger_{k_1} \hat a^\dagger_{k_2} \ket{0} = \hat a^\dagger_{k_2} \hat a^\dagger_{k_1} \ket{0} ~.
    \end{equation*}

    The general construction can be made, recalling that there is a $1-1$ correspondence between the orthonormal basis $\{\ket{e_n}\}_{n=1}^\infty$ of $\mathcal H$ and the orthonormal basis $\{\hat a_k \ket{0}\}_{k=1}^\infty$ of $\mathcal H_{S/A}$. Hence, for $N$ particles, we define
    \begin{equation*}
        \mathcal H_{S/A}^{(N)} = \{\ket{n_1, \ldots n_k, \ldots} = \frac{1}{\sqrt{ \prod_j n_j}} (\hat a_1^\dagger)^{n_1} \ldots (\hat a_k^\dagger)^{n_k} \ldots \ket{0} \} ~.
    \end{equation*} 
    If $N$ is not fixed, like the passage from canonical to grand canonical ensemble, the total Fock space is 
    \begin{equation*}
        \mathcal F = \bigoplus_{N=0}^\infty \mathcal H^{(N)}_{S/A} ~.
    \end{equation*} 
    It satisfies the following properties 
    \begin{enumerate}
        \item orthonormality, i.e. 
            \begin{equation*}
                \braket{{n'}_1, \ldots {n'}_k, \ldots}{n_1, \ldots n_k, \ldots} = \delta_{{n'}_1, n_1} \ldots \delta_{{n'}_k, n_k} \ldots  ~,
            \end{equation*}
        \item annihilation $\hat a_k \colon \mathcal H^{(N)}_{S/A} \rightarrow \mathcal H^{(N-1)}_{S/A}$, i.e.
            \begin{equation*}
                \hat a_k \ket{n_1, \ldots n_k, \ldots} = \eta_k \sqrt{n_k} \ket{n_1, \ldots (n_k - 1), \ldots} ~,
            \end{equation*}
            where for bosons $\eta_k = 1$ and for fermions $\eta_k = (-1)^{\sum_{j < k} n_j}$,
        \item creation $\hat a_k^\dagger \colon \mathcal H^{(N)}_{S/A} \rightarrow \mathcal H^{(N+1)}_{S/A}$, i.e. for bosons
            \begin{equation*}
                \hat a^\dagger_k \ket{n_1, \ldots n_k, \ldots} = \sqrt{n_k + 1} \ket{n_1, \ldots (n_k + 1), \ldots} ~,
            \end{equation*}
            and for fermions
            \begin{equation*}
                \hat a^\dagger_k \ket{n_1, \ldots n_k, \ldots} = \eta_k \sqrt{1 - n_k} \ket{n_1, \ldots (n_k + 1), \ldots} ~,
            \end{equation*}
        \item number operator $\hat n_k = \hat a_k^\dagger \hat a_k$ such that 
            \begin{equation*}
                \hat n_k \ket{n_1, \ldots n_k, \ldots} = n_k \ket{n_1, \ldots n_k, \ldots}
            \end{equation*}
        and the total number operator $\hat N = \sum_k \hat n_k = \sum_k \hat a^\dagger_k \hat a_k$ such that 
        \begin{equation*}
            \hat N \ket{n_1, \ldots n_k, \ldots} = \Big (\sum_k n_k \Big ) \ket{n_1, \ldots n_k, \ldots} ~.
        \end{equation*}
    \end{enumerate}

\section{Field operators} 

    Having constructed the Fock space, now we are looking for observables, i.e.~operators. 
    Consider a generic particle state, given by the wavefunction $\ket{f} \in \mathcal H$. We can expand it into the basis $\{\ket{e_k}\}$, using the completeness relation, to find 
    \begin{equation*}
        \ket{f} = \ket{f} \mathbb I = \sum_k \braket{e_k}{f} \ket{e_k} = \sum_k f_k \ket{e_k}
    \end{equation*}
    which by the previous construction, is equivalent to 
    \begin{equation}\label{op1}
        \ket{f} = \sum_k f_k \hat a_k^\dagger \ket{0}
    \end{equation}
    Therefore, it is natural to define also operators expanded in the basis of the ladder operators
    \begin{equation}\label{op2}
        \hat \psi^\dagger (f) = \sum_k f_k \hat a^\dagger_k ~, \quad \hat \psi (f) = \sum_k f_k^* \hat a_k ~,
    \end{equation}
    in order to get a state $\hat \psi (f) \ket{0}$. 
    \begin{proof}
        In fact, combining~\eqref{op1} and~\eqref{op2}, we have
        \begin{equation*}
            \ket{f} = \underbrace{\sum_k f_k \hat a_k^\dagger}_{\hat \psi(f)} \ket{0} = \hat \psi (f) \ket{0} ~.
        \end{equation*}
    \end{proof}
    The related commutator relations of the operators can be written in terms of the braket product of states as
    \begin{equation}\label{commf}
        [\hat \psi (f), \hat \psi^\dagger (g)]_\pm = \braket{f}{g}\mathbb I ~.
    \end{equation}
    \begin{proof}
        In fact, using~\eqref{comm}
        \begin{equation*}
        \begin{aligned}
            [\hat \psi (f), \hat \psi^\dagger (g)]_\pm & = [\sum_k f^*_k \hat a_k, \sum_m g_m \hat a^\dagger]_\pm = \sum_k \sum_m f^*_k g_m \underbrace{[\hat a_k, \hat a^\dagger_m]}_{\delta_{km} \mathbb I} \\ & = \sum_k f^*_k g_k \mathbb I = \braket{f}{g} \mathbb I ~,
        \end{aligned}
        \end{equation*}
        where we have used 
        \begin{equation*}
            \ket{f} = \sum_k f_k \ket{e_k} ~, \quad \ket{g} = \sum_m g_m \ket{e_m} ~, \quad \braket{f}{g} = \sum_k \sum_m f^*_k g_m \underbrace{\braket{e_k}{e_m}}_{\delta_{km}} = \sum_k f^*_k g_k ~.
        \end{equation*}
    \end{proof}
    
    Now, we switch into the coordinate representation. Consider a single particle state in $\mathcal H = L^2(\mathbb R^d) \ni \psi(x)$ with an orthonormal basis $u_k(x)$ such that to each ket there are ladder operators $\hat a_k$ and $\hat a_k^\dagger$. Hence, $L^2(\mathbb R^d) \ni f(x) = \sum_k f_k u_k(x)$ and we define field operators
    \begin{equation*}
        \hat \psi(x) = \sum_k u_k (x) \hat a_k ~, \quad \hat \psi^\dagger (x) = \sum_k u_k^* (x) \hat a_k^\dagger ~,
    \end{equation*}
    which is a linear superposition of annihilation and creation operators. Actually, it is called an operator-valued function because its output is an operator. In fact, $\hat \psi(x)$ and $\hat \psi (f)$ are related by
    \begin{equation*}
        \hat \psi^\dagger (f) = \sum_k \hat a_k^\dagger f_k = \sum_k \hat a_k^\dagger \int_{\mathbb R^d} d^d x ~ u^*_k(x) f(x) = \int_{\mathbb R^d} d^d x ~ \psi^\dagger (x) \sum_k u_k^* (x) \hat a_k^\dagger ~,
    \end{equation*}
    where we have exchanged sum and integral because they are convergent. 
    The commutation relation becomes
    \begin{equation*}
        [\hat \psi(x), \hat \psi^\dagger (y)] = \delta (x - y)  \mathbb I ~.
    \end{equation*}
    \begin{proof}
        In fact,
        \begin{equation*}
        \begin{aligned}
            [\hat \psi (f), \hat \psi^\dagger (g)]_\pm & = [\int d^d x ~ f^* (x) \hat \psi(x), \int d^d y ~ g(y) \hat \psi^\dagger (y)]_\pm \\ & = \int d^d x \int d^d y ~ f^*(x) g(y) [\psi(x), \psi^\dagger (y)]  ~,
        \end{aligned}
        \end{equation*}
        which, by~\eqref{commf}, must be equal to 
        \begin{equation*}
            \braket{f}{g} = \int d^d x ~ f^* (x) g(x) ~.
        \end{equation*}
        Hence, the only possibility is
        \begin{equation*}
            [\psi(x), \psi^\dagger (y)] = \delta (x - y) \mathbb I ~.
        \end{equation*}
    \end{proof}

    \begin{example}
        Consider a plane wave basis $u(x) = \exp (i \mathbf k \cdot \mathbf x) $. Then we have $\hat \psi(x) = \sum_k \hat a_k^\dagger \exp(i \mathbf k \cdot \mathbf x)$.
    \end{example}

    Notice that field operators are basis independent.
    \begin{proof}
        In fact, consider a different basis $\{\ket{b_j}\}$ with ladder operators $\{ \hat c_j, \hat c_j^\dagger\}$, such that 
        \begin{equation*}
            \hat c_j = \sum_k \braket{b_j}{e_k} \hat a_k ~, \quad \hat c_j^\dagger = \sum_k \braket{e_k}{b_j} \hat a_k^\dagger ~.
        \end{equation*}
        Hence, expanding the wavefunction into the two basis 
        \begin{equation*}
            \ket{f} = \sum_k \braket{e_k}{f} e_k = \sum_j \braket{b_j}{f} b_j ~,
        \end{equation*}
        we find that $\hat \psi(f)$ and $\hat \psi^\dagger (f)$ are left unchanged 
        \begin{equation*}
        \begin{aligned}
            \hat \psi(f) & = \sum_k f_k^* \hat a_k = \sum_k \braket{f}{e_k} \hat a_k = \sum_k \sum_j \braket{f}{e_k} \ket{b_j} \bra{b_j} \hat a_k \\ & = \sum_j \braket{f}{b_j} \underbrace{\sum_k \braket{b_j}{e_k} \hat a_k}_{\hat b_j} = \sum_j \braket{f}{b_j} \hat b_j = \sum_j f^{'*}_j  \hat b_j  ~,
        \end{aligned}
        \end{equation*}
        and 
        \begin{equation*}
        \begin{aligned}
            \hat \psi^\dagger (f) & = \sum_k f_k \hat a_k^\dagger = \sum_k \braket{e_k}{f} \hat a_k^\dagger = \sum_k \sum_j \braket{e_k}{f} \ket{b_j} \bra{b_j} \hat a_k^\dagger \\ & = \sum_j \braket{b_j}{f} \underbrace{\sum_k \braket{e_k}{b_j} \hat a_k^\dagger}_{\hat b_j^\dagger} = \sum_j \braket{b_j}{f} \hat b_j^\dagger = \sum_j f'_j  \hat b_j^\dagger  ~,
        \end{aligned}
        \end{equation*}
    \end{proof}

    We define a one-body operator in the subspace $\mathcal H^{N}$ as 
    \begin{equation*}
        \hat O^{(1)} = \sum_{j=1}^{N} \hat O(\hat x_j, \hat p_j) ~,
    \end{equation*}
    where $\hat O(\hat x_j, \hat p_j)$ is an operator on $\mathcal H$. Since it is a sum of self-adjoint operators and thus itself self-adjoint, it exists an orthonormal basis of eigenvalues $\{u_\alpha (x)\}$ such that 
    \begin{equation}\label{evop}
        \hat O(\hat p, \hat x) u_\alpha (x) = \epsilon_\alpha u_\alpha (x) ~.
    \end{equation}

    In first quantisation language, given an orthonormal basis~\eqref{onbas}
    \begin{equation*}
        \psi_{n_1 \ldots n_k \ldots} (x_i, \ldots, x_k, \ldots) = C_N \begin{bmatrix}
            \hat S \\ \hat A
        \end{bmatrix} u_{\alpha_1} (x_1) \ldots u_{\alpha_k}(x_k) \ldots ~,
    \end{equation*}
    we have
    \begin{equation*}
        \hat O^{(1)} \psi_{n_1 \ldots n_k \ldots} (x_1, \ldots x_k, \ldots) = \Big (\sum_{j=1}^{\infty} \epsilon_j n_j \Big ) \psi_{n_1 \ldots n_k \ldots} (x_1, \ldots x_k, \ldots) ~.
    \end{equation*}
    \begin{proof}
    In fact, using~\eqref{evop}, we obtain
    \begin{equation*}
    \begin{aligned}
        \hat O^{(1)} \psi_{n_1 \ldots n_k \ldots} (x_1, \ldots x_k, \ldots) & = \Big ( \sum_{j=1}^{\infty} \hat O(\hat p_j, \hat x_j) \Big) \psi_{n_1 \ldots n_k \ldots} (x_1, \ldots x_k, \ldots) \\ & = \Big ( \sum_{j=1}^{\infty} \hat O(\hat p_j, \hat x_j) \Big) c_N \begin{bmatrix} \hat S \\ \hat A \\ \end{bmatrix} u_{\alpha_1} (x_1) \ldots u_{\alpha_k} (x_k) \ldots \\ & = c_N \begin{bmatrix} \hat S \\ \hat A \\ \end{bmatrix} \Big ( \sum_{j=1}^{\infty} \hat O(\hat p_j, \hat x_j) u_{\alpha_1} (x_1) \ldots u_{\alpha_k} (x_k) \ldots \Big) \\ & = c_N \begin{bmatrix} \hat S \\ \hat A \\ \end{bmatrix} \Big ( \sum_{j=1}^{\infty}  u_{\alpha_1} (x_1) \ldots \underbrace{\hat O(\hat p_j, \hat x_j) u_{\alpha_j} (x_j)}_{\epsilon_{\alpha_j} u_{\alpha_j} (x_j) } \ldots \Big) \\ & = \Big (\sum_{j=1}^{\infty} \epsilon_j n_j \Big ) \psi_{n_1 \ldots n_k \ldots} (x_1, \ldots x_k, \ldots) ~.
    \end{aligned}
    \end{equation*}
    \end{proof}

    In the second quantisation language, we have 
    \begin{equation}\label{onebody}
        \hat O^{(1)}_F = \sum_{j=1}^{\infty} \epsilon_j \hat n_j = \sum_{j=1}^{\infty} \epsilon_j \hat a_j^\dagger \hat a_j ~,
    \end{equation}
    where 
    \begin{equation*}
        \epsilon_j = \bra{u_j (x)} \hat O (\hat p_j, \hat x_j) \ket{u_j(x)} ~.
    \end{equation*}
    Hence, we find
    \begin{equation*}
        \hat O^{(1)}_F = \sum_{j=1}^{\infty} \bra{u_j (x)} \hat O (\hat p_j, \hat x_j) \ket{u_j(x)} \hat a_j^\dagger \hat a_j ~.
    \end{equation*}
    However, we have found a definition that is dependent of the basis, because if we choose a different basis, we obtain a different one-body operator. Therefore, we redefine the one-body operator as
    \begin{equation}\label{basind}
        \hat O^{(1)}_F = \int d^d x ~ \hat \varphi^\dagger (x) \hat O (\hat p, \hat x) \hat \varphi (x) ~,
    \end{equation}
    which this time is basis independent.
    \begin{proof}
        In fact,
        \begin{equation*}
        \begin{aligned}
            \int d^d x ~ \hat \varphi^\dagger (x) \hat O (\hat p, \hat x) \hat \varphi (x) & = \int d^d x ~ \Big ( \sum_k u_k(x) \hat a^\dagger (x) \Big ) \hat O (\hat p, \hat x) \Big ( \sum_m u_m^* (x) \hat a_m (x) \Big ) \\ & = \sum_k \sum_m \hat a_k^\dagger \hat a_m \int d^d x ~ u_k (x) \underbrace{\hat O(\hat p, \hat x) u^*_m (x)}_{\epsilon_m u^*_m (x)} \\ & = \sum_k \sum_m \hat a_k^\dagger \hat a_m \epsilon_m \underbrace{\int d^d x ~ u_k (x) u^*_m (x)}_{\delta_{km}} \\ & = \sum_k \sum_m \hat a_k^\dagger \hat a_m \epsilon_m \underbrace{\delta_{km}}_{k = m} = \sum_k\hat a_k^\dagger \hat a_k \epsilon_k = \hat O^{(1)}_F ~.
        \end{aligned}
        \end{equation*}
    \end{proof}
    In a different (non-eigen) basis
    \begin{equation*}
        \psi^\dagger (x) = \sum_k u_k (x) \hat a^\dagger_k = \sum_m v_m (x) b^\dagger_m ~,
    \end{equation*}
    the one-body operator can be written as
    \begin{equation}
        \hat O^{(1)}_F = \sum_{mm'} t_{mm'} \hat b_m^\dagger \hat b_{m'} ~,
    \end{equation}
    where the transition amplitude is
    \begin{equation}\label{genbas}
        t_{km} = \bra{v_k} \hat O (\hat p, \hat x) \ket{v_m} ~.
    \end{equation}
    \begin{proof}
        In fact,
        \begin{equation*}
        \begin{aligned}
            \int d^d x ~ \hat \varphi^\dagger (x) \hat O (\hat p, \hat x) \hat \varphi (x) & = \int d^d x ~ \Big ( \sum_m v_m(x) \hat b^\dagger (x) \Big ) \hat O (\hat p, \hat x) \Big ( \sum_m' v_{m'}^* (x) \hat b_m (x) \Big ) \\ & = \sum_{mm'} \hat b_m^\dagger \hat b_{m'} \underbrace{\int d^d x ~ v_m (x) \hat O(\hat p, \hat x) v^*_{mi} (x)}_{t_{mm'}} = \sum_{mm'} t_{mm'} \hat b_m^\dagger \hat b_{m'} ~.
        \end{aligned}
        \end{equation*}
    \end{proof}

    To summarise, the one-body operator becomes
    \begin{equation*}
        \hat O^{(1)}_F = \begin{cases}
            \sum_{m m'} t_{mm'} \hat b^\dagger_m \hat b_m & \textnormal{arbitrary basis} \\
            \sum_k \epsilon_k \hat a^\dagger_k \hat a_k & \textnormal{eigenbasis} \\
        \end{cases} ~.
    \end{equation*}

    It is also possible to find different operators than the one-body one, e.g.~ the two particles operator, used to describe interaction potential 
    \begin{equation*}
    \begin{aligned}
        \hat O^{(2)} & = \sum_{i < j} V(x_i, x_j) = \frac{1}{2} \int dx \int dy V(x,y) \hat \psi^\dagger (x) \hat \psi^\dagger (y) \psi(y) \psi (x) \\ & = \frac{1}{2} \sum_{ijkl} V_{ijkl} \hat a_i^\dagger \hat a_j^\dagger \hat a_k \hat a_l ~,
    \end{aligned}
    \end{equation*}
    where the first expression is basis-independent, while in the last one it is in the eigenbasis. 

    Examples of one-body operators are density operator, number of particles and Hamiltonian.

    The density operator of a single particle $j$ is 
    \begin{equation*}
        \hat \rho_j = \delta (x - x_j)
    \end{equation*}
    and the corresponding field operator is 
    \begin{equation*}
        \hat \varphi (x_j) = \int d^d x ~ \psi (x) \delta (x - x_j) ~.
    \end{equation*}
    Therefore, the associated one-body operator is 
    \begin{equation*}
        \hat \rho^{(1)} = \sum_{j=1}^{N} \delta (x - x_j) ~,
    \end{equation*}
    which in the basis independent definition~\eqref{basind} on the Fock space becomes
    \begin{equation*}
        \hat \rho_F = \int d^d y ~ \hat \psi^\dagger (y) \delta (x - y) \hat \psi (y) = \hat \psi^\dagger \hat \psi = \sum_{kk'} u_k^* (x) u_{k'} (x) \hat a^\dagger_k \hat a_{k'} ~.
    \end{equation*}

    The number of particle operator is 
    \begin{equation*}
    \begin{aligned}
        \hat N & = \int d^d x ~ \hat \rho^{(1)}_F (x) = \int d^d x ~ \sum_{kk'} u^*_k (x) u_{k'} (x) \hat a^\dagger_k \hat a_{k'} \\ & = \sum_{kk'} \hat a^\dagger_k \hat a_{k'} \underbrace{\int d^d x ~ u^*_k (x) u_{k'} (x)}_{\delta_{kk'}} = \sum_k \hat a^\dagger_k \hat a_k = \sum_k \hat n_k ~,
    \end{aligned}
    \end{equation*}
    which is consistent with the definition of $\rho$ since it can be seen as a density of particle whose integral is indeed the number of particles.

\chapter{Quantum gases}

    In this chapter, we will study a gas of quantum particles, beginning with a useful trick to treat it with a finite volume and then continuing studying a generic system of this kind, recovering the classical and the semiclassical limit. We will always refer to the case of non-relativistic non-interacting $3$-dimensional gases.

\section{A trick to use plane waves as basis}

    Consider the Hamiltonian of a single particle described by the wave function $\psi (x) \in L^2 (\mathbb R^d)$ 
    \begin{equation*}
        \hat H_1 = \frac{\hbar^2 \hat p^2}{2m} = - \frac{\hbar^2}{2m} \nabla_x^2 ~.
    \end{equation*}
    The associated one-body operator for $N$ particles is 
    \begin{equation*}
        \hat H = \sum_{j = 1}^{N} \frac{\hbar^2 \hat p_j^2}{2m} = - \sum_{j = 1}^{N} \frac{\hbar^2}{2m} \nabla^2_{x_j} ~.
    \end{equation*}
    which on the Fock space becomes
    \begin{equation*}
        H = \sum_k \epsilon_k \hat a_k^\dagger \hat a_k ~,
    \end{equation*}
    where $\epsilon_k$ is the energy eigenvalue of a plane wave $u_k (x)$, such that
    \begin{equation*}
        \hat H_1 u_k (x) = - \frac{\hbar^2}{2m} u_k (x) = \epsilon_k u_k (x) ~.
    \end{equation*}
    However, wave plane solutions do not belong in $u_k(x) \sim \exp(i \mathbf k \cdot \mathbf x) \notin L^2 (\mathbb R^d)$, because they are not normalisable. The trick is to go into a finite volume $V \subset \mathbb R^3$ and consider the space $L^2(V)$. The simple example is the particle in a box of length $L$ described by the coordinates $(x,y,z) \in [0, L]$. The Schoredinger's equation becomes 
    \begin{equation*}
        - \frac{\hbar^2}{2m} \nabla_x^2 u_k (x,y,z) = \epsilon_k u_k (x,y,z) ~.
    \end{equation*}
    Now, we do not choose the Dirichlet or the Neumann boundary condition, but we choose the periodic boundary conditions 
    \begin{equation*}
        \begin{cases}
            u(x=0, y, z) = u(x = L, y, z) \\
            u(x, y=0, z) = u(x, y=L, z) \\
            u(x, y, z=0) = u(x, y, z=L) \\
        \end{cases} ~,
    \end{equation*}
    which transforms the cube into a $3$dimensional-torus.
    The ansatz solution is a plane wave of the kind
    \begin{equation*}
        u_\alpha (\mathbf x) = c \exp(i \mathbf k \cdot \mathbf x) ~,
    \end{equation*}
    where $c$ is a normalisation constant and the wave vector $k$ is intepreted as
    \begin{equation*}
        \nabla^2 u_\alpha (\mathbf x) = - (k_x^2 + k_y^2 + k_z^2) u_\alpha (\mathbf x) = \epsilon_{\mathbf k} = - k^2 ~.
    \end{equation*}
    Imposing the periodic boundary conditions, we obtain 
    \begin{equation*}
        u_{\mathbf k} (0,y,z) = \cancel{c} \exp(i \cancel{(k_y y + k_z z)}) = u_{\mathbf k} (L,y,z) = \cancel{c} \exp(i (k_x L + \cancel{k_y y + k_z z})) ~,
    \end{equation*}
    hence, we have found
    \begin{equation*}
        \exp(i k_x L) = 0 
    \end{equation*}
    which has discrete solution 
    \begin{equation*}
        k_x = \frac{2 \pi}{L} n_x ~,
    \end{equation*}
    where $n \in \mathbb Z$ is an integer number. Applying the same procedure for $y$ and $z$, we have found that
    \begin{equation*}
        \mathbf k = (k_x, k_y, k_z) = \frac{2\pi}{L} \mathbf n = \frac{2\pi}{L} (n_x, n_y, n_z) 
    \end{equation*}
    where $n_x, n_y, n_z \in \mathbb Z$. Finally, the energy eigenvalues are 
    \begin{equation*}
        \epsilon_{n_x, n_y, n_z} = - \frac{4\pi^2}{L^2} (n_x^2 + n_y^2 + n_z^2) 
    \end{equation*}
    and the eigenstates are 
    \begin{equation*}
        u_{n_x, n_y, n_z} = c \exp(i \frac{2\pi}{L} (n_x x + n_y y + n_z z)) \in L^2(V) ~.
    \end{equation*}
    The normalisation constant can be found by the normalisation condition
    \begin{equation*}
        1 = ||u_{n_x, n_y, n_z} ||^2 = \int_V dx ~ dy ~ dz ~ |c|^2 |\exp(i \mathbf k \cdot \mathbf x)|^2 = |c|^2 V ~.
    \end{equation*}
    which implies that 
    \begin{equation*}
        C = \frac{1}{\sqrt{V}} ~.
    \end{equation*}
    Hence, the onebody operator is 
    \begin{equation*}
        \hat O^{(1)} = \sum_{j=1}^{N} \frac{\hat p^2_j}{2m} = - \sum_{j=1}^{N} \frac{\hbar^2}{2m} \nabla^2_{\mathbf x_j} ~,
    \end{equation*}
    and choosing the orthonormal basis of wavefunctions, we have in the Fock space 
    \begin{equation*}
        \hat O_F = \sum_{\mathbf k} \epsilon_{\mathbf k} \hat a^\dagger_{\mathbf k} \hat a_{\mathbf k} ~,
    \end{equation*}
    where $\mathbf k = \frac{2\pi}{L} (n_x, n_y, n_z)$ and $\epsilon_{\mathbf k} = \frac{\hbar^2}{2m} k^2$. In order to consider the infinite $L^2 (\mathbb R^d)$, we can go to the limit $V \rightarrow \infty$, with the help of distribution theory. We will see how to develop this procedure when we will treat non-relativistic non-interacting quantum gases.
    
\section{Generic quantum gas}

    For now, consider a generic quantum gas. The Hamiltonian operator of one particle, labelled by $k$, is 
    \begin{equation*}
        \hat H_k = \epsilon_k \hat n_k = \epsilon_k \hat a^\dagger_k \hat a_k ~,
    \end{equation*}
    where $\hat n_k = \hat a^\dagger_k \hat a_k$ is the number operator and $\epsilon_k$ is the energy eigenvalue associated to the eigenbasis $\ket{u_k(x)}$ by the eigenvalue relation
    \begin{equation*}
        \hat H_k \ket{u_k(x)} = \epsilon_k \ket{u_k(x)} ~.
    \end{equation*} 
    Therefore, the Hamiltonian one-body operator~\eqref{onebody} in the Fock space $\mathcal F$, created by the ladder operators $\hat a^\dagger_k$ and $\hat a_k$ associated each to the element of the eigenbasis $\ket{u_k(x)}$, is 
    \begin{equation}\label{opham}
        \hat H = \sum_k \hat H_k =  \sum_k \epsilon_k \hat n_k =  \sum_k \epsilon_k \hat a^\dagger_k \hat a_k ~.
    \end{equation}
    We can also define the total number onebody operator in the Fock space $\mathcal F$, which is 
    \begin{equation}\label{opnumb}
        \hat N = \sum_k \hat n_k ~,
    \end{equation}
    where their eignevalues are given with respect to an orthonormal basis $\ket{n_1, \ldots n_k, \ldots}$ by the eigenvalue relation
    \begin{equation*}
        \hat n_k \ket{n_1, \ldots n_k, \ldots} = n_k \ket{n_1, \ldots n_k, \ldots} ~.
    \end{equation*}
    Notice that it is independe from the choice of basis. In particular, we distinguish the bosonic and the fermionic case
    \begin{equation}\label{bosferm}
        n_k = \begin{cases}
            0,1,2,3,\ldots & \textnormal{bosons} \\
            0,1 & \textnormal{fermions} \\
        \end{cases} ~.
    \end{equation}

    Calculation in the grand canonical ensemble are easier and, since all ensembles are equivalent, we exploit the grancanonical ensemble. The grand canonical partition function is 
    \begin{equation*}
        \mathcal Z = \tr_{\mathcal F} \exp(- \beta (\hat H - \mu \hat N)) = \prod_k \sum_{n_1, \ldots n_k, \ldots} \exp(-\beta(\epsilon_k - \mu)n_k) ~.
    \end{equation*}
    \begin{proof}
        By definition~\eqref{qgc:z}~, using~\eqref{opham} and~\eqref{opnumb} we have
        \begin{equation*}
        \begin{aligned}
            \mathcal Z & = \tr_{\mathcal F} \exp(- \beta (\hat H - \mu \hat N)) = \sum_{n_1, \ldots n_k, \ldots} \bra{n_1, \ldots n_k, \ldots} \exp(- \beta (\hat H - \mu \hat N)) \ket{n_1, \ldots n_k, \ldots} \\ & = \sum_{n_1, \ldots n_k, \ldots} \bra{n_1, \ldots n_k, \ldots} \exp(- \beta \sum_k (\epsilon_k - \mu) \underbrace{\hat n_k) \ket{n_1, \ldots n_k, \ldots}}_{n_k \ket{n_1, \ldots n_k, \ldots}} \\ & = \sum_{n_1, \ldots n_k, \ldots} \bra{n_1, \ldots n_k, \ldots} \underbrace{\exp(- \beta \sum_k}_{\prod_k \exp} (\epsilon_k - \mu) n_k) \ket{n_1, \ldots n_k, \ldots} \\ & = \sum_{n_1, \ldots n_k, \ldots} \prod_k \exp(\beta (\epsilon_k - \mu) n_k) \braket{n_1, \ldots n_k, \ldots}{n_1, \ldots n_k, \ldots} \\ & = \prod_k \sum_{n_1, \ldots n_k, \ldots} \exp(-\beta(\epsilon_k - \mu)n_k) ~,
        \end{aligned}
        \end{equation*}
        where in the last passage, we have switched the product with the sum because $n_1, \ldots, n_k, \ldots$ are independent.
    \end{proof}
    In particular, if we distinguish the bosonic and the fermionic case, wwe obtain
    \begin{equation*}
        \mathcal Z = \begin{cases}
            \prod_k \frac{1}{1 - \exp (- \beta (\epsilon_k - \mu))} & \textnormal{bosons} \\
            \prod_k \Big (1 + \exp (- \beta (\epsilon_k - \mu)) \Big ) & \textnormal{fermions} \\
        \end{cases} ~,
    \end{equation*}
    or, in compact notation, 
    \begin{equation*}
        \mathcal Z_\mp = \prod_k \Big ( 1 \mp \exp(- \beta (\epsilon_k - \mu) ) \Big)^{\mp 1} ~,
    \end{equation*}
    where the upper sign is associated to bosons and the other to fermions.
    \begin{proof}
        For fermions~\eqref{bosferm}, $n_k = 0, 1$, we have
        \begin{equation*}
            \mathcal Z_+ = \prod_k \sum_{n_1, \ldots n_k, \ldots = 0}^1 \exp(-\beta(\epsilon_k - \mu)n_k) = \prod_k \Big (1 + \exp (- \beta (\epsilon_k - \mu)) \Big ) ~.
        \end{equation*}
        For bosons~\eqref{bosferm}, $n_k = 0, 1, 2, \ldots$, we have
        \begin{equation*}
        \begin{aligned}
            \mathcal Z_- & = \prod_k \sum_{n_1, \ldots n_k, \ldots = 0}^\infty \exp(-\beta(\epsilon_k - \mu)n_k) = \prod_k \underbrace{\sum_{n_1, \ldots n_k, \ldots = 0}^\infty \exp(-\beta(\epsilon_k - \mu))^{n_k}}_{\textnormal{geometrical series}} \\ & = \prod_k \frac{1}{1 - \exp (- \beta (\epsilon_k - \mu))} ~.
        \end{aligned}
        \end{equation*}
        Convergence of the both expression implies that the spectrum is bounded from below and unbounded from above. In addition, the convergence of the geometrical series implies also that $\mu < \min \epsilon_k = 0$, which we have set to zero for convenience.
    \end{proof}
    The grand potential is 
    \begin{equation}\label{o1}
        \Omega_\mp = -\frac{1}{\beta} \ln \mathcal Z_\mp = \pm \frac{1}{\beta} \sum_k \ln \Big (1 \mp \exp (-\beta (\epsilon_k - \mu)) \Big) ~.
    \end{equation}
    \begin{proof}
        In fact, using~\eqref{qgc:o}, we have
        \begin{equation*}
        \begin{aligned}
            \Omega_\mp & = -\frac{1}{\beta} \ln \mathcal Z_\mp= - \frac{1}{\beta} \underbrace{\ln \Big (\prod_k}_{\sum_k \ln} ( 1 \mp \exp(- \beta (\epsilon_k - \mu) ))^{\mp 1} \Big ) \\ & = - (\mp) \sum_k \ln \Big (1 \mp \exp (-\beta (\epsilon_k - \mu))) = \pm \frac{1}{\beta} \sum_k \ln \Big (1 \mp \exp (-\beta (\epsilon_k - \mu)) \Big) ~.
        \end{aligned}
        \end{equation*}
    \end{proof}

    The grand canonical average number of particle in an energy level state $\overline k$ is 
    \begin{equation}\label{fdbe}
        \av{\hat n_{\overline k}}_{gc} = \frac{1}{\exp(\beta(\epsilon_k - \mu))\mp 1} ~.
    \end{equation}
    These distributions are called Fermi-Dirac distribution for fermions with $+$ and Bose-Einstein distribution for bosons with $-$.
    \begin{proof}
        In fact, using~\eqref{qgc:av} and~\eqref{qgc:o}, we have
        \begin{equation*}
        \begin{aligned}
            \av{\hat n_{\overline k}}_{gc} & = \tr_{\mathcal F} \Big (\hat n_{\overline k} \frac{\exp (-\beta \sum_k (\epsilon_k - \mu) \hat n_k)}{\mathcal Z} \Big) = \frac{1}{\mathcal Z} \tr_{\mathcal F} \Big (- \frac{1}{\beta} \pdv{}{\epsilon_{\overline k}} \exp (-\beta \sum_k (\epsilon_k - \mu) \hat n_k) \Big) \\ & = - \frac{1}{\beta \mathcal Z} \pdv{}{\epsilon_{\overline k}} \underbrace{\tr_{\mathcal F} \Big ( \exp (-\beta \sum_k (\epsilon_k - \mu) \hat n_k) \Big)}_{\mathcal Z} = - \frac{1}{\beta \mathcal Z} \pdv{}{\epsilon_{\overline k}} \mathcal Z = \pdv{}{\epsilon_{\overline k}} \underbrace{\Big ( - \frac{\ln \mathcal Z}{\beta} \Big)}_\Omega = \pdv{}{\epsilon_{\overline k}} \Omega ~.
        \end{aligned}
        \end{equation*}
        Hence, we find, 
        \begin{equation*}
        \begin{aligned}
            \pdv{}{\epsilon_{\overline k}} \Omega & = \pdv{}{\epsilon_{\overline k}}  \Big (\pm \frac{1}{\beta} \sum_k \ln (1 \mp \exp (-\beta (\epsilon_k - \mu)) ) \Big ) = \pm \frac{1}{\beta} (- \beta) \frac{\exp(- \beta(\epsilon_k - \mu))}{1 \mp \exp(- \beta(\epsilon_k - \mu))} \\ & = \mp \frac{1}{1 \mp \exp(\beta(\epsilon_k - \mu))} = \frac{1}{\exp(\beta(\epsilon_k - \mu))\mp 1} ~.
        \end{aligned}
        \end{equation*}
    \end{proof}
    Consequently, the average total number of particle is 
    \begin{equation}\label{qg:n1}
        N = \av{\hat N}_{gc} = \av{\sum_k \hat n_k}_{gc} = \sum_k \frac{1}{\exp(\beta(\epsilon_k - \mu))\mp 1} ~.
    \end{equation}
    The average energy is 
    \begin{equation*}\label{qg:e1}
        E = \av{\hat H}_{gc} = \tr_{\mathcal F} \Big (\hat H \frac{\exp (-\beta (\hat H - \mu \hat N))}{\mathcal Z} \Big) = \sum_k \epsilon_k \av{\hat n_k}
    \end{equation*}
    \begin{proof}
        In fact, using~\eqref{qgc:av}
        \begin{equation*}
        \begin{aligned}
            E & = \av{\hat H}_{gc} = \tr_{\mathcal F} \Big (\hat H \frac{\exp (-\beta (\hat H - \mu \hat N))}{\mathcal Z} \Big) = \frac{1}{\mathcal Z} \tr_{\mathcal F} \Big (- \pdv{}{\beta} \exp (-\beta (\hat H - \mu \hat N)) \Big) \\ &  = - \frac{1}{\mathcal Z} \pdv{}{\beta}\underbrace{ \tr_{\mathcal F} \Big (\exp (-\beta (\hat H - \mu \hat N)) \Big)}_{\mathcal Z} = - \pdv{}{\beta} \Big \vert_z \ln \mathcal Z \\ & =  - \pdv{}{\beta} \Big \vert_z (\mp \sum_k \ln \Big (1 \mp \exp (-\beta (\epsilon_k - \mu)) \Big))  = \mp \sum_k \frac{\epsilon_k \exp (-\beta (\epsilon_k - \mu))}{1 \mp \exp (-\beta (\epsilon_k - \mu))}  \\ & = \sum_k \frac{\epsilon_k}{\exp (\beta (\epsilon_k - \mu)) \mp 1} = \sum_k \epsilon_l \av{\hat n_k}
        \end{aligned}
        \end{equation*}
        where we have kept the fugacity $z$ constant.
    \end{proof}

    The grand canonical partition function could be also be written in a basis independent form
    \begin{equation*}
        \mathcal Z_{\mp} = \det (1 + \exp(-\beta \hat \gamma))^{\mp 1}  ~,
    \end{equation*}
    where $\hat \gamma$ is defined as $\bra{m} \hat \gamma \ket{m'} = t_{mm'} - \mu \delta_{mm'}$ with coefficient taken from~\eqref{genbas}. Therefore, using the identity 
    \begin{equation*}
        \det \hat A = \exp (\tr (\ln \hat A))~,
    \end{equation*}
    we can recover all the formulas of this section in a basis-independent form. For example, we can write for one particle
    \begin{equation*}
        \hat \rho_1 = \frac{1}{\exp(\beta \hat \gamma) \mp 1} ~, \quad \hat N = \tr \hat \rho_1 ~, \quad E = \tr (\hat \rho_1 \hat t) ~, 
    \end{equation*}
    \begin{equation*}
        S = \tr (\beta \hat \gamma \hat \rho_1 \mp \ln (1 \mp \exp(-\beta \hat \gamma))) = \tr (- \hat \rho_1 \ln \hat \rho_1 \pm (1 \pm \hat \rho_1) \ln (1 \pm \hat \rho_1)) ~,
    \end{equation*}
    \begin{equation*}
        \Omega = \frac{1}{\beta} \tr(\beta \hat \gamma \hat \rho_1 + \hat \rho_1 \ln \hat \rho_1 \mp (1 \pm \hat \rho_1) \ln (1 \pm \hat \rho_1)) ~.
    \end{equation*}

\section{Non-relativistic non-interacting quantum gas}

    So far, we have made computations for a generic quantum gas. From now on, we will deal with non-relativistic non-interacting $3$-dimensional quantum gas. As mentioned before, we have only made calculations with a finite volme, with energy eigenvalues 
    \begin{equation*}
        \epsilon_k = \frac{\hbar^2 k^2}{2m} ~ \quad \mathbf k = \frac{2\pi}{L} \mathbf n ~,
    \end{equation*}
    where $\mathbf n = (n_1, n_2, n_3) \in \mathbb Z^3$. However, now we have to go in the thermodynamic limit, in which the spectrum $\mathbf k$ becomes continuous, but $\mathbf n$ not, because
    \begin{equation*}
        \Delta k_i = \frac{2\pi}{L} (n_i + 1 - n_i) = \frac{2\pi}{L} ~.
    \end{equation*}

    The standard procedure is the following one: sums in $k$ becomes integrals in $dk$ with a constant
    \begin{equation}\label{tdlim}
        \sum_k = \sum_{n_1, n_2, n_2 = -\infty}^\infty \rightarrow \frac{V}{2\pi^2} \int dk~k^2 ~.
    \end{equation}
    \begin{proof}
        In fact, in $1$ dimensional we have 
        \begin{equation*}
        \begin{aligned}
            \sum_{n_1} \underbrace{\Delta n_1}_1 = \sum_{k_1} \frac{L}{2\pi} \Delta k_1 \rightarrow \frac{L}{2\pi} \int dk_1 ~.
        \end{aligned}
        \end{equation*}
        Similarly, in the $3$-dimensional case
        \begin{equation*}
        \begin{aligned}
            \sum_{n_1, n_2, n_3=- \infty}^\infty \underbrace{\Delta n_1 \Delta n_2 \Delta n_3}_1 & \rightarrow \Big ( \frac{L}{2\pi} \Big)^3 \int dk_1 dk_2 dk_3 = \Big ( \frac{L}{2\pi} \Big)^3 \int dk^3 \\ & = \Big ( \frac{L}{2\pi} \Big)^3 4 \pi \int dk ~ k^2 = \frac{V}{2\pi^2} \int dk~k^2 ~. 
        \end{aligned}
        \end{equation*}
    \end{proof}

    Using this procedure, we can calculate the grand potential
    \begin{equation}\label{qg:o2}
        \Omega = - \frac{2}{3} A V \int_0^\infty d\epsilon ~ \frac{\epsilon^{3/2}}{\exp(\beta(\epsilon - \mu)) \mp 1} ~,
    \end{equation}
    where we have called
    \begin{equation*}
        A = \frac{g}{4\pi^2} \Big (\frac{2m}{\hbar^2} \Big)^{3/2}
    \end{equation*}
    \begin{proof}
        In fact, using~\eqref{o1} and~\eqref{tdlim}, we have
        \begin{equation*}
        \begin{aligned}
            \Omega & = \pm \frac{1}{\beta} \sum_k \ln \Big (1 \mp \exp (-\beta (\epsilon_k - \mu)) \Big) \\ & \rightarrow \pm \frac{1}{\beta} \frac{V}{2\pi^2} \int_{-\infty}^\infty dk ~ k^2 \ln \Big (1 \mp \exp (-\beta (\epsilon_k - \mu)) \Big) ~.
        \end{aligned}
        \end{equation*}
        Now, we make a change of variable into
        \begin{equation*}
            \epsilon = \frac{\hbar^2 k^2}{2m} ~, \quad k^2 dk = \frac{1}{2} \Big (\frac{2m}{\hbar^2}\Big)^{3/2} \epsilon^{1/2} d\epsilon ~,
        \end{equation*}
        hence, we obtain 
        \begin{equation*}
        \begin{aligned}
            \Omega & = \pm \frac{V}{\beta} \underbrace{\frac{1}{2\pi^2} \frac{1}{2} \Big (\frac{2m}{\hbar^2}\Big)^{3/2}}_A \int_0^\infty d\epsilon ~ \epsilon^{1/2} \ln \Big (1 \mp \exp (-\beta (\epsilon - \mu)) \Big) \\ & = \pm \frac{A V}{\beta} \int_0^\infty d\epsilon ~ \epsilon^{1/2} \ln \Big (1 \mp \exp (-\beta (\epsilon - \mu)) \Big) \\ & = \pm \frac{AV}{\beta} (\frac{2}{3} \underbrace{\epsilon^{3/2}}_{0 \textnormal{ for } \epsilon = 0} \underbrace{\ln  (1 \mp \exp (-\beta (\epsilon - \mu)))}_{0 \textnormal{ for } \epsilon = \infty} \Big \vert_0^\infty ) \\ & \quad \mp \frac{AV}{\beta} \frac{2}{3} \int_0^\infty d\epsilon ~ \frac{\epsilon^{3/2}}{1 \mp \exp(- \beta (\epsilon - \mu))} \beta (\pm 1) \exp(-\beta(\epsilon - \mu)) \\ & = - \frac{2}{3} A V \int_0^\infty d\epsilon ~ \frac{\epsilon^{3/2}}{\exp(\beta(\epsilon - \mu)) \mp 1} ~,
        \end{aligned}
        \end{equation*}
        where we have integrated by parts and, introducing the degeneracy ($g = 2s+1$ for spin particles), we have called 
        \begin{equation*}
            A = \frac{g}{4\pi^2} \Big (\frac{2m}{\hbar^2}\Big)^{3/2} ~. 
        \end{equation*}
    \end{proof}

    The equation of state reads as 
    \begin{equation}\label{qg:e2}
        \Omega = - pV = - \frac{2}{3} E ~.
    \end{equation}
    \begin{proof}
        In fact, using~\eqref{qg:e1} and~\eqref{tdlim}, we have 
        \begin{equation*}
        \begin{aligned}
            E = \sum_k \frac{\epsilon_k}{\exp(\beta(\epsilon_k - \mu)) \pm 1} \rightarrow \frac{V}{2\pi^2} \int_{-\infty}^\infty dk~ k^2 \frac{\epsilon}{\exp(\beta(\epsilon - \mu)) \mp 1} ~.
        \end{aligned}
        \end{equation*}
        Now, we make a change of variable into
        \begin{equation*}
            \epsilon = \frac{\hbar^2 k^2}{2m} ~, \quad k^2 dk = \frac{1}{2} \Big (\frac{2m}{\hbar^2}\Big)^{3/2} \epsilon^{1/2} d\epsilon ~,
        \end{equation*}
        hence, comparing with~\eqref{qg:o2} and~\eqref{td:o2}, we obtain
        \begin{equation*}
        \begin{aligned}
            E & = V \underbrace{ \frac{1}{2\pi^2} \frac{1}{2} \Big (\frac{2m}{\hbar^2}\Big)^{3/2}}_A \int_0^\infty d\epsilon ~ \frac{\epsilon^{3/2}}{\exp(\beta(\epsilon - \mu)) \mp 1} \\ & = A V \int_0^\infty d\epsilon ~ \frac{\epsilon^{3/2}}{\exp(\beta(\epsilon - \mu)) \pm 1} = - \frac{3}{2} \Omega = \frac{3}{2} p V ~.
        \end{aligned}
        \end{equation*}
    \end{proof}

    The total number of particles is 
    \begin{equation}\label{qg:n2}
        N = A V \int_0^\infty d\epsilon ~ \frac{\epsilon^{1/2}}{\exp(\beta(\epsilon_k - \mu)) \pm 1}  ~.
    \end{equation}
    \begin{proof}
        In fact, using~\eqref{qg:n1} and~\eqref{tdlim}, we have 
        \begin{equation*}
        \begin{aligned}
            N = \sum_k \frac{1}{\exp(\beta(\epsilon_k - \mu)) \pm 1} \rightarrow \frac{V}{2\pi^2} \int_{-\infty}^\infty dk~ k^2 \frac{1}{\exp(\beta(\epsilon - \mu)) \mp 1} ~.
        \end{aligned}
        \end{equation*}
        Now, we make a change of variable into
        \begin{equation*}
            \epsilon = \frac{\hbar^2 k^2}{2m} ~, \quad k^2 dk = \frac{1}{2} \Big (\frac{2m}{\hbar^2}\Big)^{3/2} \epsilon^{1/2} d\epsilon ~,
        \end{equation*}
        hence, we obtain
        \begin{equation*}
            N = V \underbrace{ \frac{1}{2\pi^2} \frac{1}{2} \Big (\frac{2m}{\hbar^2}\Big)^{3/2}}_A \int_0^\infty d\epsilon ~ \frac{\epsilon^{1/2}}{\exp(\beta(\epsilon - \mu)) \mp 1} = A V \int_0^\infty d\epsilon ~ \frac{\epsilon^{1/2}}{\exp(\beta(\epsilon - \mu)) \pm 1} ~.
        \end{equation*}
    \end{proof}

    To summarise, defining 
    \begin{equation*}
        n (\epsilon) = \frac{1}{\exp(\beta(\epsilon - \mu)) \mp 1} ~,
    \end{equation*}
    we have 
    \begin{equation*}
        \Omega = - \frac{2}{3} A V \int_0^\infty d\epsilon ~ \epsilon^{3/2} n(\epsilon) ~, \quad E = A V \int_0^\infty d\epsilon ~ \epsilon^{3/2} n(\epsilon) ~, \quad N = A V \int_0^\infty d\epsilon ~ \epsilon^{1/2} n(\epsilon) ~.
    \end{equation*}

\section{Expansion in power series of the fugacity}

    In order to study quantum gas, it is useful to expand this integral expression in power series of the fugacity $z = \exp(\beta \mu) \geq 0$. 

    For bosons, we have
    \begin{equation}\label{ser}
        A \int_0^\infty d\epsilon ~ \epsilon^{1/2} n(\epsilon) = \frac{g}{\lambda_T^3} f_{3/2}^\pm ~, \quad A \int_0^\infty d\epsilon ~ \epsilon^{3/2} n(\epsilon) = \frac{g}{\lambda_T^3 \beta} f_{5/2}^{\pm} ~,
    \end{equation}
    where 
    \begin{equation}\label{ser2}
        f_l^\pm = \sum_{n=0}^\infty (\pm 1)^n \frac{z^{n+1}}{(n+1)^l} ~.
    \end{equation}
    Notice that for bosons, the convergence of the series implies $z < 1$, which means $\mu > 0$.
    \begin{proof}
        For $\epsilon^{1/2}$, we have 
        \begin{equation*}
        \begin{aligned}
            A \int_0^\infty d\epsilon ~ \epsilon^{1/2} n(\epsilon) & = \frac{g}{4 \pi^2} \Big(\frac{2m}{\hbar^2} \Big)^{3/2} \int_0^\infty d\epsilon ~ \epsilon^{1/2} n(\epsilon) \\ & = \underbrace{\Big ( \frac{m}{2 \pi \beta \hbar^2} \Big)^{3/2}}_{1 / \lambda_T^3} \frac{2 g \beta^{3/2}}{\pi^{1/2}} \int_0^\infty d\epsilon ~ \epsilon^{1/2} n(\epsilon)  = \frac{2 g \beta^{3/2}}{\lambda_T^3 \pi^{1/2}} \int_0^\infty d\epsilon ~ \epsilon^{1/2} n(\epsilon) ~,
        \end{aligned}
        \end{equation*}
        where we have introduced the thermal wavelength
        \begin{equation*}
            \lambda_T = \Big (\frac{2 \pi \beta \hbar^2}{m} \Big)^{1/2} ~.
        \end{equation*} 
        Now, we make a change of variable into
        \begin{equation*}
            \epsilon = \frac{x^2}{\beta} ~, \quad d \epsilon = \frac{2 x dx}{\beta} ~,
        \end{equation*}
        hence, we obtain 
        \begin{equation*}
        \begin{aligned}
            & = \frac{2 g \beta^{3/2}}{\lambda_T^3 \pi^{1/2}} \int_0^\infty dx ~ \frac{2 x}{\beta} \frac{x}{\beta^{1/2}} n(x^2 / \beta) = \frac{4 g}{\lambda_T^3 \pi^{1/2}} \int_0^\infty dx ~ \frac{x^2}{z^{-1} \exp(x^2) \mp 1} \\ & = \frac{4 g}{\lambda_T^3 \pi^{1/2}} \int_0^\infty dx ~ x^2 z \exp(-x^2) \underbrace{\frac{1}{1 \pm z \exp(-x^2)}}_{\sum_{n=0}^\infty (\pm 1)^n z^n \exp(-n x^2)} \\ & = \frac{4 g}{\lambda_T^3 \pi^{1/2}} \int_0^\infty dx ~ x^2 z \exp(-x^2) \sum_{n=0}^\infty (\pm 1)^n z^n \exp(-n x^2) \\ & = \frac{4 g}{\lambda_T^3 \pi^{1/2}} \sum_{n=0}^\infty (\pm 1)^n z^{n+1} \int_0^\infty dx ~ x^2 \exp(-(n + 1) x^2) ~.
        \end{aligned}
        \end{equation*}
        Using a change of variable 
        \begin{equation*}
            z = \sqrt{n+1} x ~,
        \end{equation*}
        we evaluate the indefinite integral (second moment of the gaussian), using the gaussian integral~\eqref{app:gauss},
        \begin{equation}\label{proof6}
        \begin{aligned}
            & \int_0^\infty ~ x^2 \exp(-(n + 1) x^2) = \frac{1}{(n+1)^{3/2}} \int dz ~ z^2 \exp(-z^2) \\ & = \frac{1}{(n+1)^{3/2}} \int dz ~ \underbrace{z}_f \underbrace{z \exp(-z^2)}_{g'} \\ & = \frac{1}{(n+1)^{3/2}} \Big ( - \underbrace{\frac{z}{2}}_{0~\text{for}~0} \underbrace{\exp(-z^2)}_{0~\text{for}~\infty} \Big \vert_0^\infty + \frac{1}{2} \int_0^\infty dz ~ \exp(- z^2) \Big) \\ & = \frac{1}{2(n+1)^{3/2}} \frac{1}{2} \underbrace{\int_{-\infty}^\infty dz~ \exp(-z^2) }_{\pi^{1/2}} = \frac{\pi^{1/2}}{4 (n+1)^{3/2}} ~,
        \end{aligned}
        \end{equation}
        with $\erf(0) = 0$ and $\erf(\infty) = 1$. Hence, we find
        \begin{equation*}
            \frac{4 g}{\lambda_T^3 \pi^{1/2}} \sum_{n=0}^\infty (\pm 1)^n z^{n+1} \frac{\pi^{1/2}}{4 (n+1)^{3/2}} = \frac{g}{\lambda_T^3} \sum_{n=0}^\infty (\pm 1)^n \frac{z^{n+1}}{(n+1)^{3/2}} ~.
        \end{equation*}
        which, for fermions, becomes 
        \begin{equation*}
            = \frac{g}{\lambda_T^3} \underbrace{\sum_{n=0}^\infty (- 1)^n \frac{z^{n+1}}{(n+1)^{3/2}}}_{f^-_{3/2}} = \frac{g}{\lambda_T^3} f^-_{3/2} 
        \end{equation*}
        and, for bosons, becomes
        \begin{equation*}
            = \frac{g}{\lambda_T^3} \underbrace{\sum_{n=0}^\infty \frac{z^{n+1}}{(n+1)^{3/2}}}_{f^+_{3/2}} = \frac{g}{\lambda_T^3} f^+_{3/2} ~.
        \end{equation*}

        Similarly, for $\epsilon^{3/2}$, we have 
        \begin{equation*}
        \begin{aligned}
            A \int_0^\infty d\epsilon ~ \epsilon^{3/2} n(\epsilon) & = \frac{g}{4 \pi^2} \Big(\frac{2m}{\hbar^2} \Big)^{3/2} \int_0^\infty d\epsilon ~ \epsilon^{3/2} n(\epsilon) \\ & = \underbrace{\Big ( \frac{m}{2 \pi \beta \hbar^2} \Big)^{3/2}}_{1 / \lambda_T^3} \frac{2 g \beta^{3/2}}{\pi^{1/2}} \int_0^\infty d\epsilon ~ \epsilon^{3/2} n(\epsilon)  = \frac{2 g \beta^{3/2}}{\lambda_T^3 \pi^{1/2}} \int_0^\infty d\epsilon ~ \epsilon^{3/2} n(\epsilon) ~.
        \end{aligned}
        \end{equation*}
        Now, we make a change of variable into
        \begin{equation*}
            \epsilon = \frac{x^2}{\beta} ~, \quad d \epsilon = \frac{2 x dx}{\beta} ~,
        \end{equation*}
        hence, we obtain 
        \begin{equation*}
        \begin{aligned}
            & = \frac{2 g \beta^{3/2}}{\lambda_T^3 \pi^{1/2}} \int_0^\infty dx ~ \frac{2 x}{\beta} \frac{x^3}{\beta^{3/2}} n(x^2 / \beta) = \frac{4 g}{\lambda_T^3 \pi^{1/2} \beta} \int_0^\infty dx ~ \frac{x^4}{z^{-1} \exp(x^2) \mp 1} \\ & = \frac{4 g}{\lambda_T^3 \pi^{1/2} \beta} \int_0^\infty dx ~ x^2 z \exp(-x^2) \underbrace{\frac{1}{1 \pm z \exp(-x^2)}}_{\sum_{n=0}^\infty (\pm 1)^n z^n \exp(-n x^2)} \\ & = \frac{4 g}{\lambda_T^3 \pi^{1/2} \beta} \int_0^\infty dx ~ x^4 z \exp(-x^2) \sum_{n=0}^\infty (\pm 1)^n z^n \exp(-n x^2) \\ & = \frac{4 g}{\lambda_T^3 \pi^{1/2} \beta} \sum_{n=0}^\infty (\pm 1)^n z^{n+1} \int_0^\infty dx ~ x^4 \exp(-(n + 1) x^2) ~.
        \end{aligned}
        \end{equation*}
        Using a change of variable 
        \begin{equation*}
            z = \sqrt{n+1} x ~,
        \end{equation*}
        we evaluate the indefinite integral (fourth moment of the gaussian), using the gaussian integral~\eqref{app:gauss} and the previous result~\eqref{proof6},
        \begin{equation*}
        \begin{aligned}
            & \int_0^\infty ~ x^4 \exp(-(n + 1) x^2) = \frac{1}{(n+1)^{5/2}} \int dz ~ z^4 \exp(-z^2) \\ & = \frac{1}{(n+1)^{5/2}} \int dz ~ \underbrace{z^3}_f \underbrace{z \exp(-z^2)}_{g'} \\ & = \frac{1}{(n+1)^{5/2}} \Big ( - \underbrace{\frac{z^3}{2}}_{0~\text{for}~0} \underbrace{\exp(-z^2)}_{0~\text{for}~\infty} \Big \vert_0^\infty + \frac{3}{2} \int_0^\infty dz ~ z^2 \exp(- z^2) \Big) \\ & = \frac{3}{2(n+1)^{5/2}} \underbrace{\int_0^\infty dz ~ z^2 \exp(- z^2)}_{\pi^{1/2} / 4} = \frac{3 \pi^{1/2}}{8 (n+1)^{5/2}} ~,
        \end{aligned}
        \end{equation*}
        Hence, we find
        \begin{equation*}
            \frac{4 g}{\lambda_T^3 \pi^{1/2}\beta} \sum_{n=0}^\infty (\pm 1)^n z^{n+1} \frac{3 \pi^{1/2}}{8 (n+1)^{5/2}} = \frac{3g}{2\lambda_T^3 \beta} \sum_{n=0}^\infty (\pm 1)^n \frac{z^{n+1}}{(n+1)^{5/2}} ~.
        \end{equation*}
        which, for fermions, becomes 
        \begin{equation*}
            = \frac{3g}{2\lambda_T^3\beta} \underbrace{\sum_{n=0}^\infty (- 1)^n \frac{z^{n+1}}{(n+1)^{5/2}}}_{f^-_{5/2}} = \frac{3g}{2\lambda_T^3 \beta} f^+_{5/2} 
        \end{equation*}
        and, for bosons, becomes
        \begin{equation*}
            = \frac{3g}{2\lambda_T^3 \beta} \underbrace{\sum_{n=0}^\infty \frac{z^{n+1}}{(n+1)^{5/2}}}_{f^-_{5/2}} = \frac{3g}{2\lambda_T^3\beta} f^-_{5/2} ~.
        \end{equation*}
    \end{proof}

    Therefore, the density~\eqref{qg:n2} becomes
    \begin{equation}\label{qg:n3}
        n = \frac{N}{V} = A \int_0^\infty d\epsilon ~ \epsilon^{1/2} n(\epsilon) = \frac{g}{\lambda_T^3} f_{3/2}^\pm ~.
    \end{equation}
    The equation of state~\eqref{qg:o2} and~\eqref{qg:e2} becomes 
    \begin{equation}\label{qg:p}
        p = \frac{2E}{3V} = - \frac{\Omega}{V} = \frac{2}{3} A \int_0^\infty d\epsilon ~ \epsilon^{3/2} n(\epsilon) = \frac{2}{3} \frac{3g}{2\lambda_T^3 \beta} f_{5/2}^\pm = \frac{g}{\lambda_T^3 \beta} f_{5/2}^\pm ~.
    \end{equation}

\section{Classical and semiclassical limit}

    In the classical limit, i.e.~for $z \ll 1$ such that we can keep only the first term in the expansion series~\eqref{ser2}  
    \begin{equation*}
        f_l^\pm = \sum_{n=0}^\infty (\pm 1)^n \frac{z^{n+1}}{(n+1)^l} \simeq z ~,
    \end{equation*}
    we recover the perfect gas 
    \begin{equation*}
        p \beta = n ~.
    \end{equation*}
    \begin{proof}
        In fact, using~\eqref{qg:n3}, we have 
        \begin{equation*}
            n = \frac{g}{\lambda_T^3} f^\pm_{3/2} \simeq \frac{gz}{\lambda_T^3} 
        \end{equation*}
        and, using~\eqref{qg:p}, we have
        \begin{equation*}
            p = \frac{g}{\lambda_T^3 \beta} f^\pm_{5/2} \simeq \frac{gz}{\lambda_T^3\beta}  ~.
        \end{equation*}
        Hence, we find
        \begin{equation*}
            p \beta = \frac{gz}{\lambda_T^3} = n ~.
        \end{equation*}
    \end{proof}

    In the semiclassical limit, i.e.~for $z \ll 1$ such that we can keep both the first and the second term in the expansion series~\eqref{ser2}  
    \begin{equation*}
        f_l^\pm = \sum_{n=0}^\infty (\pm 1)^n \frac{z^{n+1}}{(n+1)^l} \simeq z \pm \frac{z^2}{2^l} ~,
    \end{equation*}
    we find the first correction for the perfect gas 
    \begin{equation*}
        p \beta = n ( 1 \mp n \frac{\lambda_T^3}{g 2^{5/2}} ) ~.
    \end{equation*}
    Notice that for fermions there is a repulsive correction, whereas for bosons there is ana attractive one. It has also a quantum origin, since it goes to zero when we go in the limit $\hbar \rightarrow 0$.
    \begin{proof}
        In fact, using~\eqref{qg:n3}, we have 
        \begin{equation*}
            n = \frac{g}{\lambda_T^3} f^\pm_{3/2} \simeq \frac{gz}{\lambda_T^3} \pm \frac{g z^2}{2^{3/2} \lambda_T^3} ~.
        \end{equation*}
        In order to solve it, we use an ansatz 
        \begin{equation*}
            z = A n + B n^2 
        \end{equation*}
        and we find that, keeping terms of order less than $O(z^3)$
        \begin{equation*}
        \begin{aligned}
            z & = A \Big (\frac{gz}{\lambda_T^3} \pm \frac{g z^2}{2^{3/2} \lambda_T^3} \Big)+ B \Big (\frac{gz}{\lambda_T^3} \pm \frac{g z^2}{2^{3/2} \lambda_T^3} \Big)^2 = A \Big (\frac{gz}{\lambda_T^3} \pm \frac{g z^2}{2^{3/2} \lambda_T^3} \Big) + B \frac{g^2 z^2 }{\lambda_T^6} \\ & = z \Big (\frac{A g}{\lambda^3_T} \Big) + z^2 \Big (\frac{B g^2}{\lambda_T^6} \pm \frac{A g}{2^{3/2} \lambda_T^3} \Big )  
        \end{aligned}
        \end{equation*}
        which implies that, since the coefficient in $z$ must be $1$ and in $z^2$
        must be $0$,
        \begin{equation*}
            A = \frac{\lambda_T^3}{g} ~, \quad \frac{B g^2}{\lambda_T^6} \pm \frac{1}{2^{3/2}} = 0 ~,
        \end{equation*}
        \begin{equation*}
            A = \frac{\lambda_T^3}{g} ~, \quad B = \mp \frac{\lambda_T^6}{g^2 2^{3/2}} ~.
        \end{equation*}
        Hence, we find 
        \begin{equation*}
            z = \frac{\lambda_T^3}{g} n \mp \frac{\lambda_T^6}{g^2 2^{3/2}} n^2 ~.
        \end{equation*}
        Finally, using~\eqref{qg:p}, we have
        \begin{equation*}
            p = \frac{g}{\lambda_T^3 \beta} f^\pm_{5/2} \simeq \frac{gz}{\lambda_T^3\beta} \pm \frac{g z^2}{2^{5/2} \lambda_T^3 \beta} ~.
        \end{equation*}
        and, we find that
        \begin{equation*}
        \begin{aligned}
            p \beta &= \frac{gz}{\lambda_T^3} \pm \frac{g z^2}{2^{5/2} \lambda_T^3}  = \frac{g}{\lambda_T^3} (\frac{\lambda_T^3}{g} n \mp \frac{\lambda_T^6}{g^2 2^{3/2}} n^2) \pm \frac{g}{2^{5/2} \lambda_T^3} (\frac{\lambda_T^3}{g} n \mp \frac{\lambda_T^6}{g^2 2^{3/2}} n^2)^2 \\ & \simeq n \mp \frac{\lambda^3_T}{g 2^{3/2}} n^2 \pm \frac{\lambda^3_T}{g 2^{5/2}} n^2 = n \mp n^2 (\frac{2\lambda^3_T}{g 2^{5/2}} - \frac{\lambda^3_T}{g 2^{5/2}} ) \\ & = n \mp n^2 \frac{\lambda^3_T}{g 2^{5/2}}(2 - 1) = n \mp n^2 \frac{\lambda^3_T}{g 2^{5/2}} = n ( 1 \mp n \frac{\lambda_T^3}{g 2^{5/2}} ) ~.
        \end{aligned}
        \end{equation*}
    \end{proof}

    As an aside, we can recover all the formulas of the previous sections, starting from the phase space. In fact, using~\eqref{cm:denst}, we have 
    \begin{equation}\label{q:o}
        \Omega = \frac{1}{\beta} \int_0^\infty d\epsilon ~ \omega(\epsilon) \ln (1 \mp \exp(- \beta (\epsilon - \mu))) ~,
    \end{equation}
    \begin{equation}\label{q:e}
        E = \int_0^\infty d\epsilon ~ \omega(\epsilon) n(\epsilon) \epsilon ~,
    \end{equation}
    \begin{equation}\label{q:n}
        N = \int_0^\infty d\epsilon ~ \omega(\epsilon) n(\epsilon) ~.
    \end{equation}

\chapter{Fermi energy and Bose-Einstein condensate}

    In this chapter, we will restrict ourselves with the fully-quantum fermionic case, in particular we will study the $T = 0$ limit, and with the fully-quantum fermionic case, in particular we will study the Bose-Einstein condensate.

\section{Fermions at T=0}

    In the fermionic case, recall that for~\eqref{qg:n3} and~\eqref{qg:p}, we have
    \begin{equation*}
        n = \frac{g}{\lambda_T^3} f_{\frac{3}{2}}^- (z) ~, \quad \beta p = \frac{g}{\lambda_T^3} f_{\frac{5}{2}}^- (z) ~,
    \end{equation*}
    where for~\eqref{ser2}
    \begin{equation*}
        f_l^- (z) = \sum_{n=0}^\infty \frac{(-1)^n z^{n+1}}{(n+1)^l} ~.
    \end{equation*}
    Notice that this series in an alternate-sign power series in $z = \exp(\beta\mu) > 0$, always positive. It absolutely converges for $z < 1$ and pointwisely converges for $z > 1$. Moreover, it is a monotonic function in $z$. 

    For the zero temperature limit $T = 0$, the Fermi-Dirac distribution~\eqref{fdbe} becomes 
    \begin{equation}\label{fdtz}
        n(\epsilon) = \frac{1}{\exp(\beta(\epsilon - \mu)) + 1} \xrightarrow{T \rightarrow 0} \begin{cases}
            0 & \epsilon > \mu \\
            \frac{1}{2} & \epsilon = \mu \\
            1 & \epsilon < \mu \\
        \end{cases} ~.
    \end{equation} 
    A plot of the Fermi-Dirac distribution is in Figure~\ref{fig:fd}.
    \begin{figure}
        \centering
        \scalebox{0.7}{\pyc{plot3('x', '1 / ( exp(x) + 1)', '1 / ( exp(5 * x) + 1)','1 / ( exp(100 * x) + 1)', 5, 2, 17, True, False, True)}}
        \caption{A plot of the Fermi-Dirac distribution as a function of $\beta (\epsilon - \mu)$ at different temperature (red is hot, blue is cold). We have used $\beta (\epsilon - \mu)$ and $f(x) = n$.}
        \label{fig:fd}
    \end{figure}
    
    Notice that it is a discontinuous step function $\theta (\mu(T=0) - \epsilon)$ with a step in $\epsilon = \mu(T = 0)$, called Fermi energy $\epsilon_F = \mu(T=0)$. Physically, it means that all the states below this energy level are occupied. Hence, for $\epsilon < \epsilon_F$, we have as many states as particles. If we add a particle, we increase $\epsilon_F$, whereas if we remove a particle, we decrease $\epsilon_F$. This is the procedure to dope a material.
    
    In momentum space, we can define the Fermi surface $\epsilon_F = p_F^2 / 2m$, which is a sphere of radius $p_F$.

    For small $T$, it is no longer a step function, but it can be accurately approximate to it for a certain range $\Delta \epsilon$., which can be evaluated defining the Fermi temperature $T_F$
    \begin{equation*}
        \epsilon_F = \lim_{T \rightarrow 0} \mu (T) = k_B T_F ~.
    \end{equation*}
    In fact, if $\Delta \epsilon \ll \epsilon_F$, which means $T \ll T_F$, we can approximate $n(\epsilon)$ with a step function without making a big error. Physically, it means that there are no more all particles under the energy $\epsilon_F$, but more energetic particle are transfered above $\epsilon_F$.

    Consider again the case of a non-relativistic non-interacting $3$-dimensional quantum gas. In the limit at zero temperature, the density is
    \begin{equation*}
        n = A \frac{2}{3} \epsilon_F^{\frac{3}{2}} ~.
    \end{equation*}
    \begin{proof}
        In fact, using~\eqref{fdtz} and~\eqref{qg:n2}, we have
        \begin{equation*}
            n = A \int_0^\infty d\epsilon ~ \epsilon^{\frac{1}{2}} n(\epsilon) = A \int_0^{\infty} d\epsilon ~ \epsilon^{\frac{1}{2}} \theta(\epsilon_F - \epsilon) = A \int_0^{\epsilon_F} d\epsilon ~ \epsilon^{\frac{1}{2}} = A \frac{2}{3} \epsilon_F^{\frac{3}{2}} ~.
        \end{equation*}
    \end{proof}
    The energy is 
    \begin{equation*}
        E = A V \frac{2}{5} \epsilon_F^{\frac{5}{2}} ~.
    \end{equation*}
    \begin{proof}
        In fact, using~\eqref{fdtz} and~\eqref{qg:e2}, we have
        \begin{equation*}
            E  = A \int_0^\infty d\epsilon ~ \epsilon^{\frac{3}{2}} n(\epsilon) = A \int_0^\infty d\epsilon ~ \epsilon^{\frac{3}{2}} \theta(\epsilon_F - \epsilon) = A \int_0^{\epsilon_F} d\epsilon ~ \epsilon^{\frac{3}{2}} = A \frac{2}{5} \epsilon_F^{\frac{5}{2}} ~.
        \end{equation*}
    \end{proof}
    Notice that at $T = 0$, there is a positive pressure 
    \begin{equation*}
        p = \frac{2}{5} n \epsilon_F > 0~.
    \end{equation*}
    This can be seen visually, because at $T=0$, there are particle with energy $\epsilon \neq 0$, unlikely the classical case, in which $p = 0$.
    \begin{proof}
        In fact, using~\eqref{fdtz} and~\eqref{qg:e2}, we have
        \begin{equation*}
            p = \frac{2}{3} \frac{E}{V} = \frac{2}{3} \frac{E}{N} \frac{N}{V} = \frac{2}{3} n \frac{A V \frac{2}{5} \epsilon_F^{\frac{5}{2}}}{A V\frac{2}{3} \epsilon_F^{\frac{3}{2}}} = \frac{2}{5} n \epsilon_F ~. 
        \end{equation*}
    \end{proof}

    To conclude, at low temperature $T \ll 1$ but not $T=0$, we can expand in an asymptotic series, called Sommerfeld expansion, which is 
    \begin{equation*}
        f(\epsilon) \simeq \theta (\mu - \epsilon) - \frac{\pi^2 T^2}{6} \dv{}{\epsilon} \delta (\epsilon - \mu) + \ldots ~, \quad \mu = \epsilon_F \Big (1 - \frac{\pi^2 T^2}{12 \epsilon_F^2} \Big ) + \ldots ~,
    \end{equation*}
    \begin{equation*}
        E = \frac{3}{5} N \epsilon_F \Big (1 - \frac{5 \pi^2 T^2}{12 \epsilon_F^2} \Big ) + \ldots ~, \quad p = \frac{2}{5} n \epsilon_F \Big (1 - \frac{5 \pi^2 T^2 k_B^2}{12 \epsilon_F^2} \Big ) + \ldots ~.
    \end{equation*}

\section{Bosons at low temperature}

    In the bosonic case, recall that for~\eqref{qg:n3} and~\eqref{qg:p}, we have
    \begin{equation*}
        n = \frac{g}{\lambda_T^3} f_{\frac{3}{2}}^+ (z) ~, \quad \beta p = \frac{g}{\lambda_T^3} f_{\frac{5}{2}}^+ (z) ~,
    \end{equation*}
    where for~\eqref{ser2}
    \begin{equation*}
        f_l^+ (z) = \sum_{n=0}^\infty \frac{z^{n+1}}{(n+1)^l} ~.
    \end{equation*}
    Notice that this series in an positive-terms power series in $z = \exp(\beta\mu) > 0$, always positive. It absolutely converges for $z < 1$ and converges for $z > 1$ only if $l < 2$. Moreover, it is a monotonic function in $z$. At $z = 1$, it becomes the Riemann zeta 
    \begin{equation*}
        f^+_{3/2} (z=1) = \sum_{n=0}^\infty \frac{1}{(n+1)^{3/2}} = \zeta(\frac{3}{2}) ~.
    \end{equation*}
    At $z = 1$, it has a vertical derivative, and for $z > 1$, it is not defined according to the physical $\mu > 0$ in the grandcanonical ensemble. See Figure~\eqref{fig:z}.
    \begin{figure}[h!]
        \centering
        \begin{tikzpicture}
        \draw[->] (-0.5,0) -- (5,0) node[right] {$z$};
        \draw[->] (0,-0.5) -- (0,3.5) node[right] {$f^+_{3/2}$};
                
        \draw[thick] (0,0) to[bend right=40] (2.5,2.5);

        \filldraw[black] (2.5,0) circle (0.05) node[below right] {$z=1$};
        \filldraw[black] (0,2.5) circle (0.05) node[below right] {$\zeta(3/2)$};
        
        \end{tikzpicture}
        \caption{Qualitative plot of $f^+_{3/2}$ in function of $z$.}
        \label{fig:z}
    \end{figure}

    Now, we study the behaviour of the chemical potential $\mu$. It goes to $-\infty$ for $T \rightarrow \infty$ and  the equilibrium condition implies that $\pdv{\mu}{T} < 0$ for~\eqref{td:muT}, therefore, it cannot increase. If $\mu = 0$, the density becomes 
    \begin{equation*}
        n(\epsilon) = \frac{1}{\exp(\beta(\epsilon - \mu)) - 1} \Big \vert_{\mu = 0} = \frac{1}{\exp(\beta \epsilon) - 1} ~,
    \end{equation*}
    which shows that in the ground state, for $\epsilon \rightarrow 0$, all bosons are at this level and the occupation number $n_0$ becomes macroscopically important
    \begin{equation}\label{ninf}
        n(\epsilon) \xrightarrow{\epsilon \rightarrow 0} n_0 =\infty ~.
    \end{equation}
    A plot of the Bose-Enstein distribution is in Figure~\ref{fig:be}.
    \begin{figure}
        \centering
        \scalebox{0.7}{\pyc{plot3('x', '1 / ( exp(x) - 1)', '1 / ( exp(5 * x) - 1)','1 / ( exp(90 * x) - 1)', 5, 2, 18, True, True, True)}}
        \caption{A plot of the Bose-Eintein distribution as a function of $\beta (\epsilon - \mu)$ at different temperature (red is hot, blue is cold). We have used $\beta (\epsilon - \mu)$ and $f(x) = n$.}
        \label{fig:be}
    \end{figure}
    However, we can even $\mu(T=T_c) = 0$ for a different temperature, higher than $T=0$, called critical temperature 
    \begin{equation*}
        T_c = \frac{2 \pi \hbar^2}{m k_B} \Big ( \frac{n}{\zeta(3/2)} \Big)^{2/3} ~.
    \end{equation*}
    This is possible since $\mu < 0$ and it cannot pass the $T$-axis, but it cannot even decrease, so it must remain on the $T$-axis. See Figure~\eqref{fig:mu}. 
    \begin{proof}
        The condition to have $\mu = 0$ is that $z = 1$, which for~\eqref{qg:n3} implies that 
        \begin{equation*}
            n = \frac{g}{\lambda_T^3} f^+_{3/2} (z=1) = g \Big (\frac{2 \pi \hbar^2}{m k_B T} \Big)^{3/2} \zeta(3/2) ~.
        \end{equation*}
        Hence, we find
        \begin{equation*}
            T_c = \frac{2 \pi \hbar^2}{m k_B} \Big ( \frac{n}{\zeta(3/2)} \Big)^{2/3} ~.
        \end{equation*}
    \end{proof}
    \begin{figure}[h!]
        \centering
        \begin{tikzpicture}
        \draw[->] (-0.5,0) -- (5,0) node[right] {$T$};
        \draw[->] (0,-3.5) -- (0,0.5) node[right] {$\mu$};
                
        \draw[thick] (5,-3.5) to[bend right=30] (1,0);
        \draw[thick] (1,0) -- (0,0);

        \filldraw[black] (1,0) circle (0.05) node[above right] {$T=T_c$};
        
        \end{tikzpicture}
        \caption{Qualitative plot of $\mu(T)$ in function of $T$.}
        \label{fig:mu}
    \end{figure}

    Therefore, we should have density that behaves as 
    \begin{equation*}
        n = \begin{cases}
            \frac{g}{\lambda_T^3} f^+_{3/2} (z) & T > T_c \\
            \frac{g}{\lambda_T^3} \zeta(3/2) & T \leq T_c \\
        \end{cases} ~.
    \end{equation*}
    Howver, we have made a mistake, since for $T \rightarrow 0$ we have 
    \begin{equation*}
        n \propto T^{1/2} \xrightarrow{T \rightarrow 0} 0 ~,
    \end{equation*}
    but it is in contradiction with the previous result~\eqref{ninf}. To fix it, we notice that $n(\epsilon)$ must be infinitesimal with respect to $N$ in order to recover 
    \begin{equation*}
        N = \int_0^\infty d\epsilon ~ n(\epsilon) g(\epsilon) ~,
    \end{equation*}
    but, for $\epsilon = 0$ at $T \leq T_c$, this does not hold since $n(\epsilon = 0) \sim N$. This happens because in the thermodynamic limit we have lost a null-measure point which can be recovered by setting 
    \begin{equation*}
        N = \sum_k \frac{1}{\exp(\beta(\epsilon_k - \mu)) - 1} = N_0 + \sum_{k\neq 0} n(k) ~,
    \end{equation*}
    which implies that 
    \begin{equation*}
        n = n_0 + \frac{g}{\lambda_T^3} f^+_{3/2} (z) ~,
    \end{equation*}
    where the second term is the normal particle contribution that goes to zero for low T but there is the first terms which is the ground state contribution that does not go to zero and implies that all particles are in the ground state. To summarise, we have 
    \begin{equation}\label{nnnn}
        n = \begin{cases}
            \frac{g}{\lambda_T^3} f^+_{3/2} (z) & T > T_c \\
            \frac{g}{\lambda_{T_c}^3} \zeta(3/2) & T = T_c \\
            n_0(T) + n_n(T) = n_0 (T) + \frac{g}{\lambda_{T}^3} \zeta(3/2) & T < T_c \\
        \end{cases} ~.
    \end{equation}
    Furthermore, we can estimate the ground state contribution, which is 
    \begin{equation*}
        n_0 (T) = n (1 - \Big (\frac{T}{T_c} \Big)^{3/2}) ~.
    \end{equation*} 
    A plot of the density at ground state $n_0$ and normal $n_n$ for $T \leq T_c$ is in Figure~\ref{fig:nn}. Notice that it is continuous but not differentiable in $T = T_c$ and the sum of them is constant $n = n_0 + n_n$.
    \begin{proof}
        In fact, inverting the middle of~\eqref{nnnn}, we obtain 
        \begin{equation*}
            \zeta(3/2) = \frac{n}{\lambda_{T_c}^2}{g} ~,
        \end{equation*}
        hence, we find 
        \begin{equation*}
            n_n (T) = \frac{g}{\lambda_T^3} \zeta(3/2) = \frac{g}{\lambda_T^3} \frac{n}{\lambda_{T_c}^2}{g} = n \Big (\frac{T}{T_c} \Big)^{3/2} 
        \end{equation*}
        and 
        \begin{equation*}
            n_0 (T) = n - n_n(T) = n - n \Big (\frac{T}{T_c} \Big)^{3/2} = n (1 - \Big (\frac{T}{T_c} \Big)^{3/2}) ~.
        \end{equation*}
    \end{proof}
    \begin{figure}
        \centering
        \scalebox{0.7}{\pyc{plot2('x', 'Piecewise((x**(3/2), x < 1), (1, True))', 'Piecewise((1 - x**(3/2), x < 1), (0, True))', 3, 2, 19, False, True, True)}}
        \caption{A plot of the density at ground state $n_0$ in blue and normal $n_n$ in red as a function of $T$. We have used $x = T / T_c$ and $f(x) = n_0 / n$.}
        \label{fig:nn}
    \end{figure}
    In the phase transition language, $n_0$ is an order parameter, since 
    \begin{equation*}
        n_0 (T) \begin{cases}
            = 0 & T \geq T_c \\
            \neq 0 & T < T_c \\
        \end{cases} ~.
    \end{equation*}

    Now, we can study thermodynamics. The pressure is 
    \begin{equation*}
        p \beta = \frac{g}{\lambda_T^3} f^+_{5/2} (z) 
    \end{equation*}
    and energy is
    \begin{equation}\label{eeee}
        E = \frac{3pV}{2} = \frac{3 g}{2 \lambda_T^3 \beta} f^+_{5/2} (z) ~.
    \end{equation} 
    This relations are valid $\forall T$. Notice that $f^+_{5/2}$ is absolutely convergent since $5/2 > 2$. We can define the energy per particle 
    \begin{equation*}
        u(T) = \frac{E}{N} = \begin{cases}
            \frac{3}{2} k_B T \frac{f^+_{5/2} (z)}{f^+_{3/2} (z)} & T geq T_c \\
            \frac{3}{2} k_B T \Big ( \frac{T}{T_c} \Big)^{3/2} \frac{\zeta(5/2)}{\zeta(3/2)} & T < T_c \\
        \end{cases} ~.
    \end{equation*}
    It is continuous in $T=T_c$ but not differentiable. Therefore, the specific heat per particle $c_v = \partial u / \partial T$ has a cusp and we can classify this as a first order phase transition.
    \begin{proof}
        In fact, using~\eqref{eeee} and~\eqref{nnnn}, we have 
        \begin{equation*}
        \begin{aligned}
            u(T) = \frac{E}{N} = \frac{E}{V} \frac{V}{N} = \begin{cases}
                \frac{3 g}{2 \lambda_T^3 \beta} f^+_{5/2} (z)  \frac{\lambda_T^3}{g f^+_{3/2} (z)} & T \geq T_c \\
                \frac{3 g}{2 \lambda_T^3 \beta} \zeta(5/2) \frac{\lambda_T^3}{g \zeta(3/2)} \Big ( \frac{T}{T_c} \Big )^{3/2} & T < T_c \\
            \end{cases} = \begin{cases}
                \frac{3}{2} k_B T \frac{f^+_{5/2} (z)}{f^+_{3/2} (z)} & T \geq T_c \\
                \frac{3}{2} k_B T \Big ( \frac{T}{T_c} \Big)^{3/2} \frac{\zeta(5/2)}{\zeta(3/2)} & T < T_c \\
            \end{cases} ~.
        \end{aligned}
        \end{equation*}
        It is continuous, since for $T = T_c$, we have
        \begin{equation*}
            \lim_{T \rightarrow T_c^-} u(T) = \lim_{T \rightarrow T_c^+} u(T) = \frac{3}{2} k_B T_c \frac{\zeta(5/2)}{\zeta(3/2)} ~.
        \end{equation*}
    \end{proof}
    The entropy per particle is 
    \begin{equation*}
        s = \begin{cases}
            \Big ( \frac{T}{T_c} \Big)^{3/2} \frac{1}{\eta(3/2)} (\frac{5}{2} f^+_{5/2} (z) - f^+_{3/2} \ln z) & T \geq T_c \\
            \Big ( \frac{T}{T_c} \Big)^{3/2} \frac{1}{\zeta (3/2)} \frac{5}{2} \zeta(5/2)  & T \geq T_c \\
        \end{cases} ~.
    \end{equation*}
    \begin{proof}
        Using~\eqref{td:o} and~\eqref{qg:p}, we have 
        \begin{equation*}
            S = \frac{1}{T} (E - \underbrace{\Omega}_{- \frac{2}{3}E} - \underbrace{\mu}_{\frac{1}{\beta} \ln z} N) = \frac{1}{T} (\frac{5}{3} E - N \frac{1}{\beta} \ln z)  ~.
        \end{equation*}
        Hence, using~\eqref{nnnn} and~\eqref{eeee}, we find for $T \geq T_c$
        \begin{equation*}
        \begin{aligned}
            s & = \frac{S}{N} = \frac{V}{N} \frac{S}{V} = \frac{1}{n} \frac{1}{T} (\frac{5}{3} \frac{E}{V} - \frac{N}{V} \frac{1}{\beta} \ln z) \\ & = \frac{\lambda_{T_c}^3}{g \zeta(3/2)} k_B \beta ( \frac{5}{3} \frac{3g}{2 \lambda_T^3 \beta} f^+_{5/2} (z) - \frac{g}{\lambda_T^3} f^+_{3/2} (z) \frac{1}{\beta} \ln z) = \Big ( \frac{T}{T_c} \Big)^{3/2} \frac{1}{\eta(3/2)} (\frac{5}{2} f^+_{5/2} (z) - f^+_{3/2} \ln z) ~,
        \end{aligned}
        \end{equation*}
        whereas for $T > T_c$, we simply put $z = 1$ 
        \begin{equation*}
            s = \Big ( \frac{T}{T_c} \Big)^{3/2} \frac{1}{\zeta (3/2)} \frac{5}{2} \zeta(5/2) ~.
        \end{equation*}
    \end{proof}
    Notice that, for $T \rightarrow T_c^+$, we have 
    \begin{equation*}
        S \xrightarrow{T \rightarrow T_c^+} \frac{5}{2} \frac{\eta(5/2)}{\eta(3/2)} ~,
    \end{equation*}
    but for for $T \rightarrow 0$, we have 
    \begin{equation*}
        S \xrightarrow{T \rightarrow 0} 0 ~,
    \end{equation*} 
    which is consistent with the fact that all particles are in the ground state and it is the only possible state. Therefore, in the range $[0, T_c]$, we have a lose of entropy which means a latent heat transfer $\Delta Q = T_c \Delta S$
    \begin{equation*}
        \Delta S = S(T=T_c) - S(T=0) = \frac{5}{2} \frac{\eta(5/2)}{\eta(3/2)} = \frac{\Delta Q}{T_c} ~.
    \end{equation*}
    Notice that, if all particles are in the ground state, we can describe them with a single wavefunction and the system behaves like a single quantum object with macroscopic quantum effects and associated probability $|\psi_{gs}|^2 dV$. 

    To coclude, we mention some notions of how it was experimentally realised a Bose-Einstein condensate. The assumptions to have a Bose-Einstein condensate are 
    \begin{enumerate}
        \item quantum non-interacting bosonic $3$-dimensional gas, 
        \item thermal wavelength of the same order of De Broglie wavelength $\lambda_T \simeq \lambda_{DB} = h / p$,
        \item average distance between two particle is $d \simeq \lambda_{DB}$, in order to have quantum effects.
    \end{enumerate}
    In 1938, it was thought that $He^4$ was found in a BEC, but it was a false allarm since at that temperature of $T \simeq 2 K$ Helium is an highly interacting liquid. They discovered a superfluid. It was only in 1995 that a cloud of $Rb^{87}$ was experimentally found at a temperature of $T \sim 10 nK - 1 \mu K$. It was realised using a set of tecnique, called laser cooling, Doppler cooling and magneto-optical trap, which essentially uses beam of laser to cool down the gas (Doopler because the freqency is slightly different from the one of emission to have a DOppler effect) and magnetic field to trap it. It was seen by analysing the gas expand once all traps have been switching off. 

\part{Application of quantum statistical mechanics}

\section{Quantum magnetic 1/2-spin}

    Consider a system composed by $N$ distinguishable magnetic dipoles in an external magnetic field along the $z$-axis, with spin $S = 1/2$. Its hamiltonian is 
    \begin{equation*}
        \hat H = \sum_i S^{(z)}_i B ~,
    \end{equation*}
    where
    \begin{equation*}
        S^{(z)}_i = \frac{1}{2} \begin{bmatrix}
            1 & 0 \\
            0 & -1 \\
        \end{bmatrix} ~.
    \end{equation*}

    The canonical partition function is 
    \begin{equation*}
        Z = \Big ( 2 \cosh \frac{\beta B}{2} \Big)^N ~.
    \end{equation*}
    \begin{proof}
        By definition, for distinguishable particles,
        \begin{equation*}
        \begin{aligned}
            Z & = (Z_1)^N \\ & = \Big (\tr_{\mathcal H} \exp(- \beta \hat H_1) \Big)^N \\ & = \Big (\tr_{\mathcal H} \exp(- \beta B \begin{bmatrix}
                \frac{1}{2} & 0 \\ 0 & - \frac{1}{2} \\ 
            \end{bmatrix}) \Big)^N \\ & = \Big(\tr_{\mathcal H} \begin{bmatrix}
                \exp(- \frac{\beta B}{2}) & 0 \\ 0 & \exp(\frac{\beta B}{2}) \\ 
            \end{bmatrix} \Big )^N \\ & = \Big( \exp(- \frac{\beta B}{2}) + \exp(\frac{\beta B}{2}) \Big)^N \\ & = \Big ( 2 \cosh \frac{\beta B}{2} \Big)^N ~.
        \end{aligned}
        \end{equation*}
    \end{proof}

    The Helmoltz free energy $F$ is 
    \begin{equation*}
        F = - N k_B T \ln \Big ( 2 \cosh \frac{\beta B}{2} \Big) ~.
    \end{equation*}
    \begin{proof}
        By definition, 
        \begin{equation*}
            F = - \frac{\ln Z}{\beta} = - N k_B T \ln \Big ( 2 \cosh \frac{\beta B}{2} \Big) ~.
        \end{equation*}
    \end{proof}

    The internal energy $E$ is 
    \begin{equation*}
        E = - N \frac{B}{2} \tanh \frac{\beta B}{2} ~.
    \end{equation*}
    \begin{proof}
        By definition, 
        \begin{equation*}
            E = - \pdv{\ln Z}{\beta} = - N \pdv{\beta} \ln \Big (2 \cosh \frac{\beta B}{2} \Big) = - N \frac{B}{2} \tanh \frac{\beta B}{2} ~.
        \end{equation*}
    \end{proof}
    A plot of this is in Figure~\ref{qm:e}.
    \begin{figure}
        \centering
        \scalebox{0.7}{\pyc{plot1('x', '- ( tanh (1 / x))', 4, 1, 16, True, True, False)}}
        \caption{A plot of the energy $E$ as a function of $T$. We have used $x = \frac{2 k_B T}{B} $ and $f(x) = \frac{2E}{BN}$.}
        \label{qm:e}
    \end{figure}

    The magnetisation $M$ is 
    \begin{equation*}
        M = - \frac{N}{2} \tanh \frac{\beta B}{2} ~. 
    \end{equation*}
    \begin{proof}
        By definition, 
        \begin{equation*}
            M = \pdv{F}{B} = - N k_B T \pdv{}{B} \ln \Big ( 2 \cosh \frac{\beta B}{2} \Big) = - \frac{N}{2} \tanh \frac{\beta B}{2} ~.
        \end{equation*}
    \end{proof}
    A plot of this is in Figure~\ref{qm:m}.
    \begin{figure}
        \centering
        \scalebox{0.7}{\pyc{plot1('x', '- ( tanh (1 / x))', 4, 1, 17, True, True, False)}}
        \caption{A plot of the magnetisation $M$ as a function of $T$. We have used $x = \frac{2 k_B T}{B} $ and $f(x) = \frac{2M}{N}$.}
        \label{qm:m}
    \end{figure}

\section{Quantum magnetic 1-spin}

    Consider a system composed by $N$ distinguishable magnetic dipoles in an external magnetic field along the $z$-axis, with spin $S = 1$. Its hamiltonian is 
    \begin{equation*}
        \hat H = \sum_i S^{(z)}_i B ~,
    \end{equation*}
    where
    \begin{equation*}
        S^{(z)}_i = \begin{bmatrix}
            1 & 0 & 0 \\
            0 & 0 & 0 \\
            0 & 0 & - 1 \\
        \end{bmatrix} ~.
    \end{equation*}

    The canonical partition function is 
    \begin{equation*}
        Z = \Big ( 2 \cosh (\beta B) + 1 \Big)^N ~.
    \end{equation*}
    \begin{proof}
        By definition, for distinguishable particles,
        \begin{equation*}
        \begin{aligned}
            Z & = (Z_1)^N \\ & = \Big (\tr_{\mathcal H} \exp(- \beta \hat H_1) \Big)^N \\ & = \Big (\tr_{\mathcal H} \exp(- \beta B \begin{bmatrix}
                1 & 0 & 0 \\ 0 & 0 & 0 \\ 0 & 0 & - 1 \\ 
            \end{bmatrix}) \Big)^N \\ & = \Big(\tr_{\mathcal H} \begin{bmatrix}
                \exp(- \beta B) & 0 & 0 \\ 0 & 1 & 0 \\ 0 & 0 & \exp(\beta B) \\ 
            \end{bmatrix} \Big )^N \\ & = \Big( \exp(- \beta B) + 1 + \exp(\beta B) \Big)^N \\ & = \Big ( 2 \cosh (\beta B) + 1 \Big)^N ~.
        \end{aligned}
        \end{equation*}
    \end{proof}

    The Helmoltz free energy $F$ is 
    \begin{equation*}
        F = - N k_B T \ln \Big ( 2 \cosh (\beta B) + 1 \Big) ~.
    \end{equation*}
    \begin{proof}
        By definition, 
        \begin{equation*}
            F = - \frac{\ln Z}{\beta} = - N k_B T \ln \Big ( 2 \cosh (\beta B) + 1 \Big) ~.
        \end{equation*}
    \end{proof}

    The internal energy $E$ is 
    \begin{equation*}
        E = - 2 N B \frac{\sinh (\beta B)}{2 \cosh (\beta B) + 1} ~.
    \end{equation*}
    \begin{proof}
        By definition, 
        \begin{equation*}
            E = - \pdv{\ln Z}{\beta} = - N \pdv{}{\beta} \ln \Big ( 2 \cosh (\beta B) + 1 \Big) = - 2 N B \frac{\sinh (\beta B)}{2 \cosh (\beta B) + 1} ~.
        \end{equation*}
    \end{proof}
    A plot of this is in Figure~\ref{qm:e1}.
    \begin{figure}
        \centering
        \scalebox{0.7}{\pyc{plot1('x', '- (( sinh(1 / x) ) / (2 * cosh (1 /x) + 1))', 4, 1, 18, True, True, False)}}
        \caption{A plot of the energy $E$ as a function of $T$. We have used $x = \frac{k_B T}{B} $ and $f(x) = \frac{2E}{BN}$.}
        \label{qm:e1}
    \end{figure}

    The magnetisation $M$ is 
    \begin{equation*}
        M = - \frac{N}{2} \frac{\sinh (\beta B)}{2 \cosh (\beta B) + 1} ~. 
    \end{equation*}
    \begin{proof}
        By definition, 
        \begin{equation*}
            M = \pdv{F}{B} = - N k_B T \pdv{}{B} \ln \Big ( 2 \cosh (\beta B) + 1 \Big) = - 2 N \frac{\sinh (\beta B)}{2 \cosh (\beta B) + 1}  ~.
        \end{equation*}
    \end{proof}
    A plot of this is in Figure~\ref{qm:m2}.
    \begin{figure}
        \centering
        \scalebox{0.7}{\pyc{plot1('x', '- (( sinh (1 / x) ) / (2 * cosh (1 / x) + 1))', 4, 1, 19, True, True, False)}}
        \caption{A plot of the magnetisation $M$ as a function of $T$. We have used $x = \frac{2 k_B T}{B} $ and $f(x) = \frac{M}{2N}$.}
        \label{qm:m2}
    \end{figure}

\section{Quantum harmonic oscillators}

    Consider a system composed by $N$ distinguishable quantum harmonic oscillators. Its hamiltonian is 
    \begin{equation*}
        \hat H = \sum_i \hbar \omega (\hat a_i \hat a_i^\dagger + \frac{1}{2}) ~.
    \end{equation*}

    The canonical partition function is 
    \begin{equation*}
        Z = \Big ( \frac{\exp(- \frac{\beta \hbar \omega}{2})}{1 - \exp(- \beta \hbar \omega)} \Big)^N ~.
    \end{equation*}
    \begin{proof}
        By definition, for distinguishable particles,
        \begin{equation*}
        \begin{aligned}
            Z & = (Z_1)^N \\ & = \Big (\tr_{\mathcal H} \exp(- \beta \hat H_1) \Big)^N \\ & = \Big (\tr_{\mathcal H_i} \exp(- \beta \hbar \omega (\hat a_i \hat a_i^\dagger + \frac{1}{2})) \Big)^N \\ & = \Big ( \sum_i \bra{n_i}\exp(- \beta \hbar \omega (\hat a_i \hat a_i^\dagger + \frac{1}{2})) \ket{n_i }\Big)^N \\ & = \Big ( \sum_i \exp(- \beta \hbar \omega (n_i + \frac{1}{2})) \Big)^N \\ & = \Big ( \exp(- \frac{\beta \hbar \omega}{2}) \sum_i \exp(- \beta \hbar \omega)^{n_i} \Big)^N \\ & = \Big ( \exp(- \frac{\beta \hbar \omega}{2}) \frac{1}{1 - \exp(- \beta \hbar \omega)} \Big)^N
            \\ & = \Big ( \frac{\exp(- \frac{\beta \hbar \omega}{2})}{1 - \exp(- \beta \hbar \omega)} \Big)^N ~.
        \end{aligned}
        \end{equation*}
    \end{proof}

    The Helmoltz free energy $F$ is 
    \begin{equation*}
        F = N k_B T ( \frac{\beta \hbar \omega}{2} + \ln (1 - \exp(- \beta \hbar \omega))) ~.
    \end{equation*}
    \begin{proof}
        By definition, 
        \begin{equation*}
        \begin{aligned}
            F & = - \frac{\ln Z}{\beta} \\ & = - N k_B T \ln \Big ( \frac{\exp(- \frac{\beta \hbar \omega}{2})}{1 - \exp(- \beta \hbar \omega)} \Big) \\ & = - N k_B T ( \ln \exp(- \frac{\beta \hbar \omega}{2}) - \ln (1 - \exp(- \beta \hbar \omega) )) \\ & = - N k_B T ( - \frac{\beta \hbar \omega}{2} - \ln (1 - \exp(- \beta \hbar \omega))) \\ & = N k_B T ( \frac{\beta \hbar \omega}{2} + \ln (1 - \exp(- \beta \hbar \omega))) ~.
        \end{aligned}
        \end{equation*}
    \end{proof}

    The internal energy $E$ is 
    \begin{equation*}
        E = N ( \frac{\hbar \omega}{2} + \frac{\hbar \omega}{\exp(- \beta \hbar \omega) - 1} ) ~.
    \end{equation*}
    \begin{proof}
        By definition, 
        \begin{equation*}
        \begin{aligned}
            F & = - \pdv{\ln Z}{\beta} \\ & = - N \pdv{}{\beta} \ln \Big ( \frac{\exp(- \frac{\beta \hbar \omega}{2})}{1 - \exp(- \beta \hbar \omega)} \Big) \\ & = - N \pdv{}{\beta} ( \ln \exp(- \frac{\beta \hbar \omega}{2}) - \ln (1 - \exp(- \beta \hbar \omega) )) \\ & = - N \pdv{}{\beta} ( - \frac{\beta \hbar \omega}{2} - \ln (1 - \exp(- \beta \hbar \omega))) \\ & = N \pdv{}{\beta} ( \frac{\beta \hbar \omega}{2} + \ln (1 - \exp(- \beta \hbar \omega))) \\ & = N ( \frac{\hbar \omega}{2} - \frac{\hbar \omega}{1 - \exp(- \beta \hbar \omega)} ) \\ & = N ( \frac{\hbar \omega}{2} + \frac{\hbar \omega}{\exp(- \beta \hbar \omega) - 1} ) ~.
        \end{aligned}
        \end{equation*}
    \end{proof}

    The specific heat is 
    \begin{equation*}
        C_V = N \frac{\hbar^2 \omega^2}{k_B T^2} \frac{\exp(\beta \hbar \omega)}{(\exp(\beta \hbar \omega) - 1)^2} ~.
    \end{equation*}
    \begin{proof}
        In fact 
        \begin{equation*}
            C_V = \pdv{E}{T} = N \pdv{}{T} ( \frac{\hbar \omega}{2} + \frac{\hbar \omega}{\exp(- \beta \hbar \omega) - 1} ) = N \frac{\hbar^2 \omega^2}{k_B T^2} \frac{\exp(\beta \hbar \omega)}{(\exp(\beta \hbar \omega) - 1)^2} ~.
        \end{equation*}
    \end{proof}

\chapter{Fermions}

\section{Fermionic relativistic gas at T=0}

\section{Fermionic White dwarf}

    A white dwarf is an helium star woth mass $M \sim 10^{30} kg$ and a density of $\rho = 10^{10} kg/m^3$ at a temperature of $10^{7} K$. Our approxiated model is composed by $N$ electrons and $N/2$ helium nuclei.

    Assuming $M = N(m_e + 2 m_p) \sim 2 N m_p$, the electronic dentity is 
    \begin{equation*}
        n = 3 \times 10^{36} m^{-3} ~.
    \end{equation*}
    \begin{proof}
        In fact 
        \begin{equation*}
            n = \frac{N}{V} = \frac{N \rho}{M} = \frac{N \rho}{2 N m_p} = \frac{\rho}{2 m_p} = \frac{10^{10}}{2 \times 1.6 \times 10^{-27}} = 3 \times 10^{36} m^{-3} ~.
        \end{equation*}
    \end{proof}

    The Fermi momentum $p_F$ is 
    \begin{equation*}
        p_F = h \Big ( \frac{3 n}{4 \pi g} \Big)^{1/3} = 6.63 \times 10^{-34}  \times \Big ( \frac{3 \times 10^{10}}{4 \times 3.14 \times 2} \Big)^{1/3} = 0.88 Mev/c ~.
    \end{equation*}
    \begin{proof}
        In fact, using $p = \hbar k$
        \begin{equation*}
            N = g \sum_{n} \rightarrow g \frac{V}{(2\pi)^3} \int d^3 k = g \frac{V}{(2\pi \hbar)^3} \int d^3 p = g \frac{4 \pi V}{(2\pi \hbar)^3} \int_0^{p_F} dp p^2 = g \frac{4 \pi V}{(2\pi \hbar)^3} \frac{p_F^3}{3} ~,
        \end{equation*}
        hence 
        \begin{equation*}
            p_F = h \Big ( \frac{3 n}{4 \pi g} \Big)^{1/3} = 6.63 \times 10^{-34}  \times \Big ( \frac{3 \times 10^{10}}{4 \times 3.14 \times 2} \Big)^{1/3} = 0.88 Mev/c ~.
        \end{equation*}
    \end{proof}

    The Fermi energy $\epsilon_F$ is 
    \begin{equation*}
        \epsilon_F = \sqrt{(p_F c)^2 + (mc^2)^2} - mc^2 = 0.5 Mev ~.
    \end{equation*}

    The Fermi temperature $T_F$ is 
    \begin{equation*}
        T_F = \frac{\epsilon_F}{k_B} = 10^{10} K ~,
    \end{equation*}
    which means that we are in the regime $T \ll T_F$ and we can use $T=0$.

    The internal energy $E$ is 
    \begin{equation*}
        E = \frac{\pi V m^4 c^5}{\pi^2 \hbar^3} f(x_F) ~.
    \end{equation*}
    \begin{proof}
        In fact,
        \begin{equation*}
        \begin{aligned}
            E & = g \sum_{n} \epsilon \rightarrow g \frac{V}{(2\pi)^3} \int d^3 k \epsilon \\ & = g \frac{V}{(2\pi \hbar)^3} \int d^3 p \epsilon \\ & = g \frac{4 \pi V}{(2\pi \hbar)^3} \int_0^{p_F} dp p^2 \epsilon \\ & = g \frac{4 \pi V}{(2\pi \hbar)^3} \int_0^{p_F} dp p^2 c \sqrt{p^2 + (mc)^2} ~.
        \end{aligned}
        \end{equation*}

        Now we make a change of variable 
        \begin{equation*}
            x = \frac{p}{mc} ~, \quad dp = mc dx ~,
        \end{equation*}
        hence 
        \begin{equation*}
        \begin{aligned}
            E & = g \frac{4 \pi V}{(2\pi \hbar)^3} c (mc)^3 \int_0^{x_F} dx ~ x^2 (mc)\sqrt{x^2 + 1} \\ & =  \frac{4 g \pi V m^4 c^5}{h^3} \int_0^{x_F} dx ~ x^2\sqrt{x^2 + 1} \\ & = \frac{4 g \pi V m^4 c^5}{h^3} f(x_F) \\ & = \frac{V m^4 c^5}{\pi^2 \hbar^3} f(x_F) ~,
        \end{aligned}
        \end{equation*}
        where 
        \begin{equation*}
            f(x_F) = \int_0^{x_F} dx ~ x^2 \sqrt{x^2 + 1} ~.
        \end{equation*}
    \end{proof}

    The pressure $P$ is 
    \begin{equation*}
        P = \frac{m^4 c^5}{\pi^2 \hbar^3} \Big (\frac{x_F^3}{3} \sqrt{1 + x_F^2} - f(x_F)) ~.
    \end{equation*}
    \begin{proof}
        In fact,
        \begin{equation*}
        \begin{aligned}
            P & = - \pdv{E}{V} \\ & = - \pdv{}{V} \frac{ V m^4 c^5}{\pi^2 \hbar^3} f(x_F) \\ & = - \frac{\pi m^4 c^5}{\pi^2 \hbar^3} f(x_F) - \frac{ V m^4 c^5}{\pi^2 \hbar^3} \pdv{x_F}{V} \pdv{f(x_F)}{x_F} \\ & = - \frac{ m^4 c^5}{\pi^2 \hbar^3} f(x_F) - \frac{ V m^4 c^5}{\pi^2 \hbar^3} \pdv{}{V} \Big (\frac{h}{mc} \Big ( \frac{3 N}{4 \pi g V} \Big)^{1/3}) \pdv{f(x_F)}{x_F} \\ & = - \frac{m^4 c^5}{\pi^2 \hbar^3} f(x_F) - \frac{V m^4 c^5}{\pi^2 \hbar^3} \Big (\frac{h}{mc} \Big ( \frac{3 N}{4 \pi g V} \Big)^{1/3}) \pdv{}{V} V^{-1/3} \pdv{f(x_F)}{x_F} \\ & = - \frac{m^4 c^5}{\pi^2 \hbar^3} f(x_F) + \frac{1}{3} \frac{V m^4 c^5}{\pi^2 \hbar^3} \Big (\frac{h}{mc} \Big ( \frac{3 N}{4 \pi g V} \Big)^{1/3}) V^{-4/3} \pdv{f(x_F)}{x_F} \\ & = \frac{m^4 c^5}{\pi^2 \hbar^3} \Big (\frac{x_F^3}{3} \sqrt{1 + x_F^2} - f(x_F)) ~.
        \end{aligned}
        \end{equation*}
    \end{proof}

    Now, we solve the integral 
    \begin{equation*}
        f(x) = \py{indint('x**2 * sqrt(x**2 + 1)', 'x')} ~.
    \end{equation*}
    In the non-relativistic limit, $x_F \ll 1$, we can make the approximations 
    \begin{equation*}
        g(x) = \frac{x^3}{3} \sqrt{1 + x^2} = \py{Taylor('x', 'x**3 / 3 * sqrt(1 + x**2)', 0, 6)}
    \end{equation*}
    and 
    \begin{equation*}
        f(x_F) = \py{Taylor('x', 'x**5 / (4 * sqrt(x**2 + 1)) + ( 3 * x**3) / (8 * sqrt(x**2 + 1)) + x / (8 * sqrt(x**2 + 1)) - asinh(x) / 8', 0, 6)} ~.
    \end{equation*}

    In the ultra-relativistic limit, $x_F \gg 1$ or equivalemty $y_F = 1 / x_F \ll 1$, we can make the approximations 
    \begin{equation*}
        g(1/x) = \py{Taylor('x', '(1 / x)**3 / 3 * sqrt(1 + (1 / x)**2)', 0, -1)}
    \end{equation*}
    and 
    \begin{equation*}
        f(1 / x) = \py{Taylor('x', '(1 / x)**5 / (4 * sqrt((1 / x)**2 + 1)) + ( 3 * (1 / x)**3) / (8 * sqrt((1 / x)**2 + 1)) + (1 / x) / (8 * sqrt((1 / x)**2 + 1)) - (ln ( 1 / x + sqrt((1/x)**2 + 1))) / 8', 0, -1)} ~.
    \end{equation*}
    
    Imposing the equilibrium condition $dE = 0$, between the gravitational and the pressure forces, and the structure of a sphere, the pressure must be 
    \begin{equation*}
        P = \frac{\alpha G M^2}{4 \pi R^4}
    \end{equation*}
    and the Fermi momentum is 
    \begin{equation*}
        p_F = \frac{\hbar}{R} \Big ( \frac{9 \pi M}{8 m_p} \Big)^{1/3} ~.
    \end{equation*}
    \begin{proof}
        For the gravitational force 
        \begin{equation*}
            E_g = - \alpha \frac{G M^2}{R} ~, \quad dE_g = \alpha \frac{GM^2}{R^2} dR ~.
        \end{equation*}
        For the pressure force 
        \begin{equation*}
            E_p = - p V = - p \frac{4}{3} \pi R^3 ~, \quad dE_p = - 4 \pi p R^2 dR ~.
        \end{equation*}
        Imposing the equilibrium condition, 
        \begin{equation*}
            0 = dE = dE_g + dE_p = alpha \frac{GM^2}{R^2} dR - 4 \pi p R^2 dR ~,
        \end{equation*}
        hence 
        \begin{equation*}
            p = \frac{\alpha G M^2}{4 \pi R^4} ~.
        \end{equation*}

        The Fermi momentum is 
        \begin{equation*}
            p_F = h \Big ( \frac{3 n}{4 \pi g} \Big)^{1/3} = h \Big ( \frac{3}{8 \pi} \frac{M}{2 m_p \frac{4}{3} \pi R^3} \Big)^{1/3} = \frac{\hbar}{R} \Big ( \frac{9 \pi M}{8 m_p} \Big)^{1/3}  ~.
        \end{equation*}
    \end{proof}

    In the ultra-relativistic limit
    \begin{equation*}
        P = \frac{m^4 c^5}{12 \pi \hbar^3} (x_F^4 - x_F^2) = \frac{\alpha G M^2}{4 \pi R^4} ~.
    \end{equation*}

\section{Bose-Einstein condensate in 2D}

\section{Bosonic blackbody radiation}

\appendix

\part{Appendix}

\chapter{Volume of an N-dimensional sphere}

    In this appendix chapter, we will prove that the volume of an $N$-dimensional sphere of radius $R$ is 
    \begin{equation}\label{app:volumen}
        V_n (R) = \frac{\pi^{n/2} R^n}{\Gamma(n/2 + 1)} ~.
    \end{equation}
    \begin{proof}
        Consider the rotationally invariant function $f$ 
        \begin{equation*}
            f(x_1, \ldots x_n) = \exp(- \frac{1}{2} \sum_{i=1}^{n} x_i^2 ) = \prod_{i=1}^{n} \exp(- \frac{1}{2} x_i^2 ) =~.
        \end{equation*}
        Using the Gaussian integral, this function can be integrated over all $\mathbb R^n$, with volume element $dV = dx_1 \ldots dx_n$, and it gives
        \begin{equation*}
        \begin{aligned}
            \int_{\mathbb R^n} dV ~ f & = \int_{\mathbb R^n} \prod_{i=1}^n dx_i ~ f = \int_{\mathbb R^n} \prod_{i=1}^n dx_i ~ \exp(- \frac{1}{2} \sum_{i=1}^{n} x_i^2 ) \\ & = \prod_{i=1}^{n} \underbrace{( \int_{\mathbb R} dx_i ~ \exp(- \frac{1}{2} x_i^2 ))}_{(2 \pi)^{1/2}} = \prod_{i=1}^{n} (2 \pi)^{1/2} = (2 \pi)^{n/2} ~.
        \end{aligned}
        \end{equation*}
        Exploiting the rotational invariant property, we can decomposed the volume element into a surface element $dA$, which integrated gives an $(n-1)$-dimensional sphere $S^{n-1} (r)$ of radius $r$, multiplied by a length element $dr$, i.e.
        \begin{equation*}
            \int_{\mathbb R^n} dV ~ f = \int_0^\infty dr \int_{S^{n-1} (r)} dA ~ f ~.
        \end{equation*}
        Since the area is proportial to the radius, e.g.~for $n=3$ the area is $A \propto r^2$, the radius-dependence of the area is given by $A_{n-1}(r) = r^{n-1} A_{n-1} (1)$. Therefore, putting it inside the integral, we obtain 
        \begin{equation*}
            A_{n-1} (1) \int_0^\infty dr r^{n-1} \exp(- \frac{1}{2} r^2) ~.
        \end{equation*}
        Now, we make a change of variables into 
        \begin{equation*}
            t = \frac{r^2}{2} ~, \quad r = (2t)^{1/2} ~, \quad dr = 2^{-1/2} t^{-1/2} dt
        \end{equation*}
        to have the integral of the gamma function
        \begin{equation*}
        \begin{aligned}
            \int_0^\infty dr ~ r^{n-1} \exp(- \frac{1}{2} r^2) & = 2^{(n-1)/2} 2^{-1/2}\int_0^\infty dt ~ t^{(n-1)/2} t^{-1/2} \exp(-t) \\ & = 2^{n/2 - 1} \underbrace{\int_0^\infty dt ~ t^{n/2 - 1} \exp(-t)}_{\Gamma(n/2)} = 2^{n/2 - 1} \Gamma(n/2) ~.
        \end{aligned}
        \end{equation*}
        Now, we combine the two results together to obtain the surface
        \begin{equation*}
            (2 \pi)^{n/2} = A_{n-1} (1) 2^{n/2 - 1} \Gamma(n/2) ~,
        \end{equation*}
        hence 
        \begin{equation*}
            A_{n-1} (1) = \frac{2 \pi^{n/2}}{\Gamma(n/2)} ~.
        \end{equation*}
        Finally, in order to find the volume we need to integrate from $0$ to $R$ 
        \begin{equation*}
        \begin{aligned}
            V_n(R) & = \int_0^R dr A_{n-1} (r) = \int_0^R dr ~ A_{n-1} (1) r^{n-1} = \frac{2 \pi^{n/2}}{\Gamma(n/2)} \int_0^R dr ~ r^{n-1} \\ & = \frac{2 \pi^{n/2}}{\Gamma(n/2)} \frac{r^n}{n} \Big \vert_0^R = \frac{2 \pi^{n/2}}{n\Gamma(n/2)} R^n = \frac{\pi^{n/2} R^n}{\Gamma(n/2 + 1)} ~.
        \end{aligned}
        \end{equation*}
    \end{proof}

\chapter{Stirling approximation}

    In this appendix chapter, we will prove the Stirling approximation 
    \begin{equation}\label{app:stirl}
        \ln n! \simeq n \ln n - n ~.
    \end{equation}
    \begin{proof}
        The factorial can be expressed in integral form via the gamma function 
        \begin{equation*}
            \Gamma (n + 1) = n! = \int_0^\infty dt ~ t^n \exp(-t) ~.
        \end{equation*}
        Now, we make a change of variables into 
        \begin{equation*}
            t = n x ~, \quad x = \frac{t}{n} ~, \quad dx = \frac{dt}{n} ~,
        \end{equation*}
        to have 
        \begin{equation*}
        \begin{aligned}
            \int_0^\infty dt ~ t^n \exp(-t) & = int_0^\infty dt ~ \exp(\ln t^n) \exp(-t) \\ & = \int_0^\infty dt ~ \exp(n\ln t - t) \\ & = n \int_0^\infty dx ~ \exp(n \ln (nx) - nx) \\ & = n \int_0^\infty dx ~ \exp(n \ln x + n \ln n - nx) \\ & = n \exp(n \ln n) \int_0^\infty dx ~ \exp(n (\ln x - x)) ~.
        \end{aligned}
        \end{equation*}
        In the limit for which $n$ is large, we can use the Laplace approximation method 
        \begin{equation*}
            \int_a^b dx ~ \exp(n f(x)) \simeq \exp(n f(x_0)) \sqrt{\frac{2\pi}{n |f'' (x_0)|}} ~.
        \end{equation*}
        where $x_0 \in [a, b]$ is a stationary point of $f(x)$. A simple sketch of the proof is given by means of the Taylor expansion around $x_0$
        \begin{equation*}
            f(x) \simeq f(x_0) - \frac{1}{2} |f''(x_0)| (x - x_0)^2 ~,
        \end{equation*}
        hence, integrating the Gaussian integral,
        \begin{equation*}
        \begin{aligned}
            \int_a^b dx ~ \exp(n f(x)) & \simeq \exp(n f(x_0)) \int_a^b dx ~ \exp(- \frac{n}{2} |f''(x_0)| (x - x_0)^2) \\ & = \sqrt{\frac{2\pi}{n |f'' (x_0)|}} ~.
        \end{aligned}
        \end{equation*}
        In our case, $a=0$, $b=\infty$ and $f(x) = \ln x - x$, which has a maximum in $x_0 = 1$ and second derivatives equals to $|f''(x)| = 1 / x^2$. Therefore
        \begin{equation*}
            \int_0^\infty dx ~ \exp(n (\ln x - x)) \simeq \exp(n (\ln x_0 - x_0)) \sqrt{\frac{2\pi x_0^2}{n}} \Big \vert_{x_0 = 1} = \exp(- n) \sqrt{\frac{2\pi}{n}} ~.
        \end{equation*}
        Now, we combine the two results together
        \begin{equation*}
            n! \simeq n \exp(n \ln n) \exp(- n) \sqrt{\frac{2\pi}{n}} = \exp(n \ln n - n) \sqrt{2 \pi n} = n^n \exp(-n) \sqrt{\frac{2\pi}{n}} ~,
        \end{equation*}
        which can be rewritten in terms of logarithms rather than exponentials 
        \begin{equation*}
            \ln n! \simeq \ln (n^n \exp(-n) \sqrt{\frac{2\pi}{n}} ) = n \ln n - n + O(\ln n) ~.
        \end{equation*}
    \end{proof}

\chapter{Gaussian integral}

    In this appendix chapter, we will prove that the Gaussian integral is
    \begin{equation}\label{app:gauss}
        \int_{-\infty}^\infty dx ~ \exp(- x^2) = \sqrt{\pi} ~.
    \end{equation}
    \begin{proof}
        We start from the square Gaussian integral, which it is the square same integral for the mute properties of the integration variables 
        \begin{equation*}
        \begin{aligned}
            \Big (\int_{-\infty}^\infty dx ~ \exp(- x^2) \Big)^2 & = \int_{-\infty}^\infty dx ~ \exp(- x^2) \int_{-\infty}^\infty dy ~ \exp(- y^2) \\ & = \int_{-\infty}^\infty dx \int_{-\infty}^\infty dy ~ \exp(- (x^2 + y^2)) ~.
        \end{aligned}
        \end{equation*}
        Now, we make a change of variables and we use polar coordinates $(r, \theta)$
        \begin{equation*}
            r^2 = x^2 + y^2 ~, \quad \theta = \arctan \frac{y}{x} ~, \quad dx ~ dy = r ~ dr ~ d\theta ~, \quad (r, \theta) \in [0, \infty) \times [0, 2\pi] ~,
        \end{equation*} 
        to obtain
        \begin{equation*}
        \begin{aligned}
            \int_{-\infty}^\infty dx \int_{-\infty}^\infty dy ~ \exp(- (x^2 + y^2)) & = \underbrace{\int_0^{2\pi} d\theta}_{2\pi} \int_0^\infty dr ~ r \exp(- r^2) \\ & = 2 \pi \int_0^\infty dr ~ r \exp(- r^2) \\ & = \pi \int_0^\infty dr ~ 2 r \exp(- r^2) \\ & = \pi \exp(- r^2) \Big \vert_0^{\cancel \infty} = \pi ~.
        \end{aligned}
        \end{equation*}
        Now, we combine the two results together
        \begin{equation*}
            \Big (\int_{-\infty}^\infty dx ~ \exp(- x^2) \Big)^2 = \pi ~,
        \end{equation*}
        hence
        \begin{equation*}
            \int_{-\infty}^\infty dx ~ \exp(- x^2) = \sqrt{\pi} ~.
        \end{equation*}
    \end{proof}
    

\backmatter

\nocite{smlecture}
\nocite{ercolessi}
\nocite{mussardo}

\clearpage
\listoffigures

\clearpage
\listoftables

\clearpage
\listofequations

\clearpage
\phantomsection
\printbibliography

\end{document}
