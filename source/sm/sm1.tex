\part{Thermodynamics}

\chapter{The 2 laws of thermodynamics, which are 4}

    In this chapter, we will recall some notions of thermodynamics.

\section{Equilibrium}

    Thermodynamics studies large systems that reach equilbrium configurations. But, what is equilibrium? 

    Suppose we have a system immersed in its surroundings. It can be isolated or it can exchange matter and/or energy (mechanical, electric, magnetic, chemical work). The configuration it reaches and its stability can be selected by the boundary conditions, i.e. the specification on how the system is in contact and how it interacts with its surroundings. In other words, there is only one and only one final equilibrium configuration towards the system evolves, once the boundary conditions have beeen given. However, the way the system reaches the equilibrium configuration is irreversible. Equilibrium therefore means that once the system has reached its final configuration, it will stay there forever. 

\section{States}

    In thermodynamics, a state is a macroscopic configuration. It is defined by a set of macroscopic quantities, called thermodynamical variables. They can be divided into two groups, one conjugate to the other, according to their behaviour when the physical system is rescaled, i.e.~when the volume and the number of particles change: extensive variables do scale with it whereas intensive ones do not. See Table~\ref{table:1}. 
    
    \begin{table}[h!]
        \centering
        \begin{tabular}{c | c }
            Extensive & Intensive \\
            \hline
            Energy $E$ & - \\ 
            Entropy $S$ & Temperature $T$ \\ 
            Volume $V$ & Pression $p$\\ 
            Number of particles $E$ & Chemical potential $\mathbf \mu$ \\ 
            Polarization $\mathbf P$ & Electric field $\mathbf E$ \\ 
            Magnetization $\mathbf M$ & Magnetic field $\mathbf B$ \\ 
        \end{tabular}
        \caption{Extensive and intensive thermodynamical variables.}
        \label{table:1}
    \end{table}

    However, we have to be careful, since only volume is by definition extensive and all the others quantities can be considered extensive only if the surfaces terms are neglectible when we take the thermodynamic limit. An equation of state is a functional relation among them, which restrict the number of independent variables. Geometrically, it means that the only admissible states are a submanifold of the entire manifold of states. 

    \begin{example}[Perfect gas]
        A perfect gas is described by $3$ state variables $(p, V, T)$ and an equation of state $PV = N k_B T$. This means that the allowed states are in a $2$-dimensional manifold embedd in $\mathbb R^3$.
    \end{example}

\section{The laws of thermodynamics}

    Thermodynamics is governed by a set of laws that every system must obey. However, they are limitation laws, since they tell us only which processes cannot happen. 

    \begin{law}[0th]
        Let $A$ and $B$ be $2$ thermodynamic systems in thermal contact. At equilibrium, only a subset of states $\mathcal M_A \times \mathcal M_B$ is accessible. Mathematically, it means that there exists a functional relation 
        \begin{equation}\label{a1}
            F_{AB} (a,b)= 0 ~,
        \end{equation}
        with $a \in \mathcal M_A$ and $b \in \mathcal M_B$. Moreover, thermal equilibrium is an equivalence class. It can be proved that the latter means that 
        \begin{equation}\label{a2}
            F_{AB} (a,b) = f_A(a) - f_B(b) ~.
        \end{equation}
        Putting together~\eqref{a1} and~\eqref{a2}, we define the empirical temperature 
        \begin{equation*}
            t_A = f_A(a) = t_B = f_B(b) ~.
        \end{equation*}
    \end{law}

    It is a limitation law because it limit the configuration that a system can reach in thermal contact with a second one. 

    \begin{law}[1st]
        Let $\delta Q$ be the infinitesimal heat and $\delta L$ the infinitesimal work exchanged in a quasi-static process ($\delta Q > 0$ means absorbed by the system, $\delta L > 0$ means performed by the system). For any cyclic process 
        \begin{equation*}
            \oint (\delta Q - \delta L) = 0 ~.
        \end{equation*}
        This means that $\delta Q - \delta L$ is a one-form, which vanishes line-integrated along a closed curve in $\mathcal M$. This implies, by the Poincaré lemma, that it is an exact differential 
        \begin{equation*}
            dE = \delta Q - \delta L ~,
        \end{equation*}
        called the internal energy. However, notice that heat and word are not exact differential, since $\oint \delta Q \neq 0$ and $\oint \delta H \neq 0$. We can generalised to a system that can exchange matter with 
        \begin{equation}\label{first}
            \oint (\delta Q - \delta L + \mu dN) = 0 ~, \quad dE = \delta Q - \delta L + \mu dN ~,
        \end{equation} 
        where $\mu$ is the chemical potential (the necessary energy to add or remove a particle). Furthermore, we can expressed both $\delta Q$ and $\delta L$ as a linear combination of infinitesimal change of independent coordinates. We assume that the internal energy is extensive. Therefore, the chemical potential is intensive.
    \end{law}

    It is a limitation law because it limits the configuration that a system can reach in isolation to whose with $E = const$. 

    \begin{law}[2nd]
        For any cyclic process
        \begin{equation*}
            \oint \frac{\delta Q}{T} \begin{cases}
                = 0 & \textnormal{reversible process} \\
                < 0 & \textnormal{irreversible process} \\
            \end{cases} ~.
        \end{equation*}
        For reversible processes, $\pdv{\delta Q}{T} = 0$ is an exact differential. This implies that we can define 
        \begin{equation*}
            S(a) - S(b) = \int_a^b \frac{\delta S}{T} ~,
        \end{equation*}
        called the entropy. The integral is along any reversible path. Therefore, we have  
        \begin{equation}\label{second}
            dS \begin{cases}
                = 0 & \textnormal{reversible process} \\
                < 0 & \textnormal{irreversible process} \\
            \end{cases} ~.
        \end{equation}
    \end{law}

    It is a limitation law because it limits the configuration that a system can reach in isolation to whose in which entropy cannot decrease. 

    \begin{law}[3rd]
        Isothermal and adiabatic processes coincides whet $T=0$, or, equivalently, it is impossible to reach $T=0$ with a finite number of processes. Mathematically 
        \begin{equation*}
        \Delta S \rightarrow 0 ~\textnormal{as}~ T \rightarrow 0 ~.
        \end{equation*}
        Therefore, $T=0$ is a singular point. Furthermore, if it were possible to reach $T=0$, the second law $\delta Q \leq 0$ implies that it is impossible to raise the temperature. It is a thermodynamic features, since it can be proved that it is impossible to realize an engine with efficency $\eta = 1$
    \end{law}

    It is a limitation law because it limits the configuration that a system can reach in isolation to whose in which $T \neq 0$.

\chapter{Thermodynamic potentials}

\section{The fundamental equation of thermodynamics}

    Combining~\eqref{first} and~\eqref{second}, we obtain 
    \begin{equation}\label{de}
        dE \begin{cases}
            = T dS - pdV + \mu dN & \textnormal{reversible process} \\
            < T dS - pdV + \mu dN & \textnormal{irreversible process} \\
        \end{cases} ~.
    \end{equation}
    \begin{proof}
        In fact, we invert~\eqref{first}
        \begin{equation*}
            \delta Q = dE + \delta L - \mu dN ~,
        \end{equation*}
        we use $\delta L = p dV$ 
        \begin{equation*}
            \delta Q = dE + p dV - \mu dN ~,
        \end{equation*}
        and we put it into~\eqref{second}
        \begin{equation}
            dS \leq \frac{\delta Q}{T} = \frac{dE + p dV - \mu dN}{T} ~,
        \end{equation}
        \begin{equation}
            dE \leq TdS - p dV + \mu dN ~.
        \end{equation}
    \end{proof} 

    Notice that the non-differential variables are intensive and the differential variables are extensive. This tells us that $E(S, V, N)$ is a function of the extensive variables $S$, $V$ and $N$. The intensive variables $T$, $p$ and $\mu$ can be derived from $E$ by the following relations 
    \begin{equation}\label{tpm}
        T = \pdv{E}{S} \Big \vert_{V,N} ~, \quad p = - \pdv{E}{V} \Big \vert_{S,N} ~, \quad \mu = \pdv{E}{N} \Big \vert_{S,V} ~. 
    \end{equation}
    This are called the equation of state of the system, since we can calculate one variable from it, e.g. $T = T(S,V,N)$, $p = p(S,V,N)$ or $\mu = \mu(S,V,N)$. It is important to say that this is all thermodynamics can tell us, thus in order to find the explicit expression of $E$, we must go into statistical mechanics.
    \begin{proof}
        At constant $V$ and $N$,~\eqref{de} becomes
        \begin{equation*}
            dE = TdS - p \underbrace{dV}_0 + \mu \underbrace{dN}_0 = TdS ~,
        \end{equation*}
        hence 
        \begin{equation*}
            T = \pdv{E}{S} \Big \vert_{V,N} ~.
        \end{equation*}

        At constant $S$ and $N$,~\eqref{de} becomes
        \begin{equation*}
            dE = T\underbrace{dS}_0 - p dV + \mu \underbrace{dN}_0 = - p dV ~,
        \end{equation*}
        hence 
        \begin{equation*}
            p = - \pdv{E}{V} \Big \vert_{S,N} ~.
        \end{equation*}

        At constant $S$ and $V$,~\eqref{de} becomes
        \begin{equation*}
            dE = T\underbrace{dS}_0 - p \underbrace{dV}_0 + \mu dN = \mu dN ~,
        \end{equation*}
        hence 
        \begin{equation*}
            \mu = \pdv{E}{S} \Big \vert_{S,V} ~.
        \end{equation*}
    \end{proof}

    $E$ is an extensive variable, i.e.~an homogeneous function of degree one of the extensive variables 
    \begin{equation*}
        E(\lambda S, \lambda V, \lambda N) = \lambda E(S, V, N) ~, \quad \forall \lambda > 0 ~.
    \end{equation*}
    The physical meaning is that if we rescale the volume, the energy is rescaled bu the same amount.

    Moreover, the intensive variable are homogeneous function of degree zero of the extensive variables 
    \begin{equation}\label{a5}
        T(S, V, N) = T(\frac{S}{N}, \frac{V}{N}) ~, \quad p(S, V, N) = p(\frac{S}{N}, \frac{V}{N}) ~, \quad \mu(S, V, N) = \mu(\frac{S}{N}, \frac{V}{N}) ~.
    \end{equation}

    By homogeneity properties 
    \begin{equation*}
        E = N E (\frac{S}{N}, \frac{V}{N}, 1) = N e ~, \quad  S = N S(\frac{E}{N}, \frac{V}{N}, 1) = N s ~,
    \end{equation*}
    we can define specific energy and entropy 
    \begin{equation*}
        e = \frac{E}{N} = e(s, v) ~, \quad s = \frac{S}{N} = s(e, v) ~,
    \end{equation*}
    where $v = \frac{V}{N}$ is the specific volume.

    The Euler's theorem allows us to state that, if $E$ is smooth, it can be written as 
    \begin{equation*}
        E = S \pdv{E}{S} + V \pdv{E}{V} + N \pdv{E}{N} ~,
    \end{equation*}
    or, using~\eqref{de} and~\eqref{tpm}, 
    \begin{equation}\label{e}
        E = TS - pV + \mu N ~.
    \end{equation}

    The Gibbs-Duhem relation expresses $\mu$ in terms of $p$ and $T$ 
    \begin{equation}\label{gdrel}
        S dT - Vdp + N d\mu = 0 ~, \quad \mu = v dp - s dT ~.
    \end{equation}
    \begin{proof}
        Computing the differential of~\eqref{e} 
        \begin{equation*}
            dE = T dS + S dT -p dV + \mu dN + N d\mu 
        \end{equation*}
        and comparing it with~\eqref{de}
        \begin{equation*}
            dE = \cancel{T dS} + S dT - \cancel{p dV} + - V dp + \cancel{\mu dN} + N d\mu = \cancel{T dS} - \cancel{p dV} + \cancel{\mu dN} ~,
        \end{equation*}
        we obtain 
        \begin{equation*}
            S dT - V dp + N d\mu = 0 ~.
        \end{equation*}
        which can be written as 
        \begin{equation*}
            d \mu = \frac{V}{N} dp - \frac{S}{N} dT = v dp - s dT ~.
        \end{equation*}
    \end{proof}

    Inverting~\eqref{de}, we obtained the entropy differential
    \begin{equation}\label{ds}
        dS = \frac{1}{T} dE + \frac{p}{T} dV - \frac{\mu}{T} dN ~.
    \end{equation}
    Its equations of state are 
    \begin{equation}\label{ses}
        \frac{1}{T} = \pdv{S}{E} \Big \vert_{V, N} ~, \quad \frac{p}{T} = \pdv{S}{V} \Big \vert_{E, N} ~, \quad - \frac{\mu}{T} = \pdv{S}{N} \Big \vert_{E, V} ~.
    \end{equation}

\section{Integrability condition} 

    In order to be an exact differential, the exterior derivative of the right handed side of~\eqref{de} must have vanishing exterior derivative
    \begin{equation}\label{intconden}
        - \pdv{T}{V} \Big \vert_{S,N} = \pdv{p}{S} \Big \vert_{V,N} ~, \quad 
        \pdv{T}{N} \Big \vert_{S,V} = \pdv{\mu}{S} \Big \vert_{N, V} ~, \quad 
        - \pdv{p}{N} \Big \vert_{V,S} = \pdv{\mu}{V} \Big \vert_{N, S} ~. 
    \end{equation}
    \begin{proof}
        By means of the exterior derivative, we have 
        \begin{equation*}
        \begin{aligned}
            d (dE) & = d (T dS) - d (p dV) + d (\mu dN) \\ & = \pdv{T}{S} \underbrace{dS \wedge dS}_0 + \pdv{T}{V} dV \wedge dS + \pdv{T}{N} dN \wedge dS - \pdv{p}{S} dS \wedge dV - \pdv{p}{V} \underbrace{dV \wedge dV}_0 \\ & \quad - \pdv{p}{N} dN \wedge dV + \pdv{\mu}{S} dS \wedge dN + \pdv{\mu}{V} dV \wedge dN + \pdv{\mu}{N} \underbrace{dN \wedge dN}_0 \\ & = \pdv{T}{V} dV \wedge dS + \pdv{T}{N} dN \wedge dS - \pdv{p}{S} dS \wedge dV \\ & \quad - \pdv{p}{N} dN \wedge dV + \pdv{\mu}{S} dS \wedge dN + \pdv{\mu}{V} dV \wedge dN ~.
        \end{aligned}
        \end{equation*}

        At constant $N$ 
        \begin{equation*}
        \begin{aligned}
            0 & = d^2 E = \pdv{T}{V} dV \wedge dS + \pdv{T}{N} \underbrace{dN}_0 \wedge dS - \pdv{p}{S} dS \wedge dV \\ & \quad - \pdv{p}{N} \underbrace{dN}_0 \wedge dV + \pdv{\mu}{S} dS \wedge \underbrace{dN}_0 + \pdv{\mu}{V} dV \wedge \underbrace{dN}_0 \\ & = \pdv{T}{V} dV \wedge dS - \pdv{p}{S} dS \wedge dV = \pdv{T}{V} dV \wedge dS + \pdv{p}{S} dV \wedge dS ~,
        \end{aligned}
        \end{equation*}
        hence, by the linear independence of $V$ and $S$,
        \begin{equation*}
            - \pdv{T}{V} \Big \vert_{S,N} = \pdv{p}{S} \Big \vert_{V,N} ~.
        \end{equation*}

        At constant $V$ 
        \begin{equation*}
        \begin{aligned}
            0 & = d^2 E = \pdv{T}{V} \underbrace{dV}_0 \wedge dS + \pdv{T}{N} dN \wedge dS - \pdv{p}{S} dS \wedge \underbrace{dV}_0 \\ & \qquad - \pdv{p}{N} dN \wedge \underbrace{dV}_0 + \pdv{\mu}{S} dS \wedge dN + \pdv{\mu}{V} \underbrace{dV}_0 \wedge dN \\ & = \pdv{T}{N} dN \wedge dS + \pdv{\mu}{S} dS \wedge dN = \pdv{T}{N} dN \wedge dS - \pdv{\mu}{S} dN \wedge dS~,
        \end{aligned}
        \end{equation*}
        hence, by the linear independence of $N$ and $S$,
        \begin{equation*}
            \pdv{T}{N} \Big \vert_{S,V} = \pdv{\mu}{S} \Big \vert_{N, V} ~.
        \end{equation*}

        At constant $S$ 
        \begin{equation*}
        \begin{aligned}
            0 & = d^2 E = \pdv{T}{V} dV \wedge \underbrace{dS}_0 + \pdv{T}{N} dN \wedge \underbrace{dS}_0 - \pdv{p}{S} \underbrace{dS}_0 \wedge dV \\ & \qquad - \pdv{p}{N} dN \wedge dV + \pdv{\mu}{S} \underbrace{dS}_0 \wedge dN + \pdv{\mu}{V} dV \wedge dN \\ & = - \pdv{p}{N} dN \wedge dV + \pdv{\mu}{V} dV \wedge dN = - \pdv{p}{N} dN \wedge dV - \pdv{\mu}{V} dN \wedge dV ~,
        \end{aligned}
        \end{equation*}
        hence, by the linear independence of $N$ and $V$,
        \begin{equation*}
            - \pdv{p}{N} \Big \vert_{V,S} = \pdv{\mu}{V} \Big \vert_{N, S} ~.
        \end{equation*}
    \end{proof}

\section{Thermodynamic states as a manifold}

    $S$, $V$ and $N$ as independent (local) coordinates, i.e.~a chart for $\mathcal M$. In our case, it can be thought as an open subset of $\mathbb R^3$. A reversible process is a path. An irreversible process is an oriented path. Different thermodynamic systems are not connected by any process. Therefore, the manifold is path-connected and simply connected. Solving thermodynamics means find the fundamental equation 
    \begin{equation*}
        E = E(S, V, N) ~.
    \end{equation*}
    However, we could have chosen as fundamental equation 
    \begin{equation*}
        S = S(E, V, N) 
    \end{equation*}
    and a chart would have had $E$, $V$ and $N$ as coordinates. At least one of the local coordinates in any chart for $\mathcal M$ must be extensive. 
    \begin{proof}
        By the $0th$ law and~\eqref{a5}, there must exist a functional relation between intensive variables. This means that one of the three is already fixed once the other two are given and they cannot be used all three as independent coordinates.
    \end{proof}

    There are different thermodynamic potentials, which are functions of $3$ independent variables that can be used to define a different chart for $\mathcal M$. Therefore, there are different aproaches to thermodynamics.

    Thermodynamic potentials can be obtained by various kind of Legendre transforms of~\eqref{de}, which exchange the role of an extensive variable to its conjugate intensive variable as independent variable. We require that the hypothesis of the inverse function theorem are satisfied, e.g. 
    \begin{equation*}
        \pdvdu{E}{S} \Big \vert_{V,N} \neq 0 ~, \quad \pdvdu{E}{V} \Big \vert_{S,N} \neq 0 ~, \quad \pdvdu{E}{N} \Big \vert_{S, V} \neq 0 ~.
    \end{equation*}

    The thermodynamic potentials are the Helmoltz free energy $F$, the entalpy $H$, the Gibbs free energy $G$ and the granpotential $\Omega$.

\section{Helmoltz free energy} 

    The Helmoltz free energy is defined as 
    \begin{equation*}
        F = E - TS ~.
    \end{equation*}
    Its differential is 
    \begin{equation*}
        dF \leq - S dT - p dV + \mu dN ~.
    \end{equation*}
    Therefore
    \begin{equation*}
        F = F(T, V, N) ~.
    \end{equation*}
    \begin{proof}
        By a Legendre transform, which means to complete a differential
        \begin{equation*}
            dE \leq T dS - p dV + \mu dN = d(TS) - S dT - p dV + \mu dN ~,
        \end{equation*}
        hence 
        \begin{equation*}
            dF = d(E - TS) \leq - S dT - p dV + \mu dN ~.
        \end{equation*}
    \end{proof}

    The equations of state are
    \begin{equation}\label{fes}
        S = - \pdv{F}{T} \Big \vert_{V,N} ~, \quad p = - \pdv{F}{V} \Big \vert_{T,N} ~, \quad \mu = \pdv{F}{N} \Big \vert_{T,V} ~. 
    \end{equation}
    \begin{proof}
        At constant $V$ and $N$
        \begin{equation*}
            dF = - S dT - p \underbrace{dV}_0 + \mu \underbrace{dN}_0 = - S dT~,
        \end{equation*}
        hence 
        \begin{equation*}
            S = - \pdv{F}{T} \Big \vert_{V,N} ~.
        \end{equation*}

        At constant $T$ and $N$
        \begin{equation*}
            dF = - S \underbrace{dT}_0 - p dV + \mu \underbrace{dN}_0 = - pdV ~,
        \end{equation*}
        hence 
        \begin{equation*}
            p = - \pdv{F}{V} \Big \vert_{T,N} ~.
        \end{equation*}

        At constant $T$ and $V$
        \begin{equation*}
            dF = - S \underbrace{dT}_0 - p \underbrace{dV}_0 + \mu dN = \mu dN~,
        \end{equation*}
        hence 
        \begin{equation*}
            \mu = \pdv{F}{N} \Big \vert_{T,V} ~.
        \end{equation*}
    \end{proof}

    The integrability conditions are 
    \begin{equation}\label{intcondhel}
        \pdv{S}{V} \Big \vert_{T,N} = \pdv{p}{T} \Big \vert_{V,N} ~, \quad 
        - \pdv{S}{N} \Big \vert_{T,V} = \pdv{\mu}{T} \Big \vert_{N, V} ~, \quad 
        - \pdv{p}{N} \Big \vert_{V,T} = \pdv{\mu}{V} \Big \vert_{N, T} ~. 
    \end{equation}
    \begin{proof}
        By means of the exterior derivative, we have 
        \begin{equation*}
        \begin{aligned}
            d (dF) & = - d (S dT) - d (p dV) + d (\mu dN) \\ & = - \pdv{S}{T} \underbrace{dT \wedge dT}_0 - \pdv{S}{V} dV \wedge dT - \pdv{S}{N} dN \wedge dT - \pdv{p}{T} dT \wedge dV - \pdv{p}{V} \underbrace{dV \wedge dV}_0 \\ & \quad - \pdv{p}{N} dN \wedge dV + \pdv{\mu}{T} dT \wedge dN + \pdv{\mu}{V} dV \wedge dN + \pdv{\mu}{N} \underbrace{dN \wedge dN}_0 \\ & = - \pdv{S}{V} dV \wedge dT - \pdv{S}{N} dN \wedge dT - \pdv{p}{T} dT \wedge dV \\ & \quad - \pdv{p}{N} dN \wedge dV + \pdv{\mu}{T} dT \wedge dN + \pdv{\mu}{V} dV \wedge dN ~.
        \end{aligned}
        \end{equation*}

        At constant $N$ 
        \begin{equation*}
        \begin{aligned}
            0 & = d^2 F = - \pdv{S}{V} dV \wedge dT - \pdv{S}{N} \underbrace{dN}_0 \wedge dT - \pdv{p}{T} dT \wedge dV \\ & \quad - \pdv{p}{N} \underbrace{dN}_0 \wedge dV + \pdv{\mu}{T} dT \wedge \underbrace{dN}_0 + \pdv{\mu}{V} dV \wedge \underbrace{dN}_0 \\ & = - \pdv{S}{V} dV \wedge dT - \pdv{p}{T} dT \wedge dV = - \pdv{S}{V} dV \wedge dT + \pdv{p}{T} dV \wedge dT  ~,
        \end{aligned}
        \end{equation*}
        hence, by the linear independence of $V$ and $T$,
        \begin{equation*}
            \pdv{S}{V} \Big \vert_{T,N} = \pdv{p}{T} \Big \vert_{V,N} ~.
        \end{equation*}

        At constant $V$ 
        \begin{equation*}
        \begin{aligned}
            0 & = d^2 F = - \pdv{S}{V} \underbrace{dV}_0 \wedge dT - \pdv{S}{N} dN \wedge dT - \pdv{p}{T} dT \wedge \underbrace{dV}_0 \\ & \quad - \pdv{p}{N} dN \wedge \underbrace{dV}_0 + \pdv{\mu}{T} dT \wedge dN + \pdv{\mu}{V} \underbrace{dV}_0 \wedge dN \\ & = - \pdv{S}{N} dN \wedge dT + \pdv{\mu}{T} dT \wedge dN = - \pdv{S}{N} dN \wedge dT - \pdv{\mu}{T} dN \wedge dT~,
        \end{aligned}
        \end{equation*}
        hence, by the linear independence of $N$ and $T$,
        \begin{equation*}
            - \pdv{S}{N} \Big \vert_{T,V} = \pdv{\mu}{T} \Big \vert_{N, V} ~.
        \end{equation*}

        At constant $T$ 
        \begin{equation*}
        \begin{aligned}
            0 & = d^2 F = - \pdv{S}{V} dV \wedge \underbrace{dT}_0 - \pdv{S}{N} dN \wedge \underbrace{dT}_0 - \pdv{p}{T} \underbrace{dT}_0 \wedge dV \\ & \quad - \pdv{p}{N} dN \wedge dV + \pdv{\mu}{T} \underbrace{dT}_0 \wedge dN + \pdv{\mu}{V} dV \wedge dN \\ & = - \pdv{p}{N} dN \wedge dV + \pdv{\mu}{V} dV \wedge dN =- \pdv{p}{N} dN \wedge dV - \pdv{\mu}{V} dN \wedge dV ~,
        \end{aligned}
        \end{equation*}
        hence, by the linear independence of $N$ and $V$,
        \begin{equation*}
            - \pdv{p}{N} \Big \vert_{V,T} = \pdv{\mu}{V} \Big \vert_{N, T} ~.
        \end{equation*}
    \end{proof}

\section{Enthalpy} 

    The enthalpy is defined as 
    \begin{equation*}
        H = E + pV ~.
    \end{equation*}
    Its differential is 
    \begin{equation*}
        dH \leq TdS + Vdp + \mu dN ~.
    \end{equation*}
    Therefore
    \begin{equation*}
        H = H(p, S, N) ~.
    \end{equation*}
    \begin{proof}
        By a Legendre transform, which means to complete a differential
        \begin{equation*}
            dE \leq T dS - p dV + \mu dN = TdS - d(pV) + V dp + \mu dN ~,
        \end{equation*}
        hence 
        \begin{equation*}
            dH = d(E + pV) \leq TdS + Vdp + \mu dN ~.
        \end{equation*}
    \end{proof}

    The equations of state are
    \begin{equation}
        T = \pdv{H}{S} \Big \vert_{p,N} ~, \quad V = - \pdv{H}{p} \Big \vert_{S,N} ~, \quad \mu = \pdv{H}{N} \Big \vert_{S, p} ~. 
    \end{equation}
    \begin{proof}
        At constant $p$ and $N$
        \begin{equation*}
            dH = TdS + V\underbrace{dp}_0 + \mu \underbrace{dN}_0 ~,
        \end{equation*}
        hence 
        \begin{equation*}
            T = \pdv{H}{S} \Big \vert_{p,N} ~.
        \end{equation*}

        At constant $S$ and $N$
        \begin{equation*}
            dH = T\underbrace{dS}_0 + Vdp + \mu \underbrace{dN}_0~,
        \end{equation*}
        hence 
        \begin{equation*}
            V = - \pdv{H}{p} \Big \vert_{S,N} ~.
        \end{equation*}

        At constant $S$ and $p$
        \begin{equation*}
            dH = T\underbrace{dS}_0 + V\underbrace{dp}_0 + \mu dN ~,
        \end{equation*}
        hence 
        \begin{equation*}
            \mu = \pdv{H}{N} \Big \vert_{S, p} ~.
        \end{equation*}
    \end{proof}

    The integrability conditions are 
    \begin{equation}\label{intcondent}
        \pdv{V}{S} \Big \vert_{p,N} = \pdv{T}{p} \Big \vert_{S,N} ~, \quad 
        \pdv{V}{N} \Big \vert_{p,S} = \pdv{\mu}{p} \Big \vert_{N, S} ~, \quad 
        \pdv{\mu}{S} \Big \vert_{N,p} = \pdv{T}{N} \Big \vert_{S, p} ~. 
    \end{equation}
    \begin{proof}
        By means of the exterior derivative, we have 
        \begin{equation*}
        \begin{aligned}
            d (dH) & = d (T dS) + d (V dp) + d (\mu dN) \\ & = \pdv{T}{S} \underbrace{dS \wedge dS}_0 + \pdv{T}{p} dp \wedge dS + \pdv{T}{N} dN \wedge dS + \pdv{V}{S} dS \wedge dp + \pdv{V}{p} \underbrace{dp \wedge dp}_0 \\ & \quad + \pdv{V}{N} dN \wedge dp + \pdv{\mu}{S} dS \wedge dN + \pdv{\mu}{p} dp \wedge dN + \pdv{\mu}{N} \underbrace{dN \wedge dN}_0 \\ & = \pdv{T}{p} dp \wedge dS + \pdv{T}{N} dN \wedge dS + \pdv{V}{S} dS \wedge dp \\ & \quad + \pdv{V}{N} dN \wedge dp + \pdv{\mu}{S} dS \wedge dN + \pdv{\mu}{p} dp \wedge dN  ~.
        \end{aligned}
        \end{equation*}

        At constant $N$ 
        \begin{equation*}
        \begin{aligned}
            0 & = d^2 H = \pdv{T}{p} dp \wedge dS + \pdv{T}{N} \underbrace{dN}_0 \wedge dS + \pdv{V}{S} dS \wedge dp \\ & \quad + \pdv{V}{N} \underbrace{dN}_0 \wedge dp + \pdv{\mu}{S} dS \wedge \underbrace{dN}_0 + \pdv{\mu}{p} dp \wedge \underbrace{dN}_0 \\ & = \pdv{T}{p} dp \wedge dS + \pdv{V}{S} dS \wedge dp = \pdv{T}{p} dp \wedge dS - \pdv{V}{S} dS \wedge dp ~,
        \end{aligned}
        \end{equation*}
        hence, by the linear independence of $S$ and $p$,
        \begin{equation*}
            \pdv{V}{S} \Big \vert_{p,N} = \pdv{T}{p} \Big \vert_{S,N} ~.
        \end{equation*}

        At constant $S$ 
        \begin{equation*}
        \begin{aligned}
            0 & = d^2 H = \pdv{T}{p} dp \wedge \underbrace{dS}_0 + \pdv{T}{N} dN \wedge \underbrace{dS}_0 + \pdv{V}{S} \underbrace{dS}_0 \wedge dp \\ & \quad + \pdv{V}{N} dN \wedge dp + \pdv{\mu}{S} \underbrace{dS}_0 \wedge dN + \pdv{\mu}{p} dp \wedge dN \\ & = \pdv{V}{N} dN \wedge dp + \pdv{\mu}{p} dp \wedge dN = \pdv{V}{N} dN \wedge dp - \pdv{\mu}{p} dN \wedge dp ~,
        \end{aligned}
        \end{equation*}
        hence, by the linear independence of $N$ and $p$,
        \begin{equation*}
            \pdv{V}{N} \Big \vert_{p,S} = \pdv{\mu}{p} \Big \vert_{N, S} ~.
        \end{equation*}

        At constant $p$ 
        \begin{equation*}
        \begin{aligned}
            0 & = d^2 H = \pdv{T}{p} \underbrace{dp}_0 \wedge dS + \pdv{T}{N} dN \wedge dS + \pdv{V}{S} dS \wedge \underbrace{dp}_0 \\ & \quad + \pdv{V}{N} dN \wedge \underbrace{dp}_0 + \pdv{\mu}{S} dS \wedge dN + \pdv{\mu}{p} \underbrace{dp}_0 \wedge dN \\ & = \pdv{T}{N} dN \wedge dS + \pdv{\mu}{S} dS \wedge dN = \pdv{T}{N} dN \wedge dS - \pdv{\mu}{S} dS \wedge dN ~,
        \end{aligned}
        \end{equation*}
        hence, by the linear independence of $S$ and $N$,
        \begin{equation*}
            \pdv{\mu}{S} \Big \vert_{N,p} = \pdv{T}{N} \Big \vert_{S, p} ~.
        \end{equation*}
    \end{proof}

\section{Gibbs free energy} 

    The Gibbs free energy is defined as 
    \begin{equation*}
        G = E - TS + pV = F + pV = H - TS ~.
    \end{equation*}
    Its differential is 
    \begin{equation*}
        dG \leq - SdT + Vdp + \mu dN ~.
    \end{equation*}
    Therefore
    \begin{equation*}
        G = G(p, T, N) ~.
    \end{equation*}
    \begin{proof}
        By a Legendre transform, which means to complete a differential
        \begin{equation*}
            dE \leq T dS - p dV + \mu dN = d(TS) - S dT - d(pV) + V dp + \mu dN ~,
        \end{equation*}
        hence 
        \begin{equation*}
            dG = d(E - TS + pV) \leq - S dT + Vdp + \mu dN ~.
        \end{equation*}
    \end{proof}

    The equations of state are
    \begin{equation}\label{ges}
        S = - \pdv{G}{T} \Big \vert_{p,N} ~, \quad V = \pdv{G}{p} \Big \vert_{T,N} ~, \quad \mu = \pdv{G}{N} \Big \vert_{p,T} ~. 
    \end{equation}
    \begin{proof}
        At constant $p$ and $N$
        \begin{equation*}
            dG = - S dT + V\underbrace{dp }_0 + \mu \underbrace{dN}_0  ~,
        \end{equation*}
        hence 
        \begin{equation*}
            S = - \pdv{G}{T} \Big \vert_{p,N}  ~.
        \end{equation*}

        At constant $T$ and $N$
        \begin{equation*}
            dG = - S \underbrace{dT}_0  + Vdp + \mu \underbrace{dN}_0  ~,
        \end{equation*}
        hence 
        \begin{equation*}
            V = \pdv{G}{p} \Big \vert_{T,N} ~.
        \end{equation*}

        At constant $p$ and $T$
        \begin{equation*}
            dG = - S \underbrace{dT}_0  + V\underbrace{dp}_0  + \mu dN ~,
        \end{equation*}
        hence 
        \begin{equation*}
            \mu = \pdv{G}{N} \Big \vert_{p,T} ~.
        \end{equation*}
    \end{proof}

    The integrability conditions are 
    \begin{equation}\label{intcondgib}
        - \pdv{V}{T} \Big \vert_{p,N} = \pdv{S}{p} \Big \vert_{T,N} ~, \quad 
        \pdv{V}{N} \Big \vert_{p,T} = \pdv{\mu}{p} \Big \vert_{N, T} ~, \quad 
        - \pdv{S}{N} \Big \vert_{T,p} = \pdv{\mu}{T} \Big \vert_{N, p} ~. 
    \end{equation}
    \begin{proof}
        By means of the exterior derivative, we have 
        \begin{equation*}
        \begin{aligned}
            d (dG) & = - d (S dT) + d (V dp) + d (\mu dN) \\ & = - \pdv{S}{T} \underbrace{dT \wedge dT}_0 - \pdv{S}{p} dp \wedge dT - \pdv{S}{N} dN \wedge dT + \pdv{V}{T} dT \wedge dp + \pdv{V}{p} \underbrace{dp \wedge dp}_0 \\ & \quad + \pdv{V}{N} dN \wedge dp + \pdv{\mu}{T} dT \wedge dN + \pdv{\mu}{p} dp \wedge dN + \pdv{\mu}{N} \underbrace{dN \wedge dN}_0 \\ & = - \pdv{S}{p} dp \wedge dT - \pdv{S}{N} dN \wedge dT + \pdv{V}{T} dT \wedge dp \\ & \quad + \pdv{V}{N} dN \wedge dp + \pdv{\mu}{T} dT \wedge dN + \pdv{\mu}{p} dp \wedge dN ~.
        \end{aligned}
        \end{equation*}

        At constant $N$ 
        \begin{equation*}
        \begin{aligned}
            0 & = d^2 G = - \pdv{S}{p} dp \wedge dT - \pdv{S}{N} \underbrace{dN}_0 \wedge dT + \pdv{V}{T} dT \wedge dp \\ & \quad + \pdv{V}{N} \underbrace{dN}_0 \wedge dp + \pdv{\mu}{T} dT \wedge \underbrace{dN}_0 + \pdv{\mu}{p} dp \wedge \underbrace{dN}_0 \\ & = - \pdv{S}{p} dp \wedge dT + \pdv{V}{T} dT \wedge dp = - \pdv{S}{p} dp \wedge dT - \pdv{V}{T} dp \wedge dT ~,
        \end{aligned}
        \end{equation*}
        hence, by the linear independence of $p$ and $T$,
        \begin{equation*}
            - \pdv{V}{T} \Big \vert_{p,N} = \pdv{S}{p} \Big \vert_{T,N} ~.
        \end{equation*}

        At constant $T$ 
        \begin{equation*}
        \begin{aligned}
            0 & = d^2 G = - \pdv{S}{p} dp \wedge \underbrace{dT}_0 - \pdv{S}{N} dN \wedge \underbrace{dT}_0 + \pdv{V}{T} \underbrace{dT}_0 \wedge dp \\ & \quad + \pdv{V}{N} dN \wedge dp + \pdv{\mu}{T} \underbrace{dT}_0 \wedge dN + \pdv{\mu}{p} dp \wedge dN \\ & = \pdv{V}{N} dN \wedge dp + \pdv{\mu}{p} dp \wedge dN = \pdv{V}{N} dN \wedge dp - \pdv{\mu}{p} dp \wedge dN ~,
        \end{aligned}
        \end{equation*}
        hence, by the linear independence of $p$ and $N$,
        \begin{equation*}
            \pdv{V}{N} \Big \vert_{p,T} = \pdv{\mu}{p} \Big \vert_{N, T} ~.
        \end{equation*}

        At constant $p$ 
        \begin{equation*}
        \begin{aligned}
            0 & = d^2 G = - \pdv{S}{p} \underbrace{dp}_0 \wedge dT - \pdv{S}{N} dN \wedge dT + \pdv{V}{T} dT \wedge \underbrace{dp}_0 \\ & \quad + \pdv{V}{N} dN \wedge \underbrace{dp}_0 + \pdv{\mu}{T} dT \wedge dN + \pdv{\mu}{p} \underbrace{dp}_0 \wedge dN \\ & = - \pdv{S}{N} dN \wedge dT + \pdv{\mu}{T} dT \wedge dN = - \pdv{S}{N} dN \wedge dT - \pdv{\mu}{T} dN \wedge dT ~,
        \end{aligned}
        \end{equation*}
        hence, by the linear independence of $N$ and $T$,
        \begin{equation*}
            - \pdv{S}{N} \Big \vert_{T,p} = \pdv{\mu}{T} \Big \vert_{N, p} ~.
        \end{equation*}

    \end{proof}

\section{Granpotential} 

    The granpotential is defined as 
    \begin{equation}\label{ome}
        \Omega = E - TS - \mu N = F - \mu N ~.
    \end{equation}
    Its differential is 
    \begin{equation*}
        d\Omega \leq - SdT - pdV - N d\mu ~.
    \end{equation*}
    Therefore
    \begin{equation*}
        \Omega = \Omega(T, V, \mu) ~.
    \end{equation*}
    \begin{proof}
        By a Legendre transform, which means to complete a differential
        \begin{equation*}
            dE \leq T dS - p dV + \mu dN = d(TS) - SdT - p dV + (\mu N) - N d\mu ~,
        \end{equation*}
        hence 
        \begin{equation*}
            d\Omega = d(E - TS - \mu N) \leq - SdT - p dV - N d\mu ~.
        \end{equation*}
    \end{proof}

    The equations of state are
    \begin{equation}
        S = - \pdv{\Omega}{T} \Big \vert_{\mu,V} ~, \quad p = - \pdv{\Omega}{V} \Big \vert_{T,\mu} ~, \quad \mu = - \pdv{\Omega}{N} \Big \vert_{T,V} ~. 
    \end{equation}
    \begin{proof}
        At constant $\mu$ and $V$
        \begin{equation*}
            d\Omega = - SdT - p\underbrace{dV}_0 - N \underbrace{d\mu}_0 = - S dT ~,
        \end{equation*}
        hence 
        \begin{equation*}
            S = - \pdv{\Omega}{T} \Big \vert_{\mu,V} ~.
        \end{equation*}

        At constant $T$ and $\mu$
        \begin{equation*}
            d\Omega = - S \underbrace{dT}_0 - pdV - N \underbrace{d\mu}_ 0 = - p dV ~,
        \end{equation*}
        hence 
        \begin{equation*}
            p = - \pdv{\Omega}{V} \Big \vert_{T,\mu} ~.
        \end{equation*}

        At constant $T$ and $V$
        \begin{equation*}
            d\Omega = - S\underbrace{dT}_0 - p\underbrace{dV}_0 - N d\mu = - N d\mu~,
        \end{equation*}
        hence 
        \begin{equation*}
            \mu = - \pdv{\Omega}{N} \Big \vert_{T,V} ~.
        \end{equation*}
    \end{proof}

    The integrability conditions are 
    \begin{equation}\label{intcondom}
        \pdv{S}{\mu} \Big \vert_{T,V} = \pdv{N}{T} \Big \vert_{\mu,V} ~, \quad 
        \pdv{S}{V} \Big \vert_{T,\mu} = \pdv{p}{T} \Big \vert_{V, \mu} ~, \quad 
        \pdv{p}{\mu} \Big \vert_{V,T} = \pdv{N}{V} \Big \vert_{\mu, T} ~. 
    \end{equation}
    \begin{proof}
        By means of the exterior derivative, we have 
        \begin{equation*}
        \begin{aligned}
            d (d\Omega) & = - d (S dT) - d (p dV) - d (N d\mu) \\ & = - \pdv{S}{T} \underbrace{dT \wedge dT}_0 - \pdv{S}{V} dV \wedge dT - \pdv{S}{\mu} d\mu \wedge dT - \pdv{p}{T} dT \wedge dV - \pdv{p}{V} \underbrace{dV \wedge dV}_0 \\ & \quad - \pdv{p}{\mu} d\mu \wedge dV - \pdv{N}{T} dT \wedge d\mu - \pdv{N}{V} dV \wedge d\mu - \pdv{N}{\mu} \underbrace{d\mu \wedge d\mu}_0 \\ & = - \pdv{S}{V} dV \wedge dT - \pdv{S}{\mu} d\mu \wedge dT - \pdv{p}{T} dT \wedge dV \\ & \quad - \pdv{p}{\mu} d\mu \wedge dV - \pdv{N}{T} dT \wedge d\mu - \pdv{N}{V} dV \wedge d\mu ~.
        \end{aligned}
        \end{equation*}

        At constant $\mu$ 
        \begin{equation*}
        \begin{aligned}
            0 & = d^2 \Omega = - \pdv{S}{V} dV \wedge dT - \pdv{S}{\mu} \underbrace{d\mu}_0 \wedge dT - \pdv{p}{T} dT \wedge dV \\ & \quad - \pdv{p}{\mu} \underbrace{d\mu}_0 \wedge dV - \pdv{N}{T} dT \wedge \underbrace{d\mu}_0 - \pdv{N}{V} dV \wedge \underbrace{d\mu}_0 \\ & = - \pdv{S}{V} dV \wedge dT - \pdv{p}{T} dT \wedge dV = - \pdv{S}{V} dV \wedge dT + \pdv{p}{T} dV \wedge dT ~,
        \end{aligned}
        \end{equation*}
        hence, by the linear independence of $V$ and $T$,
        \begin{equation*}
            \pdv{S}{V} \Big \vert_{T,\mu} = \pdv{p}{T} \Big \vert_{V, \mu} ~.
        \end{equation*}

        At constant $V$ 
        \begin{equation*}
        \begin{aligned}
            0 & = d^2 \Omega = - \pdv{S}{V} \underbrace{dV}_0 \wedge dT - \pdv{S}{\mu} d\mu \wedge dT - \pdv{p}{T} dT \wedge \underbrace{dV}_0 \\ & \quad - \pdv{p}{\mu} d\mu \wedge \underbrace{dV}_0 - \pdv{N}{T} dT \wedge d\mu - \pdv{N}{V} \underbrace{dV}_0 \wedge d\mu \\ & = - \pdv{S}{\mu} d\mu \wedge dT - \pdv{N}{T} dT \wedge d\mu = - \pdv{S}{\mu} d\mu \wedge dT + \pdv{N}{T} d\mu \wedge dT ~,
        \end{aligned}
        \end{equation*}
        hence, by the linear independence of $\mu$ and $T$,
        \begin{equation*}
            \pdv{S}{\mu} \Big \vert_{T,V} = \pdv{N}{T} \Big \vert_{\mu,V} ~.
        \end{equation*}

        At constant $T$ 
        \begin{equation*}
        \begin{aligned}
            0 & = d^2 \Omega = - \pdv{S}{V} dV \wedge \underbrace{dT}_0 - \pdv{S}{\mu} d\mu \wedge \underbrace{dT}_0 - \pdv{p}{T} \underbrace{dT}_0 \wedge dV \\ & \quad - \pdv{p}{\mu} d\mu \wedge dV - \pdv{N}{T} \underbrace{dT}_0 \wedge d\mu - \pdv{N}{V} dV \wedge d\mu \\ & = - \pdv{p}{\mu} d\mu \wedge dV - \pdv{N}{V} dV \wedge d\mu=  - \pdv{p}{\mu} d\mu \wedge dV + \pdv{N}{V} d\mu \wedge dV ~,
        \end{aligned}
        \end{equation*}
        hence, by the linear independence of $N$ and $V$,
        \begin{equation*}
            \pdv{p}{\mu} \Big \vert_{V,T} = \pdv{N}{V} \Big \vert_{\mu, T} ~.
        \end{equation*}
    \end{proof}

\subsection{Comments}

    The thermodynamic potential are not homogeneous functions since they depend on mixed extensive and intensive variables. However, they are extensive, i.e. 
    \begin{equation}\label{a6}
        F = N f(T, v) ~, \quad H = N h(p, s) ~, \quad G = N g(T, p) ~, \quad \Omega = N f \omega (T, \mu) ~. 
    \end{equation}
    Notice that the chemical potential is also the Gibbs free energy per particle
    \begin{equation}
        g(T, p) = \mu(T, p) ~.
    \end{equation}
    \begin{proof}
        In fact, using~\eqref{ges} and \eqref{a6}
        \begin{equation*}
            \mu = \pdv{G}{N} = \pdv{Ng }{N} = g ~.
        \end{equation*}
    \end{proof}

    Notice that 
    \begin{equation*}
        \Omega = - pV ~.
    \end{equation*}
    \begin{proof}
        Using~\eqref{e} and~\eqref{ome}
        \begin{equation*}
            \Omega = E - TS - \mu N = \cancel{TS} - pV + \cancel{\mu N} - \cancel{TS} - \cancel{\mu N} = - pV ~.
        \end{equation*}
    \end{proof}

\chapter{Maxwell's relations and stability conditions}

\section{Maxwell's relations}

    Integrability condition can be written as jacobian determinant. 

    For the energy, they are
    \begin{equation*}
        \pdv{(p, S, V )}{(N, S, V)} = - \pdv{(\mu, S, N)}{(V, S, N)} = \pdv{\mu, S, N}{N, S, V} ~.
    \end{equation*}
    \begin{proof}
        Using the first of~\eqref{intconden}
        \begin{equation*}
            - \pdv{T}{V} \Big \vert_{S, N} = - \pdv{p}{S} \Big \vert_{V, N} \rightarrow \pdv{(T, N, S)}{(V, N, S)} = - \pdv{(p, N, V)}{(S, N, V)} = \pdv{(p, N, V)}{(V, N, S)} ~,
        \end{equation*} 
        hence, inverting the right-handed side
        \begin{equation*}
            1 = \pdv{(T, N, S)}{(V, N, S)} \pdv{(p, N, V)}{(V, N, S)}^{-1} = \pdv{(T, N, S)}{(V, N, S)} \pdv{(V, N, S)}{(p, N, V)} = \pdv{(T, N, S)}{(p, N, V)} ~.
        \end{equation*} 

        Using the second of~\eqref{intconden}
        \begin{equation*}
            \pdv{T}{N} \Big \vert_{S, V} = - \pdv{\mu}{S} \Big \vert_{N, V} \rightarrow \pdv{(T, V, S)}{(N, V, S)} = - \pdv{(\mu, V, N)}{(S, V, N)} = \pdv{(\mu, V, N)}{(N, V, S)} ~,
        \end{equation*} 
        hence, inverting the right-handed side
        \begin{equation*}
            1 = \pdv{(T, V, S)}{(N, V, S)} \pdv{(\mu, V, N)}{(N, V, S)}^{-1} = \pdv{(T, V, S)}{(N, V, S)} \pdv{(N, V, S)}{(\mu, V, N)} = \pdv{(T, V, S)}{(\mu, V, N)} ~.
        \end{equation*} 

        Using the third of~\eqref{intconden}
        \begin{equation*}
            \pdv{p}{N} \Big \vert_{V, S} = - \pdv{\mu}{V} \Big \vert_{N, S} \rightarrow \pdv{(p, S, V )}{(N, S, V)} = - \pdv{(\mu, S, N)}{(V, S, N)} = \pdv{(\mu, S, N)}{(N, S, V)} ~,
        \end{equation*} 
        hence, inverting the right-handed side
        \begin{equation*}
            1 = \pdv{(p, S, V )}{(N, S, V)} \pdv{(\mu, S, N)}{(N, S, V)}^{-1} = \pdv{(p, S, V )}{(N, S, V)} \pdv{(N, S, V)}{(\mu, S, N)} = \pdv{(p, S, V )}{(\mu, S, N)} ~.
        \end{equation*} 
    \end{proof}

    TO BE CONTINUED.


    Not all the Maxwell's relations are independent, but only $6$ of them 
    \begin{equation*}
        \pdv{(p, V, S)}{(\mu, N, S)} = 1 ~, \quad \pdv{(p, V, T)}{(\mu, N, T)} = 1 ~, \quad \pdv{(p, V, N)}{(T, S, N)} = 1 ~, 
    \end{equation*}
    \begin{equation*}
        \pdv{(T, S, \mu)}{(p, V, \mu)} = 1 ~, \quad \pdv{(T, S, p)}{(N, \mu, p)} = 1 ~, \quad \pdv{(T, S, V)}{(N, \mu, V)} = 1 ~.
    \end{equation*}

    The intergability conditions written in term of jacobian determinant give rise to the geometrical interpretation: the coordinate transformations, which mean that we changed into a different chart of independent variables, preserves the volume.

\section{Stability conditions}

    Every thermodynamic potential has a natural chart. In fact, the configuration of stable equilbrium can be obtained by a set of variational principle, which can be derived by fixing to constants the natural independent variables. This variations principle derive from the second law of thermodynamics, since all systems evolve spontaneously to maximise the entropy. Therefore, minima of the thermodynamic potentials correspond to stable equilibrium under boundary condition which keep constant the natural variables
    \begin{equation*}
        (T, V, N) = const \rightarrow \delta F = 0 ~, \delta^2 F > 0 ~, 
    \end{equation*}
    \begin{equation*}
        (S, p, N) = const \rightarrow \delta H = 0 ~, \delta^2 H > 0 ~, 
    \end{equation*}
    \begin{equation*}
        (T, p, N) = const \rightarrow \delta G = 0 ~, \delta^2 G > 0 ~, 
    \end{equation*}
    \begin{equation*}
        (T, V, \mu) = const \rightarrow \delta \Omega = 0 ~, \delta^2 \Omega > 0 ~.
    \end{equation*}

    Equilibrium of two subsystems requires that $T$, $p$ and $\mu$ are equal.
    \begin{proof}
        Consider two subsystems $A$ and $B$ with extensive variables $(E_A, V_A, N_A)$ and $(E_B, V_B, N_B)$. Therefore $E = E_A + E_B$, $V = V_A + V_B$ and $N = N_A + N_B$. The whole system is at fixed boundary conditions $E, V, S = const$. The entropy is additive 
        \begin{equation*}
            S = S_A + S_B = S_A(E_A, V_A, N_A) - S_B(E - E_A, V-V_A, N-N_A) ~.
        \end{equation*}
        Computing its derivative and imposing it to zero, using~\eqref{ses}
        \begin{equation*}
        \begin{aligned}
            0 & = \delta S = \pdv{S_A}{E_A} \delta E_A + \pdv{S_A}{E_A} \delta E_A + \pdv{S_A}{V_A} \delta V_A + \pdv{S_A}{N_A} \delta N_A \\ & \quad + \pdv{S_B}{E_B} \underbrace{\delta (E - E_A)}_{- \delta E_A} + \pdv{S_B}{V_B} \underbrace{\delta (V - V_A)}_{- \delta V_A} + \pdv{S_B}{N_B} \underbrace{\delta (N - N_A)}_{- \delta N_A} \\ & = \pdv{S_A}{E_A} \delta E_A + \pdv{S_A}{V_A} \delta V_A + \pdv{S_A}{N_A} \delta N_A - \pdv{S_B}{E_B} \delta E_A - \pdv{S_B}{V_B}  \delta V_A - \pdv{S_B}{N_B} \delta N_A  \\ & = \delta E_A \Big ( \underbrace{\pdv{S_A}{E_A}}_{\frac{1}{T_A}} - \underbrace{\pdv{S_B}{E_B}}_{\frac{1}{T_B}} \Big) + \delta V_A \Big ( \underbrace{\pdv{S_A}{V_A}}_{\frac{p_A}{T_A}} - \underbrace{\pdv{S_B}{E_B}}_{\frac{p_B}{T_B}} \Big) + \delta N_A \Big (\underbrace{\pdv{S_A}{N_A}}_{ - \frac{\mu_A}{T_A}} - \underbrace{\pdv{S_B}{N_B}}_{- \frac{\mu_B}{T_B}} \Big) \\ & = \delta E_A \Big ( \frac{1}{T_A} - \frac{1}{T_B} \Big) + \delta V_A \Big ( \frac{p_A}{T_A} - \frac{p_B}{T_B} \Big) + \delta N_A \Big (- \frac{\mu_A}{T_A} + \frac{\mu_B}{T_B} \Big) ~,
        \end{aligned}
        \end{equation*}
        hence, by the arbitrarity of $\delta E_A$, $\delta V_A$ and $\delta N_A$,
        \begin{equation*}
            T_A = T_B ~, \quad p_A = p_B ~, \quad \mu_A = \mu_B ~.
        \end{equation*}
    \end{proof}

    At $T, p, N = const$, the stability condition is 
    \begin{equation}\label{stab}
    \begin{aligned}
        & E_{SS} = \pdv{T}{S} \Big \vert_V > 0 ~, \quad E_{VV} = - \pdv{p}{V} \Big \vert_S > 0 ~, \\ & E_{SS}E_{VV} - E^2_{SV} = - \pdv{T}{S} \Big \vert_V \pdv{p}{V} \Big \vert_S - \Big ( \pdv{p}{S} \Big \vert_V \Big )^2 = - \pdv{T}{S} \Big \vert_V \pdv{p}{V} \Big \vert_S - \Big ( \pdv{T}{V} \Big \vert_S \Big )^2 > 0 ~, 
    \end{aligned}
    \end{equation}
    \begin{proof}
        We know that $E = E(S, V, N)$. AT constant $N$, its variation is 
        \begin{equation*}
        \begin{aligned}
            \delta E & = \underbrace{\pdv{E}{S} \Big \vert_V }_T \delta S + \underbrace{\pdv{E}{V} \Big \vert_S}_{-p} \delta V  \\ & \quad + \frac{1}{2} \Big ( \underbrace{\pdvdu{E}{S} \Big \vert_V}_{E_{SS}} \delta S^2 + 2 \underbrace{\pdvd{E}{S}{V}}_{E_{SV}} \delta S \delta V + \underbrace{\pdvdu{E}{V} \Big \vert_S }_{E_{VV}} \delta V^2 \Big) \\ & = T \delta S - p \delta V + \frac{1}{2} \Big ( E_{SS} \delta S^2 + 2 E_{SV} \delta S \delta V + E_{VV} \delta V^2 \Big) ~.
        \end{aligned}
        \end{equation*}

        The first derivative terms vanishes, since 
        \begin{equation*}
        \begin{aligned}
            \delta G &= \delta E - T \delta S + p \delta V \\ & = \cancel{T \delta S} - \cancel{p \delta V} + \frac{1}{2} \Big ( E_{SS} \delta S^2 + 2 E_{SV} \delta S \delta V + E_{VV} \delta V^2 \Big) - \cancel{T \delta S} + \cancel{p \delta V} \\ & = \frac{1}{2} \Big ( E_{SS} \delta S^2 + 2 E_{SV} \delta S \delta V + E_{VV} \delta V^2 \Big) ~.
        \end{aligned}
        \end{equation*}

        The condition to be a minimum is that 
        \begin{equation*}
            E_{SS} > 0 ~, \quad E_{VV} > 0 ~, \quad E_{SS} E_{VV} - E_{SV}^2 > 0 ~.
        \end{equation*}

        Respectively, they become 
        \begin{equation*}
            E_{SS} = \pdv{}{S} \underbrace{\pdv{E}{S}}_{T} = \pdv{T}{S} > 0 ~,
        \end{equation*}
        \begin{equation*}
            E_{VV} = \pdv{}{V} \underbrace{\pdv{E}{V}}_{-p} = - \pdv{p}{V} > 0 ~,
        \end{equation*}
        \begin{equation*}
            E_{SS}E_{VV} - E^2_{SV} = - \pdv{T}{S} \Big \vert_V \pdv{p}{V} \Big \vert_S - \Big ( \pdv{p}{S} \Big \vert_V \Big )^2 = - \pdv{T}{S} \Big \vert_V \pdv{p}{V} \Big \vert_S - \Big ( \pdv{T}{V} \Big \vert_S \Big )^2 > 0 ~.
        \end{equation*}
    \end{proof}

    We define the stability conditions in terms of the specific heat 
    \begin{equation*}
        C_V = T \pdv{S}{T} \Big \vert_{V} > 0 ~,
    \end{equation*}
    the adiabatic compressibility
    \begin{equation*}
        \chi_S = - \frac{1}{V} \pdv{V}{p} \Big \vert_S > 0
    \end{equation*}
    and the isothermal compressibility 
    \begin{equation*}
        \chi_T = - \frac{1}{V} \pdv{V}{p} \Big \vert_T > 0 ~.
    \end{equation*}
    \begin{proof}
        For the first, using~\eqref{stab} and $T > 0$
        \begin{equation*}
            C_V = T \pdv{S}{T} \Big \vert_{V} > 0 ~.
        \end{equation*}

        For the second, using~\eqref{stab} and $V > 0$
        \begin{equation*}
            \chi_S = - \frac{1}{V} \pdv{V}{p} \Big \vert_S > 0 ~.
        \end{equation*}
        
        For the third, using~\eqref{stab} and~\eqref{intcondgib}
        \begin{equation*}
        \begin{aligned}
            0 & < \pdv{T}{V} \Big \vert_S \pdv{T}{V} \Big \vert_S + \pdv{T}{S} \Big \vert_V \pdv{p}{V} \Big \vert_S \\ &  - \pdv{T}{V} \Big \vert_S \pdv{p}{S} \Big \vert_V + \pdv{T}{S} \Big \vert_V \pdv{p}{V} \Big \vert_S \\ & = \pdv{(T,p)}{(S,V)} \\ & = \pdv{(T,p)}{(S,V)} = \pdv{(T,p)}{(T,V)} \pdv{(T,V)}{(S,V)} \\ & = \pdv{p}{V} \Big \vert_T \pdv{T}{S} \Big \vert_V \\ & = \frac{T}{C_V} \pdv{p}{V} \Big \vert_T ~,
        \end{aligned}
        \end{equation*}
        hence, by $T>0$, $C_V>0$ and $V>0$, 
        \begin{equation*}
            \chi_T = - \frac{1}{V} \pdv{V}{p} \Big \vert_T > 0 ~.
        \end{equation*}
    \end{proof}

    Consequently to stability, $F$ is a concave of $T$ and covex of $V$, whereas $G$ is concave of both $T$ and $p$. 
    \begin{proof}
        For the concavity of $F$ of $T$
        \begin{equation*}
            C_V = T \pdv{S}{T} \Big \vert_{V} = - T \pdvdu{F}{T} \Big \vert_V > 0 ~,
        \end{equation*}
        hence 
        \begin{equation*}
            \pdvdu{F}{T} \Big \vert_V < 0 ~.
        \end{equation*}

        For the convexity of $F$ of $V$
        \begin{equation*}
            \chi_T = - \frac{1}{V} \pdv{V}{p} \Big \vert_T = \Big ( V \pdvdu{F}{V} \Big \vert_T \Big)^{-1} > 0 ~,
        \end{equation*}
        hence 
        \begin{equation*}
            \pdvdu{F}{V} \Big \vert_T > 0 ~.
        \end{equation*}

        For the concavity of $G$ of $T$
        \begin{equation*}
            C_P = T \pdv{S}{T} \Big \vert_{P} = - T \pdvdu{G}{T} \Big \vert_p > 0 ~,
        \end{equation*}
        hence 
        \begin{equation*}
            \pdvdu{G}{T} \Big \vert_p < 0 ~.
        \end{equation*}

        For the concavity of $G$ of $p$
        \begin{equation*}
            \chi_T = - \frac{1}{V} \pdv{V}{p} \Big \vert_T = - \frac{1}{V} \pdvdu{G}{p} \Big \vert_T > 0 ~,
        \end{equation*}
        hence 
        \begin{equation*}
            \pdvdu{G}{p} \Big \vert_T < 0 ~.
        \end{equation*}
    \end{proof}

    Furthermore, the second law of thermodynamics can be expressed, in order to maximise the entropy, by imposing that first derivatives vanish and the hessian, i.e. the matrix with its second derivatives, must be negative defined. Therefore, it must be (locally) concave in $E$, $V$ and $N$.

    When cease to work at constant $N$, the stability condition is 
    \begin{equation*}
        \pdv{N}{\mu} \Big \vert_{V, T} = \frac{N^2}{V} \chi_T > 0 ~.
    \end{equation*}
    \begin{proof}
        In fact 
        \begin{equation*}
        \begin{aligned}
            \pdv{N}{\mu} \Big \vert_{V, T} & = \pdv{(N, V, T)}{(\mu, V, T)} \\ & = \pdv{(N, V, T)}{(N, p, T)} \pdv{(N, p, T)}{(p, V, T)} 1 \pdv{(p, V, T)}{(\mu, V, T)} \\ & = \pdv{(N, V, T)}{(N, p, T)} \pdv{(N, p, T)}{(p, V, T)} \pdv{(p, V, T)}{(\mu, N, T)} \pdv{(p, V, T)}{(\mu, V, T)} \\ & = \pdv{(N, V, T)}{(N, p, T)} \pdv{(N, p, T)}{(\mu, N, T)} \pdv{(p, V, T)}{(\mu, V, T)} \\ & = - \pdv{V}{p} \Big \vert_{N, T}  \pdv{p}{\mu} \Big \vert_{V, T}  \pdv{p}{\mu} \Big \vert_{N, T} 
        \end{aligned}
        \end{equation*}

        Now we use~\eqref{gdrel}
        \begin{equation*}
            \pdv{p}{\mu} \Big \vert_{V, T} = \pdv{p}{\mu} \Big \vert_{N, T} = \Big ( \pdv{\mu}{p} \Big \vert_T \Big) = \frac{N}{V} ~,
        \end{equation*}
        hence 
        \begin{equation*}
        \pdv{N}{\mu} \Big \vert_{V, T} = - \frac{N^2}{V^2} \pdv{V}{p} \Big \vert_{N,T} = \frac{N^2}{V} \chi_T > 0 ~.
        \end{equation*}
    \end{proof}

    Moreover, we have the relation 
    \begin{equation*}
        \chi_T (C_P - C_V) = T V \alpha_p^2 ~,
    \end{equation*}
    which implies that 
    \begin{equation*}
        C_P > C_V \iff \chi_T > \chi_S ~.
    \end{equation*}

    \begin{proof}
        We start from
        \begin{equation*}
            C_V = T \pdv{S}{T} \Big \vert_V = \pdv{E}{T} \Big \vert_V ~,
        \end{equation*}
        \begin{equation*}
            C_p = T \pdv{S}{T} \Big \vert_p = \pdv{E}{T} \Big \vert_p + p \pdv{V}{T} \Big \vert_p  ~,
        \end{equation*}
        which imply that 
        \begin{equation*}
            T dS = C_V dT + ( \pdv{E}{V} \Big \vert_{T} + p) dV = C_V dT + T \pdv{p}{T} \Big \vert_V dV ~,
        \end{equation*}
        \begin{equation*}
            T dS = C_p dT + ( \pdv{E}{p} \Big \vert_T + \pdv{V}{p} \Big \vert_T ) dp = C_p dT - T \pdv{V}{T} \Big \vert_p dp ~.
        \end{equation*}

        Comparing them 
        \begin{equation*}
            (C_p - C_V) dT = T (\pdv{V}{T} \Big \vert_p dp + \pdv{p}{T} \Big \vert_V dV) ~,
        \end{equation*}
        \begin{equation*}
            (C_p - C_V) = T \pdv{V}{T} \Big \vert_p \pdv{p}{T} \Big \vert_V ~.
        \end{equation*}

        We use 
        \begin{equation*}
            \pdv{p}{T} \Big \vert_V = \pdv{(p, V)}{T, V} = \pdv{(p,V)}{(p, T)} \pdv{(p, T)}{(T, V)} = - \pdv{V}{T} \Big \vert_p \pdv{p}{V} \Big \vert_T ~,
        \end{equation*}
        hence 
        \begin{equation*}
            C_p - C_V = \frac{T}{V \chi_T} \Big ( \pdv{V}{T} \Big \vert_p \Big)^2 ~,
        \end{equation*}
        or, defining the thermal expansion coefficient
        \begin{equation}
            \alpha_p = \frac{1}{V} \pdv{V}{T} \Big \vert_p~,
        \end{equation}
        we have 
        \begin{equation*}
            \chi_T (C_P - C_V) = T V \alpha_p^2 ~,
        \end{equation*}

        Finally, we obtain 
        \begin{equation*}
            \frac{C_p}{C_V} = \frac{\chi_T}{\chi_S} ~.
        \end{equation*}
    \end{proof}
    
