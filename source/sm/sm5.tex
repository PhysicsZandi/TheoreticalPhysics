\part{Quantum statistical mechanics}

\chapter{Microcanonical ensemble}

    The microcanonical ensemble is characterised by constant volume, energy and number of particle. Since $N$ is fixed, we can work in the Hilbert space $\mathcal H_{tot}$. Given a time-independent hamiltonian operator $\hat H$, we find the energy eigenbasis $\ket{\psi_j} \in \mathcal H_{tot}$ 
    \begin{equation*}
        \hat H \ket{\psi_j} = E_j \ket{\psi_j} ~.
    \end{equation*}
    However, there could be some degeneracy we want to consider, i.e. $E_{j,\alpha} = E_{j, \beta}$ for $\ket{\psi_{j, \alpha}} \neq \ket{\psi_{j, \beta}}$. Therefore, we have 
    \begin{equation}\label{eneigen}
        \hat H \ket{\psi_{j,\alpha}} = E_j \ket{\psi_{j,\alpha}} ~,
    \end{equation}
    where $\alpha = 1, \ldots n_j$.

    The density operator for mixed states is~\eqref{mix}
    \begin{equation*}
        \rho_{mc} = \sum_{\alpha=1}^{n_j} p_\alpha \ket{\psi_{j, \beta}} \bra{\psi_{j, \beta}} ~,
    \end{equation*}
    where $p_\alpha$ is the probability for the eigenstate $\ket{\psi_{j, \beta}}$. Since $E = E_j$ is fixed, all the eigenstates have the same probability to occur. Therefore $p_\alpha = \frac{1}{n_j}$ and 
    \begin{equation*}
        \rho_{mc} = \frac{1}{n_j} \sum_{\alpha=1}^{n_j} \ket{\psi_{j, \alpha}} \bra{\psi_{j, \alpha}} = \frac{1}{n_j} \hat P_j ~,
    \end{equation*}
    where 
    \begin{equation*}
        P_j = \sum_{\alpha=1}^{n_j} \ket{\psi_{j, \alpha}} \bra{\psi_{j, \alpha}}
    \end{equation*} 
    is the projector onto the energy eigenspace. Notice that we can expand the hamiltonian using~\eqref{spec}
    \begin{equation}\label{endec}
        \hat H = \sum_j E_j \hat P_j ~.
    \end{equation}

    The average of an observable $\hat A$ in the microcanonical ensemble is 
    \begin{equation*}
        \av{A}_{mc} = \frac{1}{n_j} \sum_{\alpha=1}^{n_j} \bra{\psi_{n,\alpha}} \hat A \ket{\psi_{n,\alpha}} ~.
    \end{equation*}
    \begin{proof}
        In fact, choosing an orthonormal basis $\ket{e_j}$, the trace is 
        \begin{equation*}
            \tr_{\mathcal H_{tot}} \hat A = \sum_j \bra{e_j} \hat A \ket{e_j} ~.
        \end{equation*}
        Therefore, using~\eqref{avobs}
        \begin{equation*}
        \begin{aligned}
            \av{A}_{mc} & = \tr_{\mathcal H_{tot}} (\hat A \rho_{mc}) \\ & = \tr_{\mathcal H_{tot}} \Big ( \hat A \frac{1}{n_j} \sum_{\alpha=1}^{n_j} \ket{\psi_{j, \alpha}} \bra{\psi_{j, \alpha}} \Big) \\ & = \frac{1}{n_j} \sum_{\alpha=1}^{n_j} \tr_{\mathcal H_{tot}} \Big ( \hat A \ket{\psi_{j, \alpha}} \bra{\psi_{j, \alpha}} \Big) \\ & = \frac{1}{n_j} \sum_{\alpha=1}^{n_j} \bra{\psi_{j, \alpha}} \hat A \ket{\psi_{j, \alpha}} ~.
        \end{aligned}
        \end{equation*}
    \end{proof}

    The entropy in the microcanonical ensemble is 
    \begin{equation*}
        S_{mc} = k_B \log n_j ~,
    \end{equation*}
    where $n_j$ is the number of states with $E = E_j$. Notice that it is similar to the classical case~\eqref{entropymc}.
    \begin{proof}
        In fact, using~\eqref{unboltz}
        \begin{equation*}
        \begin{aligned}
            S_{mc} = - k_B \av{\log \rho_{mc}}_{mc} = - k_B \tr_{\mathcal H_{tot}} ( \rho_{mc} \log \rho_{mc}) ~.
        \end{aligned}
        \end{equation*}

        In matrix notation, the density operator is 
        \begin{equation*}
        \begin{aligned}
            \rho_{mc} & = \begin{bmatrix}
                \begin{bmatrix}
                    \frac{1}{n_1} & 0 & \ldots & 0 \\
                    0 & \frac{1}{n_1} & \ldots & 0 \\
                    \ldots & \ldots & \ldots & \ldots \\
                    0 & 0 & \ldots & \frac{1}{n_1} \\
                \end{bmatrix} & 0 & \ldots & 0 & \ldots & \ldots \\ 0 & 
                \begin{bmatrix}
                    \frac{1}{n_2} & 0 & \ldots & 0 \\
                    0 & \frac{1}{n_2} & \ldots & 0 \\
                    \ldots & \ldots & \ldots & \ldots \\
                    0 & 0 & \ldots & \frac{1}{n_2} \\
                \end{bmatrix} & \ldots & 0 & \ldots & \ldots \\ 
                \ldots & \ldots & \ldots & \ldots & \ldots & \ldots \\
                0 & 0 & \ldots & \begin{bmatrix}
                    \frac{1}{n_j} & 0 & \ldots & 0 \\
                    0 & \frac{1}{n_j} & \ldots & 0 \\
                    \ldots & \ldots & \ldots & \ldots \\
                    0 & 0 & \ldots & \frac{1}{n_j} \\
                \end{bmatrix} & \ldots & \ldots \\
                \ldots & \ldots & \ldots & \ldots & \ldots & \ldots \\
                \ldots & \ldots & \ldots & \ldots & \ldots & \ldots \\
            \end{bmatrix} \\ & = \sum_j \begin{bmatrix}
                0 & 0 & \ldots & 0 & \ldots & \ldots \\ 
                0 & 0 & \ldots & 0 & \ldots & \ldots \\ 
                \ldots & \ldots & \ldots & \ldots & \ldots & \ldots \\
                0 & 0 & \ldots & \begin{bmatrix}
                    \frac{1}{n_j} & 0 & \ldots & 0 \\
                    0 & \frac{1}{n_j} & \ldots & 0 \\
                    \ldots & \ldots & \ldots & \ldots \\
                    0 & 0 & \ldots & \frac{1}{n_j} \\
                \end{bmatrix} & \ldots & \ldots \\
                \ldots & \ldots & \ldots & \ldots & \ldots & \ldots \\
            \end{bmatrix}
        \end{aligned} ~.
        \end{equation*}

        In order to compute the logarithm of $0$, we use a trick: we define a small parameter $\epsilon$ and we make it go to zero. In this way, the limit becomes $\epsilon \log \epsilon \xrightarrow{\epsilon \rightarrow 0} = 0$. Finally, we compute the trace 
        \begin{equation*}
        \begin{aligned}
            \tr_{\mathcal H_{tot}} ( \rho_{mc} \log \rho_{mc}) & = \tr \begin{bmatrix}
                0 & 0 & \ldots & 0 & \ldots & \ldots \\ 
                0 & 0 & \ldots & 0 & \ldots & \ldots \\ 
                \ldots & \ldots & \ldots & \ldots & \ldots & \ldots \\
                0 & 0 & \ldots & \begin{bmatrix}
                    \frac{1}{n_j} \log \frac{1}{n_j} & 0 & \ldots & 0 \\
                    0 & \frac{1}{n_j} \log \frac{1}{n_j} & \ldots & 0 \\
                    \ldots & \ldots & \ldots & \ldots \\
                    0 & 0 & \ldots & \frac{1}{n_j} \log \frac{1}{n_j} \\
                \end{bmatrix} & \ldots & \ldots \\
                \ldots & \ldots & \ldots & \ldots & \ldots & \ldots \\
            \end{bmatrix} \\ & = \sum_j \frac{1}{n_j} \log \frac{1}{n_j} = n_j \frac{1}{n_j} \log \frac{1}{n_j} = - \log n_j ~.
        \end{aligned}
        \end{equation*}
        Hence, 
        \begin{equation*}
            S_{mc} = - k_B \tr_{\mathcal H_{tot}} ( \rho_{mc} \log \rho_{mc}) = k_B \log n_j ~.
        \end{equation*}
    \end{proof}

    Notice that entropy is always a positive function, since there is at least one state occupied $n_j \geq 1$, which implies $S \geq 0$.

\chapter{Canonical ensemble}

    The canonical ensemble is characterised by constant volume, temperature and number of particle. Energy, which can be exchange in an external reservoir, can be in one of the eigenstates~\eqref{eneigen} with probability 
    \begin{equation}\label{prob}
        p_j \propto \exp(- \beta E_j) ~.
    \end{equation}

    Consider a family of projectors $\{\hat P_j\}$, the density matrix of a mixed states is 
    \begin{equation*}
        \rho_c = \frac{1}{Z_N } \sum_j \exp(- \beta E_j) \hat P_j = \frac{\exp(- \beta \hat H)}{Z_N} ~,
    \end{equation*}
    where the quantum canonical partition function is 
    \begin{equation*}
        Z_N(T,V) = \tr_{\mathcal H_{tot}} \Big ( \frac{\exp(- \beta \hat H)}{Z_N} \Big) ~.
    \end{equation*}
    \begin{proof}
        For a mixed state, the density matrix is~\eqref{mix}
        \begin{equation*}
            \rho_c = \sum_j p_j \hat P_j = C \sum_j \exp(- \beta E_j) \hat P_J ~,
        \end{equation*}
        where the probability is given by~\eqref{prob} and $C$ is a normalisation function.

        Moreover, using~\eqref{endec}
        \begin{equation*}
        \begin{aligned}
            \rho_c & = C \sum_j \exp(- \beta E_j) \hat P_J \\ & = C \sum_j \sum_k \frac{1}{k!} (-\beta E_j)^k \underbrace{\hat P_j}_{(P_j)^k} \\ & = C \sum_j \sum_k \frac{1}{k!} (-\beta E_j \hat P_j)^k \\ & = C \sum_k \frac{1}{k!} (-\beta \sum_j E_j \hat P_j)^k \\ & = C \exp(- \beta \underbrace{\sum_j E_j \hat P_j}_{\hat H}) \\ & = C \exp(- \beta \hat H) ~,
        \end{aligned}
        \end{equation*}
        where we have used the Taylor expansion of the exponential, one of the properties of the projectors~\eqref{idem} and we have exchanged the two series.

        Finally, We set $C = \frac{1}{Z_N}$, where $Z_N$ is the quantum canonical partition function, and by the normalisation condition 
        \begin{equation*}
            1 = \tr_{\mathcal H_{tot}} \rho_c = \frac{1}{Z_N} \tr_{\mathcal H_{tot}} \exp(- \beta \hat H) ~,
        \end{equation*}
        hence 
        \begin{equation*}
            Z_N = \tr_{\mathcal H_{tot}} \exp(- \beta \hat H) ~.
        \end{equation*}
    \end{proof}

    We define the Helmoltz free energy
    \begin{equation*}
        Z_N = \exp(- \beta F) ~,
    \end{equation*}
    or equivalently 
    \begin{equation*}
        F = - \frac{1}{\beta} \log Z_N ~.
    \end{equation*}
    The average energy is 
    \begin{equation*}
        E = \av{\hat H}_c = - \pdv{}{\beta} \log Z_N ~.
    \end{equation*}
    \begin{proof}
        In fact, 
        \begin{equation*}
        \begin{aligned}
            E & = \av{\hat H}_c \\ & = \tr_{\mathcal H_{tot}} (\hat H \rho_c) \\ & = \tr_{\mathcal H_{tot}} \Big ( \hat H \frac{\exp(- \beta \hat H)}{Z_N} \Big ) \\ & = \frac{1}{Z_N} \tr_{\mathcal H_{tot}} \Big (- \pdv{}{\beta} \exp(- \beta \hat H) \Big) \\ & = - \frac{1}{Z_N} \pdv{}{\beta} \underbrace{\tr_{\mathcal H_{tot}} \exp(- \beta \hat H)}_{Z_N} \\ & = - \frac{1}{Z_N} \pdv{}{\beta} Z_N \\ & = - \pdv{}{\beta} \log Z_N ~.
        \end{aligned}
        \end{equation*}
    \end{proof}

    The entropy is 
    \begin{equation*}
        S = \frac{E - F}{T} = \pdv{F}{T} ~.
    \end{equation*}
    \begin{proof}
        In fact, using~\eqref{unboltz}
        \begin{equation*}
        \begin{aligned}
            S_c & = - k_B \av{\log \rho_c}_c \\ & = - k_B \tr_{\mathcal H_{tot}} (\rho_c \log \rho_c) \\ & = - k_B \tr_{\mathcal H_{tot}} (\frac{\exp(- \beta \hat H)}{Z_N} \log \frac{\exp(- \beta \hat H)}{Z_N}) \\ & = - k_B \tr_{\mathcal H_{tot}} \Big (\frac{\exp(- \beta \hat H)}{Z_N} (\log \exp(- \beta \hat H) - \log Z_N) \Big ) \\ & = - k_B \tr_{\mathcal H_{tot}} (\frac{\exp(- \beta \hat H)}{Z_N} (- \beta \hat H - \log Z_N)) \\ & = k_B \beta ~ \underbrace{\tr_{\mathcal H_{tot}} (\frac{\exp(- \beta \hat H)}{Z_N} \hat H )}_E + k_B \tr_{\mathcal H_{tot}} (\frac{\exp(- \beta \hat H)}{Z_N} \underbrace{\log Z_N}_{- \beta F} ) \\ & = \frac{E}{T} - k_B \beta F ~ \frac{1}{Z_N} \underbrace{\tr_{\mathcal H_{tot}} (\exp(- \beta \hat H))}_{Z_N} \\ & = \frac{E-F}{T} ~.
        \end{aligned}
        \end{equation*}
    \end{proof}
    Notice that the entropy is well defined because the trace of the exponential of the energy eigenvalues diverges only if they are negative. Thus, we assume that $E_j \geq \min E_j = 0$.

\chapter{Grancanonical ensemble}

    The grancanonical ensemble is characterised by constant volume, temperature and chemical potential. Since $N$ is not fixed, we work in the full Fock space $\mathcal F_N$. However, we restrict the hamiltonian operator in the Fock space to the condition that it conserves the number of particles, i.e. $[\hat H, \hat N] = 0$ 
    \begin{equation*}
        \hat H \Big \vert_{\mathcal F_N} = \hat H_N ~.
    \end{equation*}
    An example of physical system which does not satisfy this condition is a photons absorbed by an electron. Energy can be in one of the eigenstates, each for a fixed $N$
    \begin{equation*}
        \hat H^{(N)} \ket{\psi_{j, \alpha}^{(N)}} = E_j^{(N)} \ket{\psi_{j, \alpha}^{(N)}} ~,
    \end{equation*}
    with probability 
    \begin{equation}\label{prob2}
        p_j^{(N)} \propto \exp(- \beta (E_j - \mu N)) ~.
    \end{equation}

    Consider a family of projectors $\{\hat P_j^{(N)}\}$
    \begin{equation*}
        \hat P_j^{N} = \sum_\alpha \ket{\psi_{j, \alpha}^{(N)}} \bra{\psi_{j, \alpha}^{(N)}} ~,
    \end{equation*}  
    the density matrix of a mixed states is 
    \begin{equation*}
        \rho_{gc} = \frac{1}{\mathcal Z} \sum_N \sum_j \exp(- \beta (E_j - \mu N)) \hat P_j^{(N)} = \frac{\exp(- \beta (\hat H - \mu \hat N))}{\mathcal Z} ~,
    \end{equation*}
    where $z = \exp(\beta \mu)$ is the fugacity and the quantum grancanonical partition function is 
    \begin{equation*}
        \mathcal Z = \sum_{N=0}^{\infty} \tr_{\mathcal H_{tot}} \Big ( \exp(- \beta (\hat H - \mu \hat N)) \Big) = \sum_{N=0}^\infty z^N Z_N ~.
    \end{equation*}
    \begin{proof}
        For a mixed state, the density matrix is~\eqref{mix}
        \begin{equation*}
            \rho_{gc} = \sum_N \sum_j p_j \hat P_j^{(N)} = C \sum_N \sum_j \exp(- \beta (E_j^{(N)} - \mu N)) \hat P_j^{(N)} ~,
        \end{equation*}
        where the probability is given by~\eqref{prob2} and $C$ is a normalisation function.

        Moreover, using~\eqref{endec} and~\eqref{numb}
        \begin{equation*}
        \begin{aligned}
            \rho_{gc} & = C \sum_N \sum_j \exp(- \beta (E_j - \mu N)) \hat P_j^{(N)} \\ & = C \sum_N \sum_j \sum_k \frac{1}{k!} (-\beta (E_j^{(N)} - \mu N))^k \underbrace{\hat P_j^{(N)}}_{(P_j^{(N)})^k} \\ & = C \sum_j \sum_k \frac{1}{k!} (-\beta (E_j^{(N)} \hat P_j^{(N)} - \nu N P_j^{(N)}))^k \\ & = C \sum_k \frac{1}{k!} (-\beta \sum_N \sum_j (E_j^{(N)} \hat P_j^{(N)} - \mu N P_j^{(N)}))^k \\ & = C \exp(- \beta (\underbrace{\sum_j \sum_N E_j^{(N)} \hat P_j^{(N)}}_{\hat H}) - \mu \underbrace{\sum_j \sum_N N \hat P_j^{(N)}}_{\hat N}) \\ & = C \exp(- \beta (\hat H - \mu \hat N)) ~,
        \end{aligned}
        \end{equation*}
        where we have used the Taylor expansion of the exponential, one of the properties of the projectors~\eqref{idem} and we have exchanged the two series.

        Finally, We set $C = \frac{1}{\mathcal Z}$, where $\mathcal Z$ is the quantum canonical partition function, and by the normalisation condition 
        \begin{equation*}
            1 = \tr_{\mathcal F} \rho_{gc} = \sum_N \frac{1}{\mathcal H_{tot}} \tr_{\mathcal F} \exp(- \beta (\hat H - \mu \hat N)) ~,
        \end{equation*}
        hence 
        \begin{equation*}
            \mathcal Z = \tr_{\mathcal F} \exp(- \beta (\hat H - \mu \hat N)) = \sum_{N=0}^{\infty} \tr_{\mathcal H_{tot}} \exp(- \beta (\hat H - \mu \hat N)) = \sum_{N=0}^{\infty} z^N \underbrace{\tr_{\mathcal H_{tot}} \exp(- \beta \hat H)}_{Z_N} = \sum_N z^N Z_N ~.
        \end{equation*}
    \end{proof}

    Consider an observable $\hat A$ such that it conserves the number of particles, i.e. $[\hat A, \hat N]$, the average value is 
    \begin{equation*}
        \av{\hat A}_{gc} = \tr_{\mathcal F} (\hat A \rho_{gc}) = \frac{1}{\mathcal Z} \sum_{N=0}^{\infty} z^N Z_N \av{\hat A}_c ~.
    \end{equation*}
    \begin{proof}
        In fact, 
        \begin{equation*}
        \begin{aligned}
            \av{\hat A}_{gc} & = \tr_{\mathcal F} (\hat A \rho_{gc}) \\ & = \sum_{N=0}^{\infty} \tr_{\mathcal H_{tot}} \Big (\hat A \frac{z^N \exp(- \beta \hat H)}{\mathcal Z}) = \frac{1}{\mathcal Z} \sum_{N=0}^{\infty} z^N \tr_{\mathcal H_{tot}} (\hat A \exp(- \beta \hat H)) \\ & = \frac{1}{\mathcal Z} \sum_{N=0}^{\infty} z^N Z_N \underbrace{\frac{\tr_{\mathcal H_{tot}} (\hat A \exp(- \beta \hat H))}{Z_N}}_{\av{\hat A}_c} \\ & = \frac{1}{\mathcal Z} \sum_{N=0}^{\infty} z^N Z_N \av{\hat A}_c ~.
        \end{aligned}
        \end{equation*}
    \end{proof}
    
    We define the granpotential 
    \begin{equation*}
        \Omega = - \frac{1}{\beta} \log \mathcal Z ~,
    \end{equation*}
    the energy in the grancanonical is 
    \begin{equation*}
        E - \mu N = \av{\hat H - \mu \hat N} = - \pdv{}{\beta} \log \mathcal Z ~.
    \end{equation*}
    \begin{proof}
        In fact 
        \begin{equation*}
        \begin{aligned}
            E - \mu N & = \av{\hat H - \mu \hat N} \\ & = \tr_{\mathcal F} \Big ( (\hat H - \mu \hat N) \frac{\exp( - \beta (\hat H - \mu \hat N))}{\mathcal Z} \Big) \\ & = - \frac{1}{\mathcal Z} \pdv{}{\beta} \underbrace{\tr_{\mathcal F} (\exp(- \beta (\hat H - \mu \hat N)))}_{\mathcal Z} \\ & = - \frac{1}{\mathcal Z} \pdv{}{\beta} \mathcal Z \\ & = - \pdv{}{\beta} \log \mathcal Z ~.
        \end{aligned}
        \end{equation*}
    \end{proof}

    The entropy in the grancanonical ensemble is 
    \begin{equation*}
        S = \frac{E - \mu N - \Omega}{T} ~.
    \end{equation*}
    \begin{proof}
        In fact 
        \begin{equation*}
        \begin{aligned}
            S & = - k_B \av{\log \rho_{gc}}_{gc} \\ & = - k_B \tr_{\mathcal F} ( \rho_{gc} \log \rho_{gc}) \\ & = - k_B \tr_{\mathcal F} \Big ( \frac{\exp(- \beta (\hat H - \mu \hat N))}{\mathcal Z} \log \frac{\exp(- \beta (\hat H - \mu \hat N))}{\mathcal Z} \Big) \\ & = - k_B \tr_{\mathcal F} \Big ( \frac{\exp(- \beta (\hat H - \mu \hat N))}{\mathcal Z} (\log \exp(- \beta (\hat H - \mu \hat N)) - \log \mathcal Z) \Big) \\ & = k_B \beta \underbrace{\tr_{\mathcal F} \frac{\exp(- \beta (\hat H - \mu \hat N))}{\mathcal Z} (\hat H - \mu \hat N)}_{E - \mu N} + k_B \underbrace{\tr_{\mathcal F} \log \mathcal Z }_{- \beta \Omega} \\ & = \frac{E - \mu N - \Omega}{T} ~.
        \end{aligned}
        \end{equation*}
    \end{proof}

\chapter{Quantum gas}

\section{Generic quantum gas}

    Consider a quantum gas. The hamiltonian operator of one particle, labelled by $k$ is 
    \begin{equation*}
        \hat H_k = \epsilon_k \hat n_k = \epsilon_k \hat a^\dagger_k \hat a_k ~,
    \end{equation*}
    where $\hat n_k = \hat a^\dagger_k \hat a_k$ is the number operator and $\epsilon_k$ is the energy eigenvalue associated to the eigenbasis $\ket{u_k(x)}$ by the eigenvalue relation
    \begin{equation*}
        \hat H_k \ket{u_k(x)} = \epsilon_k \ket{u_k(x)} ~.
    \end{equation*} 
    Therefore, the hamiltonian one-body operator in the Fock space $\mathcal F$, created by the ladder operators $\hat a^\dagger_k$ each associated to the element of the eigenbasis $\ket{u_k(x)}$, is 
    \begin{equation*}
        \hat H = \sum_k \hat H_k =  \sum_k \epsilon_k \hat n_k =  \sum_k \epsilon_k \hat a^\dagger_k \hat a_k ~.
    \end{equation*}
    In $\mathcal F$, the total number onebody operator is 
    \begin{equation*}
        \hat N = \sum_k \hat n_k ~,
    \end{equation*}
    where their eignevalues are given with respect to an orthonormal basis $\ket{n_1, \ldots n_k, \ldots}$ by the eigenvalue relation
    \begin{equation*}
        \hat n_k \ket{n_1, \ldots n_k, \ldots} = n_k \ket{n_1, \ldots n_k, \ldots} ~.
    \end{equation*}
    In particular, we distinguish the bosonic and the fermionic case
    \begin{equation*}
        n_k = \begin{cases}
            0,1,2,\ldots & \textnormal{bosons} \\
            0,1 & \textnormal{fermions} \\
        \end{cases} ~.
    \end{equation*}

    We exploit the grancanonical ensemble. The grancanonical partition function is 
    \begin{equation*}
        \mathcal Z = \tr_{\mathcal F} \exp(- \beta (\hat H - \mu \hat N)) = \prod_k \sum_{n_1, \ldots n_k, \ldots} \exp(-\beta(\epsilon_k - \mu)n_k) ~.
    \end{equation*}
    \begin{proof}
        In fact, 
        \begin{equation*}
        \begin{aligned}
            \mathcal Z & = \tr_{\mathcal F} \exp(- \beta (\hat H - \mu \hat N)) \\ & = \sum_{n_1, \ldots n_k, \ldots} \bra{n_1, \ldots n_k, \ldots} \exp(- \beta \sum_k (\epsilon - \mu) \underbrace{\hat n_k) \ket{n_1, \ldots n_k, \ldots}}_{n_k \ket{n_1, \ldots n_k, \ldots}} \\ & = \sum_{n_1, \ldots n_k, \ldots} \bra{n_1, \ldots n_k, \ldots} \underbrace{\exp(- \beta \sum_k}_{\prod_k \exp} (\epsilon - \mu) n_k) \ket{n_1, \ldots n_k, \ldots} \\ & = \sum_{n_1, \ldots n_k, \ldots} \prod_k \exp(\beta (\epsilon - \mu) n_k) \braket{n_1, \ldots n_k, \ldots}{n_1, \ldots n_k, \ldots} \\ & = \prod_k \sum_{n_1, \ldots n_k, \ldots} \exp(-\beta(\epsilon_k - \mu)n_k) ~,
        \end{aligned}
        \end{equation*}
        where in the last passage, we have switched the product with the sum because $n_1, \ldots, n_k, \ldots$ are independent.
    \end{proof}

    Furthermore, for bosons and fermions, it becomes
    \begin{equation*}
        \mathcal Z = \begin{cases}
            \prod_k \frac{1}{1 - \exp (- \beta (\epsilon_k - \mu))} & \textnormal{bosons} \\
            \prod_k \Big (1 + \exp (- \beta (\epsilon_k - \mu)) \Big ) & \textnormal{fermions} \\
        \end{cases} ~,
    \end{equation*}
    or, in compact notation, 
    \begin{equation*}
        \mathcal Z_\mp = \prod_k \Big ( 1 \mp \exp(- \beta (\epsilon_k - \mu) ) \Big)^\mp ~,
    \end{equation*}
    where the minus is associated to bosons and the plus to fermions.
    \begin{proof}
        For fermions, $n_k = 0, 1$
        \begin{equation*}
            \mathcal Z_+ = \prod_k \sum_{n_1, \ldots n_k, \ldots = 0}^1 \exp(-\beta(\epsilon_k - \mu)n_k) = \prod_k \Big (1 + \exp (- \beta (\epsilon_k - \mu)) \Big ) ~.
        \end{equation*}

        For bosons, $n_k = 0, 1, 2, \ldots$
        \begin{equation*}
        \begin{aligned}
            \mathcal Z_- & = \prod_k \sum_{n_1, \ldots n_k, \ldots = 0}^\infty \exp(-\beta(\epsilon_k - \mu)n_k) \\ & = \prod_k \underbrace{\sum_{n_1, \ldots n_k, \ldots = 0}^\infty \exp(-\beta(\epsilon_k - \mu))^{n_k}}_{\textnormal{geometrical series}} \\ & = \prod_k \frac{1}{1 - \exp (- \beta (\epsilon_k - \mu))} ~.
        \end{aligned}
        \end{equation*}
        Notice that the condition of convergence of the geometrical series is $\mu < \min \epsilon_k = 0$, which we have set to zero for convenience.
    \end{proof}

    The grancanonical potential is 
    \begin{equation*}
        \Omega_\mp = -\frac{1}{\beta} \log \mathcal Z_\mp = \pm \frac{1}{\beta} \sum_k \log \Big (1 \mp \exp (-\beta (\epsilon_k - \mu)) \Big) ~.
    \end{equation*}
    \begin{proof}
        In fact 
        \begin{equation*}
        \begin{aligned}
            \Omega_\mp & = -\frac{1}{\beta} \log \mathcal Z_\mp \\ & = - \frac{1}{\beta} \underbrace{\log \Big (\prod_k}_{\sum_k \log} ( 1 \mp \exp(- \beta (\epsilon_k - \mu) ))^\mp \Big ) \\ & = - (\mp) \sum_k \log \Big (1 \mp \exp (-\beta (\epsilon_k - \mu))) \\ & = \pm \frac{1}{\beta} \sum_k \log \Big (1 \mp \exp (-\beta (\epsilon_k - \mu)) \Big) ~.
        \end{aligned}
        \end{equation*}
    \end{proof}

    The grancanonical average number of particle in an energy level state $\overline k$ is 
    \begin{equation*}
        \av{\hat n_{\overline k}}_{gc} = \tr_{\mathcal F} \Big (\hat n_{\overline k} \frac{\exp (-\beta \sum_k (\epsilon_k - \mu) \hat n_k)}{\mathcal Z} \Big) = \pdv{\Omega}{\epsilon_{\overline k}} = \frac{1}{\exp(\beta(\epsilon_{\overline k} \mp 1))} ~.
    \end{equation*}
    \begin{proof}
        In fact
        \begin{equation*}
        \begin{aligned}
            \av{\hat n_{\overline k}}_{gc} & = \tr_{\mathcal F} \Big (\hat n_{\overline k} \frac{\exp (-\beta \sum_k (\epsilon_k - \mu) \hat n_k)}{\mathcal Z} \Big) \\ & = \frac{1}{\mathcal Z} \tr_{\mathcal F} \Big (- \frac{1}{\beta} \pdv{}{\epsilon_{\overline k}} \exp (-\beta \sum_k (\epsilon_k - \mu) \hat n_k) \Big) \\ & = - \frac{1}{\beta \mathcal Z} \pdv{}{\epsilon_{\overline k}} \underbrace{\tr_{\mathcal F} \Big ( \exp (-\beta \sum_k (\epsilon_k - \mu) \hat n_k) \Big)}_{\mathcal Z} \\ & = - \frac{1}{\beta \mathcal Z} \pdv{}{\epsilon_{\overline k}} \mathcal Z = \\ & = \pdv{}{\epsilon_{\overline k}} \underbrace{\Big ( - \frac{\log \mathcal Z}{\beta} \Big)}_\Omega \\ & = \pdv{}{\epsilon_{\overline k}} \Omega ~.
        \end{aligned}
        \end{equation*}

        Therefore, 
        \begin{equation*}
        \begin{aligned}
            \pdv{}{\epsilon_{\overline k}} \Omega & = \pdv{}{\epsilon_{\overline k}}  \Big (\pm \frac{1}{\beta} \sum_k \log (1 \mp \exp (-\beta (\epsilon_k - \mu)) ) \Big ) \\ & = \pm \frac{1}{\beta} (- \beta) \frac{\exp(- \beta(\epsilon_k - \mu))}{1 \mp \exp(- \beta(\epsilon_k - \mu))} \\ & = \mp \frac{1}{1 \mp \exp(\beta(\epsilon_k - \mu))} \\ & = \frac{1}{\exp(\beta(\epsilon_k - \mu))\mp 1} ~.
        \end{aligned}
        \end{equation*}
    \end{proof}

    The average total number of particle is 
    \begin{equation*}
        N = \av{\hat N}_{gc} = \av{\sum_k \hat n_k}_{gc} = \sum_k \frac{1}{\exp(\beta(\epsilon_k - \mu))\mp 1} ~.
    \end{equation*}

    The average energy is 
    \begin{equation*}
        E = \av{\hat H}_{gc} = \tr_{\mathcal F} \Big (\hat H \frac{\exp (-\beta (\hat H - \mu \hat N))}{\mathcal Z} \Big) = \sum_k \epsilon_k \av{\hat n_k}
    \end{equation*}
    \begin{proof}
        In fact 
        \begin{equation*}
        \begin{aligned}
            E & = \av{\hat H}_{gc} \\ & = \tr_{\mathcal F} \Big (\hat H \frac{\exp (-\beta (\hat H - \mu \hat N))}{\mathcal Z} \Big) \\ & = \frac{1}{\mathcal Z} \tr_{\mathcal F} \Big (- \pdv{}{\beta} \exp (-\beta (\hat H - \mu \hat N)) \Big) \\ & = - \frac{1}{\mathcal Z} \pdv{}{\beta}\underbrace{ \tr_{\mathcal F} \Big (\exp (-\beta (\hat H - \mu \hat N)) \Big)}_{\mathcal Z} \\ & = - \pdv{}{\beta} \Big \vert_z \log \mathcal Z \\ & =  - \pdv{}{\beta} \Big \vert_z (\mp \sum_k \log \Big (1 \mp \exp (-\beta (\epsilon_k - \mu)) \Big)) \\ & = \mp \sum_k \frac{\epsilon_k \exp (-\beta (\epsilon_k - \mu))}{1 \mp \exp (-\beta (\epsilon_k - \mu))} \\ & = \sum_k \frac{\epsilon_k}{\exp (\beta (\epsilon_k - \mu)) \mp 1} \\ & = \sum_k \epsilon_l \av{\hat n_k}
        \end{aligned}
        \end{equation*}
        where we have kept the fugacity $z$ constant.
    \end{proof}

\section{Non-relativistic non-interacting quantum gas}

    So far, we have made computations for a generic quantum gas. From now on, we will deal with non-relativistic non-interacting quantum gas. The finite-volume energy eigenvalues are 
    \begin{equation*}
        \epsilon_k = \frac{\hbar^2 k^2}{2m} ~ \quad \mathbf k = \frac{2\pi}{L} \mathbf n ~,
    \end{equation*}
    where $\mathbf n = (n_1, n_2, n_3) \in \mathbb Z^3$. In the thermodynamic limit, the spectrum $\mathbf k$ becomes continuous, but $\mathbf n$ not, because
    \begin{equation*}
        \Delta K_i = \frac{2\pi}{L} (n_i + 1 - n_i) = \frac{2\pi}{L} ~.
    \end{equation*}

    Therefore, sums in $k$ becomes integrals in $dk$ 
    \begin{equation*}
        \sum_k = \sum_{n_1, n_2, n_2 = -\infty}^\infty \rightarrow \frac{V}{2\pi^2} \int dk~k^2 ~.
    \end{equation*}
    \begin{proof}
        In fact, in $1$-dimensional
        \begin{equation*}
        \begin{aligned}
            \sum_{n_1} \underbrace{\Delta n_1}_1 = \sum_{k_1} \frac{L}{2\pi} \Delta k_1 \rightarrow \frac{L}{2\pi} \int dk_1 ~.
        \end{aligned}
        \end{equation*}

        Similarly, in the $3$-dimensional case
        \begin{equation*}
        \begin{aligned}
            \sum_{n_1, n_2, n_3=- \infty}^\infty \underbrace{\Delta n_1 \Delta n_2 \Delta n_3}_1 \rightarrow \Big ( \frac{L}{2\pi} \Big)^3 \int dk_1 dk_2 dk_3 = \Big ( \frac{L}{2\pi} \Big)^3 \int dk_1 dk_2 dk_3 = \Big ( \frac{L}{2\pi} \Big)^3 \int dk^3 = \Big ( \frac{L}{2\pi} \Big)^3 4 \pi \int dk ~ k^2 = \frac{V}{2\pi^2} \int dk~k^2 ~. 
        \end{aligned}
        \end{equation*}
    \end{proof}

    The grandcanonical potential is 
    \begin{equation*}
        \Omega_\mp = \mp \frac{2}{3}AV \int_0^\infty d \epsilon^{\frac{3}{2}} \frac{1}{\exp(\beta(\epsilon - \mu)) \mp 1}  ~.
    \end{equation*}
    \begin{proof}
        In fact 
        \begin{equation*}
        \begin{aligned}
            \Omega_\mp & = \pm \frac{1}{\beta} \sum_k \log \Big (1 \mp \exp (-\beta (\epsilon_k - \mu)) \Big) \\ & \rightarrow \pm \frac{1}{\beta} \frac{V}{2\pi^2} \int_{-\infty}^\infty dk ~ k^2 \log \Big (1 \mp \exp (-\beta (\epsilon_k - \mu)) \Big) ~.
        \end{aligned}
        \end{equation*}

        Under a change of variable 
        \begin{equation*}
            \epsilon = \frac{\hbar^2 k^2}{2m} ~, \quad k^2 dk = \frac{1}{2} \Big (\frac{2m}{\hbar^2}\Big)^{\frac{3}{2}} \sqrt{\epsilon} d\epsilon ~,
        \end{equation*}
        we obtain 
        \begin{equation*}
        \begin{aligned}
            \Omega_\mp & = \pm \frac{AV}{\beta} \int_0^\infty \underbrace{d\epsilon \sqrt{\epsilon}}_{\frac{2}{3} d \epsilon^{\frac{3}{2}}} \log  (1 \mp \exp (-\beta (\epsilon_k - \mu)) ) \\ & = \pm \frac{2}{3} \frac{AV}{\beta} \int_0^\infty d \epsilon^{\frac{3}{2}} \log  (1 \mp \exp (-\beta (\epsilon_k - \mu)) ) \\ & = \pm \frac{2}{3} \frac{AV}{\beta} \underbrace{\epsilon^{\frac{3}{2}}}_{0 \textnormal{ for } \epsilon = 0} \underbrace{\log  (1 \mp \exp (-\beta (\epsilon_k - \mu)))}_{0 \textnormal{ for } \epsilon = \infty} \Big \vert_0^\infty \mp \frac{2}{3} \frac{AV}{\beta} \beta \int_0^\infty d \epsilon^{\frac{3}{2}} \frac{1}{\exp(\beta(\epsilon - \mu)) \mp 1} \\ & = \mp \frac{2}{3}AV \int_0^\infty d \epsilon^{\frac{3}{2}} \frac{1}{\exp(\beta(\epsilon - \mu)) \mp 1} \\ & = \mp \frac{2}{3} AV \int_0^\infty d \epsilon^{\frac{3}{2}} \frac{1}{\exp(\beta(\epsilon - \mu)) \mp 1} ~.
        \end{aligned}
        \end{equation*}
        where we have integrated by parts and, introducing the degeneracy ($g = 2s+1$ for spin particles), we have called 
        \begin{equation*}
            A = \frac{g}{4\pi^2} \Big (\frac{2m}{\hbar^2}\Big)^{\frac{3}{2}} ~. 
        \end{equation*}
    \end{proof}

    The equation of state reads as 
    \begin{equation*}
        \Omega = - pV = - \frac{2}{3} E ~.
    \end{equation*}

    Furthermore, we have the formulas 
    \begin{equation*}
        N = AV \int_0^\infty d\epsilon \epsilon^{\frac{1}{2}} n(\epsilon) ~, 
    \end{equation*}
    \begin{equation*}
        P = \frac{2}{3} \frac{E}{V} = \frac{2}{3} A \int_0^\infty d\epsilon \epsilon^{\frac{3}{2}} n(\epsilon) ~.
    \end{equation*}

\section{Expanding with respect to fugacity $z$}

    We can expand the density with respect to the fugacity $z = \exp(\beta \mu) \geq 0$
    \begin{equation*}
        n = \frac{g}{\lambda_T^3} f^\mp_{\frac{3}{2}} ~,
    \end{equation*}
    where 
    \begin{equation*}
        f^\mp_l = \begin{cases}
            \sum_{n=0}^\infty \frac{2^{n+1}}{(n+1)^l} & f^- \textnormal{ for bosons} \\
            \sum_{n=0}^\infty \frac{(-1)^n 2^{n+1}}{(n+1)^l} & f^+ \textnormal{ for fermions}
        \end{cases} ~.
    \end{equation*}
    \begin{proof}
        Under a change of variable 
        \begin{equation*}
            x^2 = \beta \epsilon ~, \quad \beta d \epsilon = 2 x dx ~,
        \end{equation*}
        we obtain 
        \begin{equation*}
        \begin{aligned}
            n & = A \int_0^\infty dx ~\frac{2x}{\beta} \frac{x}{\sqrt(\beta) (\exp(x^2) z^{-1}) \mp 1} \\ & = \frac{4 g}{\sqrt{\pi} \lambda_T^3} \int_0^\infty dx ~ \frac{x^2 z}{\exp(x^2) \mp 2} \\ & = \frac{4g}{\sqrt{\pi} \lambda_T^3} \int_0^\infty dx ~ x^2 z \exp(- x^2) \sum_{n=0}^\infty (\pm 1) z^n \exp(- n x^2) \\ & = \frac{4g}{\sqrt{\pi} \lambda_T^3} \sum_{n=0}^\infty (\pm 1)^n z^{n+1} \underbrace{\int_0^\infty dx ~ x^2 \exp(- x^2 (n+1)) }_{\frac{\sqrt{\pi}}{4 (n+1)^{\frac{3}{2}}}} \\ & = \frac{g}{\lambda_T^3} \sum_{n=0}^\infty (\pm 1)^n \frac{z^{n+1}}{(n+1)^{\frac{3}{2}}} \\ & = \frac{g}{\lambda_T^3} f^\mp_{\frac{3}{2}} ~.
        \end{aligned}
        \end{equation*}
    \end{proof}

    Notice that for bosons, the convergence of the series implies $z < 1$, which means $\mu > 0$.

\section{Classical limit}

\section{Semiclassical limit}

\chapter{Fermions}

    In this chapter, we restrict ourselves with the fermionic case. The equations of state are 
    \begin{equation*}
        n = \frac{g}{\lambda_T^3} f_{\frac{3}{2}}^+ (z) ~, \quad \beta p = \frac{g}{\lambda_T^3} f_{\frac{5}{2}}^+ (z) ~,
    \end{equation*}
    where 
    \begin{equation*}
        f_l^+ (z) = \sum_{n=0}^\infty \frac{(-1)^n z^{n+1}}{(n+1)^l}
    \end{equation*}
    which is an alternate-sign power series in $z = \exp(\beta\mu) > 0$, always positive. It absolutely converges for $z < 1$ and pointwisely converges for $z > 1$. Moreover, it is a monotonic function in $z$. 

    It is interesting to study its behaviour for $z \ll 1$. In fact, in the classical limit
    \begin{equation*}
        f_{\frac{3}{2}}(z) \sim f_{\frac{5}{2}}(z) \sim z  ~,
    \end{equation*}
    and in the semiclassical limit 
    \begin{equation*}
        f_{\frac{3}{2}}(z) \sim z - \frac{z^2}{2^{\frac{3}{2}}} ~, \quad f_{\frac{5}{2}}(z) \sim z - \frac{z^2}{2^{\frac{5}{2}}}  ~.
    \end{equation*}

\section{Low temperature limit}

    For the zero temperature limit $T = 0$, the Fermi-Dirac distribution becomes 
    \begin{equation*}
        n(\epsilon) = \frac{1}{\exp(\beta(\epsilon - \mu)) + 1} \xrightarrow{T \rightarrow 0} \begin{cases}
            0 & \epsilon > \mu \\
            \frac{1}{2} & \epsilon = \mu \\
            1 & \epsilon < \mu \\
        \end{cases} ~.
    \end{equation*} 
    It is a step function in $\epsilon = \mu$. This energy value is called Fermi energy $\epsilon_F$. Physically, it means that all the states below this energy level are occupied. Hence, for $\epsilon < \epsilon_F$, we have as many states as particles. If we add a particle, we increase $\epsilon_F$, whereas if we remove a particle, we decrease $\epsilon_F$. This is the procedure to dope a material.

    For small $T$, it is no longer a step function, but it can be accurately approximate to it for a certain range $\Delta \epsilon$. Physically, more energetic particle are transfered over $\epsilon_F$. We define Fermi temperature $T_F$
    \begin{equation*}
        \epsilon_F = \lim_{T \rightarrow 0} \mu (T) = k_B T_F ~.
    \end{equation*}
    In fact, if $\Delta \epsilon \ll \epsilon_F$, which means $T \ll T_F$, we can approximate $n(\epsilon)$ with a step function without making a big error. 

\section{Fermi Energy for a non-relativistic non-interacting quantum gas} 

    In the $3$-dimensional case, the density is
    \begin{equation*}
        n = A \frac{2}{3} \epsilon_F^{\frac{3}{2}} ~.
    \end{equation*}
    \begin{proof}
        In fact, using $n (\epsilon) = \theta (- \epsilon_F)$
        \begin{equation*}
        \begin{aligned}
            n & = A \int_0^\infty d\epsilon \epsilon^{\frac{1}{2}} n(\epsilon) \\ & = A \int_0^{\epsilon_F} d\epsilon \epsilon^{\frac{1}{2}} \\ & = A \frac{2}{3} \epsilon_F^{\frac{3}{2}} ~.
        \end{aligned}
        \end{equation*}
    \end{proof}
    The energy is 
    \begin{equation*}
        E = A V \frac{2}{5} \epsilon_F^{\frac{5}{2}} ~.
    \end{equation*}
    \begin{proof}
        In fact, using $n (\epsilon) = \theta (- \epsilon_F)$
        \begin{equation*}
        \begin{aligned}
            n & = A \int_0^\infty d\epsilon \epsilon^{\frac{3}{2}} n(\epsilon) \\ & = A \int_0^{\epsilon_F} d\epsilon \epsilon^{\frac{3}{2}} \\ & = A \frac{2}{5} \epsilon_F^{\frac{5}{2}} ~.
        \end{aligned}
        \end{equation*}
    \end{proof}

    Notice that at $T = 0$, there is a positive pressure 
    \begin{equation*}
        p = \frac{2}{5} n \epsilon_F > 0~.
    \end{equation*}
    This can be seen visually, because at $T=0$, there are particle with energy $\epsilon \neq 0$, unlikely the classical case, in which $p = 0$.
    \begin{proof}
        In fact 
        \begin{equation*}
            p = \frac{2}{3} \frac{E}{V} = \frac{2}{3} \frac{E}{N} \frac{N}{V} = \frac{2}{5} n \epsilon_F ~. 
        \end{equation*}
    \end{proof}

\chapter{Bosons}

    In this chapter, we restrict ourselves with the bosonic case. The equations of state are 
    \begin{equation*}
        n = \frac{g}{\lambda^3_T} f^-_{\frac{3}{2}} (z) ~, \quad \beta p = \frac{g}{\lambda^3_T} f^-_{\frac{5}{2}} (z)
    \end{equation*}
    where 
    \begin{equation*}
        f_l^- (z) = \sum_{n=0}^\infty \frac{z^{n+1}}{(n+1)^l}
    \end{equation*}
    which is a positive-terms power series in $z = \exp(\beta\mu) > 0$, always positive. It absolutely converges for $z < 1$ and converges for $z > 1$ only if $l < 2$. Moreover, it is a monotonic function in $z$. At $z = 1$, it becomes the Riemann zeta 
    \begin{equation*}
        g_{\frac{3}{2}} (z=1) = \sum_{n=0}^\infty \frac{1}{n+1}^{\frac{3}{2}} = \zeta(\frac{3}{2}) ~.
    \end{equation*}

    Notice that in $z = 1$, it has a vertical derivative, and for $z > 1$, it is not defined according to the physical $\mu > 0$ in the grandcanonical ensemble. 

    We can study the behaviour of the chemical potential $\mu$. It goes to $-\infty$ for $T \rightarrow \infty$ but the equilibrium condition implies that $\pdv{\mu}{T} < 0$, therefore, it cannot increase. 

\section{Low temperature limit}