\part{Quantum statistical mechanics}

\chapter{Quantum mechanics}

    In quantum mechanics, a pure state is described by a normalised vector in a Hilbert space $\ket{\psi} \in \mathcal H$, which is a vector space on $\mathbb C$, i.e.~in which a linear superposition is still in the space $\lambda_1 \ket{\psi_1} + \lambda_2 \ket{\psi_2}$, endowed with a scalar product $\braket{\psi}{\phi}$ through which it is possible to associate a probability and the normalisation condition $||\psi||^2 = \braket{\psi}{\psi} = 1$. Furthermore, the normalisatin condition ensures that a state is not only a vector, but a ray in a Hilbert space, since two states are physically equivalent if $\ket{\psi} \sim \exp(i \phi) \ket{\psi}$. It can be seen as an equivalence class of states.

\section{Projectors}

    It is possible to uniquely determine the state via a projection operator or projector 
    \begin{equation*}
        P_\psi = \frac{\ket{\psi} \bra{\psi}}{\braket{\psi}{\psi}} ~,
    \end{equation*}
    which for normalisation states becomes 
    \begin{equation*}
        P_\psi = \ket{\psi} \bra{\psi} ~.
    \end{equation*}
    \begin{proof}
        If $\ket{\psi} \mapsto \exp(i \phi) \ket{\psi}$ and $\bra{\psi} \mapsto \exp(- i \phi) \bra{\psi}$, we have 
        \begin{equation*}
            P_\psi \mapsto \cancel{\exp(i \phi)} \ket{\psi} \cancel{\exp(- i \phi)} \bra{\psi} = \ket{\psi} \bra{\psi} = P_\psi ~.
        \end{equation*}
    \end{proof}
    It projects onto the $1$-dimensional subspace generated by the state $\ket{\psi}$
    \begin{equation*}
        P_\psi \colon \mathcal H \rightarrow \mathcal H_\psi ~,
    \end{equation*}
    where $\mathcal H_\psi = \{\lambda \ket{\psi} \colon \lambda \in \mathbb C\}$.
    \begin{proof}
        In fact, $\forall \ket{\psi} \in \mathcal H$, we decomposed it into 
        \begin{equation*}
            \ket{\psi} = \alpha \ket{\psi} + \beta \ket{\psi^\perp}
        \end{equation*}
        and the action of the projector
        \begin{equation*}
            P_\psi \ket{\psi} = \alpha \underbrace{P_\psi}_{\ket{\psi} \bra{\psi}} \ket{\psi} + \beta \underbrace{P_\psi \ket{\psi^\perp}}_0 = \alpha \ket{\psi} \underbrace{\braket{\psi}{\psi}}_1 = \alpha \ket{\psi} ~.
        \end{equation*}
    \end{proof}

    The projector is also called density matrix $\rho_\psi$. It satisfies the following properties 
    \begin{enumerate}
        \item boundness, i.e. 
            \begin{equation*}
                ||\rho_\psi|| < C~,
            \end{equation*}
        \item hermiticity, i.e. 
            \begin{equation*}
                \rho_\psi^\dagger = \rho_\psi ~,
            \end{equation*}
        \item idempotence, i.e. 
            \begin{equation*}
                \rho_\psi^2 = \rho_\psi ~,
            \end{equation*}
        \item positive defined, i.e. $\forall \ket{\phi} \in \mathcal H$
            \begin{equation*}
                \bra{\phi} \rho_\psi \ket{\phi} \geq 0 ~,
            \end{equation*}
        \item unit trace, i.e. 
            \begin{equation*}
                \tr \rho_\psi = 1 ~.
            \end{equation*}
    \end{enumerate}
    Actually, there is a theorem which ensures that an operators such that it satifies these 5 conditions is indeed the projector.
    \begin{proof}
        For the hermiticity
        \begin{equation*}
            \rho_\psi^\dagger = (\ket{\psi} \bra{\psi})^\dagger = \bra{\psi}^\dagger \ket{\psi}^\dagger = \ket{\psi} \bra{\psi} = \rho_\psi ~.
        \end{equation*}

        For the idempotence
        \begin{equation*}
            \rho_\psi^2 = (\ket{\psi} \bra{\psi})^2 = \ket{\psi} \underbrace{\braket{\psi}{\psi}}_1 \bra{\psi} = \ket{\psi} \bra{\psi} = \rho_\psi ~.
        \end{equation*}
    \end{proof}

    Given an orthonormal basis $\{\ket{e_n}\}_{n=1}^\infty$ of a separable Hilbert space, the trace is defined as 
    \begin{equation*}
        \tr A = \sum_{n=1}^\infty A_{nn} = \sum_{n=1}^\infty \bra{e_n} A \ket{e_n} ~.
    \end{equation*}
    If it convergent, it is called a trace-class operator. Furthermore, if it is absolute convergent, the trace is independent on the choice of the basis.

\section{Observable}

    An observable is a linear hermitian operator acting on the Hilbert space. We require the self-adjointness because its eigenvalues are real and it always admit an eigenbasis, such that every state can be expanded in this basis 
    \begin{equation*}
        A \ket{\psi_n} = \lambda_n \ket{\psi_n} ~,
    \end{equation*}
    such that 
    \begin{equation*}
        \lambda_n \in \mathbb R
    \end{equation*}
    and $\forall \ket{\phi} \in \mathcal H$
    \begin{equation*}
        \ket{\phi} = \sum_{n=1}^{\infty} c_n \ket{\psi_n} ~.
    \end{equation*}

    The projectors on the eigenstates are orthogonal 
    \begin{equation*}
        P_n P_m = P_n P_m = 0 ~.
    \end{equation*}

    If we prepare the system in a state $\ket{\psi}$, a measurement of an observable $A$ can have outcomes $\lambda_n$ with probability $P_n = |c_n|^2$ where we have defined 
    \begin{equation*}
        \ket{\phi} = \sum_n c_n \ket{\psi_n}~, \quad A \ket{\psi_n} = \lambda_n \ket{\psi_n} ~.
    \end{equation*}
    Its average value is 
    \begin{equation*}
        \av{A} = \sum_{n} \lambda_n P_n = \sum_{n} \lambda_n |c_n|^2 = \tr(A \rho_\psi) ~. 
    \end{equation*}

\section{Composite system}

    For $2$ particles, the total Hilbert space is the tensor product between the single particle Hilbert spaces 
    \begin{equation*}
        \mathcal H_{tot} = \mathcal H_1 \otimes \mathcal H_2 ~. 
    \end{equation*}
    Given an orthonormal basis for each Hilbert space $\{\ket{\psi_n}\} \in \mathcal H_1$ and $\{\ket{\phi_m}\} \in \mathcal H_2$, the total orthonormal basis is 
    \begin{equation*}
        \{\ket{\psi_n}_1 \ket{\phi_m}_2 = \ket{\psi_n \phi_m}\} 
    \end{equation*}
    such that a generic state can be expanded into this basis, $\forall \ket{\phi} \in \mathcal H_{tot}$
    \begin{equation*}
        \ket{\phi} = \sum_n \sum_m a_{nm} \ket{\psi_n \phi_m} ~,
    \end{equation*}
    where the normalisation condition is $\sum_{nm} |a_{nm}|^2 = 1$.

    If the $2$ particle are identical, we have $\mathcal H_1 = \mathcal H_2 = \mathcal H$. Therefore $\mathcal H_{tot} = \mathcal H^{\otimes 2}$.

    The scalar product is 
    \begin{equation*}
        \braket{\psi_n \phi_m}{\psi_{n'} \phi_{m'}} = \braket{\psi_n}{\psi_{n'}} \braket{\phi_m}{\phi_{m'}} ~.
    \end{equation*}

\section{N distinguishable particles}

    A single particle lives in $\mathbb R^3$ and its Hilbert space is $\mathcal H = L^2 (\mathbb R^3) \ni \psi(x)$. The scalar product is 
    \begin{equation*}
        \braket{\psi}{\phi} = \int d^3 x ~ \psi^*(x) \phi(x) ~,
    \end{equation*}
    where the normalisation condition is 
    \begin{equation*}
        ||\psi||^2 = \braket{\psi}{\psi} = \int_{\mathbb R^3} d^3 x ~ |\psi(x)|^2 < infty ~.
    \end{equation*}

    For $N$ distinguishable particles, the total Hilbert space is $\mathcal H_{tot} = \mathcal H \otimes \ldots \otimes \mathcal H$ and an orthonormal basis is $\ket{\psi_{n_1} \ldots \psi_{n_N}}$ where $\ket{\psi_{n_j}}$ is a single particle orthonormal basis. Hence, $N$ distinguishable particle live in $\mathbb R^{3N}$ and their Hilbert space is $\mathcal H_N = L^2(\mathbb R^3) \otimes \ldots \otimes L^2(\mathbb R^3) = L^2 (\mathbb R^{3N}) \ni \psi(x_1, \ldots x_N)$. Therefore, an orthonormal basis is $\{u_{\alpha_1 (x_1)} \ldots u_{\alpha_N (x_N)} = u_{\alpha_1 \ldots \alpha_N} (x_1, \ldots x_N)\}$ where $\{u_\alpha (x)\}$ is the single particle orthonormal basis.

    A generic state can be expanded in this basis as 
    \begin{equation*}
        \psi(x_1, \ldots x_N) = \sum_{\alpha_1 \ldots \alpha_N} c_{\alpha_1 \ldots \alpha_N} u_{\alpha_1 \ldots \alpha_N} (x_1, \ldots x_N) ~.
    \end{equation*}
    For instance, choosing $\alpha_1 = a$ and $\alpha_2 = b$ or viceversa
    \begin{equation*}
        u_{\alpha_1 = a} (x_1) u_{\alpha_2 = b} (x_2) \neq u_{\alpha_1 = b} (x_1) u_{\alpha_2 = a} (x_2) ~.
    \end{equation*}

    If the particle are indistinguishable, they are invariant under permutations
    \begin{equation*}\label{perm}
        \psi(x_1, \ldots x_N) \mapsto \psi(P(x_1, \ldots x_N)) = \exp(i \alpha_P) \psi (x_1, \ldots x_N) ~,
    \end{equation*}
    where $P$ belongs to the permutation group.

\chapter{Permutation group}

    The premutation of $N$ elements form a group $P_N$. This group is generated by transposition $\{\sigma_i\}_{i=1}^N$. In fact, any permutation can be defined by consecutive transposition, where a transposition is defined as 
    \begin{equation*}
        \sigma_i \colon (1,2,\ldots, i, i+1, \ldots N) \mapsto (1,2,\ldots, i+1, i, \ldots N) ~.
    \end{equation*}
    However, this decomposition is not unique but the number of transposition in its decomposition is always even or odd. Therefore, we can define the sign of permutation $\forall P \in P_N$
    \begin{equation*}
        sign(P) = \begin{cases}
            + 1 & \textnormal{even number of transposition in its decomposition } \\
            - 1 & \textnormal{odd number of transposition in its decomposition } \\
        \end{cases} ~.
    \end{equation*}

    Transpositions follow the properties 
    \begin{enumerate}
        \item if $|i - j| > 2$ \begin{equation}\label{prop1}
            \sigma_i \sigma_j = \sigma_j \sigma_i ~,
        \end{equation} 
        \item \begin{equation}\label{prop2}
            \sigma_i \sigma_{i+1} \sigma_i = \sigma_{i+1} \sigma_i \sigma_{i+1} ~,
        \end{equation}
        \item \begin{equation}\label{prop3}
            (\sigma_i)^2 = \mathbb I ~.
        \end{equation}
    \end{enumerate}

    Hence, we can calculate explicitly~\eqref{perm}, which is 
    \begin{equation}
        \alpha_P = \alpha_1 + \ldots \alpha_N~.
    \end{equation}
    \begin{proof}
        In fact
        \begin{equation*}
        \begin{aligned}
        \psi(P(x_1,\ldots x_N)) & = \psi((\sigma_{\alpha_1} \ldots \sigma_{\alpha_N}) (x_1,\ldots x_N)) \\ & = \exp (i \alpha_1) \psi((\sigma_{\alpha_2} \ldots \sigma_{\alpha_N}) ) \\ & ~~ \vdots \\ & = \exp (i \alpha_1) \ldots \exp (i \alpha_N) \psi(x_1,\ldots x_N) \\ & = \exp (i (\alpha_1 + \ldots \alpha_N)) \psi(x_1,\ldots x_N) \\ &  = \exp (i \alpha_P) \psi(x_1,\ldots x_N)~,
        \end{aligned}
        \end{equation*}
        where $P = \sigma_{\alpha_1} \ldots \sigma_{\alpha_N}$.
    \end{proof}

    Furthermore, $\alpha_P = 0, m\pi$, which correspond respectively to a bosonic totally symmetric wavefunction, i.e. under $P$
    \begin{equation*}
        \psi(x_1, \ldots x_N) \xmapsto{P} + 1 \psi(x_1, \ldots x_N) ~,
    \end{equation*}
    or to a fermionic totally antisymmetric wavefunction, i.e. under $P$ 
    \begin{equation*}
        \psi(x_1, \ldots x_N) \xmapsto{P} (-1)^m \psi(x_1, \ldots x_N) = \begin{cases}
            + & \textnormal{sign(P) = +1} \\
            - & \textnormal{sign(P) = -1} \\
        \end{cases}~.
    \end{equation*}

    \begin{proof}
        For~\eqref{prop1}
        \begin{equation*}
            \psi(x_1, \ldots x_N) \xmapsto{\sigma_i} \exp(i \alpha_i) \psi(x_1, \ldots x_N)\xmapsto{\sigma_i \sigma_j} \exp(i \alpha_i) \exp(i \alpha_j) \psi(x_1, \ldots x_N) ~,
        \end{equation*}
        \begin{equation*}
            \psi(x_1, \ldots x_N) \xmapsto{\sigma_j} \exp(i \alpha_j) \psi(x_1, \ldots x_N)\xmapsto{\sigma_j \sigma_i} \exp(i \alpha_j) \exp(i \alpha_i) \psi(x_1, \ldots x_N) ~,
        \end{equation*}
        which means that 
        \begin{equation*}
            \exp(i \alpha_i) \exp(i \alpha_j) = \exp(i \alpha_j) \exp(i \alpha_i) ~.
        \end{equation*}

        For~\eqref{prop2}
        \begin{equation*}
            \psi(x_1, \ldots x_N) \xmapsto{\sigma_i} \exp(i \alpha_i) \psi(x_1, \ldots x_N)\xmapsto{\sigma_i \sigma_{i+1}} \exp(i \alpha_i) \exp(i \alpha_{i+1}) \psi(x_1, \ldots x_N) \xmapsto{\sigma_i \sigma_{i+1} \sigma_i} \exp(i \alpha_i) \exp(i \alpha_{i+1}) \exp(i \alpha_i) \psi(x_1, \ldots x_N) ~,
        \end{equation*}
        \begin{equation*}
            \psi(x_1, \ldots x_N) \xmapsto{\sigma_{i+1}} \exp(i \alpha_{i+1}) \psi(x_1, \ldots x_N)\xmapsto{\sigma_{i+1} \sigma_i} \exp(i \alpha_{i+1}) \exp(i \alpha_i) \psi(x_1, \ldots x_N) \xmapsto{\sigma_{i+1} \sigma_i \sigma_{i+1} } \exp(i \alpha_{i+1}) \exp(i \alpha_i) \exp(i \alpha_{i+1}) \psi(x_1, \ldots x_N) ~,
        \end{equation*}
        which means that 
        \begin{equation*}
            \exp(i \alpha_i) \exp(i \alpha_{i+1}) \exp(i \alpha_i) = \exp(i \alpha_{i+1}) \exp(i \alpha_i) \exp(i \alpha_{i+1}) ~,
        \end{equation*}
        where we have used the fact that $\exp(i \alpha_i) \exp(i \alpha_j)$ commutes. Therefore, $\forall i= 1, \ldots N-1$ and $\alpha_i \in [0, 2\pi[$ we have $\alpha_i = \alpha_{i+1} = \alpha$.

        For~\eqref{prop3}
        \begin{equation*}
            \exp(i \alpha)^2 = \exp (2 i \alpha) = \mathbb I = \exp(0) ~,
        \end{equation*}
        which means that 
        \begin{equation*}
            \alpha = 0, 2\pi ~.
        \end{equation*}

        Finally, there are only two possibilities 
        \begin{equation*}
            \psi(x_1, \ldots x_N) \xmapsto{\sigma_i} \underbrace{\exp(i 0)}_{+1} \psi(x_1, \ldots x_N)
        \end{equation*}
        and 
        \begin{equation*}
            \psi(x_1, \ldots x_N) \xmapsto{\sigma_i} \underbrace{\exp(i \pi)}_{-1} \psi(x_1, \ldots x_N) ~.
        \end{equation*}
    \end{proof}

\section{Symmetric/antisymmetric Hilbert space} 

    Consider $2$ distinguishable particles. In general, the Hilbert space is $\mathcal H_{tot} = \mathcal H \otimes \mathcal H$. If the particle are indistinguishable, we can decomposed the Hilbert space into $\mathcal H_{tot} = \mathcal H_S \otimes_\perp  \mathcal H_A$. In fact, given two states $\ket{a}_1 \in \mathcal H_1$ and $\ket{b}_2 \in \mathcal H_2$, we have 
    \begin{equation*}
    \begin{aligned}
        \ket{a}_1\ket{b}_2 & = \frac{2}{2} \ket{a}_1\ket{b}_2 + \frac{1}{2} \ket{b}_1\ket{a}_2 - \frac{1}{2} \ket{b}_1\ket{a}_2 \\ & = \underbrace{\frac{\ket{a}_1\ket{b}_2 + \ket{b}_1\ket{a}_2}{2}}_{\ket{\psi_S}} + \underbrace{\frac{\ket{a}_1\ket{b}_2 - \ket{b}_1\ket{a}_2}{2}}_{\ket{\psi_A}} \\ & = \ket{\psi_S} + \ket{\psi_A} ~.
    \end{aligned}
    \end{equation*}
    Notice that Pauli's exclusion principle is encoded into the antysymmetric part, because if $a = b$ we have $\ket{\psi_A} = 0$. It is also an orthogonal decomposition. In fact 
    \begin{equation*}
    \begin{aligned}
        \braket{\psi_S}{\psi_A} & = \frac{\bra{a}_1\bra{b}_2 + \bra{b}_1\bra{a}_2}{2} \frac{\ket{a}_1 \ket{b}_2 - \ket{b}_1 \ket{a}_2}{2} \\ & = \frac{1}{4} (\underbrace{\braket{a}{a}_1}_1 \underbrace{\braket{b}{b}_2}_1 - \cancel{\braket{a}{b}_1 \braket{b}{a}_2} + \cancel{\braket{b}{a}_1 \braket{a}{b}_2} - \underbrace{\braket{b}{b}_1}_1 \underbrace{\braket{a}{a}_2}_1 ) = 0 ~. 
    \end{aligned}
    \end{equation*}

    The decomposition is equivalent to define two orthogonal projectors: the symmetriser 
    \begin{equation*}
        \hat S \colon \mathcal H \rightarrow mathcal H_S
    \end{equation*}
    and the antisymmetriser 
    \begin{equation*}
        \hat A \colon \mathcal H \rightarrow mathcal H_A ~,
    \end{equation*}
    such that they satisfy the properties 
    \begin{equation*}
        \hat S^\dagger = \hat S~, \quad \hat A^\dagger = \hat A~, \quad \hat S^2 = \hat S~, \quad \hat A^2 = \hat A~, \quad \hat S \hat A = \hat A \hat S = 0 ~.
    \end{equation*}

    Generalising for $N$ particles, we have $\mathcal H_{tot} = \mathcal H \otimes \ldots \mathcal H$ and a state is $\ket{a_1}_1 \ldots \ket{a_N}_N$ where $\ket{a_j} \in \mathcal H$. The symmetriser is 
    \begin{equation*}
        \hat S \colon \ket{\psi} \mapsto \frac{1}{N!} \sum_{P \in P_N} \ket{a_{P(1)}}_1 \ldots \ket{a_P(N)}_N
    \end{equation*}
    and the antisymmetriser is 
    \begin{equation*}
        \hat A \colon \ket{\psi} \mapsto \frac{1}{N!} \sum_{P \in P_N} sgn(P) \ket{a_{P(1)}}_1 \ldots \ket{a_P(N)}_N
    \end{equation*}
    where $P(1, \ldots N) \mapsto (P(1), \ldots P(N))$. They satisfy the orthogonal projector properties. Notice that for $N > 2$ particles, the total Hilbert space is $\mathcal H_{tot} = \mathcal H_S \otimes \mathcal H_A \otimes \mathcal H'$, where bosons work only in $\mathcal H_S$, fermions work only in $\mathcal H_A$ and $\mathcal H'$ is not physical.

    For distinguishable particles, we have $\mathcal H_{tot} = \mathcal H^{\otimes N}$ with orthonormal basis $\{u_{\alpha_1}(x_1) \ldots u_{\alpha_N}(x_N)\}_{\alpha_1, \ldots \alpha_N=0}^\infty$ labelled by the ordered set $(\alpha_1, \ldots \alpha_N)$. In this case, we are specifying which particle is in which states. However, for indistinguishable particles, we lose information because we know only how many particle are in each state. We label the states with $n_1, \ldots n_j$ with $j=1, \ldots \infty$, which are the occupation number. For bosons, we have $n_k = 0, 1, \ldots, \infty$, whereas for fermions, we have $n_k = 0, 1$. For both cases, there is the constrain $N = \sum_k n_k$, which is an infinite sum but mostly are zero occupied.

\chapter{Second quantisation} 

\section{Bosonic case}

    We define creation and annihilation operators such that they satisfies the properties 
    \begin{equation*}
        [\hat a, \hat a^\dagger]_- = \hat a \hat a^\dagger - \hat a^\dagger \hat a = \mathbb I~.
    \end{equation*}
    Furthermore, the number operator $\hat N = \hat a^\dagger \hat a$ such that 
    \begin{equation*}
        [\hat N, \hat a] = - \hat a~, \quad [\hat N, \hat a^\dagger] = \hat a^\dagger ~.
    \end{equation*} 

    By analogy with the harmonic oscillator, the ground state is the vacuum 
    \begin{equation*}
        \hat a \ket{0} = 0 ~,
    \end{equation*}
    and a generic state is defined by the ladder operators
    \begin{equation*}
        \ket{\psi} = \frac{1}{\sqrt{n!}} (\hat a^\dagger)^N \ket{0} ~.
    \end{equation*}

\section{Fermionic case}

    We define creation and annihilation operators such that they satisfies the properties 
    \begin{equation*}
        [\hat a, \hat a^\dagger]_+ = \hat a \hat a^\dagger + \hat a^\dagger \hat a = \mathbb I~.
    \end{equation*}
    Furthermore, the number operator $\hat N = \hat a^\dagger \hat a$ such that 
    \begin{equation*}
        [\hat N, \hat a] = - \hat a~, \quad [\hat N, \hat a^\dagger] = \hat a^\dagger ~.
    \end{equation*} 

    The properties can be obtained from the Pauli matrices 
    \begin{equation*}
        \sigma_\pm = \sigma_1 \pm i \sigma_2 ~,
    \end{equation*}
    such that 
    \begin{equation*}
        (\sigma_+)^\dagger = \sigma_- ~, \quad (\sigma_-)^\dagger = \sigma_+ ~, \quad (\sigma_+)^2 = (\sigma_-)^2 = 0 ~, \quad [\sigma_-, \sigma_+]_+ = \mathbb I ~.
    \end{equation*}

    By analogy with the harmonic oscillator, the ground state is the vacuum 
    \begin{equation*}
        \hat a \ket{0} = 0 ~,
    \end{equation*}
    and a generic state is defined by the ladder operators
    \begin{equation*}
        \ket{\psi} = \frac{1}{\sqrt{n!}} (\hat a^\dagger)^N \ket{0} ~.
    \end{equation*}

    However, the anticommutator relation ensures the validity of the Pauli's exclusion principle. In fact, we have 
    \begin{equation*}
        a^2 = (\hat a^\dagger)^2 = 0 ~.
    \end{equation*}

\section{Fock space}

    Consider a single particle Hilbert space $\mathcal H$ with an orthonormal basis $\{\ket{e_n}\}_{n=1}^\infty$. To each $\ket{e_n}$, we associate an annihilation and a creation operators 
    \begin{equation*}
        \ket{e_n} \mapsto \{\hat a_n, \hat a_n^\dagger \}_{n=1}^\infty ~,
    \end{equation*}
    such that they satisfy 
    \begin{equation*}
        [\hat a_n, \hat a_m]_\pm = [\hat a_n^\dagger, \hat a_m^\dagger]_\pm = 0 ~, \quad [\hat a_n, \hat a_m^\dagger]_\pm = \delta_{nm} ~,
    \end{equation*}
    where the minus sign correponds to the commutator (bosons) and the plus sign to the anticommutator (fermions). 

    The vacuum state is defined as 
    \begin{equation*}
        \hat a_n \ket{0} = 0 \quad \forall n~.
    \end{equation*}

    For each $\ket{e_n}$, we associate a number operator $\hat n_k = \hat a_k^\dagger \hat a_k$ such that 
    \begin{equation*}
        \hat n_k \hat a_k^\dagger \ket{0} = 1 \hat a_k^\dagger \ket{0} ~, \quad \hat n_{k'} \hat a_k^\dagger \ket{0} = 0 \quad k' \neq k ~.
    \end{equation*}
    For a $n$ particle state, we have 
    \begin{equation*}
        \hat a_k^\dagger \ket{0} = \ket{n_1=0, \ldots n_k=1, \ldots n_N=0} = \ket{e_k} ~.
    \end{equation*}

    However, for 
    \begin{equation*}
        \hat a_{k_1}^\dagger \hat a_{k_2}^\dagger \ket{0} = \ket{e_{k_1}} \ket{e_{k_2}} 
    \end{equation*}
    we have for fermions, if $k_1 = k_2 = k$
    \begin{equation*}
    (\hat a^\dagger_k)^2 \ket{0} = 0 ~,
    \end{equation*}
    whereas for bosons 
    \begin{equation*}
        (\hat a^\dagger_k)^2 \ket{0} \neq 0 ~.
    \end{equation*}
    Furthermore, if $k_1 \neq  k_2$, we have for fermions
    \begin{equation*}
        \hat a^\dagger_{k_1} \hat a^\dagger_{k_2} \ket{0} = - \hat a^\dagger_{k_2} \hat a^\dagger_{k_1} \ket{0} ~,
    \end{equation*}
    whereas for bosons 
    \begin{equation*}
        \hat a^\dagger_{k_1} \hat a^\dagger_{k_2} \ket{0} = \hat a^\dagger_{k_2} \hat a^\dagger_{k_1} \ket{0} ~.
    \end{equation*}

\section{Alternative way}

    There is a $1-1$ correspondence between the orthonormal basis  $\{\ket{e_n}\}_{n=1}^\infty$ of $\mathcal H$ and the orthonormal basis $\{\hat a_k \ket{0}\}_{k=1}^\infty$ of $\mathcal H_{S/A}$. Hence for $N$ particles, we have 
    \begin{equation*}
        \mathcal H_{S/A}^{(N)} = \{\ket{n_1, \ldots n_k, \ldots} = \frac{1}{\sqrt{ \prod_j n_j}} (\hat a_1^\dagger)^{n_1} \ldots (\hat a_k^\dagger)^{n_k} \ldots \ket{0} \} ~.
    \end{equation*} 

    If $N$ is not fixed, like the passage from canonical to grancanonicl ensemble, the total Fock space is 
    \begin{equation*}
        \mathcal F = \bigoplus_{N=0}^\infty \mathcal H^{(N)}_{S/A} ~.
    \end{equation*} 

    It satisfies the following properties 
    \begin{enumerate}
        \item orthonormality, i.e. 
            \begin{equation*}
                \braket{{n'}_1, \ldots {n'}_k, \ldots}{n_1, \ldots n_k, \ldots} = \delta_{{n'}_1, n_1} \ldots \delta_{{n'}_k, n_k} \ldots  ~,
            \end{equation*}
        \item annihilation $\hat a_k \colon \mathcal H^{(N)}_{S/A} \rightarrow \mathcal H^{(N-1)}_{S/A}$, i.e.
            \begin{equation*}
                \hat a_k \ket{n_1, \ldots n_k, \ldots} = \eta_k \sqrt{n_k} \ket{n_1, \ldots (n_k - 1), \ldots} ~,
            \end{equation*}
            where for bosons $\eta_k = 1$ and for fermions $\eta_k = (-1)^{\sum_{j < k} n_j}$,
        \item creation $\hat a_k^\dagger \colon \mathcal H^{(N)}_{S/A} \rightarrow \mathcal H^{(N+1)}_{S/A}$, i.e. for bosons
            \begin{equation*}
                \hat a^\dagger_k \ket{n_1, \ldots n_k, \ldots} = \sqrt{n_k + 1} \ket{n_1, \ldots (n_k + 1), \ldots} ~,
            \end{equation*}
            and for fermions
            \begin{equation*}
                \hat a^\dagger_k \ket{n_1, \ldots n_k, \ldots} = \eta_k \sqrt{1 - n_k} \ket{n_1, \ldots (n_k + 1), \ldots} ~,
            \end{equation*}
        \item number operator $\hat n_k = \hat a_k^\dagger \hat a_k$ such that 
            \begin{equation*}
                \hat n_k \ket{n_1, \ldots n_k, \ldots} = n_k \ket{n_1, \ldots n_k, \ldots}
            \end{equation*}
        and the total number operator $\hat N = \sum_k \hat n_k \sum_k \hat a^\dagger_k \hat a_k$ such that 
        \begin{equation*}
            \hat N \ket{n_1, \ldots n_k, \ldots} = \Big (\sum_k n_k \Big ) \ket{n_1, \ldots n_k, \ldots} ~.
        \end{equation*}
    \end{enumerate}

\section{Field operators} 

    In the first quantisation, we quantise observables to operators, while, in the second quantisation, we quantise fields to operators. Now, a generic particle state is represented by $\ket{f} = \sum_k f_k \ket{e_k} \in \mathcal H$, which is equivalent to $sum_k f_k \hat a_k^\dagger \ket{0}$. Hence, we define the field operators
    \begin{equation*}
        \hat \psi^\dagger (f) = \sum_k f_k \hat a^\dagger_k ~, \quad \hat \psi (f) = \sum_k f_k^* \hat a_k ~,
    \end{equation*}
    in order to get a state $\hat \psi (f) \ket{0}$. The related commutator relations become
    \begin{equation*}
        [\hat \psi (f), \hat \psi^\dagger (g)]_\pm = \braket{f}{g}\mathbb I ~.
    \end{equation*}
    \begin{proof}
        In fact,
        \begin{equation*}
            [\hat \psi (f), \hat \psi^\dagger (g)]_\pm = [\sum_k f^*_k \hat a_k, \sum_m g_m \hat a^\dagger]_\pm = \sum_k \sum_m f^*_k g_m \underbrace{[\hat a_k, \hat a^\dagger_m]}_{\delta_{km} \mathbb I} = \sum_k \sum_m f^*_k g_m \underbrace{\delta_{km}}_{k= m} \mathbb I = \sum_k f^*_k g_k \mathbb I = \braket{f}{g} \mathbb I ~.
        \end{equation*}
        where we have used $\ket{f} \sum_k f_k \ket{e_k}$, $\ket{g} = \sum_m g_m \ket{e_m}$ and $\braket{f}{g} = \sum_k \sum_m f^*_k g_m \underbrace{\braket{e_k}{e_m}}_{\delta_{km}} = \sum_k f^*_k g_k$.
    \end{proof}

    Consider a single particle state in $\mathcal H = L^2(\mathbb R^d) \ni \psi(x)$ with an orthonormal basis $u_k(x)$ such that to each ket there are ladder operators $\hat a_k$ and $\hat a_k^\dagger$. Hence $L^2(\mathbb R^d) \ni f(x) = \sum_k f_k u_k(x)$ and we define field operators
    \begin{equation*}
        \hat \psi(x) = \sum_k u_k^* (x) \hat a_k ~, \quad \hat \psi^\dagger (x) = \sum_k u_k (x) \hat a_k^\dagger ~,
    \end{equation*}
    which is alinear superposition of annihilation and creation operators. Actually, it is called an operator-valued function because its output is an operator. In fact 
    \begin{equation*}
        \int_{\mathbb R^d} d^d x ~ \psi^\dagger (x) \sum_k u_k^* (x) \hat a_k^\dagger = \sum_k \hat a_k^\dagger \int_{\mathbb R^d} d^d x ~ u^*_k(x) f(x) = \sum_k \hat a_k^\dagger f_k~,
    \end{equation*}
    where we have exchanged sum and integral because they are convergent. 

    The commutation relations are 
    \begin{equation*}
        [\psi(x), \psi^\dagger (y)] = \mathbb I \delta (x - y) ~.
    \end{equation*}
    \begin{proof}
        In fact,
        \begin{equation*}
            [\hat \psi (f), \hat \psi^\dagger (g)]_\pm = [\int d^d x ~ f^* (x) \hat \psi(x), \int d^d y ~ g(y) \hat \psi^\dagger (y)]_\pm = \int d^d x \int d^d y ~ f^*(x) g(y) [\psi(x), \psi^\dagger (y)]  ~,
        \end{equation*}
        which must be equal to 
        \begin{equation*}
            \braket{f}{g} = \int d^d x ~ f^* (x) g(x) ~.
        \end{equation*}
        Hence 
        \begin{equation*}
            [\psi(x), \psi^\dagger (y)] = \mathbb I \delta (x - y) ~.
        \end{equation*}
    \end{proof}

    For instance, a plane wave $u(x) = \exp (i \mathbf k \cdot \mathbf x) $ and $\hat \psi(x) = \sum_k \hat a_k^\dagger \exp(i \mathbf k \cdot \mathbf x)$.

    Notice that field operators are basis independent

\section{Operators}

    Consider a Fock space $\mathcal F = \bigoplus_{N=0}^\infty \mathcal H^{(N)}_{B/F}$ with orthonormal basis $\ket{n_1, \ldots n_k, \ldots} = \frac{1}{\sqrt{\prod_j n_j !}} (\hat a^\dagger_1)^{n_1} \ldots (\hat a_k^\dagger)^{n_k} \ldots \ket{0}$, which is in $1-1$ correspondence to the orthonormal basis $\psi_{n_1 \ldots n_k \ldots} (x_1, \ldots x_k, \ldots) = c_N \begin{bmatrix} \hat S \\ \hat A \\ \end{bmatrix} u_{\alpha_1} (x_1) \ldots u_{\alpha_k} (x_k) \ldots$, where $hat S$ is the symmetriser and $\hat A$ is the antisymmetriser.

    We define a one-body operator, associated to a system in which all the particles are the same, as 
    \begin{equation*}
        \hat O^{(1)} = \sum_{j=1}^{N} \hat O(\hat p_j, \hat x_j) ~.
    \end{equation*}
    Since it is self-adjoint, it exists an orthonormal basis of eigenvalues $\{u_\alpha (x)\}$, such that 
    \begin{equation*}
        \hat O(\hat p, \hat x) u_\alpha (x) = \epsilon_\alpha u_\alpha (x) ~.
    \end{equation*}

    Since 
    \begin{equation*}
    \begin{aligned}
        \hat O^{(1)} \psi_{n_1 \ldots n_k \ldots} (x_1, \ldots x_k, \ldots) & = \Big ( \sum_{j=1}^{\infty} \hat O(\hat p_j, \hat x_j) \Big) \psi_{n_1 \ldots n_k \ldots} (x_1, \ldots x_k, \ldots) \\ & = \Big ( \sum_{j=1}^{\infty} \hat O(\hat p_j, \hat x_j) \Big) c_N \begin{bmatrix} \hat S \\ \hat A \\ \end{bmatrix} u_{\alpha_1} (x_1) \ldots u_{\alpha_k} (x_k) \ldots \\ & = c_N \begin{bmatrix} \hat S \\ \hat A \\ \end{bmatrix} \Big ( \sum_{j=1}^{\infty} \hat O(\hat p_j, \hat x_j) u_{\alpha_1} (x_1) \ldots u_{\alpha_k} (x_k) \ldots \Big) \\ & = c_N \begin{bmatrix} \hat S \\ \hat A \\ \end{bmatrix} \Big ( \sum_{j=1}^{\infty}  u_{\alpha_1} (x_1) \ldots \underbrace{\hat O(\hat p_j, \hat x_j) u_{\alpha_j} (x_j)}_{\epsilon_{\alpha_j} u_{\alpha_j} (x_j) } \ldots \Big) \\ & = \Big (\sum_{j=1}^{\infty} \epsilon_j n_j \Big ) \psi_{n_1 \ldots n_k \ldots} (x_1, \ldots x_k, \ldots) ~.
    \end{aligned}
    \end{equation*}

    For the Fock space, we have 
    \begin{equation*}
        \hat O^{(1)}_F = \sum_{j=1}^{\infty} \epsilon_j \hat n_j = \sum_{j=1}^{\infty} \epsilon_j \hat a_j^\dagger \hat a_j ~,
    \end{equation*}
    where 
    \begin{equation*}
        \epsilon_j = \bra{u_j (x)} \hat O (\hat p_j, \hat x_j) \ket{u_j(x)} ~.
    \end{equation*}
    Hence 
    \begin{equation*}
        \hat O^{(1)}_F = \sum_{j=1}^{\infty} \bra{u_j (x)} \hat O (\hat p_j, \hat x_j) \ket{u_j(x)} \hat a_j^\dagger \hat a_j ~.
    \end{equation*}

    Since it is dependent of the basis, because we choose the eigenbasis, we choose a different arbitrary basis 
    \begin{equation*}
        \psi^\dagger (x) = \sum_k u_k (x) \hat a^\dagger_k = \sum_m v_m (x) b^\dagger_m ~,
    \end{equation*}
    and we define the one-body operator
    \begin{equation*}
        \hat O^{(1)}_F = \int d^d x ~ \hat \varphi^\dagger (x) \hat O (\hat p, \hat x) \hat \varphi (x) ~,
    \end{equation*}
    which this time is basis independent.
    \begin{proof}
        In fact 
        \begin{equation*}
        \begin{aligned}
            \int d^d x ~ \hat \varphi^\dagger (x) \hat O (\hat p, \hat x) \hat \varphi (x) & = \int d^d x ~ \Big ( \sum_k u_k(x) \hat a^\dagger (x) \Big ) \hat O (\hat p, \hat x) \Big ( \sum_m u_m^* (x) \hat a_m (x) \Big ) \\ & = \sum_k \sum_m \hat a_k^\dagger \hat a_m \int d^d x ~ u_k (x) \underbrace{\hat O(\hat p, \hat x) u^*_m (x)}_{\epsilon_m u^*_m (x)} \\ & = \sum_k \sum_m \hat a_k^\dagger \hat a_m \epsilon_m \underbrace{\int d^d x ~ u_k (x) u^*_m (x)}_{\delta_{km}} \\ & = \sum_k \sum_m \hat a_k^\dagger \hat a_m \epsilon_m \underbrace{\delta_{km}}_{k = m} \\ & = \sum_k\hat a_k^\dagger \hat a_k \epsilon_k = \hat O^{(1)}_F ~.
        \end{aligned}
        \end{equation*}
    \end{proof}
    It can be written as 
    \begin{equation*}
        \hat O^{(1)}_F = \sum_k \sum_m t_{km} \hat b_k^\dagger \hat h_m ~,
    \end{equation*}
    where the transition amplitude is
    \begin{equation*}
        t_{km} = \bra{v_k} \hat O (\hat p, \hat x) \ket{v_m} ~.
    \end{equation*}



