\chapter{Applications}

    In this chapter, we will study different physical systems. As quantum gases, we will analysea quantum magnetic spins with $s= 1/2$ and $s =1$ and quantum harmonic oscillators. As fermionic gases, we will analyse a relativistic gas at $T=0$ and a white dwarf. As bosonic gases, we will analyse a $2$-dimensional Bose-Einstein condensate and the blackbody radiation.

\section{Quantum magnetic 1/2-spin}

    \begin{exercise}
        Consider a system composed by $N$ distinguishable magnetic dipoles in an external magnetic field along the $z$-axis, with spin $S = 1/2$, with Hamiltonian 
        \begin{equation*}
            \hat H = \sum_i S^{(z)}_i B ~,
        \end{equation*}
        where
        \begin{equation*}
            S^{(z)}_i = \frac{1}{2} \begin{bmatrix}
                1 & 0 \\
                0 & -1 \\
            \end{bmatrix} ~.
        \end{equation*}
        Since spin is the only degree of freedom, the Hilbert space is $\mathbf H = \mathbb C^{N (2S + 1)}$.
        Find the canonical partition function $Z$, the Helmholtz free energy $F$, the energy $E$ and the magnetisation $M$.
    \end{exercise}

    The canonical partition function $Z$ is 
    \begin{equation*}
        Z = \Big ( 2 \cosh \frac{\beta B}{2} \Big)^N ~.
    \end{equation*}
    \begin{proof}
        By definition, for distinguishable particles~\eqref{c:zdist} and~\eqref{qgc:cz},
        \begin{equation*}
        \begin{aligned}
            Z & = (Z_1)^N = \Big (\tr_{\mathcal H} \exp(- \beta \hat H_1) \Big)^N = \Big (\tr_{\mathcal H} \exp(- \beta B \begin{bmatrix}
                \frac{1}{2} & 0 \\ 0 & - \frac{1}{2} \\ 
            \end{bmatrix}) \Big)^N \\ & = \Big(\tr_{\mathcal H} \begin{bmatrix}
                \exp(- \frac{\beta B}{2}) & 0 \\ 0 & \exp(\frac{\beta B}{2}) \\ 
            \end{bmatrix} \Big )^N = \Big( \exp(- \frac{\beta B}{2}) + \exp(\frac{\beta B}{2}) \Big)^N \\ & = \Big ( 2 \cosh \frac{\beta B}{2} \Big)^N ~.
        \end{aligned}
        \end{equation*}
    \end{proof}

    The Helmoltz free energy $F$ is 
    \begin{equation*}
        F = - N k_B T \ln \Big ( 2 \cosh \frac{\beta B}{2} \Big) ~.
    \end{equation*}
    \begin{proof}
        By definition~\eqref{qgc:cf}, we have
        \begin{equation*}
            F = - \frac{\ln Z}{\beta} = - N k_B T \ln \Big ( 2 \cosh \frac{\beta B}{2} \Big) ~.
        \end{equation*}
    \end{proof}

    The internal energy $E$ is 
    \begin{equation*}
        E = - N \frac{B}{2} \tanh \frac{\beta B}{2} ~.
    \end{equation*}
    \begin{proof}
        By definition~\eqref{qgc:ce}, we have 
        \begin{equation*}
            E = - \pdv{\ln Z}{\beta} = - N \pdv{\beta} \ln \Big (2 \cosh \frac{\beta B}{2} \Big) = - N \frac{B}{2} \tanh \frac{\beta B}{2} ~.
        \end{equation*}
    \end{proof}
    A plot of the energy $E$ as a function of $T$ is in Figure~\ref{qm:e}.
    \begin{figure}
        \centering
        \scalebox{0.7}{\pyc{plot1('x', '- ( tanh (1 / x))', 4, 1, 19, True, True, False)}}
        \caption{A plot of the energy $E$ as a function of $T$. We have used $x = \frac{2 k_B T}{B} $ and $f(x) = \frac{2E}{BN}$.}
        \label{qm:e}
    \end{figure}

    The magnetisation $M$ is 
    \begin{equation*}
        M = - \frac{N}{2} \tanh \frac{\beta B}{2} ~. 
    \end{equation*}
    \begin{proof}
        By definition~\eqref{c:mag}, we have
        \begin{equation*}
            M = \pdv{F}{B} = - N k_B T \pdv{}{B} \ln \Big ( 2 \cosh \frac{\beta B}{2} \Big) = - \frac{N}{2} \tanh \frac{\beta B}{2} ~.
        \end{equation*}
    \end{proof}
    A plot of the magnetisation $M$ as a function of $T$ is in Figure~\ref{qm:m}.
    \begin{figure}
        \centering
        \scalebox{0.7}{\pyc{plot1('x', '- ( tanh (1 / x))', 4, 1, 20, True, True, False)}}
        \caption{A plot of the magnetisation $M$ as a function of $T$. We have used $x = \frac{2 k_B T}{B} $ and $f(x) = \frac{2M}{N}$.}
        \label{qm:m}
    \end{figure}

\section{Quantum magnetic 1-spin}

    \begin{exercise}
        Consider a system composed by $N$ distinguishable magnetic dipoles in an external magnetic field along the $z$-axis, with spin $S = 1$, with Hamiltonian 
        \begin{equation*}
            \hat H = \sum_i S^{(z)}_i B ~,
        \end{equation*}
        where
        \begin{equation*}
            S^{(z)}_i = \begin{bmatrix}
                1 & 0 & 0 \\
                0 & 0 & 0 \\
                0 & 0 & - 1 \\
            \end{bmatrix} ~.
        \end{equation*}
        Since spin is the only degree of freedom, the Hilbert space is $\mathbf H = \mathbb C^{N (2S + 1)}$. Find the canonical partition function $Z$, the Helmholtz free energy $F$, the energy $E$ and the magnetisation $M$.
    \end{exercise}

    The canonical partition function $Z$ is 
    \begin{equation*}
        Z = \Big ( 2 \cosh (\beta B) + 1 \Big)^N ~.
    \end{equation*}
    \begin{proof}
        By definition, for distinguishable particles~\eqref{c:zdist} and~\eqref{qgc:cz},
        \begin{equation*}
        \begin{aligned}
            Z & = (Z_1)^N = \Big (\tr_{\mathcal H} \exp(- \beta \hat H_1) \Big)^N = \Big (\tr_{\mathcal H} \exp(- \beta B \begin{bmatrix}
                1 & 0 & 0 \\ 0 & 0 & 0 \\ 0 & 0 & - 1 \\ 
            \end{bmatrix}) \Big)^N \\ & = \Big(\tr_{\mathcal H} \begin{bmatrix}
                \exp(- \beta B) & 0 & 0 \\ 0 & 1 & 0 \\ 0 & 0 & \exp(\beta B) \\ 
            \end{bmatrix} \Big )^N = \Big( \exp(- \beta B) + 1 + \exp(\beta B) \Big)^N \\ & = \Big ( 2 \cosh (\beta B) + 1 \Big)^N ~.
        \end{aligned}
        \end{equation*}
    \end{proof}

    The Helmoltz free energy $F$ is 
    \begin{equation*}
        F = - N k_B T \ln \Big ( 2 \cosh (\beta B) + 1 \Big) ~.
    \end{equation*}
    \begin{proof}
        By definition~\eqref{qgc:cf}, we have
        \begin{equation*}
            F = - \frac{\ln Z}{\beta} = - N k_B T \ln \Big ( 2 \cosh (\beta B) + 1 \Big) ~.
        \end{equation*}
    \end{proof}

    The internal energy $E$ is 
    \begin{equation*}
        E = - 2 N B \frac{\sinh (\beta B)}{2 \cosh (\beta B) + 1} ~.
    \end{equation*}
    \begin{proof}
        By definition~\eqref{qgc:ce}, we have
        \begin{equation*}
            E = - \pdv{\ln Z}{\beta} = - N \pdv{}{\beta} \ln \Big ( 2 \cosh (\beta B) + 1 \Big) = - 2 N B \frac{\sinh (\beta B)}{2 \cosh (\beta B) + 1} ~.
        \end{equation*}
    \end{proof}
    A plot of the energy $E$ as a function of $T$ is in Figure~\ref{qm:e1}.
    \begin{figure}
        \centering
        \scalebox{0.7}{\pyc{plot1('x', '- (( sinh(1 / x) ) / (2 * cosh (1 /x) + 1))', 4, 1, 21, True, True, False)}}
        \caption{A plot of the energy $E$ as a function of $T$. We have used $x = \frac{k_B T}{B} $ and $f(x) = \frac{2E}{BN}$.}
        \label{qm:e1}
    \end{figure}

    The magnetisation $M$ is 
    \begin{equation*}
        M = - \frac{N}{2} \frac{\sinh (\beta B)}{2 \cosh (\beta B) + 1} ~. 
    \end{equation*}
    \begin{proof}
        By definition~\eqref{c:mag}, we have
        \begin{equation*}
            M = \pdv{F}{B} = - N k_B T \pdv{}{B} \ln \Big ( 2 \cosh (\beta B) + 1 \Big) = - 2 N \frac{\sinh (\beta B)}{2 \cosh (\beta B) + 1}  ~.
        \end{equation*}
    \end{proof}
    A plot of the magnetisation $M$ as a function of $T$ is in Figure~\ref{qm:m2}.
    \begin{figure}
        \centering
        \scalebox{0.7}{\pyc{plot1('x', '- (( sinh (1 / x) ) / (2 * cosh (1 / x) + 1))', 4, 1, 22, True, True, False)}}
        \caption{A plot of the magnetisation $M$ as a function of $T$. We have used $x = \frac{2 k_B T}{B} $ and $f(x) = \frac{M}{2N}$.}
        \label{qm:m2}
    \end{figure}

\section{Quantum harmonic oscillators}

    \begin{exercise}
        Consider a system composed by $N$ distinguishable quantum harmonic oscillators, with Hamiltonian 
        \begin{equation*}
            \hat H = \sum_i \hbar \omega (\hat a_i \hat a_i^\dagger + \frac{1}{2}) ~.
        \end{equation*}
        Find the canonical partition function $Z$, the Helmholtz free energy $F$, the energy $E$ and the specific heat per particle $c_V$.
    \end{exercise}

    The canonical partition function $Z$ is 
    \begin{equation*}
        Z = \Big ( \frac{\exp(- \frac{\beta \hbar \omega}{2})}{1 - \exp(- \beta \hbar \omega)} \Big)^N ~.
    \end{equation*}
    \begin{proof}
        By definition, for distinguishable particles~\eqref{c:zdist} and~\eqref{qgc:cz},
        \begin{equation*}
        \begin{aligned}
            Z & = (Z_1)^N = \Big (\tr_{\mathcal H} \exp(- \beta \hat H_1) \Big)^N = \Big (\tr_{\mathcal H_i} \exp(- \beta \hbar \omega (\hat a_i \hat a_i^\dagger + \frac{1}{2})) \Big)^N \\ & = \Big ( \sum_i \bra{n_i}\exp(- \beta \hbar \omega (\hat a_i \hat a_i^\dagger + \frac{1}{2})) \ket{n_i }\Big)^N \\ & = \Big ( \sum_i \exp(- \beta \hbar \omega (n_i + \frac{1}{2})) \Big)^N = \Big ( \exp(- \frac{\beta \hbar \omega}{2}) \sum_i \exp(- \beta \hbar \omega)^{n_i} \Big)^N \\ & = \Big ( \exp(- \frac{\beta \hbar \omega}{2}) \frac{1}{1 - \exp(- \beta \hbar \omega)} \Big)^N = \Big ( \frac{\exp(- \frac{\beta \hbar \omega}{2})}{1 - \exp(- \beta \hbar \omega)} \Big)^N ~.
        \end{aligned}
        \end{equation*}
    \end{proof}

    The Helmoltz free energy $F$ is 
    \begin{equation*}
        F = N k_B T ( \frac{\beta \hbar \omega}{2} + \ln (1 - \exp(- \beta \hbar \omega))) ~.
    \end{equation*}
    \begin{proof}
        By definition~\eqref{qgc:cf}, we have
        \begin{equation*}
        \begin{aligned}
            F & = - \frac{\ln Z}{\beta} = - N k_B T \ln \Big ( \frac{\exp(- \frac{\beta \hbar \omega}{2})}{1 - \exp(- \beta \hbar \omega)} \Big) \\ & = - N k_B T ( \ln \exp(- \frac{\beta \hbar \omega}{2}) - \ln (1 - \exp(- \beta \hbar \omega) )) \\ & = - N k_B T ( - \frac{\beta \hbar \omega}{2} - \ln (1 - \exp(- \beta \hbar \omega))) \\ & = N k_B T ( \frac{\beta \hbar \omega}{2} + \ln (1 - \exp(- \beta \hbar \omega))) ~.
        \end{aligned}
        \end{equation*}
    \end{proof}

    The internal energy $E$ is 
    \begin{equation*}
        E = N ( \frac{\hbar \omega}{2} + \frac{\hbar \omega}{\exp(- \beta \hbar \omega) - 1} ) ~.
    \end{equation*}
    \begin{proof}
        By definition~\eqref{qgc:ce}, we have
        \begin{equation*}
        \begin{aligned}
            E & = - \pdv{\ln Z}{\beta} = - N \pdv{}{\beta} \ln \Big ( \frac{\exp(- \frac{\beta \hbar \omega}{2})}{1 - \exp(- \beta \hbar \omega)} \Big) \\ & = - N \pdv{}{\beta} ( \ln \exp(- \frac{\beta \hbar \omega}{2}) - \ln (1 - \exp(- \beta \hbar \omega) )) \\ & = - N \pdv{}{\beta} ( - \frac{\beta \hbar \omega}{2} - \ln (1 - \exp(- \beta \hbar \omega))) \\ & = N \pdv{}{\beta} ( \frac{\beta \hbar \omega}{2} + \ln (1 - \exp(- \beta \hbar \omega))) \\ & = N ( \frac{\hbar \omega}{2} - \frac{\hbar \omega}{1 - \exp(- \beta \hbar \omega)} ) = N ( \frac{\hbar \omega}{2} + \frac{\hbar \omega}{\exp(- \beta \hbar \omega) - 1} ) ~.
        \end{aligned}
        \end{equation*}
    \end{proof}

    The specific heat is 
    \begin{equation*}
        C_V = N \frac{\hbar^2 \omega^2}{k_B T^2} \frac{\exp(\beta \hbar \omega)}{(\exp(\beta \hbar \omega) - 1)^2} ~.
    \end{equation*}
    \begin{proof}
        In fact, using~\eqref{td:cv2}, we find
        \begin{equation*}
            C_V = \pdv{E}{T} = N \pdv{}{T} ( \frac{\hbar \omega}{2} + \frac{\hbar \omega}{\exp(- \beta \hbar \omega) - 1} ) = N \frac{\hbar^2 \omega^2}{k_B T^2} \frac{\exp(\beta \hbar \omega)}{(\exp(\beta \hbar \omega) - 1)^2} ~.
        \end{equation*}
    \end{proof}

\section{Fermionic relativistic gas at T=0}

    \begin{exercise}
        Consider a quantum relativistic gas of fermions confined in a finite volume $V$ at equilibrium. Since they are relativistic, the energy is $\epsilon = \sqrt{m^2 c^4 + p^2 c^2} - m c^2$. 
        Find the number of particles $N$, the energy $E$, the grand potential $\Omega$, the Fermi energy $\epsilon_F$ and at $T=0$, the energy and the pressure.
    \end{exercise}

    Since we are dealing with momentum, we can integrate with respect to it instead of the energy 
    \begin{equation}\label{tdlim2}
        \sum_k \rightarrow \frac{4 \pi V}{h^3} \int_0^\infty dp ~ p^2 ~.
    \end{equation}  
    \begin{proof}
        In fact, using $p = \hbar k$ and the polar coordinates in momentum space $(p, \theta, \phi)$, we have
        \begin{equation*}
            \sum_k \rightarrow \frac{V}{(2\pi)^3} \int d^3 k = \frac{V}{(2\pi)^3} \hbar^3 \int d^3 p = \frac{V}{(2\pi)^3} 4 \pi \hbar^3 \int_0^\infty dp ~ p^2 = \frac{4 \pi V}{h^3} \int_0^\infty dp ~ p^2 ~.
        \end{equation*}
    \end{proof}

    The number of particles $N$ is 
    \begin{equation*}
        N = \frac{4 \pi V g}{h^3} \int_0^\infty dp ~ \frac{p^2}{\exp(\beta \epsilon(p) - \mu) + 1} ~.
    \end{equation*}
    \begin{proof}
        In fact, using~\eqref{qg:n1} and~\eqref{tdlim2}, we have 
        \begin{equation}
            N = \sum_k \frac{1}{\exp(\beta(\epsilon_k - \mu)) + 1} \rightarrow \frac{4 \pi V g}{h^3} \int_0^\infty dp ~ p^2 \frac{1}{\exp(\beta(\epsilon (p) - \mu)) + 1} ~.
        \end{equation}
    \end{proof}

    The energy $E$ is 
    \begin{equation*}
        E = \frac{4 \pi V g}{h^3} \int_0^\infty dp ~ \frac{p^2 \epsilon(p)}{\exp(\beta \epsilon(p) - \mu) + 1} ~.
    \end{equation*}
    \begin{proof}
        In fact, using~\eqref{qg:e1} and~\eqref{tdlim2}, we have 
        \begin{equation}
            E = \sum_k \frac{\epsilon_k}{\exp(\beta(\epsilon_k - \mu)) + 1} \rightarrow \frac{4 \pi V g}{h^3} \int_0^\infty dp ~ p^2 \frac{\epsilon(p)}{\exp(\beta(\epsilon (p) - \mu)) + 1} ~.
        \end{equation}
    \end{proof}

    The grand potential $\Omega$ is 
    \begin{equation*}
        \Omega = - \frac{4 \pi V g}{3 h^3} \int_0^\infty dp ~ \frac{p^4 c^2}{\exp(\beta \epsilon(p) - \mu) + 1} \frac{1}{\sqrt{m^2 c^4 + p^2 c^2}} ~.
    \end{equation*}
    \begin{proof}
        In fact, using~\eqref{o1} and~\eqref{tdlim2}, we have 
        \begin{equation}
        \begin{aligned}
            \Omega & = - \frac{1}{\beta} \sum_k \ln (1 + \exp(\beta(\epsilon_k - \mu))) \rightarrow - \frac{4 \pi V g}{h^3 \beta} \int_0^\infty dp ~ p^2 \ln (1 + \exp(\beta(\epsilon_k - \mu))) \\ & = - \frac{4 \pi V g}{h^3 \beta} \Big ( \underbrace{ \frac{p^3}{3} }_{0 ~ \text{for} ~ 0} \underbrace{\ln (1 + \exp(\beta(\epsilon_k - \mu)))}_{0 ~ \text{for} ~ \infty}  \Big \vert_0^\infty - \frac{-\beta}{3} \int_0^\infty \int_0^\infty dp ~ \frac{p^3 \epsilon'(p) \exp(- \beta \epsilon(p) - \mu)}{1 + \exp(- \beta \epsilon(p) - \mu)} \Big ) \\ & = - \frac{4 \pi V g}{3 h^3} \int_0^\infty dp ~ \frac{p^3 \epsilon'(p)}{\exp(\beta \epsilon(p) - \mu) + 1} ~.
        \end{aligned}
        \end{equation}
        Now, we evaluate 
        \begin{equation}
            \epsilon'(p) = \dv{\epsilon(p)}{p} = \frac{p c^2}{\sqrt{m^2 c^4 + p^2 c^2}}
        \end{equation}
        and we find 
        \begin{equation}
            \Omega = - \frac{4 \pi V g}{3 h^3} \int_0^\infty dp ~ \frac{p^3 \epsilon'(p)}{\exp(\beta \epsilon(p) - \mu) + 1} = - \frac{4 \pi V g}{3 h^3} \int_0^\infty dp ~ \frac{p^4 c^2}{\exp(\beta \epsilon(p) - \mu) + 1} \frac{1}{\sqrt{m^2 c^4 + p^2 c^2}} ~.
        \end{equation}
    \end{proof}

    The equation of state is 
    \begin{equation}
        pV = - \Omega = \frac{4 \pi V g}{3 h^3} \int_0^\infty dp ~ \frac{p^4 c^2}{\exp(\beta \epsilon(p) - \mu) + 1} \frac{1}{\sqrt{m^2 c^4 + p^2 c^2}} ~.
    \end{equation}

    At $T=0$, we have the Fermi momentum $p_F$ and the Fermi energy $\epsilon_F$
    \begin{equation}
        p_F = \Big ( \frac{3 h^3 n}{4 \pi g} \Big)^{1/3} ~, \quad \epsilon_F = \sqrt{m^2 c^4 + p_F^2 c^2} - mc^2 ~.
    \end{equation}
    \begin{proof}
        At $T=0$, the Fermi-Dirac distribution becomes~\eqref{fdtz} and we have
        \begin{equation}
            n = \frac{4 \pi g}{h^3} \int_0^{p_F} dp ~ p^2 = \frac{4 \pi g}{3 h^3} p_F^3 ~,
        \end{equation}
        which inverted is 
        \begin{equation}
            p_F = \Big ( \frac{3 h^3 n}{4 \pi g} \Big)^{1/3} ~.
        \end{equation}
    \end{proof}

    At $T=0$, the equation of state becomes
    \begin{equation}
        p = \frac{4 \pi V g}{3 h^3} \int_0^{p_F} dp ~ \frac{p^4 c^2}{\sqrt{m^2 c^4 + p^2 c^2}}  ~.
    \end{equation}
    At $T=0$, the energy becomes
    \begin{equation}
        E = \frac{4 \pi V g}{h^3} \int_0^{p_F} dp ~ p^2 \epsilon(p) = \frac{4 \pi V g}{h^3} \int_0^{p_F} dp ~ p^2 (\sqrt{m^2 c^4 + p^2 c^2} - mc^2 ) ~.
    \end{equation}

\section{Fermionic white dwarf}

    A white dwarf is an helium star woth mass $M \sim 10^{30} kg$ and a density of $\rho = 10^{10} kg/m^3$ at a temperature of $10^{7} K$. Our approxiated model is composed by $N$ electrons and $N/2$ helium nuclei.

    Assuming $M = N(m_e + 2 m_p) \sim 2 N m_p$, the electronic dentity is 
    \begin{equation*}
        n = 3 \times 10^{36} m^{-3} ~.
    \end{equation*}
    \begin{proof}
        In fact 
        \begin{equation*}
            n = \frac{N}{V} = \frac{N \rho}{M} = \frac{N \rho}{2 N m_p} = \frac{\rho}{2 m_p} = \frac{10^{10}}{2 \times 1.6 \times 10^{-27}} = 3 \times 10^{36} m^{-3} ~.
        \end{equation*}
    \end{proof}

    The Fermi momentum $p_F$ is 
    \begin{equation*}
        p_F = h \Big ( \frac{3 n}{4 \pi g} \Big)^{1/3} = 6.63 \times 10^{-34}  \times \Big ( \frac{3 \times 10^{10}}{4 \times 3.14 \times 2} \Big)^{1/3} = 0.88 Mev/c ~.
    \end{equation*}
    \begin{proof}
        In fact, using $p = \hbar k$
        \begin{equation*}
            N = g \sum_{n} \rightarrow g \frac{V}{(2\pi)^3} \int d^3 k = g \frac{V}{(2\pi \hbar)^3} \int d^3 p = g \frac{4 \pi V}{(2\pi \hbar)^3} \int_0^{p_F} dp p^2 = g \frac{4 \pi V}{(2\pi \hbar)^3} \frac{p_F^3}{3} ~,
        \end{equation*}
        hence 
        \begin{equation*}
            p_F = h \Big ( \frac{3 n}{4 \pi g} \Big)^{1/3} = 6.63 \times 10^{-34}  \times \Big ( \frac{3 \times 10^{10}}{4 \times 3.14 \times 2} \Big)^{1/3} = 0.88 Mev/c ~.
        \end{equation*}
    \end{proof}

    The Fermi energy $\epsilon_F$ is 
    \begin{equation*}
        \epsilon_F = \sqrt{(p_F c)^2 + (mc^2)^2} - mc^2 = 0.5 Mev ~.
    \end{equation*}

    The Fermi temperature $T_F$ is 
    \begin{equation*}
        T_F = \frac{\epsilon_F}{k_B} = 10^{10} K ~,
    \end{equation*}
    which means that we are in the regime $T \ll T_F$ and we can use $T=0$.

    The internal energy $E$ is 
    \begin{equation*}
        E = \frac{\pi V m^4 c^5}{\pi^2 \hbar^3} f(x_F) ~.
    \end{equation*}
    \begin{proof}
        In fact,
        \begin{equation*}
        \begin{aligned}
            E & = g \sum_{n} \epsilon \rightarrow g \frac{V}{(2\pi)^3} \int d^3 k \epsilon \\ & = g \frac{V}{(2\pi \hbar)^3} \int d^3 p \epsilon \\ & = g \frac{4 \pi V}{(2\pi \hbar)^3} \int_0^{p_F} dp p^2 \epsilon \\ & = g \frac{4 \pi V}{(2\pi \hbar)^3} \int_0^{p_F} dp p^2 c \sqrt{p^2 + (mc)^2} ~.
        \end{aligned}
        \end{equation*}

        Now we make a change of variable 
        \begin{equation*}
            x = \frac{p}{mc} ~, \quad dp = mc dx ~,
        \end{equation*}
        hence 
        \begin{equation*}
        \begin{aligned}
            E & = g \frac{4 \pi V}{(2\pi \hbar)^3} c (mc)^3 \int_0^{x_F} dx ~ x^2 (mc)\sqrt{x^2 + 1} \\ & =  \frac{4 g \pi V m^4 c^5}{h^3} \int_0^{x_F} dx ~ x^2\sqrt{x^2 + 1} \\ & = \frac{4 g \pi V m^4 c^5}{h^3} f(x_F) \\ & = \frac{V m^4 c^5}{\pi^2 \hbar^3} f(x_F) ~,
        \end{aligned}
        \end{equation*}
        where 
        \begin{equation*}
            f(x_F) = \int_0^{x_F} dx ~ x^2 \sqrt{x^2 + 1} ~.
        \end{equation*}
    \end{proof}

    The pressure $P$ is 
    \begin{equation*}
        P = \frac{m^4 c^5}{\pi^2 \hbar^3} \Big (\frac{x_F^3}{3} \sqrt{1 + x_F^2} - f(x_F)) ~.
    \end{equation*}
    \begin{proof}
        In fact,
        \begin{equation*}
        \begin{aligned}
            P & = - \pdv{E}{V} \\ & = - \pdv{}{V} \frac{ V m^4 c^5}{\pi^2 \hbar^3} f(x_F) \\ & = - \frac{\pi m^4 c^5}{\pi^2 \hbar^3} f(x_F) - \frac{ V m^4 c^5}{\pi^2 \hbar^3} \pdv{x_F}{V} \pdv{f(x_F)}{x_F} \\ & = - \frac{ m^4 c^5}{\pi^2 \hbar^3} f(x_F) - \frac{ V m^4 c^5}{\pi^2 \hbar^3} \pdv{}{V} \Big (\frac{h}{mc} \Big ( \frac{3 N}{4 \pi g V} \Big)^{1/3}) \pdv{f(x_F)}{x_F} \\ & = - \frac{m^4 c^5}{\pi^2 \hbar^3} f(x_F) - \frac{V m^4 c^5}{\pi^2 \hbar^3} \Big (\frac{h}{mc} \Big ( \frac{3 N}{4 \pi g V} \Big)^{1/3}) \pdv{}{V} V^{-1/3} \pdv{f(x_F)}{x_F} \\ & = - \frac{m^4 c^5}{\pi^2 \hbar^3} f(x_F) + \frac{1}{3} \frac{V m^4 c^5}{\pi^2 \hbar^3} \Big (\frac{h}{mc} \Big ( \frac{3 N}{4 \pi g V} \Big)^{1/3}) V^{-4/3} \pdv{f(x_F)}{x_F} \\ & = \frac{m^4 c^5}{\pi^2 \hbar^3} \Big (\frac{x_F^3}{3} \sqrt{1 + x_F^2} - f(x_F)) ~.
        \end{aligned}
        \end{equation*}
    \end{proof}

    Now, we solve the integral 
    \begin{equation*}
        f(x) = \py{indint('x**2 * sqrt(x**2 + 1)', 'x')} ~.
    \end{equation*}
    In the non-relativistic limit, $x_F \ll 1$, we can make the approximations 
    \begin{equation*}
        g(x) = \frac{x^3}{3} \sqrt{1 + x^2} = \py{Taylor('x', 'x**3 / 3 * sqrt(1 + x**2)', 0, 6)}
    \end{equation*}
    and 
    \begin{equation*}
        f(x_F) = \py{Taylor('x', 'x**5 / (4 * sqrt(x**2 + 1)) + ( 3 * x**3) / (8 * sqrt(x**2 + 1)) + x / (8 * sqrt(x**2 + 1)) - asinh(x) / 8', 0, 6)} ~.
    \end{equation*}

    In the ultra-relativistic limit, $x_F \gg 1$ or equivalemty $y_F = 1 / x_F \ll 1$, we can make the approximations 
    \begin{equation*}
        g(1/x) = \py{Taylor('x', '(1 / x)**3 / 3 * sqrt(1 + (1 / x)**2)', 0, -1)}
    \end{equation*}
    and 
    \begin{equation*}
        f(1 / x) = \py{Taylor('x', '(1 / x)**5 / (4 * sqrt((1 / x)**2 + 1)) + ( 3 * (1 / x)**3) / (8 * sqrt((1 / x)**2 + 1)) + (1 / x) / (8 * sqrt((1 / x)**2 + 1)) - (ln ( 1 / x + sqrt((1/x)**2 + 1))) / 8', 0, -1)} ~.
    \end{equation*}
    
    Imposing the equilibrium condition $dE = 0$, between the gravitational and the pressure forces, and the structure of a sphere, the pressure must be 
    \begin{equation*}
        P = \frac{\alpha G M^2}{4 \pi R^4}
    \end{equation*}
    and the Fermi momentum is 
    \begin{equation*}
        p_F = \frac{\hbar}{R} \Big ( \frac{9 \pi M}{8 m_p} \Big)^{1/3} ~.
    \end{equation*}
    \begin{proof}
        For the gravitational force 
        \begin{equation*}
            E_g = - \alpha \frac{G M^2}{R} ~, \quad dE_g = \alpha \frac{GM^2}{R^2} dR ~.
        \end{equation*}
        For the pressure force 
        \begin{equation*}
            E_p = - p V = - p \frac{4}{3} \pi R^3 ~, \quad dE_p = - 4 \pi p R^2 dR ~.
        \end{equation*}
        Imposing the equilibrium condition, 
        \begin{equation*}
            0 = dE = dE_g + dE_p = alpha \frac{GM^2}{R^2} dR - 4 \pi p R^2 dR ~,
        \end{equation*}
        hence 
        \begin{equation*}
            p = \frac{\alpha G M^2}{4 \pi R^4} ~.
        \end{equation*}

        The Fermi momentum is 
        \begin{equation*}
            p_F = h \Big ( \frac{3 n}{4 \pi g} \Big)^{1/3} = h \Big ( \frac{3}{8 \pi} \frac{M}{2 m_p \frac{4}{3} \pi R^3} \Big)^{1/3} = \frac{\hbar}{R} \Big ( \frac{9 \pi M}{8 m_p} \Big)^{1/3}  ~.
        \end{equation*}
    \end{proof}

    In the ultra-relativistic limit
    \begin{equation*}
        P = \frac{m^4 c^5}{12 \pi \hbar^3} (x_F^4 - x_F^2) = \frac{\alpha G M^2}{4 \pi R^4} ~.
    \end{equation*}

\section{Bosonic BEC in 2D}

    \begin{exercise}
        Consider a Bose-Einstein condensate of non-relativistic particles in $2$ dimension, with energy $\epsilon = p^2 / 2m$.
        Find the density of states $\omega(E)$, the density $n$ and the critical temperature $T_c$.
    \end{exercise}

    The density of states $\omega(E)$ is 
    \begin{equation*}
        \omega(E) = \frac{A m}{2 \pi \hbar^2} ~.
    \end{equation*}
    \begin{proof}
        By definition~\eqref{cm:vol}, we have
        \begin{equation*}
        \begin{aligned}
            \Sigma (E) = \int_{H \leq \epsilon} d\Omega = \int_{H \leq \epsilon} \frac{d^2 x ~ d^2 p}{h^2} ~.
        \end{aligned}
        \end{equation*}
        In order to find the domain of integration, we compute the following inequality
        \begin{equation*}
            H = \frac{p^2}{2m} \leq \epsilon ~, \quad p^2 \leq 2 m \epsilon ~.
        \end{equation*}
        Therefore, using polar coordinates in momentum space $(p, \theta)$ and adding the degeneracy, we find
        \begin{equation*}
            \Sigma (E) = \underbrace{\int_A d^2 x}_A \int_{p^2 < 2mE} d^2 p = \frac{2 \pi A}{h^2} \int_0^{\sqrt{2mE}} dp p = \frac{A m}{2 \pi \hbar^2} \epsilon ~,
        \end{equation*}
        Finally, by definition~\eqref{cm:denst}, we have
        \begin{equation*}
            \omega (E) = \pdv{\Sigma (epsilon)}{epsilon} = \pdv{}{\epsilon} \frac{A m}{2 \pi \hbar^2} \epsilon = \frac{A m}{2 \pi \hbar^2} ~.
        \end{equation*}
    \end{proof}

    The number of particles $N$ is 
    \begin{equation*}
        N = -\frac{A m}{2 \pi \hbar^2 \beta} \ln z ~.
    \end{equation*}
    \begin{proof}
        By definition~\eqref{q:n}, we have 
        \begin{equation}
            N = \int_0^\infty d\epsilon ~ \omega(\epsilon) \frac{1}{\exp(\beta \epsilon) - 1} = \frac{A m}{2 \pi \hbar^2} \int_0^\infty d\epsilon ~ \frac{1}{\exp(\beta (\epsilon - \mu)) - 1} ~.
        \end{equation}
        Now, we make a change of variable into 
        \begin{equation}
            t = \exp(\beta (\epsilon - \mu)) ~, \quad d\epsilon = \frac{dt}{\beta t} 
        \end{equation}
        hence, we find 
        \begin{equation*}
        \begin{aligned}
            N & = \frac{A m}{2 \pi \hbar^2 \beta} \int_{\exp(-\beta \mu)}^\infty dt ~ \frac{1}{t (t+1)} = \frac{A m}{2 \pi \hbar^2 \beta} \int_{\exp(-\beta \mu)}^\infty dt ~ (\frac{1}{t} - \frac{1}{t+1} ) \\ & = \frac{A m}{2 \pi \hbar^2 \beta} (\ln t - \ln (1+t) \Big \vert_{\exp(-\beta \mu)}^{\cancel \infty} ) = \frac{A m}{2 \pi \hbar^2 \beta} (- \ln z) = -\frac{A m}{2 \pi \hbar^2 \beta} \ln z ~.
        \end{aligned}
    \end{equation*}
    \end{proof}

    The critical temperature $T_c$ is 
    \begin{equation*}
        T_c = 0 ~,
    \end{equation*}
    \begin{proof}
        In fact, putting $z = 1$, we have $T=0$ in order to keep $n$ fixed.
    \end{proof}

\section{Bosonic BEC of harmonic oscillators in 2D}

    \begin{exercise}
        Consider a Bose-Einstein condensate of harmonic oscillators in $2$ dimension, with energy $\epsilon = p^2 / 2m + m \omega^2 x^2 / 2$.
        Find the density of states $\omega(E)$, the density $n$ and the critical temperature $T_c$.
    \end{exercise}

    The density of states $\omega(E)$ is 
    \begin{equation*}
        \omega(E) = \Big (\frac{1}{\hbar \omega} \Big )^{2} \epsilon ~.
    \end{equation*}
    \begin{proof}
        Using~\eqref{harmosc} with $d = 2$ and $N = 1$, we have
        \begin{equation}
            \omega(E) = \frac{1}{\Gamma (2)} \Big (\frac{1}{\hbar \omega} \Big )^{2} \epsilon = \Big (\frac{1}{\hbar \omega} \Big )^{2} \epsilon ~.
        \end{equation}
    \end{proof}

    The number of particles $N$ is 
    \begin{equation*}
        N = \Big (\frac{1}{\hbar \omega \beta} \Big )^{2} \int_0^\infty d x ~ \frac{x}{z^{-1} \exp (x) - 1} ~.
    \end{equation*}
    \begin{proof}
        By definition~\eqref{q:n}, we have 
        \begin{equation}
            N = \int_0^\infty d\epsilon ~ \omega(\epsilon) \frac{1}{\exp(\beta \epsilon) - 1} = \Big (\frac{1}{\hbar \omega} \Big )^{2} \int_0^\infty d\epsilon ~ \frac{\epsilon}{\exp(\beta (\epsilon - \mu)) - 1} ~.
        \end{equation}
        Now, we make a change of variable into 
        \begin{equation}
            x = \beta \epsilon ~, 
        \end{equation}
        hence, we find, using the 
        \begin{equation*}
            N = \Big (\frac{1}{\hbar \omega \beta} \Big )^{2} \int_0^\infty d x ~ \frac{x}{z^{-1} \exp (x) - 1} ~.
        \end{equation*}
    \end{proof}

    The critical temperature $T_c$ is 
    \begin{equation*}
            T_c = \frac{\hbar \omega}{k_B T} (6N)^{1/2} ~.
    \end{equation*}
    \begin{proof}
        In fact, putting $z = 1$, we have 
        \begin{equation}
            N = \Big (\frac{1}{\hbar \omega \beta} \Big )^{2} \underbrace{\int_0^\infty d x ~ \frac{x}{\exp (x) - 1} }_{\zeta(2) = \pi^2 / 6} = \frac{1}{6} \Big (\frac{\pi}{\hbar \omega \beta} \Big )^{2} ~,
        \end{equation}
        which inverted is 
        \begin{equation}
            T_c = \frac{\hbar \omega}{k_B T} (6N)^{1/2} ~.
        \end{equation}
    \end{proof}

\section{Bosonic blackbody radiation}

    \begin{exercise}
        Consider a gas of photons confined in a finite volume $V$ at equilibrium. Since they are ultra-relativistic, the energy is $\epsilon = c |\mathbf p|$. 
        Find the density of states $\omega(E)$, the grand potential $\omega$, the number of particles $N$, the energy $E$, the equation of state and the specific heat $C_V$. Furthermore, find the Planck's distribution, Wien's and Rayleigh-Jeans' laws.
    \end{exercise}

    There are only two indepeendent polarisation degrees of freedome, therefore $g=2$. Since they can be absorbed and emitted, the number of particle is not conserved and $\mu (T) = 0$. 

    The density of states $\omega(E)$ is 
    \begin{equation*}
        \omega(E) = \frac{V \epsilon^2}{\hbar^3 \pi^2 c^3} ~.
    \end{equation*}
    \begin{proof}
        By definition~\eqref{cm:vol}, we have
        \begin{equation*}
        \begin{aligned}
            \Sigma (E) = \int_{H \leq \epsilon} d\Omega = \int_{H \leq \epsilon} \frac{d^3 x ~ d^3 p}{h^3} ~.
        \end{aligned}
        \end{equation*}
        In order to find the domain of integration, we compute the following inequality
        \begin{equation*}
            H = c p \leq \epsilon ~, \quad p \leq \frac{\epsilon}{c} ~.
        \end{equation*}
        Therefore, using polar coordinates in momentum space $(p, \theta, \phi)$ and adding the degeneracy, we find
        \begin{equation*}
        \begin{aligned}
            \Sigma (E) & = \frac{g}{h^3} \underbrace{\int_V d^3 x}_V \int_{p < E/c} d^3 p = \frac{8 \pi V}{h^3} \int_0^{\epsilon/c} dp ~p^2 = \frac{V}{\hbar^3 \pi^2} \frac{p^3}{3} \Big \vert_0^{\epsilon/c} = \frac{V \epsilon^3}{3 \hbar^3 \pi^2 c^3}~,
        \end{aligned}
        \end{equation*}
        Finally, by definition~\eqref{cm:denst}, we have
        \begin{equation*}
            \omega (E) = \pdv{\Sigma (epsilon)}{epsilon} = \pdv{}{\epsilon} \frac{V \epsilon^3}{3 \hbar^3 \pi^2 c^3} = \frac{V \epsilon^2}{\hbar^3 \pi^2 c^3} ~.
        \end{equation*}
    \end{proof}

    In the following, we define 
    \begin{equation}
        \zeta (n+1) = \int_0^\infty dx ~ \frac{x^n}{\exp(x) - 1} ~.
    \end{equation}

    The grand potential $\Omega$ is 
    \begin{equation*}
        \Omega = \frac{V}{3 \hbar^3 \pi^2 c^3 \beta^4} \zeta(4) 
    \end{equation*}
    \begin{proof}
        By definition~\eqref{q:o}, we have 
        \begin{equation}
        \begin{aligned}
            \Omega & = \frac{1}{\beta} \int_0^\infty d\epsilon ~ \omega(\epsilon) \ln(1 - \exp(- \beta \epsilon)) = \frac{V}{\hbar^3 \pi^2 c^3 \beta} \int_0^\infty d\epsilon ~ \epsilon^2 \ln(1 - \exp(- \beta \epsilon)) \\ & = \frac{V}{\hbar^3 \pi^2 c^3 \beta} \Big ( \underbrace{\frac{\epsilon^3}{3} }_{0 ~ \text{for} ~ 0}\underbrace{\ln(1 - \exp(- \beta \epsilon)) }_{0 ~ \text{for} ~ \infty} \Big \vert_0^\infty - \frac{\beta}{3} \int_0^\infty d\epsilon ~ \frac{\epsilon^3 \exp(- \beta \epsilon)}{1 - \exp(- \beta \epsilon)} \Big) \\ & = \frac{V}{3 \hbar^3 \pi^2 c^3} \int_0^\infty d\epsilon ~ \frac{\epsilon^3}{\exp(\beta \epsilon) - 1} ~.
        \end{aligned}
        \end{equation}
        Now, we make a change of variable into 
        \begin{equation}
            x = \beta \epsilon ~,
        \end{equation}
        hence, we find 
        \begin{equation*}
        \Omega = \frac{V}{3 \hbar^3 \pi^2 c^3 \beta^4} \underbrace{\int_0^\infty dx ~ \frac{x^3}{\exp(x) - 1}}_{\zeta(4)} = \frac{V}{3 \hbar^3 \pi^2 c^3 \beta^4} \zeta(4) ~.
        \end{equation*}
    \end{proof}

    The number of particles $N$ is 
    \begin{equation*}
        N = \frac{V}{\hbar^3 \pi^2 c^3 \beta^3} \zeta(3) ~.
    \end{equation*}
    \begin{proof}
        By definition~\eqref{q:n}, we have 
        \begin{equation}
            N = \int_0^\infty d\epsilon ~ \omega(\epsilon) \frac{1}{\exp(\beta \epsilon) - 1} = \frac{V}{\hbar^3 \pi^2 c^3} \int_0^\infty d\epsilon ~ \frac{\epsilon^2}{\exp(\beta \epsilon) - 1} ~.
        \end{equation}
        Now, we make a change of variable into 
        \begin{equation}
            x = \beta \epsilon ~,
        \end{equation}
        hence, we find 
        \begin{equation*}
            N = \frac{V}{\hbar^3 \pi^2 c^3 \beta^3} \underbrace{\int_0^\infty dx ~ \frac{x^2}{\exp(x) - 1}}_{\zeta(3)} = \frac{V}{\hbar^3 \pi^2 c^3 \beta^3} \zeta(3) ~.
        \end{equation*}
    \end{proof}
    The energy $E$ is 
    \begin{equation*}
        E = \frac{V}{\hbar^3 \pi^2 c^3 \beta^4} \zeta(4) ~.
    \end{equation*}
    \begin{proof}
        By definition~\eqref{q:n}, we have 
        \begin{equation}
            E = \int_0^\infty d\epsilon ~ \omega(\epsilon) \frac{\epsilon}{\exp(\beta \epsilon) - 1} = \frac{V}{\hbar^3 \pi^2 c^3} \int_0^\infty d\epsilon ~ \frac{\epsilon^3}{\exp(\beta \epsilon) - 1} ~.
        \end{equation}
        Now, we make a change of variable into 
        \begin{equation}
            x = \beta \epsilon ~,
        \end{equation}
        hence, we find 
        \begin{equation*}
            N = \frac{V}{\hbar^3 \pi^2 c^3 \beta^4} \underbrace{\int_0^\infty dx ~ \frac{x^3}{\exp(x) - 1}}_{\zeta(4)} = \frac{V}{\hbar^3 \pi^2 c^3 \beta^4} \zeta(4) ~.
        \end{equation*}
    \end{proof}
    The equation of state is 
    \begin{equation}
        pV = - \Omega = \frac{V}{3 \hbar^3 \pi^2 c^3 \beta^4} \zeta(4) = \frac{E}{3} ~.
    \end{equation}
    The specific heat $C_V$ is 
    \begin{equation}
        C_V = \frac{4 k_B^4 V}{\hbar^3 \pi^2 c^3} \zeta(4) T^3 ~.
    \end{equation}
    \begin{proof}
        In fact, using~\eqref{td:cv2}, we have 
        \begin{equation}
            C_V = \pdv{E}{T} = \pdv{}{T} \frac{k_B^4 T^4 V}{\hbar^3 \pi^2 c^3} \zeta(4) = \frac{4 k_B^4 V}{\hbar^3 \pi^2 c^3} \zeta(4) T^3 ~.
        \end{equation}
    \end{proof}

    The number spectral distribution is 
    \begin{equation}
        \frac{8 \pi V}{c^3} \frac{\nu^2}{\exp(\beta h \nu) - 1} ~.
    \end{equation}
    \begin{proof}
        We introduce the number of photons with an energy of $\epsilon \in [0, h\nu]$
        \begin{equation}
            N(\nu) = \frac{V}{\hbar^3 \pi^2 c^3} \int_0^{h\nu} d\epsilon ~ \frac{\epsilon^2}{\exp(\beta \epsilon) - 1} = \frac{8 \pi V}{h^3 c^3} \int_0^{h\nu} d\epsilon ~ \frac{\epsilon^2}{\exp(\beta \epsilon) - 1} ~.
        \end{equation}
        Now, we make a change of variable into 
        \begin{equation}
            \epsilon = h \nu ~,
        \end{equation}
        hence, we find 
        \begin{equation*}
            N = \frac{8 \pi V}{h^3 c^3} h^3 \int_0^{1} d\nu ~ \frac{\nu^2}{\exp(\beta h \nu) - 1} ~.
        \end{equation*}
        We define the spectral distribution 
        \begin{equation}
            f(\nu) = \dv{N(\nu)}{\nu} = \frac{8 \pi V}{c^3} \frac{\nu^2}{\exp(\beta h \nu) - 1} ~.
        \end{equation}
    \end{proof}

    The energy spectral distribution is 
    \begin{equation}
        u(\nu) = \frac{8 \pi h}{c^3} \frac{\nu^3}{\exp(\beta h \nu) - 1} ~.
    \end{equation}
    A plot of the Planck's distribution $u(\nu)$ as a function of $\nu$ is in Figure~\ref{br:p}.
    \begin{figure}
        \centering
        \scalebox{0.7}{\pyc{plot1('x', 'x**3 / (exp(x) - 1)', 10, 2, 23, True, True, True)}}
        \caption{A plot of the Planck's distribution $u(\nu)$ as a function of $\nu$. We have used $x = \beta h \nu$ and $f(x) = \frac{u c^3}{8 \pi h^4 \beta^3}$.}
        \label{br:p}
    \end{figure}
    \begin{proof}
        We introduce the energy density of photons with an energy of $\epsilon \in [0, h\nu]$
        \begin{equation}
            U(\nu) = \frac{E}{V} = \frac{1}{\hbar^3 \pi^2 c^3} \int_0^{h \nu} d\epsilon ~ \frac{\epsilon^3}{\exp(\beta \epsilon) - 1} = \frac{8 \pi}{h^3 c^3} \int_0^{h\nu} d\epsilon ~ \frac{\epsilon^3}{\exp(\beta \epsilon) - 1} ~.
        \end{equation}
        Now, we make a change of variable into 
        \begin{equation}
            \epsilon = h \nu ~,
        \end{equation}
        hence, we find 
        \begin{equation*}
            U = \frac{8 \pi}{h^3 c^3} h^4 \int_0^{1} d\nu ~ \frac{\nu^3}{\exp(\beta h \nu) - 1} ~.
        \end{equation*}
        We define the energy spectral distribution 
        \begin{equation}
            u(\nu) = \dv{U(\nu)}{\nu} = \frac{8 \pi h}{c^3} \frac{\nu^3}{\exp(\beta h \nu) - 1} ~.
        \end{equation}
    \end{proof}

    Its approximation of $\beta h \nu \gg 1$ is 
    \begin{equation}
        u(\nu) \simeq \frac{8 \pi h}{c^3} \frac{\nu^3}{\exp(\beta h \nu)} ~,
    \end{equation}
    where we have only neglected the minus at the denominator. A plot of the Wien's law $u(\nu)$ as a function of $\nu$ is in Figure~\ref{br:w}.
    \begin{figure}
        \centering
        \scalebox{0.7}{\pyc{plot1('x', 'x**3 / (exp(x))', 10, 2, 24, True, True, True)}}
        \caption{A plot of the Wien's law $u(\nu)$ as a function of $\nu$. We have used $x = \beta h \nu$ and $f(x) = \frac{u c^3}{8 \pi h^4 \beta^3}$.}
        \label{br:w}
    \end{figure}

    Its approximation of $\beta h \nu \ll 1$ is 
    \begin{equation}
        u(\nu) \simeq \frac{8 \pi}{c^3} \frac{\nu^2}{\beta} ~.
    \end{equation}
    where we have only expanded at first order the exponential. A plot of the Rayleigh-Jeans's law $u(\nu)$ as a function of $\nu$ is in Figure~\ref{br:rj}.
    \begin{figure}
        \centering
        \scalebox{0.7}{\pyc{plot1('x', 'x**2', 10, 2, 25, True, True, True)}}
        \caption{A plot of the Rayleigh-Jeans's law $u(\nu)$ as a function of $\nu$. We have used $x = \nu$ and $f(x) = \frac{u \beta c^3}{8 \pi}$.}
        \label{br:rj}
    \end{figure}
