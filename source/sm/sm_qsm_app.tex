\part{Application of quantum statistical mechanics}

\section{Quantum magnetic 1/2-spin}

    Consider a system composed by $N$ distinguishable magnetic dipoles in an external magnetic field along the $z$-axis, with spin $S = 1/2$. Its hamiltonian is 
    \begin{equation*}
        \hat H = \sum_i S^{(z)}_i B ~,
    \end{equation*}
    where
    \begin{equation*}
        S^{(z)}_i = \frac{1}{2} \begin{bmatrix}
            1 & 0 \\
            0 & -1 \\
        \end{bmatrix} ~.
    \end{equation*}

    The canonical partition function is 
    \begin{equation*}
        Z = \Big ( 2 \cosh \frac{\beta B}{2} \Big)^N ~.
    \end{equation*}
    \begin{proof}
        By definition, for distinguishable particles,
        \begin{equation*}
        \begin{aligned}
            Z & = (Z_1)^N \\ & = \Big (\tr_{\mathcal H} \exp(- \beta \hat H_1) \Big)^N \\ & = \Big (\tr_{\mathcal H} \exp(- \beta B \begin{bmatrix}
                \frac{1}{2} & 0 \\ 0 & - \frac{1}{2} \\ 
            \end{bmatrix}) \Big)^N \\ & = \Big(\tr_{\mathcal H} \begin{bmatrix}
                \exp(- \frac{\beta B}{2}) & 0 \\ 0 & \exp(\frac{\beta B}{2}) \\ 
            \end{bmatrix} \Big )^N \\ & = \Big( \exp(- \frac{\beta B}{2}) + \exp(\frac{\beta B}{2}) \Big)^N \\ & = \Big ( 2 \cosh \frac{\beta B}{2} \Big)^N ~.
        \end{aligned}
        \end{equation*}
    \end{proof}

    The Helmoltz free energy $F$ is 
    \begin{equation*}
        F = - N k_B T \ln \Big ( 2 \cosh \frac{\beta B}{2} \Big) ~.
    \end{equation*}
    \begin{proof}
        By definition, 
        \begin{equation*}
            F = - \frac{\ln Z}{\beta} = - N k_B T \ln \Big ( 2 \cosh \frac{\beta B}{2} \Big) ~.
        \end{equation*}
    \end{proof}

    The internal energy $E$ is 
    \begin{equation*}
        E = - N \frac{B}{2} \tanh \frac{\beta B}{2} ~.
    \end{equation*}
    \begin{proof}
        By definition, 
        \begin{equation*}
            E = - \pdv{\ln Z}{\beta} = - N \pdv{\beta} \ln \Big (2 \cosh \frac{\beta B}{2} \Big) = - N \frac{B}{2} \tanh \frac{\beta B}{2} ~.
        \end{equation*}
    \end{proof}
    A plot of this is in Figure~\ref{qm:e}.
    \begin{figure}
        \centering
        \scalebox{0.7}{\pyc{plot1('x', '- ( tanh (1 / x))', 4, 1, 20, True, True, False)}}
        \caption{A plot of the energy $E$ as a function of $T$. We have used $x = \frac{2 k_B T}{B} $ and $f(x) = \frac{2E}{BN}$.}
        \label{qm:e}
    \end{figure}

    The magnetisation $M$ is 
    \begin{equation*}
        M = - \frac{N}{2} \tanh \frac{\beta B}{2} ~. 
    \end{equation*}
    \begin{proof}
        By definition, 
        \begin{equation*}
            M = \pdv{F}{B} = - N k_B T \pdv{}{B} \ln \Big ( 2 \cosh \frac{\beta B}{2} \Big) = - \frac{N}{2} \tanh \frac{\beta B}{2} ~.
        \end{equation*}
    \end{proof}
    A plot of this is in Figure~\ref{qm:m}.
    \begin{figure}
        \centering
        \scalebox{0.7}{\pyc{plot1('x', '- ( tanh (1 / x))', 4, 1, 21, True, True, False)}}
        \caption{A plot of the magnetisation $M$ as a function of $T$. We have used $x = \frac{2 k_B T}{B} $ and $f(x) = \frac{2M}{N}$.}
        \label{qm:m}
    \end{figure}

\section{Quantum magnetic 1-spin}

    Consider a system composed by $N$ distinguishable magnetic dipoles in an external magnetic field along the $z$-axis, with spin $S = 1$. Its hamiltonian is 
    \begin{equation*}
        \hat H = \sum_i S^{(z)}_i B ~,
    \end{equation*}
    where
    \begin{equation*}
        S^{(z)}_i = \begin{bmatrix}
            1 & 0 & 0 \\
            0 & 0 & 0 \\
            0 & 0 & - 1 \\
        \end{bmatrix} ~.
    \end{equation*}

    The canonical partition function is 
    \begin{equation*}
        Z = \Big ( 2 \cosh (\beta B) + 1 \Big)^N ~.
    \end{equation*}
    \begin{proof}
        By definition, for distinguishable particles,
        \begin{equation*}
        \begin{aligned}
            Z & = (Z_1)^N \\ & = \Big (\tr_{\mathcal H} \exp(- \beta \hat H_1) \Big)^N \\ & = \Big (\tr_{\mathcal H} \exp(- \beta B \begin{bmatrix}
                1 & 0 & 0 \\ 0 & 0 & 0 \\ 0 & 0 & - 1 \\ 
            \end{bmatrix}) \Big)^N \\ & = \Big(\tr_{\mathcal H} \begin{bmatrix}
                \exp(- \beta B) & 0 & 0 \\ 0 & 1 & 0 \\ 0 & 0 & \exp(\beta B) \\ 
            \end{bmatrix} \Big )^N \\ & = \Big( \exp(- \beta B) + 1 + \exp(\beta B) \Big)^N \\ & = \Big ( 2 \cosh (\beta B) + 1 \Big)^N ~.
        \end{aligned}
        \end{equation*}
    \end{proof}

    The Helmoltz free energy $F$ is 
    \begin{equation*}
        F = - N k_B T \ln \Big ( 2 \cosh (\beta B) + 1 \Big) ~.
    \end{equation*}
    \begin{proof}
        By definition, 
        \begin{equation*}
            F = - \frac{\ln Z}{\beta} = - N k_B T \ln \Big ( 2 \cosh (\beta B) + 1 \Big) ~.
        \end{equation*}
    \end{proof}

    The internal energy $E$ is 
    \begin{equation*}
        E = - 2 N B \frac{\sinh (\beta B)}{2 \cosh (\beta B) + 1} ~.
    \end{equation*}
    \begin{proof}
        By definition, 
        \begin{equation*}
            E = - \pdv{\ln Z}{\beta} = - N \pdv{}{\beta} \ln \Big ( 2 \cosh (\beta B) + 1 \Big) = - 2 N B \frac{\sinh (\beta B)}{2 \cosh (\beta B) + 1} ~.
        \end{equation*}
    \end{proof}
    A plot of this is in Figure~\ref{qm:e1}.
    \begin{figure}
        \centering
        \scalebox{0.7}{\pyc{plot1('x', '- (( sinh(1 / x) ) / (2 * cosh (1 /x) + 1))', 4, 1, 22, True, True, False)}}
        \caption{A plot of the energy $E$ as a function of $T$. We have used $x = \frac{k_B T}{B} $ and $f(x) = \frac{2E}{BN}$.}
        \label{qm:e1}
    \end{figure}

    The magnetisation $M$ is 
    \begin{equation*}
        M = - \frac{N}{2} \frac{\sinh (\beta B)}{2 \cosh (\beta B) + 1} ~. 
    \end{equation*}
    \begin{proof}
        By definition, 
        \begin{equation*}
            M = \pdv{F}{B} = - N k_B T \pdv{}{B} \ln \Big ( 2 \cosh (\beta B) + 1 \Big) = - 2 N \frac{\sinh (\beta B)}{2 \cosh (\beta B) + 1}  ~.
        \end{equation*}
    \end{proof}
    A plot of this is in Figure~\ref{qm:m2}.
    \begin{figure}
        \centering
        \scalebox{0.7}{\pyc{plot1('x', '- (( sinh (1 / x) ) / (2 * cosh (1 / x) + 1))', 4, 1, 23, True, True, False)}}
        \caption{A plot of the magnetisation $M$ as a function of $T$. We have used $x = \frac{2 k_B T}{B} $ and $f(x) = \frac{M}{2N}$.}
        \label{qm:m2}
    \end{figure}

\section{Quantum harmonic oscillators}

    Consider a system composed by $N$ distinguishable quantum harmonic oscillators. Its hamiltonian is 
    \begin{equation*}
        \hat H = \sum_i \hbar \omega (\hat a_i \hat a_i^\dagger + \frac{1}{2}) ~.
    \end{equation*}

    The canonical partition function is 
    \begin{equation*}
        Z = \Big ( \frac{\exp(- \frac{\beta \hbar \omega}{2})}{1 - \exp(- \beta \hbar \omega)} \Big)^N ~.
    \end{equation*}
    \begin{proof}
        By definition, for distinguishable particles,
        \begin{equation*}
        \begin{aligned}
            Z & = (Z_1)^N \\ & = \Big (\tr_{\mathcal H} \exp(- \beta \hat H_1) \Big)^N \\ & = \Big (\tr_{\mathcal H_i} \exp(- \beta \hbar \omega (\hat a_i \hat a_i^\dagger + \frac{1}{2})) \Big)^N \\ & = \Big ( \sum_i \bra{n_i}\exp(- \beta \hbar \omega (\hat a_i \hat a_i^\dagger + \frac{1}{2})) \ket{n_i }\Big)^N \\ & = \Big ( \sum_i \exp(- \beta \hbar \omega (n_i + \frac{1}{2})) \Big)^N \\ & = \Big ( \exp(- \frac{\beta \hbar \omega}{2}) \sum_i \exp(- \beta \hbar \omega)^{n_i} \Big)^N \\ & = \Big ( \exp(- \frac{\beta \hbar \omega}{2}) \frac{1}{1 - \exp(- \beta \hbar \omega)} \Big)^N
            \\ & = \Big ( \frac{\exp(- \frac{\beta \hbar \omega}{2})}{1 - \exp(- \beta \hbar \omega)} \Big)^N ~.
        \end{aligned}
        \end{equation*}
    \end{proof}

    The Helmoltz free energy $F$ is 
    \begin{equation*}
        F = N k_B T ( \frac{\beta \hbar \omega}{2} + \ln (1 - \exp(- \beta \hbar \omega))) ~.
    \end{equation*}
    \begin{proof}
        By definition, 
        \begin{equation*}
        \begin{aligned}
            F & = - \frac{\ln Z}{\beta} \\ & = - N k_B T \ln \Big ( \frac{\exp(- \frac{\beta \hbar \omega}{2})}{1 - \exp(- \beta \hbar \omega)} \Big) \\ & = - N k_B T ( \ln \exp(- \frac{\beta \hbar \omega}{2}) - \ln (1 - \exp(- \beta \hbar \omega) )) \\ & = - N k_B T ( - \frac{\beta \hbar \omega}{2} - \ln (1 - \exp(- \beta \hbar \omega))) \\ & = N k_B T ( \frac{\beta \hbar \omega}{2} + \ln (1 - \exp(- \beta \hbar \omega))) ~.
        \end{aligned}
        \end{equation*}
    \end{proof}

    The internal energy $E$ is 
    \begin{equation*}
        E = N ( \frac{\hbar \omega}{2} + \frac{\hbar \omega}{\exp(- \beta \hbar \omega) - 1} ) ~.
    \end{equation*}
    \begin{proof}
        By definition, 
        \begin{equation*}
        \begin{aligned}
            F & = - \pdv{\ln Z}{\beta} \\ & = - N \pdv{}{\beta} \ln \Big ( \frac{\exp(- \frac{\beta \hbar \omega}{2})}{1 - \exp(- \beta \hbar \omega)} \Big) \\ & = - N \pdv{}{\beta} ( \ln \exp(- \frac{\beta \hbar \omega}{2}) - \ln (1 - \exp(- \beta \hbar \omega) )) \\ & = - N \pdv{}{\beta} ( - \frac{\beta \hbar \omega}{2} - \ln (1 - \exp(- \beta \hbar \omega))) \\ & = N \pdv{}{\beta} ( \frac{\beta \hbar \omega}{2} + \ln (1 - \exp(- \beta \hbar \omega))) \\ & = N ( \frac{\hbar \omega}{2} - \frac{\hbar \omega}{1 - \exp(- \beta \hbar \omega)} ) \\ & = N ( \frac{\hbar \omega}{2} + \frac{\hbar \omega}{\exp(- \beta \hbar \omega) - 1} ) ~.
        \end{aligned}
        \end{equation*}
    \end{proof}

    The specific heat is 
    \begin{equation*}
        C_V = N \frac{\hbar^2 \omega^2}{k_B T^2} \frac{\exp(\beta \hbar \omega)}{(\exp(\beta \hbar \omega) - 1)^2} ~.
    \end{equation*}
    \begin{proof}
        In fact 
        \begin{equation*}
            C_V = \pdv{E}{T} = N \pdv{}{T} ( \frac{\hbar \omega}{2} + \frac{\hbar \omega}{\exp(- \beta \hbar \omega) - 1} ) = N \frac{\hbar^2 \omega^2}{k_B T^2} \frac{\exp(\beta \hbar \omega)}{(\exp(\beta \hbar \omega) - 1)^2} ~.
        \end{equation*}
    \end{proof}

\chapter{Fermions}

\section{Fermionic relativistic gas at T=0}

\section{Fermionic White dwarf}

    A white dwarf is an helium star woth mass $M \sim 10^{30} kg$ and a density of $\rho = 10^{10} kg/m^3$ at a temperature of $10^{7} K$. Our approxiated model is composed by $N$ electrons and $N/2$ helium nuclei.

    Assuming $M = N(m_e + 2 m_p) \sim 2 N m_p$, the electronic dentity is 
    \begin{equation*}
        n = 3 \times 10^{36} m^{-3} ~.
    \end{equation*}
    \begin{proof}
        In fact 
        \begin{equation*}
            n = \frac{N}{V} = \frac{N \rho}{M} = \frac{N \rho}{2 N m_p} = \frac{\rho}{2 m_p} = \frac{10^{10}}{2 \times 1.6 \times 10^{-27}} = 3 \times 10^{36} m^{-3} ~.
        \end{equation*}
    \end{proof}

    The Fermi momentum $p_F$ is 
    \begin{equation*}
        p_F = h \Big ( \frac{3 n}{4 \pi g} \Big)^{1/3} = 6.63 \times 10^{-34}  \times \Big ( \frac{3 \times 10^{10}}{4 \times 3.14 \times 2} \Big)^{1/3} = 0.88 Mev/c ~.
    \end{equation*}
    \begin{proof}
        In fact, using $p = \hbar k$
        \begin{equation*}
            N = g \sum_{n} \rightarrow g \frac{V}{(2\pi)^3} \int d^3 k = g \frac{V}{(2\pi \hbar)^3} \int d^3 p = g \frac{4 \pi V}{(2\pi \hbar)^3} \int_0^{p_F} dp p^2 = g \frac{4 \pi V}{(2\pi \hbar)^3} \frac{p_F^3}{3} ~,
        \end{equation*}
        hence 
        \begin{equation*}
            p_F = h \Big ( \frac{3 n}{4 \pi g} \Big)^{1/3} = 6.63 \times 10^{-34}  \times \Big ( \frac{3 \times 10^{10}}{4 \times 3.14 \times 2} \Big)^{1/3} = 0.88 Mev/c ~.
        \end{equation*}
    \end{proof}

    The Fermi energy $\epsilon_F$ is 
    \begin{equation*}
        \epsilon_F = \sqrt{(p_F c)^2 + (mc^2)^2} - mc^2 = 0.5 Mev ~.
    \end{equation*}

    The Fermi temperature $T_F$ is 
    \begin{equation*}
        T_F = \frac{\epsilon_F}{k_B} = 10^{10} K ~,
    \end{equation*}
    which means that we are in the regime $T \ll T_F$ and we can use $T=0$.

    The internal energy $E$ is 
    \begin{equation*}
        E = \frac{\pi V m^4 c^5}{\pi^2 \hbar^3} f(x_F) ~.
    \end{equation*}
    \begin{proof}
        In fact,
        \begin{equation*}
        \begin{aligned}
            E & = g \sum_{n} \epsilon \rightarrow g \frac{V}{(2\pi)^3} \int d^3 k \epsilon \\ & = g \frac{V}{(2\pi \hbar)^3} \int d^3 p \epsilon \\ & = g \frac{4 \pi V}{(2\pi \hbar)^3} \int_0^{p_F} dp p^2 \epsilon \\ & = g \frac{4 \pi V}{(2\pi \hbar)^3} \int_0^{p_F} dp p^2 c \sqrt{p^2 + (mc)^2} ~.
        \end{aligned}
        \end{equation*}

        Now we make a change of variable 
        \begin{equation*}
            x = \frac{p}{mc} ~, \quad dp = mc dx ~,
        \end{equation*}
        hence 
        \begin{equation*}
        \begin{aligned}
            E & = g \frac{4 \pi V}{(2\pi \hbar)^3} c (mc)^3 \int_0^{x_F} dx ~ x^2 (mc)\sqrt{x^2 + 1} \\ & =  \frac{4 g \pi V m^4 c^5}{h^3} \int_0^{x_F} dx ~ x^2\sqrt{x^2 + 1} \\ & = \frac{4 g \pi V m^4 c^5}{h^3} f(x_F) \\ & = \frac{V m^4 c^5}{\pi^2 \hbar^3} f(x_F) ~,
        \end{aligned}
        \end{equation*}
        where 
        \begin{equation*}
            f(x_F) = \int_0^{x_F} dx ~ x^2 \sqrt{x^2 + 1} ~.
        \end{equation*}
    \end{proof}

    The pressure $P$ is 
    \begin{equation*}
        P = \frac{m^4 c^5}{\pi^2 \hbar^3} \Big (\frac{x_F^3}{3} \sqrt{1 + x_F^2} - f(x_F)) ~.
    \end{equation*}
    \begin{proof}
        In fact,
        \begin{equation*}
        \begin{aligned}
            P & = - \pdv{E}{V} \\ & = - \pdv{}{V} \frac{ V m^4 c^5}{\pi^2 \hbar^3} f(x_F) \\ & = - \frac{\pi m^4 c^5}{\pi^2 \hbar^3} f(x_F) - \frac{ V m^4 c^5}{\pi^2 \hbar^3} \pdv{x_F}{V} \pdv{f(x_F)}{x_F} \\ & = - \frac{ m^4 c^5}{\pi^2 \hbar^3} f(x_F) - \frac{ V m^4 c^5}{\pi^2 \hbar^3} \pdv{}{V} \Big (\frac{h}{mc} \Big ( \frac{3 N}{4 \pi g V} \Big)^{1/3}) \pdv{f(x_F)}{x_F} \\ & = - \frac{m^4 c^5}{\pi^2 \hbar^3} f(x_F) - \frac{V m^4 c^5}{\pi^2 \hbar^3} \Big (\frac{h}{mc} \Big ( \frac{3 N}{4 \pi g V} \Big)^{1/3}) \pdv{}{V} V^{-1/3} \pdv{f(x_F)}{x_F} \\ & = - \frac{m^4 c^5}{\pi^2 \hbar^3} f(x_F) + \frac{1}{3} \frac{V m^4 c^5}{\pi^2 \hbar^3} \Big (\frac{h}{mc} \Big ( \frac{3 N}{4 \pi g V} \Big)^{1/3}) V^{-4/3} \pdv{f(x_F)}{x_F} \\ & = \frac{m^4 c^5}{\pi^2 \hbar^3} \Big (\frac{x_F^3}{3} \sqrt{1 + x_F^2} - f(x_F)) ~.
        \end{aligned}
        \end{equation*}
    \end{proof}

    Now, we solve the integral 
    \begin{equation*}
        f(x) = \py{indint('x**2 * sqrt(x**2 + 1)', 'x')} ~.
    \end{equation*}
    In the non-relativistic limit, $x_F \ll 1$, we can make the approximations 
    \begin{equation*}
        g(x) = \frac{x^3}{3} \sqrt{1 + x^2} = \py{Taylor('x', 'x**3 / 3 * sqrt(1 + x**2)', 0, 6)}
    \end{equation*}
    and 
    \begin{equation*}
        f(x_F) = \py{Taylor('x', 'x**5 / (4 * sqrt(x**2 + 1)) + ( 3 * x**3) / (8 * sqrt(x**2 + 1)) + x / (8 * sqrt(x**2 + 1)) - asinh(x) / 8', 0, 6)} ~.
    \end{equation*}

    In the ultra-relativistic limit, $x_F \gg 1$ or equivalemty $y_F = 1 / x_F \ll 1$, we can make the approximations 
    \begin{equation*}
        g(1/x) = \py{Taylor('x', '(1 / x)**3 / 3 * sqrt(1 + (1 / x)**2)', 0, -1)}
    \end{equation*}
    and 
    \begin{equation*}
        f(1 / x) = \py{Taylor('x', '(1 / x)**5 / (4 * sqrt((1 / x)**2 + 1)) + ( 3 * (1 / x)**3) / (8 * sqrt((1 / x)**2 + 1)) + (1 / x) / (8 * sqrt((1 / x)**2 + 1)) - (ln ( 1 / x + sqrt((1/x)**2 + 1))) / 8', 0, -1)} ~.
    \end{equation*}
    
    Imposing the equilibrium condition $dE = 0$, between the gravitational and the pressure forces, and the structure of a sphere, the pressure must be 
    \begin{equation*}
        P = \frac{\alpha G M^2}{4 \pi R^4}
    \end{equation*}
    and the Fermi momentum is 
    \begin{equation*}
        p_F = \frac{\hbar}{R} \Big ( \frac{9 \pi M}{8 m_p} \Big)^{1/3} ~.
    \end{equation*}
    \begin{proof}
        For the gravitational force 
        \begin{equation*}
            E_g = - \alpha \frac{G M^2}{R} ~, \quad dE_g = \alpha \frac{GM^2}{R^2} dR ~.
        \end{equation*}
        For the pressure force 
        \begin{equation*}
            E_p = - p V = - p \frac{4}{3} \pi R^3 ~, \quad dE_p = - 4 \pi p R^2 dR ~.
        \end{equation*}
        Imposing the equilibrium condition, 
        \begin{equation*}
            0 = dE = dE_g + dE_p = alpha \frac{GM^2}{R^2} dR - 4 \pi p R^2 dR ~,
        \end{equation*}
        hence 
        \begin{equation*}
            p = \frac{\alpha G M^2}{4 \pi R^4} ~.
        \end{equation*}

        The Fermi momentum is 
        \begin{equation*}
            p_F = h \Big ( \frac{3 n}{4 \pi g} \Big)^{1/3} = h \Big ( \frac{3}{8 \pi} \frac{M}{2 m_p \frac{4}{3} \pi R^3} \Big)^{1/3} = \frac{\hbar}{R} \Big ( \frac{9 \pi M}{8 m_p} \Big)^{1/3}  ~.
        \end{equation*}
    \end{proof}

    In the ultra-relativistic limit
    \begin{equation*}
        P = \frac{m^4 c^5}{12 \pi \hbar^3} (x_F^4 - x_F^2) = \frac{\alpha G M^2}{4 \pi R^4} ~.
    \end{equation*}

\section{Bose-Einstein condensate in 2D}

\section{Bosonic blackbody radiation}
