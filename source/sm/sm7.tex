\part{Phase transition}

\chapter{Classical phase transitions}

    Consider the phase diagram of the water. Microscopically, they all have the same hamiltonian, however, the macroscopical variables changes. There are three phases 
    \begin{enumerate}
        \item solid, i.e. it has its own shape and volume, 
        \item liquid, i.e. it has its own volume but it has the shape of the container,
        \item gas, i.e. it has the shape and volume of the container. 
    \end{enumerate}
    There are lines, called coexistence lines, along which $2$ phases are in equilibrium. They are lines because, other than $T_1 = T_2$ and $p_1 = p_2$, we have a costrain 
    \begin{equation*}
        \mu_1(p, T) = \mu_2 (p, T) ~.
    \end{equation*}
    This reduce to a line. 

    Furthermore, there are points, called coexistence points or triple point, in which $3$ phases are in equailibrium. They are points because, other than $T_1 = T_2 = T_3$ and $p_1 = p_2 = p_3$, we have the costrains
    \begin{equation*}
        \mu_1(p, T) = \mu_2 (p, T) = \mu_3 (p, T) ~.
    \end{equation*}
    This reduce to a point.
    
\section{Symmetries} 

    We can use symmetries of the system to study it. We can distinguish solid from fluid by the translation or rotations invariance. In fact, solid has only discrete invariance, whereas fluid has continuous invariance. However, we cannot distinguish with symmetries between gas and liquid.

\section{Clausius-Clapeyron equation}

    On the coexistence line, the Clausius-Clapeyron equation is 
    \begin{equation*}
        \dv{p}{T} = \frac{s_2 - s_1}{v_1 - v_2} = \frac{\Delta q}{T \Delta v}  ~.
    \end{equation*}
    \begin{proof}
        In order to remain on the coexistence line, the costrain relation holds
        \begin{equation*}
            \mu_1(p, T) = \mu_2 (p, T) ~.
        \end{equation*}
        We differentiate it, keeping in mind that $p = p(T)$
        \begin{equation*}
            \pdv{\mu_1}{T} \Big \vert_p + \dv{\mu_1}{p} \Big \vert_T \dv{p}{T} = \pdv{\mu_2}{T} \Big \vert_p + \dv{\mu_2}{p} \Big \vert_T \dv{p}{T} ~,
        \end{equation*}
        hence 
        \begin{equation*}
            \dv{p}{T} = \frac{\pdv{\mu_1}{T} \vert_p - \pdv{\mu_2}{T} \vert_p}{\pdv{\mu_2}{p} \vert_T - \pdv{\mu_2}{p} \vert_T} ~.
        \end{equation*}

        At fixed number of particle, we can use the Gibbs free energy $G(p, T, N) = \mu (p, T) N$ or the Gibbs free energy per particle 
        \begin{equation*}
            g = \frac{G}{N} = \mu(p,T) ~.
        \end{equation*}
        Using the relations~\eqref{ges}
        \begin{equation*}
            \pdv{\mu}{p} \Big \vert_T = \pdv{g}{p} \Big \vert_T = \frac{1}{N} \pdv{G}{p} \Big \vert_T = \frac{V}{N} = v ~,
        \end{equation*}
        \begin{equation*}
            \pdv{\mu}{T} \Big \vert_p = \pdv{g}{T} \Big \vert_p = \frac{1}{N} \pdv{G}{T} \Big \vert_p = - \frac{S}{N} = - s ~,
        \end{equation*}
        we obtain
        \begin{equation*}
            \dv{p}{T} = \frac{\pdv{\mu_1}{T} \vert_p - \pdv{\mu_2}{T} \vert_p}{\pdv{\mu_2}{p} \vert_T - \pdv{\mu_2}{p} \vert_T} = \frac{s_2 - s_1}{v_2 - v_1} ~.
        \end{equation*}

        Furthermore, when there is a phase change, temperature remains constant whereas the thermal energy put in the system is transformed into latent heat
        \begin{equation*}
            \Delta s = \frac{\Delta q}{T} ~.
        \end{equation*}
        Therefore
        \begin{equation*}
            \dv{p}{T} = \frac{\Delta q}{T \Delta v} ~.
        \end{equation*}
    \end{proof}


    There could be $2$ different kind of phase transitions 
    \begin{enumerate}
        \item $1$st order phase transitions, i.e. those in which the $1$st derivatives of thermodynamic potentials are discontinuous;
        \item continuous phase transitions, i.e. those in which the higher derivatives of thermodynamic potentials are discontinuous.
    \end{enumerate}

    In our case, the former are those in which there is a jump $v_2 \neq v_1$ and $s_2 \neq s_1$ and the latter are those in which $v_2 = v_1$ and $s_2 = s_1$. 

\chapter{Theorems of Lee and Young}

    Consider a classical fluid in a volume $V \subset \mathbb R^3$. We treat it in the grancanonical ensemble. The grancanonical partition function is 
    \begin{equation*}
        \mathcal Z [V, T, z] = \sum_{N=0}^\infty z^n Z_N[T, V] ~.
    \end{equation*}

    The canonical partition function is 
    \begin{equation*}
    \begin{aligned}
        Z_N & = \frac{1}{N! h^{3N}} \int_{V^N} \prod_i d^3 q^i \underbrace{\int_{\mathbb R^{3N}} \prod_i d^3 p^i  \exp (- \beta \sum_j \frac{p_j}{2m}}_{\frac{1}{\lambda_T^{3N}}} + U_N(q^i)) \\ & = \frac{1}{N! \lambda_T^{3N}} \underbrace{\int_{V^N}\prod_i d^3 q^i \exp (- \beta U_N(q^i))}_{Q_N (T, V)} = \frac{Q_N(T, V)}{N! \lambda_T^{3N}} ~.
    \end{aligned}
    \end{equation*}
    Notice that $Q_N(T,V) > 0$.

    Therefore 
    \begin{equation*}
        \mathcal Z = \sum_{N=0}^\infty z^n \frac{Q_N}{N! \lambda_T^{3N}} ~.
    \end{equation*}
    which is a power series in $z$. Now we promote $z$ into a complex variables, keeping in mind that the physical states are only the ones for $z \in \mathbb R^+$

    We make the assumption that $U_N \geq - BN$ with $B > 0$, which means that it grows no more than $N$ order. This implies that 
    \begin{equation*}
        \exp(- \beta U_N) \leq \exp(\beta B N) ~,
    \end{equation*}
    hence 
    \begin{equation*}
        Q_N = \int_{V^N} \prod_i d^3 q^i \exp (- \beta U_N(q^i)) \leq \exp(\beta B N) \int_{V^N}\prod_i d^3 q^i = \exp(\beta B N) V^N 
    \end{equation*}
    and 
    \begin{equation*}
        Z_N = \frac{Q_N}{N! \lambda_T^{3N}} \leq \frac{V^N}{N! \lambda^{3N}_T} \exp(\beta B N) ~.
    \end{equation*}
    Therefore 
    \begin{equation*}
        |\mathcal Z| \leq \sum_{N=0}^\infty \frac{|z|^N}{N! \lambda_T^{3N}} V^N \exp(\beta B N) = \exp(\frac{V \exp(\beta B) |z|}{\lambda_T^3}) ~,
    \end{equation*}
    which, given the fact that it is an exponential, has an infinite convergence radius. We have proved that it is analytical $\forall z \in \mathbb C$, in particular for $z \in \mathbb R^+$. Furthermore, $\mathcal Z$ cannot become vanishing since it is convergent and it is a sum of positive terms. We introduce the granpotential 
    \begin{equation*}
        \Omega = - \frac{1}{\beta} \ln \mathcal Z ~,
    \end{equation*}
    which is well defined, since $\mathcal Z \neq 0$ and analytical $\forall z \in \mathbb R^+$. 

    Finally, all thermodynamic functions are analytical for $z \in \mathbb R^+$ and there are no phase transitions. How it it possible? We have not yet computed the thermodynamic limit.

    Consider a system composed of hard sphere occupying a finite volume $v$. The maximum number of particles is $M = \frac{V}{v}$. $\mathcal Z$ is a polynomial function in $z$ of degree $M$ and, by the fundamental theorem of algebra, it has $M$ zeroes but none in $\mathbb R^+$. If we go into the thermodynamic limit, $V \rightarrow \infty$, $M \rightarrow \infty$ and the number of zeroes increases. However, it holds that 
    \begin{enumerate}
        \item $\forall V, M$, there exists an open subset of $\mathbb R^+$ which does not contain zeroes, i.e. $\mathcal Z (V \rightarrow \infty, T, \mu)$ has no zeroes on $\mathbf R^+$,
        \item if zeroes accumulate towards a certain $z = z_c$, then  $\mathcal Z (V \rightarrow \infty, T, \mu)$ has a zero in $z = z_c$.
    \end{enumerate}
    This means that $\Omega$ is no longer analytical at $z = z_c$. Now, there is no more equilbrium and phase transitions come up from the dark. 

    This statements can be written down in terms of 
    \begin{equation*}
        \psi = \lim_{td} \frac{\ln \mathcal Z}{V} ~.
    \end{equation*}
    Hence 
    \begin{equation*}
        p \beta = \psi ~, \quad n = z \pdv{}{z} \psi ~.
    \end{equation*}
    \begin{proof}
        For the first 
        \begin{equation*}
            \Omega = - p V = - \frac{1}{\beta} \ln \mathcal Z ~,
        \end{equation*}
        hence
        \begin{equation*}
            p \beta = \frac{\ln \mathcal Z}{V} = \psi ~.
        \end{equation*}

        For the second,
        \begin{equation*}
            N = z \pdv{}{z} \ln \Omega = - \frac{z}{\beta} \pdv{}{z} \ln \Omega ~,
        \end{equation*}
        hence
        \begin{equation*}
            n = \frac{N}{V} = - \frac{z}{\beta} \pdv{}{z} \frac{\ln \Omega}{V} =  - \frac{z}{\beta} \pdv{}{z} \psi ~.
        \end{equation*}
    \end{proof}

    \begin{theorem}
        If $U_N \geq - BN$ with $B > 0$, if the boundary of the volume does not increases fastes than $V^{2/3}$, in order to neglect surface terms, then $\psi$ exists, it is continuous and monotonically increasing.
    \end{theorem}
    \begin{theorem}
        Given an open subset of an interval of $\mathbb R^+$ such that it doesn not contain zeroes, then $\psi$ exists and it is analytic.
    \end{theorem}
    \begin{corollary}
        A phase transitions may appear at $z = z_c$ if it is an accumulation point of zeroes. This point divides $\mathbb R^+$ into $2$ regions corresponding to $2$ different phases. Furthermore, $\psi$ is continuous but it is not analytic: $1st$ order phase transitions or continuous phase transitions.
    \end{corollary}

\chapter{Ising model}

    Consider a system composed by a discrete lattice, for example an hypercubic lattice of dimension $d$. For each vertex, there is a degree of freedom, characterised by the approximation of a spin that could have only two values $\sigma = \pm 1$. Two adjacent verteces are called nearest neighborhood. Each site has therefore $z$ nearest neighborhood, called the coordination number. For a dimension $d$ hybercube, $z = 2 d$. A possible configuration stae is defined as $\{\sigma_i\}_{i \in \mathcal L}$. The phase space is a discrete space composed by $2^N$ states $\{\{\sigma_i\}_{i \in \mathcal L}, \sigma_i = \pm 1\}$. 

    The hamiltonian of the system is 
    \begin{equation*}
        H(\sigma_i) = H_{int} + H_{field} ~,
    \end{equation*}
    where
    \begin{equation*}
        H_{int} = - J \sum_{i \text{near} j} \sigma_i \sigma_j 
    \end{equation*}
    and 
    \begin{equation*}
        H_{field} = - B \sum_{i=1}^{N} \sigma_i ~.
    \end{equation*}
    $B$ is an external magnetic field and $J$ is the interaction constant. Notice that in order to have a phase transitions, we have to allow interactions. For $J > 0$, the minimum energy configuration is the one in which all the spins are aligned $\sigma_i = \sigma_j ~, \quad \forall i, j$. This model is called ferromagnetic model. For $J < 0$, the minimum energy configuration is the one in which all the spins are antialigned $\sigma_i = - \sigma_j ~, \quad \forall i, j$. This model is called antiferromagnetic model. For $B > 0$, the minimum energy configuration is the one in which all the spins are aligned upwards $\sigma_i = + 1$. For $B < 0$, the minimum energy configuration is the one in which all the spins are aligned upwards $\sigma_i = - 1$. 

    In the canonical ensemble, the partition function is 
    \begin{equation*}
        Z_N = \sum_{\sigma_i = \pm 1} \exp(- \beta H(\sigma_i)) ~,
    \end{equation*}
    where the sum is made over all the $2^N$ states. The Helmoltz free energy is 
    \begin{equation*}
        F = E - TS = - \frac{1}{\beta} \ln Z_N ~.
    \end{equation*}
    where $E = \av{H}_c$. The thermodynamic equilibrium correspond to the configuration of minimum free energy.

    Suppose the external magnetic field is shut down. The ground state is the one with minimum energy and the entropy is small, because there are only $2$ states possible with all aligned spins. The excited state is the one with minimum energy and the entropy is big, because all spins point in all direction. Recall that entropy is $S = k_B \ln \Gamma(E)$. The minimal configuration of free energy is therefore at low $T$ with minimum $E$, i.e. all aligned, and at high $T$ with large $S$, i.e. random alignment.

    To stimate the alignment, we introduce the magnetisation 
    \begin{equation*}
        M = \av{\sum_{i=1}^N \sigma_i}_c = \sum_{i=1}^N \av{\sigma_i}_c ~,
    \end{equation*}
    where we have used the translation invariance. 

    Computing the phase diagram, we find that along the $T$-axis at $B=0$, $m \neq 0$ for $T < T_c$ and $m = 0$ for $T > T_c$, where $T_c$ is the critical temperature. The former is called the ferromagnetic phase and the latter is called the paramagnetic phase. We can use the magnetisation as an order parameter, since when it is zero there is disorder and when it is different from zero, there is order. In the neighborhood of $T_c$, we have the behaviour for $T < T_c$
    \begin{equation*}
        M \sim (T - T_c)^\beta ~,
    \end{equation*}
    where $\beta$ is a parameter. It characterise the phase transition, since it tells which speed $M \rightarrow 0$ when approacing $T \rightarrow T_c$.

\section{Correlation}

\section{Symmetry breaking}

    Consider the Ising model hamiltonian. The first term is invariant under 
    \begin{equation*}
        \sigma_i \rightarrow - \sigma_i ~, \quad \sigma_i \rightarrow \sigma_i ~.
    \end{equation*}
    Therefore, it is invariant under the global symmetry group $\mathbb Z_2$. If it were the only term, i.e. with $B=0$, the whole hamiltonian would be invariant under this group. Howver, the second term breaks explicitly the symmetry. Moreover, noticing that under this symmetry $m \rightarrow - m$, the only possible value of $m$ would be zero. Hence, for $T > T_c$ there is indeed $m=0$. But for $T < T_c$, the equilibrium state is no longer invariant under this symmetry. The hamiltonian remains the same but states are not invariant. There is a spontaneous symmetry breaking, spontaneous because $H$ is stil invariant under $\mathbb Z_2$.

\chapter{Mean-field treatment} 

    Consider the Ising model. In general, it is diffucult to compute the canonical partition function,
    \begin{equation*}
        Z_N = \sum_{\{\sigma_i = \pm 1\}} \exp(- \beta H) \neq (Z_1)^N ~,
    \end{equation*}
    since it is interacting. However, we can make an useful approximation 
    \begin{equation*}
        \sigma_i \sigma_j = ((\sigma_i - m) + m)((\sigma_j - m) + m) = m^2 + m(\sigma_i - m) + m (\sigma_j - m) + (\sigma_i - m)(\sigma_j - m) ~,
    \end{equation*}
    in which we keep only the first constant term and the second linear, but we neglect the last quadratic fluctuation term. Therefore 
    \begin{equation*}
        \sigma_i \sigma_j =  m^2 + m(\sigma_i - m) + m (\sigma_j - m) = m^2 + m\sigma_i - m^2 + m \sigma_j - m^2 = - m^2 + m (\sigma_i + \sigma_j) ~.
    \end{equation*}
    The hamiltonian in the mean-field approximation becomes 
    \begin{equation*}
        H_{mf} = - J \sum_{i \text{nn} j} (- m^2 + m(\sigma_i + \sigma_j)) - B \sum_i \sigma_i = m^2 J \sum_{i \text{nn} j} 1 - J m \sum_{i \text{nn} j} (\sigma_i + \sigma_j) - B \sum_i \sigma_i ~.
    \end{equation*}
    The number of links, given the coordination number $z$ which tells how many neighboring sites, is $NZ/2$. Hence 
    \begin{equation*}
        H_{mf} = \frac{m^2 z N J}{2} - Jmz \sum_i \sigma_i - B \sum_i \sigma_i = frac{m^2 z N J}{2} - (J m z + B) \sum_i \sigma_i ~.
    \end{equation*}
    The physical intepretation of the mean-field treatment is that we do not have to compute every links with respect to each others but only with respect to the mean field $m$. It is valid only if fluctuations are smaller than the mean-field. The partition function is 
    \begin{equation*}
    \begin{aligned}
        Z_N^{mf} & \sum_{\{\sigma_i = \pm 1\}} \exp(- \beta H_{mf}) \\ & = \exp(- \beta \frac{J z n m^2}{2}) \sum_{\{\sigma_i = \pm 1\}} \exp(\beta (B + Jmz) \sum_i \sigma_i) \\ & = \exp(- \beta \frac{J z n m^2}{2}) (\sum_{\{\sigma_i = \pm 1\}} \exp(\beta (B + Jmx) \sigma_i))^N \\ & = \exp(- \beta \frac{J z n m^2}{2}) (\exp(\beta(B + Jmz)) + \exp(- \beta (B + Jmz)))^N \\ & = \exp(- \beta \frac{J z n m^2}{2}) (2 \cosh (\beta (B + Jmz))) ~.
    \end{aligned}
    \end{equation*}
    The Helmoltz free energy is 
    \begin{equation*}
    \begin{aligned}
        F & = - \frac{1}{\beta} \ln Z_N^{mf} \\ & = - \frac{1}{\beta} (- \beta \frac{J z N m^2}{2}) N \ln (2 \cosh (\beta (B + Jmz))) \\ & = \frac{J z N m^2}{2} N \ln (2 \cosh (\beta (B + Jmz))) ~.
    \end{aligned}
    \end{equation*}
    The magnetisation is 
    \begin{equation*}
    \begin{aligned}
         m & = \frac{1}{N} \av{\sum_i \sigma_i}_c \\ & = \frac{1}{N} \sum_{\{\sigma_i = \pm 1\}} \sum_i \sigma_i \exp(- \beta H) \\ & = - \frac{1}{\beta N} \sum_{\{\sigma_i = \pm 1\}} \frac{1}{Z_N} \pdv{}{\beta} \exp(- \beta H) \\ & = - \frac{1}{\beta N} \pdv{\ln Z_N}{\beta} ~.
    \end{aligned}
    \end{equation*}
    Hence 
    \begin{equation*}
        m = \tanh (\beta (B + J m z)) ~.
    \end{equation*}
    We have a self-contistet equation for m to solve. The condition for solution is $B >0$ then $m > 0$ and $B < 0$ then $m < 0$. Particular attention we can study for $B = 0$ then 
    \begin{equation*}
        m = \tanh \frac{J m z}{k_B T} = \tanh \frac{T_c m}{T} ~,
    \end{equation*}
    where $T_c = J z / k_B$ is the critical temperature. Notice that it depends on $z$. Calling $\tilde m = T_c m / T$, we have 
    \begin{equation*}
        \frac{T \tilde m}{T_c} = \tanh \tilde m ~.
    \end{equation*}
    The solutions are points that intersect a straigh line and an hyperbolic tangent. If $T > T_c$, there is only one solution $m = 0$. If $T > T_c$, there are two solutions $\pm m_0$, one positive and one negative. See Figure~\ref{mf:m}.
    \begin{figure}
        \centering
        \scalebox{0.7}{\pyc{plot4('x', '2* x', 'x / 2', 'x', 'tanh(x)', 5, 5, 20, True, False, False)}}
        \caption{A plot of the graphical solution of $T \tilde m / T_c = \tanh \tilde m$.}
        \label{mf:m}
    \end{figure}
    We have proven that $m$ is indeed an order parameter. 
    