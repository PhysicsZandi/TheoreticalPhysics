\chapter{Applications}

    In this chapter, we will study different physical systems. In the microcanonical ensemble, we will analyse a non-relativistic ideal gas and a gas of harmonic oscillators. In the canonical ensemble, we will analyse a non-relativistic ideal gas, a gas of harmonic oscillators, an ultra-relativistic gas, a magnetic solid and the Maxwell-Boltzmann distribution. In the grand canonical ensemble, we will analyse a non-relativistic ideal gas and a gas of harmonic oscillators. Furthermore, we will investigate the counting of state approach with the Maxwell-Boltzmann, Fermi-Dirac and Bose-Einstein distributions, and the two-level system.

\section{Microcanonical non-relativistic ideal gas}

    \begin{exercise}
        Consider a non-relativistic ideal, i.e.~non-interacting, gas of $N$ particles with mass $m$, living in an $2dN$-dimensional manifold with a finite volume $V^N$: $\mathcal M^N = V^N \times \mathbb R^{dN}$, with Hamiltonian 
        \begin{equation*}
            H = \sum_{i=1}^{dN} \frac{p^2_i}{2m} ~.
        \end{equation*}
        Find the number of states $\Sigma(E)$, the density of states $\omega(E)$, the entropy $S$, the internal energy $E$ and the equation of state $p = p(V, T, N)$. Finally, express the same physical quantities in the case $d = 3$.
    \end{exercise}

    The number of states $\Sigma(E)$ is 
    \begin{equation*}
        \Sigma(E) = \frac{V^{N}}{\xi_N \Gamma(dN/2 + 1)} \Big ( \frac{2 \pi m E}{h^2}\Big)^{dN/2} ~.
    \end{equation*}
    In the case $d = 3$, we obtain 
    \begin{equation*}
        \Sigma(E) = \frac{V^{N}}{\xi_N \Gamma(3N/2 + 1)} \Big ( \frac{2 \pi m E}{h^2}\Big)^{3N/2} ~.
    \end{equation*}
    \begin{proof}
        By definition~\eqref{cm:vol}, we have
        \begin{equation*}
        \begin{aligned}
            \Sigma (E) = \int_{H (q_i, p_i) \leq E} d\Omega = \int_{H (q_i, p_i) \leq E} \frac{\prod_i d^d q_i d^d p_i}{h^{dN} \xi_N} = \frac{1}{h^{dN} \xi_N} \int_{H (q_i, p_i) \leq E} \prod_i d^d q_i d^d p_i ~.
        \end{aligned}
        \end{equation*}
        In order to find the domain of integration, we compute the following inequality
        \begin{equation*}
            H = \sum_i \frac{p^2_i}{2m} \leq E ~, \quad \sum_i p^2_i \leq 2mE ~.
        \end{equation*}
        Hence, by the volume of a $dN$-sphere of radius $\sqrt{2mE}$~\eqref{app:volumen}, we find
        \begin{equation*}
        \begin{aligned}
            \Sigma (E) & = \frac{1}{h^{dN} \xi_N} \int_{\sum_i p^2_i \leq 2mE} \prod_i d^d q_i d^d p_i = \frac{1}{h^{dN} \xi_N} \underbrace{\int_{V^N} \prod_i d^d q_i}_{V^N} \underbrace{\int_{\sum_i p^2_i \leq 2mE} \prod_i d^d p_i}_{\frac{\pi^{dN/2} (2mE)^{dN/2}}{\Gamma (dN/2 + 1)}} \\ & = \frac{V^N}{h^{dN} \xi_N} \frac{\pi^{dN/2} (2mE)^{dN/2}}{\Gamma (dN/2 + 1)} = \frac{V^N}{\Gamma (dN/2 + 1) \xi_N} \Big (\frac{2 \pi m E}{h^2} \Big)^{dN/2} ~.
        \end{aligned}
        \end{equation*}
    \end{proof}

    The density of states $\omega(E)$ is
    \begin{equation}\label{freegas}
        \omega (E) = \frac{V^N}{\xi_N \Gamma(dN/2)} \Big ( \frac{2 \pi m}{h^2} \Big )^{dN/2} E^{dN/2-1} ~.
    \end{equation}
    In the case $d = 3$, we obtain  
    \begin{equation*}
        \omega (E) = \frac{V^N}{\xi_N \Gamma(3N/2)} \Big ( \frac{2 \pi m}{h^2} \Big )^{3N/2} E^{3N/2-1} ~.
    \end{equation*}
    \begin{proof}
        By definition~\eqref{cm:denst}, we have
        \begin{equation*}
        \begin{aligned}
            \omega (E) &= \pdv{\Sigma (E)}{E} = \pdv{}{E} \Big (\frac{2 \pi m E}{h^2} \Big)^{dN/2} \frac{2 V^N}{\Gamma (dN/2)\xi_N d N} = \Big (\frac{2 \pi m}{h^2} \Big)^{dN/2} \frac{2 V^N}{\Gamma (dN/2)\xi_N d N} \pdv{E^{dN/2}}{E} \\ & = \Big (\frac{2 \pi m}{h^2} \Big)^{dN/2} \frac{2 V^N}{\Gamma (dN/2)\xi_N d N} \frac{dN}{2} E^{dN/2 - 1} = \frac{V^N}{\xi_N \Gamma(dN/2)} \Big ( \frac{2 \pi m}{h^2} \Big )^{dN/2} E^{dN/2-1} ~.
        \end{aligned}
        \end{equation*}
    \end{proof}
    Notice that 
    \begin{equation*}
        \omega(E) = \frac{dN}{2E} \Sigma(E) ~, \quad \Gamma(E) = \omega(E) \Delta E = \frac{dN}{2E} \Sigma(E) \Delta E ~,
    \end{equation*}
    which proves, in the thermodynamic limit, that the relations~\eqref{mc:tdlim} are equivalents
    \begin{equation*}
        \lim_{TD} \frac{\ln \Gamma (E)}{N} = \lim_{TD} \frac{\ln \omega (E)}{N} = \lim_{TD} \frac{\ln \Sigma (E)}{N} ~.
    \end{equation*}
    \begin{proof}
        $\omega(E)$, $\Gamma(E)$ and $\Sigma(E)$ differ only, in the logarithmic expression, by factors $\ln \Delta E$ and $\ln \frac{dN}{2E}$, which are negligible in the thermodynamic limit because they do not scale at least as $N$.
    \end{proof}
    The entropy $S$ is
    \begin{equation*}
        \frac{S}{k_B} = \ln \Gamma(E) = \ln \omega(E) + \ln \Delta E = \ln \Sigma(E) + \ln \frac{dN}{2E} + \ln \Delta E ~,
    \end{equation*}
    and, in the thermodynamic limit, it becomes
    \begin{equation*}
        S = k_B \begin{cases}
            \frac{d}{2} N + N \ln \Big ( V (\frac{4 \pi m E}{d N h^2})^{d/2} \Big) & \textnormal{for distinguishable particles} \\
            \frac{d + 2}{2} N + N \ln \Big ( \frac{V}{N} (\frac{4 \pi m E}{d N h^2})^{d/2} \Big) & \textnormal{for indistinguishable particles} \\
        \end{cases} ~,
    \end{equation*}
    In the case $d = 3$, we obtain 
    \begin{equation*}
        S = k_B \begin{cases}
            \frac{3}{2} N + N \ln \Big ( V (\frac{4 \pi m E}{3 N h^2})^{3/2} \Big) & \textnormal{for distinguishable particles} \\
            \frac{5}{2} N + N \ln \Big ( \frac{V}{N} (\frac{4 \pi m E}{3 N h^2})^{3/2} \Big) & \textnormal{for indistinguishable particles} \\
        \end{cases} ~.
    \end{equation*}
    \begin{proof}
        By definition~\eqref{mc:s}, using the Stirling approximation~\eqref{app:stirl}, we have
        \begin{equation*}
        \begin{aligned}
            \frac{S}{k_B} & = \ln \Sigma (E) = \ln \Big ( \frac{V^{N}}{\xi_N \Gamma(dN/2 + 1)} \Big ( \frac{2 \pi m E}{h^2}\Big)^{dN/2} \Big ) \\ & =  N \ln V - \ln \xi_N - \underbrace{\ln \Gamma (dN/2 + 1)}_{\frac{dN}{2} \ln \frac{dN}{2} - \frac{dN}{2}} + N \ln \Big (\frac{2 \pi m E}{h^2} \Big )^{d/2} \\ & = N \ln V - \ln \xi_N - \frac{dN}{2} \ln \frac{dN}{2} + \frac{dN}{2} + N \ln \Big (\frac{2 \pi m E}{h^2} \Big )^{d/2} \\ & = N \ln V - \ln \xi_N - N \ln \Big(\frac{dN}{2} \Big)^{d/2} + \frac{dN}{2} + N \ln \Big (\frac{2 \pi m E}{h^2} \Big )^{d/2} \\ & = - \ln \xi_N + \frac{dN}{2} + N \ln \Big ( V (\frac{4 \pi m E}{d N h^2})^{d/2} \Big) ~.
        \end{aligned}
        \end{equation*} 
        Now, we treat the distinguishable and indistinguishable case separately. For distinguishable particles $\xi_N = 1$, we find
        \begin{equation*}
            \frac{S}{k_B} = - \underbrace{\ln 1}_0 + \frac{dN}{2} + N \ln \Big (V \frac{4 \pi m E}{dNh^2} \Big )^{d/2} = \frac{d}{2} N + N \ln \Big ( V (\frac{4 \pi m E}{d N h^2})^{d/2} \Big) ~.
        \end{equation*}
        For indistinguishable particles $\xi_N = N!$, we find
        \begin{equation*}
        \begin{aligned}
            \frac{S}{k_B} & = - \underbrace{\ln N!}_{N \ln N - N} + \frac{dN}{2} + N \ln \Big ( V (\frac{4 \pi m E}{d N h^2})^{d/2} \Big) \\ & = - N \ln N + N + \frac{dN}{2} + N \ln \Big ( V (\frac{4 \pi m E}{d N h^2})^{d/2} \Big) \\ & = \frac{d + 2}{2} N + N \ln \Big ( \frac{V}{N} (\frac{4 \pi m E}{d N h^2})^{d/2} \Big) ~.
        \end{aligned}
        \end{equation*}
    \end{proof}

    The internal energy is 
    \begin{equation*}
        E = \frac{d N k_B T}{2} ~.
    \end{equation*}
    In the case $d = 3$, we obtain 
    \begin{equation*}
        E = \frac{3 N k_B T}{2}
    \end{equation*}
    \begin{proof}
        Using the first of~\eqref{td:es:e}, we have
        \begin{equation*}
            \frac{1}{T} = \pdv{S}{E} = k_B \frac{dN}{2} \pdv{}{E} \ln E = k_B \frac{dN}{2E} ~,
        \end{equation*}
        hence, we find
        \begin{equation*}
            E = \frac{d N k_B T}{2} ~.
        \end{equation*}
    \end{proof}
    The equation of state is  
    \begin{equation*}
        p V = N k_B T ~,
    \end{equation*}
    which is the same for all dimensions.
    \begin{proof}
        By the second of~\eqref{td:es:s}, we have
        \begin{equation*}
            \frac{p}{T} = \pdv{S}{V} = k_B N \pdv{}{V} \ln V = k_B \frac{N}{V}  ~,
        \end{equation*}
        hence, we find
        \begin{equation*}
            pV = N k_B T ~.
        \end{equation*}
    \end{proof}

\section{Microcanonical gas of harmonic oscillators}

    \begin{exercise}
        Consider a non-relativistic, i.e.~non-interacting, gas of $N$ particles with mass $m$, living in an $2dN$-dimensional manifold confined by an harmonic potential of frequency $\omega$, with Hamiltonian 
        \begin{equation*}
            H = \sum_{i =1}^{dN} \Big ( \frac{p^2_i}{2m} + \frac{m \omega^2}{2} q_i^2 \Big ) ~.
        \end{equation*}
        Find the number of states $\Sigma(E)$, the density of states $\omega(E)$, the entropy $S$, the internal energy $E$ and the equation of state $p = p(V, T, N)$. Finally, express the same physical quantities in the case $d = 1$.
    \end{exercise}

    The number of states $\Sigma(E)$ is 
    \begin{equation*}
        \Sigma(E) = \frac{1}{\xi_N\Gamma(dN + 1)} \Big ( \frac{2 \pi E}{h \omega}\Big)^{dN} ~.
    \end{equation*}
    In the case $d = 1$, we obtain 
    \begin{equation*}
        \Sigma(E) = \frac{1}{\xi_N \Gamma(N+1)} \Big ( \frac{2 \pi E}{h \omega}\Big)^{N} ~.
    \end{equation*}
    \begin{proof}
        By definition~\eqref{cm:vol}, we have
        \begin{equation*}
        \begin{aligned}
            \Sigma (E) = \int_{H (q_i, p_i) \leq E} d\Omega = \int_{H (q_i, p_i) \leq E} \frac{\prod_i d^d q_i d^d p_i}{h^{dN} \xi_N} = \frac{1}{h^{dN} \xi_N} \int_{H (q_i, p_i) \leq E} \prod_i d^d q_i d^d p_i ~.
        \end{aligned}
        \end{equation*}
        We make a change of variable, similar to~\eqref{cm:symplcoord}, into $x_j$, with $j = 1, \ldots 2dN$, given by
        \begin{equation*}
            p_i = \sqrt{2mE} x_j ~, \quad q_i = \sqrt{\frac{2E}{m\omega^2}} x_{dN + j} ~,
        \end{equation*}
        \begin{equation*}
            d p_i = \sqrt{2mE} dx_j  ~, \quad d q_i = \sqrt{\frac{2E}{m\omega^2}} d x_{dN + j} ~.
        \end{equation*}
        In order to find the domain of integration, we express energy in terms of the new variable and we compute the following inequality
        \begin{equation*}
            H = \sum_i \Big ( \frac{p^2_i}{2m} + \frac{m \omega^2}{2} q_i^2 \Big ) = \sum_j E x_j^2 \leq E ~, \quad \sum_j x^2_j \leq 1 ~.
        \end{equation*}
        Hence, by the volume of a $2dN$-sphere of radius $1$~\eqref{app:volumen}, we find
        \begin{equation*}
        \begin{aligned}
            \Sigma (E) & = \frac{1}{h^{dN} \xi_N} \int_{\sum_i \Big ( \frac{p^2_i}{2m} + \frac{m \omega^2}{2} q_i^2 \Big ) \leq E} \prod_i d^d q_i d^d p_i \\ & = \frac{1}{h^{dN} \xi_N} (2mE)^{dN/2} \Big (\frac{2E}{m\omega^2} \Big )^{dN/2} \underbrace{\int_{\sum_j x^2_j \leq 1} \prod_j d x_j}_{ \frac{\pi^{dN}}{\Gamma (dN + 1)}} \\ & = \frac{1}{\xi_N \Gamma (dN + 1)} \Big (\frac{2 \pi E}{h \omega} \Big )^{dN} ~.
        \end{aligned}
        \end{equation*}
    \end{proof}
    The density state $\omega(E)$ is
    \begin{equation}\label{harmosc}
        \omega (E) = \frac{1}{\xi_N \Gamma (dN)} \Big (\frac{2 \pi}{h \omega} \Big )^{dN} E^{dN-1} ~.
    \end{equation}
    In the case $d = 1$, we obtain 
    \begin{equation*}
        \omega (E) = \frac{1}{\xi_N \Gamma (N)} \Big (\frac{2 \pi}{h \omega} \Big )^{N} E^{N-1} ~.
    \end{equation*}
    \begin{proof}
        By definition~\eqref{cm:denst}, we have
        \begin{equation*}
        \begin{aligned}
            \omega (E) & = \pdv{\Sigma (E)}{E} = \pdv{}{E} \frac{1}{\xi_N dN \Gamma (dN)} \Big (\frac{2 \pi E}{h \omega} \Big )^{dN} = \frac{1}{\xi_N dN \Gamma (dN)} \Big (\frac{2 \pi}{h \omega} \Big )^{dN} \pdv{}{E} E^{dN} \\ & = \frac{1}{\xi_N dN \Gamma (dN)} \Big (\frac{2 \pi}{h \omega} \Big )^{dN} dN E^{dN-1} = \frac{1}{\xi_N \Gamma (dN)} \Big (\frac{2 \pi}{h \omega} \Big )^{dN} E^{dN-1} ~.
        \end{aligned}
        \end{equation*}
    \end{proof}
    The entropy $S$ is
    \begin{equation*}
        \frac{S}{k_B} = \ln \Gamma(E) = \ln \omega(E) + \ln \Delta E = \ln \Sigma(E) + \ln \frac{dN}{E} + \ln \Delta E ~,
    \end{equation*}
    and, in the thermodynamic limit, it becomes
    \begin{equation*}
        S = k_B \begin{cases}
            d N + N d \ln \Big (\frac{2 \pi E }{h \omega d N} \Big) & \textnormal{for distinguishable particles} \\
            (d+1) N + N d \ln \Big ( \frac{1}{N} (\frac{2 \pi E }{h \omega d N} ) \Big ) & \textnormal{for indistinguishable particles} \\
        \end{cases} ~.
    \end{equation*}
    In the case $d = 1$, we obtain 
    \begin{equation*}
        S = k_B \begin{cases}
            N + N \ln \Big (\frac{2 \pi E }{h \omega N} \Big)  & \textnormal{for distinguishable particles} \\
            2N + N \ln \Big ( \frac{1}{N} (\frac{2 \pi E }{h \omega N} ) \Big ) & \textnormal{for indistinguishable particles} \\
        \end{cases} ~.
    \end{equation*}
    \begin{proof}
        By definition~\eqref{mc:s}, using the Stirling approximation~\eqref{app:stirl}, the entropy is
        \begin{equation*}
        \begin{aligned}
            \frac{S}{k_B} & = \ln \Sigma (E) = \ln \frac{1}{\xi_N \Gamma(dN + 1)} \Big ( \frac{2 \pi E}{h \omega} \Big)^{dN} \\ & = - \ln \xi_N - \underbrace{\ln \Gamma (dN + 1)}_{dN \ln (dN) - dN} + d N \ln \Big (\frac{2 \pi E }{h \omega} \Big ) \\ & = - \ln \xi_N - dN \ln (dN) + d N + dN \ln \Big (\frac{2 \pi E }{h \omega} \Big ) \\ & = - \ln \xi_N + d N + d N \ln \Big (\frac{2 \pi E }{h \omega d N} \Big ) ~.
        \end{aligned}
        \end{equation*}
        Now, we treat the distinguishable and indistinguishable case separately. For distinguishable particles $\xi_N = 1$, we find
        \begin{equation*}
            \frac{S}{k_B} = - \underbrace{\ln 1}_0 + d N + d N \ln \Big (\frac{2 \pi E }{h \omega d N} \Big ) = d N + d N \ln \Big (\frac{2 \pi E }{h \omega d N} \Big ) ~.
        \end{equation*}
        For indistinguishable particles $\xi_N = N!$, we find
        \begin{equation*}
        \begin{aligned}
            \frac{S}{k_B} & = - \underbrace{\ln N!}_{N \ln N - N} + d N + d N \ln \Big (\frac{2 \pi E }{h \omega d N} \Big ) \\ & =  - N \ln N + N + d N + d N \ln \Big (\frac{2 \pi E }{h \omega d N} \Big ) \\ & = (d+1) N + d N \ln \Big ( \frac{1}{N} \frac{2 \pi E }{h \omega d N} \Big ) ~.
        \end{aligned}
        \end{equation*}
    \end{proof}

    The internal energy is 
    \begin{equation*}
        E = d N k_B T ~.
    \end{equation*}
    In the case $d = 3$, we obtain 
    \begin{equation*}
        E = 3 N k_B T ~.
    \end{equation*}
    \begin{proof}
        Using the first of~\eqref{td:es:s}, we have
        \begin{equation*}
            \frac{1}{T} = \pdv{S}{E} = k_B dN \pdv{}{E} \ln E = k_B \frac{dN}{E} ~,
        \end{equation*}
        hence, we find
        \begin{equation*}
            E = d N k_B T ~.
        \end{equation*}
    \end{proof}
    The equation of state is  
    \begin{equation*}
        p = 0 ~,
    \end{equation*}
    which is the same for all dimensions.
    \begin{proof}
        By the second of~\eqref{td:es:s}, we have
        \begin{equation*}
            \frac{p}{T} = \pdv{S}{V} = 0 ~,
        \end{equation*}
        hence, we find
        \begin{equation*}
            p = 0 ~.
        \end{equation*}
    \end{proof}

\section{Canonical non-relativistic ideal gas}

    \begin{exercise}
        Consider a non-relativistic ideal, i.e.~non-interacting, gas of $N$ indistinguishable particles with mass $m$, living in an $2dN$-dimensional manifold with a finite volume $V^N$: $\mathcal M^N = V^N \times \mathbb R^{dN}$, with Hamiltonian 
        \begin{equation*}
            H = \sum_{i=1}^{dN} \frac{p^2_i}{2m} ~.
        \end{equation*}
        Find the canonical partition function $Z$, the internal energy $E$, the Helmholtz partition function $F$, the entropy $S$ (and the temperature $T_c$ under which it becomes negative), the equation of state $p = p(V, T, N)$, the chemical potential $\mu$ and the specific heats $C_V$ and $C_p$. Finally, express the same physical quantities in the case $d = 3$.
    \end{exercise}

    In the following, we will use the thermal wavelength 
    \begin{equation*}
        \lambda_T = \sqrt{\frac{\beta h^2}{2 m \pi}} ~.
    \end{equation*}

    The canonical partition function $Z$ is 
    \begin{equation}\label{idcan}
        Z = \frac{V^N}{N! \lambda^{dN}_T} ~.
    \end{equation}
    In the case $d = 3$, we obtain 
    \begin{equation*}
        Z = \frac{V^N}{N! \lambda^{3N}_T} ~.
    \end{equation*}
    \begin{proof}
        By definition~\eqref{c:z}, using the gaussian integral~\eqref{app:gauss}, we have
        \begin{equation*}
        \begin{aligned}
            Z & = \int_{\mathcal M^N} d\Omega ~ \exp(- \beta H (q_i, p_i)) = \int_{\mathcal M^N} \frac{\prod_i d^d q_i d^d p_i}{h^{dN} N!} \exp(- \beta H (q_i, p_i)) \\ & = \frac{1}{h^{dN} N!} \int_{\mathcal M^N} \prod_i d^d q_i d^d p_i ~ \exp(- \beta H (q_i, p_i)) \\ & = \frac{1}{h^{dN} N!} \underbrace{\int_{ V^N} \prod_i d^d q_i}_{V^N} \underbrace{\prod_i \int_{\mathcal M^N} d^d p_i ~ \exp(- \beta \frac{p^2_i}{2m})}_{(\frac{2 m \pi}{\beta})^{dN/2}} \\ & = \frac{V^N}{h^{dN} N!} (\frac{2 m \pi}{\beta})^{dN/2} = \frac{V^N}{ N!} (\frac{2 m \pi}{\beta h^2})^{dN/2} = \frac{V^N}{N! \lambda^{dN}_T} ~.
        \end{aligned}
        \end{equation*}
    \end{proof}
    An useful intermediary formula is 
    \begin{equation*}
        \ln Z = N (1 - \ln (n \lambda_T^d)) ~.
    \end{equation*}
    \begin{proof}
        In fact, using the Stirling approximation~\eqref{app:stirl}, we have
        \begin{equation*}
        \begin{aligned}
            \ln Z & = \ln \frac{V^N}{N! \lambda^{dN}_T}= N \ln (V \lambda_T^d) - \underbrace{\ln N!}_{N \ln N - N} = N - N \frac{V \lambda_T^d}{N} \\ & = N (1 - \ln (\frac{N}{V} \lambda_T^d)) = N (1 - \ln (n \lambda_T^d))  ~.
        \end{aligned}
        \end{equation*}
    \end{proof}
    The internal energy $E$ is 
    \begin{equation*}
        E = \frac{d}{2} N k_B T ~.
    \end{equation*}
    In the case $d=3$, we obtain 
    \begin{equation*}
        E = \frac{3}{2} N k_B T ~.
    \end{equation*}
    \begin{proof}
        Using~\eqref{c:e2}, we have
        \begin{equation*}
        \begin{aligned}
            E & = - \pdv{\ln Z}\beta  = - \pdv{}{\beta} N (1 - \ln (n \lambda_T^d)) = - Nd \pdv{}{\beta} \ln (\lambda_T) \\ & = - Nd \pdv{}{\beta} \ln (\beta^{1/2}) = \frac{Nd}{2} \frac{1}{\beta} = \frac{d}{2} N k_B T ~.
        \end{aligned}
        \end{equation*}
    \end{proof}
    The Helmholtz free energy $F$ is 
    \begin{equation*}
        F = \frac{N}{\beta} (\ln (n \lambda_T^d) - 1) ~.
    \end{equation*}
    In the case $d=3$, we obtain 
    \begin{equation*}
        F = \frac{N}{\beta} (\ln (n \lambda_T^3) - 1) ~.
    \end{equation*}
    \begin{proof}
        Using~\eqref{c:f}, we have
        \begin{equation*}
            F = - \frac{\ln Z}{\beta} = \frac{N}{\beta} (\ln (n \lambda_T^d) - 1) ~.
        \end{equation*}
    \end{proof}
    The entropy $S$ is 
    \begin{equation*}
        S = N k_B \Big ( \frac{d+2}{2} - \ln (n \lambda_T^d) \Big ) ~.
    \end{equation*}
    In the case $d=3$, we obtain 
    \begin{equation*}
        S = N k_B \Big ( \frac{5}{2} - \ln (n \lambda_T^3) \Big ) ~.
    \end{equation*}
    \begin{proof}
        Using~\eqref{c:s}, we have
        \begin{equation*}
        \begin{aligned}
            S & = \frac{E - F}{T} = \frac{1}{T} \Big ( \frac{d}{2} N k_B T - \frac{N}{\beta} (\ln (n \lambda_T^d) - 1) \Big ) \\ & = \frac{N}{\beta T} \Big ( \frac{d+2}{2} - \ln (n \lambda_T^d) \Big )  = N k_B \Big ( \frac{d+2}{2} - \ln (n \lambda_T^d) \Big )
        \end{aligned}
        \end{equation*}
    \end{proof}
    Entropy becomes negative at a certain critical temperature
    \begin{equation}\label{ex:negs1}
        T_c = \frac{2 m \pi k_B}{h^2} e^{(d+2)/2} n^{-2/d} ~.
    \end{equation}
    In the case $d=3$, we obtain 
    \begin{equation*}
        T_c = \frac{2 m \pi k_B}{h^2} e^{5/2} n^{-2/3} ~.
    \end{equation*}
    A plot of the entropy as a function of $T$ is in Figure~\ref{fig:c:ent}.
    \begin{figure}
        \centering
        \scalebox{0.7}{\pyc{plot1('x', '5/2 - log(1 / x**(3/2))', 1, 3, 0, True, False, False)}}
        \caption{A plot of the entropy $S$ as a function of $T$. We have used $x = \frac{2 \pi m k_B T n^{2/3}}{h^2}$ and $f(x) = \frac{S}{N k_B}$.}
        \label{fig:c:ent}
    \end{figure}
    \begin{proof}
        In fact, after a series of manipulations, $S < 0$ for 
        \begin{equation*}
            N k_B \Big ( \frac{d+2}{2} - \ln (n \lambda_T^d) \Big ) < 0 ~, \quad \frac{d+2}{2} - \ln (n \lambda_T^d) < 0 ~, \quad \frac{d+2}{2} < \ln (n \lambda_T^d) ~,
        \end{equation*}
        \begin{equation*}
            e^{(d+2)/2} < n \lambda_T^d  ~, \quad e^{(d+2)/2} < n \Big ( \frac{h^2 \beta}{2 m \pi} \Big )^{d/2}  ~, \quad e^{(d+2)/d} n^{2/d} < \frac{h^2 \beta}{2 m \pi} ~,
        \end{equation*}
        \begin{equation*}
            \frac{2 m \pi}{h^2} e^{(d+2)/2} n^{-2/d} < \beta ~,
        \end{equation*}
        hence, we find
        \begin{equation*}
            T < \frac{2 m \pi k_B}{h^2} e^{(d+2)/2} n^{-2/d} = T_c ~.
        \end{equation*}
    \end{proof}
    The equation of state is 
    \begin{equation}\label{ides}
        p V = N k_B T ~.
    \end{equation}
    which is the same for all dimensions.
    \begin{proof}
        By the second of~\eqref{td:es:f}, we have
        \begin{equation*}
            p = - \pdv{F}{V} = - \pdv{}{V} \frac{N}{\beta} (\ln (n \lambda_T^d) - 1) = \frac{N}{\beta} \pdv{}{V} \ln V = \frac{N}{V \beta} ~,
        \end{equation*}
        hence, we find
        \begin{equation*}
            p V = N k_B T ~.
        \end{equation*}
    \end{proof}
    The chemical potential $\mu$ is 
    \begin{equation*}
        \mu = \frac{1}{\beta} \ln (n \lambda_T^d) ~.
    \end{equation*}
    In the case $d=3$, we obtain 
    \begin{equation*}
        \mu = \frac{1}{\beta} \ln (n \lambda_T^3) ~.
    \end{equation*}
    A plot of the chemical potential as a function of $T$ is in Figure~\ref{fig:c:chem}.
    \begin{figure}
        \centering
        \scalebox{0.7}{\pyc{plot1('x', 'x * log(1 / x**(3/2)) ', 2, 2, 1, True, False, False)}}
        \caption{A plot of the chemical potential $\mu$ as a function of $T$. We have used $x = \frac{2 \pi m k_B T n^{2/3}}{h^2}$ and $f(x) = \frac{2 \pi m \mu}{h^2 n^{3/2}}$.}
        \label{fig:c:chem}
    \end{figure}
    \begin{proof}
        By the third of~\eqref{td:es:f}, we have
        \begin{equation*}
            \mu = \pdv{F}{N} = \pdv{}{N} \frac{N}{\beta} (\ln (n \lambda_T^d) - 1) = \frac{1}{\beta} (\ln (n \lambda_T^d) - 1) + \frac{1}{\beta} = \frac{1}{\beta} \ln (n \lambda_T^d) ~.
        \end{equation*}
    \end{proof}
    The specific heats $C_V$ and $C_p$ are 
    \begin{equation*}
        C_V = N \frac{d}{2} k_B ~, \quad C_p = N \frac{d+2}{2} k_B ~. 
    \end{equation*}
    In the case $d=3$, we obtain 
    \begin{equation*}
        C_V = N \frac{3}{2} k_B ~, \quad C_p = N \frac{5}{2} k_B ~. 
    \end{equation*}
    \begin{proof}
        At $V$ constant, by definition~\eqref{td:cv2}, we find
        \begin{equation*}
            C_V = \pdv{E}{T} = \pdv{}{T} \frac{d}{2} N k_B T = N \frac{d}{2} k_B ~.
        \end{equation*}
        At $p$ constant, using~\eqref{td:cp2} and~\eqref{ides}, we find
        \begin{equation*}
            C_p = C_V + p \pdv{V}{T} = C_V + p \pdv{}{T} \frac{N k_B T}{p} = N \frac{d}{2} k_B + N k_B = \frac{d + 2}{2} k_B ~.
        \end{equation*}
    \end{proof}
    Notice that there are two problems: entropy cannot be negative and the specific heat $C_V \rightarrow 0$ for $T \rightarrow 0$, by thermodynamics. This means that this model is not correct and we must go quantum.

\section{Canonical gas of harmonic oscillators}

    \begin{exercise}
        Consider a non-relativistic (non-interacting) gas of $N$ distinguishable particles in an $d$-dimensional manifold confined by an harmonic potential of frequency $\omega$, with Hamiltonian 
        \begin{equation*}
            H = \sum_i \Big ( \frac{p^2_i}{2m} + \frac{m \omega^2}{2} q_i^2 \Big ) ~.
        \end{equation*}
        Find the canonical partition function $Z$, the internal energy $E$, the Helmholtz partition function $F$, the entropy $S$ (and the temperature $T_c$ under which it becomes negative), the equation of state, the chemical potential $\mu$ and the specific heats $C_V$ and $C_p$. Finally, express the same physical quantities in the case $d = 1$.
    \end{exercise}

    The canonical partition function $Z$ is 
    \begin{equation}\label{harmz}
        Z = \Big (\frac{1}{\hbar \omega \beta} \Big )^{dN} ~.
    \end{equation}
    In the case $d = 1$, we obtain
    \begin{equation*}
        Z = \Big (\frac{1}{\hbar \omega \beta} \Big )^{N} ~.
    \end{equation*}
    \begin{proof}
        By definition~\eqref{c:z}, using the gaussian integral~\eqref{app:gauss}, we have
        \begin{equation*}
        \begin{aligned}
            Z & = \int_{\mathcal M^N} d\Omega \exp(- \beta H (q_i, p_i)) = \int_{\mathcal M^N} \frac{\prod_i d^d q_i d^d p_i}{h^{dN} } \exp(- \beta H (q_i, p_i)) \\ & = \frac{1}{h^{dN} } \int_{\mathcal M^N} \prod_i d^d q_i d^d p_i \exp(- \beta H (q_i, p_i)) \\ & = \frac{1}{h^{dN} } \underbrace{\int_{ V^N} \prod_i d^d q_i \exp(- \beta \frac{m \omega^2}{2} q_i^2)}_{(\frac{2 \pi}{m \omega \beta})^{dN/2}} \underbrace{\prod_i \int_{\mathcal M^N} d^d p_i \exp(- \beta \frac{p^2_i}{2m})}_{(\frac{2 m \pi}{\beta})^{dN/2}} \\ & = \frac{1}{h^{dN} } (\frac{2 \pi}{m \omega \beta})^{dN/2} (\frac{2 m \pi}{\beta})^{dN/2} = \Big (\frac{2\pi}{h \omega \beta} \Big )^{dN} = \Big (\frac{1}{\hbar \omega \beta} \Big )^{dN}~.
        \end{aligned}
        \end{equation*}
    \end{proof}
    An useful intermediary formula is 
    \begin{equation*}
        \ln Z = - d N \ln (\hbar \omega \beta) ~.
    \end{equation*}
    \begin{proof}
        In fact, using the Stirling approximation~\eqref{app:stirl}, we have
        \begin{equation*}
            \ln Z = - \ln (\frac{}{\hbar \omega \beta})^{dN} = - d N \ln (\hbar \omega \beta)  ~.
        \end{equation*}
    \end{proof}
    The internal energy $E$ is 
    \begin{equation*}
        E = d N k_B T ~.
    \end{equation*}
    In the case $d = 1$, we obtain 
    \begin{equation*}
        E = N k_B T~.
    \end{equation*}
    \begin{proof}
        Using~\eqref{c:e2}, we have
        \begin{equation*}
            E = - \pdv{\ln Z}{\beta} = \pdv{}{\beta} d N \ln (\hbar \omega \beta) = d N \frac{1}{\beta} = d N k_B T ~.
        \end{equation*}
    \end{proof}
    The Helmholtz free energy $F$ is 
    \begin{equation*}
        F = \frac{dN}{\beta} \ln (\hbar \omega \beta) ~.
    \end{equation*}
    In the case $d = 1$, we obtain 
    \begin{equation*}
        F = \frac{N}{\beta} \ln (\hbar \omega \beta) ~.
    \end{equation*}
    \begin{proof}
        Using~\eqref{c:f}, we have
        \begin{equation*}
            F = - \frac{\ln Z}{\beta} = \frac{dN}{\beta} \ln (\hbar \omega \beta) ~.
        \end{equation*}
    \end{proof}
    The entropy $S$ is 
    \begin{equation*}
        S = d N k_B (1 - \ln (\hbar \omega \beta)) ~.
    \end{equation*}
    In the case $d=1$, we obtain 
    \begin{equation*}
        S = N k_B (1 - \ln (\hbar \omega \beta)) ~.
    \end{equation*}
    \begin{proof}
        Using~\eqref{c:s}, we have
        \begin{equation*}
            S = \frac{E - F}{T} = \frac{1}{T} \Big ( d N k_B T - \frac{dN}{\beta} \ln (\hbar \omega \beta) \Big ) = d N k_B (1 - \ln (\hbar \omega \beta)) ~.
        \end{equation*}
    \end{proof}
    Entropy becomes negative at a certain critical temperature
    \begin{equation}\label{ex:negs2}
        T_c = \frac{\hbar \omega}{k_B e} ~.
    \end{equation}
    which is the same for all dimensions.
    A plot of the entropy $S$ as a function of $T$ is in Figure~\ref{fig:c:ent2}.
    \begin{figure}
        \centering
        \scalebox{0.7}{\pyc{plot1('x', '1 - log(1 / x)', 2, 3, 2, True, False, False)}}
        \caption{A plot of the entropy $S$ as a function of $T$. We have used $x = \frac{2 \pi k_B T}{h \omega}$ and $f(x) = \frac{S}{N k_B}$.}
        \label{fig:c:ent2}
    \end{figure}
    \begin{proof}
        In fact, after a series of manipulations, $S < 0$ for 
        \begin{equation*}
            d N k_B (1 - \ln (\hbar \omega \beta)) < 0 ~, \quad 1 - \ln (\hbar \omega \beta) < 0 ~,
        \end{equation*}
        \begin{equation*}
            1 < \ln (\hbar \omega \beta) ~, \quad e < \hbar \omega \beta = \frac{\hbar \omega}{k_B T}  ~,
        \end{equation*}
        hence, we find
        \begin{equation*}
            T < \frac{\hbar \omega}{k_B e} = T_c ~.
        \end{equation*}
    \end{proof}
    The equation of state is 
    \begin{equation}\label{idesharm}
        p = 0 ~.
    \end{equation}
    which is the same for all dimensions.
    \begin{proof}
        By the second of~\eqref{td:es:f}, we have
        \begin{equation*}
            p = - \pdv{F}{V} = 0 ~.
        \end{equation*}
    \end{proof}
    The chemical potential $\mu$ is 
    \begin{equation*}
        \mu = \frac{d}{\beta} \ln (\hbar \omega \beta) ~.
    \end{equation*}
    In the case $d=1$, we obtain
    \begin{equation*}
        \mu = \frac{1}{\beta} \ln (\hbar \omega \beta) ~.
    \end{equation*}
    A plot of this the chemical potential $\mu$ as a function of $T$ is in Figure~\ref{fig:c:mu2}.
    \begin{figure}
        \centering
        \scalebox{0.7}{\pyc{plot1('x', 'x * log(1 / x)', 3, 3, 3, True, False, False)}}
        \caption{A plot of the chemical potential $\mu$ as a function of $T$. We have used $x = \frac{2 \pi k_B T}{h \omega}$ and $f(x) = \frac{2 \pi \mu}{h \omega}$.}
        \label{fig:c:mu2}
    \end{figure}
    \begin{proof}
        By the third of~\eqref{td:es:f}, we have
        \begin{equation*}
            \mu = \pdv{F}{N} = \pdv{}{N} \frac{dN}{\beta} \ln (\hbar \omega \beta) = \frac{d}{\beta} \ln (\hbar \omega \beta) ~.
        \end{equation*}
    \end{proof}
    The specific heats $C_V$ and $C_p$ are 
    \begin{equation*}
        C_V = d N k_B  ~, \quad C_p = d N k_B ~. 
    \end{equation*}
    \begin{proof}
        At $V$ constant, by definition~\eqref{td:cv2}, we find
        \begin{equation*}
            C_V = \pdv{E}{T} = \pdv{}{T} d N k_B T  = d N k_B ~.
        \end{equation*}
        At $p$ constant, using~\eqref{td:cp2} and~\eqref{idesharm}, we find
        \begin{equation*}
            C_p = C_V + p \pdv{V}{T} = C_V = d N k_B ~.
        \end{equation*}
    \end{proof}

    Notice that also here there are two problems: entropy cannot be negative and the specific heat $C_V \rightarrow 0$ for $T \rightarrow 0$, by thermodynamics. This means that this model is not correct and we must go quantum.

\section{Grand canonical non-relativistic ideal gas}

    \begin{exercise}
        Consider a non-relativistic ideal (non-interacting) gas of $N$ indistinguishable particles with mass $m$, living in an $d$-dimensional manifold with a finite volume $V^N$: $\mathcal M^N = V^N \times \mathbb R^{dN}$, with Hamiltonian 
        \begin{equation*}
            H = \sum_i \frac{p^2_i}{2m} ~.
        \end{equation*}
        Find the grand canonical partition function $Z$, the internal energy $E$, the number of particles $N$ and the equation of state.
    \end{exercise}

    The grancanonical partition function $\mathcal Z$ is 
    \begin{equation*}
        \mathcal Z = \exp(\frac{z V}{\lambda_T^d}) ~.
    \end{equation*}
    \begin{proof}
        By definition~\eqref{gc:z}, using~\eqref{idcan} and the Taylor expansion of the exponential, we have
        \begin{equation*}
            \mathcal Z = \sum_{N=0}^\infty z^N Z_N = \sum_{N=0}^\infty \frac{1}{N!} \Big ( \frac{z V}{\lambda_T^d} \Big)^N = \exp(\frac{z V}{\lambda_T^d}) ~.
        \end{equation*}
    \end{proof}
    The internal energy $E$ is 
    \begin{equation*}
        E = \frac{z V}{\lambda^d_T} \frac{d}{2 \beta} ~.
    \end{equation*}
    \begin{proof}
        Using~\eqref{gc:e2}, we have
        \begin{equation*}
        \begin{aligned}
            E & = - \pdv{\ln \mathcal Z}{\beta} \Big \vert_z = - \pdv{}{\beta} \ln \exp(\frac{z V}{\lambda_T^d}) = - \pdv{}{\beta} \frac{z V}{\lambda^d_T} \\ & = - \frac{1}{zV} \pdv{}{\beta} \Big ( \frac{2 m \pi} {\beta h^2} \Big )^{d/2} = - \frac{1}{z V} \Big ( \frac{2 m \pi}{h^2} \Big )^{d/2} \pdv{}{\beta} \beta^{-d/2} \\ & = \frac{1}{zV} \Big ( \frac{h^2}{2 m \pi} \Big )^{d/2} \frac{d}{2} \beta^{-d/2 - 1} = \frac{z V}{\lambda^d_T} \frac{d}{2 \beta} ~.
        \end{aligned}
        \end{equation*}
    \end{proof}
    The number of particle $N$ is 
    \begin{equation*}
        N = \frac{zV}{\lambda_T^d} ~.
    \end{equation*}
    \begin{proof}
        Using~\eqref{gc:n2}, we have
        \begin{equation*}
            N = z \pdv{}{z} \ln \mathcal Z = z \pdv{}{z} \frac{z V}{\lambda_T^d} = \frac{zV}{\lambda_T^d} ~.
        \end{equation*}
    \end{proof}
    The equation of state is 
    \begin{equation*}
        p = \frac{z}{\beta \lambda_T^d} ~.
    \end{equation*}
    \begin{proof}
        By the second of~\eqref{td:es:o} and~\eqref{gc:o}, we have
        \begin{equation*}
            p = - \pdv{\Omega}{V} = - \frac{1}{\beta V} \ln \mathcal Z = \frac{1}{\beta V} \frac{z V}{\lambda_T^d} = \frac{z}{\beta \lambda_T^d} ~.
        \end{equation*}
    \end{proof}

\section{Grand canonical gas of harmonic oscillators}

    \begin{exercise}
        Consider a non-relativistic (non-interacting) gas of $N$ distinguishable particles in an $d$-dimensional manifold confined by an harmonic potential of frequency $\omega$, with Hamiltonian 
        \begin{equation*}
            H = \sum_i \Big ( \frac{p^2_i}{2m} + \frac{m \omega^2}{2} q_i^2 \Big ) ~.
        \end{equation*}
        Find the grand canonical partition function $Z$, the internal energy $E$, the number of particles $N$ and the equation of state.
    \end{exercise}

    The grancanonical partition function $\mathcal Z$ is 
    \begin{equation*}
        \mathcal Z = \frac{1}{1 - \frac{z}{\hbar \beta \omega}} ~.
    \end{equation*}
    Notice that this series converges only if $\hbar \beta \omega \ll 1$, which is the high temperature limit.
    \begin{proof}
        By definition~\eqref{gc:z}, using~\eqref{harmz} and the geometrical series, we have
        \begin{equation*}
            \mathcal Z = \sum_{N=0}^\infty z^N Z_N = \sum_{N=0}^\infty \Big ( \frac{z}{\hbar \beta \omega} \Big) = \frac{1}{1 - \frac{z}{\hbar \beta \omega}}  ~.
        \end{equation*}
    \end{proof}
    The internal energy $E$ is 
    \begin{equation*}
        E = \frac{z}{\beta (\hbar \omega \beta - z)} ~.
    \end{equation*}
    \begin{proof}
        Using~\eqref{gc:e2}, we have
        \begin{equation*}
        \begin{aligned}
            E & = - \pdv{\ln \mathcal Z}{\beta} \Big \vert_z = - \pdv{}{\beta} \ln \frac{1}{1 - \frac{z}{\hbar \beta \omega}} = \pdv{}{\beta} \ln (1 - \frac{z}{\hbar \beta \omega} ) \\ & = \frac{1}{1 - \frac{z}{\hbar \beta \omega}} \frac{z}{\hbar \omega \beta^2} = \frac{z}{\beta (\hbar \omega \beta - z)} ~.
        \end{aligned}
        \end{equation*}
    \end{proof}
    The number of particle $N$ is 
    \begin{equation*}
        N = \frac{z}{\hbar \omega \beta - z} ~.
    \end{equation*}
    \begin{proof}
        Using~\eqref{gc:n2}, we have
        \begin{equation*}
            N = z \pdv{}{z} \ln \mathcal Z = z \pdv{}{z} \ln \frac{1}{1 - \frac{z}{\hbar \beta \omega}} = - z \pdv{}{z} \ln (1 - \frac{z}{\hbar \beta \omega}) = \frac{1}{1 - \frac{z}{\hbar \beta \omega}}\frac{z}{\hbar \omega \beta} = \frac{z}{\hbar \omega \beta - z}~.
        \end{equation*}
    \end{proof}
    The equation of state is 
    \begin{equation*}
        p = - \frac{1}{\beta V}\ln (1 - \frac{z}{\hbar \beta \omega}) ~.
    \end{equation*}
    \begin{proof}
        By the second of~\eqref{td:es:o} and~\eqref{gc:o}, we have
        \begin{equation*}
            p = - \pdv{\Omega}{V} = - \frac{1}{\beta V} \ln \mathcal Z = \frac{1}{\beta V} \ln (1 - \frac{z}{\hbar \beta \omega}) ~.
        \end{equation*}
    \end{proof}

\section{Canonical ultra-relativistic ideal gas}

    \begin{exercise}
        Consider an ultra-relativistic ($p_i c \gg m c^2$) ideal (non-interacting) gas of $N$ indistinguishable particles with mass $m$, living in an $3$-dimensional manifold with a finite volume $V^N$: $\mathcal M^N = V^N \times \mathbb R^{3N}$, with Hamiltonian
        \begin{equation*}
            H = \sum_i \sqrt{m^2 c^4 + p_i^2 c^2} \xrightarrow{p_i c \gg m c^2} \sum_i c |p_i| ~.
        \end{equation*}
        Find the canonical partition function $Z$, the internal energy $E$, the Helmholtz partition function $F$, the entropy $S$ (and the temperature $T_c$ under which it becomes negative), the equation of state, the chemical potential $\mu$ and the specific heats $C_V$ and $C_p$. 
    \end{exercise}

    The canonical partition function $Z$ is 
    \begin{equation*}
        Z = \frac{1}{N!} \Big (\frac{8\pi V}{(\beta h c)^3} \Big )^N ~.
    \end{equation*}
    \begin{proof}
        By definition~\eqref{c:z}, using the gaussian integral~\eqref{app:gauss}, we have
        \begin{equation*}
        \begin{aligned}
            Z & = \int_{\mathcal M^N} d\Omega \exp(- \beta H (q_i, p_i)) = \int_{\mathcal M^N} \frac{\prod_i d^3 q_i d^3 p_i}{h^{3N} N!} \exp(- \beta H (q_i, p_i)) \\ & = \frac{1}{h^{3N} N!} \int_{\mathcal M^N} \prod_i d^3 q_i d^3 p_i \exp(- \beta H (q_i, p_i)) \\ & = \frac{1}{h^{3N} N!} \underbrace{\int_{ V^N} \prod_i d^d q_i}_{V^N} \prod_i \int_{\mathcal M^N} d^d p_i \exp(- \beta c p_i)  = \frac{V^N}{h^{3N} N!} \prod_i \int_{\mathcal M^N} d^d p_i \exp(- \beta c p_i) ~.
        \end{aligned}
        \end{equation*}
        Now, in order to evaluate the integral, we use the polar coordinates in the momentum space $(p, \theta, \phi) = (p, \Omega)$ 
        \begin{equation*}
            \prod_i \int_{\mathcal M^N} d^3 p_i \exp(- \beta c p_i) = \prod_i 4 \pi \int_0^\infty dp ~ p^2 \exp(- \beta c p_i)
        \end{equation*}
        where $\int d\Omega = 4\pi$. We make a change of variables into
        \begin{equation*}
            z = \beta c p ~, \quad dz = - \beta c dp ~,
        \end{equation*}
        and we obtain 
        \begin{equation*}
            \prod_i \frac{4\pi}{(\beta c)^3} \underbrace{\int_0^\infty dz ~ z^2 \exp(- z)}_{\Gamma (3)} = \prod_i \frac{4\pi}{(\beta c)^3} \underbrace{\Gamma (3)}_2 = \prod_i \frac{8\pi}{(\beta c)^3} = \Big (\frac{8\pi}{(\beta c)^3} \Big )^N ~.
        \end{equation*}
        Hence, we find 
        \begin{equation*}
            Z = \frac{V^N}{h^{3N} N!} \Big (\frac{8\pi}{(\beta c)^3} \Big )^N = \frac{1}{N!} \Big (\frac{8\pi V}{(\beta h c)^3} \Big )^N ~.
        \end{equation*}
    \end{proof}

    An useful intermediary formula is 
    \begin{equation*}
        \ln Z = N (1 - \ln \frac{n (\beta h c)^3}{8\pi}) ~.
    \end{equation*}
    \begin{proof}
        In fact, using the Stirling approximation~\eqref{app:stirl}, we have
        \begin{equation*}
        \begin{aligned}
            \ln Z & = \ln \frac{1}{N!} \Big (\frac{8\pi V}{(\beta h c)^3} \Big )^N = - \underbrace{\ln N!}_{N \ln N - N} + N \ln \frac{8\pi V}{(\beta h c)^3} \\ & = N (1 - \ln \frac{N (\beta h c)^3}{8\pi V}) = N (1 - \ln \frac{n (\beta h c)^3}{8\pi}) ~.
        \end{aligned}
        \end{equation*}
    \end{proof}
    
    The internal energy $E$ is 
    \begin{equation*}
        E = 3 N k_B T ~.
    \end{equation*}
    \begin{proof}
        Using~\eqref{c:e2}
        \begin{equation*}
            E = - \pdv{\ln Z}{\beta} = - \pdv{}{\beta} N (1 - \ln \frac{n (\beta h c)^3}{8\pi}) = N \pdv{}{\beta} \ln (\beta^3) = 3 N \frac{\beta^2}{\beta^3} = 3 N \frac{1}{\beta} = 3 N k_B T ~.
        \end{equation*}

        As an aside, it can be also derived from the generalised equipartion theorem~\eqref{equi:thm}. In fact, we have
        \begin{equation*}
            k_B T = \av{p_i \pdv{H}{p_i}} = \av{p_i \pdv{}{p_i} c \sqrt{p_1^2 + p_2^2 + p_3^2}} = \av{c \frac{p_i^2}{\sqrt{p_1^2 + p_2^2 + p_3^2}}} ~,
        \end{equation*}
        hence, we find
        \begin{equation*}
            \av{H} = \av{c \frac{p_1^2 + p_2^2 + p_3^2}{\sqrt{p_1^2 + p_2^2 + p_3^2}}} = \sum_{i=1}^3 \underbrace{\av{c \frac{p_i^2}{\sqrt{p_1^2 + p_2^2 + p_3^2}}}}_{k_B T} = 3 k_B T ~.
        \end{equation*}
    \end{proof}
    
    The Helmholtz free energy $F$ is 
    \begin{equation*}
        F = \frac{N}{\beta} (\ln \frac{n (\beta h c)^3}{8\pi} - 1) ~.
    \end{equation*}
    \begin{proof}
        Using~\eqref{c:f}, we have
        \begin{equation*}
            F = - \frac{\ln Z}{\beta} = \frac{N}{\beta} (\ln \frac{n (\beta h c)^3}{8\pi} - 1) ~.
        \end{equation*}
    \end{proof}
    
    The entropy $S$ is 
    \begin{equation*}
        S = N k_B (4 - \ln \frac{n (\beta h c)^3}{8\pi} ) ~.
    \end{equation*}
    \begin{proof}
        Using~\eqref{c:s}
        \begin{equation*}
        \begin{aligned}
            S & = \frac{E - F}{T} = \frac{1}{T} \Big ( 3 N k_B T - \frac{N}{\beta} (\ln \frac{n (\beta h c)^3}{8\pi} - 1)  \Big ) \\ & = 3 N k_B - N k_B (\ln \frac{n (\beta h c)^3}{8\pi} - 1) = N k_B (4 - \ln \frac{n (\beta h c)^3}{8\pi} ) ~.
        \end{aligned}
        \end{equation*}
    \end{proof}
    Entropy becomes negative at a certain critical temperature
    \begin{equation*}
        T_c = \frac{hc}{k_B} \Big (\frac{n}{8\pi e^4} \Big)^{1/3} ~.
    \end{equation*}
    A plot of the entropy $S$ as a function of $T$ is in Figure~\ref{fig:c:ent3}.
    \begin{figure}
        \centering
        \scalebox{0.7}{\pyc{plot1('x', '4 - log(1 / x**3)', 3, 5, 4, True, False, False)}}
        \caption{A plot of the entropy $S$ as a function of $T$. We have used $x = \frac{(8 \pi)^{1/3} k_B T}{h c n^{1/3}}$ and $f(x) = \frac{S}{N k_B}$.}
        \label{fig:c:ent3}
    \end{figure}
    \begin{proof}
        In fact, after a series of manipulations, $S < 0$ for 
        \begin{equation*}
            N k_B (4 - \ln \frac{n (\beta h c)^3}{8\pi} ) < 0 ~, \quad  4 - \ln \frac{n (\beta h c)^3}{8\pi} < 0 ~, \quad 4 < \ln \frac{n (\beta h c)^3}{8\pi} ~,
        \end{equation*}
        \begin{equation*}
             \quad e^{4} < \frac{n (\beta h c)^3}{8\pi} ~, \quad e^{4} < \frac{n (h c)^3}{8\pi k_B^3 T^3}  ~, \quad T^3 < \frac{n (h c)^3}{8\pi k_B^3 e^4} ~,
        \end{equation*}
        hence, we find
        \begin{equation*}
            T < \frac{hc}{k_B} \Big (\frac{n}{8\pi e^4} \Big)^{1/3} = T_c ~.
        \end{equation*}
    \end{proof}
    
    The equation of state is 
    \begin{equation}\label{idesultra}
        p V = N k_B T ~.
    \end{equation}
    \begin{proof}
        By the second of~\eqref{td:es:f}, we have
        \begin{equation*}
            p = - \pdv{F}{V} = - \pdv{}{V} \frac{N}{\beta} (\ln \frac{n (\beta h c)^3}{8\pi} - 1) = \frac{N}{\beta} \pdv{}{V} \ln V = \frac{N}{V \beta} ~,
        \end{equation*}
        hence, we find
        \begin{equation*}
            p V = N k_B T ~.
        \end{equation*}
    \end{proof}
    
    The chemical potential $\mu$ is 
    \begin{equation*}
        \mu = \frac{1}{\beta} \ln \frac{n (\beta h c)^3}{8\pi} ~.
    \end{equation*}
    A plot of the chemical potential $\mu$ as a function of $T$ is in Figure~\ref{fig:c:mu3}.
    \begin{figure}
        \centering
        \scalebox{0.7}{\pyc{plot1('x', 'x * log(1 / x**3)', 3, 4, 5, True, False, False)}}
        \caption{A plot of the chemical potential $\mu$ as a function of $T$. We have used $x = \frac{(8 \pi)^{1/3} k_B T}{h c n^{1/3}}$ and $f(x) = \frac{(8 \pi)^{1/3} \mu}{h c n^{1/3}}$.}
        \label{fig:c:mu3}
    \end{figure}
    \begin{proof}
        By the third of~\eqref{td:es:f}, we have
        \begin{equation*}
            \mu = \pdv{F}{N} = \pdv{}{N} \frac{N}{\beta} (\ln \frac{n (\beta h c)^3}{8\pi} - 1) = \frac{1}{\beta} (\ln \frac{n (\beta h c)^3}{8\pi} - 1 + 1) = \frac{1}{\beta} \ln \frac{n (\beta h c)^3}{8\pi} ~.
        \end{equation*}
    \end{proof}

    The specific heats $C_V$ and $C_p$ are 
    \begin{equation*}
        C_V = 3 N k_B ~, \quad C_p = 4 N k_B T ~. 
    \end{equation*}
    \begin{proof}
        At $V$ constant, by definition~\eqref{td:cv2}, we find
        \begin{equation*}
            C_V = \pdv{E}{T} = \pdv{}{T} 3 N k_B T = 3 N k_B ~.
        \end{equation*}
        At $p$ constant, using~\eqref{td:cp2} and~\eqref{ides}, we find
        \begin{equation*}
            C_p = C_V + p \pdv{V}{T} = C_V + p \pdv{}{T} \frac{N k_B T}{p} = 3 N k_B + N k_B = 4 N k_B T ~.
        \end{equation*}
    \end{proof}

\section{Canonical magnetic solid}

    \begin{exercise}
        Consider a solid composed by $N$ atoms or molecules with an intrinsic magnetic moment $\boldsymbol \mu$ in an external magnetic field $\mathbf B$. The phase phase is a $2$-dimensional sphere that can be parametrised by $(\theta, phi)$. The conjugate coordinate and momentum are $q = \phi$ and $p = \cos \theta$. The Hamiltonian is 
        \begin{equation*}
            H = - \sum_i \boldsymbol \mu \cdot \mathbf B = - \mu B \sum_i \cos \theta_i ~,
        \end{equation*}
        where we have oriented $\mathbf B = B \mathbf k$. Find the canonical partition function $Z$, the internal energy $E$, the Helmholtz partition function $F$, the entropy $S$ (and the temperature $T_c$ under which it becomes negative), the magnetisation $\mathbf M$ and the isothermal susceptibility $\chi_\beta$. 
    \end{exercise}

    The canonical partition function $Z$ is 
    \begin{equation*}
        Z = \Big (\frac{4 \pi \sinh (\beta \mu B)}{\beta \mu B} \Big )^N ~.
    \end{equation*}
    \begin{proof}
        By definition~\eqref{c:z}, we have
        \begin{equation*}
        \begin{aligned}
            Z & = \int_{\mathcal M} d \Omega \exp(- \beta H (\theta_i)) = \underbrace{\prod_i \int_0^{2\pi} d\phi_i}_{(2\pi)^N} \prod_i \int_0^\pi d\theta_i ~ \sin \theta_i \exp(- \beta \mu B \cos \theta_i) \\ & = (2\pi)^N \prod_i \int_0^\pi d\theta_i ~ \sin \theta_i \exp(- \beta \mu B \cos \theta_i)  ~.
        \end{aligned}
        \end{equation*}
        Now, in order to evaluate the integral, we make a change of variables into
        \begin{equation*}
            x_i  = \cos \theta_i ~, \quad d x_i = - \sin \theta_i d\theta_i ~,
        \end{equation*}
        with extrema 
        \begin{equation*}
            \theta_i = 0 \rightarrow x_i = 1 ~, \quad \theta_i = \pi \rightarrow x_i = 0 ~,
        \end{equation*}
        hence,
        \begin{equation*}
        \begin{aligned}
            Z & = (2\pi)^N \prod_i \int_{-1}^1 dx_i ~ \exp(- \beta \mu B x) = (2\pi)^N \Big ( \frac{\exp(- \beta \mu B x)}{- \beta \mu B} \Big \vert_{-1}^{1} \Big )^N \\ & = (2\pi)^N \Big ( \frac{1}{- \beta \mu B} \underbrace{(\exp(- \beta \mu B) - \exp(\beta \mu B))}_{- 2 \sinh \beta \mu B}\Big )^N \\ & = (2\pi)^N \Big ( \frac{1}{\beta \mu B} (2 \sinh (\beta \mu B)) \Big )^N = \Big (\frac{4 \pi \sinh (\beta \mu B)}{\beta \mu B} \Big )^N ~.
        \end{aligned}
        \end{equation*}
    \end{proof}

    An useful intermediary formula is 
    \begin{equation*}
        \ln Z = N \ln \frac{4 \pi \sinh (\beta \mu B)}{\beta \mu B}  ~.
    \end{equation*}
    \begin{proof}
        In fact, we have
        \begin{equation*}
            \ln Z = \ln \Big (\frac{4 \pi \sinh (\beta \mu B)}{\beta \mu B} \Big )^N = N \ln \frac{4 \pi \sinh (\beta \mu B)}{\beta \mu B}  ~.
        \end{equation*}
    \end{proof}
    
    The internal energy $E$ is 
    \begin{equation*}
        E = - N \mu B (\coth (\beta \mu B) - \frac{1}{\beta \mu B} ) ~.
    \end{equation*}
    A plot of the internal energy $E$ as a function of $T$ is in Figure~\ref{fig:c:magen}.
    \begin{figure}
        \centering
        \scalebox{0.7}{\pyc{plot1('x', '- coth(1 / x) + x', 5, 1.2, 8, True, True, False)}}
        \caption{A plot of the internal energy $E$ as a function of $T$. We have used $x = \frac{1}{\beta \mu B}$ and $f(x) = \frac{E}{N \mu B}$.}
        \label{fig:c:magen}
    \end{figure}
    \begin{proof}
        Using~\eqref{c:e2}, we have
        \begin{equation*}
        \begin{aligned}
            E & = - \pdv{\ln Z}{\beta} = - \pdv{}{\beta} N \ln \frac{4 \pi \sinh (\beta \mu B)}{\beta \mu B} = - N \pdv{}{\beta} \ln \sinh (\beta \mu B) + N \pdv{}{\beta} \ln \beta \\ & = - N \mu B \coth (\beta \mu B) + \frac{N}{\beta} = - N \mu B (\coth (\beta \mu B) - \frac{1}{\beta \mu B} ) ~.
        \end{aligned}
        \end{equation*}

        We compute the limit for $\beta \rightarrow 0$ or $T \rightarrow \infty$
        \begin{equation*}
            \lim_{x \rightarrow 0} \frac{E}{N \mu B} (x) \simeq \py{limit('x', '- 1 * ((exp(x) + exp(-x))/((exp(x) - exp(-x)))) + 1 / x', 0)} ~,
        \end{equation*}
        hence, we obtain
        \begin{equation*}
            E \xrightarrow{T \rightarrow \infty} 0 ~.
        \end{equation*}
        We compute the limit for $\beta \rightarrow \infty$ or $T \rightarrow 0$
        \begin{equation*}
            \lim_{x \rightarrow \infty} \frac{E}{N \mu B} (x) \simeq \py{limit('x', '- coth(1 / x) + x', 0)} ~,
        \end{equation*}
        hence, we obtain
        \begin{equation*}
            E \xrightarrow{T \rightarrow 0} - N \mu B ~.
        \end{equation*}
    \end{proof}
    
    The Helmholtz free energy $F$ is 
    \begin{equation*}
        F = - \frac{N}{\beta} \ln \frac{4 \pi \sinh (\beta \mu B)}{\beta \mu B} ~.
    \end{equation*}
    \begin{proof}
        Using~\eqref{c:f}, we have
        \begin{equation*}
            F = - \frac{\ln Z}{\beta} = - \frac{N}{\beta} \ln \frac{4 \pi \sinh (\beta \mu B)}{\beta \mu B} ~.
        \end{equation*}
    \end{proof}
    
    The entropy $S$ is 
    \begin{equation*}
        S = N k_B \Big ( \ln \frac{4 \pi \sinh (\beta \mu B)}{\beta \mu B}  - \beta \mu B (\coth (\beta \mu B) - \frac{1}{\beta \mu B} ) \Big ) ~.
    \end{equation*}
    \begin{proof}
        Using~\eqref{c:s}, we have
        \begin{equation*}
        \begin{aligned}
            S & = \frac{E - F}{T} = \frac{1}{T} \Big (- N \mu B (\coth (\beta \mu B) + \frac{1}{\beta \mu B} ) + \frac{N}{\beta} \ln \frac{4 \pi \sinh (\beta \mu B)}{\beta \mu B}  \Big ) \\ & = N k_B \Big ( \ln \frac{4 \pi \sinh (\beta \mu B)}{\beta \mu B}  - \beta \mu B (\coth (\beta \mu B) - \frac{1}{\beta \mu B} ) \Big )~. 
        \end{aligned}
        \end{equation*}
    \end{proof}

    An useful intermediary formula is 
    \begin{equation}\label{c:mag}
        \mathbf M = - \pdv{F}{\mathbf B} \Big \vert_{\beta} ~.
    \end{equation}
    \begin{proof}
        In fact, using~\eqref{c:av} and~\eqref{c:f}, we obtain 
        \begin{equation*}
        \begin{aligned}
            \mathbf M & = \sum_i \av{\boldsymbol \mu_i}_c = \int d\Omega ~ \rho_c \sum_i \boldsymbol \mu_i  = \int d\Omega ~ \frac{\exp(\beta \sum_i \boldsymbol \mu_i \cdot \mathbf B)}{Z} \sum_i \boldsymbol \mu_i \\ & = \frac{1}{\beta Z} \pdv{}{\mathbf B} \underbrace{\int d\Omega ~ \exp(\beta \sum_i \boldsymbol \mu_i \cdot \mathbf B)}_Z = \frac{1}{\beta Z} \pdv{Z}{\mathbf B} = \frac{1}{\beta} \pdv{\ln Z}{\mathbf B} = - \pdv{F}{\mathbf B} \Big \vert_{\beta} ~.
        \end{aligned}
        \end{equation*}
    \end{proof}

    The magnetisation $\mathbf M$ is 
    \begin{equation*}
        \mathbf M = (0, 0, N\mu (\coth(\beta \mu B) - \frac{1}{\beta \mu B})) ~.
    \end{equation*}
    A plot of the magnetisation $\mathbf M$ as a function of $\beta$ is in Figure~\ref{fig:c:mag}.
    \begin{figure}
        \centering
        \scalebox{0.7}{\pyc{plot1('x', 'coth(x) - 1 / x', 7, 1, 9, True, True, True)}}
        \caption{A plot of the intrinsic magnetic moment $\mathbf M$ as a function of $\beta$. We have used $x = \beta \mu B$ and $f(x) = \frac{M_z}{N \mu}$.}
        \label{fig:c:mag}
    \end{figure}
    \begin{proof}
        Since we have oriented $\mathbf B = (0, 0, B)$, in the transversal directions we have
        \begin{equation*}
            M_x = - \pdv{F}{B_x} = M_y = - \pdv{F}{B_y} = 0 ~,
        \end{equation*}
        but in the longitudinal direction we find
        \begin{equation*}
        \begin{aligned}
            M_z & = - \pdv{F}{B} = \pdv{}{B} \frac{N}{\beta} \ln \frac{4 \pi \sinh (\beta \mu B)}{\beta \mu B} = \frac{N}{\beta} \pdv{}{\beta} \ln \sinh (\beta \mu B) - \frac{N}{\beta} \pdv{}{B} \ln B \\ & = N \mu \coth(\beta \mu B) - \frac{N}{\beta B} = N \mu \Big (\coth (\beta \mu B) - \frac{1}{\beta \mu B} \Big ) ~.
        \end{aligned}
        \end{equation*}
    \end{proof}
    
    The isothermal susceptibility $\chi_\beta$, defined as 
    \begin{equation*}
        \chi_\beta = \pdv{M}{B} \Big \vert_\beta ~,
    \end{equation*}  
    is 
    \begin{equation*}
        \chi_\beta = N \mu^2 \beta ( \frac{1}{(\beta \mu B)^2} - \frac{1}{\sinh^2 (\beta \mu H)}) ~.
    \end{equation*}
    A plot of the isothermal susceptibility $\chi_\beta$, as a function of $\beta$ is in Figure~\ref{fig:c:sus}.
    \begin{figure}
        \centering
        \scalebox{0.7}{\pyc{plot1('x', 'x * (1 / x**2 - 1 / sinh(x)**2)', 5, 0.5, 10, True, True, True)}}
        \caption{A plot of the isothermal susceptibility $\chi_\beta$, as a function of $\beta$. We have used $x = \beta \mu B$ and $f(x) = \frac{B \chi_\beta}{N \mu}$.}
        \label{fig:c:sus}
    \end{figure}
    \begin{proof}
        By definition, we have
        \begin{equation*}
        \begin{aligned}
            \chi_\beta & = \pdv{M}{B} = \pdv{}{B} N\mu (\coth(\beta \mu B) - \frac{1}{\beta \mu B}) = N \mu ( - \frac{\beta \mu}{\sinh^2 (\beta \mu B)} + \frac{\beta \mu}{(\beta \mu B)^2} ) \\ & = N \mu^2 \beta ( \frac{1}{(\beta \mu B)^2} - \frac{1}{\sinh^2 (\beta \mu H)}) ~.
        \end{aligned}
        \end{equation*}
    \end{proof}
    
    For $T \rightarrow \infty$, we can recover the Curie law 
    \begin{equation*}
        \chi_\beta = \frac{C}{T} ~,
    \end{equation*}
    where the Curie constant is 
    \begin{equation*}
        C = \frac{N \mu^2}{3 k_B} ~.
    \end{equation*}
    \begin{proof}
        To study the limit for $\beta \rightarrow 0$ or $T \rightarrow \infty$, we Taylor expand for the variable $x = \beta \mu B$
        \begin{equation*}
            \frac{B \chi_\beta}{N \mu} (x) \simeq \py{Taylor('x', 'x * ( x**(-2) - sinh(x)**(-2))', 0, 2)} ~,
        \end{equation*}
        hence, we find
        \begin{equation*}
            \frac{B \chi_\beta}{N \mu} = \frac{\beta \mu B}{3} ~,
        \end{equation*}
        which means that
        \begin{equation*}
            \chi_\beta = \frac{N \mu^2}{3 k_B} \frac{1}{T} = \frac{C}{T} ~.
        \end{equation*}
    \end{proof}

\section{Canonical Maxwell-Boltzmann distribution}

    \begin{exercise}
        Consider a non-relativistic ideal (non-interacting) gas of $N$ particles in an $3$-dimensional manifold $\mathcal M^N = \mathbb R^6$ confined into a potential $V(q_i)$, with Hamiltonian
        \begin{equation*}
            H = \sum_i \Big ( \frac{p^2_i}{2m} + V(q_i) \Big ) ~.
        \end{equation*}
        Find the probability distribution density $\rho_c$, the marginal $\rho(q^i)$, the momentum $\rho(p_i)$, the velocity $\rho(v_i$) and the square velocity $\rho(v)$ ones. Furthermore, compute the most probable velocity $v_p$, the mean velocity $\av{v}$ and the e mean square velocity $\av{v^2}$. Put $h = 1$. 
    \end{exercise}

    The probability distribution density $\rho_c$ for each particle is 
    \begin{equation*}
        \rho_c (q_i, p_i) = \frac{\exp (- \beta ( \frac{p^2_i}{2m} + V(q_i) ))}{(\frac{2\pi m}{\beta})^{3/2} \int_{\mathbb R^3} d^3 q \exp(- \beta V(q))} ~.
    \end{equation*}
    \begin{proof}
        By definition~\eqref{c:pdd}, we have
        \begin{equation*}
            \rho_c (q_i, p_i) = \mathcal N \exp (- \beta ( \frac{p^2_i}{2m} + V(q_i) )) ~,
        \end{equation*}
        where, using the gaussian integral~\eqref{app:gauss}, the normalisation constant is  
        \begin{equation*}
        \begin{aligned}
            1 & = \int_{\mathbb R^6} \prod_i d^3 q ~ d^3 p \mathcal N \exp(- \beta ( \frac{p^2}{2m} + V(q))) \\ & = \mathcal N \int_{\mathbb R^3} d^3 q ~ \exp(- \beta V(q))  \underbrace{\int_{\mathbb R^3} d^3 p ~ \exp(- \beta \frac{p^2}{2m})}_{\Big ( \frac{2\pi m}{\beta}\Big)^{3/2}} \\ & = \mathcal N  \Big ( \frac{2\pi m}{\beta}\Big)^{3/2} \int_{\mathbb R^3} d^3 q ~ \exp(- \beta V(q)) ~,
        \end{aligned}
        \end{equation*}
        hence, we find
        \begin{equation*}
            \mathcal N = \Big ( \Big ( \frac{2\pi m}{\beta}\Big)^{3/2} \int_{\mathbb R^3} d^3 q ~ \exp(- \beta V(q))  \Big )^{-1} ~.
        \end{equation*}
    \end{proof}

    The marginal probability density distribution is 
    \begin{equation*}
        \rho(q_i) = \frac{\exp (- \beta V(q_i) )}{\int_{\mathbb R^3} d^3 q \exp(- \beta V(q))} ~.
    \end{equation*}
    \begin{proof}
        By definition, we have
        \begin{equation*}
        \begin{aligned}
            \rho(q_i) & = \int_{\mathbb R^3} d^3 p ~ \rho_c (q_i, p) = \int_{\mathbb R^3} d^3 p ~ \frac{\exp (- \beta ( \frac{p^2}{2m} + V(q_i) ))}{(\frac{2\pi m}{\beta})^{3/2} \int_{\mathbb R^3} d^3 q \exp(- \beta V(q))} \\ & = \frac{\exp (- \beta V(q_i) )}{\int_{\mathbb R^3} d^3 q \exp(- \beta V(q))} \frac{\cancel{\int_{\mathbb R^3} d^3 p ~ \exp(- \beta \frac{p^2}{2m})}}{\cancel{(\frac{2\pi m}{\beta})^{3/2}}} = \frac{\exp (- \beta V(q_i) )}{\int_{\mathbb R^3} d^3 q \exp(- \beta V(q))} ~.
        \end{aligned}
        \end{equation*}
    \end{proof}

    If we have a potential defined as 
    \begin{equation*}
        V (q) = \begin{cases}
            0 & \textnormal{inside a region } \mathcal A\\
            \infty & \textnormal{outside a region } \mathcal A\\
        \end{cases} ~,
    \end{equation*}
    the probability is null outside this region and uniform inside it. 
    \begin{proof}
        In fact, we find
        \begin{equation*}
            \rho(q_i) = \frac{1}{\int_{\mathcal A} d^3 q} = \frac{1}{\mathcal A} ~. 
        \end{equation*}
    \end{proof}

    The momentum probability density distribution is 
    \begin{equation*}
        \rho(p) = (2\pi m k_B T)^{-3/2} \exp(- \beta \frac{p^2}{2m}) = \prod_i (2\pi m k_B T)^{-1/2} \exp(- \beta \frac{p^2_i}{2m}) ~.
    \end{equation*}
    A plot of the momentum probability density distribution is in Figure~\ref{fig:mb:mom}.
    \begin{figure}
        \centering
        \scalebox{0.7}{\pyc{plot1('x', 'exp(- x**2)', 3, 2, 6, True, False, True)}}
        \caption{A plot of the momentum probability density distribution. We have used $x = \sqrt{\frac{\beta}{2m}} p$ and $f(x) = (2\pi m k_B T)^{3/2} \rho$.}
        \label{fig:mb:mom}
    \end{figure}
    \begin{proof}
        By definition, we have
        \begin{equation*}
        \begin{aligned}
            \rho(p) & = \int_{\mathbb R^3} d^3 q ~ \rho_c (q, p) = \int_{\mathbb R^3} d^3 q ~ \frac{\exp (- \beta ( \frac{p^2}{2m} + V(q) ))}{(\frac{2\pi m}{\beta})^{3/2} \int_{\mathbb R^3} d^3 q' \exp(- \beta V(q'))} \\ & = \frac{\exp(- \beta \frac{p^2}{2m})}{(\frac{2\pi m}{\beta})^{3/2}} \frac{\cancel{\int_{\mathbb R^3} d^3 q ~ \exp (- \beta V(q) )}}{\cancel{\int_{\mathbb R^3} d^3 q' \exp(- \beta V(q'))}} = \frac{\exp(- \beta \frac{p^2}{2m})}{(\frac{2\pi m}{\beta})^{3/2}} \\ & = (2\pi m k_B T)^{-3/2} \exp(- \beta \frac{p^2}{2m}) = \prod_i (2\pi m k_B T)^{-1/2} \exp(- \beta \frac{p^2_i}{2m}) ~.
        \end{aligned}
        \end{equation*}
    \end{proof}
    
    The velocity probability density distribution is
    \begin{equation*}
        \rho(p) = (\frac{m}{2\pi k_B T})^{1/2} \exp(- \beta \frac{m v^2_i}{2}) ~.
    \end{equation*}
    \begin{proof}
        In order to have velocities instead of momenta, we make a change of variables into
        \begin{equation*}
            p_i = m v_i ~, \quad \rho(v_i) dv_i = \rho(p_i) dp_i = \rho(p_i) m dv_i ~,
        \end{equation*}
        hence, we find
        \begin{equation*}
            \rho(v_i) = m \rho (p_i) = (\frac{2\pi k_B T}{m})^{-1/2} \exp(- \beta \frac{m^2 v^2_i}{2m}) = (\frac{m}{2\pi k_B T})^{1/2} \exp(- \beta \frac{m v^2_i}{2}) ~.
        \end{equation*}
    \end{proof}

    The velocity modulus probability density distribution is 
    \begin{equation*}
        \rho (v) = (\frac{m}{2\pi k_B T})^{3/2} 4 \pi v^2 \exp(- \beta \frac{m v^2}{2}) ~.
    \end{equation*}
    A plot of the velocity modulus probability density distribution is in Figure~\ref{fig:mb:vel}.
    \begin{figure}
        \centering
        \scalebox{0.7}{\pyc{plot1('x', 'x**2 * exp(- x**2)', 3, 0.5, 7, True, True, True)}}
        \caption{A plot of the velocity modulus probability density distribution. We have used $x = \sqrt{\frac{\beta m}{2}} v$ and $f(x) = \rho$.}
        \label{fig:mb:vel}
    \end{figure}
    \begin{proof}
        In order to have square modulus velocities, we make a change of variable into the polar coordinates $(v, \theta, \phi)$
        \begin{equation*}
            \rho(v_1, v_2, v_3) dv_1 dv_2 dv_3 = \rho(v_1, v_2, v_3) v^2 \sin \theta d\theta d\phi dv = \rho(\theta, \phi, v) d\theta d\phi dv ~,
        \end{equation*}
        hence, we find
        \begin{equation*}
            \rho(v) = 4 \pi v^2 \prod_i \rho (v_i) = (\frac{m}{2\pi k_B T})^{3/2} 4 \pi v^2 \exp(- \beta \frac{m v^2}{2})  ~.
        \end{equation*}
    \end{proof}
    
    The most probable velocity value is 
    \begin{equation*}
        v_p = \sqrt{\frac{2 k_B T}{m}} ~.
    \end{equation*}
    \begin{proof}
        By definition, we have
        \begin{equation*}
            0 = \dv{\rho(v)}{v} = 2 v \exp(- \beta \frac{m v^2}{2}) - \beta m v^3 \exp(- \beta \frac{m v^2}{2}) ~,
        \end{equation*}
        hence, we find
        \begin{equation*}
            v_p = \sqrt{\frac{2k_B T}{m}} ~.
        \end{equation*}
    \end{proof}

    The mean velocity value is 
    \begin{equation*}
        \av{v} = \sqrt{\frac{8 k_B T}{\pi m}} ~.
    \end{equation*}
    \begin{proof}
        By definition, we have
        \begin{equation*}
            \av{v} = \int_{\mathbb R^3} dv ~ \rho(v) v = (\frac{m}{2\pi k_B T})^{3/2} 4 \pi \int_{\mathbb R^3} dv ~ v^3 \exp(- \beta \frac{m v^2}{2}) ~.
        \end{equation*}
        Now, in order to evaluate the integral, we make a change of variables into
        \begin{equation*}
            t = \frac{m \beta v^2}{2} ~, \quad dt = m \beta v dv ~,
        \end{equation*}
        hence, we find
        \begin{equation*}
        \begin{aligned}
            \av{v} & =  (\frac{m}{2\pi k_B T})^{3/2} 4 \pi \Big (\frac{2}{m \beta} \Big) \frac{1}{m \beta} \underbrace{\int_0^\infty dt ~ t \exp(- t)}_{\Gamma (2)} \\ & = \sqrt{\frac{8}{m \pi \beta}} \underbrace{\Gamma (2)}_1 = \sqrt{\frac{8}{m \pi \beta}} = \sqrt{\frac{8 k_B T}{\pi m}} ~.
        \end{aligned}
        \end{equation*}
    \end{proof}

    The mean square velocity value is 
    \begin{equation*}
        \av{v^2} = \frac{3 k_B T}{m} ~.
    \end{equation*}
    \begin{proof}
        By definition, we have
        \begin{equation*}
            \av{v^2} = \int_{\mathbb R^3} dv ~ \rho(v) v^2 = (\frac{m}{2\pi k_B T})^{3/2} 4 \pi \int_{\mathbb R^3} dv ~ v^4 \exp(- \beta \frac{m v^2}{2}) ~.
        \end{equation*}
        Now, in order to evaluate the integral, we make a change of variables into
        \begin{equation*}
            t = \frac{m \beta v^2}{2} ~, \quad dt = m \beta v dv ~,
        \end{equation*}
        hence, we find
        \begin{equation*}
        \begin{aligned}
            \av{v^2} & =  (\frac{m}{2\pi k_B T})^{3/2} 4 \pi \Big (\frac{2}{m \beta} \Big) \frac{1}{m \beta} \Big ( \frac{2}{m \beta} \Big)^{1/2} \underbrace{\int_0^\infty dt ~ t^{3/2} \exp(- t)}_{\Gamma (5/2)} \\ & = \frac{4}{\sqrt{\pi} m \beta} \underbrace{\Gamma (5/2)}_{\frac{3 \sqrt{\pi}}{4}} = \frac{3}{\beta m} = \frac{3 k_B T}{m} ~.
        \end{aligned}
        \end{equation*}
    \end{proof}

\section{Entropic Maxwell-Boltzmann distribution}

    \begin{exercise}
        Consider $N$ distinguishable particles. Find its probability distribution, using the method of counting of states.
    \end{exercise}

    We have to evaluate the two terms of~\eqref{e:count}. For the first, we can distribute $N$ particle in $p$ boxes in the following way 
    \begin{equation*}
        W^{(1)}_{n_r} = \frac{N!}{n_1! \ldots n_p!} ~,
    \end{equation*}
    whereas, for the second, there is no restriction for the states 
    \begin{equation*}
        W^{(2)}_{n_r} = \prod_r g_r^{n_r} ~,
    \end{equation*}
    hence, we obtain
    \begin{equation*}
        W_{n_r} = \frac{N!}{n_1! \ldots n_p!} \prod_{r=1}^p g_r^{n_r} = N! \prod_{r=1}^p \frac{g_r^{n_r}}{n_r!} ~.
    \end{equation*}

    Maximising the constrained entropy, we find the Boltzmann distribution 
    \begin{equation*}
        p_r^* = \frac{n_r^*}{N} = \frac{g_r \exp(- \beta E_r)}{\sum_r g_r \exp(- \beta E_r)} ~.
    \end{equation*}
    A plot of the Boltzmann distribution $p_r^*$ as a function of $\beta E_r$ is in Figure~\ref{en:bol}.
    \begin{figure}
        \centering
        \scalebox{0.7}{\pyc{plot1('x', 'exp(-x)', 5, 10, 12, True, False, True)}}
        \caption{A plot of the probability density distribution $p_r^*$ as a function of $\beta E_r$. We have used $x = \beta E_r$ and $f(x) = p_r^* \frac{\sum_r g_r \exp(- \beta E_r)}{g_r}$.}
        \label{en:bol}
    \end{figure}
    \begin{proof}
        By definition~\eqref{e:shannon2}, using the Stirling approximation~\eqref{app:stirl}, we have
        \begin{equation*}
        \begin{aligned}
            S & = \ln W_{n_r} = \ln \Big (N! \prod_{r=1}^p \frac{g_r^{n_r}}{n_r!} \Big) = \ln N! + \sum_{r=1}^p \ln \frac{g_r^{n_r}}{n_r!} = \underbrace{\ln N!}_{N \ln N - N} + \sum_{r=1}^p (\ln g_r^{n_r} - \underbrace{\ln n_r!}_{n_r \ln n_r - n_r} ) \\ & = N \ln N - \cancel{N} + \sum_{r=1}^p n_r \ln g_r - \sum_{r=1}^p n_r \ln n_r - \cancel{\sum_{r=1}^p n_r} = N \ln N + \sum_{r=1}^p n_r \ln g_r + \sum_{r=1}^p n_r \ln n_r ~.
        \end{aligned}
        \end{equation*}
        Adding the constraints~\eqref{e:constrain}, we have
        \begin{equation*}
            S =  N \ln N + \sum_{r=1}^p n_r \ln g_r - \sum_{r=1}^p n_r \ln n_r + \alpha \Big (N - \sum_{r=1}^p n_r \Big) + \beta \Big (E - \sum_{r=1}^p n_r E_r \Big ) ~.
        \end{equation*}
        Computing the maximum by putting the derivative to zero, we obtain
        \begin{equation*}
            0 = \pdv{S}{n_r} = \ln g_r - \ln n_r - 1 - \alpha - \beta E_r ~,
        \end{equation*}
        hence, we find
        \begin{equation*}
            n_r^* = \frac{g_r \exp(- \beta E_r)}{\exp(1 + \alpha)} ~.
        \end{equation*}
        We find $\alpha$ by the normalisation condition 
        \begin{equation*}
            N = \sum_r n_r^* = \sum_r \frac{g_r \exp(- \beta E_r)}{\exp(1 + \alpha)} ~,
        \end{equation*}
        hence, we find
        \begin{equation*}
            \exp(1 + \alpha) = \frac{\sum_r g_r \exp(- \beta E_r)}{N} ~.
        \end{equation*}
        Finally, if we identify $\beta = 1/k_B T$, we obtain
        \begin{equation*}
            p_r^* = \frac{n_r^*}{N} = \frac{g_r \exp(- \beta E_r)}{\sum_r g_r \exp(- \beta E_r)} ~.
        \end{equation*}
    \end{proof}
    
\section{Entropic Fermi-Dirac distribution}

    \begin{exercise}
        Consider $N$ fermionic indistinguishable particles. Find its probability distribution, using the method of counting of states.
    \end{exercise}

    We have to evaluate the two terms of~\eqref{e:count}. For the first, there is no restriction on how we can distribute $N$ particle in $p$ boxes in ways, since they are indistinguishable,
    \begin{equation*}
        W^{(1)}_{n_r} = 1 ~,
    \end{equation*}
    whereas, for the second, we can distribute $n_r$ objects in $g_r$ boxes in the following way
    \begin{equation*}
        W^{(2)}_{n_r} = \prod_r \binom{g_r}{n_r} = \prod_r \frac{g_r!}{n_r! (g_r - n_r)!} ~,
    \end{equation*}
    hence, we obtain
    \begin{equation*}
        W_{n_r} = \prod_r \binom{g_r}{n_r} = \prod_r \frac{g_r!}{n_r! (g_r - n_r)!} ~.
    \end{equation*}
    
    Maximising the constrained entropy, we find the Bose-Einstein distribution 
    \begin{equation*}
        n_r^* = \frac{g_r}{\exp(\alpha + \beta E_r) + 1} ~.
    \end{equation*}
    A plot of the Fermi-Dirac distribution $n_r^*$ as a function of $\alpha + \beta E_r$ is in Figure~\ref{en:fd}.
    \begin{figure}
        \centering
        \scalebox{0.7}{\pyc{plot1('x', '1 / ( exp(x) + 1)', 5, 2, 13, True, False, True)}}
        \caption{A plot of the Fermi-Dirac distribution $n_r^*$ as a function of $\alpha + \beta E_r$. We have used $x = \alpha + \beta E_r $ and $f(x) = \frac{n_r^*}{g_r}$.}
        \label{en:fd}
    \end{figure}
    \begin{proof}
        By definition~\eqref{e:shannon2}, using the Stirling approximation~\eqref{app:stirl}, we have
        \begin{equation*}
        \begin{aligned}
            S & = \ln W_{n_r} = \ln \Big (\prod_r \frac{g_r!}{n_r! (g_r - n_r)!} \Big) = \sum_r \Big ( \underbrace{\ln g_r!}_{g_r \ln g_r - g_r} - \underbrace{\ln n_r!}_{n_r \ln n_r - n_r} - \underbrace{\ln (g_r - n_r)!}_{(g_r - n_r) \ln (g_r - n_r) - g_r + n_r} \Big) \\ & = \sum_r \Big ( g_r \ln g_r - \cancel{g_r} - n_r \ln n_r + \cancel{n_r} - (g_r - n_r) \ln (g_r - n_r) + \cancel{g_r} - \cancel{n_r} \Big) \\ & = \sum_r \Big ( g_r \ln g_r - n_r \ln n_r - (g_r - n_r) \ln (g_r - n_r) \Big) 
             ~.
        \end{aligned}
        \end{equation*}
        Adding the constraints~\eqref{e:constrain}, we have
        \begin{equation*}
            S =  \sum_r \Big ( g_r \ln g_r - n_r \ln n_r - (g_r - n_r) \ln (g_r - n_r) \Big) + \alpha \Big (N - \sum_{r=1}^p n_r \Big) + \beta \Big (E - \sum_{r=1}^p n_r E_r \Big ) ~.
        \end{equation*}
        Computing the maximum by putting the derivative to zero, we obtain
        \begin{equation*}
        \begin{aligned}
            0 & = \pdv{S}{n_r} = - \ln n_r - \cancel{1} + \ln (g_r - n_r) + \cancel{1} - \alpha - \beta E_r \\ & = - \ln n_r + \ln (g_r - n_r) - \alpha - \beta E_r = \ln (\frac{g_r}{n_r} - 1) - \alpha - \beta E_r~ ,
        \end{aligned}
        \end{equation*}
        hence, we find
        \begin{equation*}
            \frac{g_r}{n_r} - 1 = \exp(\alpha + \beta E_r) ~,
        \end{equation*}
        \begin{equation*}
            n_r^* = \frac{g_r}{\exp(\alpha + \beta E_r) + 1} ~.
        \end{equation*}
    \end{proof}
    
\section{Entropic Bose-Einstein distribution}

    \begin{exercise}
        Consider $N$ bosonic indistinguishable particles. Find its probability distribution, using the method of counting of states.
    \end{exercise}
    
    We have to evaluate the two terms of~\eqref{e:count}. For the first, there is no restriction on how we can distribute $N$ particle in $p$ boxes in ways, since they are indistinguishable,
    \begin{equation*}
        W^{(1)}_{n_r} = 1 ~,
    \end{equation*}
    whereas, for the second, we can distribute $n_r$ objects in $g_r$ boxes in the following way
    \begin{equation*}
        W^{(2)}_{n_r} = \prod_r \binom{n_r + g_r - 1}{n_r} = \prod_r \frac{(n_r + g_r - 1)!}{n_r! (g_r - 1)!} ~,
    \end{equation*}
    hence, we obtain
    \begin{equation*}
        W_{n_r} = \prod_r \frac{(n_r + g_r - 1)!}{n_r! (g_r - 1)!} ~.
    \end{equation*}
    
    Maximising the constrained entropy, we find the Bose-Einstein distribution 
    \begin{equation*}
        n_r^* = \frac{g_r}{\exp(\alpha + \beta E_r) - 1} ~.
    \end{equation*}
    A plot of the Bose-Einstein distribution $n_r^*$ as a function of $\alpha + \beta E_r$  is in Figure~\ref{en:be}.
    \begin{figure}
        \centering
        \scalebox{0.7}{\pyc{plot1('x', '1 / ( exp(x) - 1)', 5, 5, 14, True, False, True)}}
        \caption{A plot of the Bose-Einstein distribution $n_r^*$ as a function of $\alpha + \beta E_r$. We have used $x = \alpha + \beta E_r $ and $f(x) = \frac{n_r^*}{g_r}$.}
        \label{en:be}
    \end{figure}
    \begin{proof}
        By definition~\eqref{e:shannon2}, using the Stirling approximation~\eqref{app:stirl}, we have
        \begin{equation*}
        \begin{aligned}
            S & = \ln W_{n_r} = \ln \prod_r \frac{(n_r + g_r - 1)!}{n_r! (g_r - 1)!} \\ & = \sum_r \Big (\underbrace{\ln (n_r + g_r - 1)!}_{(n_r + g_r - 1) \ln (n_r + g_r - 1) - n_r - g_r + 1} - \underbrace{\ln n_r!}_{n_r \ln n_r - n_r} - \underbrace{\ln (g_r - 1)!}_{(g_r - 1) \ln (g_r - 1) - g_r + 1} \Big ) \\ & = \sum_r \Big ( (n_r + g_r - 1) \ln (n_r + g_r - 1) - \cancel{n_r} - \cancel{g_r} + \cancel{1} \\ & \quad - n_r \ln n_r + \cancel{n_r }- (g_r - 1) \ln (g_r - 1) + \cancel{g_r} - \cancel{1} \Big ) \\ & = \sum_r \Big ( (n_r + g_r - 1) \ln (n_r + g_r - 1) \ln n_r - n_r \ln n_r - (g_r - 1) \ln (g_r - 1) \Big ) 
        \end{aligned}
        \end{equation*}
        Adding the constraints~\eqref{e:constrain}, we have
        \begin{equation*}
        \begin{aligned}
            S & = \sum_r \Big ( (n_r + g_r - 1) \ln (n_r + g_r - 1) \ln n_r - n_r \ln n_r - (g_r - 1) \ln (g_r - 1) \Big ) \\ & \quad + \alpha \Big (N - \sum_{r=1}^p n_r \Big) + \beta \Big (E - \sum_{r=1}^p n_r E_r \Big ) ~.
        \end{aligned}
        \end{equation*}
        Computing the maximum by putting the derivative to zero, we obtain
        \begin{equation*}
        \begin{aligned}
            0 & = \pdv{S}{n_r} = \ln (n_r + g_r - 1) + \cancel{1} - \ln n_r - \cancel{1} - \alpha - \beta E_r \\ & = \ln (n_r + g_r - 1) - \ln n_r - \alpha - \beta E_r = \ln (\frac{g_r - 1}{n_r} + 1) - \alpha - \beta E_r~ ,
        \end{aligned}
        \end{equation*}
        hence, for $g_r \gg 1$, we find
        \begin{equation*}
            \frac{g_r - 1}{n_r} + 1 = \exp(\alpha + \beta E_r) ~,
        \end{equation*}
        \begin{equation*}
            n_r^* = \frac{g_r - 1}{\exp(\alpha + \beta E_r) - 1} \simeq \frac{g_r}{\exp(\alpha + \beta E_r) - 1} ~.
        \end{equation*}
    \end{proof}

\section{Entropic Two-levels system}

    \begin{exercise}
        Consider a system composed by two levels of energies $\epsilon_+ = + \epsilon$ and $\epsilon_- = - \epsilon$, subjected to constrains
        \begin{equation*}
            E = \epsilon (n_+ - n_-) ~, \quad N = n_+ + n_- ~.
        \end{equation*}. Find, using the method of counting of states, the entropy $S$ and the tempretaure $T$.
    \end{exercise}

    Constraints can be inverted to find
    \begin{equation}
        n_+ = \frac{N}{2} + \frac{E}{2\epsilon} ~, \quad n_+ = \frac{N}{2} - \frac{E}{2\epsilon} ~.
    \end{equation}
    We can distribute $N$ objects in $n_+$ boxes in the following way
    \begin{equation*}
        \Omega(E) = \binom{N}{n_+} = \frac{N!}{n_+! (N - n_+)!}  = \frac{N!}{n_+! n_-!} ~.
    \end{equation*}
    
    The entropy $S$ is 
    \begin{equation*}
        S = - N k_B \Big ( (\frac{1}{2} + \frac{E}{2\epsilon N}) \ln (\frac{1}{2} + \frac{E}{2\epsilon N} ) + (\frac{1}{2} - \frac{E}{2\epsilon N}) \ln (\frac{1}{2} - \frac{E}{2\epsilon N} ) \Big ) ~.
    \end{equation*}
    A plot of the entropy $S$ as a function of $E$ is in Figure~\ref{en:s}.
    \begin{figure}
        \centering
        \scalebox{0.7}{\pyc{plot1('x', '-( (1 / 2 + x) * ln (1/2 + x) +  (1 / 2 - x) * ln (1/2 - x) )', 1, 1, 15, True, False, True)}}
        \caption{A plot of the entropy $S$ as a function of $E$. We have used $x = \frac{E}{2 \epsilon N} $ and $f(x) = \frac{S}{N k_B}$.}
        \label{en:s}
    \end{figure}
    \begin{proof}
        By definition~\eqref{mc:s}, using the Stirling approximation~\eqref{app:stirl}, we have
        \begin{equation*}
        \begin{aligned}
            \frac{S}{k_B} & = \ln \Omega (E) = \ln \frac{N!}{n_+! n_-!} = \underbrace{\ln N!}_{N \ln N - N} - \underbrace{\ln n_+!}_{n_+ \ln n_+ - n_+} - \underbrace{\ln n_-!}_{n_- \ln n_- - n_-} \\ & = N \ln N - \cancel{N} - n_+ \ln n_+ + \cancel{n_+} - n_- \ln n_- + \cancel{n_-} \\ & = N \ln N - n_+ \ln n_+ - n_- \ln n_- = (n_+ + n_-) \ln N - n_+ \ln n_+ - n_- \ln n_- \\ & = n_+ \ln \frac{N}{n_+} + n_- \ln \frac{N}{n_-} = (\frac{N}{2} + \frac{E}{2\epsilon}) \ln \frac{N}{\frac{N}{2} + \frac{E}{2\epsilon}} + (\frac{N}{2} - \frac{E}{2\epsilon}) \ln \frac{N}{\frac{N}{2} - \frac{E}{2\epsilon}} \\ & = N \Big ( (\frac{1}{2} + \frac{E}{2\epsilon N}) \ln \frac{1}{\frac{1}{2} + \frac{E}{2\epsilon N}} + (\frac{1}{2} - \frac{E}{2\epsilon N}) \ln \frac{1}{\frac{1}{2} - \frac{E}{2\epsilon N}} \Big ) \\ & = - N \Big ( (\frac{1}{2} + \frac{E}{2\epsilon N}) \ln (\frac{1}{2} + \frac{E}{2\epsilon N} ) + (\frac{1}{2} - \frac{E}{2\epsilon N}) \ln (\frac{1}{2} - \frac{E}{2\epsilon N} )\Big ) ~.
        \end{aligned}
        \end{equation*}
    \end{proof}
    
    The temperature $T$ is 
    \begin{equation*}
        T = \frac{2 \epsilon}{k_B} \frac{1}{\ln \frac{\frac{1}{2} - \frac{E}{2 \epsilon N}}{\frac{1}{2} + \frac{E}{2 \epsilon N}}} ~.
    \end{equation*}
    A plot of the temperature $T$ as a function of $E$ is in Figure~\ref{en:t}.
    \begin{figure}
        \centering
        \scalebox{0.7}{\pyc{plot1('x', '1 / (ln ( (0.5 - x) / (0.5 + x) ) )', 10, 10, 16, True, False, False)}}
        \caption{A plot of the temperature $T$ as a function of $E$. We have used $x = \frac{E}{2 \epsilon N} $ and $f(x) = \frac{k_B T}{2 \epsilon}$.}
        \label{en:t}
    \end{figure}
    \begin{proof}
        Using the first of~\eqref{td:es:s}, we have
        \begin{equation*}
        \begin{aligned}
            T & = (\pdv{S}{E})^{-1} = - (\frac{k_B}{2\epsilon} \ln (\frac{1}{2} + \frac{E}{2 \epsilon N}) + \cancel{\frac{k_B}{2\epsilon}} - \frac{k_B}{2\epsilon} \ln (\frac{1}{2} - \frac{E}{2 \epsilon N}) - \cancel{\frac{k_B}{2\epsilon}} )^{-1} \\ & = - (\frac{k_B}{2\epsilon} \ln \frac{\frac{1}{2} + \frac{E}{2 \epsilon N}}{\frac{1}{2} - \frac{E}{2 \epsilon N}})^{-1} = - \frac{2 \epsilon}{k_B} \frac{1}{\ln \frac{\frac{1}{2} + \frac{E}{2 \epsilon N}}{\frac{1}{2} - \frac{E}{2 \epsilon N}}} = \frac{2 \epsilon}{k_B} \frac{1}{\ln \frac{\frac{1}{2} - \frac{E}{2 \epsilon N}}{\frac{1}{2} + \frac{E}{2 \epsilon N}}} ~.
        \end{aligned}
        \end{equation*}
    \end{proof}
    
