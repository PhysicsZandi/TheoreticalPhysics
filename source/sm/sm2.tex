\part{Classical statistical mechanics}

\chapter{Classical mechanics}

    A state constitued by a system of N particles is described by a point in a $2N$-dimensional manifold $\mathcal M^N$, called the phase space, which is the Cartesian product of N single particle manifolds 
    \begin{equation*}
        \{(q^i, ~p_i)\} \in \mathcal M^N
    \end{equation*}
    where $i = 1, \ldots N$.

    An observable is a smooth real function 
    \begin{equation*}
        f ~\colon~ \mathcal M^N \rightarrow \mathbb R
    \end{equation*}

    and its measurement in a fixed point $(\tilde q^i, ~\tilde p_i)$ is its value in it 
    \begin{equation*}
        f = f(\tilde q^i, ~\tilde p_i)
    \end{equation*}

    The time evolution is governed by a real function, called the hamiltonian $H(q^i, ~p_i, ~t)$, which is the solution of the equations of motion, called the Hamilton's equations
    \begin{equation*}
        \dot q^i = \pdv{H}{p_i} \quad \dot p_i = - \pdv{H}{q^i}
    \end{equation*}
    
    \begin{theorem}[Conservation of energy] \label{consen}
        If the hamiltonian does not depend explicitly on time, it can be intepreted physically as the energy of the system, which is constants
        \begin{equation*}
            H(q^i(t), ~p_i(t)) = H(q^i(0), ~p_i(0)) = E = const
        \end{equation*}
    \end{theorem}

    Since they are deterministic, once the initial conditions are given, the trajectory in phase space is completely determined

\section{Probability density distribution}

    A macrostate is defined by setting the macroscopic thermodynamical quantities. A microstate is the knowledge of the phase space behaviour $(q^i, ~p_i)$. 

    In general, there are more microstates associated to the same macrostates, raising the concept of ensemble: fixing a macrostate, it is created a large number of copies of the same physical system but with different microstates. It can be studied with the introduction of a probability density distribution 
    \begin{equation*}
        \rho(q_i(t), ~p_i(t),~t)
    \end{equation*}
    such that it satisfies the following properties
    \begin{enumerate}
        \item positivity, i.e.
        \begin{equation*}
            \rho(q_i, ~p_i, ~t) \geq 0
        \end{equation*}
        \item normalisation, i.e.
        \begin{equation*}
            \int_{\mathcal M^n} \underbrace{\prod_{i=1}^N d^d q^i d^d p^i}_{d\Gamma} ~ \rho(q_i, ~p_i, ~t) = \int_{\mathcal M^n} d\Gamma ~ \rho(q_i, ~p_i, ~t) = 1
        \end{equation*}
    \end{enumerate}

    To solve the dimensional problem of the volume element $d\Gamma$, which must be adimensional but it has the dimension of an action to the power of $d$, it can be introduced the adimensional volume element 
    \begin{equation*}
        d \Omega = \frac{d\Gamma}{h^d} = \frac{\prod_{i=1}^N d^d q^i d^d p^i}{h^d}
    \end{equation*}
    where the scale factor $h$ has the dimension of an action.

    The probability to find the system in a finite portion of the phase space $\mathcal U \subset \mathcal M^N$ is 
    \begin{equation*}
        \int_{\mathcal U} d\Gamma ~ \rho(q_i, ~p_i, ~t) 
    \end{equation*}

\section{Liouville's theorem}

    The flow of a system of particles keeps trasf of all their motions. See Figure.

    \begin{theorem}[Liouville]
        The volume through the flow generated by the hamilton's equations is constant. See Figure. Mathematically
        \begin{equation*}
            vol \Omega(t=0) = vol \Omega(t) ~\Rightarrow~ \dv{\rho}{t} = \pdv{\rho}{t} + \poi{\rho}{H} = 0
        \end{equation*}
    \end{theorem}

    \begin{proof}
        Maybe in the future.
    \end{proof}

    The physical intepretation of this theorem is that particles do not appear nor disappear due to conservation of charge, mass, etc...

    For stationary systemas, i.e. when $\pdv{\rho}{t} = 0$, the necessary condition for equilibrium is $\poi{\rho}{H} = 0$, which is satisfied only if 
    \begin{equation*}
        \rho = const 
    \end{equation*}
    like in the microcanonical ensemble, and 
    \begin{equation*}
        \rho = \rho(H)
    \end{equation*}
    like in the canonical or the grancanonical ensembles.

    \begin{proof}
        Maybe in the future.
    \end{proof}

    The average value of an observable is weighted by the probability density distribution
    \begin{equation}\label{clav}
        \av{f} = \int_{\mathcal M^N} d\Gamma ~ \rho(q^i, ~p_i) f(q^i, ~p_i)
    \end{equation}
    and the standard deviation is 
    \begin{equation*}
        (\Delta f)^2 = \av{f^2} - \av{f}^2
    \end{equation*}

\section{Time-independent Hamiltonian}

    Consider a time-independent hamiltonian. Since the energy is constant for the theorem~\ref{consen}.

    \begin{equation}\label{norm}
        \int_{\mathcal M^N} \rho = 1
    \end{equation}
    \begin{equation}\label{T}
        \pdv{S}{E} = \frac{1}{T}
    \end{equation}
    \begin{equation}\label{F}
        \pdv{F}{T} = - S
    \end{equation}
    \begin{equation}\label{OM}
        \Omega = - pV = E - TS - \mu N
    \end{equation}

\chapter{Microcanonical ensemble}

    A microcanonical ensemble is a system which is isolated from the environment, i.e. it cannot exchange neither energy nor matter, so $E$, $N$ and $V$ are fixed. Since energy is conserved and the hamiltonian is time-independent, the trajectory of motion is restricted on the surface $S_E$ and not on all the phase space.
    
    Assume an a-priory uniform probability 
    \begin{equation*}
        \rho_{mc}(q^i, ~p_i) = C \delta (\mathcal H(q^i, p_i) - E)
    \end{equation*}
    where $C$ is a normalisation constant, which can be evaluated by~\eqref{norm}
    \begin{equation*}
        1 = \int_{\mathcal M^N} d\Omega \rho_{mc} = \int_{\mathcal M^N} d\Omega C \delta(\mathcal H - E) = C \int_{\mathcal M^N} d\Omega \delta(\mathcal H - E) = C \omega(E)
    \end{equation*}

    Hence
    \begin{equation*}
        \rho_{mc}(q^i, ~p_i) = \frac{1}{\omega(E)} \delta (\mathcal H(q^i, p_i) - E)
    \end{equation*}

    Consider a displacement on an infinitesimal displacement of energy $\Delta E \ll 1$, then 
    \begin{equation*}
        \Gamma (E) = \integ{E}{E+dE}{E'} \omega(E') \simeq \omega(E) \Delta E
    \end{equation*}
    and the distribution is 
    \begin{equation*}
        \rho_{mc}(q^i, p_i) = \begin{cases}
            \frac{1}{\Gamma(E)} & \mathcal H \in [E, E + \Delta E] \\
            0 & otherwise
        \end{cases}
    \end{equation*}

    Let $f(q^i, p_i)$ be an observable, then its microcanonical average is 
    \begin{equation}\label{obs}
        \avp{f(q^i, p_i)}{mc} = \int_{\mathcal M} d\Omega ~ \rho_{mc} f = \int_{\mathcal M} d\Omega ~ \frac{1}{\omega(E)} \delta (\mathcal H - E) f = \frac{1}{\omega(E)} \int_{S_E} dS_E ~ f = \avp{f}{E} 
    \end{equation}

\section{Thermodynamics potentials}

    The microcanonical entropy $S_{mc}$ is defined by 
    \begin{equation}\label{entropymc}
        S_{mc} (E, V, N) = k_B \ln \omega(E)
    \end{equation}

    The logarithm is justified by the fact that the volume of a N-particle phase space is $(W_1)^N$, where $W_1$ is the volume of a single particle phase space. According to the properties of the logarithm, entropy becomes extensive.

    In the thermodynamic limit, the following equations hold 
    \begin{equation*}
        s_{mc} = \lim_{td} \frac{S_{mc}}{N} = k_B \lim_{td} \frac{\log \omega(E)}{N} = \underbrace{k_B \lim_{td} \frac{\log \Sigma(E)}{N}}_{\mathcal H \in [0, E]} = \underbrace{k_B \lim_{td} \frac{\log \Gamma(E)}{N}}_{\mathcal H \in [E, E + \Delta E]}
    \end{equation*}

    Entropy is additive, so given two sistems $1$ and $2$
    \begin{equation*}
        s_{mc}^{tot} = s_{mc}^{(1)} + s_{mc}^{(2)}
    \end{equation*}

    \begin{proof}
        Consider two isolated systems in contact at equilibrium with the same temperature $T = T_1 = T_2$. The total energy is $E = E_1 + E_2 + E_{surface}$ but, in the thermodynamic limit, the energy exchanged by the surface is a subleading term ($E_1$ and $E_2$ go as $L^3$ whereas $E_{surface}$ goes as $L^2$) and can be neglected. The energy density is 
        \begin{equation*}
        \begin{aligned}
            \omega(E) & = \int_{\mathcal M^N} d\Gamma_1 d\Gamma_2 \delta(\mathcal H - E) \\ & = \int dE_1 \int dS_{E_1} \int dE_2 \int dS_{E_2} \delta (E - E_1 - E_2) \\ & = \int dE_1 \int dE_2 \omega_1(E_1) \omega_2(E_2) \delta (E - E_1 - E_2) \\ & = \integ{0}{E}{E_1} \omega_1(E_1) \omega_2(E_2 = E - E_1)
        \end{aligned}
        \end{equation*}
        Since the integrand is a positive function with a maximum in $_1 \in [0, E]$
        \begin{equation}\label{proof1}
        \begin{aligned}
            \integ{0}{E}{E_1} \omega_1(E_1) \omega_2(E_2 = E - E_1) & \leq \omega_1(E^*_1) \omega_2(E^*_2 = E - E^*_1) \integ{0}{E}{E_1} \\ & = \omega_1(E^*_1) \omega_2(E^*_2 = E - E^*_1) E
        \end{aligned}
        \end{equation}

        On the other hand, it is always possible to find a value for $\Delta E$ in order to have 
        \begin{equation} \label{proof2}
            \Delta E \omega_1(E^*_1) \omega_2(E^*_2) \leq \omega(E)
        \end{equation}

        Putting together~\eqref{proof1} and~\eqref{proof2}
        \begin{equation*}
            \Delta E \omega_1(E^*_1) \omega_2(E^*_2) \leq \omega(E) \leq \omega_1(E^*_1) \omega_2(E^*_2) E
        \end{equation*}
        \begin{equation*}
            \omega_1(E^*_1) \Delta E \omega_2(E^*_2) \Delta E \leq \omega(E) \Delta E \leq \frac{E}{\Delta E} \omega_1(E^*_1) \Delta E \omega_2(E^*_2) \Delta E
        \end{equation*}
        \begin{equation*}
            \Gamma_1(E^*_1) \Gamma(E^*_2) \leq \Gamma(E) \leq \frac{E}{\Delta E}\Gamma(E^*_1) \Gamma(E^*_2)
        \end{equation*}
        Since the logarithm is a monotomic function
        \begin{equation*}
            \log \Big ( \Gamma_1(E^*_1) \Gamma(E^*_2) \Big ) \leq \log \Gamma(E) \leq \log \Big ( \frac{E}{\Delta E}\Gamma(E^*_1) \Gamma(E^*_2) \Big )
        \end{equation*}
        \begin{equation*}
            k_B \log \Big ( \Gamma_1(E^*_1) \Gamma(E^*_2) \Big ) \leq k_B \log \Gamma(E) \leq k_B \log \Big ( \frac{E}{\Delta E}\Gamma(E^*_1) \Gamma(E^*_2) \Big )
        \end{equation*}
        \begin{equation*}
            k_B \log \Gamma_1(E^*_1) + k_B \log \Gamma(E^*_2) \leq k_B \log \Gamma(E) \leq k_B \log \frac{E}{\Delta E} + k_B \log \Gamma(E^*_1) + k_B \log \Gamma(E^*_2)
        \end{equation*}
        \begin{equation*}
            \frac{k_B \log \Gamma_1(E^*_1) + k_B \log \Gamma(E^*_2)}{N} \leq \frac{k_B \log \Gamma(E)}{N} \leq \frac{k_B \log \frac{E}{\Delta E} + k_B \log \Gamma(E^*_1) + k_B \log \Gamma(E^*_2)}{N}
        \end{equation*}

        In the thermodynamic limit, the last term vanishes, since $\lim_{td} \frac{1}{N} \log \frac{N}{\Delta N} = 0$. Hence 
        \begin{equation*}
            s_{mc}(E) = s_{mc}^{(1)} + s_{mc}^{(2)}
        \end{equation*}
 
    \end{proof}

    The last result tells also that at equilibrium entropy is maximum.

    In the thermodynamic limit, microcanonical entropy coincides with the thermodynamical one 
    \begin{equation*}
        s_{mc} = s_{td}
    \end{equation*}

    \begin{proof}
        Since entropy is maximum at equilibrium, also $\Gamma_1(E_1) \Gamma_2(E_2)$ is so and
        \begin{equation*}
        \begin{aligned}
            0 & = \delta (\Gamma_1(E^*_1) \Gamma_2(E^*_2 = E - E^*_1)) \\ & = \delta \Gamma_1(E^*_1) \Gamma_2 (E^*_2) + \Gamma_1(E^*_1) \delta \Gamma_2 (E^*_2) \\ & = \pdv{\Gamma_1}{E_1} \Big\vert_{E^*_1} \delta E_1 \Gamma_2 (E^*_2) + \Gamma_1(E^*_1) \pdv{\Gamma_2}{E_2} \Big\vert_{E^*_2} \delta E_2 
        \end{aligned}
        \end{equation*}

        Since $E = const$, $0 = \delta E = \delta E_1 + \delta E_2$, $\delta E_2 = \delta E_1$ and
        \begin{equation*}
            0 = \pdv{\Gamma_1}{E_1} \Big\vert_{E^*_1} \delta E_1 \Gamma_2 (E^*_2) - \Gamma_1(E^*_1) \pdv{\Gamma_2}{E_2} \Big\vert_{E^*_2} \delta E_1 
        \end{equation*}
        \begin{equation*}
            0 = \pdv{\Gamma_1}{E_1} \Big\vert_{E^*_1} \Gamma_2 (E^*_2) - \Gamma_1(E^*_1) \pdv{\Gamma_2}{E_2} \Big\vert_{E^*_2} 
        \end{equation*}
        \begin{equation*}
            \pdv{\Gamma_1}{E_1} \Big\vert_{E^*_1} \Gamma_2 (E^*_2) = \Gamma_1(E^*_1) \pdv{\Gamma_2}{E_2} \Big\vert_{E^*_2} 
        \end{equation*}
        \begin{equation*}
            \frac{1}{\Gamma_1 (E^*_1)} \pdv{\Gamma_1}{E_1} \Big\vert_{E^*_1} = \frac{1}{\Gamma_2 (E^*_2)} \pdv{\Gamma_2}{E_2} \Big\vert_{E^*_2} 
        \end{equation*}
        \begin{equation*}
            \pdv{\log \Gamma_1}{E_1} \Big\vert_{E^*_1} = \pdv{\log \Gamma_2}{E_2} \Big\vert_{E^*_2} 
        \end{equation*}

        Using the thermodynamical relation~\eqref{T}
        \begin{equation*}
            S_{mc} (E) = S_{td} (E) \times const
        \end{equation*}
        where the constant can be chosen in order to have $k_B$ in the same unit.
    \end{proof}

    The universal Boltzmann's formula is 
    \begin{equation}\label{unboltz}
        s_{mc} = s_{td} = k_B \log \omega(E) = - k_B \avp{\log \rho_{mc}}{mc} ~.
    \end{equation}

    \begin{proof}
        
    Using~\eqref{obs}, 
    \begin{equation*}
    \begin{aligned}
        \avp{\log \rho_{mc}}{mc} & = \int d\Gamma \rho_{mc} \log \rho_{mc} \\ & = \int d\Gamma \frac{1}{\omega(E)} \delta (\mathcal H - E) \log \Big ( \frac{1}{\omega(E)} \delta (\mathcal H - E) \Big) \\ & = \int dS_E \frac{1}{\omega(E)} \log \frac{1}{\omega(E)} \\ & = - \frac{1}{\omega(E)} \log \omega(E) \int dS_E \\ & = - \log \omega (E)
    \end{aligned}
    \end{equation*}
    \end{proof}

\chapter{Canonical ensemble}

    A canonical ensemble is a system which is immersed in a bigger environment or reservoir, which can exchange energy but not matter, so $T$, $N$ and $V$ are fixed. Globally, energy is conserved, since the universe composed by the union of the system and the environment can be considered as a microcanonical ensemble. 

    The canonical probability density distribution is 
    \begin{equation*}
        \rho_c (q^i, p_i) = \frac{1}{Z_N} \exp (-\beta \mathcal H(q^i, p_i))
    \end{equation*}
    where $\beta$ is 
    \begin{equation*}
        \beta = \frac{1}{k_B T}
    \end{equation*}
    and $Z_N$ is the partition function 
    \begin{equation}
        Z_N[V, T] = \int_{\mathcal M^N} d\Omega ~\exp (-\beta \mathcal H(q^i, p_i))
    \end{equation}
    which depends on the temperature through $\beta$ and volume and temperature due to the integration domain $\mathcal M^N = V \otimes \mathbb R^d$.

    Notice that the probability is a function of the hamiltonian, like Liouville's theorem said.

    \begin{proof}
        Consider the universe as a microcanonical ensemble. Its probability density distribution is 
        \begin{equation*}
            \rho_{mc} (q_i^{(1)}, p_i^{(1)}, q_i^{(2)}, p_i^{(2)}) = \frac{1}{\omega(E)} \delta (\mathcal H (q_i^{(1)}, p_i^{(1)}, q_i^{(2)}, p_i^{(2)}) - E)
        \end{equation*}
        where the total hamiltonian is 
        \begin{equation*}
            \mathcal H (q_i^{(1)}, p_i^{(1)}, q_i^{(2)}, p_i^{(2)}) = \mathcal H_1 (q_i^{(1)}, p_i^{(1)}) + \mathcal H_2 (q_i^{(2)}, p_i^{(2)})
        \end{equation*}

        Integrating it to all the possible state in the environment
        \begin{equation*}
            \rho^{(1)} = \int d\Omega_2  \rho_{mc} = \int d\Omega_2 \frac{1}{\omega(E)} \delta(\mathcal H - E) = \frac{1}{\omega(E)} \int dS_{E_2} = \frac{1}{\omega(E)} \omega(E_2 = E - E_1)
        \end{equation*}
        and the corresponding entropy is 
        \begin{equation*}
            S_2 (E_2) = k_B \ln \omega_2 (E_2)
        \end{equation*}

        Applying small variation $\delta E_1$ to $E_1$ to preserve equilibrium, the entropy trasforms, using~\eqref{T}
        \begin{equation*}
            k_B \ln \omega_2 (E_2) = S_{mc}(E) - E_1 \pdv{S_{mc}}{E} \Big \vert_{E_2} = S_{mc}(E) - E_1 \frac{1}{T} 
        \end{equation*}
        \begin{equation*}
            \ln \omega_2 (E_2) = \frac{S_{mc}(E)}{k_B} - E_1 \frac{1}{k_B T} 
        \end{equation*}
        \begin{equation*}
            \omega_2 (E_2) = \exp (\frac{S_{mc}(E)}{k_B} - E_1 \frac{1}{k_B T}) = \exp (\frac{S_{mc}(E)}{k_B}) \exp (- \frac{E_1}{k_B T}) 
        \end{equation*}

        Putting together, dropping the indices
        \begin{equation}
            \rho_c = \frac{\omega(2)(E_2)}{\omega(E)} = \frac{1}{\omega(E)} \exp (\frac{S_{mc}(E)}{k_B}) \exp (- \frac{E_1}{k_B T}) = C \exp (- \frac{E_1}{k_B T})
        \end{equation}
        where $C$ is a normalisation constant, which can be evaluated by~\eqref{norm}
        \begin{equation*}
            1 = \int_{\mathcal M^N} d\Omega \rho = \int_{\mathcal M^N} d\Omega C \exp (- \frac{E_1}{k_B T}) = C \int_{\mathcal M^N} d\Omega \exp (- \frac{E_1}{k_B T}) 
        \end{equation*}
    \end{proof}

    The partition function can also be written as 
    \begin{equation*}
        Z_N[T, V] = \integ{0}{\infty}{E} \omega(E) \exp (-\beta E)
    \end{equation*}

    \begin{proof}
        Foliating the phase space in energy hyper-surfaces 
        \begin{equation*}
            Z_N = \int_{\mathcal M^N} d\Omega \exp (- \beta \mathcal H) = \integ{0}{\infty}{E} \int dS_E ~ \exp (-\beta \mathcal H) = \integ{0}{\infty}{E} \omega(E) \exp (-\beta E)
        \end{equation*}
    \end{proof}

    Taking also in consideration indistinguishable particles, the partition function 
    \begin{equation*}
        Z_N = \int \frac{\prod_{i=1}^N d^d q^i d^d p^i}{h^{dN} \zeta_N} \exp (- \beta \mathcal H) 
    \end{equation*}
    where $\zeta_N$ is 
    \begin{equation*}
        \zeta_N = \begin{cases}
            1 & \textnormal{distinguishable} \\
            N! & \textnormal{indistinguishable}
        \end{cases}
    \end{equation*}

    The partition function of two systems is the multiplication of the single system ones
    \begin{equation}
        Z_N = Z_{N_1} Z_{N_2}
    \end{equation}

    \begin{proof}
        Since $\mathcal H = \mathcal H_1 + \mathcal H_2$, 
        \begin{equation*}
        \begin{aligned}
        \end{aligned}
        \end{equation*}
    \end{proof}

    If the hamiltonian is the sum of $N$ identical ones, like $N$ non-interacting particles
    \begin{equation*}
        \mathcal H = \sum_{i = 1}^{N} \mathcal H_i
    \end{equation*} 
    the partition function becomes 
    \begin{equation*}
        Z_N = \frac{(Z_1)^N}{\zeta_N}
    \end{equation*}

    \begin{proof}
        Denominating $Z_1$ the single-particle partition function
        \begin{equation*}
        \begin{aligned}
            Z_N & = \int_{\mathcal M^N = \mathcal M^{(1)} \otimes \ldots \otimes \mathcal M^{(1)}} \prod_{i=1}^N \frac{d^d q^i d^d p^i}{h^{dN} \zeta_N} \exp (-\beta \mathcal H) \\ & = \int_{\mathcal M^N = \mathcal M^{(1)} \otimes \ldots \otimes \mathcal M^{(1)}} \prod_{i=1}^N \frac{d^d q^i d^d p^i}{h^{dN} \zeta_N} \exp (-\beta \sum_{i = 1}^{N} \mathcal H_i) \\ & = \int_{\mathcal M^N = \mathcal M^{(1)} \otimes \ldots \otimes \mathcal M^{(1)}} \prod_{i=1}^N \frac{d^d q^i d^d p^i}{h^{dN} \zeta_N} \prod_{i=1}^{N}\exp (-\beta \mathcal H_i) \\ & = \int_{\mathcal M^N = \mathcal M^{(1)} \otimes \ldots \otimes \mathcal M^{(1)}} \prod_{i=1}^N \frac{d^d q^i d^d p^i}{h^{dN} \zeta_N} \exp (-\beta \mathcal H_i) \\ & = \frac{Z_1 Z_1 \ldots Z_1}{\zeta_N} = \frac{(Z_1)^N}{\zeta_N}
        \end{aligned}
        \end{equation*}
    \end{proof}

    Let $f(q^i, p_i)$ be an observable, then its canonical average is 
    \begin{equation*}\label{obsc}
        \avp{f(q^i, p_i)}{c} = \int_{\mathcal M} d\Omega ~ \rho_{c} f = \int_{\mathcal M} d\Omega ~ \frac{\exp (-\beta \mathcal H)}{Z_N} f
    \end{equation*}

\section{Thermodynamics variable}

    The canonical Helmotz free energy $F$ is defined by 
    \begin{equation}\label{ZF}
        Z_[V, T] = \exp(-\beta F[N, V, T])
    \end{equation}
    or, equivalently,
    \begin{equation}\label{can:f}
        F[V, N ,T] = -\frac{1}{\beta} \ln Z_N
    \end{equation}

    Furthermore, the canonical internal energy is 
    \begin{equation}\label{Ec}
        E = \avp{\mathcal H}{c} = \int d\Omega \frac{\exp(-\beta (\mathcal H))}{Z_N} \mathcal H
    \end{equation}

    \begin{proof}
        By normalisation condition 
        \begin{equation*}
            1 = \int d\Omega \frac{\exp(-\beta \mathcal H)}{Z_N} = \int d\Omega \frac{\exp(-\beta \mathcal H)}{\exp(-\beta F)} = \int d\Omega \exp (- \beta (\mathcal H - F))
        \end{equation*}

        Since $F$ depends on the temperature, it is possible to derive with respect to $\beta$
        \begin{equation*}
        \begin{aligned}
            0 & = \pdv{}{\beta} \Big ( \int d\Omega \exp (- \beta (\mathcal H - F)) \Big) \\ & = \int d\Omega \exp (-\beta (\mathcal H - F)) \Big (-(\mathcal H - F) + \beta \pdv{F}{\beta}) \\ & = - \underbrace{\int d\Omega \frac{\exp(-\beta \mathcal H)}{Z_N} \mathcal H}_{E} + F \underbrace{\int d\Omega \frac{\exp(-\beta \mathcal H)}{Z_N}}_{1} + \beta \pdv{F}{\beta} \underbrace{\int d\Omega \frac{\exp(-\beta \mathcal H)}{Z_N}}_{1} \\ & = - E + F + \beta \pdv{F}{\beta}
        \end{aligned}
        \end{equation*}
        Hence, using~\eqref{F}
        \begin{equation*}
            F = E + \beta \pdv{F}{\beta} = E + T \pdv{F}{T} = E - TS
        \end{equation*}
        showing that is indeed the Helmotz free energy.
    \end{proof}

    Notice that in the last result, the entropy can be also written as 
    \begin{equation} \label{can:s}
        S_c = \frac{E - F}{T}
    \end{equation}
    
    The internal energy can also be written as 
    \begin{equation}\label{can:en}
        E = - \pdv{}{\beta} \ln Z_N
    \end{equation}

    \begin{proof}
        Using~\eqref{Ec},
        \begin{equation*}
            - \pdv{}{\beta} \ln Z_N = - \frac{1}{Z_N} \pdv{Z_N}{\beta} = - \frac{1}{Z_N} \pdv{}{\beta} \int d\Omega \exp (-\beta \mathcal H) = \int d\Omega \frac{\exp(-\beta \mathcal H)}{Z_N}  \mathcal H = \avp{\mathcal H}{c} = E
        \end{equation*}
    \end{proof}

    The universal Boltzmann's formula is still valid
    \begin{equation*}
        S_c = -k_B \avp{\ln \rho_c}{c} 
    \end{equation*}


    \begin{proof}
        Using~\eqref{Ec} and~\eqref{can:f}
        \begin{equation*}
            \begin{aligned}
            -k_B \avp{\ln \rho_c}{c} & = -k_B \int d\Omega \rho_c \ln \rho_c \\ & = -k_B \int d\Omega \rho_c \ln \frac{\exp(-\beta \mathcal H)}{Z_N} \\ & = -k_B \int d\Omega \rho_c \ln \exp(-\beta \mathcal H) - k_B \int d\Omega \rho_c \ln Z_N \\ & = k_B \int d\Omega \beta \mathcal H - k_B \underbrace{\ln Z_N}_{\beta F} \underbrace{\int d\Omega \rho_c}_{1} \\ & =  \frac{E - F}{T} = S_c
        \end{aligned}
        \end{equation*}
    \end{proof}

\section{Equipartition theorem}

    \begin{theorem}[Generalised equipartition theorem]
        Let $\xi \in [a,b]$ and $\xi_j$ with $j \neq 1$ all the other coordinates or momenta. Suppose also 
        \begin{equation}\label{cond}
            \int \prod_{j \neq 1} d \xi_j [\xi_1 \exp(-\beta \mathcal H)]_a^b = 0
        \end{equation}
        Then 
        \begin{equation}\label{equi}
            \avp{\xi_1 \pdv{\mathcal H}{\xi_1}}{c} = k_B T
        \end{equation}
    \end{theorem}

    \begin{proof}
        By normalisation condition 
        \begin{equation*}
            1 = \int d\Omega \frac{\exp(-\beta \mathcal H)}{Z_N} = \frac{1}{Z_N} \int \prod_{j \neq 1} d \xi_j \exp(-\beta \mathcal H)
        \end{equation*}
        Using
        \begin{equation*}
            d\xi_1 (\xi_1 \exp(-\beta \mathcal H)) = d\xi_1 \exp(-\beta \mathcal H) + \xi \exp(-\beta \mathcal H) (-\beta) \pdv{\mathcal H}{\xi_1} d\xi_1
        \end{equation*}
        and integrating per parts
        \begin{equation*}
        \begin{aligned}        
            1 & = \frac{1}{Z_N} \underbrace{\int \prod_{j \neq 1} d \xi_j [\xi_1 \exp(-\beta \mathcal H)]_a^b}_{0} + \frac{\beta}{Z_N} \int \prod_{j \neq 1} d \xi_j d\xi_1 \xi_1 \pdv{\mathcal H}{\xi_1} \exp (-\beta \mathcal H) \\ & = \beta \int d\Omega \xi_1 \pdv{\mathcal H}{\xi_1} \frac{\exp (- \beta \mathcal H)}{Z_N} \\ & = \beta \avp{\xi_1 \pdv{\mathcal H}{\xi_1}}{c}
        \end{aligned}
        \end{equation*}
        Hence
        \begin{equation*}
            \avp{\xi_1 \pdv{\mathcal H}{\xi_1}}{c} = \frac{1}{\beta} = k_B T
        \end{equation*}
    \end{proof}

    Examples of system that satisfies the condition~\eqref{cond} are hamiltonians which depend on the square of momentum or confining potentials which go to infinity on the extremes $a$ and $b$. 

    \begin{corollary}[Equipartition theorem]
        If $\xi_1$ appears quadratically in $\mathcal H$, then its contribution to $E$ is $\frac{1}{2} k_B T$
    \end{corollary}

    \begin{proof}
        Consider $\mathcal H = A \xi_1^2 + B \xi_j^2$ with $j \neq 1$, then by the previous theorem 
        \begin{equation*}
            \avp{\xi_1 \pdv{\mathcal H}{\xi_1}}{c} = \avp{\xi 2 A \xi_1}{c} = k_B T
        \end{equation*}
        and 
        \begin{equation*}
            \avp{A \xi_1^2}{c} = \frac{1}{2} k_B T
        \end{equation*}
    \end{proof}

\chapter{Grancanonical ensemble}

    A grancanonical ensemble is a system which is immersed in a bigger environment or reservoir, which can exchange both energy and matter, so $T$, and $V$ are fixed. Globally, both energy and number of particles are conserved, since the universe composed by the union of the system and the environment can be considered as a microcanonical ensemble. First, with the same method used in the previous chapter, microcanonical can be transformed into canonical. Now, the universe is canonical and, globally, the number of particles is conserved. 

    The grancanonical probability density distribution is 
    \begin{equation*}
        \rho_{gc} (q^i, p_i) = \frac{\exp(-\beta \mathcal H_1)}{N_1! h^{d N_1}} \frac{Z_{N_2} [T, V_2]}{Z_N [T, V]}
    \end{equation*}
    
    \begin{proof}
        Consider the universe as a canonical ensemble. Its probability density distribution is 
        \begin{equation*}
            \rho_c (q_i^{(1)}, p_i^{(1)}, q_i^{(2)}, p_i^{(2)}) = \frac{\exp (-\beta \mathcal H (q_i^{(1)}, p_i^{(1)}, q_i^{(2)}, p_i^{(2)}))}{Z_N[T, V]}
        \end{equation*}
        where the total hamiltonian is 
        \begin{equation*}
            \mathcal H (q_i^{(1)}, p_i^{(1)}, q_i^{(2)}, p_i^{(2)}) = \mathcal H_1 (q_i^{(1)}, p_i^{(1)}) + \mathcal H_2 (q_i^{(2)}, p_i^{(2)})
        \end{equation*}

        Integrating it to all the possible state in the environment
        \begin{equation*}
        \begin{aligned}
            \rho^{(1)} & = \int d\Omega_2 ~ \rho_c \\ & = \int \prod_{i=1}^N \frac{d^d q_i^{(2)} d^d p_i^{(2)}}{N! h^{dN}} \frac{\exp(-\beta (\mathcal H_1 + \mathcal H_2))}{Z_N} \\ & = \frac{\exp(-\beta \mathcal H_1)}{N_1! h^{d N_1}} \frac{1}{Z_N} \int \prod_{i=1}^N \frac{d^d q_i^{(2)} d^d p_i^{(2)}}{N_2! h^{d N_2}} \exp(-\beta \mathcal H_2) \\ & = \frac{\exp(-\beta \mathcal H_1)}{N_1! h^{d N_1}} \frac{Z_{N_2} [T, V_2]}{Z_N [T, V]}
        \end{aligned}
        \end{equation*}
    \end{proof}

    The normalisation condition becomes 
    \begin{equation*}
        \sum_{N_1 = 0}^{N} \int_{\mathcal M^{N_1}} d\Omega_1 \rho_{gc} = 1
    \end{equation*}

    \begin{proof}
        Using the expression to evaluate the power of a sum 
        \begin{equation*}
            (a + b)^n = \sum_{i=1}^{n} \binom{n}{i} a^i b^{n-i} 
        \end{equation*}
        and 
        \begin{equation*}
        \begin{aligned}
            \int_{\mathcal M^{N_1}} d\Omega_1 ~ \rho_{gc} & = \int_{\mathcal M^{N_1}} d\Omega_1 \frac{\exp(-\beta \mathcal H_1)}{N_1! h^{d N_1}} \frac{Z_{N_2} [T, V_2]}{Z_N [T, V]} \\ & = \frac{N!}{N_1! N_2} \frac{\int_{\mathcal M^{N_1}} d\Omega_1 ~ \exp(-\beta \mathcal H_1) \int_{\mathcal M^{N_2}} d\Omega_2 ~ \exp(-\beta \mathcal H_2)}{\int_{\mathcal M^N} d\Omega ~ \exp(-\beta \mathcal H)} \\ & = \frac{N!}{N_1! N_2} \frac{\frac{\int_{\mathcal M^{N_1}} d\Omega_1 ~ \exp(-\beta \mathcal H_1)}{(V_1)^{N_1}} \frac{\int_{\mathcal M^{N_2}} d\Omega_2 ~ \exp(-\beta \mathcal H_2)}{(V_2)^{N_2}}}{\frac{\int_{\mathcal M^N} d\Omega ~ \exp(-\beta \mathcal H)}{V^N}} \frac{(V_1)^{N_1} (V_2)^{N_2}}{V^N} 
        \end{aligned}
        \end{equation*}
        which in the thermodynamical limit 
        \begin{equation*}
            \lim_{td} \frac{\frac{\int_{\mathcal M^{N_1}} d\Omega_1 ~ \exp(-\beta \mathcal H_1)}{(V_1)^{N_1}} \frac{\int_{\mathcal M^{N_2}} d\Omega_2 ~ \exp(-\beta \mathcal H_2)}{(V_2)^{N_2}}}{\frac{\int_{\mathcal M^N} d\Omega ~ \exp(-\beta \mathcal H)}{V^N}} = 1
        \end{equation*}

        Hence 
        \begin{equation*}
            \int_{\mathcal M^{N_1}} d\Omega_1 ~ \rho_{gc} = \frac{N!}{N_1! N_2} \frac{(V_1)^{N_1} (V_2)^{N_2}}{V^N} 
        \end{equation*}
        and the normalisation condition becomes, using $N = N_1 + N_2$, 
        \begin{equation*}
            \sum_{N_1 = 0}^{N} \int_{\mathcal M^{N_1}} d\Omega_1 \rho_{gc} = \sum_{N_1 = 0}^{N} \frac{N!}{N_1! N_2!} \frac{(V_1)^{N_1} (V_2)^{N_2}}{V^N} = \sum_{N_1 = 0}^{N} \binom{N}{N_1} \Big ( \frac{V}{V} \Big)^{N_1}  \Big ( \frac{V_2}{V} \Big)^{N - N_1} = \Big ( \frac{V_1 + V_2}{V} \Big)^N 
        \end{equation*}
        which in the thermodynamical limit is 
        \begin{equation*}
            \lim_{td} \Big ( \frac{V_1 + V_2}{V} \Big)^N  = 1
        \end{equation*}
    \end{proof}

\section{Thermodynamical potentials} 

    The grancanonical probability density distribution can be also written as 
    \begin{equation*}
        \rho_{gc} (q_i, p_i) = \frac{\exp(-\beta (\mathcal H (q_i, p_i) - \mu N))}{\mathcal Z}
    \end{equation*}
    where $\mu$ is the chemical potential and $\mathcal Z$ is the grancanonical partition function
    \begin{equation*}
        \mathcal Z = \sum_{N = 0}^{\infty} z^N Z_N = \exp(-\beta \Omega) 
    \end{equation*}
    where $z = \exp(\beta \mu)$ is the fugacity and $\Omega$ is the granpotential. 

    \begin{proof}
        Using~\eqref{ZF} and Taylor expanding to first order in $N_1 \ll N$ and $V_1 \ll V$, 
        \begin{equation*}
        \begin{aligned}
                \frac{Z_{N_2}[T, V]}{Z_N[T, V]} & = \frac{\exp(-\beta F(T, N_2, V_2))}{\exp(-\beta F(T, N, V))} \\ & = \exp(-\beta (F(T, N-N_1, V-V_1) - F(T, N, V))) \\ & \simeq \exp(-\beta(\underbrace{\pdv{F}{N} \Big \vert_{T, V}}_{\mu} (-N_1) + \underbrace{\pdv{F}{V} \Big \vert_{T, N}}_{-p} (-V_1))) \\ & = \exp(-\beta(-\mu N_1 + p V_1))
        \end{aligned}
        \end{equation*}

        Hence, now all the degrees of freedom of the environment has been removed
        \begin{equation*}
        \begin{aligned}
            \rho_{gc} & = \frac{\exp(\beta \mathcal H)}{N! h^{dN}} \exp (-\beta (-\mu N + p V)) \\ & = \frac{\exp(\beta \mathcal H)}{N! h^{dN}} \underbrace{\exp ( \beta \mu)^N}_{z^N} \exp(-\beta p V) \\ & = \frac{z^N \exp(\beta \mathcal H)}{N! h^{dN}} \exp(-\beta p V)
        \end{aligned}
        \end{equation*}
        where we introduced the fugacity.
        
        Recall~\eqref{OM}, the normalisation condition becomes 
        \begin{equation*}
        \begin{aligned}
            1 & = \sum_{N=0}^{\infty} \int_{\mathcal M^N} d\Omega \rho_{gc} \\ & = \sum_{N=0}^{\infty} \int_{\mathcal M^N} d\Omega \frac{z^N \exp(\beta \mathcal H)}{N! h^{dN}} \exp(-\beta p V) \\ & = \exp(-\beta p V) \sum_{N=0}^{\infty} z^N \frac{\int_{\mathcal M^N} d\Omega}{h^{dN} N!} \\ & = \exp(-\beta p V) \underbrace{\sum_{N=0}^{\infty} z^N Z_N}_{\mathcal Z} \\ & = \exp(-\beta p V) \mathcal Z
        \end{aligned}
        \end{equation*}
        Hence 
        \begin{equation*}
            \mathcal Z = sum_{N=0}^{\infty} z^N Z_N = \exp(\beta p V)
        \end{equation*}
        and 
        \begin{equation*}
            \rho_{gc} (q_i, p_i) = \frac{\exp(-\beta (\mathcal H(q_i, p_i) - \mu N))}{\mathcal Z} = \frac{\exp(-\beta \mathfrak H(q_i, p_i) )}{\mathcal Z}
        \end{equation*}
        where $\mathfrak H = \mathcal H - \mu N$ is the grancanonical hamiltonian.
    \end{proof}

    Let $f(q^i, p_i)$ be an observable, then its grancanonical average is 
    \begin{equation*}\label{obsgc}
    \begin{aligned}
        \avp{f(q^i, p_i)}{gc} & = \sum_{N = 0}^{\infty} \int_{\mathcal M} d\Omega ~ \rho_{gc} f_N \\ & = \sum_{N = 0}^{\infty} \int_{\mathcal M} d\Omega ~ \frac{\exp (-\beta (\mathcal H - \mu N ))}{\mathcal Z} f_N \\ & = \frac{1}{\mathcal Z} \sum_{N=0}^{\infty} z^N Z_N \int_{\mathcal M} d\Omega \frac{\exp(-\beta \mathcal H)}{Z_N} f_N \\ & = \frac{1}{\mathcal Z} \sum_{N=0}^{\infty} z^N Z_N \avp{f_N}{c}
    \end{aligned}
    \end{equation*}

    The grancanonical internal energy is 
    \begin{equation}\label{gran:e}
        E = - \pdv{}{\beta} \ln \mathcal Z \Big \vert_z
    \end{equation}

    \begin{proof}
        \begin{equation*}
        \begin{aligned}
            - \pdv{}{\beta} \ln \mathcal Z \Big \vert_z & = - \frac{1}{\mathcal Z} \pdv{}{\beta} \mathcal Z \Big \vert_z \\ &  = - \frac{1}{\mathcal Z} \pdv{}{\beta} \mathcal \sum_{N=0}^{\infty} z^N Z_N \Big \vert_z \\ & = - \sum_{N=0}^{\infty} \frac{z^N}{\mathcal Z} \pdv{}{\beta} \int d\Omega \exp (-\beta \mathcal H) \\ & = \sum_{N=0}^{\infty} \int d\Omega ~ \frac{\exp(-\beta (\mathcal H + \mu N))}{\mathcal Z} \mathcal H \\ & = \avp{\mathcal H}{gc} = E  
        \end{aligned}
        \end{equation*}
    \end{proof}

    The grancanonical number of particles is 
    \begin{equation}\label{gran:n}
        \avp{N}{gc} = z \pdv{}{z} \ln \mathcal Z \Big \vert_T
    \end{equation}

    \begin{proof}
        \begin{equation*}
        \begin{aligned}
            z \pdv{}{z} \ln \mathcal Z \Big \vert_T & = \frac{z}{\mathcal Z} \pdv{}{z}\mathcal Z \Big \vert_T \\ & = \frac{z}{\mathcal Z} \pdv{}{z} \sum_{N=0}^{\infty} z^N Z_N \\ & = frac{z}{\mathcal Z} \sum_{N=0}^{\infty} N z^{N-1} Z_N \\ & = \sum_{N=0}^{\infty} z^N Z_N N = \avp{N}{gc}
        \end{aligned}
        \end{equation*}
    \end{proof}

    The grancanonical potential is 
    \begin{equation}\label{gran:o}
        \Omega = - \frac{1}{\beta} \ln \mathcal Z
    \end{equation}

    The universal Boltzmann's formula is still valid
    \begin{equation*}
        S_{gc} = -k_B \avp{\ln \rho_{gc}}{gc} 
    \end{equation*}

    \begin{proof}
        Using~\eqref{gran:o},
        \begin{equation*}
        \begin{aligned}
            -k_B \avp{\ln \rho_{gc}}{gc} & = -k_B \int d\Omega ~ \rho_{gc} \ln \rho_{gc} \\ & = -k_B \sum_{N=0}^{\infty} \frac{z^N}{\mathcal Z} \int d\Omega ~ \exp(- \beta \mathcal H) \ln \rho_{gc} \\ & = = -k_B \sum_{N=0}^{\infty} \frac{z^N}{\mathcal Z} \int d\Omega ~ \exp(- \beta \mathcal H) (- \beta \mathcal H + \beta \mu N + \ln \mathcal Z) \\ & = k_B \beta \underbrace{\sum_{N=0}^{\infty} \frac{z^N}{\mathcal Z} \int d\Omega ~ \exp(-\beta \mathcal H) \mathcal H}_{E} - k_B \beta \mu \underbrace{\sum_{N=0}^{\infty} \frac{z^N}{\mathcal Z} \int d\Omega ~ \exp(-\beta \mathcal H) N}_{N} \\ & \quad + k_B \ln \mathcal Z \underbrace{\sum_{N=0}^{\infty} \frac{z^N}{\mathcal Z} \int d\Omega ~ \exp(-\beta \mathcal H)}_{1} \\ & = \frac{E - \mu N - \Omega}{T} = S
        \end{aligned}
        \end{equation*}
    \end{proof}

\chapter{Entropy}

    The Boltzmann's universal law allows us to define entropy in terms of number of states
    \begin{equation*}
        S = - k_B \av{\ln \rho} = k_B \ln \Sigma = \lim_{TD} S_{TD}
    \end{equation*}

    Thermodynamics tells us that equilibrium corresponds to maximum entropy.

    We consider a canonical ensemble with a discrete set of energy values, but it can be generalised for grancanonical and continuous energy levels. Therefore, the probability density distribution is~\eqref{candist}
    \begin{equation*}
        \rho_c (E_r) = \frac{\exp(-\beta E_r)}{Z_N}
    \end{equation*}
    where the canonical partition function~\eqref{zn} becomes 
    \begin{equation*}
        Z_N = \int_{\mathcal M^N} d\Omega~ \exp(-\beta \mathcal H(q^i, p_i)) = int_0^\infty dE \int_{S_E} dS_E ~ \exp(-\beta E) \simeq \sum_{r=1}^{p} g_r \exp(-\beta E_r)
    \end{equation*}
    where we foliated $\mathcal M^N$ in energy surfaces $S_E$ and $g_r$ is the multiplicity or degeneracy, i.e.~how many levels have the same energy.

    So far, we have started from an a-priori probability density distribution and from it derive the entropy. From now on, we will change the picture and do the converse: the probability distibution is the one corresponding to maximum entropy, given the macroscopic constains. To do so, we introduce the Shannon's information entropy
    \begin{equation*}
        H = - \sum_{i = 1}^{N} p_i \ln p_i
    \end{equation*}
    which is the only function with the following properties for a random variable $x$ such that it has $N$ possible outcomes $x_i$ with probability $p_i$ 
    \begin{enumerate}
        \item it is continuous with $p_i$,
        \item is monotonically increasing with $N$,
        \item it is invariant under compositions of subsystems, i.e.~change how we collect in group.
    \end{enumerate}

\subsection{Inference problem}

    Given a certain constraint for a function $\av{f}$, what is the expectation value for another function $g$? The answer can be found with the principle of maximum entropy, subjected to Lagrange multipliers given by the constraints 
    \begin{equation*}
        \sum_{i=1}^{N} p_i = 1 \quad \sum_{i=1}^{N} p_i f(x_i) = \av{f(x)}
    \end{equation*}
    Hence, the problem reduces to maximise the function
    \begin{equation}\label{entr}
        H = - \sum_{i=1}^{N} p_i \ln p_i + \alpha \Big( \sum_{i=1}^{N} p_i - 1 \Big) + \beta \Big( \sum_{i=1}^{N} p_i f(x_i) - \av{f} \Big)
    \end{equation}

    In particular, we need to count the number of ways $W_{\{n_r\}}$ we can find $n_r$ systems with energy $E_r$, given a set of discrete energy levels $E_r$, each of degeneracy $g_r$ on which we distribute $n_r$ particles. Hence, the probability density distribution $n_r^*$ is the one which maximises~\eqref{entr}, with entropy 
    \begin{equation*}
        S = \ln W_{\{n_r\}}
    \end{equation*} 
    and the constrains 
    \begin{equation*}
        N = \sum_{r} n_r \quad E = \sum_r n_r E_r
    \end{equation*}

    In order to count $W_{\{n_r\}}$, we need to take into account distinguishablility or not of particles. Therefore, we decomposed it into 
    \begin{equation*}
        W_{\{n_r\}} = W_{\{n_r\}}^{(1)} W_{\{n_r\}}^{(2)}
    \end{equation*}
    where $W_{\{n_r\}}^{(1)}$ countsin how many we can put $n_r$ particles in the energy level $E_r$ and $W_{\{n_r\}}^{(1)}$ consider the degeneracy of these levels.

\subsection{Boltzmann distribution}
\subsection{Bose-Einstein distribution}
\subsection{Fermi-Dirac distribution}
