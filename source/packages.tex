\usepackage[a4paper, top = 4cm, bottom = 4cm, left = 3cm, right = 3cm]{geometry}

\usepackage{xcolor}
\xdefinecolor{mycolor}{RGB}{0,175,179} 
\usepackage{hyperref}
\hypersetup{colorlinks, linkcolor={mycolor}, citecolor={mycolor}, urlcolor={mycolor}}

\usepackage{lipsum}

\renewcommand{\aftertoctitle}{\afterchaptertitle\par\nobreak\hfill{\normalfont{Page}}\par\nobreak}

\usepackage{titlesec}
\titleformat{\part}[display]
  {\normalfont\HUGE\bfseries\color{mycolor}\centering}
  {Part \thepart}{20pt}{\HUGE\normalfont\color{black}}
\titleformat{\chapter}[display]
  {\normalfont\HUGE\bfseries\color{mycolor}\centering}
  {Chapter \thechapter}{20pt}{\HUGE\normalfont\color{black}}
\titleformat{\section}
  {\normalfont\Large\bfseries\color{mycolor}\centering}
  {\thesection}{1em}{}
\titleformat{\subsection}
  {\normalfont\large\bfseries\color{mycolor}\centering}
  {\thesubsection}{1em}{}

\renewcommand{\printtoctitle}[1]{\HUGE\normalfont\color{black}#1}

\usepackage[backend=bibtex, sorting=none]{biblatex}
\addbibresource{../bibliography.bib}

\usepackage{amsmath}
\usepackage{amsthm}
\usepackage{thmtools}
\usepackage{mathtools}

\newtheorem{principle}{Principle}[chapter]
\newtheorem{lemma}{Lemma}[chapter]
\theoremstyle{definition}
\newtheorem{example}{Example}[chapter]
\renewcommand\qedsymbol{q.e.d.}

\theoremstyle{remark}
\newtheorem{case}{Case}

\newcommand{\dv}[2]{\frac{d#1}{d#2}}
\newcommand{\dvin}[3]{\frac{d#1}{d#2}\Big\vert_{#3}}
\newcommand{\dvd}[2]{\frac{d^2#1}{d#2^2}}
\newcommand{\dvf}[2]{\frac{\delta #1}{\delta #2}}
\newcommand{\pdv}[2]{\frac{\partial#1}{\partial#2}}
\newcommand{\pdvd}[3]{\frac{\partial^2 #1}{\partial#2 \partial#3}}
\newcommand{\pdvdu}[2]{\frac{\partial^2 #1}{\partial#2^2}}
\newcommand{\integ}[3]{\int_{#1}^{#2}d#3~}
\newcommand{\poi}[2]{[#1,~#2]}
\newcommand{\poiexp}[2]{\pdv{#1}{q^i} \pdv{#2}{p_i} - \pdv{#2}{q^i} \pdv{#1}{p_i}}

\newcommand{\comm}[2]{[#1,~#2]}
\newcommand{\set}[2]{\{#1\colon#2\}}
\newcommand{\inner}[2]{\langle#1,~#2\rangle}
\newcommand{\av}[1]{\langle#1\rangle}
\newcommand{\avp}[2]{\langle#1\rangle_{#2}}
\newcommand{\ket}[1]{\vert#1\rangle}
\newcommand{\bra}[1]{\langle#1\vert}
\newcommand{\braket}[2]{\langle#1\vert#2\rangle}

\newtheoremstyle{colored}{}{}{\itshape}{}{\color{mycolor}\normalfont\bfseries\indent}{}{\newline}{}

\declaretheorem[
  style=colored,
  name=Definition,
  numberwithin=chapter,
]{definition}

\declaretheorem[
  style=colored,
  name=Theorem,
  numberwithin=chapter,
]{theorem}

\declaretheorem[
  style=colored,
  name=Corollary,
  numberwithin=chapter,
]{corollary}

\declaretheorem[
  style=colored,
  name=Law,
  numberwithin=chapter,
]{law}

\declaretheorem[
  style=colored,
  name=Principle,
  numberwithin=chapter,
]{princ}

\usepackage{amsfonts}
\usepackage{dsfont}
\usepackage{yfonts}
\usepackage{amssymb}

\let\oldproof\proof
\renewcommand{\proof}{\color{darkgray}\oldproof}

\let\oldexample\example
\renewcommand{\example}{\color{darkgray}\oldexample}

\usepackage{cancel}
\usepackage{indentfirst}

\usepackage{tikz}
\usepackage{amssymb}
\usepackage{pgfplots}
\usepgfplotslibrary{patchplots}
\usetikzlibrary{patterns, positioning, arrows}
\pgfplotsset{compat=1.15}

\DeclareMathOperator{\tr}{tr}
\DeclareMathOperator{\str}{str}
\DeclareMathOperator{\real}{Re}
\DeclareMathOperator{\imm}{Im}
\DeclareMathOperator{\sgn}{sgn}
\DeclareMathOperator{\spann}{span}


\usepackage[T1]{fontenc}
\usepackage[utf8]{inputenc}
\usepackage{pythontex} 
\usepackage{nopageno} 
\usepackage{pgf}

