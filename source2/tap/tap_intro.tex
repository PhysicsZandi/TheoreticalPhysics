\chapter*{Introduction}

    The standard model is the best theory we have to describe particles and interactions. It has been tested during the last decades with enormous successes. However, there are three pillars that are not yet understood: dark matter, neutrinos and matter/antimatter asymmetry. 

    $75 \%$ of the Universe is composed by dark energy and the rest is matter. We know that baryonic matter is only the $5\%$, whereas the remaining $20 \%$ does not interact with the electromagnetic radiation. For this reason, it is called dark matter. Evidence of the existence of such matter can be trace into different scales: the rotational curve of galaxies, gravitational lensing of clusters of galaxies and the cosmological microwave background radiation (CMB) of the whole cosmos. Therefore, it cannot be explained by a modification of gravity, but by either particles or group of them, like primordial black holes.

    Neutrinos compare in three different flavours: electronic $\nu_e$, muonic $\nu_\mu$ and tauonic $\nu_\tau$. Their oscillation means that one can transform into another flavour. It is a quantum phenomenon, due to the alignment of the Hamiltonian like in the spin case. In order to do so, it needs to have mass/energy, because if they have different Hamiltonian, they have different time evolution, and they can be recognised by that. It can only explain a small fraction of dark matter. In fact, consider the decay $e^- e^+ \rightarrow \overline \nu \nu$ or $e^- e^+ \rightarrow \mu^+ \mu^-$. There are two time scales to consider: the decay rated $\Gamma \sim \sigma m$ and the Hubble time $H$. If $\Gamma > H$, the decay never happen, otherwise it does. Furthermore, if $T < m$, the decay goes out of equilibrium. Since the universe is expanding, due to redshift, the temperature is cooling down and neutrinos becomes non-relativistic. This means that at the redshift of galaxy formation $z \sim 40/50$, neutrinos where too fast although non-relativistic to explain this phenomenon.

    In the early universe, there was an equilibrium between the amount of particles and antiparticles. However, a small asymmetry of order of one particle of $10^{10}$ particles, allows that when the temperature cooled, matter and antimatter annihilated into radiation, but a small amount of matter had no pair, so it remained and formed out Universe.

    These three pillars can be studied not only looking at the early universe, but also in contemporary extreme environments. Cosmic rays produced in supernovae explosions, where the $99 \%$ of the energy is due to neutrinos, or accelerated high-energy particles in AGNs. Another observational resource is gravitational waves. They can be produced by a merger of black holes, pulsars, etc. or they can come from the early Universe, when there was a phase transition that cause a spontaneous symmetry breaking. In thermal plasma, the only contribution to mass was the thermal energy (potential like a parabola as in Higgs) but when the temperature cooled down, it becomes negligible and the normal contribution of mass arises. Suppose there is a $\mathbb Z_2$ symmetry where two patches where linked and then the symmetry breaks. A gravitational wave comes up when vacuum bubbles merge.

    Multimessenger physics is the merge of information coming from cosmic rays and gravitational waves.

