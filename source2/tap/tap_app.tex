\appendix

\chapter{The Standard Model}

    The Standard Model is based on a gauge symmetry, i.e.~it is invariant under a local symmetry which is a transformation that depends on spacetime coordinates 
    \begin{equation*}
        SU(3)_c \times SU(2)_l \times U(1)_y ~.
    \end{equation*}

\section{Covariant derivative}

    An aside regarding covariant derivatives. Consider a gauge transformation $SU(N)$ of the wave function 
    \begin{equation*}
        \psi' = U(x) \psi = \exp(i q \theta_a (x) T^a) \psi \simeq (\mathbb I + i q \theta^a (x) T_a) \psi ~,
    \end{equation*}
    where $\theta_a (x)$ are the $N^2 - 1$ spacetime coordinate dependent real parameters and $T_a$ are the $N^2 - 1$ generators. In order to have an invariant Dirac Lagrangian of the kind $\psi^\dagger \partial_\mu \psi$ under this transformation, we substitute the partial derivatives with the covariant derivative, defined as 
    \begin{equation*}
        D^\mu = \partial^\mu + i q A^\mu_a (x) T^a ~,
    \end{equation*}
    where $A^\mu_a (x)$ are compensating fields. Recall that $a$ is an index of the transformation $SU(N)$ and $\mu$ is a Lorentz index. Hence, 
    \begin{equation*}
        \psi^\dagger D_\mu \psi = \psi'^\dagger D'_\mu \psi' = \psi^\dagger U^{-1} D'_\mu U(x) \psi ~, \quad U^{-1} D'_\mu U(x) = D_\mu ~,
    \end{equation*}
    which means that the covariant derivative transforms as
    \begin{equation*}
        D'_\mu = U(x) D_\mu U^{-1} (x) = \exp(i q \theta_a (x) T^a) D_\mu \exp(-i q \theta_a (x) T^a) ~.
    \end{equation*}
    Moreover, 
    \begin{equation*}
    \begin{aligned}
        D'^\mu & = \partial^\mu + i q A'^\mu_a (x) T^a = U(x) D^\mu U^{-1} (x) \\ & = \exp(i q \theta_b (x) T^b) (\partial^\mu + i q A^\mu_a (x) T^a) \exp(-i q \theta_b (x) T^b) \\ & = \partial_\mu + \exp(i q \theta_b (x) T^b) \partial^\mu \exp(-i q \theta_b (x) T^b) \\ & \qquad + \exp(i q \theta_b (x) T^b) i q A^\mu_a (x) T^a \exp(-i q \theta_b (x) T^b) \\ & = \partial_\mu - i q \partial^\mu \theta_b (x) T^b + \exp(i q \theta_b (x) T^b) i q A^\mu_a (x) T^a \exp(- i q \theta_b (x) T^b) ~,
    \end{aligned}
    \end{equation*}
    hence,
    \begin{equation*}
    \begin{aligned}
        A'^\mu_a (x) T^a & = \exp(i q \theta_b (x) T^b) A^\mu_a (x) T^a \exp(- i q \theta_b (x) T^b) - \partial^\mu \theta_a (x) T^a \\ & = (\mathbb I + i q \theta_b (x) T^b) A^\mu_a (x) T^a (\mathbb I - i q \theta_b (x) T^b) - \partial^\mu \theta_a (x) T^a \\ & = A^\mu_a (x) T^a - i q \theta_b (x) A^\mu_a (x) (T^a T^b - T^b T^a) - \partial^\mu \theta_a (x) T^a \\ & = A^\mu_a (x) T^a - i q \theta_b (x) A^\mu_a (x) f^{ab}_{\phantom{ab} c} T^c - \partial^\mu \theta_a (x) T^a ~,
    \end{aligned}
    \end{equation*}
    and the compensating field transforms as 
    \begin{equation*}
        A'^\mu_a (x) = A^\mu_a (x) - i q \theta_b (x) A^\mu_c (x) f^{bc}_{\phantom{bc} a} - \partial^\mu \theta_a (x) ~.
    \end{equation*}

    More specifically, we have the following covariant derivatives 
    \begin{enumerate}
        \item $U(1)$ has $D^\mu = \partial^\mu + i \frac{q_1}{2} B^\mu$; 
        \item $SU(2)$ has $D^\mu = \partial^\mu + i \frac{q_2}{2} W^\mu_i \sigma^i$, where $i = 1,2,3$;
        \item $SU(3)$ has $D^\mu = \partial^\mu + i \frac{q_3}{2} G^\mu_\alpha \lambda^\alpha$, where $\alpha = 1, \ldots 8$;
    \end{enumerate}

\section{Gauge bosons}
    
    There is a number of gauge bosons equals to the number of generators. See Table~\ref{tab:bos}. After a change of basis, we find the know $\gamma, W^\pm, Z$.
    
    \begin{table}[h!]
        \centering
        \begin{tabular}{c | c | c}
            gauge group & bosons & generators \\
            \hline
            $U(1)$ & $B^\mu$ & - \\ 
            $SU(2)$ & $W^\mu_i$ with $i = 1, 2, 3$ & Pauli matrices $\sigma$ \\ 
            $SU(3)$ & $G^\mu_\alpha$ with $\alpha = 1, \ldots 8$ & Gell-Mann matrices $\lambda$ \\ 
        \end{tabular}
        \caption{Outline of the gauge bosons.}
        \label{tab:bos}
    \end{table}

    Particles are singlets, doublets or triplets of the representation of this gauge group. See Table~\ref{tab:par}. Notice that the theory is chiral, because left and right particels behave differently.

    \begin{table}[h!]
        \centering
        \begin{tabular}{c | c | c | c }
            particle & $SU(3)$ (representation) & $SU(2)$ (representation) & $U(1)$ (hypercharge)\\
            \hline
            $q_L = \begin{bmatrix} u_L \\ d_L \end{bmatrix}$ & $3$ & $2$ & $1/6$ \\ 
            $u_R$ & $3$ & $1$ & $2/3$ \\ 
            $d_R$ & $3$ & $1$ & $-1/3$ \\ 
            $l_L = \begin{bmatrix} \nu_L \\ e_L \end{bmatrix}$ & $1$ & $2$ & $-1/2$ \\ 
            $l_R$ & $1$ & $1$ & $-1$ \\ 
        \end{tabular}
        \caption{Outline of the particles.}
        \label{tab:par}
    \end{table}
    
    Suppose we have a term in the Lagrangian like $\mathcal L = y \overline e_L e_R$, where $y$ is the Yukawa constant. There is no $\nu_R$ because it is a single of all gauge groups and it can be neglected. Notice that the dimension is not okay, since $[\mathcal L] = 4$ and $[\overline e_L] = [e_R] = 3/2$. We need to add a scalar $[H] = 1$ and the term becomes $\mathcal L = y \overline e_L H e_R$. The spin is $1/2 \otimes 1/2 = 0 \otimes 1$, which means that it can come up in singlet $0$ or triplets $1$. Adding the interaction/mass term that breaks the symmetry, we need to substitute the vacuum expectation value $\av{H} = v_H / \sqrt{2}$. Therefore $\mathcal L = y \overline e_L v_H e_R / \sqrt{2} = m_e \overline e_L e_R$, where $m_e = y v_H / \sqrt{2}$ is the mass of the electron given by the Higgs boson. If the theory were not chiral, we would have a term like $\mathcal L = M \overline e_L e_R$ where $M$ has the dimension of a mass but it does not control its scale. Therefore, the chirality controls the scale of the mass. Any mass term must depend on a spontaneous symmetry breaking and all the others must be zero. The total lagrangian contains all the possible terms of particles and gauge bosons. See Table~\ref{tab:std}. 

    \begin{table}[h!]
        \centering
        \begin{tabular}{c | c }
            quarks & leptons \\
            \hline
            $q_L = \begin{bmatrix} u_L \\ d_L \end{bmatrix}, u_R, d_R \times 3$ generations&  $l_L = \begin{bmatrix} \nu_L \\ e_L \end{bmatrix}, e_R \times 3$ generations \\ 
        \end{tabular}
        \begin{tabular}{c | c }
            vector bosons & scalar bosons\\
            \hline
            $A_\mu, Z_\mu, W^\pm_\mu, G_\mu^\alpha$ & $H$  \\
        \end{tabular}
        \caption{Outline of the Standard Model.}
        \label{tab:std}
    \end{table}

\section{Spontaneous braking symmetry}

    Consider a spontaneous symmetry breaking of a global $U(1)$ group. The Lagrangian is 
    \begin{equation*}
        L = \partial_\mu \phi^* \partial^\mu \phi - m^2 |\phi|^2 - \lambda^2 |\phi|^4 = \partial_\mu \phi^* \partial^\mu \phi - V(\phi) ~.
    \end{equation*}
    We cannot have other terms otherwise the theory is not renormalisable. $m$ is the mass term that breaks the symmetry but it is not the physical mass of the particle/field. $\lambda$ is a dimensionless parameter. Suppose that $\phi$ is real and with the symmetry breaking $m^2$ becomes negative $m^2 < 0$. The potential becomes 
    \begin{equation*}
        V(\phi) = m^2 \phi^2 + \lambda \phi^4 ~.
    \end{equation*}
    See Figure~\ref{fig:bos}.

    \begin{figure}
        \centering
        \scalebox{0.7}{\pyc{plot1('x', '-x**2 + x**4 + 1 / 4', 2, 1, 1, True, False, True)}}
        \caption{A plot of the potential $V$ as a function of $\phi$. We have used $m = -1$ and $\lambda = 1$. Actually, it should be $3$-dimensional since $\phi$ is complex and the resulting plot is a so-called Mexican hat.}
        \label{fig:bos}
    \end{figure}

    There are two extrema. In fact 
    \begin{equation*}
        0 = \dv{V}{\phi} = 2 m^2 \phi + 4 \lambda \phi^3 ~, \quad \phi = 0 \lor \phi = \sqrt{- \frac{m^2}{2 \lambda}} = \frac{v_\phi}{\sqrt{2}} ~.
    \end{equation*}
    Computing the second derivative, we find 
    \begin{equation*}
        \dvd{V}{\phi} = 2 m^2 + 12 \lambda \phi^2 ~, \quad \dvd{V}{\phi} \Big \vert_{\phi = 0} = 2 m^2 < 0 ~, \quad \dvd{V}{\phi} \Big \vert_{\phi = \sqrt{- \frac{m^2}{2 \lambda}}} = 2 m^2 - 12 \frac{m^2}{2} = - 4 m^2 > 0 ~,
    \end{equation*}
    which means that $\phi = 0$ is a maximum and $\phi = v_\phi / \sqrt{2}$ is a minimum. It is better to (second) quantise the theory using as vacuum the minimum, called $|\phi|_{vev}$. Since $\phi$ is complex, we can decompose it into two real fields $\phi = (v_\phi + \phi_1 + i \phi_2) / \sqrt{2}$. The potential becomes 
    \begin{equation*}
    \begin{aligned}
        V(\phi) & = \frac{m^2}{2} |\phi|^2 + \frac{\lambda}{4} |\phi|^4 = \frac{m^2}{2} (v_\phi + \phi_1 + i \phi_2) (v_\phi + \phi_1 - i \phi_2) \\ & \qquad + \frac{\lambda}{4} ((v_\phi + \phi_1 + i \phi_2) (v_\phi + \phi_1 - i \phi_2))^2 \\ & = \frac{m^2}{2} ((v_\phi + \phi_1)^2 + \phi_2^2) + \frac{\lambda}{4} ((v_\phi + \phi_1)^2 + \phi_2^2)^2 \\ & = \frac{m^2}{2} ((v_\phi + \phi_1)^2 + \phi_2^2) + \frac{\lambda}{4} ((v_\phi + \phi_1)^4 + 2 (v_\phi + \phi_1)^2 \phi_2^2 + \phi_2^4) \\ & = \frac{m^2}{2} v_\phi^2 + \frac{m^2}{2} \phi_1^2 + m^2 v_\phi \phi_1 + \frac{m^2}{2} \phi_2^2 + \frac{\lambda}{4} v_\phi^4 + \frac{\lambda}{4} \phi_1^4 + \lambda v_\phi^3 \phi_1 + \lambda v_\phi \phi_1^3 + \frac{6}{4} \lambda v_\phi^2 \phi_1^2 \\ & \qquad + \frac{\lambda}{2} v_\phi^2 \phi_2^2 + \frac{\lambda}{2} \phi_1^2 \phi_2^2 + \lambda v_\phi \phi_1 \phi_2^2 + \frac{\lambda}{4} \phi_2^4  ~.
    \end{aligned}
    \end{equation*}
    Now, in order to study the mass term, we are interested in terms containing only $\phi_1^2$ and $\phi_2^2$
    \begin{equation*}
    \begin{aligned}
        V(\phi) & = \frac{m^2}{2} \phi_1^2 + \frac{m^2}{2} \phi_2^2 + \frac{3}{2} \lambda v_\phi^2 \phi_1^2 + \frac{\lambda}{2} v_\phi^2 \phi_2^2 \\ & = \frac{m^2}{2} \phi_1^2 + \frac{m^2}{2} \phi_2^2 - \frac{3}{2} \lambda \frac{m^2}{\lambda} \phi_1^2 - \frac{\lambda}{2} \frac{m^2}{\lambda} \phi_2^2 = - m^2 \phi_1^2 ~.
    \end{aligned}
    \end{equation*}
    This means that $\phi_1$ is massive $m_1 = \sqrt{-m^2}$ and $\phi_2$ represents a massless boson $m_2 = 0$, because there are no terms $m^2 \phi_2^2$. A Goldstone boson is a massless degree of freedom that comes up whenever there is spontaneous break of the symmetry and not when you add explicitly a term in the Lagrangian. In the Standard Model there are two accidental global symmetrues (lepton number) that give rise to pseudo-Goldstone numbers, called axions, that couple with fermions. 

    Consider now a spontaneous symmetry breaking of a gauge/local $U(1)$ group. The Lagrangian is 
    \begin{equation*}
        \mathcal L = \partial_\mu \phi^* \partial^\mu \phi - m^2 |\phi|^2 - \lambda^2 |\phi|^4 = \partial_\mu \phi^* \partial^\mu \phi - V(\phi) ~.
    \end{equation*}
    Treatment of the potential is the same as before. Consider now the kinetic term with the covariant derivative 
    \begin{equation*}
    \begin{aligned}
        K & = (D_\mu \phi)^* D^\mu \phi = (\partial_\mu \phi^* - i q A_\mu \phi^*)(\partial^\mu \phi + i q A^\mu \phi) \\ & = \partial_\mu \phi^* \partial^\mu \phi + q^2 A_\mu A^\mu |\phi|^2 + i q A^\mu (\partial_\mu \phi^* \phi - \partial_\mu \phi \phi^* ) ~.
    \end{aligned}
    \end{equation*}
    Since we have a degree of freedom to choose, we use the unitary gauge $\alpha = - \tan \phi_2 / v_\phi + \phi_1$, so that the Goldstone boson $\phi_2$ disappears. The gauge boson mass becomes 
    \begin{equation*}
        K = \frac{q^2 v_\phi^2}{2} A_\mu A^\mu = m^2 A_\mu A^\mu ~.
    \end{equation*}
    This means that a massless boson $A_\mu$ acquire mass after the spontaneous symmetry breaking.

\section{Higgs boson}

    Consider the potential part of the Lagrangian for the Higgs scalar field
    \begin{equation*}
        \mathcal L = - m^2 H^\dagger H + \lambda (H^\dagger H)^2 ~.
    \end{equation*}
    where $H$ is a matrix 
    \begin{equation*}
        H = \begin{bmatrix}
            h_+ \\ h_0 \\
        \end{bmatrix} ~.
    \end{equation*} +

    Regarding the potential for $h_0$, the same procedure of before is valid with $m^2 < 0$ and $\av{h_0} = v_H / \sqrt{2} = \sqrt{m^2 / 2 \lambda}$. 

    The kinetic part of the Lagrangian is 
    \begin{equation*}
        (D_\mu H)^\dagger D^\mu H ~,
    \end{equation*}
    where $D_\mu$ is the covariant derivative associated to the gauge group $SU(2) \times U(1)$
    \begin{equation*}
        D_\mu = \partial_\mu + i \frac{q_1}{2} B^\mu + i \frac{q_2}{2} W^\mu_i \sigma^i ~.
    \end{equation*}
    After spontaneous symmetry breaking, for $h_0 = (v_H + \tilde h_{0,1} + \tilde h_{0,2})$ where $\tilde h_{0,2}$ is a Goldstone boson that disappears in unitary gauge, $\tilde h_{0,1}$ is the real physical scalar field and $\tilde h_{0,1}, h_+$ get eaten up by $3$ massive Goldstone bosons. Therefore, with
    \begin{equation*}
        \av{H} = \begin{bmatrix}
            0 \\ v_H/\sqrt{2} \\
        \end{bmatrix} ~,
    \end{equation*}
    we obtain, neglecting the pure kinetic term $\partial_\mu$,
    \begin{equation*}
    \begin{aligned}
        D_\mu H & = \frac{i}{2} (q_1 B^\mu \mathbb I_2 + q_2 W^\mu_1 \sigma^1 + q_2 W^\mu_2 \sigma_2 + q_2 W^\mu_3 \sigma^3) H \\ & = \frac{i}{2} \begin{bmatrix}
            q_1 B^\mu + q_2 W^\mu_3 & q_2 W^\mu_1 - i q_2 W^\mu_2 \\
            q_2 W^\mu_1 + i q_2 W^\mu_2 & q_1 B^\mu - q_2 W^\mu_3 \\
        \end{bmatrix} \begin{bmatrix}
            0 \\ v_H / \sqrt{2}
        \end{bmatrix} = i \frac{v_H}{2 \sqrt{2}} \begin{bmatrix}
            q_2 W^\mu_1 - i q_2 W^\mu_2 \\ 
            q_1 B^\mu - q_2 W^\mu_3 \\ 
        \end{bmatrix}
    \end{aligned}
    \end{equation*}
    and 
    \begin{equation*}
        (D_\mu H)^\dagger = - i \frac{v_H}{2\sqrt{2}} \begin{bmatrix}
            q_2 W^\mu_1 + i q_2 W^\mu_2 &
            q_1 B^\mu - q_2 W^\mu_3 \\ 
        \end{bmatrix} ~.
    \end{equation*}
    Hence, 
    \begin{equation*}
    \begin{aligned}
        (D_\mu H)^\dagger D_\mu H & = \frac{v_H^2}{8} \begin{bmatrix}
            q_2 W^\mu_1 + i q_2 W^\mu_2 &
            q_1 B^\mu - q_2 W^\mu_3 \\ 
        \end{bmatrix} \begin{bmatrix}
            q_2 W^\mu_1 - i q_2 W^\mu_2 \\ 
            q_1 B^\mu - q_2 W^\mu_3 \\ 
        \end{bmatrix} \\ & = \frac{v_H^2}{8} \Big ( (q_2 W^\mu_1 + i q_2 W^\mu_2) (q_2 W^\mu_1 - i q_2 W^\mu_2) + (q_1 B^\mu - q_2 W^\mu_3)^2 \Big) \\ & = \frac{v_H^2}{8} \Big ( q_2^2 (W^\mu_1)^2 + q_2^2 (W^\mu_2)^2 + q_1^2 (B^\mu)^2 + q_2^2 (W^\mu_3)^2 + 2 q_1 q_2 B_\mu W^\mu_3 \Big) ~.
    \end{aligned}
    \end{equation*}
    However, it is better to diagonalise and make a change of basis of $(W^\mu_3, B^\mu)$. In fact, introducing the Weinberg angle 
    \begin{equation*}
        \sin \theta_W = \frac{q_1}{\sqrt{q_1^2 + q_2^2}} ~, \quad \cos \theta_W = \frac{q_2}{\sqrt{q_1^2 + q_2^2}} ~,
    \end{equation*}
    we find 
    \begin{equation*}
        \begin{bmatrix}
            A^\mu \\ Z^\mu \\
        \end{bmatrix} = \begin{bmatrix}
            \cos \theta_W & - \sin \theta_W \\
            \sin \theta_W & \cos \theta_W \\
        \end{bmatrix} \begin{bmatrix}
            B^\mu \\ W^\mu_3 \\
        \end{bmatrix} ~, \quad \begin{bmatrix}
            B^\mu \\ W^\mu_3 \\
        \end{bmatrix} = \begin{bmatrix}
            \cos \theta_W & \sin \theta_W \\
            - \sin \theta_W & \cos \theta_W \\
        \end{bmatrix} \begin{bmatrix}
            A^\mu \\ Z^\mu \\
        \end{bmatrix} ~,
    \end{equation*}
    or 
    \begin{equation*}
        B^\mu = A^\mu \cos \theta_W + Z^\mu \sin \theta_W ~, \quad W^\mu_3 = - A^\mu \sin \theta_W + Z^\mu \cos \theta_W ~.
    \end{equation*}
    Hence, 
    \begin{equation*}
    \begin{aligned}
        & q_1^2 (B^\mu)^2 + q_2^2 (W^\mu_3)^2 + 2 q_1 q_2 B_\mu W^\mu_3 \\ & = q_1^2 (A^\mu)^2 \cos^2 \theta_W + q_1^2 (Z^\mu)^2 \sin^2 \theta_W + 2 q_1^2 A^\mu Z_\mu \cos \theta_W \sin \theta_W \\ & \qquad + q_2^2 (A^\mu)^2 \sin^2 \theta_W + q_2^2 (Z^\mu)^2 \cos^2 \theta_W - 2 q_2^2 A^\mu Z_\mu \cos \theta_W \sin \theta_W \\ & \qquad - 2 q_1 q_2 (A^\mu)^2 \cos \theta_W \sin \theta_W + 2 q_1 q_2 (Z^\mu)^2 \cos \theta_W \sin \theta_W \\ & \qquad - 2 q_1 q_2 A_\mu Z^\mu \cos^2 \theta_W + 2 q_1 q_2 A_\mu Z^\mu \sin^2 \theta_W \\ & = (A^\mu)^2 \frac{q_1^2 q_2^2}{q_1^2 + q_2^2} + (Z^\mu)^2 \frac{q_1^4}{q_1^2 + q_2^2} + 2 A^\mu Z_\mu \frac{q_1^3 q_2}{q_1^2 + q_2^2}  + (A^\mu)^2 \frac{q_2^2 q_1^2}{q_1^2 + q_2^2} \\ & \qquad + (Z^\mu)^2 \frac{q_2^4}{q_1^2 + q_2^2} - 2 A^\mu Z_\mu \frac{q_1 q_2^3}{q_1^2 + q_2^2} - 2 (A^\mu)^2 \frac{q_1^2 q_2^2}{q_1^2 + q_2^2} \\ & \qquad + 2 (Z^\mu)^2 \frac{q_1^2 q_2^2}{q_1^2 + q_2^2} - 2 A_\mu Z^\mu \frac{q_1 q_2^3}{q_1^2 + q_2^2} + 2 A_\mu Z^\mu \frac{q_1^3 q_2}{q_1^2 + q_2^2} ~.
    \end{aligned}
    \end{equation*}
    We are interested in mass terms in which the fields are quadratic 
    \begin{equation*}
    \begin{aligned}
        \mathcal L & = \frac{v_H^2}{8} \Big ( q_2^2 (W^\mu_1)^2 + q_2^2 (W^\mu_2)^2 + (A^\mu)^2 \frac{q_1^2 q_2^2}{q_1^2 + q_2^2} + (Z^\mu)^2 \frac{q_1^4}{q_1^2 + q_2^2}  \\ & \qquad + (A^\mu)^2 \frac{q_2^2 q_1^2}{q_1^2 + q_2^2} + (Z^\mu)^2 \frac{q_2^4}{q_1^2 + q_2^2} - 2 (A^\mu)^2 \frac{q_1^2 q_2^2}{q_1^2 + q_2^2} + 2 (Z^\mu)^2 \frac{q_1^2 q_2^2}{q_1^2 + q_2^2} \Big) \\ & = \frac{v_H^2}{8} \Big (q_2^2 (W^\mu_1)^2 + q_2^2 (W^\mu_2)^2 + \frac{q_1^4 + 2 q_1^2 q_2^2 + q_2^4}{q_1^2 + q_2^2} (Z^\mu_2)^2 \Big ) \\ & = \frac{v_H^2}{8} \Big (q_2^2 (W^\mu_1)^2 + q_2^2 (W^\mu_2)^2 + (q_1^2 + q_2^2) (Z^\mu_2)^2 \Big ) ~.
    \end{aligned}
    \end{equation*}
    Last passage is to make a change also for $W_1$ and $W_2$ in 
    \begin{equation*}
        W_\pm^\mu = \frac{W_1^\mu \mp i W_2^\mu}{\sqrt{2}} ~, \quad W^\mu_1 = \frac{W_+^\mu + W_-^\mu}{\sqrt{2}} ~, \quad W^\mu_2 = \frac{ - W_+^\mu + W_-^\mu}{\sqrt{2} i} ~,
    \end{equation*}
    so that, recalling that in the complex case the square is the norm
    \begin{equation*}
    \begin{aligned}
        & q_2^2 (W^\mu_1)^2 + q_2^2 (W^\mu_2)^2 =  \frac{q_2^2}{2} (W^\mu_+)^2 + \frac{q_2^2}{2} (W^\mu_-)^2 + q_2^2 W^\mu_+ W_{\mu -} + \frac{q_2^2}{2} (W^\mu_+)^2 + q_2^2 (W^\mu_-)^2 + \frac{q_2^2}{2} W^\mu_+ W_{\mu -} \\ & = q_2^2 (W^\mu_+)^2 + q_2^2 (W^\mu_-)^2 + 2 q_2^2 W^\mu_+ W_{\mu -} ~.
    \end{aligned}
    \end{equation*}
    Putting together, for the mass terms we find
    \begin{equation*}
        \mathcal L = \frac{v_H^2}{8} \Big ( q_2^2 (W^\mu_+)^2 + q_2^2 (W^\mu_-)^2 + (q_1^2 + q_2^2) (Z^\mu_2)^2 \Big ) = \frac{v_H^2}{8} \Big ( q_2^2 (W^\mu_+)^2 + q_2^2 (W^\mu_-)^2 + (q_1^2 + q_2^2) (Z^\mu_2)^2 \Big ) ~,
    \end{equation*}
    which means that the mass of the bosons are 
    \begin{equation*}
        m_A = 0 ~, \quad m_Z = v_H \sqrt{\frac{q_1^2 + q_2^2}{8}} ~, \quad m_{W_\pm} = v_H \frac{q_2}{\sqrt{8}} ~.
    \end{equation*}
    $A_\mu$, associated to the electromagnetic field (photon) is massless, whereas $Z$ and $W^\pm$ associated to the weak interaction are massive.













\chapter{QFT}

\chapter{Cosmology}
