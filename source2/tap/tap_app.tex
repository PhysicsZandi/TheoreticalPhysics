\appendix

\chapter{The standard model}

    The standard model is based on a gauge symmetry, i.e.~it is invariant under a local symmetry which is a transformation that depends on spacetime coordinates 
    \begin{equation*}
        SU(3)_c \times SU(2)_l \times U(1)_y ~.
    \end{equation*}
    To ensure invariance under this gauge group, derivatives must be substituted with covariant derivatives 
    \begin{equation*}
        D_\mu = \partial_\mu + i g A_\mu 
    \end{equation*}
    that are invariant under a transformation generated by the group 
    \begin{equation*}
        \psi \rightarrow U \psi = \exp(i g \alpha^a(x) T^a_r) ~,
    \end{equation*}
    where $T$ are the generators depending on the group considered. There is a number of vector bosons equals to the number of generators. See Table~\ref{tab:bos}. After a change of basis, we find the know $\gamma, W^\pm, Z$.
    
    \begin{table}[h!]
        \centering
        \begin{tabular}{c | c | c}
            gauge group  & bosons & generators \\
            \hline
            $U(1)$ & 1 $B$ & - \\ 
            $SU(2)$ & 3 $W^1$, $W^2$, $W^3$ & $\sigma$ matrices \\ 
            $SU(3)$ & 8 gluons & $\gamma$ matrices\\ 
        \end{tabular}
        \caption{Outline of the gauge bosons.}
        \label{tab:bos}
    \end{table}

\chapter{QFT}

\chapter{Cosmology}
