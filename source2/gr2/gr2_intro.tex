\chapter*{Introduction}

    Black holes are regions of spacetime that are not in causal connection with the rest of the Universe. The locus of point that separate the two regions is called the event horizon. Even light cannot escape it. They are not purely mathematical objects, but there has been indirect evidences about their physical reality: gravitational waves emitted by a spiral merger of two black holes, images of the hot gas orbiting outside the event horizon. There are two kind of black holes: astrophysics black holes, like the one formed after the gravitational collapse of a star with mass greater than $10 M_\odot$; primordial black holes belonging to the early universe, thought to be candidates of dark matter, like the one at the center of our galaxy. The formers need to be of mass of order of stars, but the latter can be much bigger. They can be classified into four different types according to their charge $Q$ or angular momentum $J$. See Table~\ref{tab:bh}. Indeed, $M$, $Q$ and $J$ are the only necessary parameters to completely characterise them. 
    
    \begin{table}[h!]
        \centering
        \begin{tabular}{c | c | c}
            Black hole & Charged $Q$ & Rotation $J$ \\
            \hline
            Schwarzschild & no & no \\
            Reissner-Nordstroem & yes & no \\
            Kerr & no & yes \\
            Kerr-Newman & yes & yes \\
        \end{tabular}
        \caption{A table of all mathematically known black holes.}
        \label{tab:bh}
    \end{table}

    Black holes can be studied by means of four laws, analogous (non-accidentally) to the four laws of thermodynamics.