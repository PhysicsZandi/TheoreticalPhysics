\appendix

\chapter{Kretschmann scalar}

    In this chapter, we will compute the Kretschmann scalar $K = R^{\mu\nu\alpha\beta} R_{\mu\nu\alpha\beta}$ with the help of a python code (See https://github.com/PhysicsZandi).

\section{Metric and Christoffel symbols}

    The metric is 
    \begin{equation*}
        \pyc{SC.printmet()} ~.
    \end{equation*}

    Its inverse is 
    \begin{equation*}
        \pyc{SC.printmetinv()} ~.
    \end{equation*}

    The non-vanishing components of the Christoffel symbols are
    \begin{equation*}
        \pyc{SC.printchri()} ~.
    \end{equation*}

\section{Riemann tensor}

    The non-vanishing components of the Riemann tensor are
    \begin{equation*}
        \pyc{SC.printriem()} ~.
    \end{equation*}

    The non-vanishing components of the all upper-indices Riemann tensor is 
    \begin{equation*}
        \pyc{SC.printriemup()} ~.
    \end{equation*}

    The non-vanishing components of the all lower-indices Riemann tensor is
    \begin{equation*}
        \pyc{SC.printriemdown()} ~.
    \end{equation*}

\section{Kretschmann scalar}
    
    The Kretschmann scalar is 
    \begin{equation}\label{krescal}
        \pyc{SC.printriemscal()} ~.
    \end{equation}

\chapter{Birkhoff's theorem}

    In this chapter, we will study the Birkhoff's theorem. The signature is $-+++$.

    \begin{theorem}[Birkhoff]\label{birk}
        Let $g$ be a spherical symmetric metric that satisfies the vacuum Einstein's field equations. Then $g$ is static.
    \end{theorem}

\section{Part one of the proof}

    First, we prove that we can diagonalise the metric.

    \begin{proof}
        The most general spherical symmetric metric can be written as 
        \begin{equation*}
            ds^2 = - A(t, r) dt^2 + B(t, r) dr^2 + 2 C(t, r) drdt + D(t,r) d\Omega^2 ~,
        \end{equation*}
        where $d\Omega^2 = d\theta^2 + \sin^2 \theta d\phi^2$ is the solid angle element. We still have the freedom to rescale the time and radial coordinates. 
        Regarding the time coordinate, we make a suitable change of coordinate
        \begin{equation*}
            t = t(T,  R) ~, 
        \end{equation*}
        The differential of the transformation is 
        \begin{equation*}
            dt = \pdv{t}{T} dT + \pdv{t}{ R} d R ~, 
        \end{equation*}
        \begin{equation*}
            dt^2 = \Big ( \pdv{t}{T} dT + \pdv{t}{ R} d R \Big)^2 = \Big ( \pdv{t}{ R} \Big)^2 d R^2 + \Big ( \pdv{t}{T} \Big)^2 dT^2 + 2 \pdv{t}{T} \pdv{t}{ R} dT d R  ~.
        \end{equation*}
        The metric becomes
        \begin{equation*}
        \begin{aligned}
            ds^2 & = - A(t,  R) dt^2 + B(t,  R) d R^2 + 2 C(t,  R) d Rdt + D(t, R) d\Omega^2 \\ & = - A(t,  R) \Big ( \Big ( \pdv{t}{ R} \Big)^2 d R^2 + \Big ( \pdv{t}{T} \Big)^2 dT^2 + 2 \pdv{t}{T} \pdv{t}{ R} dT d R \Big ) + B(t,  R) d R^2 \\ & \qquad + 2 C(t,  R) d R \Big ( \pdv{t}{T} dT + \pdv{t}{ R} d R \Big ) + D(t, R) d\Omega^2 \\ & = - A (t,  R) \Big ( \pdv{t}{T} \Big)^2 dT^2 + \Big ( - A (t,  R) \Big( \pdv{t}{ R} \Big)^2 + B(t, R) + 2 C(t,  R) \pdv{t}{ R} \Big )d R^2 \\ & \qquad + \Big ( -2 A(t, R) \pdv{t}{T} \pdv{t}{ R} + 2 C(t, R) \pdv{t}{T} \Big ) dT d R + D(t, R) d \Omega^2 ~.
        \end{aligned}
        \end{equation*}
        In order to have a diagonal metric, we must impose the conditions
        \begin{equation*}
            A'(T, R) = A (t,  R) \Big ( \pdv{t}{T} \Big)^2 ~,
        \end{equation*}
        \begin{equation*}
            B'(T, R) = - A (t,  R) \Big( \pdv{t}{ R} \Big)^2 + B(t, R) + 2 C(t,  R) \pdv{t}{ R}  ~,
        \end{equation*}
        \begin{equation*}
            0 = -2 A(t, R) \pdv{t}{T} \pdv{t}{ R} + 2 C(t, R) \pdv{t}{T} ~.
        \end{equation*}
        The third one can be rewritten as 
        \begin{equation*}
            \pdv{t}{ R} = \frac{C(t, R)}{A(t, R)} ~,
        \end{equation*}
        and the other two become 
        \begin{equation*}
            A'(T, R) = A (t,  R) \Big ( \pdv{t}{T} \Big)^2 ~,
        \end{equation*}
        \begin{equation*}
            B'(T, R) = \frac{C^2(t, R)}{A(t, R)} + B(t, R) ~,
        \end{equation*}
        Hence, we have found a suitable  Rescaling of the time coordinate in such a way that the metric is diagonal
        \begin{equation*}
            ds^2 = - A'(T, R) dT^2 + B'(T,  R) d R^2 + D(T, R) d\Omega^2 ~.
        \end{equation*}
        Regarding the radius coordinate, we can simply impose
        \begin{equation*}
            D(T, R) = R^2 ~.
        \end{equation*}
        Finally, the metric is 
        \begin{equation*}
            ds^2 = - A'(T, R) dT^2 + B'(T, R) dR^2 + R^2 d\Omega^2 ~.
        \end{equation*}
    \end{proof}

    \noindent From now on, we will restore $r$, $t$, $A$, $B$.

\section{Computations}

    The metric is 
    \begin{equation*}
        \pyc{ES.printmet()} ~.
    \end{equation*}

    Its inverse is 
    \begin{equation*}
        \pyc{ES.printmetinv()} ~.
    \end{equation*}

    The non-vanishing components of the Christoffel symbols are
    \begin{equation*}
        \pyc{ES.printchri()} ~.
    \end{equation*}

    The non-vanishing components of the Riemann tensor are
    \begin{equation*}
        \pyc{ES.printriem()} ~.
    \end{equation*}

    The non-vanishing components of the Ricci tensor are
    \begin{equation*}
        \pyc{ES.printric()} ~.
    \end{equation*}

\section{Part two of the proof}  

    Now, we prove that $A$ and $B$ do not depend on time.

    \begin{proof}
        Using the $01$-th component of the Ricci tensor and the vacuum Einstein's field equations, we have 
        \begin{equation*}
            R_{10} = \frac{\dot B}{r B} = 0 ~, \quad \dot B = 0 ~, \quad B(t,r) = B(r) ~.
        \end{equation*}
        Making the time-derivative of the $22$-th component of the Ricci tensor and the vacuum Einstein's field equations, we have 
        \begin{equation*}
            \pdv{}{t} \dot R_{22} = - \pdv{}{t} \Big (\frac{r}{2 A B} \pdv{A}{r} \Big ) = 0 ~, \quad \pdv{}{t} \pdv{}{r} A = 0 ~, \quad A(t,r) = a(t) A'(r) ~.
        \end{equation*}
        Therefore, the metric becomes
        \begin{equation*}
            ds^2 = - A'(r) a(t) dt^2 + B(r) dr^2 + r^2 d\Omega^2 
        \end{equation*}
        and, with a rescaling of the time coordinate $a(t) dt^2 \rightarrow dt$, we finally obtain
        \begin{equation*}
            ds^2 = - A'(r) dt^2 + B(r) dr^2 + r^2 d\Omega^2  ~.
        \end{equation*}
        This concludes the proof.
    \end{proof}
    